%Ultimatives Tool zur Datierung:
%https://www.cc.kyoto-su.ac.jp/~yanom/pancanga/

%%%---2-DO---%%%:
% - ENTER RDGs of N2 and make apparatus negative?!
% - ENTER RDGs of D2
% - add xml ids for cladistics
% - produce diplomatic transcripts for saktumiva
% - make extra layer in Apparatus for parallels in SVARODAYA, Siddhasiddhantapaddhati and Amanaska
% - check all daṇḍas!!! now I think that it's more likely that many of them were lost in copies. Lectio difficilior! Very unconventional style of the autor! 
% - read Sarvangayogapradipika
% - maybe add second ciritical edition of yogasvarodaya?!



%%%%%%%%%%%%%%%%%%%%%%%%%%%%%%%%%%%%%%%%%%%
%MSS notes
%
%--B: i and ī are not differenciated
%
%
%
%%%%%%%%%%%%%%%%%%%%%%%%%%%%%%%%%%%%%%%%%%
%
%
%
%
%
%
%
%
%
%
%
%%%%%%%%%%%%%%%%%%%%%%%%%%%%%%%%%%%%%%%%%%%
\documentclass[12pt]{article}%{scrartcl}

%%% more functions
\usepackage{xcolor}

%%% Hyphenation settings
\usepackage{hyphenat}
\hyphenation{he-lio-trope opos-sum}
\tracingparagraphs=1
%Hyphenation in Devanāgarī of the edition still missing? Probably this needs to be modified in babel-iast package? 

%%% babel
\usepackage[english]{babel}
\usepackage{babel-iast/babel-iast}
\babelfont[iast]{rm}[Renderer=Harfbuzz, Scale=1.4]{AdishilaSan}%AdishilaSan}
\babelfont[english]{rm}{TeX Gyre Termes}

%%% ekdosis
\usepackage[teiexport=tidy,parnotes=true]{ekdosis}% =tidy cleans up HTML and XML documents by fixing markup errors and upgrading legacy code to modern standards. parnotes=footnotes below or above critical apparatus

\SetLineation{lineation=page,modulo} %lineation=pagesets thenumbering to start afresh at the top of each page.

\renewcommand{\linenumberfont}{\selectlanguage{english}\footnotesize} %sets language of lines to English

\SetTEIxmlExport{autopar=false} %autopar=falseinstructs ekdosis to ignore blank lines in the.tex sourcefile as markers for paragraph boundaries. As a result, each paragraph of the edition must be found within an environment associated with the xml <p> element

\SetHooks{
  lemmastyle=\bfseries,
  refnumstyle=\selectlanguage{english}\bfseries, 
}

\DeclareApparatus{parallel}[
  lang=english,
  sep = {] }
]

% Declare \ifinapparatus and set \inapparatustrue at the beginning of
% the apparatus criticus block. Also set the language.  
\newif\ifinapparatus
  \DeclareApparatus{default}[
  bhook=\inapparatustrue, 
  lang=english,
  bhook=\selectlanguage{english},
  sep = {] },
  delim=\hskip 0.75em,
  rule=\rule{0.7in}{0.4pt},
]

\DeclareApparatus{philcomm}[
lang=english,
sep={: },
bhook=\selectlanguage{english},
]

% Macros and Definitions for the Print of Sigla
\def\acpc#1#2#3{{#1}\rlap{\textrm{\textsuperscript{#3}}}\textsubscript{\textrm{#2}}\space}
\def\sigl#1#2{{{#1}}\textsubscript{\textrm{#2}}}
\def\None{{\sigl{N}{1}}} \def\Noneac{\acpc{N}{1}{ac}\,} \def\Nonepc{\acpc{N}{1}{pc}\,}
\def\Ntwo{{\sigl{N}{2}}} \def\Noneac{\acpc{N}{2}{ac}\,} \def\Nonepc{\acpc{N}{2}{pc}\,}
\def\Done{{\sigl{D}{1}}} \def\Doneac{\acpc{D}{1}{ac}\,} \def\Donepc{\acpc{D}{1}{pc}\,}
\def\Dtwo{{\sigl{D}{2}}} \def\Dtwoac{\acpc{D}{2}{ac}\,} \def\Dtwopc{\acpc{D}{2}{pc}\,}
\def\Uone{{\sigl{U}{1}}} \def\Uoneac{\acpc{U}{1}{ac}\,} \def\Uonepc{\acpc{U}{1}{pc}\,}                 
\def\Utwo{{\sigl{U}{2}}} \def\Utwoac{\acpc{U}{2}{ac}\,} \def\Utwopc{\acpc{U}{2}{pc}\,}

%%%%%%%%%%%%%% Tattvabinduyoga - List of Witnesses   %%%%%%%%%%%%%%%%%%%
\DeclareWitness{ceteri}{\selectlanguage{english}ceteri}{ceteri}[]   
\DeclareWitness{E}{\selectlanguage{english}E}{Printed Edition}[]    
\DeclareWitness{P}{\selectlanguage{english}P}{Pune BORI 664}[]  
\DeclareWitness{B}{\selectlanguage{english}B}{Bodleian 485}[]       
\DeclareWitness{N1}{\selectlanguage{english}N\textsubscript{1}}{NGMPP 38/31}[]
\DeclareWitness{N2}{\selectlanguage{english}N\textsubscript{2}}{NGMPP B 38/35}[]
\DeclareWitness{L}{\selectlanguage{english}L}{LALCHAND 5876}[]  
\DeclareWitness{D1}{\selectlanguage{english}D\textsubscript{1}}{IGNCA 30019}[] 
\DeclareWitness{D2}{\selectlanguage{english}D\textsubscript{2}}{IGNCA 30020}[]  
\DeclareWitness{U1}{\selectlanguage{english}U\textsubscript{1}}{SORI 1574}[] 
\DeclareWitness{U2}{\selectlanguage{english}U\textsubscript{2}}{SORI 6082}[]  

%%%%%%%%%%%%%%%%%%%%%%%%%%%%%%%%%%%%%%%%%%%
% Macro for Editing Abbrevs.
\def\om{\textrm{\footnotesize \textit{omitted in}\ }} %prints om. for omitted in apparatus
\def\korr{\textrm{\footnotesize \textit{em.}\ }} %prints em. for emended in apparatus
\def\conj{\textrm{\footnotesize \textit{conj.}\ }} %prints conj. for conjectured in apparatus

% \supplied{text} EDITORIAL ADDITION -> Within \lem oder \rdg
% \surplus{text} EDITORIAL DELETION -> Within \lem oder \rdg
% \sic{text} CRUX
% \gap{text} LACUNAE -> [reason=??, unit=??, quantity=??, extent=??]


%%%%%%%%%%%%%%%%%%%%%%%%%%%%%%%%%%%%%%%%%%% All macros of this list can be used in 
% Macro for Editing Abbrevs.
\def\eyeskip{\textrm{{ab.\,oc. }}}
\def\aberratio{\textrm{{ab.\,oc. }}}
\def\ad{\textrm{{ad}}}
\def\add{\textrm{{add.\ }}}
\def\ann{\textrm{{ann.\ }}}
\def\ante{\textrm{{ante }}} 
\def\post{\textrm{{post }}}
%\def\ceteri{cett.\,}                   
\def\codd{\textrm{{codd.\ }}}

\def\coni{\textrm{{coni.\ }}}
\def\contin{\textrm{{contin.\ }}}
\def\corr{\textrm{{corr.\ }}}
\def\del{\textrm{{del.\ }}}
\def\dub{\textrm{{ dub.\ }}}

\def\expl{\textrm{{explic.\ }}} 
\def\explicat{\textrm{{explic.\ }}}
\def\fol{\textrm{{fol.\ }}}
\def\foll{\textrm{{foll.\ }}}
\def\gloss{\textrm{{glossa ad }}}
\def\ins{\textrm{{ins.\ }}}      
\def\inseruit{\textrm{{ins.\ }}} 
\def\im{{\kern-.7pt\lower-1ex\hbox{\textrm{\tiny{\emph{i.m.}}}\kern0pt}}} %\textrm{\scriptsize{i.m.\ }}}      
\def\inmargine{{\kern-.7pt\lower-.7ex\hbox{\textrm{\tiny{\emph{i.m.}}}\kern0pt}}}%\textrm{\scriptsize{i.m.\ }}}      
\def\intextu{{\kern-.7pt\lower-.95ex\hbox{\textrm{\tiny{\emph{i.t.}}}\kern0pt}}}%\textrm{\scriptsize{i.t.\ }}}           
\def\indist{\textrm{{indis.\ }}}  
\def\indis{\textrm{{indis.\ }}}
\def\iteravit{\textrm{{iter.\ }}} 
\def\iter{\textrm{{iter.\ }}}
\def\lectio{\textrm{{lect.\ }}}   
\def\lec{\textrm{{lect.\ }}}
\def\leginequit{\textrm{{l.n. }}} 
\def\legn{\textrm{{l.n. }}}
\def\illeg{\textrm{{l.n. }}}

\def\primman{\textrm{{pr.m.}}}
\def\prob{\textrm{{prob.}}}
\def\rep{\textrm{{repetitio }}}
\def\secundamanu{\textrm{\scriptsize{s.m.}}}            \def\secm{{\kern-.6pt\lower-.91ex\hbox{\textrm{\tiny{\emph{s.m.}}}\kern0pt}}}%   \textrm{\scriptsize{s.m.}}}
\def\sequentia{\textrm{{seq.\,inv.\ }}}  
\def\seqinv{\textrm{{seq.\,inv.\ }}}
\def\order{\textrm{{seq.\,inv.\ }}}
\def\supralineam{{\kern-.7pt\lower-.91ex\hbox{\textrm{\tiny{\emph{s.l.}}}\kern0pt}}} %\textrm{\scriptsize{s.l.}}}
\def\interlineam{{\kern-.7pt\lower-.91ex\hbox{\textrm{\tiny{\emph{s.l.}}}\kern0pt}}}   %\textrm{\scriptsize{s.l.}}}
\def\vl{\textrm{v.l.}}   \def\varlec{\textrm{v.l.}} \def\varialectio{\textrm{v.l.}}
\def\vide{\textrm{{cf.\ }}}
\def\cf{\textrm{{cf.\ }}} 
\def\videtur{\textrm{{vid.\,ut}}}
\def\crux{\textup{[\ldots]} }
\def\cruxx{\textup{[\ldots]}}
\def\unm{\textit{unm.}}
%%%%%%%%%%%%%%%%%%%%%%%%%%%%%%%%%%%%

% List of Scholars
\DeclareScholar{ego}{ego}[
forename=Nils Jacob,
surname=Liersch]

% Persons:14\DeclareScholar{ego}{ego}[15forename=Robert,16surname=Alessi]17% Useful shorthands:18\DeclareShorthand{codd}{codd.}{V,I,R,H}19\DeclareShorthand{edd}{edd.}{Lit,Erm,Sm}20\DeclareShorthand{egoscr}{\emph{scripsi}}{ego}

%Useful shorthands:
%\DeclareShorthand{codd}{codd.}{V,I,R,H}
%\DeclareShorthand{edd}{edd.}{Lit,Erm,Sm}
\DeclareShorthand{egoscr}{\emph{scripsi}}{ego}
\DeclareShorthand{egomute}{\unskip}{ego}

\usepackage{xparse}

%%% define environments and commands
\NewDocumentEnvironment{tlg}{O{}O{}}{\begin{verse}}{\hfill #1\\ \end{verse}} %verse environment
\NewDocumentCommand{\tl}{m}{{\selectlanguage{iast} #1}}

\NewDocumentCommand{\extra}{m}{{\textcolor{teal}{#1}}} %command for additions to U2

\NewDocumentEnvironment{prose}{O{}}{\begin{otherlanguage}{iast}}{\end{otherlanguage}}
%\NewDocumentEnvironment{padd}{O{}}{\begin{otherlanguage}{iast}}{\end{otherlanguage}}
\NewDocumentEnvironment{tlate}{O{}}
%\NewDocumentEnvironment{tadd}{O{}}

%Define two commands: \skp ("sanskrit plus"), to be ignored by TeX in
%the edition text, but processed in the TEI output. Conversely, \skm
%("sanskrit minus") is to be processed in the edition text, but
%ignored if found in the apparatus criticus and in the TEI output:

\NewDocumentCommand{\skp}{m}{}
\TeXtoTEIPat{\skp {#1}}{#1}

%\NewDocumentCommand{\skpp}{m}{}
%\TeXtoTEIPat{\skpp {#1}}{#1}

\NewDocumentCommand{\skm}{m}{\unless\ifinapparatus#1-\fi}
\TeXtoTEIPat{\skm {#1}}{}


%%% modify environments and commands
%%% TEI mapping
\TeXtoTEIPat{\begin {tlg}}{<lg>} %lg=(Group of verse (s)) contains one or more verses or lines of verse that together form a formal unit (e.g. stanza, chorus).
\TeXtoTEIPat{\end {tlg}}{</lg>}

\TeXtoTEIPat{\begin {prose}}{<p>}
\TeXtoTEIPat{\end {prose}}{</p>}

\TeXtoTEIPat{\begin {tlate}}{<p>}
\TeXtoTEIPat{\end {tlate}}{</p>}

\TeXtoTEIPat{\\}{}
%\TeXtoTEI{tl}{l}
\TeXtoTEI{emph}{hi}
\TeXtoTEI{bigskip}{}
\TeXtoTEI{None}{N1}
\TeXtoTEI{Ntwo}{N2}
\TeXtoTEI{Done}{D1}
\TeXtoTEI{Dtwo}{D2}
\TeXtoTEI{Uone}{U1}
\TeXtoTEI{Utwo}{U2}
\TeXtoTEI{/}{|}
\TeXtoTEIPat{\korr}{em. }
\TeXtoTEIPat{\conj}{conj.}
\TeXtoTEIPat{\om}{omitted in }
\TeXtoTEIPat{english}{}
\TeXtoTEIPat{-}{ }
\TeXtoTEIPat{\textcolor {#1}{#2}}{<hi rend="#1">#2</hi>} 

% Nullify \selectlanguage in TEI as it has been used in
% \DeclareWitness but should be ignored in TEI.
\TeXtoTEI{selectlanguage}{}


\author{Nils Jacob Liersch}
\title{Yogatattvabindu of Rāmacandra\\ A Critical Edition and Annotated Translation}
\date{\today}

\parindent=3pt
\begin{document}
\maketitle
\clearpage

\section{Conventions in the Critical Apparatus}
\subsection{Sigla in the Critical Apparatus}

\begin{itemize}
\item E : Printed Edition
\item P : Pune BORI 664
\item L : Lalchand Research Library LRL5876
\item B : Bodleian Oxford D 4587
\item \None : NGMPP B 38-31
\item \Ntwo : NGMPP B 38-35 / A 1327-14
\item \Done : IGNCA 30019
\item \Dtwo : IGNCA 30020
\item \Uone : SORI 1574
\item \Utwo: SORI 6082
\end{itemize}

The order of the readings in the critical apparatus is arranged according to the quality of readings in decending order. The critical apparatus is positive. 

\subsection{Punctuation}

The very inconsistent use of punctuation marks in the witnesses at hand makes standardization necessary. Deviation of punctuation marks will not be documented in the critical apparatus. The usual standard conventions are followed:

Especially in the verse poetry, a \textit{daṇḍa} marks the end of a half verse, half of the \textit{śloka}, and the double \textit{daṇḍa} marks the end of a verse. A half verse is a \textit{pāda}, at least in some literary works, this is concluded by a \textit{daṇḍa} and the end of a \textit{śloka} by a double \textit{daṇḍa}. In prose the single \textit{daṇḍa} indicates the end of a sentence and the double \textit{daṇḍa} marks the end of a paragraph.

Variations in the usage of \textit{Avagraha} will not be recorded. 

\subsection{Sandhi}

Among the witnesses we see deviating and inconsistent application of \textit{sandhi}. There is no clear evidence that originally \textit{sandhi} was intentionally not applied. This edition will therefore apply \textit{sandhi} consistently throughout the constituted text to provide a readable text sticking to contemporary conventions in Sanskrit. To simplyfy the apparatus the variant readings concerning \textit{sandhi} are not recorded to the most part. Exceptions are made in remarkable cases. 

\subsection{Class Nasals}

Again, due to inconsistent use of class nasals among the witnesses \textit{anusvāra}s have been substituted with the respective class nasals throughout the critical edition. To simplyfy the apparatus deviating usage of class nasals is not documented in the apparatus.
\clearpage

\section{Critical Edition of the \textit{Yogatattvabindu}}
\begin{alignment}[
    texts=edition[class="edition"];
    translation[class="translation"],
    ]
  \begin{edition}
    \ekddiv{type=ed}
    \centerline{\textrm{\small{[Introduction]}}}
    \bigskip
    \begin{prose}
%--------------------------
% śrī gaṇeśāya namaḥ /                                                    rājayogāntargataḥ //  binduyogaḥ   \E 
% śrī gaṇeśāya namaḥ /                                                    atha tattvabiṃduyogaprāraṃbhaḥ     \L
% śrī ṇe ya maḥ /                                                         atha rājayoga         liṣyate      \P
% śrī gaṇeśāya namaḥ // śrī gurave namaḥ //                               atha rājayogaprakāro  likhyate //  \N1
% śrī gaṇeśāya namaḥ // śrī sarasvatyai namaḥ // śrī nirañjanāya namaḥ // atha rājayogaprakāro  likhyate //  \D1
% śrī gaṇeśāya namaḥ / oṃ śrī niraṃjanāya //                              atha rājayogaprakāra  likhyate //  \U1
% śrī gaṇeśāya namaḥ /                                                    atha rājayoga         likhyate //  \U2
%--------------------------
%Homage to Śrī Gaṇeśa. Now the methods of rājayoga are laid down.
%--------------------------          
      \app{\lem[wit={E,L,N1,D1,U1,U2}]{śrī gaṇeśāya namaḥ}
        \rdg[wit={P}]{śrī ṇe ya maḥ}
        \rdg[wit={N1}]{śrī gurave namaḥ}
        \rdg[wit={D1}]{śrī sarasvatyai namaḥ śrī nirañjanāya namaḥ}
        \rdg[wit={U1}]{oṃ śrī niraṃjanāya}}//
      \app{\lem[wit={N1,D1}]{atha rājayogaprakāro likhyate}
        \rdg[wit={U1}]{atha rājayogaprakāra  likhyate}
        \rdg[wit={E}]{rājayogāntargataḥ / binduyogaḥ}
        \rdg[wit={L}]{atha tattvabiṃduyogaprāraṃbhaḥ}
        \rdg[wit={P}]{atha rājayoga liṣyate}
        \rdg[wit={U2}]{atha rājayoga likhyate}}//
%--------------------------
% \om                       \E
% \om                       \L
% \om                       \O
% rājayogasyedaṃ phalaṃ      \P
% rājayogasya idaṃ phalaṃ    \N1
% rājayogasya idaṃ phalaṃ // \D1
% rājayogasya idaṃ phalaṃ    \U1
% rājayogasyedaṃ phalaṃ /    \U2
%--------------------------
rājayogasyedaṃ phalaṃ/
%--------------------------
%This is the result of \textit{rājayoga}:
%--------------------------
% \om                                                                                                                                                                \E
% \om                                                                                                                                                                \L
% \om                                                                                                                                                                \B
% yena rājayogenānekarājyabhogasamaya   eva    anekapārthivavinodaprekṣaṇasamaya  eva    bahutarakālaṃ śarīrasthitirbhavati    sa eva  rājayogaḥ tasyaite     bhedāḥ      \P
% yena rājayogenānekarājyabhogasamaya   eva /  anekapārthivavinodaprekṣaṇasamaya  eva /  bahutarakālaṃ śarīrasthitirbhavati    sa eva  rājayogaḥ /  tasya ete bhedāḥ /  \N1
% yena rājayogena anekarājyabhogasamaya eva // anekapārthivavinodaprekṣaṇasamaya  eva // bahutarakālaṃ śarīrasthitirbhavati // sa eva  rājayogaḥ // tasya ete bhedāḥ / \D1
% yena rājayogena anekarājyabhogasamaya eva // anekapārthivavinodaprekṣaṇasamaya  eva // bahutarakālaṃ śarīrasthitirbhavati    sa evaṃ rājayogaḥ    tasya ete bhedāḥ //   \U1 
% yena rājayogena anekarājyabhogasamaya eva // anekapārthivavinodaprekṣyaṇasamaya eva // bahutarakālaṃ śarīrasthitirbhavati // sa eva  rājayogastaisyaite     bhedāḥ //   \U2
% --------------------------
%\textit{Rājayoga} is that by which longterm durability of the body arises even amongst manifold royal pleasures even amongst the manifold royal entertainments and spectacle. This truly is \textit{rājayoga}. Of this [\textit{rājayoga}] these are the varieties: \end{tlate}
% --------------------------
      yena rājayogenānekarājyabhogasamaya eva/ anekapārthivavinoda
      \app{\lem[wit={P,N1,D1,U1}]{prekṣaṇasamaya}
        \rdg[wit={U2}]{prekṣyaṇasamaya}}
      eva/ bahutarakālaṃ śarīrasthitir-bhavati/ sa
      \app{\lem[wit={P,N1,D1,U2}]{eva}
        \rdg[wit={U2}]{evaṃ}}
      rājayogaḥ/ \bigskip
       tasyaite bhedāḥ/
     \end{prose}
     \end{edition}
      \begin{translation}
    \ekddiv{type=trans}
    \centerline{\textrm{\small{[Introduction]}}}
    \bigskip
    \begin{tlate}Homage to Śrī Gaṇeśa. Now the methods of rājayoga are laid down. This is the result of \textit{rājayoga}\footnote{This statement seems unconnected to the definition of rājayoga that follows.}: \textit{Rājayoga} is that by which longterm durability of the body arises even amongst manifold royal pleasures even amongst the manifold royal entertainments and spectacle. This truly is \textit{rājayoga}. Of this [\textit{rājayoga}] these are the varieties: \end{tlate}
      \bigskip
       \end{translation}
        
    \begin{edition}
      \ekddiv{type=ed}
%-------------------------
%
% \om                                                                                                                                                                \E
% \om                                                                                                                                                                \L
% \om                                                                                                                                                                \B
% kriyāyogaḥ 1 jñānayogaḥ 2 caryāyogaḥ 3 haṭhayogaḥ 4 karmayogaḥ 5 layayogaḥ 6 dhyānayogaḥ 7 maṃtrayogaḥ 8 lakṣyayogaḥ 9 vāsanāyogaḥ 10 śivayogaḥ 11 brahmayogaḥ 12 advaitayogaḥ 13 siddhayogaḥ 14 rājayogaḥ 15 ete paṃcadaśayogāḥ \P
% kriyāyogaḥ / jñānayogaḥ / caryāyogaḥ / haṭhayogaḥ / karmayogaḥ / layayogaḥ / dhyānayogaḥ / maṃtrayogaḥ / lakṣyayogaḥ / vāsanāyogaḥ / śivayogaḥ / brahmayogaḥ / advaitayogaḥ / rājayogaḥ / siddhayogaḥ / ete paṃcadaśayogāḥ // \N1
% kriyāyogaḥ // jñānayogaḥ // caryāyogaḥ // haṭhayogaḥ // karmayogaḥ // layayogaḥ // dhyānayogaḥ // maṃtrayogaḥ // lakṣyayogaḥ // vāsanāyogaḥ // śivayogaḥ // brahmayogaḥ // advaitayogaḥ // rājayogaḥ // siddhayogaḥ // ete paṃcadaśayogāḥ // \D1
% kriyāyogaḥ // jñānayogaḥ // tvaryāyogaḥ // haṭhayogaḥ // karmayogaḥ // layayogaḥ // dhyānayogaḥ maṃtrayogaḥ  lakṣayogaḥ  vāsanāyogaḥ  śivayogaḥ  brahmayogaḥ  advaitayogaḥ  rājayogaḥ  siddhayogaḥ ete paṃcadaśayogāḥ  \U1
% kriyāyogaḥ // jñānayogaḥ // caryāyogaḥ // haṭhayogaḥ // karmayogaḥ // nayayogaḥ // dhyānayogaḥ // maṃtrayogaḥ // lakṣyayogaḥ // vāsanāyogaḥ // śivayogaḥ // brahmayogaḥ // advaitayogaḥ // siddhayogaḥ // rājayogaḥ // evaṃ paṃcadaśāyogā bhavaṃti // \U2
%-------------------------
     \begin{prose}kriyāyogaḥ 1/\\ jñānayogaḥ 2/\\ \app{\lem[wit={P,N1,D1,U2}]{cāryayogaḥ}\rdg[wit={U1}]{tvaryāyogaḥ}} 3/\\ haṭhayogaḥ 4/\\ karmayogaḥ 5/\\ \app{\lem[wit={P,N1,D1,U1}]{layayogaḥ}\rdg[wit={U2}]{nayayogaḥ}} 6/\\ dhyānayogaḥ 7/\\ mantrayogaḥ 8/\\ \app{\lem[wit={P,N1,D1,U2}]{lakṣyayogaḥ}\rdg[wit={U1}]{lakṣayogaḥ}} 9/\\ vāsanāyogaḥ 10/\\ śivayogaḥ 11/\\ brahmayogaḥ 12/\\ advaitayogaḥ 13/\\ \app{\lem[wit={P,U2}]{siddhayogaḥ 14 /\\ rājayogaḥ 15}\rdg[wit={N1,D1,U1}]{rājayogaḥ / siddhayogaḥ}}/\linelabel{s1.z6e}\\ \\
 \note[type=philcomm, labelb=s4.z5a, lem={rājayoga}]{The initial codification of 15 \textit{yoga}s appears in \getsiglum{N1},P,\getsiglum{D1},\getsiglum{U1} and \getsiglum{U2}. It is ommitted in E and L. B can't be determined due to missing folios. P is the only witness which numbers the \textit{yoga}s with \textit{devanāgarī}-digits. I decided to include the numberation to improve the readability of the list. The other witnesses separate the list with single or double \textit{daṇḍa}s.}\app{\lem[wit={P,N1,D1,U1}]{ete pañcadaśayogāḥ}\rdg[wit={U2}]{evaṃ paṃcadaśāyogā bhavaṃti}}//\\\end{prose}
    \end{edition}
    \begin{translation}
   \ekddiv{type=trans}
\begin{tlate}1. Yoga of [mental] action (\textit{kriyāyoga}), \\ 2. Yoga of knowledge (\textit{jñānayoga}),\\ 3. Yoga of following strictly the applications (\textit{caryāyoga}),\\ 4. Yoga of force (\textit{haṭhayoga}),\\ 5. Yoga of deeds (\textit{karmayoga}),\\ 6. Yoga of absorption (\textit{layayoga}),\\ 7. Yoga of meditation (\textit{dhyānayoga}),\\ 8. Yoga of mantras (\textit{mantrayoga}),\\ 9. Yoga of fixation objects (\textit{lakṣyayoga}),\\ 10. Yoga of mental residues (\textit{vāsanāyoga}),\\ 11. Yoga of Śiva (\textit{śivayoga}),\\ 12. Yoga of Brahman (\textit{brahmayoga}),\\ 13. Yoga of non-duality (\textit{advaitayoga}),\\ 14. Yoga of completion (\textit{siddhayoga}),\\ 15. Yoga of kings (\textit{rājayoga}).\\ \\ These are the fifteen \textit{yoga}s.\footnote{At the current stage of research it is not clear if this list is a later addition by another scribe or, if indeed it originally stems from Rāmacandra. The list suggests a text following the order of yogas according to this list. However, the order and even the designation of some of the yogas given in the list is just followed very loosely in the text.}\bigskip \end{tlate}
    \end{translation}
    \end{alignment}
\begin{alignment}[
    texts=edition[class="edition"];
    translation[class="translation"],
    ]
      \begin{edition}
        \ekddiv{type=ed}
        \centerline{\textrm{\small{[Description of \textit{kriyāyoga}]}}}
        \bigskip
%--------------------------        
% \om                                      \E
% \om                                      \L
% \om                                      \B
% idānīṃ kriyāyogasya lakṣaṇaṃ kathyate/   \P
% idānīṃ kriyāyogasya lakṣaṇaṃ kathyate/   \N1
% idānīṃ kriyāyogasya lakṣaṇaṃ kathayate/  \D1
% idānīṃ kriyāyogasya lakṣaṇaṃ kathyate/   \U1
% atha   kriyāyogas   lakṣaṇaṃ          // \U2
%--------------------------
        \begin{prose}
        \app{\lem[wit={P,N1,D1,U1}]{idānīṃ}
            \rdg[wit={U2}]{atha}}
          \app{\lem[wit={P,N1,D1,U1}]{kriyāyogasya}
            \rdg[wit={U2}]{kriyāyogas}} lakṣaṇaṃ
          \app{\lem[wit={P,N1,U1}]{kathyate}
            \rdg[wit={D1}]{kathayate}
            \rdg[wit={U2}]{\om}}/\\\end{prose}
      \end{edition}
      \begin{translation}
      \ekddiv{type=trans}
      \centerline{\textrm{\small{[Description of \textit{kriyāyoga}]}}}
      \bigskip
    \begin{tlate}Now the characteristic of the Yoga of [mental] action (\textit{kriyāyoga}) described. \bigskip \end{tlate}
    \end{translation}
 \begin{edition}
 \ekddiv{type=ed}
 \begin{tlg}
%--------------------------   
% \om                                                    \E
% \om                                                    \L
% \om                                                    \B
% kriyāmuktir    ayaṃ yogaḥ    svapiṇḍe siddhidāyakaḥ    \P
% kriyāmuktir    ayaṃ yogaḥ /  svapiṇḍe siddhidāyakaḥ /  \N1 
% kriyāmuktir    ayaṃ yogaḥ    svapiṇḍe siddhidāyakaḥ /  \D1
% kriyāyuktir    ayaṃ yogaḥ /  svapiṇḍe siddhidāyakaḥ /  \U1
% kriyāmuktiḥ // ayaṃ yogaḥ    svapiṃ?  siddhidāyakaṃ // \U2 
%--------------------------
\tl{kriyāmuktir-ayaṃ yogaḥ svapiṇḍe \app{\lem[wit={P,N1,D1,U1}]{siddhidāyakaḥ}\rdg[wit={U2}]{siddhidāyakaṃ}}/}\\
%-------------------------
% \om                                                   \E
% \om                                                   \L
% \om                                                   \B
% yaṃ yaṃ karoti kallolaṃ kāryāraṃbhe manaḥ sadā         \P
% yaṃ yaṃ karoti kallolaṃ kāryāraṃbhe manaḥ sadā/        \N1
% yaṃ yaṃ karoti kallolaṃ kāryāraṃbhe manaḥ sadā/        \D1 
% yaṃ yaṃ karoti kallolaṃ kāryāraṃbhe manaḥ sadā/ 1      \U1
% yaṃ yaṃ karoti kallolaṃ kāryāraṃbhe manaḥ sadā/        \U2
%--------------------------
\tl{yaṃ yaṃ karoti kallolaṃ kāryāraṃbhe manaḥ sadā/}\\
%--------------------------
% \om                                                        \E
% \om                                                        \L
% \om                                                        \B
% tattataḥ kuñcanaṃ kurvan kriyāyogas tato bhavet           \P
% tattataḥ kuñcanaṃ kurvan kriyāyogas ato bhava    //      \N1
% tattataḥ kuñcanaṃ kurvan kriyāyogas ato bhava    //      \D1 
% taṃkṛ taṃ kuñcanaṃ kurvan kriyāyogas ato ?va     //1//   \U1
% tatastataḥ kuṃcanaṃ kurvan kriyāyogas tato bhavet //1//  \U2
%--------------------------
\tl{\app{\lem[wit={P,N1,D1}]{tattataḥ}
    \rdg[wit={U2}]{tatastataḥ}
    \rdg[wit={U1}]{taṃkṛ taṃ}}
  kuñcanaṃ kurvan-kriyāyoga\skp{s}-\app{
    \lem[wit={P,U2}, alt={tato bhavet}]{\skm{s}tato bhavet}
    \rdg[wit={N1,D1}]{ato bhava}
    \rdg[wit={U1}]{ato va}}//1//}\\
\end{tlg}
\end{edition}
\begin{translation}
\ekddiv{type=trans}
\begin{tlate}\textbf{1.} This Yoga is liberation through [mental] action. It bestows success(\textit{siddhi}) in ones own body. Each wave the mind creates at the beginning of an action, of all those one shall withdraw oneself. Then \textit{kriyāyoga} arises. \bigskip \bigskip \end{tlate}
\end{translation}
  \begin{edition}
    \ekddiv{type=ed}
    \begin{tlg}
%--------------------------      
% \om                                                                                                   \B
% \om                                                                                                   \L
% kṣamā vivekaṃ vairāgyaṃ śāntiḥ santoṣaniṣpṛhā    etadyuktiyuto yogī         kriyāyogī nigadyate       \E
% kṣamāvivekavairāgyaṃ    śāntiḥ santoṣanispṛhā    etat yuktiyuto yogī        kriyāyogī nigadyate       \N1
% kṣamāvivekavairāgyaṃ    śāntiḥ santoṣanispṛhaḥ   etat yuktiyuto yogī        kriyāyogī nigadyate       \D1
% kṣamāvivekavairāgyaṃ    śāntiḥ santoṣanispṛhāḥ   etadyuktiyuto yogī         kriyāyogī nigadyate       \P1
% kṣamāvivekavairāgya---- śāntisantoṣaniḥspṛhī     etadyuktiyuto yosau        kriyāyogī nigadyate       \U1 
% kṣamā vivekaṃ vairāgyaṃ śāntisaṃtoṣaniṣpṛhāḥ //  etatmuktiyuto yogī         kriyāyogī nigadyate //2// \U2
%--------------------------
% The text of the Printed Edition starts here ---> 
%--------------------------
\tl{kṣamā\app{\lem[wit={N1,D1,P,U1}]{viveka}\rdg[wit={E,U2}]{vivekaṃ}}vairāgyaṃ \note[type=philcomm, labelb=s6.z6a, lem={°kṣamā}]{\getsiglum{E} starts here.} śāntisantoṣa\app{\lem[wit={P}]{nispṛhāḥ}\rdg[wit={U2}]{°niṣpṛhāḥ}\rdg[wit={E,N1}]{°nispṛhā}\rdg[wit={D1}]{°nispṛhaḥ}\rdg[wit={U1}]{°niṣpṛhī}}/}\\
\tl{eta\skp{d}\app{
    \lem[wit={E,P,N1,D1,U1}, alt={yuktiyuto}]{\skm{d}-yuktiyuto}
    \rdg[wit={U2}]{muktiyuto}}
  \app{
    \lem[wit={E,P,N1,D1,U2}]{yogī}
    \rdg[wit={U1}]{yosau}}
  kriyāyogī nigadyate//2//}\\
\end{tlg}
    \end{edition}
    \begin{translation}
   \ekddiv{type=trans}
    \begin{tlate}\textbf{2.} Patience, discrimination, equanimity, peace, modesty, desireless: The \textit{yogī} who is endowed with these means is said to be a \textit{kriyāyogī}. \bigskip \bigskip \end{tlate}
    \end{translation}
    \begin{edition}
     \ekddiv{type=ed}
     \begin{tlg}
%-----------------------
% \om                                             \B
% \om                                             \L
% mātsaryaṃ mamatā māyā hiṃsā ca   madagarvitā /  \E
% mātsarya  mamatā māyā hiṃsāśā    madagarvitāḥ    \P
% mātsarya  mamatā māyā hiṃsāḥ //  madagarvatā /  \N1    -> the hiṃsā---''ḥ//'' in \nepal looks like a śā -> indicator that the others copied from \nepal? 
% mātsarya  mamatā māyā hiṃsāśā    madagarvatā /  \D1
% mātsaryaṃ mamatā māyā hiṃsāśā    madagarvatā /  \U1
% mātsaryaṃ mamatā māyā hiṃsāśā    madagarvatā /  \U2
%-----------------------
\tl{\app{\lem[wit={E,U1,U2}]{mātsaryaṃ}\rdg[wit={P,N1,D1}]{mātsarya}} mamatā māyā \app{\lem[wit={P,D1,U1,U2}]{hiṃsāśā}\rdg[wit={E}]{hiṃsā ca}\rdg[wit={N1}]{hiṃsāḥ}} madagarvatā/}\\
%-----------------------
% \om                                                   \B
% \om                                                   \L
% kāmakrodhabhayaṃ lajjā lobhamohau tathā śuciḥ //      \E
% kāmakrodhabhayaṃ lajjā lobhamohau tathā 'śuciḥ        \P
% kāmakrodhabhayaṃ lajjā lobhamohau tathā 'śuciḥ /      \N1    -> the hiṃsā---''ḥ//'' in \nepal looks like a śā -> indicator that the others copied from \nepal? 
% kāmakrodho bhayaṃ lajjā lobhamohau tathā 'śuciḥ //    \D1
% kāmakrodhau bhayaṃ lajjā lobhamohau tathā 'śuciḥ      \U1
% kāmakrodhau bhayaṃ lajjā lobhamohau tathā śuciḥ //3// \U2
%----------------------- 
\tl{kāma\app{\lem[wit={U1,U2}, alt={°krodhau}]{krodhau}\rdg[wit={E,P,N1}]{krodha°}\rdg[wit={D1}]{°krodho}} bhayaṃ lajjā lobhamohau tathā \app{\lem[wit={P,N1,D1,U1}]{'śuciḥ}\rdg[wit={E,U2}]{śuciḥ}}//3//}\\
\end{tlg}
    \end{edition}
    \begin{translation}
   \ekddiv{type=trans}
    \begin{tlate}\textbf{3.} Envy, selfishness, cheating, violence, desire and intoxication, pride, lust, anger, fear, laziness, greed, error and impurity. \bigskip \bigskip \end{tlate}
    \end{translation}
       \begin{edition}
     \ekddiv{type=ed}
      \begin{tlg}
%-----------------------
%  \om                                                           \B
%  atha dveṣo ghṛṇālasyaṃ bhrāṃtir   daṃbho kṣamā bhramaḥ //     \L
%  rāgadveṣau ghṛṇālasyaṃ bhrāntitvaṃ     mokṣamā bhramaḥ /      \E
%  rāgadveṣau ghṛṇālasyaṃ bhrāṃtir   ddaṃbhokaṣmā bhramaḥ        \P
%  rāgadveṣau ghṛṇālasyaṃ bhrāṃtir   daṃbho kṣamā bhramaḥ //4//  \N1   
%  rāgadveṣau ghṛṇālasyaṃ bhrāṃtir   debho  kṣamā bhramaḥ //     \D1
%  rāgadoṣau  ghṛṇālasyaṃ bhrāṃti    daṃbha kṣamī bhramaḥ 4      \U1
%  rāgadveṣau ghṛṇālasyaṃ bhrāṃtir   daṃbho kṣamā bhramaḥ //     \U2
%-----------------------
        \tl{\app{\lem[wit={E,P,N1,D1,U2}]{rāgadveṣau }\rdg[wit={U1}]{rāgadoṣau}\rdg[wit={L}]{athadveṣo}}\note[type=philcomm, labelb=s6.z13a, labele=s6.z13b, lem={rāga°}, labelb=3]{\getsiglum{L} starts here.} ghṛṇālasyaṃ
          \app{
            \lem[wit={P,L,N1,U2}, alt={bhraṃtir daṃbho}]{bhrantir-daṃbho}
            \rdg[wit={D1}]{bhrāṃtir debho}
            \rdg[wit={E}]{bhrāntitvaṃ}
            \rdg[wit={U1}]{bhrāṃti daṃbha}}
          \app{\lem[wit={L,N1,D1,U2}]{kṣamā bhramaḥ}\rdg[wit={E}]{mokṣamābhramaḥ}\rdg[wit={U1}]{°kṣamī bhramaḥ}}/}\\
%-----------------------
%  \om                                               \B
%  yasyai tāni na vidyaṃte kriyāyogī sa ucyate //    \L
%  yasyai tāni ca vidyante kriyāyogī sa ucyate 3     \E
%  yasyai tāni na vidyaṃte kriyāyogī sa ucyate       \P1
%  yasyai tāni na vidyaṃte kriyāyogī sa ucyate //    \N1   
%  yasyai tāni na vidyaṃte kriyāyogī sa ucyate //    \D1
%  yasyai tāni na vidyaṃte kriyāyogī sa ucyate       \U1
%  yasyai tāni na vidyaṃte kriyāyogī sa ucyate //4// \U2
%-----------------------
\tl{yasyaitāni \app{\lem[wit={P,L,N1,D1,U1,U2}]{na}\rdg[wit={E}]{ca}} vidyante kriyāyogī sa ucyate//4//}\\
\end{tlg}
    \end{edition}
    \begin{translation}
   \ekddiv{type=trans}
    \begin{tlate}\textbf{4.} Attachment and aversion, indignation and idleness, impatience and dizzyness: Whoever does not possess these is called a \textit{kriyāyogī}.\footnote{The source of the four verses seems to be unknown. It is possible that they stem from Rāmacandra himself.} \bigskip \bigskip \end{tlate}
    \end{translation}
\end{alignment}
\clearpage
\begin{alignment}[
    texts=edition[class="edition"];
    translation[class="translation"],
    ]
       \begin{edition}
     \ekddiv{type=ed}
      \begin{prose}
%-----------------------
%  \om                                                                                          \B
%  yasyāntaḥkaraṇe kṣamāvivekavairāgyaśāntisantoṣādīny                        utpadyante //     \E
%  yasyāṃtaḥkaraṇe kṣamāvivekavairāgyaśāṃtisaṃtoṣa         ityādīny           utpādyaṃte        \P
%  tasyāṃtaḥkaraṇe kṣamāvivekavairāgyaśāṃtisaṃtoṣa         ityādīnotpādyaṃte                    \L
%  yasyāṃtaḥkaraṇe kṣamāḥ vivekavairāgya / śāṃtisaṃtoṣa    ityādīni           utpādyaṃte        \N1   
%  yasyāṃtaḥkaraṇe kṣamā // vivekavairāgya // śāṃtisaṃtoṣa ityādīni           utpādyaṃte //     \D1
%  yasyāṃtaḥkaraṇe kṣamāvivekavairāgyaśāṃtisaṃtoṣa         ityādīna niraṃtaram   utyaṃte        \U1
%  yasyāṃtaḥkaraṇe kṣamāvivekavairāgyaśāṃtisaṃtoṣa         ityādayoniraṃtaraṃ utpādyaṃte        \U2
%-----------------------      
        yasyāntaḥkaraṇe
        \app{\lem[wit={E,P,L,D1,U1,U2},alt={kṣamā°}]{kṣamā}
          \rdg[wit={N1}]{kṣamāḥ}}
        vivekavairāgyaśānti\app{
          \lem[wit={P,N1,D1}, alt={°santoṣa ityādīny}]{santoṣa ityādīny\skm{-u}} %the°-problem
          \rdg[wit={E}]{°santoṣādīny}
          \rdg[wit={L}]{°santoṣa ity ādīno°}
          \rdg[wit={U1}]{°santoṣa ity ādīna niraṃtaram}
          \rdg[wit={U2}]{°santoṣa ity ādayo niraṃtaraṃ}}\app{\lem[wit={P,N1,D1,U2}]{\skp{-}utpādyante}
          \rdg[wit={E}]{utpadyante}
          \rdg[wit={U1}]{utyaṃte}}/ \\
%-----------------------
% \om \oxford
%  sa eva bahukriyāyogī kathyate /      \E
%  sa eva bahukriyāyogī kathyate        \P
%  sa eva bahukriyāyogī kathyate //     \L
%  sa eva bahukriyāyogī kathyate /      \N1
%  sa eva bahukriyāyogā sa kathyate //  \D1
%  sa eva bahukriyāyogī kathyate /      \U1
%  sa eva bahukriyāyogī tkacyate /      \U2
%----------------------- 
        sa eva
        \app{\lem[wit={E,P,L,N1,U1,U2}]{bahukriyāyogī}
          \rdg[wit={D1}]{bahukriyāyogā}}
        \app{\lem[wit={E,P,L,N1,U1}]{kathyate}
          \rdg[wit={D1}]{sa kathyate}
          \rdg[wit={U2}]{tkacyate}}/
      \end{prose}
    \end{edition}
    \begin{translation}
   \ekddiv{type=trans}
    \begin{tlate} Patience, discrimination, equanimity, peace, contentment etc. are generated in his mind. He alone is called a \textit{yogī} of many actions (\textit{bahukriyāyogī})\footnote{The term \textit{bahukriyāyogī} seems to be unique in the whole yoga literature.}. \bigskip\end{tlate}
    \end{translation}  
    \begin{edition}
    \ekddiv{type=ed}
    \begin{prose}
%-----------------------
% \om \B
%                kāpaṭyaṃ      vittaṃ   hiṃsā    tṛṣṇā    mātsaryam    ahaṃkāraḥ    roṣaḥ kṣayaṃ   lajjālobhamohā      aśucitvaṃ                       pākhaṃḍatvaṃ       bhrāntiḥ indriyavikāraḥ kāmaḥ          ete yasya manasi pratidinaṃ vyunā bhavanti /  \E
%                kāpaṭyaṃ      vittaṃ   hiṃsā    tṛṣṇā    mātsaryaṃ    ahaṃkāraḥ    roṣo bhayaṃ    lajjā lobhaḥ mohaḥ  aśucitvaṃ rāgaḥdveṣaḥ   ālasyaṃ pākhaṃḍitvaṃ       bhrāṃtiḥ indriyaṃ vikāraḥ kāmaḥ        ete yasya manasi pratidinaṃ nyunā bhavanti   \P
%                kāpayaṃ     //vitaṃ // hiṃsā // tṛṣṇā // mātsaryaṃ // ahaṃkāraḥ // roṣo bhayaṃ // lajjālobhaḥ // moha aśucitvaṃ // rājadveṣa  alasyaṃ // pākhaṃḍitvaṃ // bhrāṃtiḥ // itivikāraḥ // kāmaḥ        eta yasya manasi pratidinaṃ nyunā bhavaṃti//\L
% yasyāṃtakaraṇe kapatyaṃ māyā vitvaṃ   hiṃsā    tṛṣṇā    mātsaryaṃ    ahaṃkāraḥ    roṣobhayaṃ     lajjā // lobhamohā  asucitvaṃ rāgadveṣaḥ // alasyaṃ pāṣaṃḍitvaṃ        bhraṃtiḥ / iṃdriyaivikāraḥ / kāmaḥ     ete yasya manasi pratidinaṃ nyunā bhavaīti / \N1
%                kāpaṭyaṃ māya vitvaṃ   hiṃsā    tṛṣṇā    mātsarya     ahaṃkāraḥ    roṣobhayaṃ     lajjā // lobhamohā  asucitvaṃ rāgadveṣaḥ // ālasyaṃ pāṣaṃḍitvaṃ        bhraṃtiḥ // iṃdriyavikāraḥ // kāmaḥ // ete yasya manasi pratidinaṃ nyunā bhavaṃti //  \D1
%                kāpachaṃ yāya vitvaṃ   hiṃsā    tṛṣṇā    mātsarya     ahaṃkāraḥ    roṣaḥ bhayaṃ   lajā lobhamohā      aśucitvaṃ rāgadveṣaḥ    ālasyaṃ pākhaṃḍitvaṃ       bhraṃtiḥ iṃdriyavīkāraḥ    kāmaḥ       rāte yasya manasi pratidinaṃ nyunā bhavaṃti //      \U1
%                kāpaṭyaṃ pāpā titaṃ    hiṃsā    tṛṣṇā    mātsaryaṃ // ahaṃkāraḥ    roṣobhayaṃ     lajjā ----mohā      aśucitvaṃ rāgadveṣaḥ    ālasyaṃ pākhaṃḍitvaṃ //    bhraṃtiḥ iṃdriyavikāraḥ //-----        etate yasya manasi pratidinaṃ nyunā bhavaṃti // \U2
%-----------------------
      \app{\lem[wit={E,P,D1,U2}]{kāpaṭyaṃ}
        \rdg[wit={N1}]{yasyāntaḥkaraṇe kapatyaṃ}
        \rdg[wit={L}]{kāpayaṃ}
        \rdg[wit={U1}]{kāpachaṃ}}
      \app{\lem[wit={N1}]{māyā}
        \rdg[wit={D1}]{māya}
        \rdg[wit={U1}]{yāya}
        \rdg[wit={U2}]{pāpa}
        \rdg[wit={E,P,L}]{\om}}
        %\rdg[wit={E,P,L}]{\textbf{omitted in}}}
      \app{\lem[wit={E,P}]{vittaṃ}
        \rdg[wit={L}]{vitaṃ}
        \rdg[wit={N1,D1,U1}]{vitvaṃ}
        \rdg[wit={U2}]{titaṃ}}
      hiṃsā tṛṣṇā
      \app{\lem[wit={E}, alt={mātsaryam}]{mātsarya\skm{ma}}
        \rdg[wit={P,L,N1,U2}]{mātsaryaṃ}
        \rdg[wit={D1,U1}]{mātsarya}}\skp{m-a}haṃkāraḥ
      \app{\lem[wit={P,L,N1,D1,U2}]{roṣobhayaṃ}
        \rdg[wit={E,U1}]{roṣaḥ bhayaṃ}}
      \app{\lem[wit={E,P,L,N1,D1,U2}]{lajjā}
        \rdg[wit={U1}]{lajā}}
      \app{\lem[wit={E,N1,D1,U1}]{lobhamohā}
        \rdg[wit={P,L}]{lobhaḥ mohaḥ}
        \rdg[wit={U2}]{mohā}}
      aśucitvaṃ
      \app{
        \lem[type=emendation, resp=egoscr]{rāgo dveṣa }
        \rdg[wit={P}]{\korr rāgaḥ dveṣaḥ}
        \rdg[wit={N1,D1,U1,U2}]{rāgadveṣaḥ}
        \rdg[wit={L}]{rājadveṣa}\rdg[wit={E}]{\om}}\note[type=philcomm, labelb=s8.z2a, lem={rāgo dveṣaḥ}]{I conjectured to \textit{rāgo dveṣaḥ} to provide a sentence with correct grammar. Another possible conjecture would be to read \textit{rāgadveṣau}.}
      \app{\lem[wit={P,L,N1,D1,U1,U2}]{ālasyaṃ}
        \rdg[wit={E}]{\om}}
      \app{\lem[wit={P,L,U1,U2}]{pākhaṃḍitvaṃ}
        \rdg[wit={D1,N1}]{pāṣaṃḍitvaṃ}
        \rdg[wit={E}]{pākhaṃḍatvaṃ}} bhrānti\skp{r-}\app{
        \lem[wit={E,N1,D1,U2}, alt={indriyavikāraḥ}]{\skm{r}indriyavikāraḥ}
        \rdg[wit={U1}]{iṃdriyavīkāraḥ}
        \rdg[wit={P}]{iṃdriyaṃ vīkāraḥ}
        \rdg[wit={L}]{itivikāraḥ}}
      \app{\lem[wit={E,P,L,N1,D1,U1}]{kāmaḥ}
        \rdg[wit={U2}]{\om}}
      \app{\lem[wit={E,P,D1,N1}]{ete}
        \rdg[wit={L}]{eta}\rdg[wit={U1}]{rāte}
        \rdg[wit={U2}]{etate}} yasya manasi pradidinaṃ nyūna
      \app{\lem[wit={E,P,L,D1,U1,U2}]{bhavanti}
        \rdg[wit={N1}]{bhavīti}}/ \\
%-----------------------       
%sa eva bahukriyāyogī kathyate // \E
%sa eva bahukriyāyogī kathyate // \P
%sa eva bahukriyāyogī kathyate // \L
%sa eva bahukriyāyogī kathyate // \N1
%sa eva bahukiyāyogī kathyate //  \D1
%sa eva bahukiyāyogī kathyaṃte // \U1
%sa eva bahukiyāyogī kathyaṃte // \U2
%-----------------------     
      sa eva bahukriyāyogī
      \app{\lem[wit={E,P,L,N1,D1,U2}]{kathyate}
        \rdg[wit={U1}]{kathyaṃte}}//  
    \end{prose}
    \end{edition}
    \begin{translation}
   \ekddiv{type=trans}
    \begin{tlate}Fraud, illusion, property,violence, craving, envy, ego, anger, anxiety, shame, greed, error, impurity, attachment, aversion, idleness, heterodoxy, false view, affection of the senses, sexual desire: He who diminishes these from day to day in is mind, he alone is called a yogī of many actions (\textit{bahukriyāyogī}).\end{tlate}
    \end{translation}
    \end{alignment}
\clearpage
\begin{alignment}[
    texts=edition[class="edition"];
    translation[class="translation"],
  ]
  \begin{edition}
    \ekddiv{type=ed}
    \bigskip
    \centerline{\textrm{\small{[Varieties of \textit{rājayoga}: Siddhakuṇḍalinīyoga and Mantrayoga]}}}
    \bigskip
     \begin{prose}
%-----------------------   
% \om                                   \B
%idānīṃ rājayogasya bhedāḥ kathyante // \E
%idānīṃ rājayogasya bhedāḥ kathyaṃte    \P
%idānīṃ rājayogasya bhedāḥ              \L
%idānīṃ rājayogasya bhedāḥ kathyaṃte    \N1
%idānīṃ rājayogasya bhedāḥ kathyaṃte // \D1     
% \om                                   \U1
%idānīṃ rājayogasya bhedāḥ kathyaṃte // \U2
%-----------------------   
       idānīṃ rājayogasya bhedāḥ
       \app{\lem[wit={E,P,N1,D1,U2}]{kathyante}
         \rdg[wit={L}]{\om}}/\note[type=philcomm, labelb=s8.z5a, lem={kathyante}]{The whole sentence is \om in \getsiglum{U1}.}
 %-----------------------
%te ke    \E
%te ke    \P
%te ke    \L
%ke te // \D1
%ke te /  \N1 
%ke te    \U1
%te ke    \U2
%-----------------------
       \app{\lem[wit={D1,N1,U1}]{ke te}
         \rdg[wit={E,P,L,U2}]{te ke}}/
%-----------------------
%\om                                       \B
%ekaḥ siddhakuṇḍalinīyogaḥ / mantrayogaḥ / \E
%ekaḥ siddhakuṃḍaṃliṃ yogaḥ maṃtrayogaḥ    \P
%ekaḥ siddhakuṇḍalanīyoga /                \L 
%ekaḥ siddhakuṇḍalinīyogaḥ maṃtrayogaḥ /   \N1
%ekaḥ siddhakuṃḍalanīyogaḥ mantrayogaḥ //  \D1 
%ekaḥ siddhakuṇḍaliniyogaḥ mantrayogaḥ     \U1
%ekaḥ siddhakuṇḍalinīyoga // mantrayogaḥ   \U2
%-----------------------
       ekaḥ
       \app{\lem[wit={E,N1}]{siddhakuṇḍalinīyogaḥ}
         \rdg[wit={U1}]{siddhakuṇḍalinīyogaḥ}
         \rdg[wit={U2}]{siddhakuṇḍalinīyoga}
         \rdg[wit={D1}]{siddhakuṃḍalanīyogaḥ}
         \rdg[wit={P}]{siddhakuṃḍaṃliṃ yogaḥ}}
       \app{\lem[wit={E,P,N1,D1,U1,U2}]{mantrayogaḥ}
         \rdg[wit={L}]{\om}}/ \note[type=philcomm, labelb=s8.z5aa, lem={mantrayogaḥ}]{The sudden appearance of \textit{mantrayoga} seems very odd. Esspecially considering that this section of the text doesn't mention the practice of mantra at all. It might me a mistake, or a later insertion. However, the most reliable witnesses preserve this reading exept of \getsiglum{L}.}
%-----------------------
% \om                         \B
%astu rājayogaḥ kathyate /    \E
%amū rājayogau kathyete       \P
%amū rājayogau kathyate //    \L
%amū rājayogau kathyate       \N1
%amū rājayogau kathyate //    \D1 
%amū rājayogau kathyate       \U1
%amū rājayogau kathyaṃte //   \U2
%-----------------------
       \app{\lem[wit={P,L,N1,D1,U1,U2}]{amū}
         \rdg[wit={E}]{astu}}
       \app{\lem[wit={P,L,N1,D1,U1,U2}]{rājayogau}
         \rdg[wit={E}]{rājayogaḥ}}
       \app{\lem[wit={P}]{kathyete}
         \rdg[wit={E,L,N1,D1,U1}]{kathyate}
         \rdg[wit={U2}]{kathyaṃte}}/
%-----------------------
% \om                                                              \B
%mūlakandasthāne    ekā tejorūpā    mahānāḍī varttate /            \E
%mūlaṃ kaṃdasthāne  ekā tejorūpā    mahānāḍī varttate              \P
%mūlakaṃdasthāne    ekā tejorūpā    mahānāḍī vartate               \L
%mūlakaṃdasthāne    eka tejorūpā    mahānāḍī varttate /            \N1
%mūlakaṃdasthāne    ekā tejorūpā    mahānāḍī varttate //           \D1 
%mūlakaṃdasthāne    ekā tejorūpā    mahānāḍī vartate /             \U1
%mūlakaṃdasthāne // ekā tejorūpā // mahānāḍī pravarttate /         \U2
%-----------------------
       \app{\lem[wit={E,L,N1,D1,U1,U2}]{mūlakandasthāne}
         \rdg[wit={P}]{mūlaṃ kaṃdasthāne}}
       \app{\lem[wit={E,P,L,D1,U1,U2}]{ekā}
         \rdg[wit={N1}]{eka}}
       tejorūpā mahānāḍī
       \app{\lem[wit={E,P,L,N1,D1,U1}]{vartate}
         \rdg[wit={U2}]{pravartate}}/
%-----------------------
% \om                                                            \B
%iyamekanāḍī /  iḍāpiṃgalāsuṣumṇā      etān bhedān prāpnoti /    \E
%iyaṃ ekanāḍī   iḍāpiṃgalāsuṣumṇā      etān bhedān prāpnoti      \P
%trayaṃ kā nāḍī iḍāpiṃgalāsuṣumnā //   etān bhedān prāpnoti      \L
%iyaṃ ekā nāḍī  iḍāpiṃgalāsuṣumnān /   ete  bhedān prāpnoti      \N1
%iyaṃ ekā nāḍī  iḍāpiṃgalasuṣumnān //  ete  bhedān prāpnoti      \D1 
%iyaṃ ekā nāḍī  iḍāpiṃgalāsuṣumnā      etān bhedān prāpnoti      \U1
%iyaṃ eka nāḍī  iḍāpiṃgalāsuṣumṇā      etān bhegān prāpnoti      \U2
%-----------------------
   \app{
         \lem[wit={E},alt={iyam}]{iyam\skm{-e}}
         \rdg[wit={E,P,N1,D1,U1,U1}]{iyaṃ}
         \rdg[wit={L}]{trayaṃ}}\app{
         \lem[wit={N1,D1,U1,U2}, alt={ekā}]{\skp{-e}kā}
         \rdg[wit={E,P}]{eka}
         \rdg[wit={L}]{kā}}
       nāḍī iḍāpiṅgalā
       \app{
         \lem[wit={N1,D1},alt={°suṣumṇān}]{suṣumṇān}
    \rdg[wit={E,P,U1,U2}]{suṣumṇā}}
       \app{
         \lem[wit={E,P,L,U1,U2}]{etān}
    \rdg[wit={N1,D1}]{ete}}
  bhedān prāpnoti/\end{prose}
    \end{edition}
    \begin{translation}
      \ekddiv{type=trans}
        \bigskip
    \centerline{\textrm{\small{[Varieties of \textit{rājayoga}: Siddhakuṇḍalinīyoga and Mantrayoga]}}}
    \bigskip
    \begin{tlate}Now varieties of \textit{rājayoga} will be described. Which are these? One is \textit{siddhakuṇḍalinīyoga} [and one] is \textit{mantrayoga}. These two rājayogas are described [in the following]. At the location of the root-bulb exists one major vessel in the form of energy. This single vessel splits up into these openings which are \textit{iḍā}, \textit{piṅgalā} and \textit{suṣumnā}. \end{tlate}
    \end{translation}
    \begin{edition}
      \ekddiv{type=ed}
      \begin{prose}
%-----------------------
% \om                                                      \oxford
%vāmabhāge candrarūpā iḍā nāḍī varttate /      \E
%vāmabhāge caṃdrarūpā iḍā nāḍī varttate        \P
%vāmabhāge caṃdrarūpā iḍā nāḍī varttate //     \L
%vāmabhāge caṃdrarūpā iḍā nāḍī varttate /      \N1
%vāmabhāge caṃdrarūpā iḍā nāḍī varttate /      \D1 
%vāmabhāge caṃdrarūpā iḍā nāḍī vartate         \U1
%vāmabhāge caṃdrarūpā     nāḍī pravarttate //  \U2
%-----------------------
        vāmabhāge candrarūpā
        \app{\lem[wit={E,P,L,N1,D1,U1}]{iḍā}
          \rdg[wit={U2}]{\om}} nāḍī
        \app{\lem[wit={E,P,L,N1,D1,U1}]{vartate}
          \rdg[wit={U2}]{pravarttate}}/
%-----------------------
% \om                                                \B
%dakṣiṇabhāge sūryarūpā piṅgalā  nāḍī    varttate /  \E
%dakṣiṇabhāge sūryarūpā piṃgalā  nāḍī    varttate    \P
%dakṣiṇabhāge sūryarūpā piṃgalā  nāḍī    varttate // \L
%dakṣiṇabhāge sūryarūpā piṃgalā  nāḍī    varttate // \N1
%dakṣiṇabhāge sūryarūpā piṃgalā  nāḍī    varttate // \D1 
%dakṣiṇe bhāge sūryarūpā piṃgalā nāḍī    vartate     \U1
%dakṣiṇabhāge sūryarūpā piṃgalā  nāḍī pravartate //  \U2
%-----------------------
        \app{\lem[wit={E,P,L,N1,D1,U2}]{dakṣiṇabhāge}
          \rdg[wit={U1}]{dakṣiṇe bhāge}}
        sūryarūpā piṅgalā nāḍī
        \app{\lem[wit={E,P,L,N1,D1,U1}]{vartate}
          \rdg[wit={U2}]{pravarttate}}/
%-----------------------
% \om                                                                   \B
%madhyamārge `tisūkṣmā padminī taṃtusamākārā  koṭividyutsamaprabhā      \E
%madhyamārge `tisūkṣmā padmanī taṃtusamākāra! koṭividyutsamaprabhā      \P
%madhyamārge `tisūkṣmā padmanī taṃtusamākārā  koṭividyutsamaprabhā      \L
%madhyamārge atisūkṣmā padmanī taṃtusamākārā  koṭividyutsamaprabhā //   \N1
%madhyarge   atisūkṣmā padminī taṃtusamākārā  koṭividyutsamaprabhā //   \D1 
%madhyamārge atisūkṣmā padminī taṃtusamākārā  koṭividyutsamaprabaḥ      \U1
%madhyamārge  tisūkṣmā padminī taṃtusamākārā  koṭividyutsamaprabhā //   \U2
%-----------------------         
        \app{\lem[wit={E,P,L,N1,U1,U2}]{madhyamārge}
          \rdg[wit={D1}]{madhyarge}}
        'tisūkṣmā
        \app{\lem[wit={E,D1,U1,U2}]{padminī}
          \rdg[wit={P,L,N1}]{padmanī}}/
        \app{\lem[wit={E,L,N1,D1,U1,U2}]{tantusamākārā}
          \rdg[wit={P}]{taṃtusamākāra°}}
        \app{\lem[wit={E,P,L,N1,D1,U2},alt={°prabhā}]{koṭividyutsamaprabhā}
          \rdg[wit={U1}]{°prabhaḥ}}/
      \end{prose}
    \end{edition}
    \begin{translation}
      \ekddiv{type=trans}
      \begin{tlate}On the left side is the iḍā-channel, being a resemblence of the moon. On the right side exists the piṅgalā-channel, being a resemblence of the sun. Within the middle path is a lotuspond being very subtle. [It is] made from a web of light [and it] shines like a thousand lightnings. \end{tlate}
    \end{translation}
    \begin{edition}
      \ekddiv{type=ed}
      \begin{prose}
%-----------------------
%\om                                                                                                                                                                 \B
%bhuktimuktipradā                                     'syā jñānotpattau satyaṃ puruṣaḥ sarvajño  bhavati      idānīṃ suṣumṇāyāṃ jñānotpattāv---upāyāḥ kathyante      \E
%bhuktimuktidā                                        asyā jñānotpattau satyāṃ puruṣaḥ sarvajño  bhavati      idānīṃ suṣumṇāyā  jñānotpattau   upāyāḥ kathyaṃte      \P
%bhuktimuktipradā //                                  asyā jñānotpattau satyāṃ puruṣaḥ sarvajño  bhavati   // idānīṃ suṣumnā    jñānotpattau   upāyaḥ kathyate //    \L
%bhuktimukti--------------------------------------------------dotpanne  sati---puruṣaḥ sarrvajño bhavati    / idānīṃ suṣumnāyāḥ jñanotpanno    'pāyāḥ kathyaṃte //   \N1
%bhuktimukti--------------------------------------------------dotpanne  sati---puruṣaḥ sarrvajño bhavati    / idānīṃ suṣumnāyāḥ jñanotpattau   upāyāḥ kathyaṃte //   \D1 
%bhuktimukti--------------------------------------------------dotpanne  sati---puruṣaḥ sarrvajño bhavati    / idānīṃ  suṣumnāya-jñanotpattau   upāyāḥ kathyaṃte //   \U1
%bhuktimuktidā śivarūpiṇī suṣumṇā nāḍī pravarttate // asyā jñānotpattau satyāṃ puruṣa--sar-vajño bhavati   // idānīṃ suṣumṇāyā  jñānotpattau   upāyā  kathyaṃte //   \U2
%-----------------------
\app{\lem[wit={P,N1,D1,U1,U2}]{bhuktimuktidā}
  \rdg[wit={E,L}]{bhuktimuktipradā}
  \rdg[wit={N1,D1,U1}]{bhuktimukti}}
 % \rdg[wit={U2}]{bhuktimuktidā śivarūpiṇī suṣumṇā nāḍī pravarttate}} %Lesart oder einfach zusätzliches Material? 
%\textcolor{red}{śivarūpiṇī suṣumṇā nāḍī pravarttate/}
\extra{śivarūpiṇī suṣumṇā nāḍī pravarttate/}
    \app{\lem[resp=egoscr, type=emendation]{'syāṃ}  
      \rdg[wit={E,P,L,U2}]{\korr asyā}
      \rdg[wit={N1,D1,U1}]{\om}}
    \app{\lem[wit={E,P,L,U2}]{jñānotpattau}
      \rdg[wit={N1,D1,U1}]{utpanne}}
    \app{\lem[wit={P,L,U2}]{satyāṃ}
      \rdg[wit={E}]{satyaṃ}
      \rdg[wit={N1,D1,U1}]{sati}}
    sarvajño bhavati/ idānīṃ
    \app{\lem[wit={E}]{suṣumṇāyāṃ}
      \rdg[wit={P,U2}]{suṣumṇāyā}
      \rdg[wit={U1}]{suṣumnāya°}
      \rdg[wit={N1,D1}]{suṣumṇāyāḥ}
      \rdg[wit={L}]{suṣumnā°}}
    \app{\lem[wit={E}, alt={jñānotpattāv-upāyāḥ}]{jñānotpattāv-upāyāḥ}
      \rdg[wit={P,L,D1,U1}]{jñānotpattau upāyāḥ}
      \rdg[wit={U2}]{jñānotpattau upāyā}
      \rdg[wit={N1}]{jñānotpanno 'pāyāḥ}}
    \app{\lem[wit={E,P,N1,D1,U1,U2}]{kathyante}
      \rdg[wit={L}]{kathyate}}//
  \end{prose}
  \end{edition}
   \begin{translation}
    \ekddiv{type=trans}
      \begin{tlate}She \extra{emerges as the central channel, assuming the form of benevolence (\textit{śiva}),} is the bestower of enjoyment and liberation. While abiding in (\textit{satyāṃ}) her (\textit{asyāṃ}) knowledge arises [to the point of which] the person becomes all-knowing. The means for the genesis of knowledge in the central channel will now be described.\footnote{It is not clear if the list given at the beginning of the text codifying the fifteen \textit{yoga}s belongs to the original text or was a later addition by a another hand. One primary reason for this suspicion is that the structure of the \textit{yoga}s in the text does not equal the list. The text begins with a description of \textit{kriyāyoga} and continues to describe \textit{siddhakuṇḍaliniyoga} and somewhat suprisingly mentions \textit{mantrayoga} in the same breath. One starts wondering why the structure of the text does not follow the codification. However the mention of \textit{jñānotpattau upāyaḥ} might be a clue why the second \textit{yoga} in the list might be \textit{jñānayoga}. So far it seems to me that there are three options or a combination of these to explain these apparent inconsistencies: 1. The text is highly corrupted. 2. The codification was a later addition of another hand. 3. The term \textit{jñānayoga} is listed due to the results of \textit{siddhakuṇḍalinīyoga}, which is the generation of knowledge due to the practice of a certain \textit{yoga} involving the central channel, as mentioned in this section of the text.}\end{tlate}
   \end{translation}
   \end{alignment}
\clearpage
\begin{alignment}[
    texts=edition[class="edition"];
    translation[class="translation"],
  ]
   \begin{edition}
     \ekddiv{type=ed}
     \bigskip
    \centerline{\textrm{\small{[Description of the first Cakra]}}}
    \bigskip
    \begin{prose}
%-----------------------
%\om                                       \B
%ādau caturdalaṃ mūlaṃ cakraṃ varttate /   \E
%ādau caturddalaṃ mūlaṃ cakraṃ varttate /  \P
%ādau caturdalamūlacakraṃ varttate //      \L
%ādau caturdalaṃ mūlacakraṃ varttate       \N1
%ādau caturdalaṃ mūlacakraṃ varttate       \D1 
%ādau caturdalaṃ mūlaṃ cakraṃ vartate      \U1
%ādau caturdalaṃ mūlacakraṃ pravarttate // \U2
%-----------------------
      ādau \app{\lem[wit={N1,D1,U2}]{caturdalaṃ mūlacakraṃ}
        \rdg[wit={E,P,U1}]{caturdalaṃ mūlaṃ cakraṃ}
        \rdg[wit={L}]{caturdalamūlacakraṃ}}
      \app{\lem[wit={E,P,L,N1,D1,U1}]{vartate}
        \rdg[wit={U2}]{pravartate}}/
%-----------------------
%
%\om                                       \B
%prathamādhāracakraṃ varttate / gudāsthānaṃ    raktavarṇaṃ    gaṇeśadaivataṃ    siddhibuddhiśaktimuṣakavāhanam       kurmaṛṣiḥ /  ākuṃcamudrā /    apānavāyuḥ                                   caturdaleṣu     rajaḥsattvatamomanāṃsi /  vaṃ śaṃ ṣaṃ saṃ    madhyatrikoṇe triśikhāt    tanmadhye trikoṇākāraṃ kāmapīthaṃ varttate//    \E
%prathamaṃ ādhāracakraṃ         gudāsthānaṃ    raktavarṇaṃ    gaṇeśāṃ daivataṃ  siddhibuddhiśaktir mukhako vāhanam   kurmaṛṣiḥ    ākuṃcanamudrā    apānavāyuś-----------------------------------caturddaleṣu    rajaḥsattvatamomanāṃsi    vaṃ śaṃ ṣaṃ saṃ    madhyatrikoṇe triśikhā     tanmadhye trikoṇākāraṃ kāmapīthaṃ varttate //   \P
%prathamaṃ ādhāracakraṃ         gudāsthānaṃ    raktavarṇaṃ    gaṇeśadaivataṃ    siddhibuddhiśaktimuṣako vāhanaṃ //   kurmaṛṣiḥ    ākuṃcanamudrā    apānavāyuḥ                                   caturddaleṣu    rajaḥsattvatamomanāṃsi // vaṃ śaṃ ṣaṃ saṃ    madhyatrikoṇe triśikhā     tanmadhyatrikoṇākāraṃ kāmapīthaṃ vartate        \L
%prathamaṃ ādhāracakraṃ         gudāsthānaṃ // raktavarṇaṃ // gaṇeśadaivataṃ // siddhibuddhiśaktiḥ muṣako vāhanaṃ // kurmaṛṣiḥ // ākuṃcanamudrā // apānavāyu // umīrkalā // ojasvinīdhāraṇā // caturddaleṣu // rajaḥsattvatamomanāṃsi //  vaṃ śaṃ ṣaṃ saṃ // madhyatrikoṇe trirekhā //  tanmadhye trikoṇākāraṃ kāmapīthaṃ varttate //   \U2    
%---------------------------------------------------------------------------------------------------------------------------------------------------------------------------------------------------------------------------------------------------------------------------------------tanmadhyatrikoṇākāraṃ kāmapiṭhaṃ varttate /   \N1
%---------------------------------------------------------------------------------------------------------------------------------------------------------------------------------------------------------------------------------------------------------------------------------------tanmadhye trikoṇākāraṃ kāmapiṭhaṃ varttate /  \D1 
      %---------------------------------------------------------------------------------------------------------------------------------------------------------------------------------------------------------------------------------------------------------------------------------------tanmadhye trikoṇākāraṃ kāmapiṭhaṃ varttate /   \U2
%-----------------------
                  \extra{\app{\lem[wit={P,L,U2}]{prathamaṃ ādhāracakraṃ}
                 \rdg[wit={E}]{prathamādhāracakraṃ vartate}}/
                  gudāsthānaṃ/ raktavarṇaṃ/
            \app{\lem[wit={E,L,U2}]{gaṇeśadaivataṃ}
                 \rdg[wit={P}]{gaṇeśāṃ daivataṃ}}
            \app{\lem[type=emendation, resp=egoscr]{siddhibuddhiśaktiṃ muṣako vāhanaṃ} %Emendation!!!
                 \rdg[wit={E}]{\korr siddhibuddhiśaktimuṣakavāhanam}
                 \rdg[wit={P}]{siddhibuddhiśaktir mukhako vāhanam}
                 \rdg[wit={L}]{siddhibuddhiśaktimuṣako vāhanaṃ}
                 \rdg[wit={U2}]{siddhibuddhiśaktiḥ muṣako vāhanaṃ}}/
kurmaṛṣiḥ /
            \app{\lem[wit={P,L,U2}]{ākuñcanamudrā} 
              \rdg[wit={E}]{ākuṃcamudrā}}/
            \app{\lem[wit={E,L}]{apānavāyuḥ}
                 \rdg[wit={P}]{°vāyuś}
                 \rdg[wit={U2}]{°vāyu}}/ 
            \extra{umīrkalā/ ojasvinīdhāraṇā/} caturdaleṣu/ rajaḥsattvatamomanāṃsi/ vaṃ śaṃ ṣaṃ saṃ/ madhyatrikoṇe
            \app{\lem[wit={P,L}]{triśikhā}
                 \rdg[wit={E}]{triśikhāt}
                 \rdg[wit={U2}]{trirekhā}}/}
            \app{\lem[wit={E,P,D1,U1,U2}]{tanmadhye}
                 \rdg[wit={L,N1}]{tanmadhya}}
            trikoṇākāraṃ kāmapiṭhaṃ vartate/\note[type=philcomm, labelb=s10.zx, lem={prathamaṃ ... triśikhā}]{The whole section from \textit{prathamaṃ} to \textit{triśikhā} is missing in \getsiglum{N1},\getsiglum{D1} and \getsiglum{U1}.}
%-----------------------
 %\om                                                     \B
%tatpīṭhamadhye 'gniśikhākāraikā mūrtir varttate /        \E
%tatpīṭhamadhye magniśikhākārā ekā mūrtir varttate /      \P
%tatpīṭhamadhye   jniśikhāka!rāṇakā mūrti varttate //     \L
%tatpīṭhamadhye  agniśikhākārā ekā mūrttir varttate //    \N1
%tatpīṭhamadhye  agniśikhākārā ekā mūrttir varttate //    \D1 
%tatpīṭhamadhye  agniśikhākārā ekā mūrttir varttate //    \U1
%tatpīṭhamadhye  agniśikhākārā ekā mūrttirasmi      //    \U2
%-----------------------
  tatpīṭhamadhye
\app{\lem[wit={E}]{'gniśikhākāraikā}
  \rdg[wit={N1,D1,U1,U2}]{agniśikhākārā ekā}
  \rdg[wit={P}]{magniśikhākārā ekā}
  \rdg[wit={L}]{jñiśikhākarāṇakā}}
murti\skp{r-}\app{\lem[wit={E,P,L,N1,D1,U1}, alt={vartate}]{\skm{r}vartate}
  \rdg[wit={U2}]{asmi}}/
%-----------------------%
%\om                                       \oxford
%tasyāḥ mūrtirdhyānakāraṇāt   sakalaśāstrakāvya-nāṭakādi-sakalavāṅmayaṃ vinābhyāsena puruṣasya manomadhye sphurati,     \E
%tasyā mūrter dhyānakaraṇāt   sakalaśāstrakāvya-nāṭakādi-sakalavāṅmayaṃ vinābhyāsena puruṣasya manomadhye sphurati      \P
%tasyā mūrtir dhyānakāraṇāt   sakalaśāstrakāvya-nāṭakādi //----vāṅmayaṃ vinābhyāsena puruṣasya manomadhye sphuraṃti!    \L
%tasyāḥ mūrter dhyānakaraṇāt  sakalaśāstrakāvya-nāṭakādi-sakalavāgmayaṃ vinābhyāsena puruṣasya manomadhye sphurati      \N1
%tasyāḥ mūrter dhyānakaraṇāt  sakalaśāstrakāvya-nāṭakādi-sakalavāgmayaṃ vinābhyāsena puruṣasya manomadhye sphurati      \D1 
%tasyā  mūrtair dhyānakaraṇāt sakalaśāstrakāvya-nāṭakādi-sakalavāgmayaṃ vinābhyāsena puruṣasya manomadhye sphurati      \U1
%tasyā          dhyānakaraṇāt sakalaśāstrakāvya-nāṭakādi-sakalavāṅmayaṃ vinābhyāsena puruṣasya manomadhye sphurati // asya bahir mānaṃdā // yogānaṃdā virānaṃdā // uparamānaṃdā // ajapājapa śāt // 600 // ghaṭi 1 palāni 40 // \U2
%-----------------------
\app{\lem[wit={P,L,U1,U2}]{tasyā}
  \rdg[wit={E,N1,D1}]{tasyāḥ}}
\app{\lem[wit={P,N1,D1}, alt={mūrter}]{mūrte\skp{r}\skm{r-dhyā}}
  \rdg[wit={E,L}]{mūrtir}
  \rdg[wit={U1}]{mūrtair}
  \rdg[wit={U2}]{\om}}\skp{-dhyā}nakaraṇāt sakalaśāstrakāvyanāṭakādi
  \app{\lem[wit={E,P,N1,D1,U1,U2}, alt={°sakala}]{sakala}
    \rdg[wit={L}]{\om}}vāṅmayaṃ vinābhyāsena puruṣasya manomadhye
  \app{\lem[wit={E,P,N1,D1,U1,U2}]{sphurati}
    \rdg[wit={L}]{sphuraṃti}}/
 % \rdg[wit={U2}]{sphurati // asya bahir mānaṃdā // yogānaṃdā virānaṃdā // uparamānaṃdā // ajapājapaśāt // 600 // ghaṭi 1 palāni 40 //}} / % Lesart oder zusätzliches Material? 
  \extra{asya bahir-mānandā/ yogānandā virānandā/ uparamānandā/}
    \end{prose}
   \end{edition}
\begin{translation}
  \ekddiv{type=trans}
  \bigskip
    \centerline{\textrm{\small{[Description of the first Cakra]}}}
    \bigskip
 \begin{tlate}At the beginning [of the central channel?] exists the root-cakra having four petals. \extra{The first cakra of support (\textit{ādhāra}) is at the anus, [it] is red-colored, [it] has Gaṇeśa as its deity, [he] is success, intelligence and power, [and has] a rat as [his] mount, the Ṛṣi [of it] is Kūrma, [its seal] is the seal of contraction (\textit{ākuñcanamudrā}), [its] vitalwind is \textit{apāna}, \extra{[its] \textit{kalā} is \textit{umīr}, its \textit{dhāraṇā} is \textit{ojasvinī}} in the four petals [of it resides] \textit{rajas}, \textit{sattva}, \textit{tamas} and the mind-faculties (\textit{manāṃsi}) [symbolized by the syllables] “\textit{vaṃ}”, “\textit{śaṃ}”, “\textit{ṣaṃ}” and “\textit{saṃ}”, in the middle [of it] is a triangle.} In the middle is a trident, and \textit {kāmapīṭha} in the shape of a triangle. In the middle of this seat (\textit{pīṭha}) exists a single form having the shape of a flame. Trough the practice of meditation on this form the whole literature, all \textit{śāstra}s, all poems, dramas etc., everything [related to] elocution, appears in the mind of the person without [prior] learning. \extra{[Assigned to it] is external bliss, yogic bliss, heroic bliss [and] the bliss of coming to rest.}\footnote{It is very strange that only the first \textit{cakra} adds a detailled description of mounts, Ṛṣis, gods, seals and so forth among the current majority of witnesses at hand: \getsiglum{E}, \getsiglum{P}, \getsiglum{L} and \getsiglum{U2}. All other descriptions of the remaining eight \textit{cakra}s leave this out. The only exception is \getsiglum{U2}, a relatively late witness that adds those descriptions for the other \textit{cakra}s as well. Since it is probable that those descriptions are later additions to the text and the witnesses are partially quite conflated, I think this is very interesting for the history of this text, they are added to the edition as well as the translation and are highlighted in color.}\footnote{Find out more about the four blisses.} \end{tlate}
   \end{translation}
   \begin{edition}
     \ekddiv{type=ed}
     \bigskip
    \centerline{\textrm{\small{[Description of the second Cakra]}}}
    \bigskip
    \begin{prose}
%-----------------------
% \om                                       \oxford
%idānīṃ dvitīyaṃ svādhiṣṭānacakraṃ   ṣaḍdalaṃ upāyanapīṭhasaṃjñakaṃ bhavati //  \E
%idānīṃ dvitīyaṃ svādhiṣṭānacakraṃ   ṣaṭdalaṃ uḍḍīyānapīṭhaṃ saṃjñakaṃ bhavati  \P
%idānīṃ dvitīyaṃ svādhiṣṭānacakraṃ   ṣaṭdalaṃ uḍḍīyān pīṭhaṃ saṃjñakaṃ bhavati  \L
%idānīṃ dvitīyaṃ svādhiṣṭānacakraṃ   ṣaṭdalaṃ uḍyānapīṭhasaṃjñakaṃ bhavati /    \N1
%idānīṃ dvitīyaṃ svādhiṣṭānacakraṃ   ṣaṭdalaṃ uḍyāṇāpīṭhasaṃjñikaṃ bhavati //   \D1 
%idānīṃ dvitīyaṃ svādhiṣṭhānacakraṃ  ṣaṭdalaṃ uḍāganapīṭasaṃjñakaṃ bhavati      \U1
%idānīṃ dvitīye svādhiṣṭānacakraṃ // ṣaṭdalaṃ // uḍḍīyāṇapīṭhasaṃjñakaṃ bhavati // liṃgasthānaṃ // pītavarṇaṃ // pītaprabhā // rajoguṇa // brahmādevatā // vaikharīvāca //sāvitrīśaktiḥ //haṃsavāhanaṃ // vahaṇaṛṣiḥ // kāmāgniprabhā //sthūladehā // jāgradavasthā // ṛgveda // ācāryaliṃgaṃ // braṃhmasalokatāmokṣaḥ // śuddhabhumikātatvaṃ // gaṃdho viṣayaḥ // apānavāyuḥ // aṃtarmātṛkā // vaṃ bhaṃ maṃ yaṃ raṃ laṃ // bahirmātrā // kāmā // kāmākhyā // tejasī // ceṣṭṛikā // alasā // mithunā // ajapājapaḥ sahasra // 6000 //gha 0 16 pa 0 40// \U2
%-----------------------
        idānīṃ
        \app{\lem[wit={E,P,L,N1,D1,U1}]{dvitīyaṃ}
        \rdg[wit={U2}]{dvitīye}}
          \app{\lem[wit={U1}]{svādhiṣṭhānacakraṃ}
            \rdg[wit={E,P,L,N1,D1,U2}]{svādhiṣṭānacakraṃ}}
            \app{\lem[wit={P,L,N1,D1,U1,U2}]{ṣaṭdalaṃ}
              \rdg[wit={E}]{ṣaḍdalaṃ}}
       \app{\lem[wit={U2},alt={uḍḍīyāṇapīṭha°}]{uḍḍīyāṇapīṭha}
            \rdg[wit={E}]{upāyanapīṭha°}
            \rdg[wit={L}]{uḍḍīyān pīṭhaṃ}
            \rdg[wit={N1}]{uḍyānapīṭha°}
            \rdg[wit={D1}]{uḍyāṇāpīṭha°}
            \rdg[wit={U1}]{uḍāganapīṭa°}}saṃjñakaṃ bhavati/
       %         \rdg[wit={U2}]{bhavati // liṅgasthānaṃ // pītavarṇaṃ // pītaprabhā // rajoguṇa // brahmādevatā // vaikharīvāca //sāvitrīśaktiḥ // haṃsavāhanaṃ // vahaṇaṛṣiḥ // kāmāgniprabhā //sthūladehā // jāgradavasthā // ṛgveda // ācāryaliṃgaṃ // braṃhmasalokatāmokṣaḥ // śuddhabhumikātatvaṃ // gaṃdho viṣayaḥ // apānavāyuḥ // aṃtarmātṛkā // vaṃ bhaṃ maṃ yaṃ raṃ laṃ // bahirmātrā // kāmā // kāmākhyā // tejasī // ceṣṭṛikā // alasā // mithunā // ajapājapaḥ sahasra // 6000 //gha 0 16 pa 0 40//}} /
\extra{liṅgasthānaṃ/ pītavarṇaṃ/ pītaprabhā/ rajoguṇa/ brahmādevatā/ vaikharīvāca/ sāvitrīśaktiḥ/ haṃsavāhanaṃ/ vahaṇaṛṣiḥ/ kāmāgniprabhā/ sthūladehā/ jāgradavasthā/ ṛgveda/ ācāryaliṅgaṃ/ braṃhmasalokatāmokṣaḥ/ śuddhabhumikātatvaṃ/ gaṃdho viṣayaḥ/ apānavāyuḥ/ aṃtarmātṛkā/ vaṃ bhaṃ maṃ yaṃ raṃ laṃ/ bahir-mātrā/ kāmā/ kāmākhyā/ tejasī/ ceṣṭṛikā/ alasā/ mithunā/}    %-----------------------
%
% \om                                        \B
%tanmadhye atiraktavarṇaṃ tejo varttate /    \E
%tanmadhye 'tiraktavarṇaṃ tejo varttate      \P
%tanmadhye  tiraktavarṇaṃ tejo varttate //   \L
%tanmadhye  atiraktavarṇaṃ tejo varttate     \N1
%tanmadhye  atiraktavarṇaṃ tejo varttate     \D1 
%tanmadhye  atiraktavarṇatejo varttate       \U1
%tanmadhye 'tiraktavarṇaṃ tejo vartate //    \U2
%-----------------------%
       tanmadhye \app{\lem[wit={E,P,L,N1,D1,U2}]{'tiraktavarṇaṃ}
         \rdg[wit={U1}]{atiraktavarṇa°}}
       tejo vartate/
%-----------------------
% \om                                          \B
%tasya dhyānāt sādhako 'tisundaro bhavati /    \E
%tasya dhyānāt sādhako  tisuṃdaro bhavati      \P
%tasya dhyānāt sādhako  tisuṃdaro bhavati //   \L
%tasya dhyānāt sādhakaḥ  atisuṃdaro bhavati // \N1
%tasya dhyānāt sādhakaḥ  atisuṃdaro bhavati // \D1 
%tasya dhyānāt sādhakaḥ  atisuṃdaro bhavati    \U1
%tasya dhyānāt sādhako 'tisundaro bhavati //   \U2
%-----------------------%
tasya dhyānāt sādhako 'tisundaro bhavati/
%-----------------------
% \om                                  \B
%pratidinam-āyur vardhate /           \E
%pratidinam-āyur vardhate             \P
%pratidinam-āyur vardhate //2//        \L
%dinaṃ dinaṃ prati āyurvarddhate // //  \N1
%dinaṃ prati āyurvarddhate //2//        \D1 
%dinaṃ dinaṃ prati āyurvarddhate       \U1
%pratidinaṃ āyur varddhate //          \U2
%-----------------------
\app{\lem[wit={E,P,L,U2}, alt={pratidinam}]{pratidinam\skm{-ā}}
  \rdg[wit={N1,U1}]{dinaṃ dinaṃ prati}
  \rdg[wit={D1}]{dinaṃ prati}}\skp{-ā}yur-vardhate/
    \end{prose}
    \end{edition}
    \begin{translation}
    \ekddiv{type=trans}
    \bigskip
    \centerline{\textrm{\small{[Description of the second Cakra]}}}
    \bigskip
    \begin{tlate}
      Now the second [will be described]. The \textit{svādhiṣṭānacakra} having six petals is known as the seat of \textit{uḍḍīyāṇa}. \extra{[It is] located at the gender, [its] yellow in color, [its] shine is yellow, [it is assigned to the] \textit{rajas}-quality, [its] god is Brahmā, the divinity of speech (\textit{vaikharīvāca}) [is presiding over it], [its] power is Sāvitrī, [its] mount is the goose, [its] \textit{Rṣi} is Vahaṇa, [it has] the shine of desire, [it belongs to] the gross body, [it is assigned to] the waking state, the Ṛgveda, the \textit{guruliṅga}, the liberation of the world of Brahma, the pure land?, [it is] in the range of smell, [its] vitalwind is \textit{apāna}. [Its] inner measure: [endowed with the syllables] vaṃ bhaṃ maṃ yaṃ raṃ laṃ. [Its] outer measure: desire, \textit{kāmākhyā}, the twofold glow?, ceṣṭṛikā?, laziness [and] copulation.} In its middle exists extremely red glow. The adept becomes very handsome through meditation on it. The vital force increases from day to day. \end{tlate}
    \end{translation}
\end{alignment}
\begin{alignment}[
    texts=edition[class="edition"];
    translation[class="translation"],
  ]
\begin{edition}
 \ekddiv{type=ed}
  \bigskip
    \centerline{\textrm{\small{[Description of the third Cakra]}}}
    \bigskip
 \begin{prose}
%-----------------------
% \om                                                 \B
%tṛtīye nābhisthāne     daśadalaṃ padmaṃ vartate      \E
%tṛtīyaṃ nābhisthāne    daśadalaṃ padmaṃ vartate     \P
%tṛtīyaṃ nābhisthāne // daśadalapadme vartate        \L
%tṛtīyaṃ nābhisthāne    daśadalaṃ padma varttate //  \N1
%tṛtīyaṃ nābhisthāne    daśadalaṃ padma varttate //  \D1 
%tṛtīyaṃ nābhisthāne    daśadalakaṃ padmaṃ varttate   \U1
%atha tṛtīyaṃ maṇipūracakraṃ nābhisthāne // kapilavarṇaṃ // viṣṇudevatā // lakṣmīśaktiḥ // vāyuṛṣiḥ // samānavāyuḥ // garuḍavāhanaṃ // sūkṣmaliṃgadevatāha // svapnāvasthā // madhyamāvāk // yajurvedaḥ // dakṣināgniḥ // samipatāmokṣaḥ // guruliṃgaviṣṇuḥ // āpastatvaṃ // rajoviṣayaḥ daśadalāni // daśamātrāḥ // aṃtarmātrā // ḍaṃ ṭaṃ ṇaṃ taṃ thaṃ daṃ dhaṃ naṃ paṃ phaṃ // bahirmātrāḥ // śāṃtiḥ // kṣamā // medhā // tanyā // medhāvinī // puṣkarā // ahaṃsagamanā // lakṣyā //tanmayā // amṛtā // ajapājapa // 6000 gha 016 pa 040 //    \U2
%
%-----------------------
    \app{\lem[wit={P,L,N1,D1,U1}]{tṛtīyaṃ}
      \rdg[wit={E}]{tṛtīye}
      \rdg[wit={U2}]{atha tṛtīyaṃ maṇipūracakraṃ}}
    nābhisthāne
    \app{\lem[wit={E,P,N1,D1}]{daśadalaṃ}
      \rdg[wit={L}]{daśadala°}
      \rdg[wit={U1}]{daśadalakaṃ}
      \rdg[wit={U2}]{\om}}
    \app{\lem[wit={E,P,U1}]{padmaṃ}
      \rdg[wit={L}]{padme}
      \rdg[wit={N1,D1}]{padma}
      \rdg[wit={U2}]{\om}}
    \app{\lem[wit={E,P,L,N1,D1,U1}]{vartate}
      \rdg[wit={U2}]{\om}}/
     % \rdg[wit={U2}]{kapilavarṇaṃ // viṣṇudevatā // lakṣmīśaktiḥ // vāyuṛṣiḥ // samānavāyuḥ // garuḍavāhanaṃ // sūkṣmaliṃgadevatāha // svapnāvasthā // madhyamāvāk // yajurvedaḥ // dakṣināgniḥ // samipatāmokṣaḥ // guruliṃgaviṣṇuḥ // āpastatvaṃ // rajoviṣayaḥ daśadalāni // daśamātrāḥ // aṃtarmātrā // ḍaṃ ṭaṃ ṇaṃ taṃ thaṃ daṃ dhaṃ naṃ paṃ phaṃ // bahirmātrāḥ // śāṃtiḥ // kṣamā // medhā // tanyā // medhāvinī // puṣkarā // ahaṃsagamanā // lakṣyā //tanmayā // amṛtā // ajapājapa // 6000 gha 016 pa 040 //}}
    \extra{kapilavarṇaṃ/ viṣṇudevatā/ lakṣmīśaktiḥ/ vāyuṛṣiḥ/ samānavāyuḥ/ garuḍavāhanaṃ/
      \app{\lem[type=emendation, resp=egoscr]{sūkṣmaliṅgadevatā}
   \rdg[wit={U2}]{\korr sūkṣmaliṅgadevatāha}}/ svapnāvasthā/ madhyamāvāk/ yajurvedaḥ/ dakṣināgniḥ/ samipatāmokṣaḥ/ guruliṅgaviṣṇuḥ/ āpastatvaṃ/ rajo viṣayaḥ daśadalāni/ daśamātrāḥ/ antarmātrā/ ḍaṃ ṭaṃ ṇaṃ taṃ thaṃ daṃ dhaṃ naṃ paṃ phaṃ/ bahir-mātrāḥ/ śāṃtiḥ/ kṣamā/ medhā/ tanyā/ medhāvinī/ puṣkarā/ ahaṃsagamanā/ lakṣyā/ tanmayā/ amṛtā/}
%-----------------------
% \om                                       \B
%tanmadhye paṃcakoṇaṃ cakraṃ varttate //    \E
%tanmadhye paṃcakoṇaṃ cakraṃ varttate       \P
% \om  \L
%tanmadhye paṃcakoṇaṃ cakraṃ varttate //    \N1
%tanmadhye paṃcakoṇaṃ cakraṃ varttate //    \D1 
%tanmadhye paṃcakoṇaṃ cakraṃ varttate       \U1
%tanmadhye paṃcakoṇaṃ cakraṃ vartate //     \U2
%-----------------------
tanmadhye pancakoṇaṃ cakraṃ vartate/ \note[type=philcomm, labelb=s14.z5, lem={tanmadhye ... cakraṃ vartate}]{This sentence is \om \getsiglum{L}.}
%-----------------------
% \om                                  \B
%tanmadhye ekā mūrtir vartate /        \E
%tanmadhye ekā mūrtir vartate          \P
%\om                                   \L
%tanmadhye ekā mūrttir varttate //     \N1
%tanmadhye ekā mūrttir varttate //     \D1 
%tanmadhye ekā mūrtir vartate          \U1
%tanmadhye ekā mūrtir asmi //          \U2
%-----------------------
tanmadhye ekā mūrti\skp{r-}\app{\lem[wit={E,P,N1,D1,U1}, alt={vartate}]{\skm{r}vartate}
  \rdg[wit={U2}]{asmi}}/ \note[type=philcomm, labelb=s14.z6, lem={tanmadhye ... mūrtir vartate}]{This sentence \om in \getsiglum{L}.}
%-----------------------
% \om                                          \B
%tasyās tejo jihvayā kathayituṃ na śakyate /   \E
%tasyās tejo jihvayā kathayituṃ na śakyate     \P
%tasyās tejo jihvayā kathyituṃ na śakyate      \L
%tasyā tejo jihvayā kathayituṃ na śakyate //   \N1
%tasyā tejo jihvayā kathayituṃ na śakyate //   \D1 
%tasyāstejo jihvayā kathatuṃ na śakyate        \U1
%tasyāstejo jihvayā vaktuṃ na śakyate //       \U2
%-----------------------
 \app{\lem[wit={E,P,L,U1,U2}, alt={tasyās}]{tasyā\skp{s-}}
  \rdg[wit={N1,D1}]{tasyā}}\skm{s}tejo jihvayā
  \app{\lem[wit={E,P,N1,D1}]{kathayituṃ}
    \rdg[wit={L}]{kathyituṃ}
    \rdg[wit={U1}]{kathatuṃ}
    \rdg[wit={U2}]{vaktuṃ}}
  na śakyate/
%-----------------------
% \om                                                                   \B
%tasyāḥ mūrter dhyānakāraṇāt    puruṣasya śarīraṃ sthiraṃ bhavati //     \E
%tasyā  mūrter dhyānakaraṇāt    -------------------------------------    \P
%tasyā  mūrtir dhyānakaraṇāt // puruṣasya śarīraṃ sthiram bhavati //     \L
%tasyāḥ mūrter dhyānakaraṇāt    puruṣasya śarīraṃ sthiraṃ bhavati /      \N1
%tasyāḥ mūrter dhyānakaraṇāt    puruṣasya śarīraṃ sthiraṃ bhavati /      \D1 
%tasyāḥ mūrter dhyānakaraṇāt    puruṣasya śarīraṃ sthiraṃ bhavati vā     \U1
%tasyāḥ dhyānakaraṇāt           puruṣasya śarīraṃ sthiraṃ bhavati //     \U2
%-----------------------
 tasyāḥ
  \app{\lem[wit={E,P,N1,D1,U1}, alt={mūrter}]{mūrte\skp{r-}}
      \rdg[wit={L}]{mūrtir}
      \rdg[wit={U2}]{\om}}\skm{r-}dhyāna\app{\lem[wit={P,L,N1,D1,U1,U2}, alt={°karaṇāt}]{karaṇāt}
      \rdg[wit={E}]{°kāraṇāt}}
    \app{\lem[wit={E,L,N1,D1,U1,U2}]{puruṣasya śarīraṃ sthiraṃ}
    \rdg[wit={P}]{\om}}
  \app{\lem[wit={E,L,N1,D1,U2}]{bhavati}
  \rdg[wit={U1}]{bhavati vā}
  \rdg[wit={P}]{\om}}/
 \end{prose}
\end{edition}
\begin{translation}
  \ekddiv{type=trans}
     \bigskip
    \centerline{\textrm{\small{[Description of the third Cakra]}}}
    \bigskip
 \begin{tlate}
The third, a lotus with ten petals exists at the location of the navel.\extra{[It is] monkey-colored, [has] Viṣṇu as its god, Lakṣmi [as its] power, Vāyu [as its] Rṣi, Samāna [as its] vitalwind, [its] mount is Garuḍa, [it belogns to] the suble body, [it is assigned] to the sleeping-state, the inaudible speech (\textit{madhyamāvāg}), the Yajurveda,  the fire of Dakṣina, the liberation of Samipatā\footnote{The second type of liberation. Additional information will be added in the near future}, Viṣṇu's Guruliṅga, the Tattva [of it is] water, [being in] the range of Rajas. It has ten parts [and] ten measures\footnote{What kind of measures?}. [The] inner measure: \textit{ḍaṃ ṭaṃ ṇaṃ taṃ thaṃ daṃ dhaṃ naṃ paṃ phaṃ}. External measure: peace, patience, insight, \textit{tanyā}?, a leared teacher, the lotus, ahaṃsagamanā?, an object aimed at, absorbed in and immortality.} In its middle exists a \textit{cakra} with five angles. In its middle is a single (divine) form. It is not possible to describe her shine with speech (lit. with the tongue). Through the execution of meditation on this (divine) form the body of the person is going to be strong. 
 \end{tlate}
\end{translation}
\begin{edition}
  \ekddiv{type=ed}
   \bigskip
    \centerline{\textrm{\small{[Description of the fourth Cakra]}}}
    \bigskip
  \begin{prose}
%-----------------------
% \om                                                     \B
%caturthaṃ hṛdayamadhye dvādaśadalaṃ kamalaṃ vartate /   \E
%caturthaṃ hṛdayamadhye dvadaśadalaṃ kamalaṃ varttate /  \P
%caturthaṃ hṛdayamadhye dvadaśadalaṃ kamalaṃ varttate /  \L
%caturthaṃ hṛdayamadhye dvadaśadalaṃ kamalaṃ varttate / \N1 
%caturthaṃ hṛdayamadhye dvadaśadalaṃ kamalaṃ varttate   \D1 
%caturthaṃ hṛdayamadhye dvadaśadalaṃ kamalaṃ varttate / \U1   
%caturthaṃ hṛdayamadhye dvadaśadalaṃ kamalamasti      / \U2
%
% anāhatacakraṃ hṛdayasthānaṃ // śvetavarṇaṃ tamoguṇaḥ // rudrodevatā // umāśaktiḥ // hiraṇyagarbhaṛṣiḥ // naṃdivāhanaṃ // prāṇavāyuḥ // jyotiḥ kalākāraṇaṃ dehe // suṣuptir avasthā // paśyaṃtivācā // sāmavedaḥ // gārhasyatyogniḥ? // śivaliṇgaṃ // prāptibhūmikā // sarū?patāmuktiḥ // dvādaśādalāni //dvādaśamātrā // kaṃ khaṃ gaṃ ghaṃ ṇaṃ caṃ chaṃ jaṃ jhaṃ yaṃ taṃ thaṃ // bahirmātrā // rudrāṇī // tejasā // tāpinī // spha?kadā // caitanyā // śivadā // Śānti // umā // gaurī // mātara // jvālā // prajvālinī // ajapājapasahasra // cha 000 gha 0 1 6? pa 040 // U2
caturthaṃ hṛdayamadhye dvādaśadalaṃ
    \app{\lem[wit={E,P,L,N1,D1,U1}]{vartate}
      \rdg[wit={U2}]{asti}}/
    \extra{anāhatacakraṃ hṛdayasthānaṃ/ śvetavarṇaṃ tamoguṇaḥ/ rudrodevatā /umāśaktiḥ/ hiraṇyagarbhaṛṣiḥ/ nandivāhanaṃ/ prāṇavāyuḥ/ jyotiḥ kalākāraṇaṃ dehe/ suṣuptir-avasthā/ \app{\lem[type=emendation, resp=egoscr]{paśyantīvācā}\rdg[wit={U2}]{\korr paśyaṃtivācā}}/ sāmadedaḥ/ \app{\lem[type=emendation, resp=egoscr]{gārhapatyāgniḥ}\rdg[wit={U2}]{\korr gārhasyatyogniḥ}}/ śivaliṇgaṃ/ prāptibhūmikā/ sarū?patāmuktiḥ/ dvādaśādalāni/ dvādaśamātrā/ kaṃ khaṃ gaṃ ghaṃ ṇaṃ caṃ chaṃ jaṃ jhaṃ yaṃ taṃ thaṃ/ bahir-mātrā/ rudrāṇī/ tejasā/ tāpinī/ sphakadā/ caitanyā/ śivadā/ śānti/ umā/ gaurī/ mātara/ jvālā/ prajvālinī/} 
%-----------------------
% \om                                          \B
%atitejomayatvād   dṛṣṭigocaraṃ na bhavati \E  
%atitejomayatvāt   dṛṣṭigocaraṃ na bhavati    \P
%atitejomayatvād   dṛṣṭigocaraṃ na bhavati // \L
%atitejomayatvāt / dṛṣṭigocaraṃ na bhavati / \N1
%atitejomayatvāt / dṛṣṭigocaraṃ na bhavati / \D1
%atitejomayatvāt / dṛṣṭigocaraṃ na bhavati / \U1
%atitejomayatvād   dṛṣṭigocaratāṃ na yāti // \U2 
%-----------------------
atitejomayatvād-dṛṣṭi\app{\lem[wit={E,P,L,N1,D1,U1}, alt={°gocaraṃ}]{gocaraṃ}
                         \rdg[wit={U2}]{gocaratāṃ}}
na
    \app{\lem[wit={E,P,L,N1,D1,U1}]{bhavati}
      \rdg[wit={U2}]{yāti}}/   
%-----------------------
% \om                                               \B
%tanmadhye 'ṣṭadalam adhomukhaṃ kamalaṃ varttate // \E  
%tanmadhye 'ṣṭadale  mukhaṃ kamalaṃ varttate //     \P
%tanmadhye ṣṭadalaṃ    adhomukhakamalaṃ vartate //  \L
%tanmadhye aṣṭadalaṃ adhomukhaṃ kamalaṃ vartate //  \N1
%tanmadhye aṣṭadalaṃ adhomukhaṃ kamalaṃ vartate //  \D1
%tanmadhye aṣṭadalaṃ adhomukhaṃ kamalaṃ vartate /   \U1
%tanmadhye 'ṣṭadalaṃ adhomukhaṃ kamalaṃ asti /      \U2
%-----------------------
    tanmadhye \app{\lem[wit={E,L,N1,D1,U1,U2},alt={'ṣṭadalam}]{'ṣṭadalam\skm{a}}
      \rdg[wit={P}]{'ṣṭadale}}\app{\lem[wit={E,N1,D1,U1,U2},alt={adhomukhaṃ kamalaṃ}]{\skp{-a}dhomukhaṃ kamalaṃ}
        \rdg[wit={L}]{adhomukhakamalaṃ}
        \rdg[wit={P}]{mukhaṃ kamalaṃ}}
      \app{\lem[wit={E,P,L,N1,D1,U1}]{vartate}
        \rdg[wit={U2}]{asti}}/    
\end{prose}
\end{edition}
\begin{translation}
  \ekddiv{type=trans}
       \bigskip
    \centerline{\textrm{\small{[Description of the fourth Cakra]}}}
    \bigskip
  \begin{tlate}
The fourth lotus having twelve-petals exists in the middle at the heart. \extra{[The] Anāhatacakra is placed in the heart. [It is] white in color, has the quality of \textit{tamas}, [its] deity is Rudra, [its] power is Umā, [its] Ṛṣi is Hiraṇyagarbha, [its] mount is Nandi, [its] vitalwind is Prāṇa, in the body it is the light that causes fragmentation? (\textit{kalākaraṇa}), [its] state is deep sleep, [its] speech is \textit{paśyantī}\footnote{Add footnote of entry in \textit{Tāntrikābhidhānakośa}.}, [it is attributed to the] Sāmaveda, the fire of the house, Śivaliṅgam, the ability to attach everything on the earth [and] the uniform liberation. [It has] twelve petals, [associated with] twelve measures, [having the syllables] kaṃ khaṃ gaṃ ghaṃ ṇaṃ caṃ chaṃ jaṃ jhaṃ yaṃ taṃ [and] thaṃ. [Its] external measure [is]: Rudras wife, light (\textit{tejasā?}), glow, sphakadā?, consciousness (\textit{caitanyā}), bestower of Śiva, peace, Umā, Gaurī, Mātara, the flame [and] Prajvālinī.} Due to being made of [such an] intense light [the fourth lotus] is not in the range of sight. In its middle exists a lotus facing downward having eight petals.
  \end{tlate}
   \end{translation}
\clearpage
  \begin{edition}
     \ekddiv{type=ed}
\begin{prose}
      \extra{manaś-cakre/ manodevatā/
        \app{\lem[type=emendation, resp=egoscr]{bhaiśaktiḥ}
          \rdg[wit={U2}]{\korr bahiśaktiḥ}}
        / ātmaṛṣih/ nābhimadhye sthitaṃ padmaṃ nālaṃ tasya
        \app{\lem[type=emendation, resp=egoscr]{daśāṅgulaṃ}
          \rdg[wit={U2}]{\korr daśāgulaṃ}}/
        komalaṃ tasya tan-nālaṃ nirmalaṃ cāpy-adhomukhaṃ/ kadalīpuṣpasaṃkāśaṃ tanmadhye ca pratiṣṭhitaṃ/ mana unnatyasaṃkalpa/ vikalpātmakameva ca/ pūrvadale svetavarṇe yadā viśrāmate manaḥ/ dharmakīrtividyādi sadbuddhir-bhavati/ agnikoṇe āraktavarṇe nidrā ālasyamāyāmandamatir-bhavati/ dakṣiṇe kṛṣṇavarṇeti tadā krodhotpattir-bhavati/ naiṛtye nīlavarṇe mamatāmatir-bhavati/ paścime kapilavarṇe/ krīḍāhāsotsavotsāhamatir-bhavati/ vāyavye śāmavarṇe cintodvegamatir-bhavati/ uttare pītavarṇe bhogaśṛṇgāramahodayamatir-bhavati/ īśāne gauravarṇe
        \app{\lem[type=emendation, resp=egoscr, alt={jñānasaṃdhāna°}]{jñānasaṃdhāna}
          \rdg[wit={U2}]{\korr jñānasaṃdhāne}}
        matir-bhavati/}
 %The mind resides in this \textit{cakra}, [the] god [presiding over it] is the mind [itself], [its] power is Bhai, [its] Ṛṣi is the self. In the middle of the navel [exists] a place, being a lotus, its tube measures ten \textit{aṅgula}s, the water [being in] the tube is pure and facing upwards. In its middle is the location of a shining banana-flower. The mind is intended to rise upwards?. [There are] several options to arise in oneself. If the mind takes rest in the eastern petal [which is] while in color the natural law, fame, knowledge etc. [and] a clear intellect arises. [If the mind rests] in south-east, [which is] reddish in color, sleep, laziness, illusion and a weak mind arises. [If it rests] on the right south, [which is] black in color then anger is generated. [If it rests] in the southwest, [which is] blue in color a mind that is selfish arises. [If it rests] in the west, [which is] brown in color a mind of payfulness, laughing, and party-mood arises. [If it rests] in the northwest, [which is] dark in color a mind of restless thought arises. [If it rests] in the north, [which is] yellow in color a mind of great happiness, erotic and enjoyment arises. [If it rests] in north-east [which is] whitish in color a mind endowed with unified knowledge arises.      
%-----------------------
% \om                                                     \B      
%tanmadhye prāṇavāyoḥ sthānam    aṣṭadalakamalamadhye liṃgākārā karṇikā  kathyate /  \E 
%tanmadhye prāṇavāyoḥ sthānam    aṣṭadalakamalamadhye liṃgākārā karṇikā  kathyate /  \P
%tanmadhye prāṇavāyoḥ sthānam    aṣṭadalakamalamadhye liṃgākārā karṇikā  kathyate // \L
%tanmadhye prāṇavāyoḥ sthānam    aṣṭadalakamalamadhye liṃgākārā karṇikā  kathyate // \N1
%tanmadhye prāṇavāyoḥ sthānam // aṣṭadalakamalamadhye liṃgākārā karṇi    kathyate // \D1
%tanmadhye prāṇavāyo  sthānam    aṣṭadalakamalamadhye liṃgākārā karṇikā  kathyate    \U1
%tanmadhye prāṇavāyo  sthānam // aṣṭadalakamalamadhye liṃgākārā karṇikā  kathyate    \U2
%-----------------------        
tanmadhye prāṇavāyoḥ sthānam-aṣṭadalakamalamadhye liṃgākārā \app{\lem[wit={E,P,L,N1,U1,U2}]{karṇikā}\rdg[wit={U2}]{karṇi}} kathyate/   
%-----------------------
% \om                                                     \B
%tasyāḥ karṇiketi saṃjñā tatkarṇikāmadhye padmarāgasamānavarṇāṃ guṣṭhapramāṇaikā puttalikā varttate //          \E  
%tasyāḥ kaliketi saṃjñā tatkalikāmadhye   padmarāgaratnasamānavarṇāṃ aṃguṣṭhapramāṇā ekā puttalikā varttate     \P
%tasyāḥ kalikeli                 madhye   padmaratnasamānavarṇā // aṃguṣṭhapramāṇā // ekā puttalikā varttate // \L
%tasyāḥ kaliketi saṃjñā tatkalikāmadhye   padmarāgaratnasamānavarṇāṃ aṃguṣṭhapramāṇā ekā puttalikā varttate     \N1
%tasyāḥ kaliketi saṃjñā tatkalikāmadhye   padmarāgaratnasamānavarṇā aṃguṣṭhapramāṇāt ekā puttalikā varttate /   \D1
%tasyāḥ kaliketi saṃjñā tatkalikāmadhye   padmarāgaratnasamānavarṇā aṃguṣṭhapramāṇāt ekā puttalikā varttate /   \U1
%tasyāḥ kaliketi saṃjñā tatkalikāmadhye   padmarāgaratnasamānavarṇā  // aṃguṣṭhapramāṇā ekā puttalikā varttate / \U2
%-----------------------
tasyāḥ \app{\lem[wit={P,N1,D1,U1,U2}]{kaliketi}
  \rdg[wit={L}]{kalikeli}
  \rdg[wit={E}]{karṇiketi}}
\app{\lem[wit={E,P,N1,D1,U1,U2}]{saṃjñā}
  \rdg[wit={L}]{\om}}
\app{\lem[wit={E,P,N1,D1,U1,U2}]{tatkalikāmadhye}
  \rdg[wit={L}]{\om}}
\app{\lem[type=emendation, resp=egoscr]{padmarāgaratnasamānavarṇāṅguṣṭhapramāṇaikā}
  \rdg[wit={E}]{\korr padmarāgasamānavarṇāṃguṣṭhapramāṇaikā}
  \rdg[wit={P,N1}]{padmarāgaratnasamānavarṇāṃ// aṃguṣṭhapramāṇā// ekā}
  \rdg[wit={L}]{padmaratnasamānavarṇā aṃguṣṭhapramāṇā ekā}
  \rdg[wit={D1,U1}]{padmarāgaratnasamānavarṇā aṃguṣṭhapramāṇāt ekā}
  \rdg[wit={U2}]{padmarāgaratnasamānavarṇā// aṃguṣṭhapramāṇā ekā}} puttalikā vartate/   
%The technical designation of her is kalikā. In the middle of this kalikā exists a single thumbsized (divine) figurine (puttalikā) being similiar to a ruby-gem in color. Her technical designation is embodied soul (jīva).
%-----------------------
%
%tasyā  jīvasaṃjñā           tasyā  balamadhyasvarūpaṃ        koṭijihvābhir  vaktuṃ naiva śakyate // \E
%tasyā  jīvasaṃjñā           tasyā  balam atha svarūpaṃ       koṭijihvābhir  vaktuṃ naiva śakyate // \P 
%tasya                              bala sappa svarūpaṃ       koṭijihvāyābhi vaktuṃ na    śakyate // \L 
%tasyāḥ jīveti saṃjñāḥ       tasyāḥ balaṃ atha ca svarūpaṃ    koṭijihvābhir  vaktuṃ na    śakyate // \N1
%tasyāḥ jīveti saṃjña /      tasyāḥ balaṃ atha ca svarūpaṃ    koṭijihvābhir  vaktuṃ na    śakyate // \D1
%tasyāḥ jīveti saṃjñā        tasyāḥ balaṃ atha ca svarūpaṃ    koṭijihvābhir  vaktuṃ na    śakyate // \U1
%tasyā  jīvasaṃjñā //        tasya  balaṃ tasya atha svarūpaṃ koṭijihvābhir  vaktuṃ na    śakyate // \U2
%-----------------------
\app{\lem[wit={E,P}]{tasyā}
     \rdg[wit={N1,D1,U1}]{tasyāḥ}
     \rdg[wit={L}]{tasya}}
\app{\lem[wit={U2}]{jīveti saṃjñā}
     \rdg[wit={N1}]{jīveti saṃjñāḥ}
     \rdg[wit={D1}]{jīveti saṃjña}
     \rdg[wit={E,P,U2}]{jīvasaṃjñā}
     \rdg[wit={L}]{\om}}
\app{\lem[wit={E,P}]{tasyā}
     \rdg[wit={N1,D1,U1}]{tasyāḥ}
     \rdg[wit={U2}]{tasya}}
\app{\lem[wit={N1,D1,U1,U2}]{balaṃ atha ca svarūpaṃ}
     \rdg[wit={P}]{balam atha svarūpaṃ}
     \rdg[wit={U2}]{balaṃ tasya atha svarūpaṃ}
     \rdg[wit={L}]{bala sappa svarūpaṃ}
     \rdg[wit={E}]{balamadhyasvarūpaṃ}}
\app{\lem[wit={E,P,N1,D1,U1,U2}, alt={koṭijihvābhir}]{koṭijihvābhi\skp{r-}\skm{r-va}}
    \rdg[wit={L}]{koṭijihvāyābhi}}\skp{-va}ktuṃ
\app{\lem[wit={L,N1,D1,U1,U2}]{na}
    \rdg[wit={E,P}]{naiva}}
  śakyate/
%-----------------------  
%Her technical designation is embodied soul. Not even with a thousand tongues it is possible to talk about her nature and her power.
%-----------------------
%asyā  mūrter   dhyānakāraṇāt      svarga-pātāl--ākaśamanuṣyagandharvakinnaraguhyakavidyādharalokasambandhinyaḥ strīyo 'pi-------------------- vaśyā bhavanti / \E
%asyā  mūrter   dhyānakaraṇāt      svarga-pātāl--ākāśamanuṣyagandharvakiṃnaraguhyakavidyādharalokasaṃbaṃdhinyaḥ strīyo 'pi-------------------- vaśyā bhavanti / \P
%asyā  mūrtir   dhyānāt            svarga-pātāl--ākāśamanuṣyagaṃdharvakinnaraguhyakavidyādharalokasambandhinyaḥ strīyo 'pi-------------------- vaśyā bhavanti /L
%asyāḥ mūrter  dhyānakaraṇāt      svarga-pātāla ākāśamanuṣyagaṃdharvakinnaraguhyakavidyādharalokasaṃbaṃdhinyaḥ strīyaḥ sādhakasya puruṣasya   vaśyā bhavanti // \N1
%asyāḥ mūrter  dhyānakaraṇāt      svarga-pātāla ākāśamanuṣyagaṃdharvakiṃnaraguhyakavidyādharalokasaṃbaṃdhinyaḥ strīyaḥ sādhakasya puruṣasya   vaśyā bhavanti // \D1
%asyāḥ mūrter  dhyānakaraṇāt      svarga-pātāla ākāśamanuṣyagaṃdharvakiṃnaraguhyakavidyādharalokasaṃbaṃdhinyaḥ strīyaḥ sādhakasya puruṣasya   vaśyā bhavanti // \U1
%pṛthvī lokasaṃbaṃdhanyo pi striyaḥ vaśyā bhavaṃti/  
%tasyāḥ mūrter dhyānaṃ karaṇāt // svarga-pātāl--ākāśamanuṣyagandharvakinnaraguhyakavidyādharalokasaṃbadhinya---striyo  pi---------------------vaśyā bhavaṃti // \U2
%-----------------------
%“Because of the exercise of meditation on this form the inhabitants of the universe (which are) Humans, Gandharvas, Kinnaras, Guhyakas, Vidyādharas and (their) females, in the heavenly world, underworld and open space are obedient to the will of the practicing person.”, is what said here.  
%-----------------------
   \app{\lem[wit={E,P,L}]{asyā}
    \rdg[wit={N1,D1,U1}]{asyāḥ}
    \rdg[wit={U2}]{tasyāḥ}}
 \app{\lem[wit={E,P,N1,D1,U1,U2}, alt={mūrter}]{mūrte\skp{r-}}
    \rdg[wit={L}]{mūrtir}}\app{\lem[wit={E,P,N1,D1,U1}, alt={dhyānakāraṇāt}]{\skm{r-}dhyānakāraṇāt}
    \rdg[wit={U2}]{dhyānaṃ karaṇāt}
    \rdg[wit={L}]{dhyānāt}}
  svargapātālākaśamanuṣyagandharvakinnaraguhyakavidyādharaloka\app{\lem[wit={E,P,L,N1,D1,U1}]{saṃbandhinyaḥ}\rdg[wit={U2}]{saṃdadhinya}}
  \app{\lem[wit={N1,D1,U1}]{strīyaḥ sādhakasya puruṣasya}
    \rdg[wit={E,P,L}]{strīyo 'pi}
      \rdg[wit={U2}]{striyo pi}}
vaśyā bhavanti/\note[type=philcomm, labelb=s16, lem={bhavanti}]{\getsiglum{U1} adds a flawed phrase hereafter: \textit{pṛtvī lokasaṃbaṃdhanyo pi striyaḥ vaśyā bhavaṃti/}. I refrained to include it in the apparatus due to its redundance.}
%-----------------------
%ityatra kathyate// /E
%ityatra kathyate// \P
%ityatra kathyate// \L
%ityatra kiṃ kathyate // \N1
%ityaṃtra kiṃ kathyate // \D1
%ityatra kiṃ kathyate vā \U1
%ityatra kathyate // \U2
%-----------------------
ityatra \app{\lem[wit={N1,D1,U1}]{kiṃ}
  \rdg[wit={E,P,L,U2}]{\om}}
\app{\lem[wit={E,P,L,N1,D1,U2}]{kathyate}
  \rdg[wit={U1}]{kathyate vā}}//
  \end{prose}
\end{edition}
\begin{translation}
  \ekddiv{type=trans}
  \begin{tlate}
   \extra{The mind resides in this \textit{cakra}, [the] god [presiding over it] is the mind [itself], [its] power is Bhai, [its] Ṛṣi is the self. In the middle of the navel [exists] a place, being a lotus, its tube measures ten \textit{aṅgula}s, the water [being in] the tube is pure and facing upwards. In its middle is the location of a shining banana-flower. The mind is intended to rise upwards?. [There are] several options to arise in oneself. If the mind takes rest in the eastern petal [which is] while in color the natural law, fame, knowledge etc. [and] a clear intellect arises. [If the mind rests] in south-east, [which is] reddish in color, sleep, laziness, illusion and a weak mind arises. [If it rests] on the right south, [which is] black in color then anger is generated. [If it rests] in the southwest, [which is] blue in color a mind that is selfish arises. [If it rests] in the west, [which is] brown in color, a mind of playfulness, laughing, and party-mood arises. [If it rests] in the northwest, [which is] dark in color, a mind of restless thought arises. [If it rests] in the north, [which is] yellow in color, a mind of great happiness, erotic and enjoyment arises. [If it rests] in north-east [which is] whitish in color a mind endowed with unified knowledge arises.} It is said that in its middle is the place of the \textit{prāṇa}-vitalwind [and] in the middle [of] the eight-petalled lotus is a pericarp (\textit{karṇikā}) in the form of a \textit{liṅga}. The technical designation of her is kalikā. In the middle of this kalikā exists a single thumbsized [divine] figurine (\textit{puttalikā}) being similiar to a ruby-gem in color. Her technical designation is embodied soul (\textit{jīva}). Not even with a thousand tongues it is possible to talk about her nature and her power. “Because of the exercise of meditation on this form the inhabitants of the universe [which are] Humans, Gandharvas, Kinnaras, Guhyakas, Vidyādharas and [their] females, in the heavenly world, underworld and open space are obedient to the will of the practicing person.”, is said here.
  \end{tlate}
\end{translation}
\end{alignment}
\clearpage
\begin{alignment}[
    texts=edition[class="edition"];
    translation[class="translation"],
  ]
\begin{edition}
  \ekddiv{type=ed}
  \bigskip
    \centerline{\textrm{\small{[Description of the fifth Cakra]}}}
    \bigskip  
    \begin{prose}
%-----------------------      
%-------pañcamaṃ kaṇṭhasthāne ṣoḍaśadalaṃ kamalaṃ      vartate //  \E
%-------paṃcamaṃ kaṃṭhasthāne ṣoḍaśadalaṃ kamalaṃ      vartate     \P
%-------paṃcamaṃ kaṃṭhasthāne ṣoḍaśadalaṃ kamalaṃ      vartate     \L
%idānīṃ paṃcamaṃ kamalaṃ      ṣodaśadalaṃ kaṃṭhasthāne varttate // \N1
%idānīṃ paṃcamaṃ kamalaṃ      ṣodaśadalaṃ kaṃṭhasthāne varttate // \D1 --------> Was in diesem Falle machen?
%idānīṃ paṃcamaṃ kamalaṃ      ṣodaśadalaṃ kaṃṭhasthāne varttate // \U1
%-------paṃcamaṃ viśuddhacakraṃ           kaṃṭhastāne              \U2     
%-----------------------
      %dhūmra?varṇe jīvodevatā// avidyāśaktiḥ// virāṭharṣiḥ// vāyurvāhanaṃ// udānavāyuḥ// jvālākalā jālaṃdharobaṃdhaḥ mahākāraṇadeha// tūryāvasthā// parāvācā// atharvaṇavedaḥ// jaṃgamaliṅgaṃ jīvaprāptābhūmikā// sāyujyatāmokṣaḥ// ṣoḍaśadalāni// ṣoḍaśamātrāḥ// atarmātrār-carāḥ// aṃ āṃ iṃ īṃ u ūṃ ṛṃ ṝṃ ḷṃ ḹṃ eṃ aiṃ oṃ auṃ aṃ aṃḥ// bahirmātrāvidyā// avidyā// ichā// śakti// jñānaśaktiḥ// śatalā// mahāvidyā// mahāmāyā// buddhiḥ// tamasī// maitrā?// kumārī// maitrāyaṇī// rudrā// puṣṭa// siṃhanī// ajapājapasahasra/ 1000 gha 02 pa 046 akṣara 40//
      %
      
%Now (follows the description of) the fifth lotus having sixteen petals (which) exists at the location of the throat.
%-----------------------      
  \app{\lem[wit={N1,D1,U1}]{idānīṃ}
\rdg[wit={E,P,L,U2}]{\om}}
pañcamaṃ
\app{\lem[wit={N1,D1,U1}]{kamalaṃ ṣodaśadalaṃ kaṇṭhasthāne}
  \rdg[wit={E,P,L}]{kaṇṭhasthāne ṣoḍaśadalaṃ kamalaṃ}
  \rdg[wit={U2}]{viśuddhacakraṃ kaṃṭhastāne}}
\app{\lem[wit={E,P,L,N1,D1,U1}]{vartate}
  \rdg[wit={U2}]{\om}}/
\extra{dhūmravarṇe jīvodevatā/ avidyāśaktiḥ/ virāṭharṣiḥ/ vāyurvāhanaṃ/ udānavāyuḥ/ jvālākalā jālaṃdharobandhaḥ mahākāraṇadeha/ tūryāvasthā/ parāvācā/ atharvaṇavedaḥ/ jaṃgamaliṅgaṃ jīvaprāptābhūmikā/ sāyujyatāmokṣaḥ/ ṣoḍaśadalāni/ ṣoḍaśamātrāḥ/ antarmātrār-carāḥ/ aṃ āṃ iṃ īṃ u ūṃ ṛṃ ṝṃ ḷṃ ḹṃ eṃ aiṃ oṃ auṃ aṃ aṃḥ/ bahirmātrāvidyā/ avidyā/ ichā/ śakti/ jñānaśaktiḥ/ śatalā/ mahāvidyā/ mahāmāyā/ buddhiḥ/ tamasī/ maitrā/ kumārī/ maitrāyaṇī/ rudrā/ puṣṭa/ siṃhanī/}
%----------------------- 
%tanmadhye koṭisūryasamāna       ekaḥ puruṣo vartate / \E
%tanmadhye koṭicaṃdrasamaprabhaḥ ekaḥ puruṣo vartate   \P
%tanmadhye koṭicaṃdrasamaprabhā  ekaḥ puruṣo vartate   \L
%tanmadhye koṭicaṃdrasamaprabhaḥ ekaḥ puruṣo varttate  \N1
%tanmadhye koṭicaṃdrasamaprabhā  eka--puruṣo varttate  \D1
%tanmadhye koṭicaṃdrasamaprabhaḥ ekaḥ puruṣo varttate  \U1
%tanmadhye koṭicaṃdrasamaprabhaḥ // eka pumān varttate // \U2
%----------------------- 
%In its  middle exists a single person which shines like a thousand moons.
%----------------------- 
tanmadhye
\app{\lem[wit={P,N1,U1,U2}]{koṭicandrasamaprabhaḥ}
  \rdg[wit={L,D1}]{°prabhā}
  \rdg[wit={E}]{koṭisūryasamāna}}
\app{\lem[wit={E,P,L,N1,U1}]{ekaḥ puruṣo}
  \rdg[wit=D1]{ekapuruṣo}
  \rdg[wit={U2}]{eka pumān}}
vartate/
%----------------------- 
%tasya puruṣasya dhyānakāraṇād--- asādhyarogā naśyanti // \E
%tasya puruṣasya dhyānakāraṇād--- asādhyarogā naśyanti // \L
%tasya puruṣasya dhyānakāraṇād--- asādhyarogā naśyaṃti // \P
%tasya puruṣasya dhyānakaraṇāt--  asādhyarogā naśyaṃti // \N1
%tasya puruṣasya dhyānakaraṇāt    asādhyarogā naśyaṃti    \N2
%tasya puruṣasya dhyānakaraṇāt /  asādhyarogā naśyaṃti // \D1
%tasya puruṣasya dhyānakaraṇāt /  asādhyarogā naśyaṃti    \U1
%tasya puṃsaḥ    dhyānakaraṇāt // asādhyarogā naśyaṃti // \U2
%----------------------- 
%Because of the exercise of meditation on this person all diseases which are (otherwise) not possible to be controlled vanish.
%----------------------- 
tasya
\app{\lem[wit={E,L,P,N1,N2,D1,U1}]{puruṣasya}
  \rdg[wit={U2}]{puṃsaḥ}}
dhyānakaraṇād-asādhyarogā naśyanti/
%----------------------- 
%ekasahasravarṣaparyaṃtaṃ sa puruṣo jīvatīdānīṃ     \E
%ekasahasravarṣaparyaṃtaṃ sa puruṣo jīvati          \P
%ekasahasravarṣa             puruṣo jīvati //       \L
%ekasahasravarṣaparyaṃtaṃ    puruṣo jīvati /        \N1
%ekasahasravarṣaparyaṃta     puruṣo jīvati /        \N2
%ekasahasravarṣaparyaṃtaṃ    puruṣo jīvati /        \D1
%ekasahasravarṣaparyaṃtaṃ    puruṣo jīvati          \U1
%ekasahasravarṣaparyaṃtaṃ    puruṣo jīvati //       \U2
%----------------------- 
%The person lives up to 1001 years.
%----------------------- 
ekasahasravarṣa\app{\lem[wit={E,P,N1,D1,U1,U2},alt={°paryantaṃ}]{paryantaṃ}
  \rdg[wit={N2}]{°paryaṃta}
  \rdg[wit={L}]{\om}}
\app{\lem[wit={L,N1,N2,D1,U1,U2}]{puruṣo}
\rdg[wit={E,P}]{sa puruṣo}}
  jīvati//
    \end{prose}
\end{edition}
\begin{translation}
  \ekddiv{type=trans}
  \bigskip
    \centerline{\textrm{\small{[Description of the fifth Cakra]}}}
    \bigskip
   \begin{tlate}Now the fifth lotus having sixteen petals exists at the location of the throat.\extra{[It is] smoke-colored, [its] god is the embodied soul (\textit{jīva}), [its] power is ignorance (\textit{avidyā}), [its] Ṛṣi is Virāṭha, [its] mount is the vitalwind (\textit{vāyu}), [its] vitalwind is \textit{udāna}, [it belongs to] Jvālākalā (?), [associated with it is] Jālandharabandha, [and the] supra-causel body (\textit{mahākāraṇadeha}), [its] state is the fourth state (\textit{tūrya}), [its] speech is Parā\footnote{Im Kaśm. Śiv. °das ewige Wort, in welchem potentiell alle Begriffe und Worte ruhen; vgl. das śabdabrahma des Vyākaraṇa. [B.]― Schmidt S. 246}, [it is associated with the] Atharvaveda, Jaṅgamaliṅga [and] Jīvaprāptābhūmikā?, [its] liberation is absorption into the divine essence (\textit{sāyujyatāmokṣaḥ}), [it has] sixteen petals [with] sixteen measures. [Its] internal measures sounds are: aṃ āṃ iṃ īṃ u ūṃ ṛṃ ṝṃ ḷṃ ḹṃ eṃ aiṃ oṃ auṃ aṃ aṃḥ. [Its] external measures are: knowledge, ignorance, desire, power, the power of knowledge, \textit{śatala}?, great knowledge, great illusion, intellect, \textit{tamasī}?, love, young girl?, Maitrāyaṇī?, sun-ray?, abundance, lioness?.} In its  middle exists a single person which shines like a thousand moons. Because of the exercise of meditation on this person all diseases which are (otherwise) not possible to be controlled vanish. The person lives up to 1001 years.\end{tlate}
\end{translation}
\begin{edition}
  \ekddiv{type=ed}
   \bigskip
    \centerline{\textrm{\small{[Description of the sixth Cakra]}}}
    \bigskip
 \begin{prose}
%----------------------- 
%īdānīṃ ṣaṣṭhaṃ bhrūmadhye ājñācakraṃ                vartate//   \E
%īdānīṃ ṣaṣṭhaṃ bhrūmadhye ājñācakraṃ                vartate//   \P
%īdānīṃ ṣaṣṭhaḥ bhrūmadhye ājñācakraṃ                vartate//   \L
%idānīṃ ṣaṣṭhacakraṃ       ajñānāmakaṃ               varttate // \N1
%idānīṃ ṣaṣṭhacakraṃ       ajñānāmaka                varttate    \N2
%idānīṃ ṣaṣṭhacakraṃ       ajñānāmakaṃ               varttate // \D1
%idānīṃ ṣaṣṭhacakraṃ        ājñānāmakaṃ               vartate     \U1
%idānīṃ ṣaṣṭa   bhrūmadhye ājñācakraṃ raktavarṇaṃ //             \U2
%-----------------------
   %āgnirdevatā suṣumṇāśaktiḥ// hiṃsaṛṣiḥ// caitanyavāhanaṃ// jñānadehī// vijñānāvathā// anupamavācā// sāmadevaḥ// pramādaliṃgaṃ// ardhamātrā// ākāśātatvaṃ// jīvahiṃsa// caitanyalīlraṃbhaḥ// dvemātrā// hiṃkṣaṃ// aṃtarmātrā// bahirmātrā//sthiti//prabhā?// ajapājapasahasra// 1000 gha 02 pa 046 akṣara 40// \U2
%-----------------------
   idānīṃ
    \app{\lem[wit={N1,N2,D1,U1}]{ṣaṣṭhacakraṃ}
       \rdg[wit={E,P}]{ṣaṣṭhaṃ bhrūmadhye}
       \rdg[wit={L}]{ṣaṣṭhaḥ bhrūmadhye}
       \rdg[wit={U2}]{ṣaṣṭa bhrūmadhye}}
    \app{\lem[wit={U1}]{ājñānāmakaṃ}
       \rdg[wit={N1,D1}]{ajñānāmakaṃ}
       \rdg[wit={N2}]{ajñānāmaka}
       \rdg[wit={E,P,L}]{ājñācakraṃ}
       \rdg[wit={U2}]{ājñācakraṃ raktavarṇaṃ}
       \rdg[wit={N1,D1,U1}]{ajñānāmakaṃ}
       \rdg[wit={N2}]{ajñānāmaka}}
   \app{\lem[wit={E,P,L,N1,N2,D1,U1}]{vartate}
       \rdg[wit={U2}]{\om}}/
       \extra{āgnirdevatā suṣumṇāśaktiḥ/ hiṃsaṛṣiḥ/ caitanyavāhanaṃ/ jñānadehī/ vijñānāvasthā/ anupamavācā/ sāmavedaḥ/ pramādaliṃgaṃ/ ardhamātrā/ ākāśātatvaṃ/ jīvahiṃsa/ caitanyalīlāraṃbhaḥ/ dvemātrā/ haṃ kṣaṃ/ aṃtarmātrā/ bahirmātrā/ sthiti/ prabhā?/}
   %[Its] god is Āgni?, [its] power is the godess of the centre (\textit{suṣumṇā}), [its] Ṛṣi is Hiṃsa, [its] mount is Caitanya, [its] body is Jñāna, [its] state is Vijñāna, [its] speech is incomparable (\textit{anupama}), [its] Veda is Sāma, [its] liṅgaṃ is intoxication (\textit{pramāda}), [its] half-measure? is Jīvahiṃsa [and] the support of play of Caitanya. [It has] two measures haṃ [and] kṣam [as its] inner measure. [Its] external measures [are] contemplation (\textit{sthiti}) [and] splendour (\textit{prabhā}).
%----------------------- 
                                       %dvidalaṃ tanmadhye  'gnijvālākārakamalaṃ     kiṃcid vastu vartate/    \E
                                       %dvidalaṃ tanmadhye  agnijvālākārakamalaṃ     kiṃcid vastu vartate/    \P
                                       %dvidalaṃ tanmadhye  agnijvālākārakamalaṃ     kiṃcid vastu vartate/    \L
%                                                           agnijvālākārakamalaṃ     kiṃcid vastu vartate/    \B
%tac cakraṃ bhruvor madhye dvidalakaṃ sthitaṃ // tanmadhye  agnijvālākāraṃ akalaṃ    kiṃcid vastu varttate/   \N1
%tac-cakraṃ bhruvor-madhye dvidalakaṃ sthitaṃ /  tanmadhye  agnijvālākāraṃ akalaṃ    kiṃcid-vastu vartate/    \N2
%tac cakraṃ bhruvor madhye dvidalakaṃ sthitaṃ // tanmadhye  agnijvālākāraṃ akalaṃ    kiṃcid vastu varttate/   \D1
%tac-cakraṃ bhruvor-madhye dvidalakaṃ sthitaṃ    tanmadhye  agnijvālākāraṃ akala     kiṃcit vastu vartate/    \U1  
%                                                tanmadhye  agnijvālākārakamalaṃ //  kiṃcid-vastu varttate/ \U2   
%-----------------------    
%taccakraṃ bhrūvormadhye dvidalakaṃ sthitaṃ \varc{taccakraṃ bhrūvormadhye dvidalakaṃ sthitaṃ \nepal \dehlia}{dvidalaṃ \edprint \pune \lalchand} / tanmadhye agnijvālākāramakalaṃ\varc{akalaṃ \nepal \dehlia}{\om \edprint \pune \lalchand \oxford}\notes{agnijvālākārakamalaṃ}{\englishnote{\small \oxford starts here. All other folios before are missing.}} kiṃcidvastu vartate /
   \app{\lem[wit={N1,N2,D1,U1}, alt={tac cakraṃ bhruvor madhye dvidalakaṃ sthitaṃ}]{tac-cakraṃ bhruvor-madhye dvidalakaṃ sthitaṃ}
     \rdg[wit={E,P,L}]{dvidalaṃ}
     \rdg[wit={U2}]{\om}}
   tanmadhye
   \app{\lem[wit={N1,N2,D1}]{'gnijvālākāraṃ akalaṃ}
     \rdg[wit={E,P,L,B}]{agnijvālākāraṃ akalaṃ}
     \rdg[wit={U1}]{agnijvālākāraṃ akala}}\note[type=philcomm, labelb=s20.z11a, lem={agnijvālākāra°}]{Witness \getsiglum{B} starts here.}
   kiṃcidvastu vartate/
%-----------------------  
%na strī pumān     / tasya dhyānakāraṇāt  puruṣasya  śarīraṃ  ajarāmaraṃ bhavati /     \E
%na strī pumān    // tasyā dhyānakaraṇāt  puruṣasya  śarīraṃ  ajarāmaro  bhavati /     \B
%na strī pumān    // tasyā dhyānakaraṇāt  puruṣasya  śarīraṃ  ajarāmaro  bhavati /     \L
%na strī na pumān // tasyā dhyānakaraṇāt  puruṣasya  śarīraṃ  ajarāmaro  bhavati /     \P
%na strī na pumān /  tasya dhyānakaraṇāt  puruṣasya  śarīraṃ  ajarāmaraṃ bhavati      \N1
%na strī na pumān /  tasya dhyānakaraṇāt  puruṣasya  śarīraṃ  ajarāmaraṃ bhavati //   \N2
%na strī na pumān /  tasya dhyānakaraṇāt  puruṣasya  śarīraṃ  ajarāmaraṃ bhavati      \D1
%na strī na pumān    tasya dhyānakaraṇāt  puruṣasya  śarīraṃ  ajarāmaraṃ bhavati vā   \U1
%na strī na pumān /  tasya dhyānakāraṇāt/ puruṣasya--śarīram--ajarāmaraṃ bhavati /    \U2   
%-----------------------
   na strī
   \app{\lem[wit={P,N1,N2,D1,U1,U2}]{na pumān}
     \rdg[wit={E,B,L}]{pumān}}/
   puruṣasya \app{\lem[wit={E,N1,N2,D1,U1,U2}, alt={°ajarāmaraṃ}]{śarīramajarāmaraṃ}
     \rdg[wit={B,L,P}]{°ajarāmaro}}
   \app{\lem[wit={E,B,L,P,N1,N2,D1,U2}]{bhavati}
     \rdg[wit={U2}]{bhavati vā}}//   
 \end{prose}
\end{edition}
\begin{translation}
  \ekddiv{type=trans}
   \bigskip
    \centerline{\textrm{\small{[Description of the sixth Cakra]}}}
    \bigskip
  \begin{tlate}Now it exists a sixth cakra named Ājñā. \extra{[Its] god is Āgni?, [its] power is the godess of the centre (\textit{suṣumṇā}), [its] Ṛṣi is Hiṃsa, [its] mount is Caitanya, [its] body is Jñāna, [its] state is Vijñāna, [its] speech is incomparable (\textit{anupama}), [its] Veda is Sāma, [its] liṅgaṃ is intoxication (\textit{pramāda}), [its] half-measure? is Jīvahiṃsa [and] the support of play of Caitanya. [It has] two measures haṃ [and] kṣam [as its] inner measure. [Its] external measures [are] contemplation (\textit{sthiti}) [and] splendour (\textit{prabhā}).} This cakra is located in the middle of the eyebrows and is two-petalled. In its middle exists a certain object being a form of blazing fire without parts, not being female not being male. Because of the exercise of meditation on it the body of the person becomes non-aging and immortal.\end{tlate}
\end{translation}
\clearpage
\begin{edition}
  \ekddiv{type=ed}
   \bigskip
    \centerline{\textrm{\small{[Description of the seventh Cakra]}}}
    \bigskip
    \begin{prose}
%-----------------------
% idānīṃ saptamaṃ  tālumadhye catuḥṣaṣṭidalaṃ              amṛtapūrṇaṃ vartate / \E
% idānīṃ saptamaṃ  tālumadhye catuḥṣaṣṭhidalaṃ             amṛtapūrṇaṃ vartate / \P
% idānīṃ saptamaṃ  // tāludeśe madhye catuḥṣaṣṭhidala      amṛtapūrṇaṃ vartate / \L
% idānīṃ saptamaṃ  // tāludeśe madhye catuḥṣaṣṭhidala      amṛtapūrṇaṃ vartate / \B
% idānīṃ saptamaṃ  cakraṃ     catuḥṣaṣṭhidalaṃ tālumadhye  amṛtapūrṇaṃ varttate // \N1
% idānīṃ saptamaṃ  cakraṃ     catuṣaṣṭhidalaṃ tālumadhye   amṛtapūrṇa  varttate // \N2      
% idānīṃ saptamaṃ  cakraṃ     catuḥṣaṣṭhidalaṃ tālumadhye  amṛtapūrṇaṃ varttate // \D1
% idānīṃ saptamaṃ  cakraṃ     catuḥṣaṣṭhidalaṃ tālumadhye  amṛtapūrṇaṃ varttate // \U1
% idānīṃ saptamaṃ  tālumadhye catuḥṣaṣṭidalaṃ //           amṛtapūrṇaṃ vartate / \U2      
%-----------------------
% Now the seventh cakra having 64 petals and being full of nectar exists in the middle of the palate.
%-----------------------
%\extra{lalāṭamaṃḍalaṃ// caṃdrodevatā// amṛtāśaktiḥ// paramātmāṛṣiḥ// amṛtavāsinīkalāsaptadaśī amṛtakallolanadī// mahākāśa// aṃbikā// laṃbikā// ghaṃṭikā// tālikā// ajapāgāyatrīdehasvarūpaṃ// kākamukhī// naranetrāgośṛṃgālalāṭabrahmapaṭhāhayagrīvā// mayūramukhā// haṃsavadaṃgāni// ajapāgāyatrīsvarūpaṃ// 
%-----------------------
%Circle on the forehead, [its] god [is] the moon, [its] power [is] the nectar of immortality, [its] Rṣi is the supreme self, seventeen parts with the scent of nectar, sounding like a wave of immortality, [it is attributed to] the great space, the mother, the uvula, a small bell, having the nature of the body of the unspeakable Gayatrī, [having] the face of a crow, Mann-Auge-Kuh-Horn-Stirn-Brahmapaṭhā-Viṣṇu, [having] the face of a peacock, [having] limbs like a goose, [having] the nature of the unspeakable Gayatrī.    
%-----------------------
%ich fange mal so an: bei der Angabe der Elemente eines
%tantrischen Mantras finden sich in den
%Ritualhandbüchern (paddhati) Stellen wie die folgende:
%
%  śrīmahāgaṇapatimantrasya brahmā ṛṣiḥ gāyatraṃ chandaḥ
%  śrīmahāgaṇapatirdevatā gaṃ bījaṃ hrīṃ śaktiḥ namaḥ kīlakaṃ mama
%  śrīmahāgaṇapatiprasādātsarvasiddhyarthe śrījape viniyogaḥ/
%
%"Für dieses śrīmahāgaṇapatimantra ist Brahmā der Ṛṣi, gāyatraṃ das Metrum, Gaṇeśa dies Gottheit ...".
%Bis hier ist die Angabe nicht anders als für vedische Mantras, wo
%auch der Ṛṣi (also Autor), das Metrum und die Gottheit genannt
%werden müssen, sonst wirkt die Rezitation nicht.
%
%Dann kommen die tantrischen Elemente:
%
%gaṃ ist die Keimsilbe (im Mantra der Gottheit, nämlich: oṃ gaṃ mahāgaṇapataye namah. etc)
%hrīṃ = Śakti usw. Hier folgen dann noch beliebig viele tantrische Elemente.
%Am Ende kommt dann noch die Anwendung des Mantra.
%
%Dein Text scheint diese Struktur nachzubilden, aber merkwürdigerweise in der Beschreibung
%eines cakra. Man muß also vielleicht lesen:
%
%lalāṭa(ṃ) maṃḍalaṃ    Die Stirn ist das Maṇḍala
%caṃdro devatā        Mond die Gottheit
%amṛtā śaktiḥ
%paramātmā ṛṣiḥ
%amṛtavāsinī kalā saptadaśī
%amṛtakallolanadī mahākāśa
%aṃbikā laṃbikā
%ghaṃṭikā tālikā
%
%ajapāgāyatrīdehasvarūpaṃ
%kākamukhī//
%naranetrā
%gośṛṃgā
%lalāṭa brahmapaṭhā
%hayagrīvā//
%mayūra mukhā//
%haṃsavad aṃgāni//
%ajapāgāyatrī svarūpaṃ
%
%Die ajapā gāyatrī ist das mantra, welches der Atem ganztätig als so 'ham = haṃsa vollzieht.
%Steht auch in meiner Sahib Kaul-Paddhati. Diesem Mantra wird nun ein Körper zugeschrieben,
%der genauer beschrieben wird, mit Gesicht, Augen, Hörnern (?).
%
%Aber klar ist mir das auch nicht, jedenfalls wird hier ein cakra wie das mantra einer Gottheit
%behandelt. In jedem Fall interessant.
%
%Liebe Grüße
%Jürgen   
%-----------------------
idānīṃ saptamaṃ
      \app{\lem[wit={N1,D1,U1}]{cakraṃ catuḥṣaṣṭhidalaṃ tālumadhye}
        \rdg[wit={N2}]{cakraṃ catuṣaṣṭhidalaṃ tālumadhye}
        \rdg[wit={E,P,U2}]{tālumadhye catuḥṣaṣṭidalaṃ}
        \rdg[wit={L,B}]{tāludeśe madhye catuḥṣaṣṭhidala}}
      \app{\lem[type=emendation, resp=egoscr]{'mṛtapūrṇaṃ}
        \rdg[wit={E,P,L,B,N1,D1,U1,U2}]{\korr amṛtapūrṇaṃ}
        \rdg[wit={N2}]{amṛtapūrṇa}}
      vartate/ \extra{lalāṭamaṃḍalaṃ/ caṃdrodevatā/ amṛtāśaktiḥ/ paramātmāṛṣiḥ/ amṛtavāsinīkalāsaptadaśī amṛtakallolanadī/ mahākāśa/ aṃbikā/ laṃbikā/ ghaṃṭikā/ tālikā/ ajapāgāyatrīdehasvarūpaṃ/ kākamukhī/ naranetrāgośṛṃgālalāṭabrahmapaṭhāhayagrīvā/ mayūramukhā/ haṃsavadaṃgāni/ ajapāgāyatrīsvarūpaṃ/ adhikataraprabhā?muktaṃ/ atiśvetaṃ/ tanmadhye raktavarṇaṃ ghaṭikāsaṃjñā/}
%-----------------------
%adhikaśobhāyuktam-----atiśvetaṃ       tanmadhye       raktavarṇaṃ ghāṃṭikāsaṃjñaikā      karṇikā varttate / \E 
%adhikataraśobhayuktaṃ atiśvetaṃ       tanmadhye       raktavarṇaṃ ghaṭikāsaṃjñā ekā      karṇikā varttate / \P
%adhikataraśobhayuktaṃ // atiśvetaṃ // tanmadhye       raktavarṇaṃ ghaṇikāsaṃjñā ekā ekā  karṇikā varttate / \L
%adhikataraśobhayuktaṃ // atiśvetaṃ // tanmadhye       raktavarṇaṃ ghaṃṭikāsaṃjñā ekā ekā karṇikā varttate / \B
%adhikataraśobhayuktaṃ atiśvetaṃ       tanmadhye       raktavarṇaṃ ghaṃṭikāsaṃjñā ekā     karṇikā varttate / \N1
%adhikataraśobhāyuktaṃ  atiśvetaṃ      tanmadhye       raktavarṇa--ghaṇṭikāsaṃjñā ekā     karṇikā vartate /  \N2
%adhikataraśobhayuktaṃ atiśvetaṃ       tanmadhye       raktavarṇaṃ ghaṃṭikāsaṃjñā ekā     karṇikā varttate / \D1
%adhikataraśobhayuktaṃ atiśvetaṃ       tanmadhye       raktavarṇaṃ ghaṃṭikāsaṃjñā ekā     karṇikā varttate / \U1      
%adhikataraprabhāmuktaṃ // atiśvetaṃ //tanmadhye       raktavarṇaṃ ghaṃṭikāsaṃjñā// ekā   karṇikā varttate / \U2   
%-----------------------
%[It is] endowed with superabundant beauty. [It is] very bright. In its middle, red in color [is that] known as "uvula" (\textit{ghāṃṭikā}). [It] exists as a single pericarp.  
%-----------------------      
      adhi\app{\lem[wit={P,L,B,N1,D1,U1}]{°kataraśobhayuktaṃ}
        \rdg[wit={N2}]{°kataraśobhāyuktaṃ}
        \rdg[wit={E}]{°kaśobhāyuktam}
          \rdg[wit={U2}]{°kataraprabhāmuktaṃ}}/
        atiśvetaṃ/
        tanmadhye
        \app{\lem[wit={E,P,L,B,N1,D1,U1,U2}]{raktavarṇaṃ}
          \rdg[wit={N2}]{raktavarṇa°}}
        \app{\lem[wit={B,N1,N2,D1,U1,U2},alt={ghaṇṭikā°}]{ghaṇṭikā}
          \rdg[wit={E}]{ghāṃṭikā°}
          \rdg[wit={P}]{ghaṭikā°}
          \rdg[wit={L}]{ghaṇikā°}}saṃjñā/
          \app{\lem[wit={E,P,N1,N2,D1,U1,U2}]{ekā}
            \rdg[wit={L,B}]{ekā ekā}}
          karṇikā vartate/
%-----------------------          
%tanmadhye bhūmiḥ / \E
%tanmadhye bhūmiḥ / \P
%tanmadhye bhūmiḥ / \L
%tanmadhye bhūmiḥ / \B
%tanmadhye bhūmiḥ / \N1
%tanmadhye bhūmiḥ / \N2
%tanmadhye bhūmiḥ / \D1
%tanmadhye bhūmis- / \U1
%tanmadhye bhūmi   / \U2         
%-----------------------
%In its middle is a place. 
%-----------------------        
       tanmadhye
       \app{\lem[wit={E,P,L,B,N1,N2,D1}]{bhūmiḥ}
         \rdg[wit={U1}]{bhūmis°}
         \rdg[wit={U2}]{bhūmi}}/
%-----------------------  
%tanmadhye prakaṭacandrakalā 'mṛtādhārā bhavati         / \E
%tanmadhye prakaṭacandrakalā 'mṛtādhārā sravati         / \P
%tanmadhye prakaṭacandrakalā 'mṛtādhārā sravaṃti        / \L
%tanmadhye prakaṭacandrakalā 'mṛtādhārā sravaṃti        / \B
%tanmadhye prakaṭacandrakalā amṛtādhārāsravaṃtī varttate/ \N1
%tanmadhye prakaṭacaṃdrakalā amṛtādhārāsravaṃtī varttate/ \N2
%tanmadhye prakaṭacandrakalā 'mṛtādhārāsravaṃtī varttate/ \D1 %sravantī f. Fluss Nom Sg
%tanmadhye pragaṭacaṃdrakalā amṛtadhārāsravaṃtī varttate  \U1
%tanmadhye-ṃdrakaṭaṃ caṃdrakalā amṛtadhārā sravati       /\U2       
%-----------------------
%In its middle exists a flow of nectar like a river (\textit{amṛtādhārāsravantī}, appearing from the digits of the moons disc.
%-----------------------
       tanmadhye
       \app{\lem[wit={E,P,L,B,N1,N2,D1},alt={prakaṭa°}]{prakaṭa}
         \rdg[wit={U1}]{pragaṭa}
         \rdg[wit={U2}]{°ṃdrakaṭaṃ}}candrakalā
       \app{\lem[wit={N1,N2,D1,U1}]{amṛtadhārāsravantī}
         \rdg[wit={L,B}]{'mṛtādhārā sravaṃti}
         \rdg[wit={P,U2}]{'mṛtādhārā sravati}
         \rdg[wit={E}]{'mṛtādhārā bhavati}}
       \app{\lem[wit={N1,N2,D1,U1}]{vartate}
         \rdg[wit={E,P,L,B,U2}]{\om}}/
%-----------------------
%tasyāḥ kalāyā     dhyānakāraṇāt tasya samīpe maraṇaṃ nāyāti/     \E -> does not come near to death -> na-ā-yāti
%tasyāḥ kalāyā     dhyānakaraṇāt tasya samīpe maraṇaṃ nāyāti/     \P
%tasyāḥ karṇikāyā  dhyānakaraṇāt tasya samīpe maraṇaṃ na yāti     \L
%tasyāḥ karṇikāyā  dhyānakaraṇāt tasya samīpe maraṇaṃ na yāti     \B
%tasyāḥ kalāyāḥ    dhyānakaraṇāt tasya samīpe maraṇaṃ nāyāti      \N1
%tasyāḥ kalāyāḥ    dhyānakaraṇāt tasya samīpe maraṇaṃ nāyāti/     \N2       
%tasyāḥ kalāyāḥ    dhyānakaraṇāt tasya samīpe maraṇaṃ nāyāti      \D1
%tasyāḥ kalāyā     dhyānakaraṇāt tasya samīpe maraṇaṃ nāyāti/     \U1
%tasyāḥ kalāyā     dhyānakāraṇāt// tasya samīpe maraṇaṃ na yāti/  \U2
%-----------------------
%Because of the exercise of meditation on this digit death does not come near him. 
%-----------------------
       tasyāḥ
       \app{\lem[wit={E,P,U1,U2}]{kalāyā}
         \rdg[wit={N1,N2,U1}]{kalāyāḥ} %Sandhi-mistake in apparatus in this case?
         \rdg[wit={L,B}]{karṇikāyā}}
       dhyānakaraṇāt tasya samīpe maraṇaṃ
       \app{\lem[wit={E,P,N1,N2,D1,U1}]{nāyāti}
         \rdg[wit={L,B,U2}]{na yāti}}/    
%-----------------------
%nirantaradhyānād        -amṛtadhārāyāḥ sajīvo bhavati /  \E
%niraṃtaradhyānāt---------amṛtadhārā plāvanaṃ   bhavati /  \P
%niraṃtaradhyānakaraṇād   amṛtadhārā           sravati /  \L
%niraṃtaradhyānakaraṇād   amṛtadhārā           sravati /  \B
%niraṃtaradhyānakaraṇāt / amṛtadhārā           sravaṃti / \N1
%niraṃtaradhyānakaraṇāt   amṛtadhārā            sravaṃti    \N2
%niraṃtaradhyānakaraṇāt / amṛtadhārā           sravaṃti / \D1
%niraṃtaradhyānakaraṇāt   amṛtadhārā             sravati /  \U1
%niraṃtaradhyānakaraṇāt / amṛtadhārā plavanaṃ  bhavati / \U2
%-----------------------
%Due to uninterrupted meditation the stream (\textit{dhārā}) of nectar flows. 
%-----------------------
       \app{\lem[wit={L,B,N1,N2,D1,U1,U2},alt={niraṃtaradhyānakaraṇād}]{niraṃtaradhyānakaraṇād\skm{a}}
         \rdg[wit={E,P}]{nirantaradhyānād}}\app{\lem[wit={L,B,N1,N2,D1,U1}, alt={amṛtadhārā}]{\skp{-a}mṛtadhārā}
         \rdg[wit={E}]{amṛtadhārāyāḥ sajīvo}
         \rdg[wit={P}]{amṛtadhārā plāvanaṃ}
         \rdg[wit={U2}]{amṛtadhārā plavanaṃ}}
       \app{\lem[wit={L,B,U1}]{sravati}
         \rdg[wit={N1,N2,D1}]{sravaṃti}
         \rdg[wit={E,P,U2}]{bhavati}}/       
%-----------------------
%tadā  yakṣam-aroga----pittajvarahṛdayadāha-śiroroga-jihvā--jaḍa-bhāvā           naśyanti / \E
%tadā     kṣayaroga----pittajvarahṛdayadāha-śiroroga-jihvā--jaḍa-bhāvān          naśyanti / \P
%tadā     kṣayaroga----pittajvarahṛdayadāha-----roga-jihvāyājaḍa-bhāvān          naśyanti / \L
%tadā     kṣayaroga----pittajvarahṛdayadāha-----roga-jihvāyājaḍa-vān             naśyanti / \B
%         kṣayarogaṃ   pittajvarahṛdayadāha-śiroroga-jihvāyājaḍa-bhāvā           naśyanti / \N1 %besser kṣayarogaṃ emendieren zu vollem Kompositum?
%         kṣayarogaṃ   pittajvarahṛdayadāha-śiroroga-jihvāyājaḍa-bhāvātā         naśyanti / \N2
%         kṣayaṃ rogaṃ pittajvarahṛdayadāha-śiroroga-jihvāyājaḍa-bhāvā           naśyanti / \D1
%         kṣayaroga----pittajvarahṛdayadāha-śiroroga-jihvāyājaḍa-bhāvā           naśyanti / \U1  
%tadā     kṣayarogo----ptatti// jvara hṛdayadāha// śiroroga// jihvājaḍatā// dayo naśyanti / \U2       
%-----------------------
%Then the appearances of emaciation (\textit{kṣayaroga}), fever due to disordered bile (\textit{pittajvara), heartburn (\textit{hṛdayadāha}), head-disease (\textit{śiroroga}) and tongue insensibility (\textit{jihvājaḍa}) vanish. %!!!Krankheiten in Ayurvedabuch checken! medizinische Identifikationen!
%-----------------------
       \app{\lem[wit={E,P,L,B,U2}]{tadā}
         \rdg[wit={N1,N2,D1,U1}]{\om}}
       \app{\lem[type=emendation, resp=egoscr]{kṣayarogapittajvarahṛdayadāhaśirorogajihvājaḍabhāvā}
         \rdg[wit={E}]{\korr yakṣamarogapittajvarahṛdayadāhaśirorogajihvājaḍabhāvā}
         \rdg[wit={P}]{kṣayarogapittajvarahṛdayadāhaśirorogajihvājaḍabhāvān}
         \rdg[wit={L}]{kṣayarogapittajvarahṛdayadāharogajihvāyājaḍabhāvān}
         \rdg[wit={B}]{kṣayarogapittajvarahṛdayadāharogajihvāyājaḍavān}
         \rdg[wit={N1}]{kṣayarogaṃ pittajvarahṛdayadāhaśirorogajihvāyājaḍabhāvā}
         \rdg[wit={N2}]{kṣayarogaṃ pittajvarahṛdayadāhaśirorogajihvāyājaḍabhāvātā}
         \rdg[wit={D1}]{kṣayaṃ rogaṃ pittajvarahṛdayadāhaśirorogajihvāyājaḍabhāvā}
         \rdg[wit={U1}]{kṣayarogapittajvarahṛdayadāhaśirorogajihvāyājaḍabhāvā}
         \rdg[wit={U2}]{kṣayarogoptatti// jvara hṛdayadāha// śiroroga// jihvājaḍatā// dayo}}
         naśyanti/
%-----------------------       
%bhakṣitam--api   viṣan    na bādhate / \E
%bhakṣitam--api   viṃṣa    na bādhate / \P
%bhākṣitam--api   viṣaṃ    na bādhyate / \L
%bhākṣitamār pi   viṣaṃ    na bādhyate / \B
%bhakṣitam        viṣamapi na bādhyate / \N1
%bhakṣitaṃ        viṣamapi na bādhate / \N2
%bhakṣitāṃ        viṣamapi na bādhyate / \D1
%bhakṣitaṃ        viṣamapi na bādhyate   \U1
%bhakṣitam--api   viṣaṃ    na bādhyate / \U2       
%-----------------------       
%Also eaten venom doesn't trouble him. 
%-----------------------
         \app{\lem[wit={N2,U1}]{bhakṣitaṃ}
           \rdg[wit={N1}]{bhakṣitam}
           \rdg[wit={D1}]{bhakṣitāṃ}
           \rdg[wit={E,P,L,U2}]{bhakṣitam api}
           \rdg[wit={B}]{bhākṣitamār pi}}
         \app{\lem[wit={N1,N2,D1,U1}, alt={viṣam api}]{viṣam-api}
           \rdg[wit={L,B,U2}]{viṣaṃ}
           \rdg[wit={E}]{viṣan}
           \rdg[wit={P}]{viṃṣa}}
         na
         \app{\lem[wit={E,P,N2}]{bādhate}
           \rdg[wit={L,B,N1,D1,U1,U2}]{bādhyate}}/       
%-----------------------
%yady-atra manaḥ sthiraṃ   bhavati /  \E
%yady-atra manaḥ sthiraṃ   bhavati /  \P
%yady-atramapi manasthiraṃ bhavati /  \L              %VARIANTE UNSICHER!!!WAS MEINT JÜRGEn??
%yady-atramapi manasthiraṃ bhavati /  \B
%yady-atra     manasthiraṃ bhavati /  \N1
%yadyanna      manasthiraṃ bhavati // \N2
%yadyanna      manasthiraṃ bhavati /  \D1
%yadyatra      manasthiraṃ bhavati    \U1
%yadyatra      manasthiraṃ bhavati//  \U2       
%-----------------------
%If here the mind becomes stable.       
%-----------------------
         \app{\lem[wit={E,P,N1,U1,U2}]{yadyatra}
           \rdg[wit={L,B}]{yadyatram api}
           \rdg[wit={N1,D1}]{yadyanna}}
         \app{\lem[wit={E,P}]{manaḥ sthiraṃ}
           \rdg[wit={L,B,N1,N2,D1,U1,U2}]{manasthiraṃ}}
         bhavati//         
    \end{prose}
\end{edition}
\begin{translation}
  \ekddiv{type=trans}
    \bigskip
    \centerline{\textrm{\small{[Description of the seventh Cakra]}}}
    \bigskip
  \begin{tlate}
    Now the seventh cakra having 64 petals and being full of nectar exists in the middle of the palate. \extra{The forehead is [its] Maṇḍala, the moon [its] deity, the nectar of immortality [its] power, the highest self [its] Ṛṣi, [endowed with] seventeen parts [having] the scent of nectar, [it is attributed to] the great space sounding like a wave of immortality. [Its] mother is Laṃbikā?, the uvula [which is attributed to it] being a kind of gem?. [Its] body has the nature of the unspeakable Gayatrī (\textit{ajapāgāyatrī}), [which has] the face of a crow, the eye of a human, the horn of a cow, the forehead of Brahmapaṭhā Hayagrīvā, the face of a peacock and the limbs of a goose. [This is] the nature of the unspeakable Gayatrī (\textit{ajapāgāyatrī}).}   
%lalāṭa(ṃ) maṃḍalaṃ    Die Stirn ist das Maṇḍala
%caṃdro devatā        Mond die Gottheit
%amṛtā śaktiḥ
%paramātmā ṛṣiḥ
%amṛtavāsinī kalā saptadaśī
%amṛtakallolanadī mahākāśa
%aṃbikā laṃbikā
%ghaṃṭikā tālikā
%
%ajapāgāyatrīdehasvarūpaṃ
%kākamukhī//
%naranetrā
%gośṛṃgā
%lalāṭa brahmapaṭhā
%hayagrīvā//
%mayūra mukhā//
%haṃsavad aṃgāni//
%ajapāgāyatrī svarūpaṃ
%
%Die ajapā gāyatrī ist das mantra, welches der Atem ganztätig als so 'ham = haṃsa vollzieht.
%Steht auch in meiner Sahib Kaul-Paddhati. Diesem Mantra wird nun ein Körper zugeschrieben,
%der genauer beschrieben wird, mit Gesicht, Augen, Hörnern (?).
%
[It is] endowed with superabundant beauty. [It is] very bright. In its middle, red in color [is that] known as "uvula" (\textit{ghāṃṭikā}). [It] exists as a single pericarp. In its middle is a [certain] site. In the middle of it exists a flow of nectar like a river (\textit{amṛtādhārāsravantī}), appearing from the digits of the moons disc. Because of the exercise of meditation on this digit death does not come near him. Due to uninterrupted meditation the stream (\textit{dhārā}) of nectar flows.Then the appearances of emaciation (\textit{kṣayaroga}), fever due to disordered bile (\textit{pittajvara}), heartburn (\textit{hṛdayadāha}), head-disease (\textit{śiroroga}) and tongue insensibility (\textit{jihvājaḍa}) vanish. Also eaten venom doesn't trouble him. If here the mind becomes stable.     
  \end{tlate}
\end{translation}
\end{alignment}
\clearpage
\begin{alignment}[
    texts=edition[class="edition"];
    translation[class="translation"],
  ]
\begin{edition}
  \ekddiv{type=ed}
      \bigskip
    \centerline{\textrm{\small{[Description of the eigth Cakra]}}}
    \bigskip
    \begin{prose}
%-----------------------
%idānīṃ brahmarandhrasthāne 'ṣṭamaṃ śatadalaṃ cakraṃ varttate / \E
%idānīṃ brahmaraṃdhrasthāne 'ṣṭamaṃ śatadalaṃ cakraṃ vartate / \P
%idānīṃ brahmaraṃdhrasthāne aṣṭamaṃ śatadalaṃ cakraṃ vartate / \L
%idānīṃ brahmaraṃdhrasthāne aṣṭamaṃ śatadalaṃ cakraṃ vartate / \B
%idānīṃ aṣṭamacakraṃ brahmaraṃdhrasthāne śatadalaṃ   vartate / \N1
%idānīṃ aṣṭamacakraṃ brahmaraṃdhrasthāne śatadalaṃ   vartate  \N2
%idānīṃ aṣṭamacakraṃ brahmaraṃdhrasthāne śatadalaṃ   vartate / \D1
%idānīṃ aṣṭamaṃ cakraṃ brahmaraṃdhrasthāne śatadalaṃ   vartate . \U1
%idānīṃ brahmaraṃdhrasthāne 'ṣṭamaṃ śatadalaṃ cakraṃ varttate // \U2
%-----------------------
%gurudevatā// caitanyaśaktiḥ// virāṭu?ṛṣiḥ// sarvotkṛṣṭasākṣiḥ// bhūtaturyātītacaitanyātmakaṃ// sarvavarṇāḥ// sarvamātrāḥ// sarvadalāni virāṭudeha sthitāvasthā prajñāvācā sohaṃ veda anupamasthānaṃ// ajapājapasahasra/ 1000 gha 02 pa 046 akṣara40// sarvajapasaṃkhyā// 21600// ekaviṃśatisahasrāṇiṣaṭśatāni// tathaivaca niśāhevahate// prāṇaḥ yojānātisapaṃḍitaḥ// sakāreṇa bahiryātihakāreṇaviśotpunaḥ// haṃsaḥ sohaṃ// tato maṃtraṃ jīvojapati sarvadā//    
%-----------------------
%Now exists the eighth \textit{cakra} having one hundred petals located at the aperture of Brahman.
%-----------------------
      idānīṃ
      \app{\lem[wit={N1,N2,D1}]{aṣṭamacakraṃ brahmaraṃdhrasthāne śatadalaṃ}
        \rdg[wit={E,P,U2}]{brahmarandhrasthāne 'ṣṭamaṃ śatadalaṃ cakraṃ}
        \rdg[wit={L,B}]{brahmaraṃdhrasthāne aṣṭamaṃ śatadalaṃ cakraṃ}
        \rdg[wit={U1}]{cakraṃ brahmaraṃdhrasthāne śatadalaṃ}}
      vartate/ \extra{gurudevatā/ caitanyaśaktiḥ/ virāṭuṛṣiḥ/ sarvotkṛṣṭasākṣiḥ/ bhūta-turyātīta-caitanyātmakaṃ/ sarvavarṇāḥ/ sarvamātrāḥ/ sarvadalāni virāṭudehasthitāvasthā prajñāvācā sohaṃ veda anupamasthānaṃ/ sarvajapasaṃkhyā/ ekaviṃśatisahasrāṇiṣaṭśatāni/ tathaiva ca niśāhevahate/ prāṇaḥ yojānātisapaṃḍitaḥ/ sakāreṇa bahir-yāti hakāreṇa viśotpunaḥ/ haṃsaḥ sohaṃ/ tato mantraṃ jīvo japati sarvadā/}
%[Its] deity is the Guru, [its] power is consciousness (\textit{caitanya}), [its] Ṛṣi is Virāṭu, [attributed to it is] the witness being eminent in everything, [being] characterized by the soul that is beyond the fourth state of beings. [It has] all colours. [It has] all measures. [It has] all petals. [Its] state is being established in the body of Virāṭu. [Its] speech is wisdom. [It is attributed to] the "I am that"-[expression] (\textit{soham}), the Veda [in general] [and] the matchless place (\textit{anupamasthāna}). [It is associated with] the count of all whispered utterings [of Mantras]. [It is associated with the number] 21600. And in this way Niśāhevahate. The breath is a the pair of yojānātisapaṃḍitaḥ? With the sound of "sa" externally he goes, with the sound of "ha" viśotpunaḥ: "I am he, he is I". Because of that the embodied soul constantly utters the Mantra. 
%----------------------
%tasya kamala----jātyadharaṇīpīṭha iti saṃjñā / \E
%tasya kamalasya jālaṃdharapīṭha iti saṃjñā / \P
%tasya kamalasya jālaṃdharapīṭha iti saṃjñā ...  \L
%tasya kamalasya jālaṃdharapīṭhasaṃjñā ...  \B
%tasya kamalasya jālaṃdharapīṭha iti saṃjñā ...  \N1
%tasya kamalasya jālaṃdharapīṭha iti saṃjñā ...  \N2
%tasya kamalasya jālaṃdharapīṭha iti saṃjñā ...  \D1
%tasya kamalasya jālaṃdharapīṭha iti saṃjñā ...  \U1      
%tasya kamalasya jālaṃdharapīṭha iti saṃjñā //   \U2
      %----------------------
%``The (divine) seat of  Jālaṃdhara'' is the designation of the lotus of it. 
%----------------------      
      tasya \app{\lem[wit={P,L,B,N1,N2,D1,U1,U2}]{kamalasya}
        \rdg[wit={E}]{kamala°}}
      \app{\lem[wit={P,L,N1,N2,D1,U1,U2}]{jālandharapīṭha}
        \rdg[wit={B}]{jālandharapīṭha°}
        \rdg[wit={E}]{jātyadharaṇīpīṭha}}
      \app{\lem[wit={E,P,L,N1,N2,D1,U1,U2}]{iti}
        \rdg[wit={B}]{\om}}
      \app{\lem[wit={E,P,L,N1,N2,D1,U1,U2}]{saṃjñā}
        \rdg[wit={B}]{°saṃjñā}}/
      \linebreak
%---------------------- 
%siddhapuruṣasya sthānam / \E
%siddhapuruṣasya sthānam / \P
%siddhapuruṣasya sthānam mūrti vartate // \L                         %%% schwerer Satz -> wie soll ich hier entscheiden?! 
%siddhapuruṣasya sthānam mūrti vartate // \B %Zeilensprung
%siddhapuruṣasya sthānam // \N1
%siddhapuruṣasya sthānam // \N2
%siddhapuruṣasya sthānam // \D1
%siddhapuruṣasya sthānam    \U1
%siddhapuruṣasya sthānaṃ   \U2
%----------------------      
%[It is] the place of the accomplished person.
%----------------------
            siddhapuruṣasya
      \app{\lem[wit={E,P,N1,N2,D1,U1,U2}]{sthānaṃ}
        \rdg[wit={L,B}]{sthānam mūrti vartate}}/
%----------------------
%tanmadhye    'gnidhūmākārarekhā     yādṛśy    ādṛśy ekā  puruṣasya mūrttir varttate / \E
%tanmadhye    'gnidhūmākārarekhā     yādṛśī   tādṛśy ekā  puruṣasya mūrttir varttate / \P
%tanmadhye    'gnidhūmākārārekhā     yādṛśī   tādṛśy ekā  puruṣasya mūrttir varttate / \L               
%tanmadhye    'gnidhūmākārārekhā     yādṛśī   tādṛśy ekā  puruṣasya mūrttir varttate / \B
      
%tanmadhye    'gnidhūmākārāreṣā      yādṛśī   tādṛśī ekā  puruṣasya mūrttir varttate / \N1
%tanmadhye    agnidhūmrākārarekhā    yādṛśī / tādṛśī ekā  puruṣasya mūrttir varttate / \N2
%tanmadhye    agnidhūmākārāreṣā      yādṛśī   tādṛśī ekā  puruṣasya mūrttir varttate / \D1
%tanmadhye    agnidhūmrākārārekhā    yādṛśī   tādṛśī ekā  puruṣasya mūrtir  vartate    \U1
%tanmadhye    'gnidhūmrākārārekhāyāḥ  etādṛśī         ekā  puruṣasya mūrtir  vartate // \U2
%----------------------      
%In its middle [is] something like a streak having the form of smoke and fire. Such a single [divine] form of the person (\textit{puruṣa}) exists [there].        
%---------------------      
      tanmadhye
      \app{\lem[wit={E,P,L,B}]{'gnidhūmākārarekhā}
        \rdg[wit={N1,D1}]{'gnidhūmākārāreṣā}
        \rdg[wit={N2,U1}]{agnidhūmrākārarekhā}
        \rdg[wit={U2}]{'gnidhūmrākārārekhāyāḥ}}
      \app{\lem[wit={P,L,B,N1,N2,D1,U1,U2}]{yādṛśī}
        \rdg[wit={E}]{yādṛśy°}
        \rdg[wit={U2}]{etādṛśī}}/
      \app{\lem[wit={P,L,B}]{yādṛśy}
        \rdg[wit={E}]{ādṛsy}
        \rdg[wit={N1,N2,D1,U1}]{yādṛśī}
        \rdg[wit={U2}]{\om}}ekā puruṣasya mūrtir-vartate/
%---------------------
%tasyā  nādir nāṃto 'sti / \E
%tasyā  nādināṃ 'to sti / \P
%tasyā  nādir nāṃto sti / \L -> vor dem bei allen anderen vorigen Satz!?!?!?! 
%tasyā  nādir nāṃto sti / \B -> vor dem bei allen anderen vorigen Satz!?!?!?! 
%tasyāḥ nāstyaṃtaḥ ādir-api nāsti / \N1????
%tasyāḥ nāstyaṃtaḥ ādir-api nāsti / \N2
%tasyāḥ nāstyaṃtaḥ ādir api nāsti / \D1 
%tasyāḥ nāstyaṃtaḥ ādir-api nāsti    \U1
%tasyā  nādir naṃto sti              \U2
%---------------------
% Of her exists no end, nor a beginning.
%---------------------      
      \app{\lem[wit={E,P,L,B}]{tasyā} %Sandhi-difference included! 
        \rdg[wit={N1,N2,D1,U1}]{tasyāḥ}}
      \app{\lem[alt={nādir nānto 'sti}, wit={E,L,B,U2}]{nādir-nānto 'sti}
        \rdg[wit={N1,N2,D1,U1}]{nāstyaṃtaḥ ādir api nāsti}
        \rdg[wit={P}]{nādināṃ 'to sti}}/        
%---------------------    
%tasyā  mūrtter dhyānakāraṇāt pratyakṣaṃ niraṃtaraṃ  puruṣasyākāśe   gamāgamau   bhavataḥ / \E
%tasyā  mūrtter dhyānakaraṇāt pratyakṣaniraṃtaraṃ    puruṣasyākāśe   gamāgamau   bhavataḥ / \P
%tasyā  mūrtir  dhyānakaraṇāt pratyakṣaniraṃtaraṃ    puruṣasyākāśe   gamāgamau   bhavataḥ / \L         
%tasyā  mūrtir  dhyānakaraṇāt pratyakṣaṃ niraṃtaraṃ  puruṣasyākāśe   gamāgamau   bhavataḥ / \B
%tasyāḥ mūrttir dhyānakaraṇāt pratyakṣaniraṃtaraṃ    puruṣasya ākāśe gamāgamau   bhavataḥ / \N1
%tasyāḥ mūrttir dhyānakaraṇāt pratyakṣaniraṃtaraṃ    puruṣa ākāśe    gamāgame    bhavataḥ / \N2
%tasyāḥ mūrtir  dhyānakaraṇāt pratyakṣaniraṃtaraṃ    puruṣasya ākāśe gamāgamau   bhavataḥ / \D1
%tasyāḥ mūrter  dhyānakaraṇāt/ pratyakṣaniraṃtaraṃ   puruṣasya ākāśi gamāmamo   bhavataḥ   \U1
%tasyāḥ mūrter  dhyānakaraṇāt pratyakṣaniraṃtaraṃ    puruṣasyākāśa---gamāgamau bhavata //      \U2
%---------------------    
%BEDEUTUNG DES SATZES BIS JETZT UNKLAR! Idee: Zeilensprung aus übernächstem Satz! Streiche pratyakṣaṃ niraṃtaraṃ und der Satz ergibt Sinn!  
%gamāgamau nom.  dual = coming and going ; bhavataḥ = 3p du ind pres von bhū
%Due to the exercise of meditation on this (divine) form both coming and going of the person in space occurs. 
%Kolloquium: Meinung zu Kompositum pratyakṣaniraṃtaraṃ = macht wenig Sinn oder?
%{\englishnote{\small Even though every single witness at hand transmits the latter reading right after \textit{°karaṇāt}, several considerations make it reasonable to conject that the original sentence is corrupted and was written without it. The main consideration to assume the corruption is that \textit{pratyakṣaṃ nirantaraṃ} is ungrammatical. The second is that the sentence is way more meaningful without it. The third that two sentences later we get the phrase in a meaningful context. Due to the last consideration my best guess is an interlace at an early stage of transmission.}}
%---------------------
      tasyā \app{\lem[alt={mūrter},wit={E,P,U1,U2}]{mūrte\skp{r-}}
        \rdg[wit={L,B,N1,N2,D1}]{mūrtir}}\app{\lem[alt={dhyānakaraṇāt},type=conjecture, resp=egoscr]{\skm{-r}-dhyānakaraṇāt}
        \rdg[wit={E,B}]{dhyānakāraṇāt pratyakṣaṃ niraṃtaraṃ}
        \rdg[wit={P,L,N1,N2,D1,U1,U2}]{dhyānakaraṇāt pratyakṣaniraṃtaraṃ}}
      \note[type=philcomm, labelb=s22.z4, lem={°kāraṇāt pratyakṣaṃ niraṃtaraṃ}]{Even though every single witness at hand transmits the latter reading right after °\textit{karaṇāt}, several considerations make it reasonable to conject that the original sentence is corrupted and was written without it. The main consideration to assume the corruption is that the syntactical units \textit{pratyakṣaṃ nirantaraṃ} is ungrammatical in this construction. The second is that the sentence is way more meaningful without it. The third that two sentences later we get the phrase in a meaningful context. Due to the last consideration my best guess is an interlace at an early stage of transmission.}
      \app{\lem[wit={E,P,L,B,N1,D1}]{puruṣasyākāśe}
        \rdg[wit={N2}]{puruṣa ākāśe}
        \rdg[wit={U2}]{puruṣasyākāśa°}
        \rdg[wit={U1}]{puruṣasya ākāśi}}
      \app{\lem[wit={E,P,L,B,N1,D1,U2}]{gamāgamau}
        \rdg[wit={U1}]{°gamo}
        \rdg[wit={N2}]{°game}}
        \app{\lem[wit={E,P,L,B,N1,N2,D1,U1}]{bhavataḥ}
          \rdg[wit={U2}]{bhavata}}/
%---------------------     
%pṛthvīmadhye  sthitasyāpi    pṛthvī-------bādho   na bhavati / \E
%pṛthvīmadhye  sthitasyāpi    pṛthaka                 bhavati   \P %Zeilenspringer führt zu Verlust von Zeile in Pune
%pṛthvīmadhye  sthitasyāpi    pṛthvī-------bādho   na bhavati / \L
%pṛthivīmadhye sthitasyāpi // pṛtvī--------bādho   na bhavati // \B
%pṛthvīmadhye  sthitāv-api    pṛthvī kṣato bādho   na bhavati // \N1
%pṛthvīmadhye  sthitāv-api    pṛthvī kṣato bādho   na bhavati // \N2      
%pṛthvīmadhye  sthitāv-api    pṛthvī kṣato bādho   na bhavati // \D1
%pṛthvīmadhye  sthitāv-api    pṛthvī kṣato bādho   na bhavati     \U1
%pṛthīvīmadhye sthitasyāpi    pṛthvī       bādhoko na bhati     \U2
%---------------------
%Affliction from the earth-element does not arise [anymore] even if one is situated in the middle of the earth.        
%---------------------
        \app{\lem[wit={E,P,L,N1,N2,D1,U1}]{pṛthvīmadhye}
          \rdg[wit={B,U2}]{pṛtivīmadhye}}
        \app{\lem[wit={E,P,L,B,U2}]{sthitasyāpi}     
          \rdg[wit={N1,N2,D1,U1}]{sthitāv-api}}
        \app{\lem[wit={E,L}]{pṛthvībādho}
          \rdg[wit={B}]{pṛtvībādho}
          \rdg[wit={N1,N2,D1,U1}]{kṣato bādho}
          \rdg[wit={P}]{pṛthaka}
          \rdg[wit={U2}]{pṛthvī bādhoko}}
        \app{\lem[wit={E,L,B,N1,N2,D1,U1}]{na bhavati}
          \rdg[wit={P}]{bhavati}
          \rdg[wit={U2}]{na bhati}}/
%---------------------
%sakalān pratyakṣaṃ niraṃtaraṃ paśyati ca pṛthagbhavati / \E
% \om                                                       \P      
%sakalāḥ pratyakṣaṃ niraṃtara paśyatī  ca pṛthak bhavati // \B
%sakalāḥ pratyakṣaṃ niraṃtara paśyatī  ca pṛthak bhavati / \L
%sakalāpratyakṣaniraṃtaraṃ    paśyati  ca pṛthak ca bhavati // \N1
%sakalapratyakṣaniraṃtaraṃ    paśyati  ca pṛthak ca bhavati    \N2      
%sakalāpratyakṣaniraṃtaraṃ    paśyati  ca pṛthak pṛthak bhavati \D1
%sakalāpratyakṣaniraṃtaraṃ    paśyati  ca/ pṛthak ca bhavati // \U1
%\om                                                     \U2
%---------------------
%He constantly sees everything in front of his eyes and he becomes separated (from the material world).
%---------------------
        \app{\lem[type=emendation, resp=egoscr]{sakalaṃ pratyakṣaṃ nirantaraṃ}
          \rdg[wit={N1,N2,D1,U1}]{\korr sakalāpratyakṣaṃ nirantaraṃ}
          \rdg[wit={B,L}]{sakalāḥ pratyakṣaṃ niraṃtara}
          \rdg[wit={E}]{sakalān pratyakṣaṃ niraṃtaraṃ}
          \rdg[wit={P,U2}]{\om}}
        \app{\lem[wit={E,N1,N2,D1,U1}]{paśyati}
          \rdg[wit={L,B}]{paśyatī}
          \rdg[wit={P,U2}]{\om}}
        \app{\lem[wit={E}]{pṛthagbhavati}
          \rdg[wit={B,L}]{ca pṛthak bhavati}
          \rdg[wit={N1,N2,U1}]{ca pṛthak ca bhavati}
          \rdg[wit={P,U2}]{\om}}/  
%---------------------
%atiśayenāyur vardhate /   \E
%atiśayenāyur vardhate     \P      
%atīśayanāyur vardhayate / \B
%atīśayanāyur vardhayate // \L
%atiśayena āyur varddhate // \N1
%atiśayena āyur varddhate // \N2     
%atiśayena āyur varddhate // \D1
%atiśayena āyur varddhate // \U1
%\om                         \U2
%---------------------
% The force of life increases eminently. 
%---------------------
        \app{\lem[alt={atiśayenāyur},wit={E,P}]{atiśayenāyu\skp{r-}}
          \rdg[wit={B,L}]{atīśayanāyur}
          \rdg[wit={N1,N2,D1,U1}]{atiśayena āyur}
          \rdg[wit={U2}]{\om}}\app{\lem[alt={vardhate},wit={E,P,N1,N2,D1,U2}]{\skm{r-}vardhate}
          \rdg[wit={B,L}]{vardhayate}}//        
    \end{prose}
\end{edition}
\begin{translation}
  \ekddiv{type=trans}
      \bigskip
    \centerline{\textrm{\small{[Description of the eigth Cakra]}}}
    \bigskip
  \begin{tlate}
    Now [there] exists the eighth \textit{cakra} having one hundred petals located at the aperture of Brahman. \extra{[Its] deity is the Guru, [its] power is consciousness (\textit{caitanya}), [its] Ṛṣi is Virāṭu, [attributed to it is] the witness being eminent in everything, [being] characterized by the soul that is beyond the fourth state of beings. [It has] all colours. [It has] all measures. [It has] all petals. [Its] state is being established in the body of Virāṭu. [Its] speech is wisdom. [It is attributed to] the "I am that"-[expression] (\textit{soham}), the Veda [in general] [and] the matchless place (\textit{anupamasthāna}). [It is associated with] the count of all whispered utterings [of Mantras]. [It is associated with the number] 21600. And in this way Niśāhevahate. The breath is a the pair of yojānātisapaṃḍitaḥ? With the sound of "sa" externally he goes, with the sound of "ha" viśotpunaḥ: "I am he, he is I". Because of that the embodied soul constantly utters the Mantra.} ``The (divine) seat of  Jālaṃdhara'' is the designation of the lotus of it. [It is] the place of the accomplished person. In its middle looking like a streak [and] having the form of smoke and fire, exists such a single [divine] form of the person (\textit{puruṣa}). Of her exists no end, nor a beginning. Due to the exercise of meditation on this [divine] form both coming and going of the person in space occurs. Affliction from the earth-element does not arise [anymore] even if one is situated in the middle of the earth. He constantly sees everything in front of his eyes and he becomes separated [from the material world]. The force of life increases eminently.    
     \end{tlate}
   \end{translation}
%\end{alignment}
\clearpage
%\begin{alignment}[
%    texts=edition[class="edition"];
%    translation[class="translation"],
%  ]
\begin{edition}
 \ekddiv{type=ed}
   \bigskip
    \centerline{\textrm{\small{[Description of the ninth Cakra]}}}
    \bigskip
 \begin{prose}
%---------------------
%idānīṃ navamacakrasya   bhedāḥ kathyante /  \E
%idānīṃ navamacakrasya   bhedāḥ kathyante /  \P
%idānīṃ navamacakrasya   bhedāḥ kathyate     \L
%idānīṃ navamaṃ cakrasya bhedāḥ kathyate //  \B
%idānīṃ navamacakrasya   bhedāḥ kathyaṃte // \N1
%idānīṃ navamacakrasya   bheda  kathyate  // \N2
%idānīṃ navamacakrasya   bhedāḥ kathyaṃte // \D1
%idānīṃ navamaś cakrasya bhedāḥ kathyaṃte    \U1   
%idānīṃ navamacakrasya   bhedaḥ kathyate /   \U2
%---------------------
%Now the divisions/differentiations of the ninth cakra are explained.
%---------------------
idānīṃ
\app{\lem[wit={E,P,L,N1,N2,D1,U2}]{navamacakrasya}
  \rdg[wit={B}]{navamaṃ cakrasya}
  \rdg[wit={U1}]{navamaś cakrasya}}
\app{\lem[wit={E,P,B,L,N1,D1,U1,U2}]{bhedāḥ}
  \rdg[wit={N2}]{bheda}}
\app{\lem[wit={E,P,N1,D1,U1}]{kathyante}
  \rdg[wit={L,B,N2,U2}]{kathyate}}/
%------------------------------
%tasya mahāśūnyacakram    iti  saṃjñā /  \E
%tasya mahāśūnyacakram    iti  saṃjñā /  \P
%tasya mahāśūnye cakram   iti  saṃjñā    \L
%tasye mahāśūnye cakram   iti  saṃjñā    \B
%tasya mahāśūnye cakreti       saṃjñā // \N1
%tasya mahāśūnyacakreti        saṃjñā // \N2
%tasya mahāśūnyacakreti        saṃjñā // \D1
%tasya mahāśūnyacakreti        saṃjñā /  \U1
%\om /                                   \U2
%---------------------
%The designation of it is ``the \textit{cakra} of the great void (\textit{mahāśūnyacakra})''.
%------------------------------
tasya \app{\lem[wit={E,P,N2,D1,U1}, alt={mahāśūnya°}]{mahāśūnya}
  \rdg[wit={L,B}]{mahāśūnye}
  \rdg[wit={N1}]{mahāśūnye}
  \rdg[wit={U2}]{\om}}\app{\lem[wit={N1,N2,D1,U1}]{cakreti}
  \rdg[wit={E,P}]{°cakram iti}
  \rdg[wit={L,B}]{cakram iti}
  \rdg[wit={U2}]{\om}}
\app{\lem[wit={E,P,L,B,N1,N2,D1,U1}]{saṃjñā}
  \rdg[wit={U2}]{\om}}/
%------------------------------
%tadupary aparaṃ kimapi nāsti / \E
%tadupary aparaṃ kimapi nāsti \P
%tadupary        kimapi nāsti \B ??-> auch mögliche Lesart
%tadupari        kimapi nāsti \L
%tadupari aparaṃ kiṃapi nāsti / \N1
%tadupari aparaṃ kiṃapi nāsti / \N2
%tadupari aparaṃ kiṃapi nāsti / \D1
%tadupari aparaṃ kiṃapi nāsti   \U1
% \om                           \U2
%---------------------
%kim api: somewhat, to a considerable extent, rather, much more, still, further. Śa
%---------------------
%Above that there is no other. 
%---------------------
\app{\lem[wit={E,P,B},alt={tadupary}]{tadupary\skm{a-}}
  \rdg[wit={L,N1,N2,D1,U1,U2}]{tadupari}
  \rdg[wit={U2}]{\om}}\app{\lem[wit={E,P,N1,N2,D1,U1}]{\skp{-a}paraṃ}
  \rdg[wit={B,L,U2}]{\om}}
\app{\lem[wit={E,P,L,B,N1,N2,D1,U1}]{kiṃ api nāsti}
  \rdg[wit={U2}]{\om}}/
%------------------------------
%tadeva-mahāsiddhacakraṃ kathyate // \E
%tadeva-mahāsiddhacakraṃ kathyate    \P 
%tadeva-mahāsiddhacakraṃ kathyate // \B
%tadeva-mahāsiddhacakraṃ kathyate // \L
%tadeva-mahāsiddhacakraṃ kathyate // \N1
%tadeva-mahāsiddhacakraṃ kathyate // \N2
%tadeva-mahāsiddhacakraṃ kathyate // \D1
%tadeva-mahāsiddhacakraṃ kathyate /  \U1
% \om                                \U2
%---------------------
%Therefore it is declared to be the \textit{cakra} of the great perfection (\textit{mahāsiddhacakra}).
%---------------------
tadeva mahāsiddhacakraṃ kathyate/
%------------------------------
%       tasya           pūrṇagiripīṭha               etadṛśaṃ nāma /  \E 
%       tasya           pūrṇagiripīṭham-iti          etādṛśaṃ nāma    \P
%       tasya           pūrṇagiripīṭham-iti saṃjñā   etādṛsaṃ nāma    \B ->!!! 
%       tasya           pūrṇagiripīṭham-iti saṃjñā   etādṛsaṃ nāma    \L
%       tasya cakrasya  pūrṇagiri                    etādṛśaṃ nāma /  \N1
%       tasya cakrasya  pūrṇagiri                    etādṛśaṃ nāma /  \N2
%       tasya cakrasya  pūrṇagiri                    etādṛśaṃ nāma /  \D1
%       tasya cakrasya  pūrṇagire                    etādṛśaṃ nāmaḥ   \U1
%madhye tasya           pūrṇagiripīṭham-iti          ekādaśaṃ nāma // \U2   
%-----------------------------
%Such a name of it is ``(divine) seat of Pūrṇagiri''.   
%------------------------------
\app{\lem[wit={E,P,B,L,N1,N2,D1,U1}]{tasya}
  \rdg[wit={N1,N2,D1,U1}]{tasya cakrasya}
  \rdg[wit={U2}]{madhye tasya}}
\app{\lem[wit={E,P,B,L,U2},alt={pūrṇagiri°}]{pūrṇagiri}
  \rdg[wit={N1,N2,D1}]{pūrṇagiri}
  \rdg[wit={U1}]{pūrṇagire}}\app{\lem[wit={P,B,L,U2}, alt={pīṭham°}]{pīṭham\skm{i-}}
  \rdg[wit={E}]{pīṭha}
  \rdg[wit={N1,N2,D1,U1}]{\om}}\app{\lem[wit={P,U2},alt={iti}]{\skp{-i}ti}
  \rdg[wit={B,L}]{iti saṃjñā}
  \rdg[wit={E,N1,N2,D1,U1}]{\om}}
\app{\lem[wit={P,B,L,N1,N2,D1,U1}]{etādṛśaṃ}
  \rdg[wit={E}]{etadṛśaṃ}
  \rdg[wit={U2}]{ekādaśaṃ}}
\app{\lem[wit={E,P,L,B,N1,N2,D1,U2}]{nāma}
  \rdg[wit={U1}]{nāmaḥ}}/
%------------------------------
%------------------------------
%tasya mahāśūnyacakrasya madhye ūrdhvamukham iti raktavarṇaṃ sakalaśobhāspadam    \E
%tasya mahāśūnyacakrasya madhye ūrdhvamukham iti raktavarṇa--sakalaśobhāspadaṃ     \P
%tasya mahāśūnyacakrasya madhye ūrdhvamukhem iti raktavarṇaṃ sakalaśobhāspadaṃ // \B    
%tasya mahāśūnyacakrasya madhye ūrdhvamukham iti raktavarṇaṃ sakalaśobhāspadaṃ // \L
%tasya mahāśūnyacakramadhye     ūrdhvamukhaṃ atiraktavarṇaṃ  sakalaśobhāspadaṃ /   \N1 ->!!!
%tasya mahāśūnyacakramadhye     ūrdhvamukhaṃ atiraktavarṇaṃ  sakalaśobhāspadaṃ     \N2
%tasya mahāśūnyacakramadhye     ūrdhvamukhaṃ atiraktavarṇaṃ  sakalaśobhāspadaṃ /   \D1
%tasya mahāśūnyacakramadhye     ūrdhvamukhaṃ atiraktavarṇaṃ  sakalaśobhāspadaṃ     \U1
%tasya mahāśūnyacakrasya        urdhvamukham-ativarṇaṃ       sakalaśobhanāsyadaṃ / \U2                                             
%------------------------------
%anekakalyāṇapūrṇaṃ sahasradalan      ekaṃ kamalaṃ  varttate / \E
%anekakalyāṇapūrṇaṃ sahasradalaṃ      ekaṃ kamalaṃ  vartate    \P
%anekakalyāṇapūrṇa--sahasradalaṃ      ekaṃ kamalaṃ  vartato    \B
%anekakalyāṇapūrṇaṃ sahasradalaṃ      ekaṃ kamalaṃ  vartate    \L
%anekakalyāṇapūrṇaṃ sahasradalaṃ      eka--kamalaṃ  varttate   \D1
%anekakalyāṇapūrṇaṃ sahasradalaṃ      ekaṃ kamalaṃ  vartate    \N1
%anekakalyāṇapūrṇa--sahasradalaṃ      ekaṃ kamalaṃ  varttate    \N2
%anekakalyāṇapūrṇaṃ sahasradalaṃ           kamalaṃ  vartate /   \U1
%anekakalyāṇapūrṇaṃ // sahasradalaṃ   ekaṃ kamalaṃ  vartate / \U2
%Fragezeichen in |nepal ... schreiber Einfügung? 
%------------------------------
%In the middle of the \textit{mahāśūnyacakra} exists one lotus facing upward, very red in color with a thousand petals - an abode of brilliance and wholeness.
%------------------------------
tasya
\app{\lem[wit={N1,N2,D1,U1}]{mahāśūnyacakramadhye}
  \rdg[wit={E,P,B,L}]{mahāśūnyacakrasya madhye}
  \rdg[wit={U2}]{mahāśūnyacakrasya}
}
\app{\lem[wit={N1,N2,D1,U1}]{ūrdhvamukhaṃ}
  \rdg[wit={E,P,L}]{ūrdhvamukham}
  \rdg[wit={U2}]{urdhvamukham}
  \rdg[wit={B}]{ūrdhvamukhem}
}
\app{\lem[wit={N1,N2,D1,U1}]{ atiraktavarṇaṃ}
  \rdg[wit={E,L,B}]{iti raktavarṇaṃ}
  \rdg[wit={P}]{iti raktavarṇa°}
  \rdg[wit={U2}]{ativarṇaṃ}
}
\app{\lem[wit={P,B,L,N1,N2,D1,U1}]{sakalaśobhāspadaṃ}
  \rdg[wit={E}]{sakalaśobhāspadam}
  \rdg[wit={U2}]{sakalaśobhanāsyadaṃ}
}
\app{\lem[wit={E,P,L,D1,N1,U1,U2}]{anekakalyāṇapūrṇaṃ}
  \rdg[wit={B,N2}]{°pūrṇa°}
}
sahasradalaṃ
\app{\lem[wit={E,P,L,B,N1,N2,U2}]{ekaṃ}
  \rdg[wit={D1}]{eka°}
  \rdg[wit={U1}]{\om}
}
kamalaṃ
\app{\lem[wit={E,P,L,N1,N2,D1,U1,U2}]{vartate}
  \rdg[wit={B}]{vartato}
}/
%---------------------
%yasya           parimalo manaso vacaso na gocaraḥ // \E
%yasya           parimalo manasā vacasā na gocaraḥ /  \P
%yasya           parimalo manasā vacasā    gocaraḥ /  \L
%yasya           parimalo manasā vacasā na gocaraḥ /  \B
%yasya           parimalo manasā vacasā na gocaraḥ /  \N1
%yasya           parimalo manasā vacasā na gocara /   \N2
%yasya           parimalo manasā vacasā na gocaraḥ /  \D1
%yasya           parimalo vacasā manasā na gocaraḥ    \U1
%yasya kamalasya parimalo manasā vācā   na gocara ..  \U2
%---------------------
%Whose fragrance is not in range by mind and speech. 
%Dessen Duft ist nicht in Reichweite von Geist und Sprache. 
%---------------------
\app{\lem[wit={E,B,N1,N2,D1,P,U1,U2}]{yasya}
  \rdg[wit={U2}]{yasya kamalasya}}
pariomalo
\app{\lem[wit={E}]{manaso vacaso}
  \rdg[wit={P,L,B,N1,N2,D1}]{manasā vacasā}
  \rdg[wit={U1}]{vacasā manasā}
  \rdg[wit={U2}]{manasā vācā}
}
\note[type=philcomm, labelb=s22.z4, lem={°manaso vacaso}]{All manuscripts at hand share this usage of the instrumentals. Only the printed edition conjectures the forms into the exspected genitiv. I adopted the variant of the printed edition to arrive at a grammatical text.}
\app{\lem[wit={E,P,B,N1,N2,D1,U1,U2}]{na}
  \rdg[wit={L}]{\om}
}
\app{\lem[wit={E,P,B,N1,D1,U1}]{gocaraḥ}
  \rdg[wit={N2,U2}]{gocara}}/
%---------------------
%tasya kamalasya madhye trikoṇarūpa-ikā karṇikā varttate /    \E
%tasya kamala----madhye trikoṇārūpā ekā karṇikā varttate/ \P
%tasya kamalasya madhye trikoṇarūpā ekā karṇikā varttate/     \L
%tasya kamalasya madhye trikoṇarūpā ekā karṇikā varttate/     \B
%tasya kamalasya madhye trikoṇarūpā eka karṇikā varttate/     \N1
%tasya kamalasya madhye trikoṇarūpā eka karṇikā varttate/     \N2
%tasya kamalasya madhye trikoṇarūpā ekā karṇikā varttate/     \D1
%tasya kamalasya madhye trikoṇarūpā ekā karṇikā vartate       \U1
%tasya kamalasya madhye trikoṇarūpā ekā karṇikā vartate //    \U2
%---------------------
%In the middle of this lotus exists one pericarp having the shape of a triangle. 
%------------------------------
tasya
\app{\lem[wit={E,L,B,N1,N2,D1,U1,U2}]{kamalasya}
  \rdg[wit={P}]{kamala°}}
madhye
\app{\lem[wit={E}]{trikoṇarūpaikā}
  \rdg[wit={P,L,B,D1,U1,U2}]{trikoṇārūpā ekā}
  \rdg[wit={N1,N2}]{trikoṇārūpā eka}
}
karṇikā vartate//
%------------------------------
%tatkarṇikāmadhye saptadaśī         niraṃjanarūpā kalā varttate/ \E
%tatkarṇikāmadhye saptadaśireṇa ekā niraṃjanarūpā kalā vartate// \L
%tatkarṇikāmadhye saptadaśireṇa ekā niraṃjanarūpā kalā vartate// \B
%tatkarṇikāmadhye saptadaśī     ekā niraṃjanarūpā kalā vartate// \P
%tatkarṇikāmadhye saptadaśī     ekā niraṃjanarūpā kalā vartate// \N1
%tatkarṇikāmadhye saptadaśī     ekā niraṃjanarūpā kalā vartate/  \N2
%tatkarṇikāmadhye saptadaśī     ekā niraṃjanarūpā kalā vartate// \D1
%tatkarṇikāmadhye saptadaśī     ekā niraṃjanarūpā kalā vartate  \U1
%tatkarṇikāmadhye saptadaśī     eka niraṃjanarūpā kalā varttate/ \U2
%---------------------
%In the middle of the pericarp exists one seventeenth digit in the shape of a immaculé form.
%---------------------
tatkarṇikāmadhye
\app{\lem[wit={E,P,N1,N2,D1,U1,U2}]{saptadaśī}
  \rdg[wit={L,B}]{saptadaśireṇa}}\note[type=philcomm, labelb=s22.z4, lem={saptadaśī}]{A \textit{saptadaśī kalā} appears frequently in Śaiva literature. References need to be added here.}
\app{\lem[wit={P,L,B,N1,N2,D1,U1,U2}]{ekā}
  \rdg[wit={E}]{\om}}
nirañjanarūpā kalā varttate/
%---------------------
%koṭisūryasamaprabhaṃ kalāyās tejo vartate /    \E
%koṭisūryasamaprabhā kalāyās tejo vartate /     \L
%koṭisūryasamaprabhā kalāyās tejo vartate /     \B
%koṭisūryasamaprabha kalāyās tejo vartate /     \P
%koṭisūryasamaprabhaṃ kalāyās tejo vartate /    \N1
%koṭisūryasamaprabhaṃ kalāyā  tejo varttate //  \N2
%koṭisūryasamaprabhaṃ kalāyās tejo vartate /    \D1
%koṭisūryasadṛṣaprabhaṃ kalāyās tejo vartate /  \U1
%koṭisūryasamaprabhā // kalāyās tejo varttate / \U2
%---------------------
%A light of the part exists shining like a thousand suns. 
%------------------------------
koṭisūrya\app{\lem[alt={°samaprabhaṃ}, wit={E,N1,N2,D1}]{samaprabhaṃ}
  \rdg[wit={L,B,U2}]{samaprabhā}
  \rdg[wit={P}]{samaprabha}
  \rdg[wit={U1}]{sadṛṣaprabhaṃ}}
kalāyās-tejo vartate/
%------------------------------
%param udbhavo nāsti /     \E
%parim uṣṇabhavo nāsti /   \P
%parim uṣṇabhavo nāsti /   \L
%parim uṣṇabhavo nāsti /   \B
%parim uṣṇabhāvo nāsti /   \N1
%para  uṣṇabhāvo nāsti     \N2
%parim auṣṇabhāvo nāsti /  \D1
%paraṃ uṣṇabhāvo nāsti     \U1
%param uṣṇabhāvo nāsti /   \U2
%---------------------
%[But] excessive heat is not arising. 
%------------------------------
\app{\lem[alt={param},wit={E,U1,U2}]{param\skm{-u}}
  \rdg[wit={U1}]{paraṃ}
  \rdg[wit={N2}]{para}
  \rdg[wit={P,L,B,D1}]{parim}
}\app{\lem[wit={N1,N2,U1,U2}, alt={uṣṇabhāvo}]{\skp{-u}ṣṇabhāvo}
  \rdg[wit={P,L,B}]{uṣṇabhavo}
  \rdg[wit={D1}]{auṣṇabhāvo}
  \rdg[wit={E}]{udbhavo}
}
nāsti/
%------------------------------
%koṭicandrasamaprabhā    śītalaṃ paraṃ   śītabhāvo   nāsti / \E
%koṭicandrasamaprabhā    śītalaṃ paraṃ   śītabhavo   nāsti / \P
%\om /                                                      \L
%koṭicandrasamaprabhā    śītalaṃ paraṃ   śītabhavo   nāsti / \B
%koṭicandrasamaprabhaṃ   śītalaparaṃ         bhavo   nāsti / \N1
%koṭicandrasamaprabhaṃ   śītalapara----------bhavo   nāsti // \N2
%koṭicaṃdrasamaprabhaṃ   śītalaparaṃ         bhavo   nāsti / \D1
%koṭicaṃdrasamaṃ prabhaṃ śītalaṃ paraṃ       bhavo   nāsti / \U1
%koṭicaṃdrasamaprabhā    śītalaṃ paraṃ śītalabhāvo   nāsti / \U2
%---------------------
%Shining like a thousand moons, excess of cold is not arising.
%---------------------
koṭicandra\app{\lem[alt={°samaprabhaṃ},wit={N1,N2,D1}]{samaprabhaṃ}
  \rdg[wit={E,P,B,U2}]{°samaprabhā}
  \rdg[wit={U1}]{°samaṃ prabhaṃ}
  \rdg[wit={L}]{\om}
}
\app{\lem[wit={N1,D1}]{śītalaparaṃ}
  \rdg[wit={E,P,B,U1,U2}]{śītalaṃ paraṃ}
  \rdg[wit={N2}]{śītalapara}
  \rdg[wit={L}]{\om}
}
\app{\lem[wit={N1,N2,D1,U1}]{bhāvo} 
  \rdg[wit={E,P,B}]{śītabhāvo}
  \rdg[wit={U2}]{śītalabhāvo}
  \rdg[wit={L}]{\om}
}
nāsti/
%------------------------------
%asyāḥ kalāyā   dhyānayogāt    sādhakasya manasi duḥkhaṃ na bhavati / \E
%asyāḥ kalādhyānayogāt         sādhakasya manasi duḥkhaṃ na bhavati / \P

%asyāḥ kalāyāḥ  dhyānakaraṇāt  sādhakasya manasi duḥkhaṃ na bhavati / N1
%asyā kalāyā    dhyānakaraṇāt  sādhaka----manasi duḥkhaṃ na bhavati / N2
%asyāḥ kalāyāḥ  dhyānakaraṇāt  sādhakasya manasi duḥkhaṃ na bhavati / D1
%
%asyāḥ kalāyā   dhyānayogāt    sādhakasya manasi duḥkhaṃ bhavati /B
%asyāḥ kalāyā   dhyānayogāt    sādhakasya manasi duḥkhaṃ bhavati /L
%asyāḥ kalāyā   dhyānakaraṇāt/ sādhakasya manasi duḥkhaṃ na bhavati / U1
%asyā  kalāyāḥ  dhyānayogāt//  sādhakasya manasi duḥkhaṃ na bhavati // \U2
%atrastāne 'haṃ devatā// sohaṃ śaktiḥ// ātmāṛṣiḥ// mokṣamārhaḥ// haṃbhrahmordhaṃ// haṃcakra iti// agnicakre sakaro bhavatī// prāṇīrūḍho bhave jjīva ārohaty avarohati bhavaguhāsthānaṃ pitavarṇaṃ// koṭisūryapratikāśaṃ tejaḥ sadoditaprabhā śīvodevatā// mūlamāyāśaktiḥ// hara ātmālayāvsthā dhvanisthirānādātmako khaṃḍa 'dhvani// adhorāmudrā// mūlamāyā// prakṛtidehaḥ// vāṅmanogocaraḥ// niḥprapaṃcaḥ// niḥsaṃśayaḥ// nistaraṃhanir lopalakṣaṃ laya// dhyānasamādhi 
%---------------------
%asyāḥ kalāyā dhyānakaraṇāt\varc{\emend kalāyāḥ dhyānakaraṇāt \nepal \dehlia}{kalāyā dhyānayogāt \nepal \dehlia kalādhyānayogāt \pune} sādhakasya manasi duḥkhaṃ na\varc{na \edprint \pune \nepal \dehlia}{\om \oxford \lalchand} bhavati /
%Due to the exercise of meditation upon the digit suffering does not arise in the mind of the practitioner (anymore). 
%------------------------------
\app{
  \lem[wit={E,P,N1,D1,B,L,U1}]{asyāḥ}
  \rdg[wit={N2,U2}]{asyā}}
\app{
  \lem[wit={N2,U1}]{kalāyā dhyānakaraṇāt}
  \rdg[wit={N1,D1}]{kalāyāḥ dhyānakaraṇāt}
  \rdg[wit={E,B,L}]{kalāyā dhyānayogāt}
  \rdg[wit={U2}]{kalāyāḥ dhyānayogāt}
  \rdg[wit={P}]{kalādhyānayogāt}}
\app{\lem[wit={E,P,B,L,N1,D1,U1,U2}]{sādhakasya}
  \rdg[wit={N2}]{sādhaka°}}
duḥkhaṃ
\app{\lem[wit={E,P,N1,N2,D1,U1,U2}]{na}
  \rdg[wit={B,L}]{\om}}
bhavati/
 \end{prose}
\end{edition}
\begin{translation}
  \ekddiv{type=trans}
   \bigskip
    \centerline{\textrm{\small{[Description of the ninth Cakra]}}}
    \bigskip
  \begin{tlate}
    Now the divisions/differentiations of the ninth cakra are explained. The designation of it is ``the \textit{cakra} of the great void'' (\textit{mahāśūnyacakra}). Above that there is no other. Therefore it is declared to be the \textit{cakra} of the great perfection (\textit{mahāsiddhacakra}). In the middle of the \textit{mahāśūnyacakra} exists one lotus facing upward, very red in color with a thousand petals - an abode of brilliance and wholeness, whose fragrance is not in range of mind and speech. In the middle of this lotus exists one pericarp having the shape of a triangle. In the middle of the pericarp exists one seventeenth digit in the shape of a immaculé form. A light of the part exists shining like a thousand suns. [But] excessive heat is not arising. Shining like a thousand moons, excess of cold is not arising.
  \end{tlate}
   \end{translation}
  \clearpage
 \begin{edition}
 \ekddiv{type=ed}
 \begin{prose} 
   \extra{atra
   \app{\lem[type=emendation, resp=egoscr]{sthāne}
    \rdg[wit={U2}]{\korr stāne}} 'haṃ devatā/
   \app{\lem[type=emendation, resp=egoscr]{so 'haṃ}
    \rdg[wit={U2}]{\korr sohaṃ}} śaktiḥ/ ātmāṛṣiḥ/ mokṣamārgaḥ/
   \app{\lem[type=emendation, resp=egoscr]{ahaṃ brahmordhvaṃ}
    \rdg[wit={U2}]{\korr haṃ brahmordhaṃ}}/
   \app{\lem[type=emendation, resp=egoscr]{ahaṃ cakra iti}
    \rdg[wit={U2}]{\korr haṃcakra iti}}/ agnicakre
   \app{\lem[type=emendation, resp=egoscr]{sakarā}
    \rdg[wit={U2}]{\korr sakaro}} bhavatī/ prāṇī rūḍho bhavej-jīva ārohaty-avarohati bhavaguhāsthānaṃ pitavarṇaṃ/ koṭisūryapratikāśaṃ tejaḥ sadoditaprabhā śīvo devatā/ mūlamāyāśaktiḥ/ hara ātmālayāvsthā dhvanisthirānādātmako khaṃḍadhvani/ aghorāmudrā/ mūlamāyā/ prakṛtidehaḥ/ vāṅmanogocaraḥ/ niḥprapañcaḥ/ niḥsaṃśayaḥ/ nistaraṃ hanir-lopalakṣaṃ laya/ dhyānasamādhi/}
%---------------------
%tadupari anaṃtaparamānandasya sthānam / \E
%tadupari anaṃtaparamānandasya sthānaṃ   \P
%tadupari anantaparamānaṃdasya sthānam / \N1
%tadupari anantaparamānaṃdasya sthānam / \N2
%tadupari anantaparamānaṃdasya sthānaṃ / \D1
%tadupari anantaparamānaṃdasya sthānam vartate/ \B
%tadupari anaṃtaparamānaṃdasya sthānam vartate/ \L
%tadupari alakṣaparamānaṃdasya sthānam   \U1
%tadupari anaṃtaparamānaṃdasya sthānaṃ// U2
%---------------------
%Above that is the place of infinite supreme bliss.
%---------------------
tadupari
\app{\lem[wit={E,P,B,L,N1,N2,D1,U2}, alt={ananta°}]{ananta}
  \rdg[wit={U1}]{alakṣa}}paramānaṃdasya
\app{\lem[wit={E,P,N1,N2,D1,U1,U2}]{sthānam}
  \rdg[wit={D1,U2}]{stānaṃ}
  \rdg[wit={B,L}]{sthānam vartate}}/
%---------------------
%tatrordhvaśaktiḥ / \E
%tatordhvaśaktiḥ \P
%rdhaśakti ardhaśakti \B
%rdhaśakti ardhaśakti \L
%tatrordhvaśaktiḥ / \N1
%tatra ūrdhva śaktiḥ / \D1
%tatra ūrdhva śakti / \N2
%urdhvaśaktir         \U1
%tatrordhvaśaktiḥ// \U2
%---------------------
%There above is \textit{śakti},
%------------------------------
\app{\lem[wit={E,N1,U2}]{tatrordhvaśaktiḥ}
  \rdg[wit={P}]{tatordhvaśaktiḥ}
  \rdg[wit={U1}]{urdhvaśaktir}
  \rdg[wit={D1}]{tatra ūrdhva śaktiḥ}
  \rdg[wit={N2}]{tatra ūrdhva śakti}
  \rdg[wit={B,L}]{rdhaśakti ardhaśakti}}/
%------------------------------
%etādṛśī  saṃjñā   ekā kalā vartate / \E
%ekādaśā  saṃjñā   ekā kalā vartate   \P
%etādṛśī  saṃjñā   ekā kalā vartate /  \N1
%etādṛśī  saṃjñā   ekā kalā varttate / \N2
%etādṛsaṃ saṃjñā   ekā kalā vartate / \D1
%ekādaśā  saṃjñā   ekā kalā vartate / \B
%ekādaśā  saṃjñā   ekā kalā vartate / \L
%etādṛśī  saṃjñakā ekā kalā vartate /  \U1
%etādṛśā  saṃjñā   ekā kalā vartate/ \U2 
%---------------------
%Being designated as such she is one single digit. 
%------------------------------
\app{\lem[wit={E,N1,N2,U1}]{etādṛśī}
  \rdg[wit={U2}]{etādṛśā}
  \rdg[wit={D1}]{etādṛsaṃ}
  \rdg[wit={P,B,L}]{ekādaśā}}
\app{\lem[wit={E,P,B,L,N1,N2,D1,U2}]{saṃjñā}
  \rdg[wit={U1}]{saṃjñakā}}
ekā kalā vartate/ 
%------------------------------
%asyāḥ  kalāyā   dhyānakāraṇāt     puruṣo yadicchati / \E
%asyāḥ  kalāyā   dhyānakāraṇāt     puruṣo yadicchati ?Zeichen? \P
%asyāḥ  kalāyā   dhyānakāraṇāt     puruṣo yadicchati  tad bhavati \N1
%tasyāḥ kalāyāḥ  dhyānakāraṇāt     puruṣo yadicchati  tad bhavati \N2
%asyāḥ  kalāyā   dhyānakāraṇā      puruṣo yadicchati  tad bhavati \D1
%asyāḥ  kalāyā   dhyānakāraṇāt /   puruṣo yadicchati / \B
%asyāḥ  kalāyā   dhyānakāraṇāt /   puruṣo yadicchati / \L
%asyā   kalāyā   dhyānakāraṇāt     puruṣo yadicchati tad bhavati vā \U1
%asyāḥ  kalāyāḥ  dhyānakāraṇāt //  puruṣo yadicchati // \U2
%---------------------
%Due to the exercise of meditation on this part the person manifests whatever he wishes for.
%------------------------------
\app{\lem[wit={ceteri}]{asyāḥ}
  \rdg[wit={U1}]{asyā}
\rdg[wit={N2}]{tasyāḥ}}
\app{\lem[wit={ceteri}]{kalāyā}
  \rdg[wit={N2,U2}]{kalāyāḥ}}
\app{\lem[wit={ceteri}]{dhyānakāraṇāt}
  \rdg[wit={D1}]{dhyānakāraṇā}}
puruṣo yad-icchati
\app{\lem[wit={N1,N2,D1}, alt={tad bhavati}]{tad-bhavati}
  \rdg[wit={U1}]{tad bhavati vā}
  \rdg[wit={ceteri}]{\om}}/ 
%------------------------------
%tasya sukhabhogavataḥ / \E
%tasya sukhabhogavataḥ \P
%rājya-sukhabhogavataḥ \N1
%rājya-sukhabhogavataḥ \N2
%rājya-sukhabhogavṛtaḥ \D1 !!!
%tasya-khaṃ bhogavataṃ / \B
%tasya-sukhaṃ bhogavaṃtaṃ / \L
%rājya-sukhabhogavataḥ \U1
%tasya-sukhabhogavataḥ / \U2
%---------------------
%He is furnished with royal pleasure and enjoyment. 
%------------------------------
\app{\lem[wit={D1}]{rājyasukhabhogavṛtaḥ}
  \rdg[wit={N1,N2,U1}]{rājyasukhabhogavataḥ}
  \rdg[wit={E,P,U2}]{tasya sukhabhogavataḥ}
  \rdg[wit={B}]{tasya-khaṃ bhogavataṃ}
  \rdg[wit={L}]{tasya-sukhaṃ bhogavaṃtaṃ}}/
%------------------------------
%strīmadhye     vilāsavataḥ    saṃgītavilāsavataḥ vinodaprekṣāvataḥ        puruṣasya pratidinaṃ śuklapakṣe candrakalāvat   kalā     vardhate/   \E
%strīmadhye     vilāsavataḥ    saṃgītavinodaprekṣāvataḥ              eva   puruṣasya pratidinaṃ śuklapakṣe candrakalāvat   kalā     vardhate /  \P
%strīmadhye     vilāsavaṃtaṃ   saṃgītaṃ prekṣāvatāḥ //               evaṃ  puruṣasya pratidinaṃ śuklapakṣe caṃdrakalāvat / kalā     vartate /   \L
%strīmadhye     vilāsavaṃtaṃ   saṃgītaṃ vinodavaṃtaṃ prekṣāvaṃtāḥ // eva   puruṣasya pratidinaṃ śuklapakṣe caṃdrakalāvat / kalā     vartate /   \B
%strīmadhye     vilāsavataḥ    saṃgītavinodaprekṣyāvataḥ             evaṃ  puruṣasya pratidinaṃ śuklapakṣe candrakalā vṛddhivato?   vardhate / \N1
%śrī strīmadhye vilāsavataḥ    saṃgītavinodaprekṣāvataḥ              evaṃ  puruṣasya pratidinaṃ śuklapakṣa candrakalā vṛddhi vaṃto  varttate /  \N2
%strīmadhye     vilāsavataḥ // saṃgītavinodaprekṣyāvataḥ //          evaṃ  puruṣasya pratidinaṃ śuklapakṣe candrakalā vṛddhivato    vardhate / \D1
%strīmadhye     vilāśavataḥ    saṃgītavinodaprekṣyāvataḥ             eka   puruṣasya pratidinaṃ śuklapakṣe caṃdrakalā vṛddhir       varddhate / \U1
%strīmadhye     vilāsavata     saṃgītavinodaprekṣāvata//             evaṃ  puruṣasya pratidinaṃ śuklapakṣe candrakalāvat   kalā     varttate/   \U2
%---------------------
%(Selbst) bei einem Menschen, der sich inmitten von Frauen vergnügt, (und) ein Musikvergnügen
%ansieht, wächst täglich die Kraft (kalā = śakti?) wie die "kalā" (Phase) des Mondes in der hellen Monatshälfte.
%The \textit{kalā} of a person grows daily, like the \textit{kalā} of the moon in the bright half of the month, even amusing oneself amongst women and watching a musical pleasure.
%(Even) amusing oneself amongst women, and watching musical pleasures, the \textit{kāla} of the person grows daily like the \textit{kalā} of the moon in the bright half of the month. 
%------------------------------
\app{\lem[wit={ceteri}]{strīmadhye}
  \rdg[wit={N2}]{śrī strīmadhye}}
\app{\lem[wit={ceteri}]{vilāsavataḥ}
  \rdg[wit={U2}]{vilāsavata°}
  \rdg[wit={L,B}]{vilāsavaṃtaṃ}} 
\app{\lem[wit={N1,D1,U1}]{saṃgītavinodaprekṣyāvataḥ}
  \rdg[wit={P,N2}]{saṃgītavinodaprekṣāvataḥ}
  \rdg[wit={U2}]{saṃgītavinodaprekṣāvata}
  \rdg[wit={B}]{saṃgītaṃ vinodavaṃtaṃ prekṣāvaṃtāḥ}
  \rdg[wit={E}]{saṃgītavilāsavataḥ vinodaprekṣāvataḥ}
  \rdg[wit={L}]{saṃgītaṃ prekṣāvatāḥ}}
 \app{\lem[wit={P,B}]{eva}
  \rdg[wit={L,N1,N2,D1,U2}]{evaṃ}
  \rdg[wit={U1}]{eka}}
puruṣasya pratidinaṃ śuklapakṣe
candrakalā\app{\lem[wit={E,P,L,B,U2},alt={°vat kalā}]{vat kalā}
  \rdg[wit={N1,D1}]{vṛddhivato}
  \rdg[wit={N2}]{vṛddhi vaṃto}
  \rdg[wit={U1}]{vṛddhir}}
\app{\lem[wit={E,P,N1,D1,U1}]{vardhate}
  \rdg[wit={ceteri}]{vartate}}/  
%------------------------------
%puṇyapāpe  'sya śarīraṃ   na spṛśataḥ /    \E
%\om                                     \P
%puṇyapāpe  asya śarīrena     spṛśataḥ /      \N1
%puṇyapāpe  asya śarīrena     spṛśataḥ /      \N2
%puṇyapāpe  asya śarīrena     spṛśataḥ /      \D1
%puṇyapāpe  asya śarīrasya na spṛśataḥ // \B
%puṇyapāpe  asya śarīrasya na spṛśataḥ // \L
%puṇyapāpau asya śarīrena     spṛśāt         \U1
%puṇyapāpe  asya śarīraṃ   na spṛśataḥ // \U2
%---------------------
%puṇyapāpe\varc{puṇyapāpe \edprint \lalchand \oxford \nepal \dehlia}{\om \pune} 'sya\varc{'sya \edprint}{asya \nepal \dehlia \oxford \lalchand \om \pune} śarīrasya\varc{śarīrasya \lalchand \oxford}{śarīraṃ \edprint śarīrena \nepal \dehlia \om \pune} na\varc{na \edprint \oxford \lalchand}{\om \nepal \dehlia \pune} spṛśataḥ\varc{spṛśataḥ \edprint \lalchand \oxford \nepal \dehlia}{\om \pune} /
%---------------------
%His body is not affected by merit and sin. 
%------------------------------
\app{\lem[wit={ceteri}]{puṇyapāpe}
  \rdg[wit={U1}]{puṇyapāpau}
\rdg[wit={P}]{\om}}
\app{\lem[wit={E}]{'sya}
  \rdg[wit={P}]{\om}
  \rdg[wit={ceteri}]{asya}}  
\app{\lem[wit={B,L}]{śarīrasya}
  \rdg[wit={N1,N2,D1,U1}]{śarīrena}
  \rdg[wit={E,U2}]{śarīraṃ}
  \rdg[wit={P}]{\om}}
\app{\lem[wit={E,B,L,U2}]{na}
  \rdg[wit={N1,N2,D1,U1,P}]{\om}}
\app{\lem[wit={ceteri}]{spṛśataḥ}
  \rdg[wit={U1}]{spṛśāt}}/
%------------------------------
%                          nirantaradhyānakaraṇāt     nijasvarūpaṃ prakāśanasāmarthyaṃ bhavati / \E
%                          \om until .....            nijasvarūpaprakāśasāmarthyaṃ     bhavati / \P
%                          niraṃtaraṃ dhyānakaraṇāt   nijasvarūpaprakāśasāmarthyaṃ     bhavati / \B
%                          niraṃtaraṃ dhyānakaraṇāt// nijasvarūpaprakāśasāmarthyaṃ     bhavati / \L
%                          nirantaradhyānakaraṇāt /   nijasvarūpaprakāśasāmarthyaṃ     bhavati / \N1 <-----
%                          niraṃtaradhyānakaraṇāt /   nijasvarūpaprakāśasāmarthyaṃ     bhavati // \N2
%                          nirantaradhyānakaraṇāt /   nijasvarūpaprakāśasāmarthyaṃ     bhavati / \D1
%                          nirantaradhyānakaraṇāt /   nijasvarūpaprakāśasāmarthyaṃ     bhavati    \U1
%evaṃ puruṣasya pratidinaṃ niraṃtaraṃ dhyānakaraṇāt   nijasvarūpaṃ prakāśanasāmarthyaṃ bhavati// \U2 
%---------------------
%Due to uninterrupted meditation the power of the light of the innate nature arises. 
%------------------------------
\app{\lem[wit={ceteri}]{nirantaradhyānakaraṇāt}
  \rdg[wit={B,L}]{niraṃtaraṃ dhyānakaraṇāt}
  \rdg[wit={U2}]{evaṃ puruṣasya pratidinaṃ niraṃtaraṃ dhyānakaraṇāt}
  \rdg[wit={P}]{\om}}
\app{\lem[wit={ceteri}]{nijasvarūpaprakāśasāmarthyaṃ}
  \rdg[wit={E,U2}]{nijasvarūpaṃ prakāśanasāmarthyaṃ}}
bhavati/
%------------------------------
%dūrasthopi ca dūrasthavastu                   samīpa iva   paśyati // \E
%dūrasthamapi                                  samīpam iva  paśyati // \N1
%dūrasthamapi                                  samīpaṃ iva  paśyati // \N2
%dūrasthamapy-arthaṃ                           samīpa iva   paśyati // \D1
%dūrasthamapi padārthaṃ                        samīpa iva   paśyati // \B
%dūrasthamapi parārthaṃ                        samīpa iva   paśyati // \L
%dūrasthamapi padārthaṃ                        samīpa iva   paśyati // \P
%dūrasthamapy-arthaṃ                           samīpameva   paśyati // \U1
%dūrasthamapi bhavati //dūrasthamapi padārthaṃ samīpa iva   paśyati// \U2
%------------------------------
%dūrasthamapyarthaṃ\varc{dūrasthamapyarthaṃ \dehlia}{dūrasthamapi padārthaṃ \oxford \pune durasthamapi parārthaṃ \lalchand sūrastamapi \nepal ca dūrasthavastu \edprint} samīpa\varc{samīpa \dehlia \edprint \lalchand \oxford \pune}{samīpam \nepal} iva paśyati //
%------------------------------
%He sees remotely located objects as if they'd be near.
%------------------------------
\app{\lem[wit={D1,U1},alt={dūrasthamapy arthaṃ}]{dūrasthamapy-arthaṃ}
  \rdg[wit={B,P}]{dūrasthamapi padārthaṃ}
  \rdg[wit={L}]{dūrasthamapi parārthaṃ}
  \rdg[wit={E}]{dūrasthopi ca dūrasthavastu}
  \rdg[wit={N1,N2}]{dūrasthamapi}
  \rdg[wit={U2}]{dūrasthamapi bhavati// dūrasthamapi padārthaṃ}}
\app{\lem[wit={ceteri}]{samīpa iva}
  \rdg[wit={N1}]{samīpam iva}
  \rdg[wit={N2}]{samīpaṃ iva}
  \rdg[wit={U1}]{samīpameva}}
paśyati//
%------------------------------
\end{prose}
\end{edition}
\begin{translation}
  \ekddiv{type=trans}
  \begin{tlate}
    \extra{Here at this location the "I"(\textit{aham}) is the deity. The "I am that" (\textit{so 'ham}) is [its] power. This self is [its] seer. [Its] path is liberation, being the "I am Brahman" after death, the "I am the [Brahman]-wheel?". [Here] in the wheel of fire is the being (\textit{bhavatī}) full of rays. [From here] the living being arises. [Here] the embodied soul approaches [and] reaches the hidden place of existence [which is] yellow-colored. [Here] is the light of a million suns. [Here] is Śiva, the deity always shining from above. [Here] is the power of the original Illusion (\textit{māyā}). [Here] is the state of the dissolution of the self [which is] Hara. [Attributed to this] is a [certain] type of sound, having the nature of a stable resonance being the transcendental sound. [It is attributed to the] the fearless seal (\textit{aghoramudrā}). [Attributed to it] is the original illusion. [Its] body is the primal material matter. [Its] range is speech and mind.? [Attributed to it] is non-diversity [and] no doubts. [Its] dissolution is the weapon cutting the aims to final liberation?. [Attributed to it] is Meditation and Absorption.} Above that is the place of infinite supreme bliss. There above is power (\textit{śakti}). Being designated as such she is one single digit. Due to the exercise of meditation on this part the person manifests whatever he wishes for. He is furnished with royal pleasure and enjoyment. [Even] amusing oneself amongst women, and watching musical pleasures, the \textit{kāla} of the person grows daily like the \textit{kalā} of the moon in the bright half of the month. His body is not affected by merit and sin. Due to uninterrupted meditation the power of the light of the innate nature arises. He sees remotely located objects as if they'd be near.
\end{tlate}
\end{translation}
\clearpage
\begin{edition}
 \ekddiv{type=ed}
   \bigskip
    \centerline{\textrm{\small{[Lakṣyayoga, the yoga of fixation]}}}
    \bigskip
    \begin{prose}
%------------------------------
%idānīṃ sukhasādhyo lakṣyayogaḥ kathyate / \E
%idānīṃ sukhasādho  lakṣyayogaḥ kathyate / \P
%idānīṃ sukhasādho  lakṣayogaḥ  kathyate / \B
%idānīṃ sukhasādhe  lakṣayogaḥ  kathyate // \L
%idānīṃ sukhasādhyo lakṣyayogaḥ kathyate / \N1
%idānīṃ sukhasādhya lakṣanayogaḥ kathyate / \N2
%idānīṃ sukhasādhyo lakṣyayogaḥ kathyate / \D1
%idānīṃ sukhasādhyopalakṣayogaḥ kathyate / \U1
%idānīṃ sukhasādhyo lakṣyayogaḥ kathyate / \U2
%------------------------------
%Now the yoga of fixation{\textit{lakṣyayoga}}, which is easily accomplished is explained. 
%------------------------------
      idānīṃ
      \app{\lem[wit={E,N1,D1,U2}]{sukhasādhyo}
        \rdg[wit={N2}]{°sādhya}
        \rdg[wit={P,B}]{°sādho}
        \rdg[wit={L}]{°sādhe}
        \rdg[wit={U1}]{°sādhyopa°}}
    \app{\lem[wit={E,P,N1,D1,U2}]{lakṣyayogaḥ}
        \rdg[wit={B,L}]{lakṣayogaḥ}
        \rdg[wit={U1}]{°lakṣayogaḥ}
        \rdg[wit={N2}]{lakṣanayogaḥ}}
      kathyate/
%------------------------------      
%asya lakṣyayogasya  paṃcabhedā     bhavanti   ūrdhvalakṣyam / adholakṣyam / lakṣyam /      bāhyalakṣyam /  aṃtaralakṣyam /  \E
%asya lakṣyayogasya  paṃcabhedā     bhavanti   ūrdhvalakṣyam   adholakṣyam / madhyalakṣyam  bāhyalakṣyam    aṃtaralakṣyam /  \P
%asya lakṣayogasya   paṃce bhedāḥ   bhavaṃtī   ūrdhvalakṣam//  adholakṣam// bāhyakṣam//                     aṃtaralakṣam //  \B
%asya lakṣayogasya   paṃcabhedāḥ    bhavaṃti   ūrdhvalakṣam    adholakṣam// madhyalakṣam//  bāhyakṣam//     aṃtaralakṣam //  \L
%     lakṣyayogasya  paṃcabhedā     bhavaṃti// urdhvalakṣya    adholakṣya   bāhyalakṣya     madhyalakṣya    antaralakṣya //  \N1
%     lakṣanayogasya paṃcabhedā     bhavati//  urdhvalakṣa     adholakṣa    bāhyalakṣa      madhyalakṣa     antaralakṣa //   \N2
%     lakṣyayogasya  paṃcabhedā     bhavaṃti// urdhvalakṣya    adholakṣya   bāhyalakṣya     madhyalakṣya    antaralakṣya //  \D1
%a----lakṣayogasya   paṃcabhedā     bhavati    urdhvalakṣa                  bāhyalakya      madhyalakṣa     aṃtaralakṣya     \U1
%asya lakṣayogasya   paṃcabhedā     bhavaṃti// ūrdhvalakṣam//  adholakṣam/  bāhyalakṣyam /  madhyalakṣaṃ/   sarvalakṣyam /   \U2
%------------------------------
%Of this yoga of fixation (\textit{lakṣyayoga}) there are five subdivisions: 1. The upward directed fixation {\textit{ūrdhvalakṣya}), 2. the downward directed fixation (\textit{adholakṣya}),3. the central fixation (\textit{madhyalakṣya}) 4. the outer fixation (\textit{baḥyalakṣya}), 5. the inner fixation (\textit{antaralakṣya}).
%------------------------------
      \app{\lem[wit={E,P,B,L,U2}]{asya}
        \rdg[wit={N1,N2,D1,U1}]{\om}}
      \app{\lem[wit={E,P,N1,D1}]{lakṣyayogasya}
        \rdg[wit={B,L,U2}]{lakṣayogasya}
        \rdg[wit={U1}]{alakṣayogasya}
        \rdg[wit={N2}]{lakṣanayogasya}}
      \note[type=philcomm, labelb=s28.z2, lem={lakṣyayogasya}]{The designation of this type of yoga is transmitted in various variants. Given the list of the 15 yogas at the beginning of the text it is very likely that the correct name of the yoga is \textit{lakṣyayoga} and not \textit{lakṣayoga} or \textit{lakṣanayoga}.}
      \app{\lem[wit={E,P,N1,N2,D1,U1,U2}]{pañcabhedā}
        \rdg[wit={L}]{paṃcabhedāḥ}
        \rdg[wit={B}]{paṃce bhedāḥ}}
     \app{\lem[wit={E,P,B,L,N1,D1,U2}]{bhavanti}
       \rdg[wit={B}]{bhavaṃtī}
       \rdg[wit={N2,U1}]{bhavati}}/\\\\
    1 \app{\lem[wit={E,P}]{ūrdhvalakṣyam}
       \rdg[wit={L,B,N2}]{ūrdhvalakṣam}
       \rdg[wit={N1,D1}]{urdhvalakṣya}
       \rdg[wit={N2,U1}]{urdhvalakṣa}}/\\
    2 \app{\lem[wit={E,P}]{adholakṣyam}
       \rdg[wit={B,L,U2}]{adholakṣam}
       \rdg[wit={N1,D1}]{adholakṣya}
       \rdg[wit={N2}]{adholakṣa}
       \rdg[wit={U1}]{\om}}/\\
    3 \app{\lem[wit={U2}]{bāhyalakṣyam}
       \rdg[wit={N1,D1}]{bāhyalakṣya}
       \rdg[wit={N2}]{bāhyalakṣa}
       \rdg[wit={U1}]{bāhyalakya}
       \rdg[wit={B}]{bāhyakṣam}
       \rdg[wit={E}]{lakṣyam}
       \rdg[wit={P}]{madhyalakṣyam}
       \rdg[wit={L}]{madhyalakṣam}}/\\
    4 \app{\lem[type={emendation}, resp={egoscr}]{madhyalakṣyam}
       \rdg[wit={N1,D1}]{\korr madhyalakṣya}
       \rdg[wit={N2,U1}]{madhyalakṣa}
       \rdg[wit={U2}]{madhyalakṣaṃ}
       \rdg[wit={E,P}]{bāhyalakṣyam}
       \rdg[wit={L}]{bāhyakṣam}
       \rdg[wit={B}]{\om}}/\\
    5 \app{\lem[wit={E,P}]{antaralakṣyam}
       \rdg[wit={N1,D1,U1}]{antaralakṣya}
       \rdg[wit={B,L}]{aṃtaralakṣam}
       \rdg[wit={N2}]{antaralakṣa}
       \rdg[wit={U2}]{sarvalakṣyam}}/\\\\
\end{prose}
\end{edition}
\begin{translation}
  \ekddiv{type=trans}
     \bigskip
    \centerline{\textrm{\small{[Lakṣyayoga, the yoga of fixation]}}}
    \bigskip
 \begin{tlate}
Now the yoga of fixation (\textit{lakṣyayoga}), which is easily accomplished is explained. Of this yoga of fixation there are five subdivisions:\\\\ 1. The upward directed fixation (\textit{ūrdhvalakṣya}),\\ 2. the downward directed fixation (\textit{adholakṣya}),\\3. the outer fixation (\textit{baḥyalakṣya}),\\4. the central fixation (\textit{madhyalakṣya}),\\ 5. the inner fixation (\textit{antaralakṣya}).\\\\
 \end{tlate}
\end{translation}
\clearpage
 \begin{edition}
   \ekddiv{type=ed}
   \bigskip
     \centerline{\textrm{\small{[1. Ūrdhvalakṣya - The upward directed fixation]}}}
     \bigskip
\begin{prose}    
%------------------------------      
%prathamam ūrdhvalakṣyaṃ kathyate/  \E
%prathamam ūrdhvalakṣyaḥ kathyate/  \P
%atha      ūrdhvalakṣaṃ          // \L
%athama    urdhalakṣaṃ           // \B
%prathamaṃ urdhvalakṣaḥ  kathyate/  \N1
%prathamaṃ urdhvalakṣaḥ  kathyate/  \N2
%prathamaṃ urdhvalakṣaḥ  kathyate/  \D1
%prathamaṃ urdhvalakṣya/ kathyate/  \U1
%prathamaṃ urdhvalakṣaṃ  kathyate/  \U2
%------------------------------
%At first the upward directed fixation{\textit{adholakṣya} is explained. 
%------------------------------
     \app{\lem[wit={E,P},alt={prathamam}]{prathama\skp{mū}}
       \rdg[wit={N1,N2,D1,U1,U2}]{prathamaṃ}
       \rdg[wit={L}]{atha}
       \rdg[wit={B}]{athama}}\app{\lem[wit={E},alt={ūrdhvalakṣyaṃ}]{\skm{mū}rdhvalakṣyaṃ}
       \rdg[wit={P}]{ūrdhvalakṣyaḥ}
       \rdg[wit={U1}]{urdhvalakṣya}
       \rdg[wit={L}]{ūrdhvalakṣaṃ}
       \rdg[wit={U2}]{urdhvalakṣaṃ}
       \rdg[wit={N1,N2,D1}]{urdhvalakṣaḥ}
       \rdg[wit={B}]{urdhalakṣaṃ}}
     \app{\lem[wit={E,P,N1,N2,D1,U1,U2}]{kathyate}
       \rdg[wit={L,B}]{\om}}/ 
%------------------------------     
%ākāśamadhye dṛṣṭiḥ / \E
% \om                 \P
%ākāśamadhye dṛṣṭiḥ / \L
%ākāśamadhye dṛṣṭi    \B
%ākāśamadhye dṛṣṭiḥ / \N1
%ākāśamadhye dṛṣṭiḥ / \N2
%ākāśamadhye dṛṣṭiḥ / \D1
%ākāśamadhye dṛṣṭiḥ / \U1
%ākāśamadhye dṛṣṭiḥ / \U2
%------------------------------
%The gaze (\textit{dṛṣṭi)) [should be] in the middle of the sky. 
%------------------------------
  \app{\lem[wit={E,L,B,N1,N2,D1,U1,U2}]{ākāśamadhye}
    \rdg[wit={P}]{\om}}
  \app{\lem[wit={E,L,N1,N2,D1,U1,U2}]{dṛṣṭiḥ}
    \rdg[wit={B}]{dṛṣṭi}
    \rdg[wit={P}]{\om}}/
%------------------------------     
%kadā ca    mana    ūrdhvaṃ      kṛtvā sthāpayati /     \E x
%atha ca    mana    ūrdhvaṃ      kṛtvā sthāpyate /      \P x
%atha vā            ūrdhvaṃ mana kṛtvā sthāpyate        \L
%atha vā            ūrdhvamana   kṛtvā sthāpyate        \B
%atha ca // mana    urdhvaṃ      kṛtvā sthāpyate /      \N1 x
%atha ca mana       ūrdhvaṃ      kṛtvā sthāpyate /      \N2 x
%atha vā mana       ūrdhaṃ       kṛtvā sthāpyate        \D1 x
%atha ca maner------ddhvaṃ       kṛtvā sthāpyate        \U1
%atha    mana       urdhvaṃ      kṛtvā sthāpyate//      \U2 x
%------------------------------
%And then having caused the mind to be directed upwards, it is caused to be fixed there. 
%------------------------------
  \app{\lem[wit={P,N1,N2,U1}]{atha ca}
    \rdg[wit={L,B,D1}]{atha vā}
    \rdg[wit={U2}]{atha}
    \rdg[wit={E}]{kadā ca}}
  \app{\lem[wit={E,P,N2}]{mana ūrdhvaṃ}
    \rdg[wit={N1,U2}]{mana urdhvaṃ}
    \rdg[wit={D1}]{mana ūrdhaṃ}
    \rdg[wit={U1}]{manerddhvaṃ}
    \rdg[wit={L}]{ūrdhvaṃ mana}
    \rdg[wit={B}]{ūrdhvamana}}
  kṛtvā
  \app{\lem[wit={P,L,B,N1,N2,D1,U1,U2}]{sthāpyate}
    \rdg[wit={E}]{sthāpayati}}/
%------------------------------ 
%etasya lakṣyasya  dṛḍhakaraṇāt   parameśvarasya tejasā saha dṛṣṭer-aikyaṃ  bhavati /  \E
%etasya lakṣyasya  dṛḍhakaraṇāt   parameśvarasya tejasā saha dṛṣṭer-aikyaṃ  bhavati /  \P
%etasya lakṣasya   dṛḍhīkṛtvā//   parameśvarasya teja---saha dṛṣṭair-aikā   bhavati //  \L
%etasya lakṣasya   dṛḍhīkṛtvā//   parameśvarasya teja---saha dṛṣṭair-aikā   bhavati //  \B
%etasya lakṣyasya  dṛḍhīkaraṇāt / parameśvarasya tejasā saha dṛṣteḥ aikyaṃ  bhavati /  \N1
%etasya lakṣaṇasya dṛḍhīkaraṇāt   parameśvarasya tejasā saha dṛṣteḥ ekaṃ    bhavati //  \N2
%etasya lakṣasya   dṛḍhīkaraṇāt// parameśvarasya tejasā saha dṛṣṭeḥ aikyaṃ  bhavati // \D1
%etasya lakṣasya   dṛḍhīkaraṇāt/  parameśvarasya tejasā saha dṛṣṭer-aikyaṃ  bhavati/ \U1
%etasya lakṣasya   dṛḍhīkaraṇāt   parameśvarasya tenasā saha dṛṣṭer-aikyaṃ  bhavati // \U2
%------------------------------
%Due to the exercise of stabilizing of this fixation (\textit{lakṣya}) arises unity of the gazing point (\textit{dṛṣṭi}) with the light of the highest lord (\textit{parameśvara}). 
%------------------------------
etasya
  \app{\lem[wit={E,P,N1}]{lakṣyasya}
    \rdg[wit={L,B,D1,U1,U2}]{lakṣasya}
    \rdg[wit={N2}]{lakṣaṇasya}}
  \app{\lem[wit={N1,N2,D1,U1,U2}]{dṛḍhīkaraṇāt}
    \rdg[wit={E,P}]{dṛḍhakaraṇāt}
    \rdg[wit={L,B}]{dṛḍhīkṛtvā}}
  parameśvarasya
\app{\lem[wit={E,P,N1,N2,D1,U1}]{tejasā}
  \rdg[wit={U2}]{tenasā}
  \rdg[wit={L,B}]{teja°}}
saha
\app{\lem[wit={E,P,U1,U2}]{dṛṣṭer-aikyaṃ}
  \rdg[wit={N1,D1}]{dṛṣṭeḥ aikyaṃ}
  \rdg[wit={N2}]{dṛṣteḥ ekaṃ}
  \rdg[wit={L,B}]{ dṛṣṭair aikā}}
bhavati/  
%------------------------------
%atha cākāśa----madhye    yaḥ kaścidadṛṣṭaḥ   padārtho bhavati /  \E x
%atha cākāśa----madhye    yaḥ kaścidadṛṣṭaḥ   padārtho bhavati /  \P  x
%atha vākāśa----madhye    yaḥ kacciddṛṣṭaḥ    padārtho bhavati    \L  x
%athā cākāśa----madhye    yaḥ kaccit dṛṣṭaḥ   padārtho bhavati    \B   x
%atha ca ākāśa--madhye    yaḥ kaścitadṛṣtaḥ   padārthe bhavati /  \N1   x
%atha// ākāśa---madhye    yaḥ kaścita adṛṣtaḥ padārtha bhavati /  \N2  x 
%atha ca ākāśa--madhye    yaḥ kaścitadṛsṭaḥ   padārtho bhavati /  \D1    x
%atha ca/ ākāśa-madhye    yaḥ kaścidadṛsṭaḥ   padārtho bhavati    \U1    x
%atha cākāśa----madhye    yaḥ kaściddṛsṭa-----padārtho bhavati /  \U2
%------------------------------
%And then an indefinable invisible object arises in the middle of the sky.
%------------------------------
\app{\lem[wit={E,P,L,N1,N2,D1,U1,U2}]{atha}
  \rdg[wit={B}]{athā}}
\app{\lem[wit={E,P,B,U2},alt={cākāśa°}]{cākāśa}
  \rdg[wit={N1,D1,U1}]{ca ākāśa°}
  \rdg[wit={L}]{vākāśa°}
  \rdg[wit={N2}]{ākāśa°}}madhye
yaḥ
\app{\lem[wit={E,P,N1,D1,U1}]{kaścid-adṛṣṭaḥ}
  \rdg[wit={L}]{kacciddṛṣṭaḥ}
  \rdg[wit={B}]{kaccit dṛṣṭaḥ}
  \rdg[wit={N2}]{kaścita adṛṣtaḥ}
  \rdg[wit={U2}]{kaściddṛsṭa°}}
\app{\lem[wit={E,P,L,B,D1,U1,U2}]{padārtho}
  \rdg[wit={N1}]{padārthe}
  \rdg[wit={N2}]{padārtha}}
bhavati/ 
%------------------------------
%sa sādhakasya dṛṣṭigocaro bhavati//  \E
%sa sādhakasya dṛṣṭigocaro bhavati//  \P
%   sādhakasya dṛṣṭigocaro bhavati//  \L
%   sādhakasya dṛṣṭigocaro bhavatī    \B
%sa sādhakasya dṛṣṭigocare bhavati // \D1  saḥ-Sonderregel -> ḥ fällt aus vor allen Konsonanten
%sa sādhakasya dṛṣṭigocare bhavati // \N1
%   sādhakasya dṛṣṭigocarā bhavati // \N2
%sa sādhakasya dṛṣṭigocaro bhavati    \U1
%   sādhakasya dṛṣṭigocare bhavati // \U2
%------------------------------
%It arises in the range of sight of the practitioner.  
%------------------------------
\app{\lem[wit={E,P,D1,N1,U1}]{sa}
  \rdg[wit={L,B,N2,U2}]{\om}}
sādhakasya
\app{\lem[wit={D1,N1,U2}]{dṛṣṭigocare}
  \rdg[wit={E,P,L,B,U1}]{dṛṣṭigocaro}
  \rdg[wit={N2}]{dṛṣṭigocarā}}
\app{\lem[wit={E,P,L,D1,N1,D1,N2,U1,U2}]{bhavati}
  \rdg[wit={B}]{bhavatī}}/
%------------------------------
%ayam evordhvalakṣyaḥ      \E
%ayam evordhvalakṣyaḥ      \P
%ayam evordhvalakṣaḥ  //   \L
%ayam evordhalakṣaḥ  //    \B
%ayam evordhvalakṣya  //   \N1
%ayam eva vodhalakṣaṇam // \N2
%ayam evordhvalakṣyaḥ //   \D1
%ayam evordhvalakṣyaḥ      \U1
%ayam evordhvalakṣya //    \U2
%------------------------------
%This is truly the upward directed fixation (\textit{ūrdhvalakṣya}).
%------------------------------
aya\skp{m-e}\app{\lem[wit={E,P,D1,U1},alt={evordhvalakṣyaḥ}]{\skm{m-e}vordhvalakṣayaḥ}
  \rdg[wit={L}]{°lakṣaḥ}
  \rdg[wit={B}]{evordhalakṣaḥ}
  \rdg[wit={N1,U2}]{°lakṣya}
  \rdg[wit={N2}]{eva vodhalakṣaṇam}}/
\end{prose}
\end{edition}
\begin{translation}
  \ekddiv{type=trans}
     \bigskip
    \centerline{\textrm{\small{[1. Ūrdhvalakṣya - The upward directed fixation]}}}
    \bigskip    
  \begin{tlate}
At first the upward directed fixation (\textit{ūrdhvalakṣya}) is explained. The gaze (\textit{dṛṣṭi}) [should be] in the middle of the sky. And then having caused the mind to be directed upwards, it is caused to be fixed there. Due to the exercise of stabilizing of this fixation (\textit{lakṣya}) arises unity of the gazing point (\textit{dṛṣṭi}) with the light of the highest lord (\textit{parameśvara}). And then an indefinable invisible object arises in the middle of the sky. It arises in the range of sight of the practitioner. This is truly the upward directed fixation (\textit{ūrdhvalakṣya}).
  \end{tlate}
\end{translation}  
\clearpage
\begin{edition}
  \ekddiv{type=ed}
   \bigskip
    \centerline{\textrm{\small{[2. Adholakṣya - The downward directed fixation]}}}
    \bigskip
 \begin{prose}
%------------------------------
%                            nāsikāyāḥ  upari     dvādaśāṃgulamūlaparyantaṃ dṛṣṭiḥ sthirā karttavyā /   \E
%       athādholakṣaḥ        nāsikāyā   upari     dvādaśāṃgulaparyantaṃ     dṛṣṭiḥ sthirā karttavyā /   \P
%       athādholakṣaḥ //     nāsikāyā   upari     dvādaśāṃgulaparyaṃtaṃ     dṛṣṭiḥ sthirā karttavyā     \L
%       athādholakṣa //      nāsikāyā   upari     dvādaśāṃgulaparyaṃtaṃ     dṛṣṭiḥ sthirā karttavyā     \B
%       atha adholakṣyaḥ //  nāsikāyā   upari     dvādaśaṃgulaparyaṃtaṃ     dṛṣṭiḥ sthirā karttavyā //  \N1
%       atha adholakṣanaḥ // nāsikāyā   upari     dvādaśāṃgulaparyaṃtaṃ     dṛṣṭiḥ sthirā karttavyā //  \N2
%       atha adholakṣaḥ //   nāsikāyā   upari     dvādaśaṃgulaparyaṃtaṃ     dṛṣṭiḥ sthirā karttavyā //  \D1
%       atha adholakṣa       nāsikāyā   upari     dvādaśaṃgulaparyaṃtaṃ     dṛṣṭi--sthirā karttavyā     \U1
%                            nāsikāyāḥ  upariṣṭāt    daśāṃgulaparyaṃtaṃ     dṛṣṭiḥ sthirā karttavyā //  \U2
%------------------------------
%Now the downward directed fixation object (\textit{adholakṣya}). One should stabilize the gaze within the circumference (\textit{paryanta}) of twelve \textit{aṅgula}s beyond the nose.
%------------------------------
\app{\lem[type=emendation, resp=egoscr]{athādholakṣyaḥ}
  \rdg[wit={N1}]{\korr atha adholakṣyaḥ}
  \rdg[wit={P,L}]{athādholakṣaḥ}
  \rdg[wit={B}]{athādholakṣa}
  \rdg[wit={N2}]{atha adholakṣanaḥ}
  \rdg[wit={D1}]{atha adholakṣaḥ}
  \rdg[wit={U1}]{atha adholakṣa}
  \rdg[wit={E,U2}]{\om}}/
\app{\lem[wit={P,L,B,N1,N2,D1,U1}]{nāsikāyā}
  \rdg[wit={E,U2}]{nāsikāyāḥ}}
\app{\lem[wit={E,P,L,B,N1,N2,D1,U1}]{upari}
  \rdg[wit={U2}]{upariṣṭāt}}
\app{\lem[wit={P,L,B,N1,N2,D1,U1}]{dvādaśāṃgulaparyantaṃ}
  \rdg[wit={E}]{°mūlaparyantaṃ}
  \rdg[wit={U2}]{daśāṃgulaparyaṃtaṃ}}
\app{\lem[wit={E,P,L,B,N1,N2,D1,U2}]{dṛṣṭiḥ}
  \rdg[wit={U1}]{dṛṣṭi°}}
sthirā karttavyā/
%------------------------------
%atha vā nāsikāyā agre dṛṣṭiḥ sthirā karttavyā / \E
%atha vā nāsikāyā agre dṛṣṭiḥ sthirā karttavyā / \P
%\om / \L
%\om / \B
%atha vā nāsikāyā  agre dṛṣṭiḥ sthirā karttavyā // \N1
%atha vā nāsikā    agre dṛṣṭi-sthirā karttavyā      \N2
%atha vā nāsikāyā  agre dṛṣṭiḥ sthirā karttavyā // \D1
%atha vā nāśikāyāḥ/ agre dṛṣṭiḥ/ sthirā karttavyā / \U1
%atha vā nāsikāyā  agre dṛṣṭiḥ sthirā karttavyā // \U2
%------------------------------
%Or one should stabilize the gaze onto the tip of the nose.
%------------------------------
\app{\lem[wit={E,P,N1,N2,D1,U1,U2}]{atha vā}
  \rdg[wit={L,B}]{\om}}
\app{\lem[wit={E,P,N1,D1,U2}]{nāsikāyā}
  \rdg[wit={U1}]{nāsikāyāḥ}
  \rdg[wit={N2}]{nāsika}}
\app{\lem[wit={E,P,N1,N2,D1,U1,U2}]{agre}
  \rdg[wit={L,B}]{\om}}
\app{\lem[wit={E,P,N1,D1,U1,U2}]{dṛṣṭiḥ}
  \rdg[wit={N2}]{dṛṣṭi°}}
\app{\lem[wit={E,P,N1,N2,D1,U1,U2}]{sthirā}
  \rdg[wit={L,B}]{\om}}
\app{\lem[wit={E,P,N1,N2,D1,U1,U2}]{karttavyā}
  \rdg[wit={L,B}]{\om}}/ 
%------------------------------
%lakṣadūyasya  dṛḍhīkaraṇāt / dṛṣṭiḥ sthirā bhavati / \E
%lakṣadvayasya dṛṣṭīkaraṇāt / dṛṣṭiḥ sthirā bhavati / \P
%lakṣadvayasya dṛḍhīkaraṇāt   dṛṣṭi--sthiro bhavati / \L
%lakṣadvayasya dṛḍhīkaraṇān---dṛṣṭiḥ sthiro bhavatī   \B
%lakṣadvayasya dṛdhīkaraṇāt   dṛṣṭiḥ sthirā bhavati / \N1
%lakṣadvayasya dṛḍhīkaraṇād---dṛṣṭi--sthirā bhavati / \N2
%lakṣadvayasya dṛḍhīkaraṇāt   dṛṣṭiḥ sthirā bhavati / \D1
%lakṣadvayasya dṛḍhīkaraṇāt   dṛṣṭiḥ sthirā bhavati / \U1
%lakṣadvayasya dṛḍhīkaraṇāt   dṛṣṭi--sthirā bhavati // \U2
%------------------------------
%The fixation becomes stable due to firm exercise [on one] of the twofold aims [of fixation]. 
%------------------------------
\app{\lem[wit={P,L,B,N1,N2,D1,U1,U2}]{lakṣadvayasya}   %emend to lakṣyadvayasya??? 
  \rdg[wit={E}]{lakṣadūyasya}} 
\app{\lem[wit={N2}, alt={dṛḍhīkaraṇād}]{dṛḍhīkaraṇād\skm{-dṛ}}
  \rdg[wit={E,L,N1,D1,U1,U2}]{dṛḍhīkaraṇāt}
  \rdg[wit={P}]{dṛṣṭīkaraṇāt}
  \rdg[wit={B}]{dṛḍhīkaraṇān}}\app{\lem[wit={E,P,B,N1,D1,U1}, alt={dṛṣṭiḥ}]{\skp{-dṛ}ṣṭiḥ}
  \rdg[wit={L,N2,U2}]{dṛṣṭi°}}
\app{\lem[wit={E,P,N1,N2,D1,U1,U2}]{sthirā}  
  \rdg[wit={B}]{sthiro}
  \rdg[wit={L}]{°sthiro}}
\app{\lem[wit={E,P,L,N1,N2,D1,U1,U2}]{bhavati}
  \rdg[wit={B}]{bhavatī}}/
%------------------------------
%pavanaḥ sthiro bhavati / \E
%pavanaḥ sthiro bhavati / \P
%\om                    / \L
%\om                    / \B
%pavanaḥ sthiro bhavati / \N1
%pavana--sthiro bhavati /   \N2
%pavanaḥ sthiro bhavati / \D1
%pavana--sthiro bhavati  / \U1
%pavana--sthiro bhavati  / \U2
%------------------------------
%The breath becomes stable. 
%------------------------------
\app{\lem[wit={E,P,N1,D1}]{pavanaḥ}
  \rdg[wit={N2,U1,U2}]{pavana°}
  \rdg[wit={L,B}]{\om}}
\app{\lem[wit={E,P,N1,N2,D1,U1,U2}]{sthiro}
  \rdg[wit={L,B}]{\om}}
\app{\lem[wit={E,P,N1,N2,D1,U1,U2}]{bhavati}
  \rdg[wit={L,B}]{\om}}/
%------------------------------
%āyurvarddhate / \E
%āyurvarddhate / \P
%āyurvarddhate / \L
%āyurvardhate /  \B
%āyurvardhate /  \N1
%āyurvardhate /  \N2
%āyurvardhate /  \D1
%āyurvarddhate   \U1
%āyurvarddhate //  \U2
%------------------------------
%Vitality increases. 
%------------------------------
āyur-varddhate/
%------------------------------
%etad dūyam       api bāhyalakṣyam eva  bhavati      bāhyāṃtara       ākāśe         śūnyalakṣyaṃ    karttavyaḥ / \E
%etad dvayam      api bāhyalakṣyam eva  bhavati      bāhyābhyaṃtare   ākāśe cet     śūnyalakṣyaṃ    karttavyaḥ / \P
%etad dvayam      api bāhyalakṣam  eva  bhavati//    bāhyābhyaṃtare   ākāśacen      śūnyaṃ lakṣaṃ   karttavyā // \L
%etad dvayadvayam api bāhyalakṣam  eva  bhavatī//    bāhyābhyaṃtare   ākāśacvat     śūnyaṃ lakṣaṃ   karttavyā // \B
%etat advayam     eva bāhyalakṣam  api  kathyate //  bāhyo bhyaṃtaraṃ ākāśavat------śūnyalakṣyaḥ    karttavyaḥ / \N1
%etad dvayam      eva bāhyalakṣam  api  kathyate //  bāhyābhyaṃtaram--ākāśavat------śūnyalakṣaḥ     karttavyaḥ   \N2
%etat advayam     eva bāhyalakṣam  api  kathyate //  bāhyo bhyaṃtaraṃ ākāśavat //   śūnyalakṣyaḥ    karttavyaḥ / \D1
%etat dvayam      eva bāhyalakṣyam api  kathyate/    bāhyābhyaṃtare   ākāśavat------śūnyalakṣyaḥ    karttavyaḥ  \U1
%etat dvayam      api bāhyalakṣyam eva  bhavati//    bāhyābhyaṃtare   ākāśe cet     śūnyalakṣyaṃ    karttavyaḥ / \U2
%------------------------------
%Just as this [aim] is twofold, also the external fixation is said to be [like this]. Internally or externally the aim of fixation is to be done onto the heavenly void.  
%------------------------------
\app{\lem[wit={P,L,N2},alt={etad dvayam}]{etad-dvayam\skm{-e}}
  \rdg[wit={E}]{etad dūyam}
  \rdg[wit={B}]{etad dvayadvaya}
  \rdg[wit={N2,D1}]{etat advayam}
  \rdg[wit={U1,U2}]{etat dvayam}}\app{\lem[wit={N1,N2,D1,U1}, alt={eva}]{\skp{-e}va}
  \rdg[wit={E,P,L,B,U2}]{api}} 
\app{\lem[wit={E,P,U1,U2},alt={bāhyalakṣyam}]{bāhyalakṣyam\skm{-a}}
  \rdg[wit={L,B,N1,N2,D1}]{°lakṣam}}\app{\lem[wit={N1,N2,D1,U1},alt={api}]{\skp{-a}pi}
  \rdg[wit={E,P,L,B,U2}]{eva}}
\app{\lem[wit={N1,N2,D1,U1}]{kathyate}
  \rdg[wit={E,P,L,U2}]{bhavati}
  \rdg[wit={B}]{bhavatī}}/
\app{\lem[wit={N2},alt={bāhyābhyantaram}]{bāhyābhyantaram\skm{-ā}}                %Übersetzung nochmal überdenken! 
  \rdg[wit={N1,D1}]{bāhyo bhyaṃtaraṃ}
  \rdg[wit={P,L,B,U1,U2}]{bāhyābhyaṃtare}
  \rdg[wit={E}]{bāhyāṃtara}}\app{\lem[wit={N1,N2,D1,U1},alt={ākāśavat}]{\skp{-ā}kāśavat}
  \rdg[wit={B}]{ākāśacvat}
  \rdg[wit={L}]{ākāśacen}
  \rdg[wit={P,U2}]{ākāśe cet}
  \rdg[wit={E}]{ākāśe}}
\app{\lem[wit={N1,D1,U1}]{śūnyalakṣyaḥ}
  \rdg[wit={E,P,U2}]{śūnyalakṣyaṃ}
  \rdg[wit={N2}]{śūnyalakṣaḥ}
  \rdg[wit={L,B}]{śūnyaṃ lakṣaṃ}}
\app{\lem[wit={E,P,N1,N2,D1,U1,U2}]{karttavyaḥ}
  \rdg[wit={L,B}]{karttavyā}}/ 
%------------------------------
%jāgraddaśāyāṃ    calanadaśāyāṃ   bhojanadaśāyāṃ   sthitikāle sarvasthāne   śūnyasya dhyānakāraṇāt //                              \E
%jāgraddaśāyāṃ    calanadaśāyāṃ   bhojanaṃ daśāyāṃ sthitikāle sarvasthāne   śūnyasya dhyānakāraṇāt //                              \P
%jāgradādidaśāyāṃ calanadaśāyāṃ// bhojanadaśāyāṃ   sthitikāle sarvasthāneṣu śūnyasya dhyānakāraṇāt //                              \L
%jāgradādidaśāyāṃ calanadaśāyāṃ// bhojanadaśāyāṃ   sthitikāle sarvasthāneṣu śūnyasya dhyānakaraṇāt //                              \B
%jāgraddaśāyāṃ    cakabadaśāyāṃ   bhojanadaśāyāṃ   sthitikāle sarvvasthāne  śūnyasya dhyānakaraṇāt  maraṇatrāso na bhavati//       \N1
%jāyadaśāyāṃ      calanadaśāyāṃ/  bhojanadaśāyāṃ   sthitikāle sarvasthāne   śūnyasya dhyānakaraṇāt  maraṇatrāśo na bhavati//       \N2
%jāgraddaśāyāṃ    calanadaśāyāṃ   bhojanadaśāyāṃ   sthitikāle sarvvasthāne  śūnyasya dhyānakaraṇāt  maraṇatrāso na bhavati// śūnya \D1
%jāgraddaśāyāṃ    calanadaśāyāṃ                    sthitikāle sarvasthāne   śūnyasya dhyānakaraṇāt/ maraṇasautrāṃ na bhavati vā    \U1
%jāgṛaddaśāyāṃ    calanadaśāyāṃ   bhojanadaśāyāṃ   sthitikāle sarvasthāne   śūnyasya dhyānakaraṇāt//                               \U2
%------------------------------
%The fear of dying does not arise due to the exercise of meditation on the void at all places during ones life - while eating, moving and waking. 
%------------------------------
\app{\lem[wit={E,P,N1,D1,U1}]{jāgraddaśāyāṃ}
    \rdg[wit={N2}]{jāgṛaddaśāyāṃ}
    \rdg[wit={N2}]{jāyadaśāyāṃ}
    \rdg[wit={L,B}]{jāgradādidaśāyāṃ}}
\app{\lem[wit={E,P,L,B,N2,D1,U1,U2}]{calanadaśāyāṃ}
    \rdg[wit={N1}]{cakabadaśāyāṃ}}
\app{\lem[wit={E,L,B,N1,N2,D1,U2}]{bhojanadaśāyāṃ}
    \rdg[wit={P}]{bhojanaṃ daśāyāṃ}
    \rdg[wit={U1}]{\om}}
  sthitikāle
\app{\lem[wit={E,P,N1,N2,D1,U1,U2}]{sarvasthāne}
    \rdg[wit={L,B}]{sarvasthāneṣu}}
  śūnyasya dhyānakāraṇāt
\app{\lem[wit={N1,D1}]{maraṇatrāso}
    \rdg[wit={N2}]{maraṇatrāśo}
    \rdg[wit={U1}]{maraṇasautrāṃ}
    \rdg[wit={E,P,L,B,U2}]{\om}}
\app{\lem[wit={N1,N2,D1,U1}]{na}
    \rdg[wit={E,P,B,U2}]{\om}}
\app{\lem[wit={N1,N2}]{bhavati}
    \rdg[wit={D1}]{bhavati// śūnya}
    \rdg[wit={U1}]{bhavati vā}
    \rdg[wit={E,P,L,B,U2}]{\om}}/
 \end{prose}
\end{edition}
\begin{translation}
  \ekddiv{type=trans}
   \bigskip
    \centerline{\textrm{\small{[2. Adholakṣya - The downward directed fixation]}}}
    \bigskip
  \begin{tlate}
    Now the downward directed fixation object (\textit{adholakṣya}). One should stabilize the gaze within the circumference (\textit{paryanta}) of twelve \textit{aṅgula}s beyond the nose. Or one should stabilize the gaze onto the tip of the nose. The breath becomes stable. Vitality increases. Just as this [aim] is twofold, also the external fixation is said to be [like this]. Internally or externally the aim of fixation is to be done onto the heavenly emptiness. The fear of dying does not arise due to the exercise of meditation on the void at all places during ones life - while eating, moving and waking.\\ \\
  \end{tlate}
\end{translation}
\clearpage
\begin{edition}
  \ekddiv{type=ed}
       \bigskip
    \centerline{\textrm{\small{[Description of Rājayogin's Body]}}}
    \bigskip
 \begin{prose}
%------------------------------  
%idānīṃ rājayogayuktasya           śarīre yaccihnaṃ  tat    kathyate / \E
%idānīṃ rājayogayuktasya puruṣasya yaccharīracihnaṃ         kathyate / \P
%idānīṃ rājayogayuktasya puruṣasya          cinhnaṃ         kathyate / \L
%idānīṃ rājayogayuktasya puruṣasya          cinhnaṃ         kathyate // \B
%idānīṃ rājayogayuktasya puruṣasya yaccarīracihnaṃ   tat    kathyate / \N1
%idānīṃ rājayogayuktasya puruṣasya yaccharīracihūṃ   tat    kathyate// \N2
%idānīṃ rājayogayuktasya puruṣasya yaccarīracihnaṃ   tat    kathyate / \D1
%idānīṃ rājayogayuktasya puruṣasya yaccharīre cinhaṃ tata   kathyate \U1
%idānīṃ rājayogayuktasya puruṣasya yat śarīracinhaṃ         kathyate / \U2
%------------------------------
%Now it is said that this is the characteristic of the embodied person who is endowed with the royal yoga:
%------------------------------
  idānīṃ rājayogayuktasya
  \app{\lem[wit={P,L,B,N1,N2,D1,U1,U2}]{puruṣasya}
    \rdg[wit={E}]{\om}}
  \app{\lem[wit={N1,D1,P},alt={yac carīracihnaṃ}]{yac-carīracihnaṃ}
    \rdg[wit={U2}]{yat śarīracinhaṃ}
    \rdg[wit={E}]{śarīre yac cihnaṃ}
    \rdg[wit={U1}]{yac charīre cinhaṃ}
    \rdg[wit={N2}]{yac charīracihūṃ}
    \rdg[wit={L,B}]{cinhnaṃ}}
  \app{\lem[wit={E,N1,N2,D1}]{tat}
    \rdg[wit={U1}]{tata}
    \rdg[wit={P,L,B,U2}]{\om}}
  kathyate/
%------------------------------  
%tatsarvatra pūrṇo bhavati / \E
%tatsarvatra pūrṇā bhavati / \P
%tatsarvatra pūrṇo bhavati / \L
%tatsarvatra pūrṇo bhavatī / \B
%  sarvvatra pūrṇo bhavati / \N1
%  sarvvatra pūrṇā bhavati  \N2
%  sarvvatra pūrṇo bhavati  \D1
%  sarvvatra pūrṇo bhavati   \U1
%tatsarvatra pūrṇo bhavati// \U2
%------------------------------
%Abundance arises at all times. %Alternative=permanent Abundance arises because of that.   
%------------------------------
\app{\lem[wit={N1,N2,D1,U1},alt={sarvatra°}]{sarvatra}
  \rdg[wit={E,P,L,B,U2}]{tatsarvatra°}}
\app{\lem[wit={E,L,B,N1,D1,U1,U2}, alt={°pūrṇo}]{pūrṇo}
  \rdg[wit={P,N2}]{pūrṇā}}
\app{\lem[wit={E,P,L,N1,N2,D1,U1,U2}]{bhavati}
  \rdg[wit={B}]{bhavatī}}/
%------------------------------  
%pṛthivyāḥ dūre tiṣṭhati / \E
%pṛthivyāḥ hare tiṣṭhati / \P
%\om                      \L
%\om                      \B
%pṛthivyāḥ dūre  tiṣṭhati / \N1
%pṛthivyāḥ dūra  tiṣṭhati / \N2
%pṛthivyāḥ dūre  tiṣṭhati / \D1
%pṛthivyāḥ ddūre tiṣṭhati / \U1 %emend to na tiṣṭhati? 
%pṛthivyā dūraṃ  tiṣṭhati // \U2 !!dūraṃ
%------------------------------
%No distances exist on earth.
%------------------------------
\app{\lem[type=conjecture, resp=egoscr]{pṛthivyāṃ}
  \rdg[wit={E,P,N1,N2,D1,U1}]{\conj pṛthivyāḥ}
  \rdg[wit={U2}]{pṛthivyā}
  \rdg[wit={L,B}]{\om}} 
\app{\lem[wit={U2}]{dūraṃ}
  \rdg[wit={E,N1,D1}]{dūre}
  \rdg[wit={U1}]{ddūre}
  \rdg[wit={N2}]{dūra}
  \rdg[wit={L,B}]{\om}}
\app{\lem[type=conjecture, resp=egoscr]{na tiṣṭhati}
  \rdg[wit={E,P,N1,N2,D1,U1,U2}]{\conj tiṣṭhati}
  \rdg[wit={L,B}]{\om}}/
%------------------------------
%pṛthivyāṃ vyāpya tiṣṭhati / \E
%pṛthi-----vyāpya tiṣṭhati / \P
%\om                         \L
%\om                         \B
%pṛthvāṃ vyāpya   tiṣṭhati /   \N1
%pṛthvīṃ vyāpya   tiṣṭhati /   \N2
%pṛthvīṃ vyāpya   tiṣṭhati /   \D1  %geht auch pṛthu für Erde? 
%\om   \U1
%pṛthivyā vyāti   tiṣṭhati     \U2
%------------------------------
%He dwells on earth having pervaded [it]. 
%------------------------------
\app{\lem[type=emendation, resp=egoscr]{pṛthivīṃ}
  \rdg[wit={E}]{pṛthivyāṃ}
  \rdg[wit={P}]{pṛthi°}
  \rdg[wit={N1}]{pṛthvāṃ}
  \rdg[wit={N2,D1}]{pṛthvīṃ}
  \rdg[wit={U2}]{pṛthivyā}
  \rdg[wit={L,B,U2}]{\om}}
\app{\lem[wit={E,P,N1,N2,D1,U2}]{vyāpya}
  \rdg[wit={U2}]{vyāti}
  \rdg[wit={L,B,U1}]{\om}} 
\app{\lem[wit={E,P,N1,N2,D1,U2}]{tiṣṭhati}
  \rdg[wit={L,B,U2}]{\om}}/
%------------------------------
% yasya janmamaraṇe  na staḥ sukhaṃ na bhavati /  \E
% yasya janmamaraṇe  na staḥ sukhaṃ na bhavati /  \P
% \om                                            \L
% \om                                            \B
% yasya janmamaraṇe  na staḥ sukhaṃ na bhavati /  \N1
% yasya janmamaraṇe  na staḥ sukhaṃ na bhavati /  \N2
% yasya janmamaraṇe  na staḥ sukhaṃ na bhavati /  \D1
% \om                                            \U1
% yasya jananamaraṇe na staḥ sukhaṃ na bhavati /  \U2 maraṇe nom/acc dual! staḥ von as 3. dual 
%------------------------------
% Birth and death both do not exist. Happiness does not exist. 
% ------------------------------
\app{\lem[wit={E,P,N1,N2,D1,U2}]{yasya}
  \rdg[wit={L,B,U1}]{\om}}
\app{\lem[wit={E,P,N1,N2,D1}]{janmamaraṇe}
  \rdg[wit={U2}]{jananamaraṇe}
  \rdg[wit={L,B,U1}]{\om}}
\app{\lem[wit={E,P,N1,N2,D1,U2}]{na}
  \rdg[wit={L,B,U1}]{\om}}
\app{\lem[wit={E,P,N1,N2,D1,U2}]{staḥ}
  \rdg[wit={L,B,U1}]{\om}}/
\app{\lem[wit={E,P,N1,N2,D1,U2}]{sukhaṃ}
  \rdg[wit={L,B,U1}]{\om}}
\app{\lem[wit={E,P,N1,N2,D1,U2}]{na}
  \rdg[wit={L,B,U1}]{\om}}
\app{\lem[wit={E,P,N1,N2,D1,U2}]{bhavati}
  \rdg[wit={L,B,U1}]{\om}}/
% \om                 \E
% \om                 \P
% \om                 \L
% \om                  \B
% duḥkhaṃ na bhavati / \N1
% duḥkhaṃ na bhavati / \N2
% duḥkham na bhavati / \D1
% \om                  \U1
% \om                  \U2
% ------------------------------
%Suffering does not exist. 
%------------------------------
\app{\lem[wit={N1,N2,D1}]{duḥkhaṃ}
  \rdg[wit={E,P,L,B,U1,U2}]{\om}} 
\app{\lem[wit={N1,N2,D1}]{na}
  \rdg[wit={E,P,L,B,U1,U2}]{\om}} 
\app{\lem[wit={N1,N2,D1}]{bhavati}
  \rdg[wit={E,P,L,B,U1,U2}]{\om}}/
%------------------------------
% \om               \E
% kalaṃ na bhavati  \L
% kulaṃ na bhavatī// \B
% kūlaṃ na bhavati / \P
% kūlaṃ na bhavati / \N1
% kūlaṃ na bhavati / \N2
% kūlaṃ na bhavati / \D1
% \om               \U1
% kulaṃ na bhavatī// \U2
%------------------------------
%Impediment does not exist.
%------------------------------
\app{\lem[wit={P,N1,N2,D1}]{kūlaṃ}
  \rdg[wit={B,U2}]{kulaṃ}
  \rdg[wit={L}]{kalaṃ}
  \rdg[wit={E,U1}]{\om}}
\app{\lem[wit={ceteri}]{na}
  \rdg[wit={E,U1}]{\om}}
\app{\lem[wit={ceteri}]{bhavati}
  \rdg[wit={B,U2}]{bhavatī}
  \rdg[wit={E,U1}]{\om}}/
%------------------------------
% \om                  \E
% śītalaṃ na bhavati / \P
% \om                  \L
% \om                  \B
% śīlaṃ na bhavati /   \N1
% śīlaṃ na bhavati /   \N2
% śīlaṃ na bhavati /   \D1
% śīlaṃ na bhavati /   \U1
% śīlaṃ na bhavati /   \U2
%------------------------------
% Habit doesn't exist. 
% ------------------------------
\app{\lem[wit={ceteri}]{śīlaṃ}
  \rdg[wit={P}]{śītalaṃ}
  \rdg[wit={E,L,B}]{\om}}
\app{\lem[wit={ceteri}]{na}
  \rdg[wit={E,L,B}]{\om}}
\app{\lem[wit={ceteri}]{bhavati}
  \rdg[wit={E,L,B}]{\om}}/
%------------------------------
% \om                 \E
% sthānaṃ na bhavati / \P
% \om                  \L
% \om                  \B
% sthānaṃ na bhavati / \N1
% sthānaṃ na bhavati / \N2
% sthānaṃ na bhavati / \D1
% sthānaṃ na bhavati / \U1
% sthānaṃ na bhavati / \U2
%------------------------------
% Place does not exist. 
%------------------------------
\app{\lem[wit={ceteri}]{sthānaṃ}
  \rdg[wit={E,L,B}]{\om}}
\app{\lem[wit={ceteri}]{na}
  \rdg[wit={E,L,B}]{\om}}
\app{\lem[wit={ceteri}]{bhavati}
  \rdg[wit={E,L,B}]{\om}}/
%------------------------------
% \om                                                                             \E
%asya siddhasya manomadhye īśvarasaṃbaṃdhī prakāśo niraṃtaraṃ     pratyakṣo bhavati  \P
%asya siddhasya manomadhye īśvarasaṃbaṃdhi prakāśo  niraṃtaraṃ    pratyakṣo bhavati  \L
%asya siddhasya manomadhye īśvaraṃ saṃbaṃdhī prakāśo  niraṃtaraṃ  pratyakṣo bhavatī//  \B
%asya siddhasya manomadhye īśvarasaṃbaṃdhī prakāśaḥ niraṃtaraṃ    pratyakṣa bhavati  \N1
%asya siddhasya manomadhye īśvarasaṃbaṃdhī prakāśaḥ niraṃtaraṃ    pratyakṣa bhavati/  \N2
%asya siddhasya manomadhye īśvarasaṃbaṃdhi prakāśaḥ niraṃtaraṃ    pratyakṣo bhavati  \D1
%asya siddhasyaṃ pṛthivī vyāpya tiṣṭhati yasya yanma maraṇai na saḥ sukhaṃ na bhati kulaṃ na bhavati śīlaṃ na bhavati sthānaṃ na bhavati ..... asya siddhasya manomadhye īśvarasaṃbaṃdhī prakāśaḥ niraṃtaraṃ pratyakṣo bhavati  \U1
%asya siddhasya manomadhye īśvarasaṃbaṃdhī prakāśo nirattaraṃ  pratyakṣo bhavati//  \U2
%------------------------------
%The manifestation of permanent perception of the connection with god arises in the middle of the mind of this accomplished one. 
%------------------------------
\app{\lem[wit={ceteri}]{asya}
  \rdg[wit={E}]{\om}}
\app{\lem[wit={ceteri}]{siddhasya}
  \rdg[wit={U1}]{siddhasyaṃ pṛthivī vyāpya tiṣṭhati yasya yanma maraṇai na saḥ sukhaṃ na bhati kulaṃ na bhavati śīlaṃ na bhavati sthānaṃ na bhavati asya siddhasya}
  \rdg[wit={E}]{\om}}
\note[type=philcomm, labelb=s34.z3, lem={asya siddhasyaṃ}]{U1 repeats the whole section from pṛthivī to ... sthānaṃ na bhavati due to an eyeskip in the process of copying.}
\app{\lem[wit={ceteri}]{manomadhye}
  \rdg[wit={E}]{\om}}
\app{\lem[wit={ceteri}]{īśvarasaṃbandhī}
  \rdg[wit={B}]{īśvaraṃ saṃbaṃdhī}
  \rdg[wit={E}]{\om}}
\app{\lem[wit={ceteri}]{prakāśo}
  \rdg[wit={N1,N2,D1,U1}]{prakāśaḥ}
  \rdg[wit={E}]{\om}}
\app{\lem[wit={ceteri}]{nirantaraṃ}
  \rdg[wit={U2}]{nirattaraṃ}
  \rdg[wit={E}]{\om}}
\app{\lem[wit={ceteri}]{pratyakṣo}
  \rdg[wit={N1}]{prakyakṣa}
  \rdg[wit={E}]{\om}}
\app{\lem[wit={ceteri}]{bhavati}
  \rdg[wit={B}]{bhavatī}
  \rdg[wit={E}]{\om}}/
%------------------------------
%sa ca prakāśo na śīto na coṣṇo na śveto na pīto bhavati / \E
%sa ca prakāśo na śīto na coṣṇo na śveto na pīto bhavati / \P
%sa ca prakāśo na śīto na coṣṇo na śveto na pīto bhavatī // \L
%sa ca prakāśo na śīto na coṣṇo na śveto na pīto bhavatī // \B
%sa ca prakāśo na śīto na coṣṇo na śveto na pīto bhavati    \D1
%sa ca prakāśo na śīto na coṣṇo na kheto na pīto bhavati/ \N2
%sa ca prakāśo na śīto na ?hbho?na kheto na pīto bhavati // \U1
%sa ca prakāśo// na śīto na coṣṇo na śveto pīto na bhavati // \U2
%------------------------------
%And he is shining - not cold, and not hot, not white [and] not yellow. 
%------------------------------
sa ca prakāśo na śīto na
\app{\lem[wit={ceteri}]{coṣṇo}
  \rdg[wit={U1}]{...o}}
na
\app{\lem[wit={ceteri}]{śveto}
  \rdg[wit={N2,U1}]{kheto}}
\app{\lem[wit={ceteri}]{na pīto}
  \rdg[wit={U2}]{pīto na}}
\app{\lem[wit={ceteri}]{bhavati}
  \rdg[wit={L,B}]{bhavatī}}/
%------------------------------
%tasya na jātir na kiñciccihnam  \E
%tasya na jātir na kiñciccihnaṃ  \P
%tasya na jātir na kiṃciccinhaṃ  \L
%tasya na jātir na kiṃciccinhaṃ  \B
%tasya na jātir na kiṃciccihūṃ  \N1
%tasya na jāti na kiṃciccihūṃ//  \D1
%tasya na jāti na  kiṃciccihūṃ  \N2
%tasya na jātir na kiṃcit khecha cinhaṃ  \U1
%tasya na jānāti na kiṃcit cinhaṃ //  \U2
%------------------------------
%Neither is there birth of him, nor does he have any attributes.
%------------------------------
tasya na
\app{\lem[wit={ceteri}, alt={jātir}]{jātir\skm{-na}}
  \rdg[wit={D1,N2}]{jāti}
  \rdg[wit={U2}]{jānāti}
}\skp{-na}
\app{\lem[wit={ceteri}, alt={kiñcic cihnaṃ}]{kiñcic\skp{-}cihnaṃ}
  \rdg[wit={E}]{°cihnam}
  \rdg[wit={D1,N1,N2}]{°cihūṃ}
  \rdg[wit={U1}]{kiṃcit khecha cinhaṃ}
  \rdg[wit={U2}]{na kiṃcit cinhaṃ}}/
%------------------------------
%ayaṃ   ca niṣkalo   niraṃjanaḥ   alakṣyaś ca bhavati \E
%ayaṃ   ca niṣkalo   niraṃjanaḥ   alakṣyaś ca bhavati \P
%vyayaṃ ca niṣkalo   niraṃjanaṃ// alakṣaś  ca bhavati// \L
%vyayaṃ ca nīṣkalo   niraṃjanaṃ// alakṣaś  ca bhavatī// \B
%ayaṃ   ca niṣkalo   niraṃjanaḥ// alakṣyaś ca bhavati// \D1
%ayaṃ   ca nīṣkalo   niraṃjanaḥ   alakṣaś  ca bhavati// \N1
%ayaṃ   ca niṣkalo   niraṃjanaḥ   alakṣaś  ca bhavati// \N2
%ayaṃ   ca niḥkalo   niraṃjanaḥ   alakṣyaḥ    bhavati/ \U1
%ayaṃ   ca nīṣkalo   niraṃjanaḥ// alakṣyaḥ    bhavati// \U2
%------------------------------
%And he is without parts, immacule and uncharacterized.  
%------------------------------
\app{\lem[wit={ceteri}]{ayaṃ}
  \rdg[wit={L,B}]{vyayaṃ}}
ca
\app{\lem[wit={ceteri}]{niṣkalo}
  \rdg[wit={B,U2}]{nīṣkalo}
  \rdg[wit={U1}]{niḥkalo}}
nirañjanaḥ/
\app{\lem[wit={ceteri}, alt={alakṣyaś}]{alakṣyaś\skm{-ca}}
  \rdg[wit={U1,U2}]{alakṣyaḥ}
  \rdg[wit={L,B,N1,N2}]{alakṣaś}
}\app{\lem[wit={ceteri}, alt={ca}]{\skp{ca}}
  \rdg[wit={U1,U2}]{\om}}
\app{\lem[wit={ceteri}]{bhavati}
  \rdg[wit={B}]{bhavati}}/
%------------------------------
%atha ca phaladvaṃde  na         kāminy āder   yasyecchā         na bhavati // \E
%atha ca phalacaṃda   na         kāminy āder   yasyochā          na bhavati  \P
%atha ca phalavaṃda   na         kāminy ādir   yasya             na bhavati  \L
%atha ca phalaṃ jaṃda na         kāminy ādar   yasye             na bhavatī  \B
%atha ca phalacaṃdra  na         kāminy āder   yasya  yasyeccha   bhavati/  \N1
%atha ca phalacaṃda   na         kāminy āde    yasya  yasyechā    bhavati//  \D1
%atha ca phalaṃ/caṃdra           kāminy āder   yasya  yasyeccha   bhavati/  \N2
%atha ca phalaṃ caṃda na         kāminy āder   yasya  yaṃ         bhavati   \U1
%atha ca phalacaṃda   na         kāminy āder   yasye  chā         bhavati//  \U2
%------------------------------
%His desire etc. doesn't arise in [situations of] lust [and] is not located within the duality of the result.  
%------------------------------
atha ca
\app{\lem[wit={E}]{phaladvande}
     \rdg[wit={P,D1,U2}]{phalacaṃda}
     \rdg[wit={U1}]{phalaṃ caṃda}
     \rdg[wit={L}]{phalavaṃda}
     \rdg[wit={B}]{phalaṃ jaṃda}
     \rdg[wit={N1}]{phalacaṃdra}
     \rdg[wit={N2}]{phalaṃ/ caṃdra}}
\app{\lem[wit={ceteri}]{na}
     \rdg[wit={N2}]{\om}}
\skp{kāminy-}\app{\lem[wit={ceteri}, alt={āder}]{\skm{kāminy}āde\skp{r}}
     \rdg[wit={D1}]{āde}
     \rdg[wit={B}]{ādar}
     \rdg[wit={L}]{ādir}
}\app{\lem[wit={E},alt={yasyecchā}]{\skm{r}-yasyecchā}
     \rdg[wit={P}]{yasyochā}
     \rdg[wit={L}]{yasya}
     \rdg[wit={B}]{yasye}
     \rdg[wit={N1,N2}]{yasya yasyeccha}
     \rdg[wit={D1}]{yasya yasyechā}
     \rdg[wit={U1}]{yasya yaṃ}
     \rdg[wit={U2}]{yasye chā}}
\app{\lem[wit={E,P,L,B}]{na}
     \rdg[wit={ceteri}]{\om}}
\app{\lem[wit={ceteri}]{bhavati}
     \rdg[wit={B}]{bhavatī}}/ 
\end{prose}
\end{edition}
\begin{translation}
  \ekddiv{type=trans}
    \bigskip
    \centerline{\textrm{\small{[Description of Rājayogin's Body]}}}
    \bigskip
    \begin{tlate} Now it is said that this is the characteristic of the embodied person who is endowed with the royal yoga: Abundance arises at all times. No distance exists on earth. He dwells on earth having pervaded [it]. Birth and death both don't exist. Happiness does'nt exist. Suffering does'nt exist. Impediment does'nt exist. Habit doesn't exist. Place does'nt exist. The manifestation of permanent perception of the connection with god arises in the middle of the mind of this accomplished one. And he is shining - not cold, and not hot, not white [and] not yellow. Neither is there birth of him, nor does he have any attributes. And he is without parts, immacule and uncharacterized. His desire etc. doesn't arise in [situations of] lust [and] is not located within the duality of the result.    
    \end{tlate}
   \end{translation}
 \end{alignment}
\end{document}
%------------------------------
% \om                   \E
% \om                   \L
% \om                      \B
%taṃ taṃ bhogaṃ prāpnoti \D1
%taṃ taṃ bhogaṃ prāpnoti \N1
%taṃ taṃ bhogaṃ prāpnoti// \N2
%tataṃ bhogaṃ prāpnoti \U1
%------------------------------
%With regard to that? he attains happiness.  
%------------------------------
% \om \L
% \om \B
%atha vāyasya mana eva sthāne 'nurāgaṃ na prāpnoti// \D1
%atha vāsya/vātya mana eva sthāne 'nurāgaṃ na prāpnoti/ \N1
%atha vā syamana eva sthāne 'nurāgaṃ na prāpnoti/ \N2
%atha vā svāmana etata sthāne nu rāgaṃ/rāge? na prāpnoti/ \U1
%------------------------------
%And also his mind does does not suffer the state of attachment. ?! %überdenken 
%------------------------------
%anyad  rājayogasya cihnaṃ kathyate \E
%anyata rājayogasya cinhaṃ kathyate// \L
%anyata rājayogasya cinhaṃ kathyate// \B
%anyat  rājayogasya cinhaṃ kathyate// \N1 yasyecchā bhavati??? taṃ taṃ bhogaṃ prāpnoti/ atha vāsya mana eva sthāne 'nu rāgaṃ na prāpnoti/ anyat rājayogasya cinhaṃ kathate//
%anyat  rājayogasya cihuṃ  kathyate// \D1
%anyad  rājayogasya ciṃhuṃ kathyate// \U2
%anyat  rājayogacinhaṃ     kathyate/ \U1
%------------------------------
%[Now] another attribute of Rājayoga is described. 
%------------------------------
%yasya rājyādilābhe 'pi   phalalābho na bhavati/ \E
%yasya rājādilābhetya phalalābho na bhavatī \L
%yasya rājādilābhetya phalalābho na bhavatī \B
%yasya rājyādilābhe  pi   phalalābho ba bhavati/ \N1
%yasya rājyādilābhe  pi   phalalābho na bhavati// \D1
%yasya rājyādilobhe  pi ca  phalalābho na bhavati// \N2
%yasya rājyādilābe  'pi ca palalābho na bhavati/ \U1
%------------------------------
%The attainment of the result does not arise even
%------------------------------
%hānāv api manomadhye duḥkhaṃ na bhavati/ \E
%hananād pi mānomadhye duḥkahṃ na bhavatī/ \L
%hananād pi mānomadhye duḥkahṃ na bhavatī/ \B
%hānāv api manomadhye duḥkhaṃ na bhavati/ \U1
%hānāv api manomadhye duḥkhaṃ na bhavati/ \N1 %emend to hānau loc. sg. of hāni -> abandonment
%hānāv api manomadhye duḥkhaṃ na bhavati// \D1
%hānāv  pi manomadhye duḥkhaṃ na bhavati// \N2
%------------------------------
%While cessation in the middle of the mind suffering does not arise. 
%------------------------------
%atha ca tṛṣṇā na bhavati/ \E
%atha ca tṛṣṇā na bhavati/ \L
%atha ca tṛṣṇā na bhavatī/ \B
%atha ca tṛṣṇā na bhavati/ \U1
%atha ca tṛṣṇā na bhavati/ \N1
%atha ca tṛṣṇā na bhavati  \D1
%atha ca tṛṣṇā na bhavati/ \N2
%------------------------------
%And then also desire does not arise. 
%------------------------------
%atha ca kasmin     padārthasyoparyanicchā na bhavati/ \E
%atha ca kasminna   padārtho prāpte kasyāpi padārthasyopari ānīcha na bhavati//   \L
%atha ca kasminna   padārthau prāpte kasyāpi padārthāsyopari ānīchā ni bhavati//   \B
%atha ca kasminnpi  padārthe prāptakasyāpi padārthasya upari anusthāna bhavaṃti// \N1 %emend to anuṣṭhāna?! n. performance, religious practice
%atha ca kasminnapi padārthe prāpte kasyāpi padārthasya upari anichā    bhavaṃti  \D1
%atha ca kasminnpi  padārthe prāpte kasyāpi padārthasya upari anisthāna bhavati// \N2
%atha ca kasminnpo  padārthe prātpe kasyāpi padārthasya upari anuṣṭāna bhavati//  \U1
%atha ca kasmin adhipadārtha prāpte kābhyādi padārther pari aniccha na bhavati//  \U2
%------------------------------
%And then some regarding the object/in the object which has been attained by him of the object power arises upon him???!?! -> weird translation recheck    
%------------------------------
%kasmin padārthe manasonurāgo na bhavati/    \E
%asminn padārthe manasonurāgo na bhavatī/    \L
%asminn padārthe manasonurāgo na bhavatī/    \B
%asminnapi padārthe manasaḥ anurāgo bhavati// \D1
%asminnapi padārthe manasaḥ anurāgo bhavati/ \N2
%kasminnpi padārthe anurāgo na bhavati//     \U2
%------------------------------
%He in whom the Object is affection does not arise. 
%------------------------------
%ayam api rājayogaḥ kathyate/ \E
%atha samarājayogaḥ kathyate/ \L
%ayam api rājayogaḥ kathyate/ \B
%ayam api rājayogaḥ kathyate// \D1
%ayam api rājayoga kathyate// \N2
%ayam api rājayoga kathyate// \U2
%------------------------------
%Now a similar Rājayoga is explained. 
%------------------------------
%athacaḥ yasya manaḥ munividvatpuruṣeṣu maitre ca samaṃ bhavati/ \E
%atha ca yasya manaḥ bhunividvat puruṣe maitreśatrau ca samaṃ bhavati/ \L
%atha ca yasya manaḥ śrunividvat puruṣe maitreśatro ca samaṃ bhavatī/ \B
%atha ca yamanaḥ śrutividyutpuruṣe mitre śatrau ca samaṃ bhavati// \D1
%atha ca yasya manaḥ śrutividyutpuruṣe  mitre śatrau ca samaṃ bhavati/ \N2
%atha ca yasya manaḥ śuciviśuddhapuruṣe mitre śatrau ca samaṃ bhavati// \U2
%------------------------------
%And then his mind in the person of perfect purity is equal towards friend and enemy. %something like this \U2 
%------------------------------
%dṛṣṭiś ca samā bhavati/   \E
% \om                      \L
% \om                      \B
%dṛṣṭiś ca samā bhavati//  \D1
%dṛṣṭiś ca samā bhavati//  \N2
%dṛṣṭiś ca samā bhavati/   \U2
%------------------------------
%And a similar gaze arises. 
%------------------------------
%sakalapṛthvīmadhye gamanavataḥ       sukhabhogavataḥ      yasya manasi karttṛtvābhimāno   nāsti/ \E
%sakalapṛtvīmadhye  gamanāgamanataḥ   sukhabogho bhavataḥ  yasya manasi kartu tvābhimano   nāsti/ \L
%sakalapṛthvīmadhye  gamanāgamanataḥ   sukhabogho bhavataḥ  yasya manasi kartutvābhimano   nāsti// \B
%sakalapṛthvīmadhye gamanaṃvataḥ//    sukhabhogavataḥ      yasya manasi kartṛtvādyabhimāno nāsti// \D1
%sakalapṛthvīmadhye gamavataḥ         sukhabhogavataḥ      yasya manasi kartṛtvādyabhimāno nāsti// \N2
%sakalapṛthvīmadhye gamanāgamanavat// sukhabhogavat        yasya manasi kartṛtvābhimāno    nāsti// \U2
%------------------------------
%Being situated anywhere on earth he is furnished with motion. In the mind of him who is possesing enjoyment and happiness there does not arise the desire for Täterschaft.
%------------------------------
%atha ca lokamadhye gamanavataḥ sukhabhogavataḥ yasya manasi karttṛtvābhimāno nāsti/atha ca lokamadhye kartṛtvaṃ na jñāpayati/ \E
%anucaralokamadhya kartṛtvābhi mano nāsti \L
%anucaralokamadhyakartṛtvābhimano nāsti// \B
%anucalokamadhye kartṛtvaṃ na jñātvā payati/ \D1
%anucalokamadhye kartṛtvaṃ na jñāpayati/ \N2
%anucalokamadhye kartṛtvaṃ na jñāpayati \U2
%
%------------------------------
%
%------------------------------
%sopi rājayogaḥ kathyate// \E
%so pi rājayoga kathyate/ \L
%so pi rājayoga kathyate/ \B
%so pi rājayoga kathyate//  \D1
%so pi rājayoga kathyate  \U2
%so 'pi rājayoga kathyate// N2
%------------------------------
%This is also said to be Rājayoga. 
%------------------------------
%navīnāni paṭṭasūtramayadhṛtāni vastrāṇi \E
%navinīnīśpī paṭṭasūtramayāni dhṛtāni vastrāṇi// \L
%navinīnīr api paṭṭasūtramayāni dhṛtāni vastrāṇi// \B
%navīnāni paṭasūtramayāni dhṛtāni vastrāṇi// D1
%navīnāni paṭasūtramayāni dhṛtāni vastrāṇi/ N2
%
%------------------------------
%Neue, aus Seidenfaden gemachte, getragene Kleider.?! 
%------------------------------
%atha vā jīrṇāni chidrāṇi dhṛtāni kastūrīcandanalepairvā kardamalepena yasya manasi harṣaśokau na staḥ/ \E
%atha vā jīrṇāni svachidrāṇi dhṛtāni kasturīcaṃdanalepocākardamalepovā yasya manasi harṣaśokau na sthaḥ// \L
%atha vā jīrṇāni svachidrāṇi dhṛtāni kastūrīcaṃdanalepo vā kardamalepo vā yasya manasi harṣaśokau na sthaḥ// \B
%atha vā jīrṇāni sacchidrāṇi dhṛtāni// kas turikā caṃdanalepo vā / kardamalepo vā yasya manasi harṣaśoko na sthaḥ  \D1
%atha vā jīrṇāni sacchidrāṇi dhṛtāni // kas turikā caṃdanalepo vā / kardamalepo vā yasya manasi harṣaśoka na sthāḥ \N2
%------------------------------
%
%------------------------------
%sa evātra tiṣṭhati/ \E
%sa eva rājayogaḥ// idānīṃ// \L
%sa eva rājayogaḥ// idānīṃ// \B
%sa eva rājayogaḥ// \D1
%sa eva rājayogaḥ// \N2
%------------------------------ 
%yasya janmamaraṇe na staḥ sukhaṃ na bhavati/ kulaṃ na bhavati śīlaṃ na bhavati/ sthānaṃ na bhavati/ \E
%\om D1
%\om \L
%\om \B
%\om \N2
%------------------------------
%
%----------------------------
%rājayogaḥ naramadhye atha ca vanamadhye yuddhe saṃgrāma madhye vā yasya manaḥ bhayapūrṇaṃ vā na bhavati/ sopi rājayogaḥ kathyate// \E
%nagaramadhye tha ca vanamadhye %udvasta?! %cant read% grāmamadhye vā yasya manaḥ unaṃ pūrṇaṃ vā na bhavati so pi rājayogaḥ// \L
%nagaramadhye (')tha ca vanamadhye udvasta %cant read% grāmaṃmadhye vā yasya manaḥ unaṃ pūrṇaṃ vā na bhavatī so pi rājayogaḥ// \B
%ṣagaramadhye atha ca vanamadhye//udve???!sūgrāmamadhye svetapūrṇagrāmamadhye vā yasya manaḥ ū???nannapūrṇaṃ vā na bhavati / so pi rājayogaḥ// \D1
%nagaramadhye atha ca vanamadhye//udve???!sūgrāmamadhye svetapūrṇagrāmamadhye vā yasya manaḥ ū???nannapūrṇaṃ vā na bhavati / so pi rājayogaḥ// \N2
%------------------------------
%
%----------------------------
%idānīṃ yogaḥ kathyate/ \E
%idānīṃ caryāyogaḥ kathyate// \L
%idānīṃ caryāyogaḥ kathyate// \B
%idānīṃ caryāyogaḥ kathyate// \D1 [S.7, Z.7]
%idānīṃ caryāyoga kathyate// \N2
%------------------------------
%Now caryāyogaḥ is explained.
%----------------------------
%nirākāro      nityo'bhedyaḥ  sa etādṛśaḥ ātmani     mano yasya niścalaṃ tiṣṭhati/ \E
%nirākāro calo nityo bhedhyaḥ sa etādṛa   ātmā sa etādṛśe ātmani niścala tiṣṭhati/ \L     %daṇḍa nach ātmā besser -> emend? oder in weiteren Hss?
%nirākāro calo nityo bhedhyaḥ sa etādṛa   ātmā sa etādṛśye ātmani niścalaṃ tiṣṭhati/ \B
%nirākālo nityo calo 'bhedhyaḥ// sa etādṛśaḥ ātmā etādṛśe ātmani manaḥ yasya niścalaṃ tiṣṭhati \D1
%nirākālo nityo calo 'bhedhyaḥ sa etādṛśaḥ ātmā etādṛśa ātmani manaḥ yasya niścala tiṣṭhati/ \N2
%------------------------------
%The is self as such: shapeless, (em zu acala) unchangeable, permanent [and] unsplitable. Unmovable it is situated in such a person. 
%------------------------------
%tasyātmanaḥ puṇyapāpasparśo na bhavati/ \E
%tasyātmanaḥ puṇyapāpasparśo na bhavati/ \L
%tasyātmanaḥ puṇyapāpasparśo na bhavatī/ \B
%tasyātmanaḥ punyapāpasparśo na bhavati// \D1
%tasyātmanaḥ puṇyapāpasparśo na bhavati/ \N2
%------------------------------
%His self is not touched by sin and merit. 
%------------------------------
%udakamadhye sthitasya padmapatre      yatho dakasya sparśo    bhavati/ \E
%udakamadhye sthitasya padmanī patrasya yatho dakasya sparśo na bhavati/ \L
%udakamadhye sthitasya padmanī patrasya yatho dakasya sparśā na bhavatī/ \B
%udakamadhye sthitasya padminī patrasya yathā udakasparśo na bhavati// \D1
%udakamadhye sthitasya padminī patrasya yathā udakasparśo na bhavati/ \N2
%------------------------------
%Just as the heap of lotusses-leaves situated in the middle of the water doesn't touch the water;  
%------------------------------
%tathaivātmani yathā kāśamadhye pavanaḥ svecchayā bhramati/ \E
%tathaivātmani yathā ākāśamadhye pavanaḥ svechayā bhramati/ \L
%tathaivātmani yathā ākāśamadhye pavanaḥ svechayā bhramatī/ \B
%tathaivātmani yathā ākāśamadhye pavanasvachayā bhramati \D1
%tathaivātmani yathā ākāśamadhye pavanasvachayā bhramati/ \N2
%------------------------------
%likewise the self [is not touched]. Just as the wind wanders according to its own will in space,...  
%------------------------------
%tathā yasya manaḥ nirākāramadhye līnaṃ bhavati/ sa eva caryāyogaḥ// \E
%tathā yasya manaḥ nirākāramadhye līnaṃ bhavati sa eva caryāyogaḥ// \L
%tathā yasya manaḥ nirākāramadhye līnaṃ bhavatī sa eva caryāyogaḥ// \B
%tathā yasya manaḥ nirākāramadhye līnaṃ bhavati/ sa eva kriyāyogaḥ// \D1 ?!?!!?!? obvious mistake! stemma point?! 
%tathā pavananirākāramadhye līnaṃ bhavati/ sa eva kriyāyogaḥ// \N2
%------------------------------
%Likewise is the mind of whom is absorbed into the universal spirit [wanders according to its own will in space]. This is \textit{\caryāyoga}.  
%------------------------------
%[p.23]
%idānīṃ grahayogaḥ kathyate/ \E
%idānīṃ haṭhayogaḥ kathyate/ \L
%idānīṃ haṭayoga kathyate/ \B
%idānīṃ haṭhayogaḥ kathyate/ \D1
%idānīṃ haṭhayoga kathyate// \N2
%------------------------------
%Now \textit{haṭhayoga} is explained. 
%------------------------------
%recakapūrakakumbhaka ityādi prakāreṇa pavanasādhanaṃ kartavyam/ \E
%recakapūrakakumbhaka ityādi prakāreṇa pavanasya sādhanaṃ kartavyam// \L
%recakapūrakakuṃbhaka ityādi  prakāreṇa// pavanasya sādhanaṃ kartavyam \B
%recakapūrakakuṃbhaka ityādi  prakāreṇa pavanasya sādhanaṃ kartavyaṃ// \D1
%recakapūrakakuṃbhaka ityādhi prakāreṇa pavanasya sādhanaṃ kartavyaṃ// \N2
%------------------------------
%The practice shall be done by the mode of breath: "Exhalation, Inhalation [and] Retention etc.
%------------------------------
%atha ca dhautyādiṣaṭkarmakāraṇāt śarīrasya śuddhir bhavati/ \E
%atha ca dhautyādiṣaṭkarmakāraṇāt// śarīrasya śuddhir bhavati \L
%atha ca dhotyādiṣaṭkarmakaraṇāt// śarīrasya śuddhir bhavatī \B
%atha ca dhautyādiṣaṭkarmakaraṇāt   śarīrasya śuddhir bhavati// \D1
%atha ca dhautyādiṣaṭkarmakaraṇāt// śarīrasya śuddhir bhavati// \N2
%%------------------------------
%And then due to the six practices(\textit{ṣaṭkarma}) the purification of the body arises. 
%------------------------------
%sūryanāḍīmadhye pavanaḥ pūrṇo yadā tiṣṭati/ \E %!
%sūryanāḍīmadhye pavanapūrṇo yadāti/ \L
%sarvasūryanāḍīmadhye pavanapūrṇo yadāti/ \B
%sūryanāḍīmadhye pavanaḥ pūrṇo yadā tiṣṭhati \D1
%sūryanāḍīmadhye pvanaḥ pūrṇo yadā tiṣṭhati/ \N2
%
%------------------------------
%When the full breath abides in the middle of the sun-channel.  
%------------------------------
%tadā mano niścalaṃ bhavati/ \E
%tadā mano niścalo bhavati/ \L
%tadā mano niścalo bhavatī// \B
%tadā manaḥ niścalaṃ bhavati \D1
%tadā manaḥ niścalaṃ bhavati \N2
%------------------------------
%Then the mind is unmovable. 
%------------------------------
%manaso niścalatvena ānandarūpaṃ pratyakṣaṃ bhāsate/ haṭhayogakāraṇāt manaḥ śūnyamadhye līnaṃ bhavati/ kālaḥ samīpe nāgacchati/ \E
%manaso niścalatve ānandaṃ svarūpaṃ pratyakṣaṃ bhāsate/ haṭhayogakāraṇāt manaḥ śūnyamadhye līnaṃ bhavati/ kālaḥ samīpe nāgacchati// \L
%manaso niścalatve ānaṃdaṃ svarūpaṃ pratyakṣaṃ bhāsate// haṭayogākāraṇāt manaḥ// śūnyamadhye līnaṃ bhavatī/ kālāsamīpe nāma gacchati// \B
%manasaḥ niścalatve ānaṃdasvarūpaṃ pratyakṣaṃ bhāsate/ haṭhayogakaraṇāt manaḥ śūnyamadhye līnaṃ bhavati// kālaḥ samīpe nāgachaṃti// \D1
%manasaḥ niścalatve ānaṃdasvarūpaṃ pratyakṣaṃ bhāṣate/ haṭhayogakaraṇāt manaśūnyamadhye līnaṃ bhavati/ kālasamīpe nāgachati// \N2
%------------------------------
%The form of bliss immediately shines through the motionless mind. Due to the execution of haṭhayoga the mind becomes absorbed into emptiness. The time of death does not approach.  
%------------------------------
%idānīṃ haṭhayogasya dvitīyo bhedaḥ kathyate/ \E
%idānīṃ haṭhayogasya dvitīyabhedāḥ kathyante/ \L
%idānīṃ haṭayogasya dvitīyaṃ bhedāḥ kathyaṃte// \B
%idānīṃ haṭhayogasya dvitīyabhedaḥ kathyate \D1
%idānīṃ haṭhayogasya dvitīyabhedaḥ kathyate/ \U2 
%------------------------------
%Now, the second division of haṭhayoga is explained. 
%------------------------------
%pādādārabhya śiraḥparyaṃtaṃ svaśarīre koṭisūryatejaḥ   samānaṃ śvetaṃ pītaṃ raktaṃ kiṃcidvarṇaṃ ciṃtyate/ \E
%pādādārabhya śiraparyaṃtaṃ svaśarīre koṭisūryatejaḥ    samānaśvetaṃ nīlaṃ raktaṃ tiṃdrupaṃ ciṃtate/ \L
%pādādārabhya śiraparyaṃtaṃ svaśarīre koṭisūryatejaḥ//  samānaśvetanīlaṃ raktaṃ kiṃdrupaṃ ciṃtate// \B
%pādādārabhyā śiraḥ paryaṃtaṃ svaśarīre koṭisūryatejaḥ  samānaṃ śvetaṃ pītaṃ nīlaṃ raktaṃ kiṃcidrūpaṃ ciṃtyate \D1
%pādādārabhya śiraḥ paryaṃtaṃ svaśarīre koṭisūryyatejaḥ samānaśvetaṃ pītaṃ nīlaṃ raktaṃ kiṃcidrūpaṃ? ciṃtyate \U2
%------------------------------
%The shine of ten million suns in one's own body beginning from the feet to the top of head is contemplated in any color equal to white, yellow [or] red.
%------------------------------
%ttad dhyānakāraṇāt sakalaṃ rogajvalanaṃ bhavati/ āyur vardhate/ \E
%tad dhyānakāraṇāt sakalaṃ gerogajvalanaṃ na bhavati/ āyur vardhate/ \L
%tat dhyānakāraṇāt sakalaṃ gerogajvalanaṃ na bhavati/ āyur vṛddhir bhavatī/ \B
%na/ta dhyānaṃkaraṇāt// sakalāṃ geroga jvalanaṃ na bhavati//\D1
%tad-dhyānaṃkaraṇāt / sakalāṃ geroga na bhavati jvaranaṃ na bhavati āyuvṛddhir bhavati// \N2
%------------------------------
%Due to the execution of meditation on that all types of disease does'nt arise. Vitality grows. %recheck translation. 
%------------------------------
%idānīṃ jñānayogasya lakṣaṇaṃ kathyate/ \E
%idānīṃ jñānayogasya lakṣaṇaṃ// \L
%idānīṃ jñānayogasya lakṣaṇaṃ// \B
%idānīṃ jñānayogasya lakṣaṇaṃ// \D1
%idānīṃ jñānayogasya lakṣaṇaṃ kathyate// \N2
%------------------------------
%Now the characteristic of jñānayoga is explained. 
%---------------------------------------------------------------------------
%---------------------------------------------------------------------------
%---------------------------------------------------------------------------
%SVARODAYA PT
%ekameva jagat paśyedviśvātmā viśvabhāvanaḥ |
%iti kṛtvā tu vai yukto jñānayogaṃ samācaret |
%yatra tatra sthito vāpi sarvajñānamayaṃ jagat |
%ya evamasti bodhena so'pi jñānādhikāravān |                                 %this prooves that the tattvabinduyoga is derived from the Yogasvarodaya!!! 
%prāpnoti śāmbhavīmantrān sadā nityaparāyaṇaḥ |
%yathā nyagrodhavījaṃ hi kṣitau vapturdrumāyate | ādāvekastato'nekaḥ
%---------------------------------------------------------------------------
%---------------------------------------------------------------------------
%----------------------------------------------------------------------------
%ekam eva jagat paśyed viśvāvasu vibhāsvaram/       
%avikalpatayā yuktyā jñānayogaṃ samācaret//1// \E
%
%ekam evājagat paśyed viśvātmāsu vibhāsvaraṃ//
%avikalpatayā yuktā jñānayogaṃ samācaret// \L
%
%ekam evājagat paśyad visvātmāsu vibhāsvaraṃ//
%avikalpatayā yuktā jñānayogaṃ samācaret// \B
%
%ekameva jagat paśyedviśvātmā viśvabhāvanaḥ/
%iti kṛtvā tu vai yukto jñānayogaṃ samācaret// SVARODAYA

%ekameva jagat paśyeddviśvātmāsu vibhāsvaraṃ//
%avikalpatayā yuktyā jñānayogaṃ samācaret//1// \D1

%ekameva jagat paśyeddviśvātmāsu vibhāsvaraṃ//
%avikalpatayā yuktyā jñānayogaṃ samācaret//1// \N2
%------------------------------
%He shall see the world truly as being one, shining in all selves. 
%By applying indistinctness he shall accomplish \textit{jñānayoga}.   
%------------------------------
%
%yatra yatra sthito vāpi sarvajñānamayaṃ jagat/ 
%sa evaṃ vetti bodhena so pi jñānādhikāraṇāt//2// \E 
%
%yatra yatra sthito vāpi sarvajñānamayaṃ jagat//  
%ya evaṃ vetti bodhena so pi jñānādhikāravān// \L
%
%yatra yatra sthito vāpi sarvajñānamayaṃ jagat//  
%ya evaṃ ve bodhena so pi jñānādhikāravān// \B
%
%yatra tatra sthito vāpi sarvajñānamayaṃ jagat/
%ya evam asti bodhena so'pi jñānādhikāravān/ \SVARODAYA
%
%yatra yatra sthito vāpi sarvajñānamayaṃ jagat//
%ya evaṃ vetti bodhena so pi jñānādhikāravān//2//\D1
%
%yatra yatra sthito vāpi sarvajñānamayaṃ jagat//
%ya evaṃ vetti bodhena so pi jñānādhikāravān//2//\N2
%------------------------------
%Wherever the world is established in the resemblence of omniscience,
%He who knows such by means of insight he is a like expert of knowledge.    %revise translation    
%------------------------------
%
%\om!!!!!                                       \E
%
%prāpnoti śāmbhavīmantrān sadā nityaparāyaṇaḥ/
%yathā nyagrodhavījaṃ hi kṣitau vaptur drumāyate/ \SVARODAYA  %emend to bīja 
%
%prāpnoti śāmbhavīṃ sattān sadā dvaitaparāyaṇaḥ//
%yathā nyagrodhavīja hi kṣitāv utpadyate yathā// \L
%
%prāpnoti śāmbhaviṃ sattāṃ sadādvaitaparāyaṇaḥ//
%yathā nyagrodhabījāṃ hi kṣitī utpadyate// \B
%
%prāproti sāṃbhavīsattāṃ sadādvaitaparāyaṇaḥ//
%yathā nyagrodhavījaṃ hi kṣitāvuptaṃ drumāyate// \D1
%
%prāproti sāṃbhavī sattā sadādvaitaparāyaṇaḥ//
%yathā nyagrodhavījaṃ hi kṣitāvuptaṃ drumāyate// \N2 %drumaayate=denom. wie ein beim  sein 
%
%------------------------------
%He always attains the śāmbhavī state - the innate highest goal. 
%Just as the seed of the Nyagrodha scattered onto the soil [always] becomes a tree. 
%------------------------------
%ekāntaṃ naikadā svena dṛśyate daśadhā kṛtaḥ/
%mūlāṅkurasya coddaṇḍāḥ śākhākuṇḍalapallavāḥ//3// \E
%
%ekāṃte nekadhā svena dṛśyaṃte daśadhāt kṛp?tā/
%mūlāṃ kurutva kudaṃjaḥ? śākhākilekālapallavā \B
%
%ekāṃtaṃ naikadhā śvetana dṛśyate daśadhā kṛtā//
%mūlāṃ kurutva codarāṭaḥ śālavākumapadṛtravā//4// \D1
%
%ekāṃtaṃ naikadhāśvetanadṛśyet śadhākṛtā//
%mūlāṃkurutva codarāṭaḥ śākhākumbhala pallavā//4// \N2
%------------------------------
%Nur eines, nicht zusammen mit dem Ich wird das zehnfach gemachte gesehen. 
%Die aufgerollten Sprossen am Ast, welche die austreibendem Stöcke sind vom Spross der Wurzel. 
%------------------------------
%srehapuṇyaphalaṃ bīje vistaro yaṃ svabhāvataḥ/ %%%%%%srehapuṇyaphalaṃ = die reine Frucht des männlichen Samen  
%tathāsau nirmalo nityo nirvikāro niraṃjanaḥ//4// \E
%
%snehe puṣpaphalabījavistāro yasva bhāvatāḥ       %%%%snehapuṣpaphalaṃ = Frucht der Liebesblüte   
%yāthāsau nirmalo nityo nirvikāro niraṃjanaḥ//    \B
%
%snehapuṣpaphalaṃ bīje vistārā yasya bhāvataḥ//        %%%%snehapuṣpaphalaṃ = Frucht der Liebesblüte   
%tathāsau nirmalo nityo nirvikāro niraṃjanaḥ//5// \D1
%
%snehapuṣpaphalaṃvīje vistāro yaṃ svabhāvataḥ//
%tathāsau nirmalo nityo nirvikāro niraṃjanaḥ//5// \N2
%------------------------------
%Aufgrund der eigenen Natur ist diese Menge, welche die Frucht der Blüte der Liebe ist, im Samen.
%Gewiss, ist jenes rein, ewig, unveränderlich und makellos. 
%------------------------------
%eko nekaḥ svayaṃbhūśca dhāmnā ca bahudhā sthitaḥ/ 
%paṃcatattvamanobuddhimāyāhaṃkāravikriyāḥ//5//   \E
%
%eko nekaḥ svayaṃbhaśca svadhā...ṣ bahudhā sthitāḥ//
%paṃcatatvamanobuddhirmāyāhaṃkāravikriyā//6// \D1
%
%eko neka svayaṃbhūśca dhāmnāya bahudhā sthitaḥ//
%paṃcatatvamanobuddhimāyāhaṃkāravikriyā// \B
%
%eko neka svayaṃbhūśca svadhāmnāva bahudhā sthitaḥ//
%paṃcatatvamanobuddhirmāyāhaṃkāravikriyā//6// \N2
%------------------------------
%Eins, nicht eins und aus sich selbst heraus seiend mittels dieser Macht vielfach sich befindet.
%Fünf So-Heiten, welche da sind Manas, Buddhi, Māya, Ahaṃkāra und Modifikation (Umgestaltung). 
%------------------------------
%evaṃ daśavidhaṃ viśvaṃ lokālokasavistaram/
%eka eva na cānyo sti yo jānāti sa tattvavit//6// \E
%
%evaṃ daśavidhaṃ viśvaṃ lokālokasavistaraṃ//
%eka eva na cānyo sti yojānāti sa tatvavit//7// \D1
%
%evaṃ daśavidhāviśvaṃ lokālokasavistaraṃ//
%eka eva na cānyā sti yojānāti sa tatvavit// \B
%
%evaṃ daśavidhā viśvaṃ lokālokasavistaraṃ//
%eka eva na cānyo sti yo jānāti sa tatvavit//7// \N2
%------------------------------
%Auf diese Weise die zehn Variationen überall die Gesamtheit der Welt und der Nicht-Welt
%Nur eines ist und nicht etwas anderes, wer das weiß ist ein Tattvakenner.  
%------------------------------
%pṛthvīvanaspatiparvatādisthārarūpaḥ              saṃsāramanuṣyahastyaśvapakṣītyādiko jaṃgamarūpaḥ saṃsāraḥ// \E
%pṛthvīvanaspatī parvatādisthāvararūpā saṃsāraḥ/ manuṣyahasteśvapakṣītyādiko jaṃgamarūpaḥ saṃsāraḥ// \B
%pṛthvīvanaspatī parvato tyādisthāṃvararūpaḥ saṃsāraḥ manuṣyahastī aśvapakṣītyādiko jaṃgamaḥ rūpaḥ saṃsāraḥ// \D1
%pṛthvīvanaspat/o/i/e/parvate 'thyādisthāvararūpa saṃsāraḥ manuṣyahastipakṣītyādiko jaṃgamarūpaḥ saṃsāraḥ// \N2
%------------------------------
%
%------------------------------
%atha ca yo dṛṣṭiviṣayaḥ sa dṛśya ucyate/ yo dṛṣṭyā na vīkṣyate sa adṛśya ityucyate/ \E
%atha ca// yodaṣṭiviṣayaḥ sa dṛśya ucyate// yo dṛṣṭyā na vīkṣyate sa adṛśyatyucyate// \B
%atha vā yadārṣṭiviṣayaḥ sa dṛśya ucyate/ yo dṛṣṭyā na vīkṣyate sa adṛśya ityucyate// \D1
%atha ca ya drṣṭiviṣayaḥ sa dṛśya ucyate/ yo dyā na vīkṣyate sa adṛśya śatyucyate/ \N2
%------------------------------
%
%------------------------------
%evaṃ saṃsārasya svātmano bhedaṃ dūrīkṛtyaikamevadarśanaṃ sa eva jñānayogaḥ/ \E
%evaṃ saṃsārasya svātmano  bhedadūrīkṛtya aikyo na darśanaṃ jñānayogaḥ/ \B
%evaṃ saṃsārasya svātmanaḥ bhedāṃ dūrīkṛtya ekyena darśanaṃ jñānayogaḥ/ \D1
%evaṃ saṃsārasya svātmanaḥ bhedaṃ dūrīkṛtya ekena darśanaṃ jñānayogaḥ/ \N2
%------------------------------
%
%------------------------------
%tasya kāraṇāt kālaḥ śarīranāśaṃ na karoti/ \E
%tasya karaṇāt kālaḥ śarīranāśaṃ na karoti// \B
%tasya karaṇāt kālaḥ śarīranāśaṃ na karoti// \D1
%tasya karaṇāt kālaḥ śarīranāśaṃ karoti/ \N2
%------------------------------
%
%------------------------------
%idānīṃ tasyabhedaḥ kathyate/ yathā vaṭabījam/ \E
%idānī  svābhāvabhedaḥ kathyate// yathāvaṭabīje \B
%idānīṃ svabhāvabhedaṃ kathyate// yathāvaṭabījaṃ \D1
%idānīṃ svabhāva bheda kathyate// yathāvathabījaṃ \N2
%------------------------------
%
%------------------------------
%vaṭarūpeṇa pariṇataṃ sat daśadhā bhedaṃ [P.27] svabhāvata eva prāpnoti/ \E
%narūpeṇa pariṇamate/ śatadaśadhā bhedaṃ svābhāva eva prāpnotī// \B
%vaṭarūpeṇa pariṇataṃ/ sadaśadhā bhedaṃ svabhāvata eva prāpnoti// \D1
%vaṭarūpeṇa pariṇataṃ/ satudaśadhā bhedaṃ svabhāvata eva prāpnoti/ \N2
%------------------------------
%
%------------------------------
%mūlāṃ kuratvag daṇḍaśākhākalikāpallavapuṣpaphalasnehā iti daśa bhedān prāpnoti// \E
%mūlaṃ aṃkuratvak daṃdaśākhā kilakālapallavā// vistāroyaṃ svābhāvataḥ iti daśabhedān prāpnoti// \B DSCN7160 Z. 4
%mūlāṃ kuratvak daṇdaśākhāṃ kalikāpallavapuṣpaphalasnehaṃ iti bhedo daśadhā prāpnoti// \D1
%mūlāṃ kuratvak daṇdaśākhāṃ kalikāpallavapuṣpaphalasneha/ iti bhedo daśadhā prāpnoti// \N2
%------------------------------
%
%------------------------------
%yathā nirmalo nirvikāraḥ niraṃjana eka etādṛśa ātmā svabhāvādeva/ \E
%tathā nirmalo nirvikāraḥ niraṃjanaḥ eka etādṛśa ātmasvabhāvādeva... \B
%tathā nirmalaḥ nirvikāraḥ niraṃjanaḥ ekaetādṛśaḥ ātmasvabhāvādeva... \D1
%tathā nirmalaḥ nirvikāraḥ niraṃjanaḥ ekaḥ etādṛśaḥ ātmasvabhāvādeva... \N2
%------------------------------
%
%------------------------------
%pṛthivyaptejovāyvākāśamanobuddhimāyāvikārarūpabhedān prāpnoti/ \E
%pṛthvyā patejovādvyākāśamanobuddhimāyāvikārarūpabhedāna prāpnoti// \B
%pṛthtivya ete jīvā??[S.9, Z.4 hinten]kāśamanobuddhir māyāvikārarūpabhedān prāpnoti/ \D1
%pṛthvīpate/ jīvīkāśamanobuddhir māyāvikārarūpabhedāt prāpnoti/ \N2  
%------------------------------
%
%------------------------------
%jñānayogaprabhāvādeka eva ātmā iti niścayo bhavati// \E
%jñānayogaḥ// prabhāvād eka eka ātmā iti niścayā bhavatī// \B
%jñānayogaprabhāvāt eka eva ātmā iti niścayo bhavati// \D1
%jñānayogaprabhāvāt eka eva ātmā iti niścayo bhavati// \N2
%------------------------------
%
%------------------------------
%yathaikaiva pṛthvī kvacitkomalarūpā kvacitparimalarūpara[p.28]hitā kvacitsuvarṇarūpā                                     kvacidraupyarūpā kvacidratnamayīkvacicca śvetā kvacidratnamayī kvacicca śvetā kvacidraktā kvacitpītā kvacitkarburā kvacinnānāvidharūpā kvacidviṣarūpā kvacidamṛtarūpamayī svabhāvata eva bhavati// \E
%
%
%yathā ekaika pṛthvī kvacitkomalarūpā// kvacitmano hararūpā// kvacitparimalarūpayuktā// kvacitparimalarohitā// kvacitsuvarṇarūpa kvacidratnamaī// kvacitśverūpā// kvacitkṛṣṇā//kvacidraktā/ kvacitpītā// kvacitkarburā// kvacinnānāvidhaphalarūpā kvacit viṣarūpā// kvacidamṛtamaī/ svabhāvata eva bhavataḥ// \B
%
%yathā ekaiva pṛthivī kvacitkomalarūpa kvacitmano harā// kvacitparimalarūpāyuktā/ kvacitparimalarohitā kvacitsuvarṇarūpa// kvacitrūpyarūpa// kvacidratnamayī// kvacitśvetā// kvacitkṛ....[S8., Z.7]kvacidrakta kvacitpītā// kvacitkarburā kvacinnānāvidhaphalarūpā// kvacidpuṣparūpā// kvacidamṛtamayī/ svabhāvata eva bhavati// \D1
%
%
%yathā ekaṃ ca pṛthivī kvacitkomalarūpa kvacitmano ha?rā kvacitparimalarūpāyuktaḥ/ kvacitparimalarohitā kvacitsuvarṇarūpā kvacitrūpyarūpa kvacidratnamayī kvacitśveta kvacitkṛ....[S6. verso]kvacidrakta kvacitpītā kvacitkarburā kvacinnānāvidhaphalarūpā kvacidamṛtamayī/ svabhāvata eva bhavati// \N2
%------------------------------
%
%------------------------------
%tathaivātmā manuṣyapakṣihariṇahastividyādharagandharvakinnaramahāpaṃḍitamahāmūrkharogyarogikrodhiśāṃtarūpaḥ svabhāvādeva bhavati/ \E
%tathaivātmā// manuṣyapakṣihariṇahastividyādharagaṃdharvakinnaramahāpiṃḍatamahāmūrkharogī krodhadhiśāṃtarūpaḥ svabhāvādeva bhavatī/ \B
%tathātmā// manuṣyapakṣi// hariṇahastīvidyādharagandharvakinnaramahāpaṃḍitamahāmūrvarogī arogīkrodhīśāṃtarūpasvabhāvādeva bhavati/ \D1
%tathātmā//  manuṣyapakṣihariṇahastividyādharagandharvakinnaramahāpaṇḍitamahāmūrkharogīarogīkrodhīśāṃtarūpasvabhāvādeva bhavati/ \N2
%------------------------------
%
%------------------------------
%jñānayogādhikārarūparahito jñāyate/ yathā plakṣasyotpattiḥ/ sthānam eva bhavati// \E
%jñānayogādhikārarūparahito jñāyate// yathā phalasyotpattisthānam ekameva bhavatī// \B
%jñānayogadhikārarūparahito jñāyate// yathā phalasyotpattisthānam ekaseva bhavati// \D1
%jñānayogadhikārarūparahito jñāyate// yathā phalasyotpattisthānam eva kameva bhavati// \N2 
%------------------------------
%
%------------------------------
%atha ca phalasya gatir bahudhā dṛśyate/ \E
%atha ca phalasya gatir bahudhā dṛśyate// \B
%atha ca phalasya gatir bahudhā dṛśyate// \D1
%atha ca phalasya gati bahudhā dṛśyate/ \N2
%------------------------------
%
%------------------------------
%ekaṃ phalaṃ pṛthvīmadhye patati/ śuklaṃ bhavati/ ekasya phalasya makaraṃdaṃ bhramaraḥ pibati/ ekasya phalasya mālāṃ kāminī tuṃgakucamaṃḍalopari dadhāti/ \E
%ekaṃ phalaṃ pṛthvīmadhye patiśuklaṃ bhavatī// ekasya karaṃdaṃ bhramaraṃ pibatī/ ekasya phalasya mālāṃ kāminītuṃ gakucamaṃḍalopari dadhātī// \B
%ekaṃ phalapṛthvīmadhye patati// śuklaṃ bhavati// ekasya phalasya makaraṃdabhramaraḥ pibati/ ekasya phalasya mālāṃ kāmibītuṃgakucamaṇḍalopari dadhāti// \D1
%eva phalapṛthvīmadhye patati śuklaṃ bhavati// ekasya phalasya makaraṃdaṃ bhramara pibati/ ekasya phalasyaṃ mālākāminī tuṃgakucamaṇḍalopari dadhovati// \N2
%------------------------------
%
%------------------------------
%ekaṃ phalaṃ mṛtamanuṣyopari kṣipyate/ ayaṃ vastunaḥ svabhāvaḥ/ tathā eka evātmā svīyabhāvādevāṣṭau bhogān bhunakti/ \E
%ekaṃ phalaṃ mṛtamanuṣyopari kṣapyate// ayaṃ vastunaḥ svabhāvaḥ/ tathā eka evātmā svabhāvādevāṣṭau bhogān bhunakte// \B
%ekaphalaṃ mṛtamanuṣyopari kṣipyate/ ayaṃ vastunaḥ svabhāvaḥ/ tathā ekaevātmā svīyabhāvādevāṣṭau bhogān bhunakti// \D1
%ekaphalaṃ mṛtamanuṣyopari kṣipyate/ ayaṃ castunaḥ svabhāvaḥ/ tathā ekaevātmā svīyabhāvādevāstau bhogāt bhunakti/ \U2
%------------------------------
%
%------------------------------
%ke teṣṭau bhogāḥ – suvāsaśca suvasrañca suśayyā sunitaṃbinī/ susthānañcānnapānāni aṣṭau bhogāśca dhīmatām/ padṛsūtramayāni vasrāṇi// \E
%aṣṭau bhogāḥ suvāsacca suvasaś ca suśayyāḥ sūnitaṃbinī/ susthānaśvānnapānānyaṣṭau bhogāścā sudhīmatām//1// paṭasūtrāmayāni vasrāṇi// \B
%ke teṣṭau bhogāḥ – suvāsaś ca// suśayyāsunitaṃ vinītavinī// susthātāścānmanaspāṣṭau bhogāḥ sudhiṣaṇapadṛsūtrayāni vasrāṇi   \D1
%ke teṣṭau bhogāḥ   suvāyaśca            suśayya sunitaṃbinī/ susthānāścānmanasyāṣṭau bhogāḥ sudhiyane padṛsūtrayāni vasrāṇi \N2
%------------------------------
%
%------------------------------
%paṃcasaptādṛ(??)likāyuktāni harmyāṇi teṣu vāsaḥ ativipulā mṛdutasukhā suśayyā/        padminī tārūṇyavatī manoharā guṇavatī tatropaviṣṭā kāṃtā/ sādhu āśanam/ atimūlyañca/ manoramamannaṃ। tathāvidhaṃ pānam/ \E
%paṃcasatyādāti kāyuktāni harmyāṇi teṣu vāstu ativipulā mṛdutara lāśayyā//2// padminī tārūnyavatī manoharā guṇavatī// tatrāpavistā kāṃtā 4 sādhu āsanaṃ 5 atiamūlyo asvaṃ//6 manoramyamattaṃ // 7 tathā vidhapānaṃ// 8 \B
%paṃcavāsaptavādṛāṃlikāyuktāni harmyāṇi          ativapulā mṛdu uttara chaṃdavatīśayyā/ padminī tārūnyavatī manoharā guṇavatī tatropavistā// sādhyāsanaṃ// amūlyo svaśca// manoramamattaṃ tathā vidhaṃ pānaṃ// \D1
%paṃcavāsaptavātālikāyuktāni harmyāni            ativipulā mṛdu uttara chaṃdavatīśayyā padminī tārūnyavatī manoharā guṇavatī tatropavistā sādhyāsanaṃ amūlyo svaśca manotamamannaṃ tathā vidhapānaṃ// \N2
%------------------------------
%
%------------------------------
%ete  ṣṭau bhogāḥ kathitāḥ/ eke duḥkhaṃ bhajante/ bhikṣāṃ yācante// \E
%ete ṣṭau bhogḥḥ// eka duḥkhā bhajaṃte/ bhikṣāyāṃ ca te ca// \B
%ete aṣṭau bhogā kathyaṃte// ete duḥkhaṃ bhajaṃte/ bhikṣyāṃ yācaṃte ca// \D1
%ete aṣṭau ghogā kathyate// ete duḥkhataṃ bhajate bhikṣāṃ yācate ca// \N2
%------------------------------
%
%------------------------------
%kiñca yathā sūryasya tejaḥ dugdhasya ghṛtam agnerjvalanaṃ viṣānmūrchā tilāttailam/ vṛkṣācchāyā/ phalātparimalaḥ kāṣṭhādagniḥ śarkarādibhyo madhuro rasaḥ/ \E
%      yathā sūryasye tejāḥ dugdhaghṛtḥ agne dvāhaḥ// viṣān mūrchāti lāttailaṃ// vṛkṣāchāyā phalāsarimalaḥ kaṣṭādagniḥ śarkadībhyo madhuro sahī mānībhyaḥ śītaḥ/ \B
%      yathā sūryasya tejaḥ// dugdhasya ghṛtaṃ// agne dadhiḥ viṣānmūrchāti lāttailaṃ// vṛkṣāchāyā// phalātpalātparimalaḥ// [S.9 Ende] kāṣṭhādāgniḥ śarkarādibhyo madhuro rasaḥ/ \D1
%      yathā sūryasya tejaḥ dusya     ghṛtaṃ agne dadhi viṣānmūrchāti latailaṃ      vṛkṣāchāyā   phalātparimalaḥ kāṣṭhādāgniḥ śarkarādibhyo madhuro rasaḥ/ \N2 
%------------------------------
%
%------------------------------
%himānībhyaḥ śītamityādipadārthānāṃ svabhāvaḥ tathā saṃsāro'pi parameśvarasvarūpamadhye tiṣṭhati/ \E
%ityādipadārthā svabhāvataḥ// eva tathā saṃsāro piparemesvara svarūpasya madhye tiṣṭhatī/ \B 
%himānitpaśaityāśatyādipadārthartha svabhāva eva// tathā saṃsāro pi parameśvarasvarūpamadhye tiṣṭhati \N2
%------------------------------
%
%------------------------------
%parameśvaro'khaṇḍaparipūrṇaḥ/ idānīṃ lakṣyaṃ kathyate/ \E
%parameśvaro khaṃdaparipūrṇaśca// idānīṃ ṣāhyalakṣa kathyate// \B
%parameśvaro yarāṇdaparipūrṇaśca// // idānīṃ bāhyalakṣaṇa kathyate/ \N2
%------------------------------
%
%------------------------------
%nāsāgrādārabhyāṃgulacatuṣṭayapramāṇaṃ nīlākāraṃ tejaḥ pūrṇamākāśaṃ lakṣyaṃ karttavyam/ \E
%nāsāgrādārabhyāṃgulacatuṣṭayaṃ pramāṇaṃ nilākāraṃjaḥ pūrṇam ākāśalakṣaṃ kartavyaṃ// \B
%nāsāgrādārabhyāṃgulacatuṣṭayapramāṇaṃ nirākāraṃ tejapūrṇamākāśalakṣaṇaṃ karttavyaṃ// \N2
%
%------------------------------
%
%------------------------------
%athavā nāsāgrādārabhyāṃgulacatuṣṭayapramāṇaṃ nīlākāraṃ tejaḥ pūrṇamākāśaṃ  lakṣyaṃ karttavyam/ \E
%athavā nāsāgrādārabhyaṣaḍaṃgulaṃ pramāṇaṃ ?bi?ṣi?īnāvarṇaṃ .. .. .. ..??.  lakṣyaṃ kartavyam/ \B
%\om                                                                                           \N2
%------------------------------
%
%------------------------------
%athavā nāsāgrādārabhya ṣaḍaṃgulapramāṇaṃ pavanatattvaṃ dhūmrākāraṃ lakṣyaṃ karttavyam// \E
%\om \B?
%athavā nāsāgrārabhyaṣadaṃgulapramāṇaṃ pavanatatvaṃ dhūmrākāraṃ lakṣaṇaṃ karttavyaṃ// \N2
%
%------------------------------
%
%------------------------------
%atha vā nāsāgrādārabhya tattvaṃ dādaśāṃgulapramāṇaṃ pītavarṇaṃ pṛthvītattvaṃ lakṣyaṃ karttavyam/ \E
%athavā nāsāgrādārabhya dvadaśāṃgulapramāṇaṃ pītavarṇaṃ pṛthvītatvaṃ lakṣaṃ kartavyaṃ// \B
%atha vā nāsāgrādārabhyaṃ ṣṭāṃgulapramāṇamati rakṭaṃ tejo lakṣaṇaṃ kartavyaṃ// \N2
%------------------------------
%
%------------------------------
%athavā nāsāgrādārabhya koṭisūryasamaprabhaṃ tejaḥ/ \E
%athavā nāsāgrādārabhya koṭisūryasamaprabhaṃ tejaḥ/ \B
%athavā nāsāgrādārabhya daśāṃgulapramāṇaṃ śuklaṃ caṃcalamudakaṃ lakṣaṇaṃ kartavyaṃ// \U2 [S.7 Verso, Zeile 1]
%------------------------------
%
%------------------------------
%\om                                                                                       \E
%\om \B
%athavā nāsāgrādārabhya dvadaśāṃgulapramāṇaṃ pītavarṇaṃ prthvītatvaṃ lakṣaṇaṃ karttavyaṃ// \N2
%------------------------------
%
%------------------------------
%\om                                                                 \E
%\om \B  DSCN7161.JPG letzte 3 Zeilen! 
%athavā nāsāgrādārabhyakoṭisūryasamaprabhaṃ tejaḥ                    \N2
%------------------------------
%
%------------------------------
%pūrṇam ākāśatattvaṃ lakṣyaṃ karttavyam/ ākāśamadhye ākāśopari    dṛṣṭiṃ kṛtvā dhyānakāraṇāt// sūryaṃ vinā sūryasambandhinī sahasrakiraṇapaṅktīḥ paśyati/ \E
%pūrṇam ākāśatatvaṃ lakṣaṃ kartavyaṃ//               ākāśopari    dṛṣti  kṛtvā ākāśamadhyedhyānakaraṇāt// sūryaṃ vinā sūryasaṃbaṃdhīnī sahasrakiraṇāvali paśyatī//\B
%pūrṇa ākāśatatvaṃ lakṣaṇaṃ karttavyaṃ//ākāśamadhye  ākāśopari vā dṛṣṭiṃ kṛtvā dhyānakaraṇātsūryavināsūryasaṃbaṃdhinī sahasrāṇapi kiraṇāṇi paśyate//     \N2
%------------------------------
%
%-----------------------------
%athavā śivopari vṛddhaṃ saptadaśāṃgulapramāṇaṃ tejaḥ puṃjalakṣyaṃ karttavyam/ \E
%athavā śiroparir urdhvaṃ saptadaśāṃgulapramāṇaṃ tejaḥ pūṃjaṃlakṣaṇaṃ kartavyaṃ/ \B
%athavā śiropari ūrddhvaṃ saptadaśāṃgulaṃ parāṇaṃ tejaḥ puṃjalakṣaṇaṃ kartavyaṃ// \N2
%-----------------------------
%
%-----------------------------
%athavā dṛṣṭer agre tatparaṃ svarṇākāraṃ pṛthvītattvaṃ lakṣyaṃ kartavyam/ \E
%athavā dṛṣṭeragne taptasuvarṇavarṇapṛthivītatvaṃ lakṣaṃ kartavyaṃ/ \B
%athavā dṛṣṭer agre taptavarṇākāraṃ pṛthvītatvaṃ lakṣaṇaṃ karttavyaṃ/ \N2
%-----------------------------
%
%-----------------------------
%uktānāṃ lakṣyāṇāṃ madhye yasya kasyāpyekasya lakṣyakaraṇāt valita palitā dūre bhavanti/ \E
%uktānāṃ lakṣaṇaṃ madhye yasya kasyāpi kasya lakṣakaraṇāt// valitaṃ palitādi dūre bhavatī/ \B
%uktānāṃ lakṣāṇāmadhye yasya lasyāpyelasya lakṣaṇasya karaṇāt /va/oder??/dva/lita palitādidūre bhavati/ \N2
%-----------------------------
%
%-----------------------------
%aṃgarogāḥ vinauṣadhaṃ dūrībhavanti/ samagrāḥ śatravaḥ svapnepyamitrannāyāṃti/ sahasravarṣamāyur bhavati/ \E
%aṃgirogādi vinauṣadhaṃ dūro bhavatī samagrāḥ śatrave svapnevyamitratāmayāṃti// sahasravarṣam āyur vardhate// \B
%aṃgarogā vinauṣadhaṃ dūre bhavati// samagrā śatravaḥ svapinevan nityaṃ nāyāti// sahasravarṣaparyaṃtamāyuṣaṃ vardhate// \N2
%-----------------------------
%
%-----------------------------
%apaṭhitaṃ śāstraṃ jihvāgreṇoccarati/ etādṛśaṃ phalaṃ bahutaraṃ bhavati// \E
%apaṭhitaṃ śāstraṃ jihvāgreṇoccaratī/ etādṛśaṃ phalaṃ bahutaraṃ bhavatī// \B
%apathitaṃ śāstraṃ jihvāgreṇoccarate// etādṛśaṃ bahutaraṃ phalaṃ bhavati// \N2
%-----------------------------
%
%-----------------------------
%idānīm anyataraṃ lakṣyaṃ kathyate/ \E
%idānīṃ antaralakṣaṃ kartavyaṃ// \B
%idānīṃ antaralakṣyakaṃ kathyate// \N1
%idānīṃ aṇtaralakṣyaṇaṃ kathyate// \N2
%-----------------------------
%
%-----------------------------
%mūlakandasthāne brahmadaṇḍotpannā nāḍī śvetavarṇā brahmadaṇḍaparyantam ekā brahmanāḍī varttate/ \E
%mūlakaṃ sthāne  brahmānaṃḍādutpannā śvetāvarṇā brahmaraṃdhraparyaṃtaṃ ekā nāḍī vartate/ \B
%mūlakaṃdasthāne brahmadaṃḍa ityannā śvetavarṇā brahmaraṃdhraparyaṃtaṃ ekā brahmanāḍī varttate/ \N1
%mūlakaṃdasthāne brahmadaṇḍadūtpannā śvetavarṇā brahmaraṃdhraparyaṃtaṃ ekā brahmanāḍī varttate/ \N2
%-----------------------------
%
%-----------------------------
%brahmanāḍīmadhye kamalatantusamānākārā koṭisūryavidyutsamaprabhā ūrdhvaṃ calati/ \E
%brahmanāḍīmadhye kamalataṃtusamānākārā koṭisūryavidyutsabhāprabhā ūrdhvaṃ calati/ \B
%brahmanāḍīmadhye kamalatantusamānākārā koṭisūryavidyutsamaprabhā ūrdhvaṃ calati/ \N1
%\om                                                                              \N2
%-----------------------------
%
%-----------------------------
%etādṛśyekā mūrttir varttate/ tanmūrterdhyānakāraṇāt         aṣṭamahāsiddhayo'ṇimādayastasya puruṣasya samīpamāgatya tiṣṭhanti// \E
%etādṛśyekā mūrttir varttate/ tasyāmūrtedhyānakaraṇāt//      aṣṭamahāsiddhayo// aṇimādyāḥ// aṇimāmahimāladhimāgirimādurevāyadivā yadi vā dure śrutvā parakāyāpraveśitā// puruṣasya samīpe māgatya tiṣṭhati// \B
%etādṛśī ekā mūrttir varttate/ tasyāḥ mūrtter dhyānakāraṇāt/ aṇimādīsiddhiḥ                  puruṣasya samīpe? āgatya tiṣṭhanti// \N1
%\om \N2
%-----------------------------
%
%-----------------------------
%atha vā lalāṭoparyākāśamadhye śuklasadṛśasya tejaso dhyānakāraṇāt       śarīrasambandhinaḥ kuṣṭhādayo rogā naśyanti/ āyur vṛddhir bhavati/ \E
%atha vā lalāṭo pari ākāśamadhye śuklasadṛśasya tejaso dhyānakāraṇāt//   charīrasambandhinaḥ kuṣṭhādayo rogā naśyaṃtī// āyur vṛddhir bhavatī/B
%atha vā lalāṭo pari ākāśamadhye śuklasadṛśasya tejaso dhyānakāraṇāt śarīrasambandhī kuṣṭhādayo rogāḥ naśyaṃti/ āyur vṛddhir bhavati/       \N1
%                                             tasyā mūrtter dhyānakaraṇāccharīrasaṃbaṃdhi---kuṣṭadayorogāḥ  naśyaṃti āyur vṛddi bhavati/    \N2
%-----------------------------
%
%-----------------------------
% bhruvor madhyetiriktavarṇasyātisthūlasya tejaso dhyānakāraṇādbahulānāṃ pārthivānāṃ tatpuruṣāṇāṃ ca vallabho bhavati/ \E
%atha vā// bhruvor madhye 'tiraktavarṇasyātisthūlasya tejaso dhyānaṃ karaṇāt-sakalānāṃ pārthiva puruṣāṇāṃ vallabho bhavati/ \B DSCN7163.jpg Z.1
%atha vā bhruvor madhye 'tiraktavarṇasyātisthūlasya tejaso dhyānakaraṇāt-sakalānāṃ pārthiva puruṣāṇāṃ vallabho bhavati/ \N1
%atha vā bhruvor madhyetiraktavarṇasyātisthūlasya tejaso dhyānakaraṇāt-sakālānāṃ pārthiva puruṣāṇāṃ vallabho bhavati/ \N2
%-----------------------------
%
%-----------------------------
%jagadvallabho pi bhavati/ asya puruṣasyāvalokanena sarveṣāṃ dṛṣṭiḥ sthirā bhavati// \E
%taṃ puruṣaṃ vrati sarveṣāṃ dṛṣṭisthirā bhavatī// \B
%taṃ puruṣaṃ dṛṣṭvā sarveṣāṃ dṛṣṭisthirā bhavati// \N1
%taṃ puruṣaṃ dṛṣṭā       sarveṣāṃ dṛṣṭisthirā bhavati// \N2
%-----------------------------
%
%-----------------------------
%idānīṃ śarīramadhye nāḍīnāṃ bhedāḥ kathyante daśa mukhyanāḍyaḥ/ \E
%idānī  śarīramadhye nāḍībhedaḥ kathyate// daśamukhye nāḍyā \B
%idānīṃ śarīramadhye nāḍīnāmaparo bhedaḥ kathyate// daśa mukhyanādhyaḥ/ \N1
%idānī  śarīramadhye nāḍīnāmaparo bhedāḥ kathyate// daśa mukhyanāḍyaḥ// \N2
%-----------------------------
%Now the divisions of channels within the body are explained. There are ten channels belonging to the face. 
%-----------------------------
%tanmadhye dvayamijāpiṃgalāsaṃjñakaṃ nāsā dvāre tiṣṭhati/ \E
%tanmadhye nāḍīdvayaṃ /idāpiṃgalā saṃjñīkāḥ nāsādvāre tiṣṭhati//  \B
%tanmadhye nāḍīdvayam/ idāpiṃgalā saṃjñakaṃ nāsādvāre tiṣṭhati//  \N1
%tanmadhye nāḍīdvayam/ idānīṃ piṃgalā saṃjñakaṃ nāsādvāre tiṣṭhati//  \N2
%-----------------------------
%
%-----------------------------
%suṣumṇā tālumārge brahmadvāraparyantaṃ vahati tiṣṭhati/ \E
%suṣumṇā tālumārge brahmaraṃdhraparyantaṃ vahatī tiṣṭhati \B
%suṣumṇā tu tālumārgeṇa brahmadvāraparyantaṃ vahatī tiṣṭhati \N1
%suṣumṇā tu tālumārge brahmadvāraparyantaṃ vahatī tiṣṭhati// \N2
%-----------------------------
%The wind is located at the end of the door of Brahma by the path of the palate.? 
%-----------------------------
%sarasvatī mukhamadhye tiṣṭhati/ \E
%sarasvatī mukhamadhye tiṣṭhatī/ \B
%sarasvatī mukhamadhye varttate/ \N1
%sarasvatī mukhamadhye varttate/ \N2
%-----------------------------
%
%-----------------------------
%gāṃdhārī hyasti jihvākarṇayor madhye vahalyau tiṣṭhataḥ/ \E
%gāṃdhārī hastī jihvē karṇavahatyo tiṣṭhati// B
%gāṃdhārī hastinī jihvē karṇayor madhye vahatyau tiṣṭhataḥ// N1
%gāṃdhārī hastinī jihvē karṇayor madhye vahatyau tiṣṭhataḥ// N2
%-----------------------------
%
%-----------------------------
%pūṣā lambusemā netrayormadhyervahalyā tiṣṭhataḥ/ \E
%pūṣo ḍalabuṣenetramadhye vahatyo tiṣṭhati/ \B
%pūṣāṃ alaṃbuṣenetrayor madhye vahatyo tiṣṭhataḥ/ \N1
%pūṣāṃ alaṃbuṣenetayor madhye vahatyo tiṣṭhataḥ/ \N2
%-----------------------------
%
%-----------------------------
%śaṃkhinī liṃgadvārādārabhyeḍāmārgeṇa brahmasthānaparyaṃtaṃ tiṣṭhatīti/ \E
%śaṃkhinī liṃgadvārādārabhya iḍāmārgeṇa brahmasthānaparyaṃtaṃ tiṣṭhati/ \B
%śaṃkhanī liṃgadvārādārabhya iḍāmārgeṇa brahmasthānaparyaṃtaṃ tiṣṭhati/ \N1
%śaṃkhinī liṃgadvārādārabhya iḍānīṃ mārgeṇa brahmasthānaparyaṃtaṃ tiṣṭhati/ \N2
%-----------------------------
%
%-----------------------------
%etādṛśanāḍyo daśasu dvāreṣu tiṣṭhanti/ \E
%etādṛṣyānāḍyo daśasu dvāreṣu tiṣṭhaṃti/ \B
%etādaśanāḍyo daśasu dvāreṣu tiṣṭhaṃti/ \N1
%etādaśasudvāreṣu tiṣṭhaṃti/ \N2
%-----------------------------
%
%-----------------------------
%anyā dvisaptatisahasraparimitā nāḍayo lomnāṃ mūleṣu sūkṣmarūpeṇa tiṣṭanti// \E
%anyā dvisaptatīsahasraparimitā nāḍyo lomnāmūleṣu sūkṣmarūpeṇa tiṣṭaṃti// \B
%anyā dvisaptatisahasraparamitā nāḍyā lomnāṃ mūleṣu sūkṣmarūpeṇa tiṣṭaṃti// \N1
%anyā dvisaptatrisahasraparimitā nāḍyā lomnāṃ mūleṣu sūkṣmarūpeṇa tiṣṭaṃti// \N2
%-----------------------------
%
%-----------------------------
%[p.36]
%idānīṃ śarīramadhye vāyavo daśa tiṣṭhanti/ \E
%idānīṃ śarīramadhye .....\om \B
%idānīṃ śarīramadhye vāyavastiṣṭhaṃti/ \N1
%idānīṃ śarīramadhye vāyavastiṣṭhaṃti/ \N2
%-----------------------------
%
%-----------------------------
%teṣāṃ nāmāni kāryāṇi kathyante/ \E
%\om
%teṣāṃ kāryāṇi kathyante/ \N1
%teṣāṃ kāryāṇi kathyate/ \N2
%-----------------------------
%
%-----------------------------
%prāṇavāyurtdṛdayamadhye śvāsocchāsaṃ karoti/ \E
%------------------------śvāsośvaroti/ \B
%prāṇavāyuhṛdayamadhye utsvāsaprasvāsasaṃ karoti// \N1
%prāṇavāyuhṛdayamadhye ūrdhvaśvāsapraśvāsaṃ karoti// \N2
%-----------------------------
%
%-----------------------------
%aśanapānecchā bhavati/ gudamadhye samāno vāyur vartate/                                                             sapta samagrā nāḍīḥ śoṣayati/  \E
%aśanapānechā bhavati// gudamadhye appānāvāyor tiṣṭhatī sa āṃkucanastaṃbhanaṃ karotī/ nābhīmadhye smānā vartate/ sa samagrānāḍī śoṣayati// \B
%asitapittecha bhavati/ guḍamadhye apānavāyu tiṣṭhati sa ākuṃcanaṃ sthaṃbhanaṃ karoti/ nābhimadhye samāno varttate/ sa samāgraṃ nādhyaṃ śoṣayati/  \N2
%aśitapiteccha bhavati/ gudamadhye apānavāyu tiṣthati/ sa ākuṃcanaṃ staṃbhanaṃ karoti/ nābhimadhye samāno varttate/ sa samagraṃ nādhyaṃ śoṣayati// \N1
%-----------------------------
%
%-----------------------------
%tathā nāḍīśoṣaṇāt rucimutpādayati/ vahniṃ dīpayati/ \E
%tathā   pośayatī/ tathā poṣayatī// rucirutpādayatī vahnī dīpayatī/ \B
%tathā nāḍīṃ pośayati/ kvacit-utpādayati/ āgniṃ dīpayati \N1
%tathā nāḍīṃ pośayati/ kvacit-utpādayati/ āgniṃ dīpayati \N2
%-----------------------------
%
%-----------------------------
%tālumadhye udāno vāyus-tiṣṭhati/ sa vāyuḥ ratnaṃ līlati/ pānīyaṃ pibati/ nāgavāyuḥ sarvaśarīre varttate/ tasmād-vāyoḥ śarīraṃ cālayati/ śokamāpnoti// vivilaḥ \E
%tālumadhye udānavāyus-tiṣṭhati/ sa vāyur annaṃ galayatī/ pānīyaṃ pibatī/ nāgavāyuḥ sakalaśarīre varttate tasmād-vāyoḥ// śarīrae cālatī/ śokamāpnoti vi??kru??taḥ// \B DSCN7163.JPG Z.11
%tālumadhye udānavāyus-tiṣṭhati/ sa vāyuḥ ratnaṃ śilati/ pānīyaṃ pibati/ nāgavāyuḥ sakale śarīre varttate// tasmād-vāyoḥ śarīraṃ calati/                \N1
%tālumadhye udānāni vāyus-tiṣṭhati/ sa vāyur-annaṃ gīlati/ pānīyaṃ pibati/ nāgavāyuḥ sakale śarīre varttate// tasmād-vāyoḥ śarīraṃ calati/                \N2
%-----------------------------
%
%-----------------------------
%kūrmavāyur netramadhye tiṣṭhati। nimeṣonmeṣaṃ karoti/ \E
%kūrmavāyoḥ netramadhye nimeṣonmeṣaṃ karotī/ \B
%kūrmo vāyunetramadhye tiṣṭhati/ unmeṣaṃ nimeṣaṃ karoti/ \N1
%kūrmo vāyunetramadhye tiṣṭhati/ unmeṣaṃ nimeṣaṃ karoti/ \N2
%-----------------------------
%
%-----------------------------
%kṛkalakartā vāyurudgāraṃ karoti devadattavāyoḥ jṛmbhaṇaṃ bhavati/ dhanaṃjayavāyoḥ śabda utpadyate// \E
%kṛkalavāyur udhāraṃ karotī devadattavāyor  jumbhābhavaṃtī dhanaṃjayavāyoḥ śabda utpadyate// \B
%kṛkalavāyor-ūdgāto bhavati/ devadattavāyor jṛṃbha utpadyate/ dhanaṃ jayavāyo śabda utpadyate// \N1
%kṛkaravāyor-ūdgāro bhavati/ devadattavāyo  jṛṃbhotpadyate/ dhanaṃ jayavāyo śabdotpadyate// \N2
%-----------------------------
%
%-----------------------------
%\om                              \E
%idānīṃ madhyalakṣaṇaṃ kathyate//  \B DSCN7164 Z.1
%idānīṃ madhyalakṣyaṃ kathyate//  \N1
%idānīṃ madhyalakṣaṇaṃ kathyate// \N2
%-----------------------------
%
%-----------------------------
%aṃtha ca pītavarṇaṃ raktavarṇaṃ vā dhūmrākāraṃ yannīlavarṇaṃ vā agniśikhāsadṛśaṃ vidyutsamānaṃ sūryamaṇḍalasadṛśaṃ arddhacandrasadṛśaṃ jvaladākāśasamākāraṃ svaśarīraparimitaṃ tejomanomadhye tathyaṃ kartavyam// \E
%śvetavaraṃ atha pītavarṇaṃ// rakta vā dhūmrākāraṃ yannīlavarṇaṃ vā// agniśikhāsadṛśaṃ vidyutsamānaṃ sūryamaṇdalasadṛśaṃ/ ūrdhvacaṃdrasadṛśaṃ jvaladākāśasamākāraṃ// svaśarīraparimitaṃ tejomanomadhye lakṣaṃ kartavyaṃ//  \B
%śvetavarṇā/ athavā pītavarṇaṃ raktaṃ vā dhūmārava nīlavarṇaṃ vā agniśikhāsadṛśaṃ vidyutsamānaṃ sūryamaṇdalaṃ sadṛśaṃ/ ūrdhvacaṃdrasadṛśaṃ jvaladākāśasamānakāraṃ// svaśarīraparimitaṃ tejomanomadhye lakṣyaṃ karttavyaṃ//  \N1
%śvetavarṇā athavā pītavarṇa raktavarṇa dhūmravarṇa nīlavarṇaṃ vā agniśikhāsadṛśaṃ vidyutsamānaṃ sūryamaṇdalasadṛśaṃ ūrdhvacaṃdrasadṛśaṃ jvaladākāśasamānakāraṃ// svaśarīraparimitaṃ tejomanomadhye lakṣaṇaṃ karttavyaṃ//  \N2
%-----------------------------
%
%-----------------------------
%ekasmin lakṣye kṛte sati manomadhye sthitasya malasya[p.38]%dāho bhavati/ \E
%etasmin lakṣe kṛte satī manomadhye sthitasya malasya dāho bhavati \B
%ekasmin lakṣye kṛte sati manomadhye sthitasya malasya dāho bhavati/ \N1
%ekasmin lakṣaṇo kṛte sati manomadhye sthitasya malasya dāho bhavati/ \N2
%
%-----------------------------
%
%-----------------------------
%manasaḥ sattvaguṇaprakāśo bhavati/ puruṣa ānandamayo bhūtvā tiṣṭhati// \E
%manasaḥ// sattvaguṇo prakaṭo bhavati// puruṣa ānandamayo bhūtvā tiṣṭhati// \B
%manasaḥ sattvaguṇe prakaṭo bhavati/ puruṣa ānandamayo bhūtvā tiṣṭhati// \N1
%manasaḥ sattvaguṇo prakaṭo bhavati/ puruṣa ānandamayo bhūtvā tiṣṭhati// \N2
%-----------------------------
%
%-----------------------------
%idānīm-ākāśabhedāḥ kathyante/ \E
%idānīṃ ākaśabhedāḥ kathyaṃte/ \B
%idānīṃ ākaśabhedāḥ kathyaṃte/ \N1
%idānīṃ ākāśabhedāḥ kathyate/ \N2
%-----------------------------
%
%-----------------------------
%te ākāśaḥ paramākāśaḥ mahākāśaḥ tattvākāśaḥ sūryākāśaḥ/ bāhyābhyantare nirmalaṃ nirākāramākāśalakṣyaṃ karttavyam/ \E
%teṣāṃ lakṣyāni kathyate//  ākāśa,parākāśa,mahākāśa,tatvākāśa,sūryakāśa// bāhyābhyaṃtare nirmalaṃ nirākāraṃ ākāśalakṣyaṃ kartavyaṃ// \N1
%ākāśaḥ paramākāśaḥ// mahākāśa// tattvākāśaḥ sūryākāśa// bāhyābhyaṃtaro nirmalaṃ nirākāramākāśaṃ lakṣaṃ kartavyaṃ// \E
%teṣāṃ lakṣaṇāni kathyate// ākāśaparākāśamahākāśatatvākāśasūryakāśaḥ bāhyābhyaṃtare nirmalaṃ nirākāraṃ ākāśalakṣaṇaṃ kartavyaṃ// \N2
%
%-----------------------------
%
%-----------------------------
%tataḥ paraṃ bāhyābhyantareṣvanandhakārasadṛśaṃ parākāśaikyaṃ lakṣyaṃ karttavyam// \E
%tataḥ paraṃ bāhyābhyaṃtare ghanāṃ ghakārasadṛśaḥ parākāśalakṣaṃ kartavyaṃ// \B
%tataḥ paraṃ bāhyābhyantare ghanāṃ dhakārasadṛśaparākāśasya lakṣyaṃ kattavyam// \N1
%tataḥ paraṃ bāhyābhyantare ghanāṃ dhakārasadṛśaparākāśasya lakṣaṇaṃ karttavyam// \N2
%-----------------------------
%
%-----------------------------
%tataḥ paraṃ pralayakālīnajvaladdāvānalapūrṇaṃ bāhyābhyantare, mahākāśalakṣyaṃ karttavyam/ \E
%tataḥ paraṃ pralayakālīnaḥ jaladvaḍavānalapūrṇaṃ bāhyābhyaṃtare mahākāśalakṣaṃ kartavyaṃ// \B
%tataḥ paraṃ pralayakālīnajvaladvṛddha?[S.9 verso letzte Zeile] nalapūrṇa bāhyābhyaṃtare mahākāśalakṣyaṃ karttavyaṃ// \N1
%tataḥ paraṃ pralayakālīnajvaladvṛ? nalapūrṇa  bāhyābhyaṃtare  mahākāśalakṣaṃ karttavyaṃ// \N2
%-----------------------------
%
%-----------------------------
%\om                                                                                                                  tataḥ bāhyābhyantare prakāśamānayarsūsahitaṃ sūryākāśaṃ lakṣyaṃ[p.39]karttavyam/ \E
%tataḥ paraṃ bāhyābhyaṃtare koṭidīpānāṃ prakāśaprāpto yādṛśamau jvalaṃ bhavatī/ tādṛśaṃ tatvāśalakṣaṃ kartavyaṃ// paccā bāhyābhyaṃtare prakāśamān sūryabiṃbasahitasūryakāśalakṣaṃ kartavyaṃ mataḥ ... \B
%tataḥ paraṃ bāhyābhyaṃtare koṭidīpānāṃ prakāśaprāptau yādṛśamau jvalyaṃ bhavati/ tādṛśaṃ tatvākāśaṃ lakṣyaṃ kartavyaṃ// tataḥ paścāt bāhyābhyaṃtare prakāśamānasūryabiṃbasahitaṃ sūryakāśaṃ lakṣyaṃ karttavyaṃ// N1
%tataḥ paraṃ bāhyābhyaṃtare koṭidīpānāṃ prakāśaprāptau yādṛśamau jvala bhavati/ tādṛśaṃ tatvākāśaṃ lakṣaṃ kartavyaṃ// tataḥ paścādābhyaṃtare prakāśamānasūryabiṃbasahitaṃ sūryakāśaṃ lakṣaṃ karttavyaṃ// N2
%-----------------------------
%
%-----------------------------
%eteṣāṃ lakṣyāṇāṃ kāraṇāt śarīraṃ rogāsaṃsargi bhavati// \E
%eteṣāṃ lakṣaṇaṃ karaṇāt// śarīre rogasaṃsargo na bhavatī/ \B
%eteṣāṃ lakṣyaṇāṃ karaṇāt śarīrarohasaṃsarge na bhavati/ \N1
%eteṣāṃ lakṣāṇākāraṇāccharīrarogāsaṃsargo na bhavati// \N2
%-----------------------------
%
%-----------------------------
%tathā valitapalitaṃ puṇyaṃ pāpaṃ na bhavati// \E
%tathā// valitapalitaṃ puṇyāṃ pāpaṃ ca na bhavatī// \B
%tathā valitaṃ palitaṃ puṇyaṃ pāpaṃ ca na bhavati// \N1
%tathā valitaṃ palitaṃ puṇyaṃ pāpaṃ ca na bhavati// \N2
%-----------------------------
%
%-----------------------------
%navacakraṃ kalādhāraṃ trilakṣyaṃ vyomapaṃcakam/ \E
%śloka navacakraṃ kalādhāraṃ trilakṣaṃ vyomapaṃcakam/ \B
%navacakrakalādhāraṃ trilakṣyaṃ vyomapaṃcakaṃ/ \N1
%navacakrakalādhāraṃ trilakṣaṃ vyomapaṃcakaṃ/ \N2
%-----------------------------
%
%-----------------------------
%svadehe yo na jānāti sa yogīnāmadhārakaḥ// \E
%svadehe yo na jānāti sa yogīnāmadhārakaḥ//1// \B
%samakriyānajānāti sayogīnāmadhāraka// \N1
%samakriyānajānātisayogīnāmadhāraka// \N2
%-----------------------------
%
%-----------------------------
%idānīṃ cakrāṇām anukramaḥ kathyate/ \E
%idānīṃ cakrāṇām anukramaḥ// \B
%idānīṃ cakrāṇām anukrama  kathyaṃte/ \N1
%idānīṃ cakrānām-anukramā  kathyaṃte/ \N2
%-----------------------------
%Now the practice of the cakras is explained. 
%-----------------------------
%ādhāre brahmacakram/ ādhāropari liṃgamūle sbādhiṣṭhānacakram/ nābhau maṇipūrakacakram/ hṛdaye anāhatacakram/ kaṇṭhasthāne viśuddhicakram/ ṣaṣṭhaṃ tālucakram/ bhruvor madhye ājñācakram/ brahmasthāne kālacakram/ navamamākāśacakram/ etatparaṃ śūnyam/ \E
%ādhāro brahmacakram/ ādhāropari liṃgamūle svādhiṣṭhānacakraṃ//2// nābhau maṇipūrakacakram//3  hṛdaye anāhatacakram// 4 kaṇṭhasthāne viśuddhicakraṃ// ṣaṣṭhaṃ tālucakre/6 bhruvormadhye ājñāyacakraṃ/ brahmasthāne kālacakraṃ// 8 navamaṃ ākāśacakraṃ/9 tatparaṃ śūnyam/ \B
%ādhāre brahmacakraṃ liṃge svādhiṣṭhānacakram/ nābhau maṇipūrakacakram/ hṛdaye viśuddhacakraṃ/ kaṇṭhasthāne anāhatacakraṃ/ ṣaṣṭhaṃ tālucakram/ bhruvor madhye ājñācakram brahmaraṃdhrasthāne kālacakraṃ/ navamaṃ ākāśacakram/ tatparamaśūnyaṃ/ \N1
%ādhāre brahmacakraṃ liṃge svādhiṣṭhānacakram// nābhau maṇipūrakacakram/ hṛdaye viśuddhacakraṃ/ kaṇṭhasthāne anāhatacakraṃ ṣaṣṭhaṃ tālucakram/ bhruvor madhye ājñācakram brahmaraṃdhrasthāne kālacakraṃ/ navama ākāśacakram tata paraśūnyaṃ/ \N2
%-----------------------------
%
%-----------------------------
%idānīm ādhāracakrasya bhedāḥ kathyanta/ \E
%idānīmādhāracakrasya bhedā kathyaṃte/ \B DSCN7165.jpg Z.3
%idānīm ādhāracakrasya bhedaḥ kathyate/  \N1
%idānī  ādhāracakrasya bhedaḥ kathyaṃte/ \N2
%-----------------------------
%
%-----------------------------
%pādayor aṃguṣṭhe tejaso lakṣyakāraṇāt dṛṣṭiḥ sthirā bhavati/ \E
%pādayor aṃguṣṭhai tejasaṃ lakṣaṃ kartavyaṃ kāraṇāt// dṛṣṭiḥ sthirā bhavati/ \B
%pādayor aṃguṣṭhe tejaso lakṣyakāraṇāt dṛṣṭisthirā bhavati/ \N1
%pādayor aṃguṣṭhe tejaso lakṣakāraṇāt dṛṣṭisthirā bhavati/ \N2
%-----------------------------
%
%-----------------------------
%dvitīyo mūlādhāraḥ/ pādāṃguṣṭhasya mūle parapādasya pārṣṇiḥ sthāpyate tadāgniḥ prabalo bhavati/ \E
%dvitīyo mūlādhāraḥ/ pādāṃguṣṭhasya mūle aparasya pādapārṣṇiḥ syāpyate tadāgniḥ prabalo bhavatī/ \B
%dvitīyo mūlādhāraḥ/ pādāṃguṣṭhasya mūle aparapādasya pārṣṇiḥ sthāpyate agniḥ prabalo bhavati/   \N1
%dvitīyo mūlādhāraḥ  pādāṃguṣṭhasya mūle aparapādasya pārṣṇiḥ sthāpyate/ \om                     \N2
%-----------------------------
%
%-----------------------------
%ekaḥ pārṣṇirādau[P.41]mūlādhāre sthāpyate/ tasya pādasyāṃguṣṭhamūle parasya pādasya pārṣṇiḥ sthāpyate// tadagniḥ pradīpyate// \E
%ekā pārṣṇirādau[P.41]mūlādhāra sthāpyate   tasya pādasyāṃguṣṭhamūle aparasya pādasya pārṣṇiḥ sthāpyate// tadagnīḥ pradipyate// \B
%ekā pārṣṇiḥ mūladdhāre sthāpyate/          tasya pādasya aṃguṣṭhamūlaṃ/ aparasya pādasya pārṣṇiḥ sthāpyaṃ agnir dāpyate?!/ \N1   
% \om ------------------------------------- tasya pādasyāṃguṣṭhamūle// aparasya pādasya pārṇisthāpyaṃ agni dīpate// \N2
%-----------------------------
%
%-----------------------------
%tṛtīyaṃ gudādhārasthānaṃ tanmadhye saṃkocavikāsākuṃcanakāraṇāt pavanaḥ sthiro bhavati// \E
%tṛtīyaṃ gudādhārasthāne tanmadhye saṃkocavikāśākuṃcanakāraṇāt pavanaḥ sthiro bhavati// \B
%tṛtīyaṃ gudādhārasthānaṃ tanmadhye saṃkocavikāśākuṃcanakāraṇāt pavanaḥ sthiro bhavati// \N1
%tṛtīyaṃ gudādhārasthānaṃ taṃmadhye saṃkocavikāśākuṃcanaṃ kāraṇāt pavanasthiro bhavati// \N2
%-----------------------------
%
%-----------------------------
%anyacca/ puruṣasya maraṇaṃ na bhavati/ \E
%anucarapuruṣasya maraṇaṃ bhavatī/ \B
%anūca puruṣasya maraṇaṃ na bhavati ve?/ \N1
%anūca puruṣasya maraṇaṃ na bhavati// \N2
%-----------------------------
%
%-----------------------------
%caturthaṃ liṃgādhāraṃ tanmadhye/ liṃgasaṃkocanābhyāsāt  paścimadaṇḍamadhye prajñā nāḍī bhavati/ tanmadhye punarabhyāsakaraṇānmanaḥ pavanayoḥ saṃcāro bhavati/ \E
%caturthaliṃgādhāraṃ tanmadhye liṃgasaṃkocanābhyāsāt paścīmadaṇḍamadhye vajñā nāḍī bhavatī/ tanmadhye punar abhyāsakaraṇāt punaḥ pavanayo saṃcāro bhavatī/ \B
%caturthaṃ liṃgādhāraṃ tanmadhye/ liṃgasaṃkocanābhyāsāt/ paścimadaṇḍamadhye vajranāḍī bhavati/ tanmadhye punaḥ abhyāsakaraṇāt manaḥ pavanayoḥ saṃcāro bhavati/ \N1
%caturthaṃ liṃgādhāraṃ tanmadhye  liṃgasakoṇābhyāsāt//   paścimadaṇḍamadhye vajranāḍī bhavati/ tanmadhye punarābhyāsakaraṇāt manaḥ pavanayoḥ saṃcāro bhavati// \N2
%-----------------------------
%
%-----------------------------
%tayoḥ saṃcārān madhye granthitrayaṃ truṭyati/ tatroṭanāt pavano brahmakamalamadhye pūrṇo bhūtvā tiṣṭhati/ tato vīryastambho bhavati/ puruṣaḥ sadaiva yuvā bhavati/ \E
%tayo saṃcārān madhye granthitrayaṃ truṭyatī/ tatroṭanāt pavano brahmakamadhye pūrṇā bhūtvā tiṣṭhati// tato vīryastambho bhavatī// puruṣaḥ sadaiva yuvai bhavatī/ \B
%tayoḥ saṃcārān madhye granthitrayaṃ truṭyati/ tat troṭanāt pavanaḥ brahmakamalamadhye pūrṇo bhūtvā tiṣṭhati/ tato vīryastambho bhavati/ puruṣaḥ sadaiva yuvā/e va bhavati// \N1 %truṭyati="zerbrechen"
%tayoḥ saṃcārān madhye granthitrayaṃ ... ..ti/ tata troṭanāt pavanaḥ brahmakamalamadhye pūrṇo bhūtvā tiṣṭhati/ tato vīryastambho bhavati/ puruṣa sadaiva  yurvaiva bhavati// \N2
%-----------------------------
%
%-----------------------------
%paṃcama udgīryāṇāṃ svādhiṣṭhānaṃ tatra bandhanānmalamūtrayornāśo bhavati/ ṣaṣṭho nābhyādhāraḥ/ tasmin sthāne prāṇavāyornirodhāt ṣaḍapi kamalānyūrdhvamukhāni vikasaṃti// \E
%paṃcama uḍḍiyānāṃ svādhiṣṭhānaṃ tatra baṃdha dīyate/ malamūtrayor nāśo bhavatī// ṣaṣṭho nābhyādhāraḥ tatra prāṇavābhyāsād anāhato nādaḥ// svayam utpadyate// \B
%paṃcamaṃ udyānaṃ tatra baṃdhanāt malamūtrayornāśe/o[s.10, verso, z4] bhavati// ṣaṣṭho nābhyādhāraḥ/ tatra praṇavābhyāsāt anāhato nādaḥ svayamūtpadyate/  \N1
%paṃcamodyānaṃ tatra baṃdhanāt malamūtrayor nāśo bhavati/                       ṣaṣṭho nābhyādhāraḥ  tatra praṇavābhyāsāt anāhato tādaḥ svayaṃ utpadyate/ \N2
%
%-----------------------------
%
%-----------------------------
%\om                                                                         \E
%                           tasmin sthāne prāṇavāyo nirodhaāt/ yaḍapi kamalāny ūrdhvamukhāni vikasaṃti// \B
%saptamo hṛdayarūpa ādhāraḥ tasmin sthāne prāṇacā?yor nirūṃdhanāt/ ṣadapi kamalānyūrdhvamukhaṃ vikasaṃti// \N1 
%saptamo hṛdayarūpādhāraḥ tasminsthāne prāṇavāyor nirūṃdhanāt/ ṣadapi kamalānyūrdhvemukhaṃ [S.9, recto, z.4] vikasaṃti// \N2
%-----------------------------
%
%-----------------------------
%aṣṭamaṃ kaṇṭhādhāraḥ/ tatra jālaṃdharo bandho dīyate/ tasmin satīḍāyāṃ piṃgalāyāṃ[p.43]pavanaḥ sthiro bhavati/ \E
%aṣṭame kaṇṭhādhāraḥ/ tatra jalaṃ baṃdho dīyate tasmin satīyāṃ piṃgalāyāṃ pavanaḥ sthiro bhavatī/ \B  %%%%DSCN7166.jpg Z.3
%aṣṭamaḥ kaṇṭhādhāraḥ/ tatra jālaṃdharo baṃdho dīyate/ tasmin sati iḍāyāṃ piṃgalāyāṃ[p.43]pavanaḥ sthiro bhavati/ \N1
%aṣṭamakaṇṭhādhāraḥ/ tatra jālaṃdharabandho dīyate// tasminsatiśadāyāṃ piṃgalāyāṃ pavanaḥ sthiro bhavati/ \N2
%-----------------------------
%
%-----------------------------
%navamo ghaṃṭikādhāraḥ/ tatra jihvāgraṃ lagnaṃ bhavati/ tatomṛtakalāyā amṛtaṃ sravati/ tadamṛtapānāt śarīramadhye rogasaṃcāro na bhavati/ \E
%navo ghaṃṭikādhāraḥ// tatra jihvāgraṃ lagnaṃ bhavatī/ tatomṛtakalāyā amṛtaṃ sravati/ tadamṛtakalāyāṃ amṛtapānīcharīramadhye rogasaṃcāro bhavatī/ \B
%navamo ghaṃṭikādhāraḥ/ tatra jihvāgraṃ lagnaṃ bhavati/ tatomṛtakalāyā amṛtaṃ sravati/ tadamṛtapānāt śarīramadhye rogasaṃcāro na bhavati/ \N1
%navamo ghaṃṭikādhāraḥ/ tatra jihvāgraṃ lagnaṃ bhavati/ tatomṛtakalāyā amṛtaṃ sravati/ tadamṛtapānāt śarīramadhye rogasaṃcāro na bhavati/ \N2
%-----------------------------
%
%-----------------------------
%daśamaṃ tālvādhāraḥ/ tanmadhye vānaṃdollahanaṃ ca kṛtvā laṃbikāpraveśe sati tālunimagnā jihvā tiṣṭhati/ \E
%daśamaṃ stālvādhāraḥ/ tanmadhye cālanaṃ dohanaṃ cakratvā laṃbikāpraveśe sati tālumagnā jihvā tiṣṭhati/ \B
%daśamatālvādhāraḥ// tanmadhye cānanaṃ dohanaṃ ca kṛtvā laṃbikāpraveśe grati tālunimagnā jihvā tiṣṭhati/ \N1
%daśamatālvādhāraḥ tanmadhye cālanaṃ dohanaṃ ca kṛtvā laṃbikāpraveśe gratitālūnimagnā                    \N2
%-----------------------------
%
%-----------------------------
%ekādaśo jihvādhāraḥ/ tasmin jihvāgreṇa manthanaṃ kriyate tasmin kṛtetimadhuraṃ pānīyaṃ sravati/ tadā ca kavitvacchandonāṭakādiviṣayajñānamutpadyate/ \E
%ekādaśo jihvātale jihvādhāraḥ// tasmin jihvāgreṇa manthanaṃ kṛtvā// tasmiṃ kṛte satimadhuraṃ pānīyaṃ sravatī// tathā kvacit vacchaṃdonāṭakādiviṣayapānam utpadyaṃte/ \B
%ekādaśo jihvādhāraḥ/ tasmin jihvāgreṇa manthanaṃ kriyate/ tasmin kṛte atimadhuraṃ pānīyaṃ sravati/ tathā ca kavitvagītacchaṃdanāṭakādiviṣaye jñānam utpadyate/ \N1
%                            jihvāgreṇa manthanaṃ kriyate// tasmin kṛte atimadhuraṃ pānīyaṃ sravati// kaminnāsikā phatkāravat// tathā ca kavitvagītachaṃdanāṭakādiviṣaye jñānam utpadyate/ \N2
%-----------------------------
%
%----------------------------
%[p.44] tadupari dvādaśadantayomadhye dantādhāraḥ/ tasmin sthāne jihvāyā agraṃ ghaṭīmātraṃ valātkāreṇa sthāpyate/ tasmin sati sādhakasya samagrā rogā naśyanti// \E
%        tadupari dvādaśo daṃtayor madhye// daṃtādhāraḥ// tasmin sthāne jihvāyā 'agnaṃ ghaṭīmātramārghaghaṭimātraṃ bālākāreṇa sthāpyate// tasmiṃ sādhakasya samagrā rogā naśyaṃtī// \B
%        tadupari dvādaśayor madhye daṃtādhāraḥ/ tasmin sthāne jihvāyā agraṃ ghaṭīmātraṃ arddhaghaṭimātraṃ valātkāreṇa sthāpyate// tasmin sati sādhakasya samagrā rogā naśyaṃti// \N1
%        tadupari dvādaśayor madhye daṃtādhāraḥ// tasmin sthāne jihvāyāgraṃ ghaṭīmātraṃ arddha?ghaṭimātraṃ valātkāreṇa sthāpyate// tasmin sati sādhakasya samagrā rogā naśyanti \N2
%-----------------------------
%
%----------------------------
%trayodaśo nāsikāgrādhāraḥ/ tasmin lakṣye kṛte sati manaḥ sthiraṃ bhavati/ \E
%trayodaso nāsikādhāraḥ/ tasmin ḍraṣṭe kṛte minasthire bhavati/ \B
%trayodaśo nāsikādhāraḥ/ tasmin lakṣe kṛte sati manasthiraṃ bhavati/ \N1
%trayodaśo nāsikādhāraḥ/ tasmin lakṣe kṛte sati manasthiraṃ bhavati/ \N2
%-----------------------------
%
%----------------------------
%caturdaśo nāsāmūlādhāraḥ/ tasmin dṛṣṭeḥ sthairyakāraṇātṣaṣṭhe māsi svīyantejaḥ pratyakṣaṃ bhavati/ tejasaḥ pratyakṣatve pārthivaṃ sakalaṃ bandhanaṃ tuṭyati/ \E
%caturdaśo nāso mūlādhāraḥ// tasmin llakṣe krute satī sthairyakāraṇāt// ṣaṣṭhe māse svayaṃ tejaḥ pratyakṣaṃ bhavati// tejasaḥ pratyakṣatve pārthivaṃ sakalaṃ baṃdhanaṃ truṭhayati/ \B
%caturdaśo nāsāmūle vāyvā? bāybā? dhāraḥ/ tasmin dṛṣṭeḥ sthairyakāraṇāt ṣaṣṭhe māsi svīyaṃ tejaḥ pratyakṣaṃ bhavati/ tejasaḥ pratyakṣatve pārthivaṃ sakalaṃ baṃdhanaṃ trudyati/ \N1
%caturdaśo nāsāmūle vāyvādhāraḥ??/        tasmin dṛṣṭeḥ sthairyakāraṇāt ṣaṣṭhe māsi svayaṃ tejaḥ pratyakṣaṃ bhavati tejasaḥ pratyakṣatve pārthiva sakalaṃ bandhanaṃ trudyati// \N2
%-----------------------------
%
%----------------------------
%pañcadaśo bhruvormadhyādhārastasmin dṛṣṭeḥ sthirīkaraṇāt koṭikiraṇāḥ sphuraṃti/ \E
%paṃcadaśo bhruvormadhye dhāraḥ// tasmin ḍṛṣṭeḥ sthirikaraṇāt// koṭikiriṇā sphuraṃti// \B
%pañcadaśo bhruvormadhye ādhāraḥ/ asmin dṛṣṭeḥ sthirīkaraṇāt koṭikiraṇāni sphuraṃti/ \N1
%pañcadaśo bhruvormadhye ādhāraḥ tasmin dṛṣṭeḥ sthirīkaraṇāt koṭikiraṇāni sphuraṃti/ \N2 [S.9]
%-----------------------------
%
%----------------------------
%ṣoḍaśo netrādhāraḥ/ ayamaṃgulyagreṇa cālyate/ tadabhyāsāt/ pṛthvīmadhye yatkiṃcintejo [p.45] varttate/ \E
%ṣoḍaśo netrā// ayamaṃgulyagreṇa cālyate// tadabhyāsāt pṛthivīmadhye yatkiṃcittejo vartate// \B %%%%%%%%%%%%%%%%DSCN7167.jpg Z. 1
%ṣoḍaśaḥ netrādhāraḥ/ ayaṃ agulyagreṇa cālyate/ tadabhyāsāt pṛthvīmadhye yatkiṃcittejaḥ varttate/ \N1
%ṣoḍaśaḥ netrādhāraḥ/ ayaṃ aṃgugreṇa cālyate/ tadabhyāsāt pṛthvīmadhye yatkiṃcittejaḥ varttate/ \N2
%-----------------------------
%
%----------------------------
%tatsarvaṃ tejo dṛṣṭiviṣayaṃ bhavati/ taddarśanāt puruṣaḥ sarvajño bhavati// \E
%tatsarvaṃ tejo dṛṣṭiviṣayaṃ bhavatī// taddarśanāt puruṣaḥ sarvajño bhavatī// \B
%tatsarvvatejo dṛṣṭiviṣayaṃ bhavati taddarśanāt puruṣaḥ sarvvajño bhavati// \N1
%tatsarvatejo dṛṣṭiviṣayaṃ bhavati taddarśanāt puruṣaḥ sarvajño bhavati// \N2
%
%
%-----------------------------
%
%----------------------------
%idānīm aṣṭāṃgayogavicāraḥ kathyate/ yamaniyamāsanaprāṇāyāmapratyāhāradhyānadhāraṇāsamādhir iti/ eteṣāṃ lakṣaṇāni kathyante/ \E
%idānīm aṣṭāṃgayogasya vicāraḥ kathyate/ yamaniyamāsanaprāṇāyāmapratyāhāradhāraṇādhyānasamādhir iti/ eteṣāṃ lakṣaṇāni kathyaṃte/ \B
%idānīm aṣṭāṃgayogasya vicāraḥ kathyate/ yamaniyamāsanaprāṇāyāmapratyāhāradhāraṇādhyānasamādhir iti/ eteṣāṃ lakṣaṇāni kathyate/ \L
%idānīm aṣṭāṃgayogasya vicāraḥ kathyate// yamaniyamāsanaprāṇāyāmapratyāhāradhyānadhāraṇāsamādhiyaḥ eteṣāṃ lakṣaṇāni kathyaṃte/ \N1
%idānīṃ aṣṭāṃgayogasya vicāraḥ kathyate// yamaniyamāsanaprāṇāyāmapratyāhāradhyānadhāraṇāsamādhiyaḥ eteṣāṃ lakṣaṇāni kathyaṃte/ \N2
%-----------------------------
%Now the procedure of the eightfold yoga (\textit{aṣṭāṅgayoga})is explained: "Yama, niyama, āsana, prāṇāyāma, pratyāhāra, dhyāna, dhāraṇā and samādhi." Their characteristics will be explained.   
%----------------------------
%śāntiḥ/ ṣaṇṇāmindriyāṇāṃ jayaḥ/ svalpāhāraḥ/ nidrājayaḥ/ śītoṣṇajayaḥ/ ete yamāḥ/ \E
%śāntiḥ ṣaṇāṃ iṃdriṇāṃ jayaḥ// ahāraḥ svalpaḥ nidrāyā jayaḥ// śaityajayaḥ/ uṣṇājayaḥ// ya te yamaḥ// \B
%śāntiḥ/ ṣaṇṇāṃ iṃdriyāṇāṃ jayaḥ// ahāraḥ// svalpaḥ// nidrāyāḥ jayaḥ/ śaityajayaḥ/ uṣṇajayaḥ ya te yamaḥ/ \L
%śāntiṣaṇṇāṃ indriyāṇāṃ jayaḥ/ svalpāḥ nidrājayaḥ/ śītyajayaḥ/ uṣṇajayaḥ/ ete yamāḥ/ \N1
%śāntiṣaṇṇāṃ indriyāṇāṃ jayaḥ/ ahāraḥ svalpāḥ nidrājayaḥ/ śaityajayaḥ uṣṇajayaḥ/ ete yamāḥ/ \N2
%----------------------------
%These are the Yāmas: Peace, conquer of the senses, little food, conquer of sleep, conquer of cold and heat.
%----------------------------
%niyamāḥ khalu cāpalabhāvānnivāryasthairye sthāpyate/ ekāṃte sevanam/ prāṇimātre [P.46] samā buddhiḥ/ audāsīnyaṃ kasyāpi vastuna icchā na karttavyā yathālābhasaṃtoṣaḥ/ parameśvaranāma na vismaraṇīyam/ manomadhye dainyaṃ karttavyam/ iti niyamāḥ// \E
%
%svalu cāpalabhāvānnirvārya sthāpyate// ekāṃta sevānāṃ prāṇimātre samābuddhiḥ udāsīnyaṃ kasyāpi vastunaḥ// icchā na kartavyā yathālābhasaṃtoṣaḥ/ parameśvaranāma na vismaraṇīyaṃ manomadhye dainyaṃ kartavyaṃ// iti niyamaḥ// \B
%
%niyamaḥ khalu capalabhāvānnivāryasthairye sthāpyate/ ekāṃte sevanam/ prāṇimātre [P.46] samā buddhiḥ/ udāsīnya/ kasyāpi vastunaḥ icchā na karttavyā// yathā lābhasaṃtoṣaḥ/ parameśvaranāma vismaraṇīyam/ manomadhye dainyaṃ na karttavyam/ //[S.11] \N1
%
%niyamaḥ khalū manaḥ capalabhāvānnivāryasthairye sthāpyate ekāṃtasevanam/ prāṇimātre [P.46] samā buddhiḥ udāsīnya kasyāpi vastunaḥ icchā na karttavyā/ yathā lābhasaṃtoṣaḥ parameśvaranāmavismanīyam/ manomadhye dainyaṃ na karttavyam// // \N2  \em zu vismāra 
%----------------------------
%Niyamās are truly: Keeping the mind from the unsteady state [and] ground it in calmness, retreating to a lonely place, refraining from contact to living beings, unchanging intellect, keeping equanimous one shall not crave for things, as well as being contend with what is given, never forgetting the name of the highest lord, one shall not bring the mind into depression. 
%----------------------------
%āsanalakṣaṇaṃ bahuṣu grantheṣu nirūpitamasti tenātra na nirūpyate/ \E
%āsanaṃ lakṣaṇāṃ bahūgraṃtheṣu nirūpyamasti tenātranirūpyate/ \B
%āsanasya lakṣaṇaṃ bahūgraṃthe nirūpitam/ ataḥ atrāyaṃ nirūpyate/ \N1
%āsanasya lakṣaṇaṃ bahugraṃthe nirūpitam// ataḥ atrāyaṃ nirūpyate/ \N2
%----------------------------
%The characteristic of posture has been discussed in many works and will not be discussed here.  
%----------------------------
%prāṇāyāmastu sukumāreṇa sādhituṃ na śakyate atastasya nāmamātraṃ kathyate/ \E
%prāṇāyāmastu kumāreṇa sādhituṃ na śakyate// ataḥ nāma kathyate/ \B
%prāṇāyāmastu kūmāreṇa puruṣeṇa sādhituṃ na śakyate/ ataḥ tasya nāmamātraṃ kathitaṃ/ \N1
%prāṇāyāmastu kūmāreṇa puruṣeṇa sādhituṃ na śakyate// ata tasya nāmamātre kathitaṃ/ \N2
%----------------------------
%Breath-control can't be practiced by young persons. That's why it is just mentioned by name. 
%----------------------------
%pratyāhāraḥ pratyato manaḥ saṃsārānnivartyātmani sthāpyate// manomadhye [P.47] ye vikārā utpadyante/ tepi nivāraṇīyāḥ/ anekacamatkāriṇī buddhirutpadyate/ sāṃgopāṃgaṃ dhyānaṃ ca bahutaraṃ prāguktam/ tenātra nocyate// \E XX! this one?!%pratyāhāraḥ kathyate// manaḥ saṃsārān nivṛtyātmanī sthāpyate// manomadhye ye vikārā utpadyaṃte tepi nivāraṇīyā anekacamatkāriṇī buddhir utpadyate/ sāgopyā// dhyānaṃ ca bahutaraṃ prāguktam tenātra nocyate// \B 

%pratyāhāraḥ kathyate// manaḥ saṃsārān nivṛtya ātmani sthāpyate/ manomadhye ye vikārā utpadyante/ tepi nivāraṇīyāḥ/ anekacamatkārakarakāraṇī buddhi utpadyate sāṃgopyāḥ/ dhyānaṃ ca bahutaraṃ uktam tena atra nocyate/ \N1
%pratyāhāraḥ kathyate// manaḥ saṃsārānnivṛtya ātmani vāraṇīyāḥ// anekacamatkārakarakāraṇī buddhi utpadyate sāgopyāḥ/ dhyānaṃ ca bahuttaraṃ uktam tenātra nocyate// \N2
%-----------------------------
%Pratyāhāra is [when] the mind is intend on escaping from Saṃsāra and caused to remain in the self. Changes within the mind arise, but they are [supposed to be] kept off. Not just one miracle arises in the buddhi. They are secret. Dhyāna has been taught many times. Because of that is not discussed here.     
%----------------------------
%idānīṃ piṃḍabrahmāṃḍayoraikyamasti tasmāt brahmāṇḍamadhye ye padārthāstepi piṃḍamadhye santīti kathyante/ padastale talaṃ varttate/ pādopari talātalaṃ varttate/ gulphayormahātalaṃ varttate/ jaṃghāmadhye sutalaṃ varttate/ jānumadhye vitalaṃ varttate/ ūrvormadhye'talaṃ varttate// \E %[P.48]
%idānīṃ piṃḍabrahmāṃḍayor ekyam asti// tasmā brahmāṃḍamadhye ye padārthāste piṃḍamadhye sati kathyate// pādayas talās talaṃ vartate// pādopari talātalaṃ vartate/ gulphayor mahātalaṃ vartate// jaṃghāmadhye stutalaṃ vartate// jānubhyāṃvitalaṃ vartate// ūrvo madhye atalaṃ vartate// \B %%%%%%%%%%%%DSCN7168.jpg Z.2
%idānīṃ piḍabrahmāḍayoḥ aikyam asti// tasmāt brahmāṇḍamadhye ye padārthāḥ tepi piṃḍamadhye saṃti// te kathyante// padayoraṃguṣṭale talaṃ varttate/ nādupari talātalaṃ varttate/ gulpho parimahātalaṃ varttate/ jaṃghāmadhye sutalaṃ/ jānvomadhye vitalaṃ/ ūrvormadhye atalaṃ// \N1
%idānīṃ piṇḍabrahmāḍayoḥ ekamasti/ tasmānte brahmāṇḍamadhye ye padārthā tepi piṇḍamadhye saṃti te kathyante// padayoraṃguṣṭale talaṃ varttate tādupari talātalaṃ varttate/ gulpho parimahātalaṃ varttate jaṃghāmadhye sutalaṃ/ jānvomadhye vitalaṃ/ ūrvormadhye atalaṃ// \N2
%-----------------------------
%
%----------------------------
%idānīṃ śarīramadhye lokatrayaṃ kathyate/ mūlādhāre bhūr lokaḥ/ liṃgāgre bhuvar lokaḥ/ liṃgamadhye svarlokaḥ// \E
%idānīṃ piḍopiri lokatrayaṃ kathyate// mūlādhāre bhūr lokaḥ liṃgāgre bhuvar lokaliṃgamadhye svarlokaḥ// \B
%idānīṃ piṃḍamadhye lokatrayaṃ kathyate/ mūlādhāre bhūr lokaḥ/ liṃgamūle svarlokaḥ   \N1
%idānīṃ piṃḍamadhye lokatrayaṃ kathyate/ mūlādhāre bhūr lokaḥ liṃgmūle svargalokaḥ// \N2
%-----------------------------
%
%----------------------------
%idānīm uparitanaṃ lokacatuṣka kathyate/ pṛṣṭhadaṃḍāṃkure maharlokaḥ/ daṇḍacchidramadhye janalokaḥ/ taddaṇḍanāḍīmadhye tapolokaḥ/ daṇḍamalamadhye satyalokaḥ/ \E
%idānīm uparitanulokaḥ catuṣṭayaṃ kathyate// daṃḍaṣṭaṭheṃ skure maharlokā/ daṇḍachidramadhye janaloka taddaṃḍanālikāmadhye ..polokaḥ daṇḍakamalamadhye satyalokaḥ// \B
%idānīṃ uparijanaṃ                  lokacatuṣkaṃ kathyate/pṛṣṭhadaṃḍāṃ kure maharllokaḥ/ daṇḍacchidramadhye janalokaḥ/ taddaṇḍanālī \om \N1 !!!!!!!!!!!!!!!!!!!!!!important omission stemmapoint
%idānīṃ uparijanaṃ                  lokacatuṣkaṃ kathyate// pṛṣṭhadaṃḍākūle maharllokaḥ/ uchidramadhye janalokaḥ/ taddaṇḍanālī - - - - - [text indicates lacunae in Vorlage] \om \N1 !!!!!!!!!!!!!!!!!!!!!!important omission stemmapoint
%-----------------------------
%
%
%
%
%
%
%
%
%
%
%
%
%
%
%----------------------------
%atha brahmāṇḍamadhye caturdaśalokāni sthānāni tānyapi piṃḍe varttante// \E
%atha brahmāṇḍamadhye caturdaśalokasthānānī tānyapi piṃḍo vartate... \B
%----------------------------
%śarīramadhye dvau kukṣī dve sakthinī vakṣaḥsthalaṃ kaṃṭhamūlaṃ kaṃṭhamadhyaṃ laṃbikāmūlaṃ tāludvāraṃ tālumadhyaṃ lalāṭamadhye śrṛṃgāṭikā kapolamadhye kamalinīmadhye brahmaraṃdhra kamalinyastrikūṭasthānam/ \E
%śarīramadhye// dvau kukṣau dve sakṭhinī vakṣaḥsthalaṃ kaṃṭhamūlaṃ kamardhye laṃbikāmūlaṃ tāludvāraṃ tālumadhyaṃ lalāṭamadhyaṃ//śṛṃgāṭikā// kapolamadhye// kamalinīmadhyaṃ brahmaraṃdhraṃ kamalīnyāṃ strikūṭasthānam// \B
%----------------------------
%evam ekaviṃśati sthāne ekaviṃśatibrahmāṃḍāni vasaṃti// \E
%ekamekaṃviṃśasthāne kekaviṃśabrahmāḍānī vasaṃtī// \B
%-----------------------------
%
%----------------------------
%idānīṃ saptadvīpāni piṃḍamadhye kathyante// \E
%idānī satyadvīpāni piṃḍamadhye kathyate// \B
%----------------------------
%majjāmadhye jaṃbudvīpaḥ। asthimadhye śākadvīpaḥ śirāmadhye śālmalidvīpaḥ/ \E
%majjāmadhye jaṃbudvīpaḥ/ astimadhye śākaladvīpaḥ// śirāmadhye śākaladvīpaḥ// \B
%----------------------------
%māṃsamadhye kuśadvīpaḥ/ tvacāmadhye krauṃcadvīpaḥ/ śarīrasthalomamadhye gomedadvīpaḥ/ nakhamadhye puṣkaradvīpaḥ// etāni dvīpāni madhye tiṣṭhanti// \E [p.50]
%māṃsamadhye kuśadvīpaḥ tvacāmadhye krauṃcadvīpaḥ// śarīrasya lomamadhye gomedadvīpaḥ// nakhamadhye puṣkaradvīpaḥ// etāni dvīpāni guptānimadhye tiṣṭhaṃti// \B
%-----------------------------
%
%----------------------------
%idānīṃ piṃḍamadhye saptasamudrāḥ kathyante// prasvedamadhye kṣārasamudraḥ/ lalāṭamadhye kṣīraḥ samudraḥ/ vāṅmadhye madhusamudraḥ/ kaphamadhye dadhisamudraḥ/ medomadhye ghṛtasamudraḥ/ rasamadhye ikṣurasasamudraḥ// vīryamadhye svādusamudraḥ/ pādamadhye kūrmasthānam// \E
%
%idānīṃ piṃḍamadhye samudrāḥ kathyate// prasvedamadhye kṣārasamudraḥ// lalāṭamadhye kṣīrasamudraḥ// raktamadhye vasāmadhye madasamudraḥ kaphamadhye dadhisamudraḥ// medomadhye ghṛtasamudraḥ// ikṣusamudraḥ/ vīryamadhye svādukasamudraḥ/karmasthāna pādasamadhye/ \B
%-----------------------------
%
%----------------------------
%idānīṃ navadvāreṣu nāsikayoḥ kinnarakhaṃḍanaraharikhaṃḍauḥ netrayoḥ ketumāla bhadrāśvau/ karṇayoḥ hiraṇmayakhaṃḍaramyakakhaṃḍau/ gude kurukhaṃḍaḥ liṃge ilāvṛtakhaṇḍaḥ// \E  [p.51]
%idānīṃ navakhaṃḍāni kathyaṃte/ mukhe bharatakhaṃḍaḥ nāsikayor madhye kināraharikhaṃḍā/ netrayo ketumāla bhadrāsve/ karṇayor hiraṇyamayaramyakhaṃḍaḥ/ gude kurukhaṃḍāḥ/ liṃge iḍṛttaṃ??/ \B DSCN7169.JPG Z.4 
%-----------------------------
%
%----------------------------
%idānīmaṣṭamakulaparvatāḥ kathyante/ \E
%idānīm aṣṭamakulaparvatāḥ kathyaṃte// \B
%-----------------------------
%
%----------------------------
%merudaṇḍamadhye merumaṃdaraḥ/ brahmakapāṭamadhye kailāsaḥ/ \E
%merudaṇḍamadhye merumaṃdaraḥ/ brahmakapāṭamadhye kailāsaḥ/ \B
%-----------------------------
%
%----------------------------
%pṛṣṭhamadhye himācalaḥ/ vāmaskandhe malayācalaḥ/ dakṣiṇaskandhe mandarācalaḥ/ dakṣiṇakarṇe vindhyācalaḥ/ \E
%pṛthvīamadhye himācalaḥ/ vāmaskaṃdhe malayācalaḥ/ dakṣiṇaskaṃdhe maṃdarācalaḥ/ dakṣiṇakarṇe vindhyācalaḥ/ \B
%-----------------------------
%
%----------------------------
%vāmakarṇe mainākaḥ/ lalāṭamadhye śrīśailaḥ/ apare śailāḥ hastayoḥ pādayoraṃgulīnāṃ mūleṣu varttaṃte// \E
%vāmakarṇe mainākaḥ/ lalāṭamadhye śrīśailāsaḥ/ apare śailā hastayoḥ/ pādayoraṃgulimūleṣu vartate// \B
%-----------------------------
%
%----------------------------
%idānīṃ śarīramadhye nava nāḍyastiṣṭhanti tanmadhye navanadīnāṃ sthānāni varttante/ \E
%idānīṃ śarīre navanaḍyas tiṣṭhanti// tanmadhye navānāṃ nadīnāṃ sthānāni vartate/ \B
%-----------------------------
%
%----------------------------
%gaṃgāyamune vitastā candrabhāgā sarasvatī vipāśā śatahradā irāvatī narmadā/ \E [p.52]
%gaṃgāyamunā vitastā caṃdrabhāgā sarasvatī vipāśā śāśatahṛdā irāvati narmadā/ \B
%-----------------------------
%
%-----------------------------
%aparā nadyo nadāni srotāṃsi taṭākāni vāpīkūpādi saptatisahasranāḍīmadhye tiṣṭhanti/ \E
%aparā nadyo nadānir jñārāstyetāṃ sītaṭānī vāpīkūpādvisaptatī sahasranāḍīnāmadhye tiṣṭaṃti/ \B
%-----------------------------
%
%-----------------------------

%saptaviṃśatinakṣatrāṇi dvisaptatikoṣṭhakābhyantare vasaṃti। dvādaśa rāśayaḥ। meṣaḥ vṛṣaḥ mithunaḥ karkaḥ siṃhaḥ kanyā tulā vṛściko dhanurmakarakumbhamīnāḥ। nava grahāḥ। ādityasomamaṃgalabudhaguruśukraśanirāhuketavaḥ। paṃcadaśatithayotra madhye vasaṃti।।
%[P.53]
%yathā samudramadhye laharī varttate। tathā śarīramadhye kūrmmīnāma laharī bhavati। ūrmyaścalāstataḥ calanaṃ bhavati। tanmadhye samagraṃ tārāmaṇḍalaṃ varttate। trayastriṃśatkoṭidevatāḥ। bāhuromamadhye vasaṃti। hṛdayaromamadhye takṣakaḥ mahānāgaḥ। śaṃkhaḥ takṣakaḥ। vāsukiḥ। anantaśeṣaḥ ete nāga vasaṃti। udararomamadhye apare nāgā vasaṃti guṇagandharvakinnarāpsaro vidyādharaguhyakāḥ। śarīramadhye anekatīrthāni vasaṃti।।
%aśrupātamadhye meghamaṇḍalaṃ vasati। anaṃtāḥ siddhayo buddhayaśca prakāśamadhye varttante। caṃdra-
%[p.54]
%sūryau dvayornetrayormadhye varttete। anekavanaspatigulmalatātṛṇāni jaṃghāromamadhye vasaṃti। puruṣasya nṛtyadarśanāt gītaśravaṇāt। vallabhavastuno darśanāt। yaḥ ānanda utpadyate saḥ svargalokaḥ kathyate। rogapīḍito durjanebhyaḥ puruṣasya yat duḥkhamutpadyate tadbahutaraṃ narakaṃ kathyate।।
%atha ca yatkarmakaraṇāt manomadhye śaṃkā na bhavati tatkarma muktikāraṇam।
%-----------------------------
%
%----------------------------
%idānīṃ rājayogāccharīre yādṛśāni cihnāni bhavanti tāni kathyante। sakalaroganāśaḥ sakalapṛthvīṃ paśyati। tadanaṃtaraṃ jñānamutpadyate।।
%[p.55]
%samagrā bhāṣā jānāti। tataḥ puruṣasya deho vajramayo bhavati। sarpadaṃśena maraṇaṃ na bhavati। tataḥ puruṣasya bubhukṣāpipāsānidrollatāśītoṣṇatā bādhāṃ na kurvanti। vāksiddhirbhavati। vidyatpāte kācidbādhāpi na bhavati।।
%tadanaṃtaraṃ pavanarūṣī puruṣī bhavati। samagrāṃ pṛthvīṃ dṛṣṭyā paśyati। aṇimādyaṣṭasiddhirbhavati। mahāpadmādyā nava nidhayaḥ samīpa āgacchanti। ākāśamadhye daśasu dikṣu gamanāgamane bhavataḥ balaṃ bhavati। parameśvaraṃ 
%[P.56]
%samīpe paśyati। karaṇe haraṇe sāmarthyaṃ bhavati।।
%idaṃ guru bhakteḥ phalaṃ ātmamadhye manaso viśrāmakaraṇamicchatā puruṣeṇa sadguroḥ sevāṃ kṛtvā sāvadhānaṃ manaḥ karaṇīyam। abhyāsabalāt paramaprāptiḥ। tena svaśiṣyamanasaḥ svāsthyaṃ karttavyam। candrasūryyau yāvatpiṃḍe niścalau bhavataḥ।।
%[p.57]
%samyaksvabhāvakiraṇodayacidvilāsagrastaṃ svaśāṃtisamatāṃ svayameva yāti। graste svaveganicaye padapiṃḍamaikyaṃ satyaṃ bhavetsamarasaṃ guruvatsalāṃ ca।। 1।।
%-----------------------------
%
%----------------------------
%idānīm avadhūtapuruṣasya lakṣaṇaṃ kathyate। yasya haste dhairyadaṇḍaḥ kharparaṃ śūnyamāsanam। yogaiśvaryeṇa saṃpannaḥ sovadhūta udātdṛtaḥ।। 2।।
%bhedābhedau yasya bhikṣā bharaṇaṃ jāraṇaṃ tathā। 
%etādṛśopi puraṣaḥ sovadhūta udātdṛtaḥ।। 3।।
%[p.58]
%ātmā hyakāro vijñeyo vakāro bhavavāsanā।
%dhūtaṃ saṃtāpanaṃ proktaṃ sovadhūto nigadyate।। 4।।
%akārārtho jīvabhūto vakārārthotha vāsanā।
%etaddūyaṃ japaṃ kuryātsovadhūta udāhṛtaḥ।। 5।।
%yaḥ puruṣo dvitīyaṃ na paśyati kevalaṃ svasvarūpaṃ paśyati sovadhūtaḥ। athavo yasya manaścaṃcalabhāvaṃ na dadhāti sovadhūtaḥ kathyate। yanna dṛśyate tadavyaktamityucyate। tadavyaktaṃ pratyakṣeṇa paśyati। yatkiṃcidṛśyate tatsarvaṃ grastā%timuktamiti jñānaṃ paśyati। sovadhūtaḥ kathyate। avadhūtatanuḥ somo nirākārapade 
%%p.59]
%sthitaḥ। sarveṣāṃ darśanānāṃ ca svasvarūpaṃ prakāśyate।। 1।। satyamekamajaṃ nityamanaṃtamakṣayaṃ dhruvam। jñātvā hyevaṃ vadeddhīmān satyavādī sa kathyate।। 2।।
%yatkiṃcinna paśyati, sa eko hyevaṃ manaso vijānāti nāśā na tādṛśaṃ padārthaṃ jñātvā kāle ceṣṭā bhavati। sa satyavādī kathyate।।
%[p.60]
%vāsvare bhāsvare śaktiḥ saṃkoco bhāsvarepi ca। tayoḥ saṃyogakarttā yaḥ sa bhavetsatyayogabhāk।। 3।।
%viśvānītatayā viśvam ekam eva virājate। 
%saṃyogo na sadā yasya siddhayogī sa gadyate।। 4।।
%sarvāsāṃ nijavṛttīnāṃ vismṛtīrbhajate ttu yaḥ।
%sa bhavetsiddhasiddhānto siddhayogī sa gadyate।। 5।। %This quote stems from the Siddhasiddhāntapaddhati 6.66
%[p.61]
%udāsīnaḥ sadā śānto brahmānandamayopi ca। yo bhavetsiddhayogena siddhayogī sa kathyate।। 6।।
%adhunā kamalānāṃ tu śrṛṇu saṃketamadbhutam। anekākārabhedotthaṃ kaṃ svarūpātmakaṃ malam। kamalaṃ tena vikhyātaṃ trividhaṃ tatra dehagam।। 7।।
%ādhārakamalamasya kamalamiti kaṃ kasmāt। kamātmā tasmātkamalamiti saṃjñā asyā-
%[p.62]
%dhāraḥ kamaladalasya catuṣṭayaṃ bhavati। prathamaṃ sattvaguṇasya dvitīyaṃ rājayogasya tṛtīyaṃ tamoguṇaḥ caturtho dale manastiṣṭhati। etaddalacatuṣṭayaṃ ca saṃgādātmā sādhu karoti। tasmin kamale niścalīkṛte sati puruṣasya samīpe maraṇaṃ% na gacchati।।
%-----------------------------
%
%----------------------------
%idānīṃ tdṛyakamalabhedāḥ kathyaṃte। asya dvādaśadalāni siddhapuruṣāḥ kathayaṃti। tathā dviṣaśaktistṛtīyalokāṃtaḥ samyak samudrā khecarī cidānandādvayaścandracaṃdrikā vetināmānvitaḥ।।
%[P.63]
%paramātmanā saha raśmipuṃjaprakāśaḥ prakāśānandayoraikyaṃ prakarttavyaṃ nirantaraṃ svayaṃ manasi mahājyotirābhāti paramaṃ padam।।
%sadoditamanaścandraḥ sūryodayamavekṣate। tena grasto manaścandraḥ sopi lipyaḥ svayaṃ pade।।
%padameva mahānagniryame grastaṃ kalāmayam। evaṃ candrārkavahnīnāṃ saṃketaḥ paramārthataḥ।।
%[p.64]
%-----------------------------
%
%----------------------------
%idānīṃ yogasiddheranaṃtarametādṛśaṃ jñānamatpadyate। yadā nāsti svayaṃ karttā kāraṇaṃ na kulākulam ।। avyaktaṃ na paraṃ tattvamanāmā vidyate tadā।। 1।।
%anāmā ekaḥ kaścitpuruṣo varttate। anāmnaśca parāvaraḥ parātparaḥ paraṃ padaṃ paramapadātparaṃ śūnyaṃ śūnyānniraṃjanamanāmnaḥ paṃcaguṇāsteṣvanutattvam akhaṇḍatva manuparṇadalānām aṣṭadalānāṃ madhya ekaṃ kaṭhinaṃ bhavati। tadaṣṭadalaṃ kamalaṃ hṛdaye tiṣṭhati। te ubhaye hṛdaye tiṣṭhataḥ। prathame dale śabdāstiṣṭhanti। dvitīyadale sparśaḥ। tṛtīye dale rūpaṃ tiṣṭhati। caturthe dale rasastiṣṭhati। paṃcame dale 
%[p.65]
%gandhaṃ tiṣṭhati। paṣṭhadale cittaṃ tiṣṭhati। saptame dale buddhistiṣṭhati। aṣṭame dalehaṃkārastiṣṭhati। etadaṣṭadalamadhye pṛthivyākāro varttate। atha ca tatkamalamadhye mukhaṃ tiṣṭhati। asya kamalasya nādātprakāśo bhavati।।
%prakāśānaṃtaraṃ kamalamūrdhvamukhaṃ bhavati। tathā sūryaprakāśānantaraṃ tadā saromadhye kamalaṃ
%[p.66]
%vikasati। tathedamapyātmā prakāśānantaramūrdhvamukhaṃ vikasati। tanmadhye paramānandarūpā bhūmirbhavati। tasyāhaṃ sohamiti saṃjñā tasyā madhye svātmano dhyānāddinedine hyāyurvarddhate। rogo dūre bhavati।।
%guṇāḥ kartṛtvaṃ jñātṛtvamabhyāsatvaṃ kalatvaṃ sarvajñatvaṃ prakāśasya guṇāḥ sakalaḥ niṣkalaḥ sarvaiḥ saha samatā viśrāṃtiḥ tata etādṛśamutpadyate। ādyaḥ ātmā ātmana ākāśaḥ ākāśādvāyuḥ vāyostejaḥ tejaso jalaṃ jalāt pṛthvī। atrātmanaḥ pañcaguṇāḥ agrāhyaḥ, anantaḥ, avācyaḥ, agocaraḥ, 
%[p.67]
%aprameyaśca। ākāśasya pañcaguṇāḥ। praveśaḥ niṣkramaṇaṃ, chiṃdraṃ, śabdādhāraḥ, bhrāṃtinilayatvam। mahāvāyoḥ pañcaguṇāḥ। calanaṃ śeṣasaṃcāraḥ, sparśaḥ, dhūmravarṇatā, tejaḥ saṃcaraḥ tejasaḥ pañcaguṇāḥ। dahanaṃ, jvālarūpaṃ, uṣṇatā, rakto varṇaḥ।।
%apāṃ paṃca guṇāḥ। pravāhaḥ śithilatā dravaḥ madhuratā śvetavarṇaḥ। pṛthivyāḥ paṃca guṇāḥ।
%[p.68]
%sthūlatā sākāratā kaṭhinatā gandhavattā pītavarṇatā avayavatvamananyatvaṃ ceti। parāvarasya paṃca guṇāḥ – niścalatvaṃ niṣkarmatvaṃ paripūrṇaṃtvaṃ vyāpakatvamakalatvaṃ ceti। paramapadasya paṃca guṇāḥ nityaṃ nirantaraṃ nirākāraṃ nirniketanaṃ niścalatvaṃ ceti। śūnyasya pañcaguṇāḥ – līnatā ghūrṇatā mūrchā unmanībhāvaḥ alasatvaṃ ceti। niraṃjanasya paṃca guṇāḥ satyā sahabhāvā sattā svarūpatā samatā ceti।
%-----------------------------
%
%----------------------------
%idānīṃ piṃḍotpattiḥ kathyate।।
%[p.69]
%anāditaḥ paramātmā paramātmanaḥ paramānaṃdaḥ paramānaṃdātprabodhaḥ prabodhāccidudayaḥ cidudayātprakāśaḥ। tatra paramātmanaḥ paṃca guṇāḥ – akṣayaḥ, abhedyaḥ, acchedyaḥ, adāhyaḥ, avināśī। paramānaṃdasya paṃcaguṇāḥ – sphu raṇaḥ, kiraṇaḥ, visphuraṇaḥ, ahaṃtā, harṣavattvam। prabodhasya paṃca guṇāḥ – layaḥ, ullāsaḥ, vibhāsaḥ, vicāraḥ, prabhā। cidudayasya paṃca śarīramadhye paṃca mahābhūtāni।।
%[P.70]
%teṣāṃ guṇāḥ kathyante tatra pṛthivyā guṇāḥ – asthimāṃsanāḍīlomāni vāk। tatrodakaguṇāḥ- lālā, mūtraṃ, śuklaṃ, raktaṃ, prasvedaḥ। tejaso guṇāḥ- kṣudhā tṛṣā nidrā glāniḥ ālasyam। vāyorguṇāḥ - dhāvanaṃ majjanaṃ nirodhanaṃ prasāraṇamākuṃcanaṃ ceti। ākāśasya guṇāḥ – rāgadveṣau bhayaṃ lajjā mohaḥ। tadanaṃtaramekādaśīkā buddhirutpadyate। manobuddhyihaṃkārāścittaṃ caitanyaṃ ceti। ete paṃcaprakārā aṃtaḥ karaṇasya । manasaḥ ye ca guṇāḥ saṃkalpavikalpamūrkhatvālasatā mananaṃ ceti।।
%[p.71]
%buddheḥ paṃca guṇāḥ। viveko vairāgyaṃ śāntiḥ santoṣaḥ kṣamā ceti। ahaṃkārasya paṃca guṇāḥ। ahaṃ mameti etasya duḥkhaṃ svataṃtratā। cittasya paṃcaguṇāḥ। dhṛtiḥ smṛtiḥ। rāgadveṣau matiḥ। caitanyasya paṃcaguṇāḥ। ārṣaṃ vimarśaḥ dhairyaṃ ciṃtanaṃ nispṛhatvam। ataḥ paraṃ kulapaṃcakasya bhedāḥ kathyante। sattvaṃ rajaḥ tamaḥ kālaḥ jīvanam। tatra sattvaguṇāḥ। dayā dharmaḥ kṛpā bhaktiḥ śraddhā ceti। rajasoguṇāḥ। tyāgaḥ। bhogaḥ śṛṃgāraḥ svārthaḥ। vastusaṃgrahaśceti।।
%[p.72]
%tamaso guṇāḥ vivādaḥ kalahaḥ śokaḥ baṃdhaḥ vañcanam। kālasya guṇāḥ kalanā kalmaṣaṃbhrāntiḥ prasādaḥ unmādaḥ। jīvasya guṇāḥ jāgradavasthā svapnāvasthā suṣuptāvasthā turīyāvasthā। turīyātītāvasthā tadanaṃtarametādṛśamekajñānamutpadyate। icchāyāḥ paṃca guṇāḥ। unmanyavasthā। vāṃchā cittaṃ veṣṭanam vibhramaḥ। kriyāyāḥ paṃca guṇāḥ। smaraṇaṃ udyamaḥ udvegaḥ। kāryaniścayaḥ। satkulācāratvam।।
%[p.73]
%māyāyāḥ paṃca guṇāḥ। madamātsaryādayaḥ। kīrtiḥ asatyabhāvāḥ। prakṛteḥ paṃca guṇāḥ āśā tṛṣṇā spṛhā kāṃkṣā mithyātvam। vācāyāḥ paṃca guṇāḥ। parā paśyantī madhyamā vaikharī। mātṛkā tadanaṃtarametādṛśaṃ jñānamutpa dyate। karmakāraḥ। candraḥ। sūryaḥ। agniḥ eta tpaṃcakaṃ pratyakṣaṃ karttavyam tatra karmaṇaḥ paṃcaguṇāḥ kāmasya guṇāḥ ratiḥ prītiḥ krīḍā kāmanā anustutā।।
%[p.74]
%-----------------------------
%
%----------------------------
%idānīṃ caṃdrasya ṣoḍaśakalāḥ kathyante। dallolā kallolinī uścalinī unmādinī taraṃgiṇī poṣayaṃtī laṃpaṭā laharī lolā lelihānā prasarantī pravṛttiḥ plavantī pravāhā saumyā prasannā।।
%candrasya saptadaśamī kalā varttate tasyā nāma nivṛttisametā kalā kathyate.
%-----------------------------
%
%----------------------------
%idānīṃ sūryasya kalāḥ kathyante। tapanī grāsakā ugrā akocanī śoṣaṇī prabodhinī ghasmarā ākarṣiṇī tuṣṭivarddhinī kūrmī reṣā kiraṇavatī 
%[p.75]
%prabhavati sūryasya trayodaśī kalā vidyate।
%tasya nāma nijakalāsvaprakāśā ca।।
%-----------------------------
%
%----------------------------
%idānīm agnisaṃbaṃdhinyo daśa kalāḥ kathyante। dīpikā jvālā visphuliṃginī pracaṃḍā pācikā raudrī dāhikā rāvaṇī। śikhāvatī । agnerekādaśī nijakalā jyotiḥ saṃjñā varttate।।
%-----------------------------
%
%----------------------------
%idānīṃ yogasya māhātmyaṃ kathyate। guroranugrahāt śāstrasya paṭhanāt ācārakaraṇāt
%[P.76]
%vedāṃtarahasya śravaṇāt dhyānakaraṇāt upavāsakaraṇāt caturaśītyāsane sādhanāt vairāgyasyotpatteḥ nairāśye karaṇāt haṭhayogasya karaṇāt iḍāpiṃgalayoḥ pavanadhāraṇāt mahāmudrādidaśamudrāsādhanāt maunakaraṇāt vanavāsāt bahutarakleśakaraṇāt bahukālayaṃtramaṃtrādisādhanāttapaḥ karaṇāt bahutarārpaṇadānāt āśramācārapālanāt saṃnyāsagrahaṇāt ṣaḍdarśanagrahaṇāt śiromuṃḍanāt anyopāyakaraṇāt yogatattvaṃ na prāpyate।।
%[p.76]
%sa tu yogaḥ gurusevayā prāpyate। gurukṛpātaḥ pātrāṇāṃ dṛḍhānāṃ satyavādinām। kathanādṛṣṭipātādvā sāṃnidhyādavalokanāt। sadguruprasādāt samyak paramaṃ padaṃ pāpyate। ata evaṃ vacaḥ proktaṃ na guroradhikaṃ param।। 1।।
%vāṅmātrādbodhadṛkpātādyaḥ karoti śamaṃ kṣaṇāt। prasphuṭadbhrāṃtihṛttoṣaṃ svacchaṃ vaṃde guruṃ param।। 2।।
%[p.78]
%samyagānandajananaḥ sadguruḥ sobhidhīyate। nimeṣārddhaṃ vā tatpādaṃ yadvākyādavalokanāt।। 3।।
%svātmā sthiratvamāyāti tasmai śrīgurave namaḥ। nānāviplavaviśrāntiḥ kathanātkuru te tataḥ।। 4।। sadguruḥ sa tu vijñeyo na tu vai priyajalpakaḥ।। 5।।
%ata eva paramapadasya prāptyarthaṃ sadguruḥ sevyaḥ sarvadā yaḥ puruṣaḥ satyavādī bhavati। niraṃtaraṃ
%[p.79]
%gurusevātatparo bhavati। yasya manasi pāpaṃ na bhavati। svācārarataḥ snānādiśīlo bhavati। kāpaṭyaṃ na bhavati yasya vaṃśaparaṃparā jñāyate। etādṛśasya sadguroḥ saṃgatiḥ karttavyā tena puruṣasya manaḥ śāṃtiṃ prāpnoti। atha ca yasya manomadhye sthira ānanda utpadyate sopi sadguruḥ kathyate। kasyāpi duḥkhaṃ na dīyate। prāṇimātreṇa saha maitrī kriyate kasyāpi doṣaṃ na kathayati sopi sadguruḥ kathyate।।
%[p.80]
%ajñātakulaśīlānāṃ yatīnāṃ brahmacāriṇām।
%upadeśaṃ na gṛhṇīyādanyathā narakaṃ dhruvam।।
%yasya vacasi manasidhṛte sati svātmanaḥ parameśvarasyaikyaṃ bhavati। etādṛśo manomadhye niścayo bhavati। taṃ sadguruṃ vijānīyāt vikalpa etādaśo yathā samudramadhye mahattaraṃ kallolāḍambaram। prapaṃce vāsanā tādṛśī yathodakamadhye mahattaraṃgāḥ। tādṛśasya saṃsārasāgarasya yaḥ svavākyanāvā paraṃ pāraṃ prāpayati। sa sadguruḥ kathyate।।
%[p.81]
%yasya puruṣasya mano'khaṇḍe paramamade līnaṃ bhavati। yaḥ puruṣaḥ svakulaṃ trividhāttāpānnivartya parame muktipade rakṣati। etādṛśasya puruṣasya śravaṇāddarśanāt samagravighnā naśyanti। dinedine kalyāṇaṃ bhavati। niṣkalaṃkā buddhirutpadyate। idaṃ yogaśāstrasya rahasyaṃ samastaśāstraprameyasya manaḥ yathāṃdhakārasya madhye dīpatejaḥ praviśati। tathā śāstramadhye mano praviśati। yasya rājño madhye kalaho nāsti। yasmin dṛṣṭe deśikatrāso na bhavati। tasya manaḥ śuddhaṃ bhavati। yasya pṛthvyāṃ vītirbhavati। yasya manomadhye taspuruṣasya vaco viśvāso bhavati। yo rājā sadānaṃdarūpo bhavati।।
%[p.82]
%yasya pārśve pratyakṣam anekamanohārivastūni tiṣṭhaṃti। etādṛśasya rājña idaṃ yogarahasyaṃ kathanīyam। na snehānna bhayānna lobhānna mohānna dhanādbalānna maitrībhāvānnaudāryānna sauṃdaryānna sevanāt। sāmānyāgre yogo na kathanīyaḥ। yaḥ paraniṃdārato bhavati। durācāro bhavati। durmaitryānyasya vastu na dadāti।
%[p.83]
%ya asatyaṃ vadati। yo yoganindāṃ karoti। yasya manomadhye dayā na bhavati। yaḥ kalahapriyo bhavati। svakāryakaraṇe sāvadhāno bhavati। guroḥ kāryakaraṇe na dattacitto bhavati। etādṛśasyāgre na yogaḥ kriyate na paṭhyate।।
%śrṛṇvan prītādikān śabdān paśyan rūpaṃ manoharam। jāgrat sphuran spṛśansparśamṛdupriyam svādān manoramān bhrāmyan deśān।
%[p.84]
%manoramān bhāṣamāṇaḥ ramamāṇaḥ svalīlayā।
%bhāvābhāvavinirmukto sarvagrahavivarjitaḥ।। 1।।
%sadānaṃdamayo yogī sadābhyāsī sadā bhavet।
%viruddhaduḥkhade deśe virūpetibhayānake।। 1।।
%iṣṭādyaniṣṭasaṃsparśe rase ca lavaṇādike। pratyā-
%dāvapi gaṃdhe ca kaṃkoṣṇādi vivarjayet।। 2।।
%[p.85]
%sarvadaiva sadābhyāsaḥ samaḥ syātsukhaduḥkhayoḥ।
%evaṃ yogasya karmmāṇi saṃkalparahitāni ca।। 3।।
%gacchannṝṇāṃ ca saṃsparśāttapaḥ kurvanna lipyate।
%utpannatattvabodhasya hyudāsīnasya sarvadā।। 4।।
%tadā dṛṣṭiviśeṣaḥ syādvividhānyāsanāni ca।
%aṃtaḥ karaṇajā bhāvā yogino nopayoginaḥ।। 5।। %Amanaska!!!!
%sarvarājapadasthasya niṣkalādhyātmavedinaḥ।
%yadyatprayatnaniḥ pāyaṃ tattatsarvamakāraṇam।। 6।।
%[p.86]
%vilāsinīnāṃ manohārigānaśravaṇāt। atisauṃdaryakāminīnāṃ rūpadarśanāt। kastūrīkarpūrayorgaṃdhagrahaṇāt। manaḥśaityakāri komalavastunaḥ sparśakāraṇāt। atimādhuryaṃ citte karoti। tādṛśaḥ svādanāt। anekadeśānāṃ sādhvasādhusthānadarśanāt। mitreṇa saha komalavacanāt। śatruṇā saha kaṭhinavacanāt। yasya manasi harṣo vā dveṣo na bhavati sa puruṣa īśvaropadeśiko jñeyaḥ।।
%[p.87]
%svalīlayā vadati calati bhāvābhāvayościttamudāsīnaṃ bhavati kasyāṃcidvārtāyāṃ harṣaviṣādaṃ na karoti yasya manaḥ sahajānaṃde magnaṃ bhavati। tena puruṣeṇa dṛṣṭiḥ sthirā karttavyā। āsanaṃ dṛḍhaṃ karttavyam। pavanaḥ sthiraḥ karttavyaḥ। etādṛśaḥ kaścinniyamaḥ। siddhasya noktaḥ manaḥpavanābhyāṃ yadā sahajānaṃdasvasvarūpeṇa prakāśyate sa sahajayogaḥ kathyate। te madhye iti cakravarttikathanam।।
%
%iti śrīsarvaguṇasampannapaṃḍitasukhānandamiśrasūrisūnupaṇḍitajvālāprasādamiśrakṛtabhāṣāṭīkāsahito rājayoge binduyogaḥ samāptaḥ।। śubhamastu।। śrīrastu।।


      
%\begin{alignment}[
%    texts=edition[class="edition"];
%    translation[class="translation"],
%  ]
%\begin{edition}
% \ekddiv{type=ed}
%\begin{prose}homa\end{prose}
%\end{edition}
%\begin{translation}
%  \ekddiv{type=trans}
%  \begin{tlate}\end{tlate}
%   \end{translation}
% \end{alignment}


