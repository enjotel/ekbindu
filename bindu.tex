%Ultimatives Tool zur Datierung:
%https://www.cc.kyoto-su.ac.jp/~yanom/pancanga/
%skp = ignored in edition
%skm = ignored in xml
%%%---2-DO---%%%:
% - ENTER RDGs of N2 and make apparatus negative?!
% - ENTER RDGs of D2!!
% - add xml ids for cladistics
% - produce diplomatic transcripts for saktumiva
% - make extra layer in Apparatus for parallels in SVARODAYA, Siddhasiddhantapaddhati and Amanaska
% - check all daṇḍas!!! now I think that it's more likely that many of them were lost in copies. Lectio difficilior! Very unconventional style of the autor! 
% - read Sarvangayogapradipika, Maya Burger! 
% - maybe add second ciritical edition of yogasvarodaya?!
% - Korrekturlesen von \E!! 
% - Verspattern einbauen!
% - add all Testtimonia of SSP & Ysv
% - Sigla alphabetisch ordnen und! daṇḍas mit einkollationieren
% - präambel auslagern wie Jürgen
% - grep-search alle Verse!!!!
% - Mss spreadsheet
% - sort N1,D1,B2 zu N1,N2,D1
% - sort all sigla alphabetically 
% - additions to U2: make footnotes for the bahir mātrā-s: explaining the inventions of female deities and tell that this is "schwer interpretierbar"
% - Belege für source und testimonia einfügen!!!
% - GIVE UNIQUE LABELS for TESTIMONIO AND SOurces
% - Edition mit Sätzen übereinander nennt sich: Synoptische Edition
% - Consider changing Lakṣya to Lakṣa
% - vEREINHEITLICHUNG von souce und testium! 
%%%%%%%%%%%%%%%%%%%%%%%%%%%%%%%%%%%%%%%%%
% Don't forget
% Siddhasiddhantapaddhati Yogic Body descriptions are followed by Rāmacandra
% Quotes of the Yogasvarodaya in the Yoga Karṇikā
% Rāmacandra more a compiler than an author!!!
% Identify quotes of YTB in Haṭhasanketacandrikā 
%%%%%%%%%%%%%%%%%%%%%%%%%%%%%%%%%%%%%%%%%%%
%MSS notes
%
%--B: i and ī are not differenciated
%--P: no punctuation no daṇdas nothing
%--U1: dot . serves as daṇḍa 
%--\L and \U2 very similar
%--figure out for U2: // ajapājapaḥ sahasra // 6000 //gha 0 16 pa 0 40// \U2?!?!?!?!?!?
%%%%%%%%%%%%%%%%%%%%%%%%%%%%%%%%%%%%%%%%%%
%
% Einleitung Ideen 
% - sprachliche Simplizität
% - Potenzial als Anfängertext
% - Großartige Einführung in die Textkritik -> Synoptische Edition 
% - Gelegenheit Yogasvarodaya und Yogatattvabindu zu edieren 
% - Historische Evidenz entweder für das königliche Leben in einer Maṭha in der Nähe von Benares während der Muslimischen Herrschaft, oder sogar Lehrtext für die Bildung junger Prinzen  
% - eines der raren Beispiele der engen Verknüfung mehrerer Texte 
% - eines der raren Beispiele der Prosaisierung eines metrischen Textes 
% - Anwendung rezenter Technologie! 
% - How the text was construed -> intermingling of Ysv and SSP
% - Martin Straube: "jeder kleine Dorfhäuptling kann Rāja genannt werden". 
%%%%%%%%%%%%%%%%%%%%%%%%%%%%%%%%%%%%%%%%%%%
\documentclass[10pt]{memoir}
\setstocksize{220mm}{155mm} 	        
\settrimmedsize{220mm}{155mm}{*}	
\settypeblocksize{170mm}{116mm}{*}	
\setlrmargins{18mm}{*}{*}
\setulmargins{*}{*}{1.2}
%\setlength{\headheight}{5pt}
\checkandfixthelayout[lines]
\linespread{1.16}

%%% more functions
\usepackage[dvipsnames]{xcolor}
%\usepackage[flushmargin]{footmisc}

%%% Hyphenation settings
\usepackage{hyphenat}
\hyphenation{he-lio-trope opos-sum}
\tracingparagraphs=1
%Hyphenation in Devanāgarī of the edition still missing? Probably this needs to be modified in babel-iast package? 

%%% babel
\usepackage[english]{babel}
\usepackage{babel-iast/babel-iast}
\babelfont[iast]{rm}[Renderer=Harfbuzz, Scale=1.3]{AdishilaSan}%AdishilaSan}
\babelfont[english]{rm}{Adobe Text Pro}


%%% more functionality
\PassOptionsToPackage{hyphens}{url}
\usepackage{hyperref}
\usepackage{cleveref}
\usepackage{url}
\usepackage{cleveref}
\usepackage{microtype}
\usepackage{lineno}
%\linenumbers
%\usepackage[type=lowerleftT]{fgruler}


%%%% test better pagebreaks
\def\fussy{%
  \emergencystretch\z@
  \tolerance 200%
  \hfuzz .1\p@
  \vfuzz\hfuzz}

\interfootnotelinepenalty=10000\relax

\usepackage[maxfloats=256]{morefloats}

\maxdeadcycles=500

\raggedbottomsectiontrue
\checkandfixthelayout

%%%
%\setlength{\parskip}{2cm plus0.5cm minus0.5cm}


%Index
\usepackage[backend=biber, bibstyle=verbose, citestyle=authoryear]{biblatex}

\DefineBibliographyStrings{german}{
  references = {Literaturverzeichnis},
  bibliography = {Bibliographie},
  shorthands = {Abkürzungen der Zeugen des kritischen Apparatus},
}

\makeatletter
\renewcommand*{\mkbibnamefamily}[1]{%
  \ifdefstring{\blx@delimcontext}{parencite}
    {\textsc{#1}}
    {#1}}
\makeatother

\DeclareFieldFormat{postnote}{#1}
\renewcommand{\postnotedelim}{:}
\addbibresource{bindu.bib}

%%% ekdosis
\usepackage[teiexport=tidy,parnotes=true]{ekdosis}% =tidy cleans up HTML and XML documents by fixing markup errors and upgrading legacy code to modern standards. parnotes=footnotes below or above critical apparatus

\SetLineation{lineation=page} %lineation=pagesets thenumbering to start afresh at the top of each page. modulo makes every fifth line numbered. 

\renewcommand{\linenumberfont}{\selectlanguage{english}\footnotesize} %sets language of lines to English

\SetTEIxmlExport{autopar=false} %autopar=falseinstructs ekdosis to ignore blank lines in the.tex sourcefile as markers for paragraph boundaries. As a result, each paragraph of the edition must be found within an environment associated with the xml <p> element

\SetHooks{
  lemmastyle=\bfseries,
  %refnumstyle=\selectlanguage{english}\bfseries,
  refnumstyle=\selectlanguage{english}\color{blue}\bfseries,
  appheight=0.8\textheight,
}

\newif\ifinapparatus
\DeclareApparatus{testium}[
%bhook=\inapparatustrue,
lang=english,
notelang=english,
% bhook=\selectlanguage{english},
bhook=\selectlanguage{english}\textbf{Testimonia:},
%maxentries=4, 
%ehook=.]
%sep={: },
]

\newif\ifinapparatus
\DeclareApparatus{source}[
%bhook=\inapparatustrue,
lang=english,
notelang=english,
% bhook=\selectlanguage{english},
bhook=\selectlanguage{english}\textbf{Sources:},%
%maxentries=4, 
%ehook=.]
%sep={: },
]

% Declare \ifinapparatus and set \inapparatustrue at the beginning of
% the apparatus criticus block. Also set the language.  
\newif\ifinapparatus
  \DeclareApparatus{default}[
  %bhook=\inapparatustrue, 
  lang=english,
  %maxentries=33,
  %bhook=\selectlanguage{english},
  sep = {] },
  delim=\hskip 0.75em,
  rule=\rule{0.7in}{0.4pt},
]

\newif\ifinapparatus
\DeclareApparatus{philcomm}[
%bhook=\inapparatustrue,
lang=english,
notelang=english,
bhook=\selectlanguage{english}\textbf{Philological Commentary:},
%bhook=\selectlanguage{english},
sep={: },
]

\ekdsetup{
showpagebreaks,
spbmk = \textcolor{blue}{spb},
hpbmk = \textcolor{red}{hpb}
}

% Macros and Definitions for the Print of Sigla
\def\acpc#1#2#3{{#1}\rlap{\textrm{\textsuperscript{#3}}}\textsubscript{\textrm{#2}}\space}
\def\sigl#1#2{{{#1}}\textsubscript{\textrm{#2}}}
\def\None{{\sigl{N}{1}}} \def\Noneac{\acpc{N}{1}{ac}\,} \def\Nonepc{\acpc{N}{1}{pc}\,}
\def\Ntwo{{\sigl{N}{2}}} \def\Noneac{\acpc{N}{2}{ac}\,} \def\Nonepc{\acpc{N}{2}{pc}\,}
\def\Done{{\sigl{D}{1}}} \def\Doneac{\acpc{D}{1}{ac}\,} \def\Donepc{\acpc{D}{1}{pc}\,}
\def\Dtwo{{\sigl{D}{2}}} \def\Dtwoac{\acpc{D}{2}{ac}\,} \def\Dtwopc{\acpc{D}{2}{pc}\,}
\def\Uone{{\sigl{U}{1}}} \def\Uoneac{\acpc{U}{1}{ac}\,} \def\Uonepc{\acpc{U}{1}{pc}\,}                 
\def\Utwo{{\sigl{U}{2}}} \def\Utwoac{\acpc{U}{2}{ac}\,} \def\Utwopc{\acpc{U}{2}{pc}\,}

%%%%%%%%%%%%%% Tattvabinduyoga - List of Witnesses   %%%%%%%%%%%%%%%%%%%
\DeclareWitness{ceteri}{\selectlanguage{english}cett.}{ceteri}[]   
\DeclareWitness{E}{\selectlanguage{english}E}{Printed Edition}[]    
\DeclareWitness{P}{\selectlanguage{english}P}{Pune BORI 664}[]  
\DeclareWitness{B}{\selectlanguage{english}B}{Bodleian 485}[]       
\DeclareWitness{N1}{\selectlanguage{english}N\textsubscript{1}}{NGMPP 38/31}[]
\DeclareWitness{N2}{\selectlanguage{english}N\textsubscript{2}}{NGMPP B 38/35}[]
\DeclareWitness{L}{\selectlanguage{english}L}{LALCHAND 5876}[]  
\DeclareWitness{D}{\selectlanguage{english}D}{IGNCA 30019}[] 
%\DeclareWitness{D2}{\selectlanguage{english}D\textsubscript{2}}{IGNCA 30020}[]  
\DeclareWitness{U1}{\selectlanguage{english}U\textsubscript{1}}{SORI 1574}[] 
\DeclareWitness{U2}{\selectlanguage{english}U\textsubscript{2}}{SORI 6082}[]
%%%%%%%%%%%%% Testimonia
\DeclareWitness{Ysv}{\selectlanguage{english}Ysv}{Yogasvarodaya}[] %%%add infos!  

%%%%%%%%%%%%%%%%%%%%%%%%%%%%%%%%%%%%%%%%%%%
% Macro for Editing Abbrevs.
\def\om{\textrm{\footnotesize \textit{om.}\ }} %prints om. for omitted in apparatus
\def\korr{\textrm{\footnotesize \textit{em.}\ }} %prints em. for emended in apparatus
\def\conj{\textrm{\footnotesize \textit{conj.}\ }} %prints conj. for conjectured in apparatus

% \supplied{text} EDITORIAL ADDITION -> Within \lem oder \rdg
% \surplus{text} EDITORIAL DELETION -> Within \lem oder \rdg
% \sic{text} CRUX
% \gap{text} LACUNAE -> [reason=??, unit=??, quantity=??, extent=??]


%%%%%%%%%%%%%%%%%%%%%%%%%%%%%%%%%%%%%%%%%%% All macros of this list can be used in 
% Macro for Editing Abbrevs.
\def\eyeskip{\textrm{{ab.\,oc. }}}
\def\aberratio{\textrm{{ab.\,oc. }}}
\def\ad{\textrm{{ad}}}
\def\add{\textrm{{add.\ }}}
\def\ann{\textrm{{ann.\ }}}
\def\ante{\textrm{{ante }}} 
\def\post{\textrm{{post }}}
%\def\ceteri{cett.\,}                   
\def\codd{\textrm{{codd.\ }}}

\def\coni{\textrm{{coni.\ }}}
\def\contin{\textrm{{contin.\ }}}
\def\corr{\textrm{{corr.\ }}}
\def\del{\textrm{{del.\ }}}
\def\dub{\textrm{{ dub.\ }}}

\def\expl{\textrm{{explic.\ }}} 
\def\explica t{\textrm{{explic.\ }}}
\def\fol{\textrm{{fol.\ }}}
\def\foll{\textrm{{foll.\ }}}
\def\gloss{\textrm{{glossa ad }}}
\def\ins{\textrm{{ins.\ }}}      
\def\inseruit{\textrm{{ins.\ }}} 
\def\im{{\kern-.7pt\lower-1ex\hbox{\textrm{\tiny{\emph{i.m.}}}\kern0pt}}} %\textrm{\scriptsize{i.m.\ }}}      
\def\inmargine{{\kern-.7pt\lower-.7ex\hbox{\textrm{\tiny{\emph{i.m.}}}\kern0pt}}}%\textrm{\scriptsize{i.m.\ }}}      
\def\intextu{{\kern-.7pt\lower-.95ex\hbox{\textrm{\tiny{\emph{i.t.}}}\kern0pt}}}%\textrm{\scriptsize{i.t.\ }}}           
\def\indist{\textrm{{indis.\ }}}  
\def\indis{\textrm{{indis.\ }}}
\def\iteravit{\textrm{{iter.\ }}} 
\def\iter{\textrm{{iter.\ }}}
\def\lectio{\textrm{{lect.\ }}}   
\def\lec{\textrm{{lect.\ }}}
\def\leginequit{\textrm{{l.n. }}} 
\def\legn{\textrm{{l.n. }}}
\def\illeg{\textrm{{l.n. }}}

\def\primman{\textrm{{pr.m.}}}
\def\prob{\textrm{{prob.}}}
\def\rep{\textrm{{repetitio }}}
\def\secundamanu{\textrm{\scriptsize{s.m.}}}            \def\secm{{\kern-.6pt\lower-.91ex\hbox{\textrm{\tiny{\emph{s.m.}}}\kern0pt}}}%   \textrm{\scriptsize{s.m.}}}
\def\sequentia{\textrm{{seq.\,inv.\ }}}  
\def\seqinv{\textrm{{seq.\,inv.\ }}}
\def\order{\textrm{{seq.\,inv.\ }}}
\def\supralineam{{\kern-.7pt\lower-.91ex\hbox{\textrm{\tiny{\emph{s.l.}}}\kern0pt}}} %\textrm{\scriptsize{s.l.}}}
\def\interlineam{{\kern-.7pt\lower-.91ex\hbox{\textrm{\tiny{\emph{s.l.}}}\kern0pt}}}   %\textrm{\scriptsize{s.l.}}}
\def\vl{\textrm{v.l.}}   \def\varlec{\textrm{v.l.}} \def\varialectio{\textrm{v.l.}}
\def\vide{\textrm{{cf.\ }}}
\def\cf{\textrm{{cf.\ }}} 
\def\videtur{\textrm{{vid.\,ut}}}
\def\crux{\textup{[\ldots]} }
\def\cruxx{\textup{[\ldots]}}
\def\unm{\textit{unm.}}
%%%%%%%%%%%%%%%%%%%%%%%%%%%%%%%%%%%%

% List of Scholars
\DeclareScholar{ego}{ego}[
forename=Nils Jacob,
surname=Liersch]

% Persons:14\DeclareScholar{ego}{ego}[15forename=Robert,16surname=Alessi]17% Useful shorthands:18\DeclareShorthand{codd}{codd.}{V,I,R,H}19\DeclareShorthand{edd}{edd.}{Lit,Erm,Sm}20\DeclareShorthand{egoscr}{\emph{scripsi}}{ego}

%Useful shorthands:
%\DeclareShorthand{codd}{codd.}{V,I,R,H}
%\DeclareShorthand{edd}{edd.}{Lit,Erm,Sm}
\DeclareShorthand{egoscr}{\emph{scripsi}}{ego}
\DeclareShorthand{egomute}{\unskip}{ego}

\usepackage{xparse}

%%% define environments and commands
\NewDocumentEnvironment{tlg}{O{}O{}}{\vspace{-1ex}\begin{verse}}{\hfill #1\\ \vspace{-1ex}\end{verse}} %verse environment
%\NewDocumentEnvironment{tlg}{O{}O{}}{\begin{verse}}{॥#1\hskip-4pt ॥\\ \end{verse}}
\NewDocumentCommand{\tl}{m}{{\selectlanguage{iast} #1}}

\NewDocumentCommand{\extra}{m}{{\textcolor{MidnightBlue}{#1}}} %command for additions to U2
\NewDocumentCommand{\crazy}{m}{{\textcolor{red}{#1}}} %totally corrupted passage 

\NewDocumentEnvironment{prose}{O{}}{\begin{otherlanguage}{iast}}{\end{otherlanguage}}
% \NewDocumentEnvironment{padd}{O{}}{\begin{otherlanguage}{iast}}{\end{otherlanguage}}
\NewDocumentEnvironment{tlate}{O{}}
%\NewDocumentEnvironment{tadd}{O{}}

%Define two commands: \skp ("sanskrit plus"), to be ignored by TeX in
%the edition text, but processed in the TEI output. Conversely, \skm
%("sanskrit minus") is to be processed in the edition text, but
%ignored if found in the apparatus criticus and in the TEI output:

\NewDocumentCommand{\skp}{m}{}
\TeXtoTEIPat{\skp {#1}}{#1}

%\NewDocumentCommand{\skpp}{m}{}
%\TeXtoTEIPat{\skpp {#1}}{#1}

\NewDocumentCommand{\skm}{m}{\unless\ifinapparatus#1-\fi}
\TeXtoTEIPat{\skm {#1}}{}

\NewDocumentCommand{\dd}{}{/\hskip-4pt/}
\TeXtoTEIPat{\dd {}}{//}


%%% modify environments and commands
%%% TEI mapping
\TeXtoTEIPat{\begin {tlg}}{<lg>} %lg=(Group of verse (s)) contains one or more verses or lines of verse that together form a formal unit (e.g. stanza, chorus).
\TeXtoTEIPat{\end {tlg}}{</lg>}

\TeXtoTEIPat{\begin {prose}}{<p>}
\TeXtoTEIPat{\end {prose}}{</p>}

\TeXtoTEIPat{\begin {tlate}}{<p>}
\TeXtoTEIPat{\end {tlate}}{</p>}

\TeXtoTEIPat{\\}{}
\TeXtoTEIPat{\linebreak}{<br/>}
\TeXtoTEIPat{\noindent}{}
%\TeXtoTEI{tl}{l}
\TeXtoTEI{emph}{hi}
\TeXtoTEI{bigskip}{}
\TeXtoTEI{None}{N1}
\TeXtoTEI{Ntwo}{N2}
\TeXtoTEI{Done}{D1}
\TeXtoTEI{Dtwo}{D2}
\TeXtoTEI{Uone}{U1}
\TeXtoTEI{Utwo}{U2}
%\TeXtoTEIPat{/}{ |}
%\TeXtoTEI{//}{ ||}
\TeXtoTEIPat{\korr}{em. }
\TeXtoTEIPat{\conj}{conj.}
\TeXtoTEIPat{\om}{om.}
\TeXtoTEIPat{english}{}
\TeXtoTEIPat{\hskip}{}
\TeXtoTEIPat{\hskip-4pt}{}
\TeXtoTEIPat{\hskip-2pt}{}
\TeXtoTEIPat{-}{ }
\TeXtoTEIPat{4pt}{}
\TeXtoTEIPat{2pt}{}
\TeXtoTEIPat{\textcolor {#1}{#2}}{<hi rend="#1">#2</hi>} 

% Nullify \selectlanguage in TEI as it has been used in
% \DeclareWitness but should be ignored in TEI.
\TeXtoTEI{selectlanguage}{}

\author{Nils Jacob Liersch}
\title{Yogatattvabindu of Rāmacandra\\ A Critical Edition and Annotated Translation}
\date{\today}

\parindent=15pt
\begin{document}
\maketitle
\clearpage
\tableofcontents
\addtocounter{page}{-1}
\thispagestyle{empty}
\clearpage


\chapter{The List of the 15 Yogas}
\begin{itemize}
\item It's not entirely clear if the list given at the beginning of the text codifying the fifteen \textit{yoga}s belongs to the original text or was a later addition by a another hand. One primary reason for this possibility is the structure of the \textit{yoga}s in the actual course of the text does not equal the list. The text begins with a description of \textit{kriyāyoga} and continues to describe \textit{siddhakuṇḍaliniyoga} and somewhat suprisingly mentions \textit{mantrayoga} in the same breath. One starts wondering why the structure of the text does not follow the codification. However the mention of \textit{jñānotpattav upāyaḥ} might be a clue why the second \textit{yoga} in the list might be \textit{jñānayoga}. So far it seems to me that there are three options or a combination of these to explain these apparent inconsistencies: 1. The text is highly corrupted. 2. The codification was a later addition of another hand. 3. The term \textit{jñānayoga} is listed due to the results of \textit{siddhakuṇḍalinīyoga}, which is the generation of knowledge due to the practice of a certain \textit{yoga} involving the central channel, as mentioned in this section of the text.
\end{itemize}

\chapter{Conventions in the Critical Apparatus}
\section{Sigla in the Critical Apparatus}

\begin{itemize}
\item E : Printed Edition
\item P : Pune BORI 664
\item L : Lalchand Research Library LRL5876
\item B : Bodleian Oxford D 4587
\item \None : NGMPP B 38-31
\item \Ntwo : NGMPP B 38-35 / A 1327-14
\item \Done : IGNCA 30019
\item \Uone : SORI 1574
\item \Utwo: SORI 6082
\end{itemize}

The order of the readings in the critical apparatus is arranged according to the quality of readings in decending order. The critical apparatus is positive. Gemitation is not recorded. 

\section{Punctuation}

The very inconsistent use of punctuation marks in the witnesses at hand makes standardization necessary. A close examination of the overall usage of punctuation suggest that in the course of the texts transmission punctuations have been dropped frequently or even have been added. Particularly in the lists given in the text the copists negliance or not properly dealing with punctuation resulted in various forms of those lists with and without punctuations. Due to missing punctuation in many instances copists either made up case endings, changed the text and combined the lists' items into compounds that weren't present in the assumed original text. Even though punctuation plays a role that should'nt be underestimated, the deviation of punctuation at the end of sentences, lists and verse-numbering will only be documented in the critical apparatus of the printed edition to meaningful extend. That means, for example that emendations of obvious mistakes in punctuation will not be recorded in the critical apparatus. However, the digital edition of this work provides a way more detailled documentation of deviations in punctuation in the form of diplomatic transcripts of each witness and even a function to display sentences cummulativly on top of each other.

In the printed edition of the \textit{Tattvayogabindu} the standard conventions of punctuation are followed:

In verse poetry, a \textit{daṇḍa} marks the end of a half verse, half of the \textit{śloka}, and the double \textit{daṇḍa} marks the end of a verse. A half verse is a \textit{pāda}, at least in some literary works, this is concluded by a \textit{daṇḍa} and the end of a \textit{śloka} by a double \textit{daṇḍa}. In prose the single \textit{daṇḍa} indicates the end of a sentence and the double \textit{daṇḍa} marks the end of a paragraph.

Variations in the usage of \textit{Avagraha} will be recorded. Items of lists will be separated by a single \textit{daṇḍa}. 

\section{Sandhi}

Among the witnesses we see deviating and inconsistent application of \textit{sandhi}. There is no clear evidence that originally \textit{sandhi} was intentionally not applied. This edition will therefore apply \textit{sandhi} consistently throughout the constituted text to provide a readable text sticking to contemporary conventions in Sanskrit. The variant readings concerning \textit{sandhi} are recorded consistently in the apparatus criticus. This is due to various textcritical problems arising from the inconsistent usage of punctuation which results in application or non-application of \textit{sandhi} wheter the respective witness applied a \textit{daṇḍa} or not. This is particularly the case within lists, which frequently occur in our compilation. Items were most likely originally separated by \textit{daṇḍa}. 

\section{Class Nasals}

Again, due to inconsistent use of class nasals among the witnesses \textit{anusvāra}s have been substituted with the respective class nasals throughout the edition.

\section{Lists}

Lists are very frequent in the \textit{Yogatattvabindu}. In fact, the text initially gives a list of 15 Yogas in the beginning and many more lists are have been utilized throughout the text. Many witness lost punctuation in the process of copying and as a consequence applied \textit{sandhi}, to arrive at a consistent and conveniently readable edition of the text, all list have been identified as such and normalized to the Nominativ Singular or Nominativ Plural form of the respective item. Items are separated by a double \textit{daṇḍa}. The differences in punctuation, as well as simple emendations regarding punctuation  won't be documented in the apparatus criticus. 
\clearpage

\chapter{Critical Edition}
\clearpage
\begin{ekdosis}
  \begin{tlg}
%-----------------------------
%samyag ānandajananaḥ  sadguruḥ sobhidhīyate/   nimeṣārddhaṃ vā tatpādaṃ    yad vākyād avalokanāt// 3// \E[p.78]
%samyag ānandajananaḥ  sadguruḥ sobhidhīyate    nimiṣārddhaṃ vā tatpādaṃ    yad vākyād avalokanāt       \P
%samyag ānandajananaḥ/ sadguruḥ sobhidhīyate/   nimeṣārddhā  vā tatpāda     yad vākyād avalokanāt// 3//  \B
%samyag ānandajananaḥ  sadguruḥ sobhidhīyate//  nimeṣārddhā  vā tatpāda     yad vākyād avalokanāt// 3// \L %%%%0039.jpg
%samyag ānandajananaṃ  sadguruḥ sobhidhīyate/   nimeṣārddhaṃ ca    pādaṃ vā yad vākyād avalokanāt/ \N1
%samyag ānandajananaṃ  sadguruḥ sobhidhīyate/   nimiṣārddhaṃ ca    pādaṃ vā yad vākyād avalokanāt  \N2
%samyag ānandajananaṃ  sadguruḥ sobhidhīyate/   nimeṣārddhaṃ ca    pādaṃ vā yad vākyād avalokanāt/ \D
%samyag ānandajananaḥ  sadguruḥ  sobhidhīyate// nimeṣārddhaṃ vā tatpādaṃ    yad vākyād avalokanāt// \U2
%\om \U1
%-----------------------------
%He, who is the progenitor of absolute bliss, is known to be the true teacher. Because of a glance for just half a wink on the [teacher's] feet [or] by [just] talking about [it], ...  
%-----------------------------
\tl{
  \note[type=source, labelb=408, lem={samyag}]{SSP 5.64cd-5.65ab: samyag ānandajanakaḥ sadguruḥ so 'bhidhīyate | nimiṣārdhārdhapātād vā yad vā pādāvalokanāt |}
  \note[type=testium, labelb=409, lem={samyag}]{Ysv\textsuperscript{PT}: nimeṣārddhena tasyaiva ājñāpālanato bhavet | mahānandaśataprāptis tasmai śrīgurave namaḥ |}
samyag-ānanda\app{\lem[wit={B,E,L,P,U2},alt={°jananaḥ}]{jananaḥ}
  \rdg[wit={D,N1,N2}]{jananaṃ}} sadguruḥ sobhidhīyate/}\\
\tl{
  \app{\lem[wit={ceteri}]{nimeṣārddhaṃ}
    \rdg[wit={P,N2}]{nimiṣārddhaṃ}
    \rdg[wit={B,L}]{nimeṣārddhā}}
  \app{\lem[wit={ceteri}]{vā}
    \rdg[wit={D,N1,N2}]{ca}}
  \app{\lem[wit={E,P,U2}]{tatpādaṃ}
    \rdg[wit={B,L}]{tatpāda}
    \rdg[wit={D,N1,N2}]{pādaṃ vā}}
yad-vākyād-avalokanāt\dd{}4\hskip-2pt\dd{}}
\end{tlg}
\end{ekdosis}
%-----------------------------
%svātmā  sthiratvam  āyāti   tasmai śrīgurave  namaḥ/    nānāviplava--viśrāntiḥ  kathanāt  kurute tataḥ// 4// \E
%svātmā/ sthiratvam  āyāti   tasmai śrīgurave  namaḥ     nānāvikalpaḥ viśrāṃtiḥ  kathanāt  kurute tu yaḥ      \P einziges daṇḍa im Text! 
%svātmā  sthiraṃtvam āyāti   tasmai śrīgurave  namaḥ/    nānāvikalpa--viśrāṃti   kathanāt/ kurute tu yaḥ/   \B
%svātmā  sthiratvam  āyāti   tasmai śrīgurubho namaḥ//   nānāvikalpa--viśrāṃti   kathanāt  kurute tu yaḥ//   \L
%svātmā  sthiratvam  āyāti   tasmai śrīgurave  namaḥ/    nānāvikalpa--viśrāṃtiṃ  kathanāt  kurute tu saḥ/   \N1
%svātmā  sthiratvam  āyāti   tasmai śrīgurave  namaḥ/    nānāvikalpa--viśrāṃti   kathanāt  kurute tu saḥ//5// \N2
%svātmā  sthiratvam  āyāti   tasmai śrīgurave  namaḥ//   nānāvikalpaṃ viśrāṃtiṃ  kathanāt  kurute tu saḥ/      \D
%svātmā  sthiratvam  āyāti   tasmai śrīguru namo namaḥ// nānāvikalpaviśrāntiṃ    kathanāt  kurute tu yaḥ// \U2
%\om \U1
%-----------------------------
%... the own self goes into stability, homage to that teacher, who brings all doubts to stop because of [his] advice.   
%-----------------------------
\begin{ekdosis}
\begin{tlg}
\tl{
  \note[type=source, labelb=410, lem={svātmā}]{SSP 5.65cd-5.66ab: svātmānaṃ sthiram ādhatte tasmai śrīgurave namaḥ | nānāvikalpaviśrāntiṃ kathayā kurute tu yaḥ |}
 \note[type=testium, labelb=411, lem={nānāvikalpa°}]{Ysv\textsuperscript{PT}: nānāvikalpavibhrāntināśañca kurute tu yaḥ | sadguruḥ sa tu vijñeyo na tu vairaprakalpakaḥ |}
svātmā sthiratvam-āyāti tasmai
\app{\lem[wit={ceteri}]{śrīgurave}
  \rdg[wit={L}]{śrīgurubho}
  \rdg[wit={U2}]{śrīguru namo}} namaḥ/}\\
\tl{
\app{\lem[wit={N1,U2}]{nānāvikalpaviśrāntiṃ}
  \rdg[wit={D}]{nānāvikalpaṃ viśrāṃtiṃ}
  \rdg[wit={E}]{nānāviplavaviśrāntiḥ}
  \rdg[wit={P}]{nānāvikalpaḥ viśrāṃtiḥ}
  \rdg[wit={B,L}]{nānāvikalpaviśrāṃti}
  \rdg[wit={N2}]{nānāvikalpaviśrāṃti}}
\app{\lem[wit={ceteri},alt={kathanāt}]{kathanā\skp{t-ku}}
  \rdg[wit={B}]{kathanāt |}
}\skm{t-ku}rute
\app{\lem[wit={B,L,P,U2}]{tu yaḥ}
  \rdg[wit={E}]{tataḥ}
  \rdg[wit={D,N1,N2}]{tu saḥ}}/}\\
%-----------------------------
%sadguruḥ sa tu vijñeyo na tu vai priyajalpakaḥ// 5// \E
%sadguruḥ sa tu vijñeyo na tu vipriyajalpakaḥ      5 \P
%sadguruḥ sa tu vijño   nnu   viprāyajalākaḥ// 5//        \B
%sadguruḥ sa tu vijño   nnu   viprāyajalākaḥ// 5//        \L
%sadguruḥ sa tu vijñeyo na tu vipriyajalpakaḥ//       \N1
%sadguruḥ sa tu vijñeyo na tu vipriyajalpakaḥ//6//    \N2
%sadguruḥ sa tu vijñeyo na tu vipriyajalpakaḥ/       \D
%sadguruḥ sa tu vijñeyo na tu vipriyajalpakaḥ// \U2
%\om \U1
%-----------------------------
%He is known to be true teacher, not an unpleasant disputant.  
%-----------------------------
\tl{
  \note[type=source, labelb=410, lem={sadguruḥ}]{SSP 5.66cd: sadguruḥ sa tu vijñeyo na tu mithyāviḍambakaḥ||5.67||}
  sadguruḥ sa tu
  \app{\lem[wit={ceteri}]{vijñeyo}
    \rdg[wit={B,L}]{vijño}}
  \app{\lem[wit={ceteri}]{na tu}
    \rdg[wit={B,L}]{nnu}}
\app{\lem[wit={ceteri}]{vipriyajalpakaḥ}
  \rdg[wit={B,L}]{viprāyajalākaḥ}
  \rdg[wit={E}]{vai priyajalpakaḥ}}\dd{}5\hskip-2pt\dd{}}
\end{tlg}
\end{ekdosis}
\bigskip
\begin{ekdosis}
  \begin{prose}
%-----------------------------
%ata eva paramapadasya prāpty arthaṃ sadguruḥ sevyaḥ sarvadā           yaḥ puruṣaḥ satyavādī bhavati/ \E
%ata eva paramapadasya prāpty arthaṃ sadguruḥ sevyaḥ sarvadā           yaḥ puruṣaḥ satyavādī bhavati  \P %%%7677.jpg
%ata eva paramapada----prāpty arthaṃ sadguruḥ sevya--sarvadā/          yaḥ puruṣaḥ satyavādī bhavati/ \B
%ata eva paramapada----prāpty arthaṃ sadguruḥ sevya--sarvadā//         yaḥ puruṣaḥ satyavādī bhavatī// \L
%ata eva paramapada----prāpty arthaṃ sadguruḥ        sarvadā vaṃdyaḥ?/ yaḥ puruṣaḥ satyavādī bhavati/ \N1
%ata eva paramapada----prāpty arthaṃ sadguruḥ        sarvadā vaṃdyaḥ/  yaḥ puruṣaḥ satyavādī bhavati/ \N2 S. 11 verso Zeile 6. 
%ata eva paramapada----prāpty arthaṃ sadguruḥ        sarvadā vaṃdyaḥ/  yaḥ puruṣaḥ satyavādī bhavati \D
%ata eva paramapada----prāpty arthaṃ sadguruḥ sevyaḥ sarvadā           yaḥ puruṣaḥ satyavādī bhavati// \U2
%\om \U1
%-----------------------------
%Hence the true teacher is always is to be praised to attain the highest place. That person is a speaker of truth. 
%-----------------------------
\note[type=source, labelb=411, lem={ata eva}]{SSP 5.67: ata eva paramapadaprāpty arthaṃ sa sadguruḥ sadā vandanīyaḥ |}
\note[type=testium, labelb=412, lem={ata eva}]{Ysv\textsuperscript{PT}: ata eva maheśāni sadguruḥ śiva āditaḥ | satyavādī ca sacchīlo gurubhakto dṛḍhavrataḥ |}
ata eva
\app{\lem[wit={ceteri},alt={paramapadaprāpty}]{paramapadaprāpt\skp{y-a}}
  \rdg[wit={E,P}]{paramapadasya prāpty}
}arthaṃ sadguruḥ
\app{\lem[wit={D,N1,N2}]{sarvadā vandyaḥ}
  \rdg[wit={E,P,U2}]{sevyaḥ sarvadā}
  \rdg[wit={B,L}]{sevyasarvadā}}/
yaḥ puruṣaḥ satyavādī bhavati/ 
%-----------------------------
%niraṃtaraṃ gurusevā tatparo bhavati/ \E [p.79]
%niraṃtara--gurusevā rato bhavati   \P
%niraṃtaraṃ gurusevā taro bhavati/   \B
%niraṃtaraṃ gurusevā rato bhavati...   \L
%niraṃtaraṃ gurusevā rato bhavati   \N1
%niraṃtaraṃ gurusevā rato bhavati   \N2
%niraṃtaraṃ gurusevā rato bhava   \D
%niraṃtaraṃ gusevā   rato bhavati// \U2
%\om \U1
%-----------------------------
%Uninterrupted devotion for frequenting the teacher arises. 
%-----------------------------
\app{\lem[wit={ceteri}]{nirantaraṃ}
  \rdg[wit={P}]{niraṃtara°}}
\app{\lem[wit={ceteri}]{gurusevā}
  \rdg[wit={U2}]{gusevā}}
\app{\lem[wit={ceteri}]{rato}
  \rdg[wit={B}]{taro}
  \rdg[wit={E}]{tatparo}}
\app{\lem[wit={ceteri}]{bhavati}
  \rdg[wit={D}]{bhava}}/
%-----------------------------
%yasya manasi pāpaṃ na bhavati/ \E
%yasya manasi pāpaṃ na bhavati  \P
%yasya manasi pāpa nna bhavati/ \B
%yasya manasi pāpaṃ nna bhavati/ \L
%yasya manasi pāpaṃ nna bhavati/ \N1
%yasya manasi pāpaṃ na bhavati/ \N2
%yasya manasi pāpaṃ nna bhavati/ \D
%\om \U1
%yasya manasi pāpa na bhavati// \U2
%-----------------------------
%He becomes one in whose mind evil does not arise. 
%-----------------------------
yasya manasi
\app{\lem[wit={ceteri}]{pāpaṃ}
  \rdg[wit={B}]{pāpa}}
na bhavati/ 
%-----------------------------
%svācārarataḥ   snānādiśīlo bhavati/ \E
%svācārarataḥ   snānodiśīlo bhavati \P
%svācāraratāḥ   snānādiśīlo bhavatī/ \B
%svācāraratāḥ   snānādiśīlo bhavatī/ \L
%svasyācārarato snānādiśīlo bhavati/ \N1
%svasyācārarato snānādiśīlo bhavati/ \N2
%svasyācārarato snānādiśīlo bhavati/ \D
%\om \U1
%svācārataḥ// snānādiśīlo bhavati// \U2
%-----------------------------
%Being someone who is devoted to good habits, habits such as ceremonial bathing etc. arise. 
%-----------------------------
\note[type=source, labelb=413, lem={svācārarataḥ}]{Ysv\textsuperscript{PT}: svalpācāraratātmā yo dānādiśīlasaṃyutaḥ | kāpaṭyalobhavinyāsau mahāvaṃśasamudbhavaḥ |}
\app{\lem[wit={E,P}]{svācārarataḥ}
  \rdg[wit={B,L}]{svācāraratāḥ}
  \rdg[wit={U2}]{svācārataḥ ||}
  \rdg[wit={D,N1,N2}]{svasyācārarato}}
snānādiśīlo bhavati/ 
%-----------------------------
%kāpaṭyaṃ na bhavati   yasya vaṃśaparaṃparā jñāyate/ \E
%kāpaṭyaṃ na bhavati   yasya vaṃśaparaṃparā jñāyate  \P
%kāpaṭyaṃ bhavati/     yasya vaṃśaparaṃparā jñāyate/ \B
%kāpaṭyaṃ na bhavati/  yasya vaṃśaparaṃparā jñāyate... \L
%kāpaṭyaṃ nāsti/       yasya vaṃśaparaṃparā jñāyate/ \N1
%kāpaṭyaṃ nāsti/       yasya vaṃśaparaṃparā jñāyate/ \N2
%kāpaṭyaṃ nāsti/       yasya parāparaṃparā  jñāyate/ \D
%\om \U1
%kāpaṭyaṃ na bhavati// yasya vaṃśaparaṃ parā jñāyate// \U2
%-----------------------------
%Deceiving does not arise. His noble race is recognized by him.
%-----------------------------
kāpaṭyaṃ \app{\lem[wit={E,P,L,U2}]{na bhavati}
  \rdg[wit={B}]{bhavati}
  \rdg[wit={D,N1,N2}]{nāsti}}/
yasya \app{\lem[wit={ceteri}]{vaṃśaparaṃparā}
  \rdg[wit={D}]{parāparaṃparā}}
jñāyate/ 
%-----------------------------
%etādṛśasya sadguroḥ saṃgatiḥ karttavyā  tena puruṣasya manaḥ śāṃtiṃ prāpnoti/ \E
%etādṛśasya sadguroḥ saṃgati  karttavyā       puruṣasya manaḥ śāṃtiṃ prāpnoti  \P
%etādṛśasya sadguroḥ saṃgatī  karttavyā/      puruṣasya manaḥ śāṃti  prāpnoti/ \B
%etādṛśasya sadguroḥ saṃgatī  karttavyā       puruṣasya manaḥ śāṃti  prāpnoti... \L
%etādṛśasya sadguroḥ saṃgatiḥ  kattavyāḥ      puruṣasya manaḥ śāṃtiṃ  prāpnoti/ \N1
%etādṛśasya sadguroḥ saṃgati   karttavyāḥ     puruṣasya manaḥ śāṃtiṃ  prāpnoti// \N2
%etādṛśasya sadguroḥ saṃgatiḥ  kattavyāḥ/     puruṣasya manaḥ śāṃtiṃ  prāpnoti/ \D
%etādṛśasya guroḥ    saṃgatiḥ  karttavyā//    puruṣasya mano śāṃtiṃ  prāpnoti// \U2
%\om \U1
%-----------------------------
%One shall associate with such a true teacher. The mind of such a person attains peace.
%-----------------------------
\note[type=source, labelb=414, lem={etādṛśasya}]{Ysv\textsuperscript{PT}: īdṛśaḥ sadgurustasya saṅgatau yatnavān bhavet | tad eva manasaḥ śāntiṃ prāpnoti paramaṃ padam |}
etādṛśasya
\app{\lem[wit={ceteri}]{sadguroḥ}
  \rdg[wit={U2}]{guroḥ}}
  \app{\lem[wit={E,D,N1,U2}]{saṃgatiḥ}
    \rdg[wit={P,N2}]{saṃgati}
    \rdg[wit={B,L}]{saṃgatī}}
  \app{\lem[wit={B,E,L,P,U2}]{karttavyā}
    \rdg[wit={D,N1}]{kattavyāḥ}
    \rdg[wit={N2}]{karttavyāḥ}}
  \app{\lem[wit={E}]{tena}
    \rdg[wit={ceteri}]{\om}}
  puruṣasya
  \app{\lem[wit={ceteri}]{manaḥ}
    \rdg[wit={U2}]{mano}}
  \app{\lem[wit={ceteri}]{śāntiṃ}
    \rdg[wit={B,L}]{śāṃti}}
  prāpnoti/
%-----------------------------
%atha ca yasya manomadhye sthira ānanda utpadyate  so pi sadguruḥ kathyate/ \E
%atha ca yasya manomadhye sira   ānanda utpadyate  so pi sadguruḥ kathyate \P
%atha ca yasya manomadhye sīraḥ  ānaṃda utpadyate  so pi sadguruḥ kathyate... \B
%atha ca yasya manomadhye sīraḥ  ānaṃda utpadyate  so pi sadguruḥ kathyate \L
%atha ca yasya manomadhye sthira ānanda utpadyate  so pi sadguruḥ kathyate \N1
%atha ca yasya manomadhye sthīrānaṃda   utpadyate/ so pi sadguruḥ kathyate// \N2
%atha ca yasya manomadhye sthira ānaṃda utpadyate/ so pi sadguruḥ kathyate/ \D
%atha ca       manomadhye sthira ānanda utpadyate//so pi sadguruḥ kathyate// \U2
%\om \U1
%-----------------------------
%And he in whose mind arises steady bliss is also called a true teacher. 
%-----------------------------
  atha ca \app{\lem[wit={ceteri}]{yasya}
    \rdg[wit={U2}]{\om}}
  manomadhye
  \app{\lem[wit={E,D,N1,N2}]{sthira}
    \rdg[wit={B,L}]{sīraḥ}
    \rdg[wit={P}]{sira}
    \rdg[wit={N2}]{sthīrā°}}ānanda utpadyate/ so 'pi sadguruḥ kathyate/ 
%-----------------------------
% \om                                                                                                           \E
%atha ca                ghaṭikārdhaṃ   ghaṃṭikā   caturthāṃśo       vā  yasya pārśvam upaviṣṭe satyatādṛṣo bhāvo manomadhya utpadyate    \P
%atha ca                ghaṭikārdhaṃ   ghaṃṭikā   caturthāṃśo       vā  yasya pārśvam upaviṣṭe satyatādṛṣo bhāvo manomadhya uppapadyate/ \B
%atha ca                ghaṭikārdhaṃ   ghaṭikā    caturthaṃśo       vā  yasya pārśvam upaviṣṭe satyetādṛṣo bhāvo manomadhya upapadyate/  \L
%atha ca ghaṭī .. .. mo ghaṭikārddhaṃ  ghaṭikāyāḥ caturtho ḍaṃśo    vā  yasya pārśvam upaviṣṭe satyetādṛśo bhavo manomadhye utpadyate/  \N1 
%atha ca ghaṭimātra-----ghaṃṭikārddhaṃ ghaṭikā----caturtho daṃśo    vā  yasya pārśvem upaviṣṭe satyatādṛṣo .. .. manomadhye utpadyate/ \N2
%atha ca ghaṭīṃ mātraṃ  ghaṭikārddhaṃ  ghaṭikāyāḥ caturtho  aṃśo    vā  yasya pārśvam upaviṣṭe satyetādṛṣo bhāvo manomadhye utpadyate/ \D %%%%p.18 verso
%atha                   ghaṭikā        ghaṭikā    caturthāṃśo       vā  yasya pārśvam upaviṣṭe satyatādṛṣo bhāvo manomadhye utpadyate//    \U2
%\om \U1
%-----------------------------
%And then such a state of reality is generated of one who is seated at the side [of the teacher] for a \textit{ghaṭikā}\footnote{[1/60th part of a day (24 minutes). siehe Wörterbuch], half a \textit{ghaṭikā} or a quarter part of a \textit{ghaṭikā}. 
%-----------------------------
\note[type=philcomm, labelb=415, lem={atha ca ghaṭīmātraṃ \ldots utpadyate}]{E omits this sentence.}
atha
\app{\lem[wit={ceteri}]{ca}
  \rdg[wit={U2}]{\om}} 
\app{\lem[type=emendation, resp=egoscr]{ghaṭimātraṃ}
  \rdg[wit={N2}]{\korr ghaṭimātra°}
  \rdg[wit={D}]{ghaṭīṃ mātraṃ}
  \rdg[wit={N1}]{ghaṭī....mo}
  \rdg[wit={ceteri}]{\om}}
\app{\lem[wit={B,L,P,N1,D}]{ghaṭikārdhaṃ}
  \rdg[wit={N2}]{°ghaṃṭikārddhaṃ}
  \rdg[wit={U2}]{ghaṭikā}}
\app{\lem[wit={L,N2,U2},alt={ghaṭikā°}]{ghaṭikā}
  \rdg[wit={N1,D}]{ghaṭikāyāḥ}
  \rdg[wit={B,P}]{ghaṃṭikā°}}\app{\lem[wit={B,P,L,U2}]{caturthāṃśo}
  \rdg[wit={N1}]{caturtho ḍaṃśo}
  \rdg[wit={N2}]{caturtho daṃśo}
  \rdg[wit={D}]{caturtho aṃśo}}
vā yasya pārśvaṃ upaviṣṭe satyatādṛṣo bhāvo manomadhye \app{\lem[wit={ceteri}]{utpadyate}
  \rdg[wit={B,L}]{uppapadyate}}/
\end{prose}
\end{ekdosis}
\end{document}
%-----------------------------
%\om                           \E
%gatvā vanamadhye  sthiyate  gṛhaṃ tyajyate so pi sadguruḥ kathyate    \P
%gatvā vanamadhye/ sthiyate/ gṛhaṃ tyajyate so pi sadguruḥ kathyate/\B
%gatvā vanamadhye sthīyate... gṛhaṃ tyajyate so pi sadguruḥ kathyate...\L
%gatvā vanamadhye sthīyate/ gṛhaṃ tyajyate \N1
%gatvā vanamadhye sthīyate/ gṛhaṃ tyajyate \N2
%gatvā vanamadhye sthīyate/ gṛhaṃ tyajyate \D
%\om \U1
%gatvā vanamadhye  sthīyate// gṛhaṃ tyajyate// so pi sadguruḥ kathyate//  \U2
%-----------------------------
%A true teacher is taught to be one that has left the house and went into the forest...
%-----------------------------
%-----------------------------
%kasyāpi duḥkhaṃ na dīyate/ \E
%kasyāpi duḥkhaṃ na dīyate  \P
%kasyāpi duḥkhaṃ na dīyate/ \B
%kasyāpi duḥkhaṃ na dīyate... \L
%kasyāpi duḥkhaṃ na dīyate  \N1
%kasyāpi duḥkhaṃ na dīyate//  \N2
%kasyāpi duḥkhaṃ na dīyate//  \D
%\om \U1
%kasyāpi duḥkhaṃ na dīyate/ \U2
%-----------------------------
%... who does not capitulate to suffering.  
%-----------------------------
%prāṇimātreṇa saha maitrī kriyate  kasyāpi doṣaṃ na kathayati         so pi sadguruḥ kathyate// \E
%prāṇimātreṇa saha maitrī kriyate  kasyāpi doṣo  na prakāśyate        so pi sadguruḥ kathyate  \P %%%7678.jpg
%prāṇimātreṇa saha maitrī krīyate/ kasyāpi doṣau na prākāśate/        so pi sadguruḥ kathyate/  \B
%prāṇimātreṇa saha maitrī krīyate  kasyāpi doṣo  na prākāśate         so pi sadguruḥ kathyate \L
%prāṇimātreṇa saha maitrī krīyate/ kasyāpi doṣo  na prākāśyate/  yena so pi sadguruḥ kathyate// \N1
%prāṇimātreṇa saha maitrī yate/    kasyāpi doṣaṃ na prakāśyate// yena so pi sadguruḥ kathyate// \N2
%prāṇimātreṇa saha maitrī krīyate/ kasyāpi doṣo  na prākāśyate/  yena so pi sadguruḥ kathyate/ \D
%\om \U1
%prāṇimātre   saha maitrī kriyate// kasyāpi doṣo na prakāśyate//  so pi sadguruḥ kathyate// \U2 %%%430.jpg
%-----------------------------
%... by whom loving kindness is practiced towards living beings. One for whom error does not appear. He is said to be a true teacher.
%-----------------------------
%ajñātakulaśīlānāṃ yatīnāṃ brahmacāriṇām/   upadeśaṃ na  gṛhṇīyād anyathā narakaṃ dhruvam// \E %[p.80]
%ajñātakulaśīlānāṃ yatīnāṃ brahmacāriṇām    upadeśo  na  gṛhṇīyād anyathā narakaṃ dhruvam 1 \P
%ajñānakulaśilānāṃ yatīnāṃ brahmacāriṇām    upadeśa  na  gṛhītyāsthā/ yadānyathā na narakaṃ dhruvaṃ// 1// \B
%ajñānakulaśilānāṃ yatīnāṃ brahmacāriṇām//  upadeśaṃ na  gṛṇhīyād anyathā narakaṃ dhruvaṃ// 1// \L  0040.jpg
%ajñānakulaśīlānāṃ, yatīnāṃ brahmacāriṇām/  upadeśaṃ na  gṛhnīyāt anyathā narakaṃ dhruvaṃ// \N1 %%%S.14 
%ajñātakulaśīlānāṃ yatīnāṃ brahmacāriṇām//  upadeśaṃ na  gṛhnīyāt anyathā narakaṃ dhruvaṃ//1// \N2 
%ajñānakulaśīlānāṃ yatīnāṃ brahmacāriṇām/   upadeśaṃ na  gṛhnīyāt anyathā narakaṃ dhruvaṃ/ \D
%\om \U1
%ajñātakulaśīlānāṃ yatīnāṃ brahmacāriṇām    upadeśo na gṛhṇīyāt// anyathā narakaṃ dhruvaṃ// \U2
%-----------------------------
%Man soll nicht die Lehre der Zölibatäre, der Asketen und von denen deren Abstammung und Charakter unbekannt sind akzeptieren, andernfalls [käme man] in Folge dessen für immer in die Hölle.
%----------------------
%yasya vacasi manasi dhṛte sati  svātmanaḥ parameśvarasyaikyaṃ bhavati/ \E
%yasya vacasi manasi dhṛte sati  svātmanaḥ parameśvarasyaikyaṃ bhavati  \P
%yasya vacasi manasi dhṛte sati  svātmanaḥ parameśvarasakyaṃ bhavati/ \B
%yasya vacasi manasi dhṛte sati  svātmanaḥ parameśvarasakyaṃ bhavati// \L
%yasya vacasi manasi dhṛte sati/ svātmanaḥ parameśvarasyaikyaṃ bhavati/ \N1
%yasya vacasi manasi dhṛte sati/ svātmanaḥ parameśvarasyaikaṃ bhavati/ \N2
%yasya vacasi manasi dhṛte sati/ svātmanaḥ parameśvarasyaikyaṃ bhavati/ \D
%\om \U1
%yasya cavi          dhṛte sati  svātmanaḥ parameśvarasyaikyaṃ bhavati// \U2
%-----------------------------
%Unitiy of the supreme deity and the own self arises, for one who abides calm in mind and speech.  
%-----------------------------
%etādṛśo manomadhye niścayo bhavati/ \E
%etādṛśo manomadhye niścayo bhavati \P
%etādṛśo manomadhye niścayo bhavati/ \B
%etādṛśo manomadhye niścayo bhavati// \L
%etādṛśo manomadhye niścayo bhavati \N1
%etādṛśo manomadhye niścayo bhavati \N2
%etādṛśo manomadhye niścayo bhavati/ \D
%etādṛśo manomadhye niścayo bhavati// \U2
%\om \U1
%-----------------------------
%Certainty of this kind arises in the mind.
%Such is the certainty that arises in the mind. 
%-----------------------------
%taṃ sadguruṃ vijānīyāt vikalpa etādaśo  yathā samudramadhye mahattaraṃ kallolāḍambaram/ prapaṃce vāsanā   tādṛśī yathodakamadhye mahattaraṃgāḥ/    \E
%taṃ sadguruṃ jānīyāt   vikalpa etādṛśo  yathā samudramadhye mahattara--kallolāḍambaraṃ  prapaṃca-vāsanā  etādṛśī yathodakamadhye mahattarati        \P
%taṃ sadguruṃ jānīyāt   vikalpa etādṛśo  yathā samudramadhye mahattara--kallolāḍambara   prapaṃca-vāsanā  etādṛśī yathodakamadhye mahattarati ....   \B
%taṃ sadguruṃ jānīyāt   vikalpa etādṛśo  yathā samudramadhye mahattara--kallolāḍambara   prapaṃca-vāsanā  etādṛśī yathodakamadhye mahattarati ....   \L
%taṃ sadguruṃ jānīyāt// vikalpa etādṛśo  yathā samudramadhye mahattara--kallolāḍambaraḥ/ prapaṃca-vāsanā  etādṛśī yathodakamadhye mahattarati        \N1
%taṃ sadguruṃ jānīyāt// vikalpa etādṛśaṃ yathā samudramadhye mahattara--kallolāḍambaraḥ  prapaṃca-vāsanā  etādṛśī yathodakamadhye mahattarati        \N2 %%S.12
%taṃ sadguruṃ jānīyāt/  vikalpa etādṛśo  yathā samudramadhye mihattara--kallolāḍambaraḥ/ prapaṃca-vāsanā  etādṛśī yathodakamadhye mahattarati        \D
%taṃ sadguruṃ jānīyāt// vikalpa etādṛśo  yathā samudramadhye mahattara--kallolāḍambaraṃ  prapaca vāsanā// etādṛśī yathodakamadhye mahattarī          \U2
%\om \U1
%-----------------------------
%E: One should know this true teacher. Uncertainty is like the roar of waves within the ocean. The manifold mental imprints are like the ripples in the water...
%-----------------------------
%tādṛśasya saṃsārasāgarasya yaḥ svavākyanāvā paraṃ pāraṃ  prāpayati/ sa sadguruḥ kathyate// \E
%tādṛśya   saṃsārārgāvād    yo              nāvā paraṃ       prāpayati  sa sadguruḥ kathyate  \P
%tādṛśāt   saṃsārārṇavavād  yau           nāvā paraṃ       prāpayati/ sa sadguruḥ kathyate// \B
%tādṛśāt   saṃsārārṇavād    yau             nāvā paraṃ       prāpayati  sa sadguruḥ kathyate// \L
%tādṛśāt   saṃsārāt araṇavād yo           nāvaraṃ          prāpayati/ sa sadguruḥ kathyate// \N1 %%%S.14 Z. 3
%tādṛśāt   saṃsārāt araṇavād yo           nāvaraṃ          prāpayati/ sa sadguruḥ kathyate// \N2
%tādṛśāt   saṃsārāt aṛṇavād  yo            nāvā paraṃ       prāpayati/    sadguruḥ kathyate/ \D
%tādṛśāt   saṃsārārṇavād   yo              nāvā pāraṃ pāraṃ prāpayati// sa sadguruḥ kathyate// \U2
%\om \U1
%-----------------------------
%Weil der Ozean des Weltengangs etc. solcher Art ist, verursacht [der Lehrer] durch das Boot zur Traditionsfolge zu bringen?!.
%E:He who with the boat of his own speech causes to convey one to the other shore of such an ocean of Saṃsāra is said to be a true teacher.
%U2: He who  The boat which is the speech for the flood of Samsāra which causes to bring one from shore to shore is called a true teacher.
%He who causes to convey the boat from such the ocean of Saṃsāra to the other shore is called a true teacher. 
%-----------------------------
%yasya puruṣasya mano  'khaṇḍe paramamade līnaṃ bhavati/ [p.81]     \E
%yasya puruṣasya mano  'khaṃḍe       pade līnaṃ bhavati/                 \P
%yasya puruṣasya manaḥ akhaṃḍe       pade līnaṃ bhavatī/ DSCN7180.JPG   \B
%yasya puruṣasya manaḥ akhaṃḍe       pade līnaṃ bhavati/                \L
%yasya puruṣasya mano   khaṃḍe   parapadalīna bhavati/                \N1
%yasya puruṣasya mano   khaṃḍe   paramadalita bhavati/                \N2
%yasya puruṣasya mano   khaṃḍe   parapadalīnaṃ bhavati/                \D
%yasya puruṣasya mano  'khaṇḍe   parapade līnaṃ bhavati     \U2
%\om \U1
%----------------------------
%The mind of the person becomes absorbed into the indivisible supreme place.  
%-----------------------------
%yaḥ puruṣaḥ svakulaṃ     trividhāt   tāpānnivartya  parame  muktipade  rakṣati/ \E
%yaḥ puruṣaḥ svīyaṃ kulaṃ trividhātra pānnivartta----parama--muktipade  rakṣati \P %%%7579.jpg
%yaḥ puruṣaḥ svikulaṃ     trividhaṃ/  tāpānnivartya  para----muktipade  rakṣati/ \B
%yaḥ puruṣaḥ sviyaṃ kulaṃ trividhat   āpān nivartya   para----muktipade  rakṣati... \L
%yaḥ puruṣaḥ svīyaṃ kūlaṃ trividhāt   tāpān nivarttya parama--muktipade  rakṣati/ \N1
%yaḥ puruṣa  svīyaṃ kūlaṃ trividhāt   tāyānivartya    paramamamuktipade  rakṣati/ \N2
%yaḥ puruṣaḥ svīyaṃ kūlaṃ trividhāt---tāpān mivarttya parama--muktipade  rakṣati/ \D
%yaḥ puruṣa  svīyaṃ kulaṃ trividhat---āpān nivartya   parama--muktipakṣe rakṣati// \U2
%\om \U1
%-----------------------------
%The person situated in the place of supreme liberation who turned away from the threefold misery (adhyātmika, adhibhautika, adhidaivika) protects the own kula (lineage? noble family? tribe?). 
%-----------------------------
%etādṛśasya puruṣasya śravaṇād   darśanāt  samagravighnā naśyanti/    \E
%etādṛśā    puruṣasya śravaṇā    darśanāt  samagravighnā naśyaṃti     \P
%etādṛśā    puruṣasya śravaṇāt   darśanāt/ samagravighnā na naśyaṃtī/ \B
%etādṛśā    puruṣasya śravaṇāt   darśanāt  samagravighnā na naśyaṃti... \L
%etādṛśa/   puruṣaṃ   śravaṇād   darśanāt  samagravighnā naśyaṃti/ \N1
%etādṛśaṃ   puruṣaṃ   śravaṇād   darśanāt  samagravighnā naśyaṃti/ \N2
%etādṛśa----puruṣaṃ   śravaṇād   darśanāt  samagravighnā naśyaṃti/ \D
%etādṛśaṃ   puruṣaṃ   śravaṇād   darśanāt  samagraviśvaś ca vaśāṃ? bhavati \U1
%etādṛśa    puruṣasya śravaṇāt// darśanāt  samagravighna naśyaṃti//    \U2
%-----------------------------
%From hearing [or] from seeing about such a person all obstacles are destroyed. 
%-----------------------------
%dine dine kalyāṇaṃ  bhavati/ niṣkalaṃkā buddhir utpadyate/    \E
%dine dine kalyāṇaṃ  bhavati  niṣkalaṃkā buddhir utpadyate     \P
%dine dine kalyāṇaṃ  bhavati  niṣkalaṃkā buddhir utpadyate/    \B
%dine dine kalyāṇaṃ  bhavati  niṣkalaṃkā buddhir utpadyate...  \L
%dine dine kalyāṇaṃ  bhavati/ niṣkalaṃ   buddhir utpadyate//     \N1
%dine dine kalyāṇaṃ  bhavati/ niṣkalaṃ   buddhir utpadyate//     \N2
%dine dine kalyāṇaṃ  bhavati/ niṣkalaṃkā buddhir utpadyate/    \D
%dine      kalyāṇāṃ  bhavatīr niṣkalaṃkā buddhir utpadyate     \U1
%dine dine kalyāṇaṃ  bhavati/ niṣkalaṃko buddhir utpadyate// \U2
%-----------------------------
%Day by day prosperity arises. A flawless intellect arises. 
%-----------------------------
%idaṃ yogaśāstrasya rahasyaṃ samastaśāstraprameyasya manaḥ           yathāṃdhakārasya    madhye dīpatejaḥ praviśati/        \E
%idaṃ yogaśāstrasya rahasyaṃ samagraśāstramadhye           yasya manaḥ yathāṃdhakārasya    madhye dīpasya tejaḥ praviśati   \P
%idaṃ yogaśāstrarahasyaṃ/    samastaśāstramadhye     manaḥ yasya manayathāṃdhakārasya    madhye dīpasya tejaḥ praviśyati... \B
%idaṃ yogaśāstrarahasyaṃ//   samastaśāstramadhye     mano  yasya manayathāṃdhakārasya    madhye dīpasya tejaḥ praviśyati...  \L
%idaṃ yogaśāstrarahasyaṃ/    samagraśāstramadhye           yasya mana/ yathāṃdhakāras    madhye dīpasya tejaḥ praviśati/     \N1
%idaṃ yogaśāstrarahasya      samagraśāstramadhye                                                                              \N2
%idaṃ yogaśāstrarahasya/     samagraśāstramadhye           yasya mana/ yathāṃdhakāramadhye dīpasya tejaḥ praviśati/     \D
%idaṃ yogaśāstreṣu rahasyaṃ  samagraśāstramadhye           yasya manaḥ yathāṃdhakārasya    madhye dīpasya tejaḥ praviśyati...  \U1
%idaṃ yogaśāstrarahasyaṃ     samagraśāstramadhye yasya manaḥ//             yathāṃdhakārasya    madhye dīpasya tejaḥ vipraśati//        \U2
%-----------------------------
%This is the secret of the Yogaśāstra. The mind of whom understands? the entire teaching, just as the light of the fire enters the centre of darkness  
%-----------------------------
%tathā śāstramadhye mano praviśati/     \E
%tathā śāstramadhye manaḥ praviśati     \P
%tathā.... \om                          \B
%tathā.... \om                          \L
%tathā śāstramadhye tasya manaḥ praviśati/     \N1
%                   tasya manaḥ praviśati/     \N2
%tathā śāstramadhye tasya manaḥ praviśati/     \D
%tathā.... \om                          \U1
%yathā śāstramadhye mano praviśati//     \U2
%-----------------------------
%in the same way the mind enters into the teaching. 
%-----------------------------
%yasya rājño madhye kalaho nāsti/ \E
%yasya rājño manomadhye kapaṭaṃ nāsti \P
%yasya rājño madhye manasi kapaṭaṃ nāsti/ \B
%yasya rājño madhye manasi kapaṭaṃ nāsti... \L
%yasya rājño manomadhye kapaṭaṃ nāsti/ \N1
%yasya rājño manomadhye kapaṭaṃ nāsti  \N2
%yasya rājño manomadhye kapaṭaṃ nāsti/ \D
% rājño manomadhye    kapaṭaṃ nāsti \U1
%yasya rājño manomadhye kapaṭaṃ nāsti// \U2
%-----------------------------
%For such a king there is no deceit. 
%-----------------------------
%yasmin dṛṣṭe deśikatrāso na bhavati/ \E
%yasmin dṛṣṭe deśikasya trāso na bhavati  \P
%yasmiṃ dṛṣṭe deśikasya trāso na bhavati/ \B
%yasmiṃ dṛṣṭe deśikasya trāso na bhavati \L
%yasmiṃ dṛṣṭe deśakasya trāso na bhavati/  \N1
%yasmin dṛṣṭe deśakasya trāso na bhavati/  \N2
%yasmiṃ dṛṣṭe deśakasya trāso na bhavati/  \D
%yasmiṃ dṛṣṭe darśakasya trāso na bhavati \U1
%yasmin dṛṣṭe deśikasya trāso na bhavati//  \U2
%-----------------------------
%Fear does not aries while the gaze is fixed on the teacher.  
%-----------------------------
%tasya manaḥ śuddhaṃ bhavati/  \E
%tasya manaḥ śuddhaṃ bhavati   \P
%tasya manaḥ śuddhaṃ bhavati/  \B
%tasya manaḥ śuddhaṃ bhavati   \L
%tasya manaḥ śuddhaṃ bhavati/  \N1
%tasya manaḥ śuddhaṃ bhavati/  \N2
%tasya manaḥ śuddhaṃ bhavati/  \D
%yasya manaḥ śuddhaṃ bhavati   \U1
%tasya manaḥ śuddhaṃ bhavati//  \U2
%-----------------------------
%His mind becomes pure. 
%-----------------------------
%yasya pṛthvyāṃ vītir  bhavati/ \E
%yasya pṛthivyāṃ kīrttir bhavati \P
%yasya pṛthvyāṃ kīrtir bhavati/ \B
%yasya pṛthivyāṃ kīrtir bhavati \L
%yasya pṛthivīkīrttir bhavati/ \N1
%yasya pṛthivīkīrttir bhavati/ \N2
%yasya pṛthivīkīrttir bhavati/ \D
% pṛithīvī kīrti bhavati \U1
%yasya pṛthvyāṃ kītīr  bhavati// \U2
%-----------------------------
%Fame on earth arises for him. 
%-----------------------------
%yasya manomadhye taspuruṣasya vaco viśvāso bhavati/        yo rājā sadānaṃdarūpo bhavati// \E
%yasya manomadhye satpuruṣavacanaviśvāso bhavati            yo rājā sadānaṃdapūrṇo bhavati   \P
%yasya manomadhye satpuruṣavacanaviśvāso bhavati            yo rājā sadānaṃdapūrṇo bhavati/  \B
%yasya manomadhye satpuruṣavacanaviśvāso bhavati            yo rājā sānaṃdapūrṇo bhavati   \L
%yasya manomadhye satpuruṣavacanaviśvāso bhavati/           yo rājā sadānaṃdapūrṇo bhavati   \N1
%yasya manomadhye satpuruṣavacanaḥ viśvāso bhavati/         yo rājā sadānaṃdapūrṇo bhavati   \N2
%yasya manomadhye satpuruṣavacanaviśvāso bhavati/           yo rājā sadānaṃdapūrṇo bhavati/   \D
%yasya manomadhye satpuruṣasya vacanavihyābhyāso??? bhavati yo rājā sadānaṃdapūrṇo bhavati   \U1
%yasya manomadhye satpuruṣavacanaviśvāso bhavati//          yo rājā sadānaṃdapūrṇo bhavati// \U2
%-----------------------------
%In the middle of his mind the confidence of speech of a wise man arises. He who is prince is one who is full of permanent bliss. 
%-----------------------------
%yasya pārśve pratyakṣam anekamanohārivastūni tiṣṭhaṃti/ \E
%yasya pārśve pratyakṣam anekaṃ manohārivastu bhavati  \P
%yasya pārśve pratyakṣam anekamanohārivastu bhavati/ \B
%yasya pārśve pratyakṣam anekamanohārivastu bhavati \L
%yasya pārśve pratyakṣam anekaṃ manohārivastu bhavati/ \N1
%yasya pārśve pratyakṣam anekaṃ manohārivastu bhavati// \N2
%yasya pārśve pratyakṣam anekaṃ manohārivastu bhavati/ \D
%yasya pārśve pratyakṣam anekaṃ manohārivastu bhavati \U1
%yasya pārśve pratyakṣam anekaṃ manohārivastu bhavati// \U2 %%%431
%-----------------------------
%E: Next to him in front of his eyes manifold beautiful things arise. 
%-----------------------------
%etādṛśasya rājña idaṃ yogarahasyaṃ kathanīyam/ \E
%etādṛśasya rājño ye yogarahasyaṃ kathanīyaṃ \P
%etādṛśasya rājño yethogarahasyaṃ kathyaniyaṃ/ \B
%etādṛśasya rājño yadyogarahasyaṃ kathyanīyaṃ// \L
%etādṛśasya rājño 'gre yogarahasyaṃ karttavyaṃ// \N1
%etādṛśasya rājño 'gre yogarahasyaṃ karttavyaṃ// \N2
%etādṛśasya rājño gre yogarahasyaṃ karttavya/ \D
%etādṛśasya rājño gre yogarahasyaṃ karttavyaṃ \U1
%etādṛśasya rājño ye yogarahasyaṃ kathyate// \U2
%-----------------------------
%This secret of Yoga of such a prince is the foremost secret of yoga that has to be told. 
%-----------------------------
%na snehān na bhayān na lobhān na mohān na dhanādbalān na maitrībhāvān naudāryān na sauṃdaryān na sevanāt/ \E
%na snehānno bhayāllno? bhāvānno dānānnasau daryānna sevanāt  \P %%%7680.jpg
%ni śnehānnā bhayāllobhānna mohān na dhanābdalāt/ na  maitrībhāvānno dānāt na sauṃdaryānna sevanāt/ \B
%ni śnehānnā bhayāllobhānna mohānna nadhanābdalāta// na maitrībhāvānno dānāt na sauṃdayanni sevanāt// \L
%na śnehāna bhayālobhān na mohān na dhanād balāt/ na maitrībhāvān na dāsān na sauṃdaryān na sevanāt/ \N1
%na śnehāna bhayālobhān na mohān na dhanād balāt// na maitrībhāvanā dānān na saudaryān na sevanāt// 1 \N2
%na śnehāna bhayāl lobhān na mohān na dhanād balāt/ na maitrī  \D
%na śnehānna bhayān lobhān na mohān na dhanāt balāt na maitrībhāvān na dāsān na sauṃdaryān na sevatā \U1
%na snehān na bhayā llonna mohānna dhanād balāt//na maitrī bhāvān nodānān na sauṃdaryān na sevanāt// \U2
%-----------------------------
%
%-----------------------------
%sāmānyāgre yogo na kathanīyaḥ/ \E
%sāmānyād agre yogo na kathanīyaḥ \P
%sāmānyāgre yogo na kathaniyaṃ/ \B
%sāmānyāgre yogo na kathanīyaṃ// \L
%sāmānyād gre yogo na kathanīyaḥ/ \N1
%sāmānyād gre yogo na kanīyaḥ/ \N2
%\om \D
%sāmānyāgre yogo na kathanīyaḥ \U1
%sāmānyād gre yogo na kathanīyaḥ \U2
%-----------------------------
%
%-----------------------------
%yaḥ paraniṃdārato bhavati/ \E
%yaḥ paraniṃdārato bhavati \P
%yaḥ paraniṃdāṃ karoti/ \B
%yaḥ paraniṃdāṃ karoti \L
%yaḥ paraniṃdārato bhavati \N1
%yaḥ paraniṃdārato bhavati \N2
%\om \D
% paraniṃdāṃ  rato bhavati \U1
%yaḥ paraniṃdārato bhavati// \U2
%-----------------------------
%Who has the pleasure of insulting others, 
%-----------------------------
%durācāro bhavati/ \E
%durācāro bhavati  \P
% \om              \B
% \om              \L
% dūrācāro bhavati/ \N1
% dūrācāro bhavati/ \N2
%\om \D
%dūrācāro bhavati  \U1
%dūrācāro bhavati//  \U2
%-----------------------------
%who is behaving badly, 
%-----------------------------
%durmaitryānyasya vastu na dadāti/ \E
%bhrātur mitrasya yogyaṃ vastu na dadāti \P
%durmaitryānyasya vastu na dadāti/ \B
%\om                               \L
%bhrātumitrasya ca yogyaṃ ca vastu na dadāti/ \N1
%bhrātumitrasya ca yogyaṃ vastu na dadāti/ \N2
%\om \D
%bhrātṛ rmitraṃ ca yogyaṃ vastu na dadāti \U1
%bhrātur mitrasya yogyaṃ  vastu na dadāti// \U2
%-----------------------------
%who does not gives [single] thing, which benefits friend and brother,  
%-----------------------------
%[p.83]
%ya asatyaṃ vadati/ yo yoganindāṃ karoti/ \E
%yo 'satyaṃ vadati yo yogināṃ manomadhye niṃdāṃ karoti \P
% \om                                     \B
% \om                                     \L
% so 'satyaṃ vadati/ yoginā manomadhye niṃdāṃ karoti/ \N1
% so 'satyaṃ vadati/ yoginā manomadhye niṃdāṃ karoti/ \N2
%\om \D
%so satyaṃ vadati yogināṃ manomadhye nikaroti \U1
%yo 'satyaṃ vadati// yogināṃ manomadhye niṃdāṃ karoti// \U2
%-----------------------------
%who does not speak the truth and complains about yoga, 
%-----------------------------
%yasya manomadhye dayā na bhavati/ \E
%yasya manomadhye dayā na bhavati  \P
%yasya manomadhye dayā na bhavati/ \B
%yasya manomadhye dayā na bhavati/ \L
%yasya manomadhye dayā na bhavati  \N1
%yasya manomadhye dayā na bhavati  \N2
%\om \D
%yasya manomadhye dayā na bhavati  \U1
%yasya manomadhye dayā na bhavati// \U2
%-----------------------------
%in this mind compassion does not arise.  
%-----------------------------
%yaḥ    kalahapriyo bhavati/ \E
%yasya  kalaha priyo bhavati/ \P
%yasya  kalahaṃ priyo na bhavati/ \B
%yasya  kalahaṃ priyo na bhavati// \L
%yaḥ    kalahapriyo     bhavati/ \N1 [em. to  yasya kalahapriyo bhavati ||
% \om N2
%\om \D
%yaḥ    kalahapriyo     bhavati \U1
%yasya  kalahaḥ priyo bhavati// \U2
%-----------------------------
% For him devotion to quarrel arises.  
%-----------------------------
%svakāryakaraṇe  sāvadhāno bhavati/ \E
%svakāryakaraṇe  sāvadhāno bhavati  \P
%svakāryākaraṇeṃ sāvadhāno bhavati/ \B
%svakāryākaraṇe  sāvadhāno bhavati// \L
%svakāryyākaraṇe  sāvadhāno bhavati/ \N1
%svakāryyākaraṇā  sāvadhāno bhavati/ \N2
%\om \D
%svakāryakaraṇe  sāvadhāno bhavati \U1
%svakāryakaraṇe  sāvadhāno bhavati// \U2
%-----------------------------
%, attention arises for him only with regard to his selfish intentions,  [em. zu kāryakāraṇa]
%-----------------------------
%guroḥ  kāryakaraṇe na dattacitto bhavati/ \E
%guroḥ  kāryakaraṇe 'nādṛto??? bhavati        \P
%guro   kārye karaṇe anādarano    bhavati/ \B [em. to anādaro = disrespect] 
%guroḥ  kārykaraṇe  anādare no    bhavati/ \Ĺ
%guroḥ  kārykaraṇe  ādaro na     bhavati/ \N1
%guro   kārykaraṇe  ādaro na     bhavati/ \N2
%\om \D
%guroḥ  kāryakaraṇe  ādaro na    bhavati  \U1
%guro   kāryakaraṇe  nādṛto      bhavati//  \U2
%-----------------------------------
%disrespect arises towards the intentions of the teacher.
%-----------------------------
%etādṛśasyāgre na yogaḥ  kriyate  na paṭhyate// \E
%etādṛśasyāgre ta yogaḥ  kriyate  na paṭhyate    \P
%etādṛśasyāgre na yogaḥ  kriyate/ na paṭhayate/ \B
%etādṛśasyāgre na yogaḥ  kriyate  na paṭhayate... \L
%etādṛśasyāgre na kriyate/  na padyaṃte//  \N1
%etādṛśasyāgre na kriyate/  na padyaṃte//  \N2
%\om \D
%etādṛśasya agre na      kriyate  na paṭhyate \U1
%etādṛśasyāgre na yogaḥ  kriyate//  na paṭhyate// \U2
%-------------------------
%As a result, he can neither do yoga nor read.
%-----------------------------
%śrṛṇvan prītādikān śabdān paśyan rūpaṃ manoharam/ \E
%śrṛṇvan gītādikān śabdān paśyan rūpaṃ manoharaṃ   \P
%śrṛṇvan gītādikān  śabdān paśyan rūpaṃ manoharaṃ/ \B
%śṛṇvan gītādikān  śabdān/ paśyan rūpaṃ manoharaṃ/ \N1
%śuśvanagītādikāna śabdāt/ paśyan rūpaṃ manoharaṃ// \N2
%\om \D
%śṛṇvan gītādikān  śabdān paśyan rūpaṃ manoharaṃ// \L
%śṛṇvan gītādikān  śabdān paśyat rūpaṃ manoharaṃ \U1
%śrṛṇvan gītādikān// śabdān paśyan rūpaṃ manoharam// \U2
%-----------------------------
%
%-----------------------------
%jāgrat sphuran spṛśansparśamṛdupriyam svādān manoramān bhrāmyan deśān/ \E
%jighran gaṃdhāṃśca surabhin spṛśan sparśaṃ mṛḍupriyaṃ  svādān manoramān khādan bhrāmyan deśān ... \P
%jighranagachan sphurabhi spṛśaṃ mṛdupriyaṃ svādān manorathān khādavan bhrāman deśān ... \B
%jighra na gachan surabhin spṛśaṃ mṛdupriyaṃ svādān manorathān khādavan bhrāman deśān ... \L
%jighran gaṃdhāṃ śusurabhīn spṛśyanasya ???mṛdupriyaṃ/ svādān manomān svādan bhrāmye na deśān\N1
%jighraṃ gaṃdhāṃ śusurabhīn spṛśyanasyaṃ śarmṛdu??priyaṃ svādān manomān bhrāmya na deśān \N2
%\om \D
%jighraṃ nāṃdhaśca surabhīn sparśaṃ mṛdupriyaṃ svādāna manoramān khādaṃtabhrāmyan tveṣāṃn  \U1 %%%302.jpg
%jighran spṛśan  gaṃdhan surabhīn spṛśan sparśaṃ mṛḍu// priyaṃ  svādān manoramān bhrāmyan deśān  \U2
%-----------------------------
%
%-----------------------------
%manoramān bhāṣamāṇaḥ ramamāṇaḥ svalīlayā/ bhāvābhāvavinirmukto sarvagrahavivarjitaḥ// 1// %[p.84] \E
%manoramān bhāṣamāṇaḥ sumadhuraṃ ramamāṇaḥ svalīlayā bhāvābhāvavinirmuktaḥ sarvagrahavivarjjitaḥ   \P  %%%7681.jpg 
%manoramān bhakṣamāṇa samaghuraramāṇa svalilayā bhāvāvinirmuktaḥ/ sarvagrāhavivarjitaḥ/            \B
%manoramān bhakṣamāṇaḥ samadhuraramamāṇaṃ svalīlayā bhāvāvinir muktaḥ sarvagrāhavivarjitaḥ...      \L
%N1 CHECK!!! 
%manoramān bhāṣamāṇasya madhuraṃ rasamānaḥsvalīlayā // bhāvābhāvavinirmuktaḥ sarvāgrahavivarjitaḥ// N2
%\om \D
%manoramān/ bhāṣamāṇasya madhuraṃ ramamāṇaḥ svalīlayā// bhāvābhāvavinirmuktaḥ sarvagrāhavivarjitaḥ  \U1
%manoramān// bhāṣamāṇaḥ sumadhuraṃ ramamāṇaḥ svalīlayā// bhāvābhāvavinirmuktaḥ sarvāgrahavivarjjitaḥ// \U2
%-----------------------------
%
%-----------------------------
%sadānaṃdamayo yogī sadābhyāsī sadā bhavet/ viruddhaduḥkhade deśe virūpeti bhayānake//1// \E
%sadānaṃdamayo yogī sadābhyāsī sadā bhavet  viruddhaduḥkhade deśe virūpeti bhayānake      \P
%sadāmayo yogī sadābhyāsī sadā bhavet/ viruddhe duḥkhe deśe śovirūpe bhayānake/ \B
%sadāmayo yoyogī sadābhyāsī sadā bhavet// viruddhe duḥkhadeśe śovirūpe bhayānake... \L
%sadānaṃdamayo yogī sadābhyāsī sadā bhavet/ viruddhe duḥkhade deśe śovirūpeti bhayānake/ \N1
%sadānaṃdamayo yogī sadābhyāsī sadā bhavet//1// viruddhe duḥkhade deśe virūpeti bhayānake/ \N2 %%%last folio verso!!!!
%\om \D
%sadānaṃdamayo yogī sadābhyāso sadā bhavet  dviruddhe duḥkhade deśe vivarūpe bhayānake \U1
%\om \U2
%-----------------------------
%
%-----------------------------
%iṣṭādyaniṣṭasaṃsparśe rase ca lavaṇādike/ pratyādāvapi gaṃdhe ca kaṃkoṣṇādi vivarjayet//2// \E
%iṣṭādhaniṣṭaṃ saṃsparśe rase ca lavaṇādike pratyādāvapi gaṃdhe ca kaṃṭakoṣyādi vivarjjite   \P
%iṣṭādyaniṣṭasaṃsparśe rase ca lavaṇādike  pratyādāvapi gaṃdhe ca kaṭakoṣmādi varji/         \B
%iṣṭādyaniṣṭasaṃsparśe rase ca lavaṇādike  pūtyādāvapi gaṃdhe ca kaṃṭakoṣmādi varji//        \L
%iṣṭādyaniṣṭasaṃsparśe/ rase ca lavaṇādike  pūtyādāvapi gaṃdhe ca kaṃṭakoṣmādi varjjite/     \N1
%\om \D
%iṣṭādyaniṣṭasaṃsparśe rase ca lavaṇādike  pūjādāvapi gaṃdhe ca kuṃṭakoṣmādi varjite         \U1
% \om \U2
%-----------------------------
%
%-----------------------------
%sarvadaiva sadābhyāsaḥ samaḥ syāt sukhaduḥkhayoḥ/ evaṃ yogasya karmmāṇi saṃkalparahitāni ca//3// %[p.85] \E
%sarvadaiva sadābhyāsaḥ samaḥ syāt sukhaduḥkhayoḥ  evaṃ yogasya karmāṇi  saṃkalparahitāni ca              \P
%sarvadeva  sadābhyāsāḥ samaḥ syāt sukhaduḥkhayoḥ/ evaṃ yogasya karmmāṇi saṃkalparahitāni ca/ \B
%sarvadeva  sadābhyāsāḥ samaḥ sya/ tsukhaduḥkhayoḥ// evaṃ yogasya karmmāṇi saṃkalparahitāni ca// \L
%sarvadeva  sadābhyāsāḥ samasyāt sukhaduḥkhayoḥ/ evaṃ yogasya karmmāṇi saṃkalparahitāni ca/ \N1
%\om \D
%sarvadeva  sadābhyāsāḥ samasyā sukhaduḥkhayoḥ evaṃ bhūtakarmāṇī saṃkalparahitāni ca        \U1
%sarvadaivaṃ sadābhyāsaḥ samaḥ syāt sukhaduḥkhayoḥ// evaṃ yogasya karmāṇi saṃkalparahitāni ca//
%-----------------------------
%
%-----------------------------
%gacchan nṝṇāṃ ca saṃsparśāttapaḥ kurvanna lipyate/  utpannatattvabodhasya hyudāsīnasya sarvadā//4// \E %%%% Amanaska 2.36 
%gacchan nṝṇāṃ ca saṃsparśātpāpaḥ kurvanna lipyate   utpannatattvabodhasya udāsīnasya sarvadā        \P
%gacchan nṝṇāṃ ca saṃsparśotpāpaṃ kurvaṃ na lipyate/ utpannatattvabodhasya udāsīnasya sarvadā    \B
%gachan nṝṇāṃ ca saṃsparśāt pāpaṃ kurvan na lipyate/ utpannatattvabodhasya udāsīnasya sarvadā    \L
%gacchan nṝṇāṃ ca saṃsparśotpāpaṃ kurvan na lipyate/ utpannatattvabodhasya udāsīnasya sarvadā//    \N1
%\om \D
%gacha  nṛṇāṃ ca saṃsparśāt pāpaṃ kurvaṃ nna lipyate utpannatatvabodhasyād udāsīnasya sarvadā    \U1
%gacchan nṝṇāṃ ca saṃsparśāt pāpaṃ kurvanna lipyate// utpannatattvabodhasya udāsīnasya sarvadā// \U2
%-----------------------------
%
%-----------------------------
%tadā dṛṣṭiviśeṣaḥ syādvividhānyāsanāni ca/ aṃtaḥ karaṇajā bhāvā yogino nopayoginaḥ//5// \E %%%Amanaska 2.37
%tadā dṛṣṭiviśeṣasyād vidhānyāsanāni ca   aṃtaḥ karaṇajā bhāvā yogino nopayoginaḥ \P
%tadā dṛṣṭiviśeṣasyād vidhānyāsanāni ca/ aṃtaḥ karaṇajā bhāvā yogino nopayoginaḥ/ \B
%tadā dṛṣṭiviśeṣasyād vidhānyāsanāni ca// aṃtaḥ karaṇajā bhāvā yogino nopi yoginaḥ// \L
%tadā dṛṣṭiviśeṣaḥ syād vidhānyāsanāni ca/ aṃtaḥ karaṇajā bhavā yogino nopa yoginaḥ/ \N1
%\om \D
%tadā dṛṣṭir viśeṣasyād vidhānyāsanāni ca  aṃtaḥ karaṇayo bhāvā yogino nopa yoginaḥ \U1
%tadā dṛṣṭiviśeṣaḥ syād vividhāny āsanāni ca// aṃtaḥ karaṇajā bhāvā yogino nopi yoginaḥ// \U2
%-----------------------------
%
%-----------------------------
%sarvarājapadasthasya niṣkalādhyātmavedinaḥ/ yadyatprayatnaniḥ pāyaṃ tattatsarvamakāraṇam// 6// \E
%sarvadā sahajasthasya niṣkalādhyātmavedinaḥ yadyatprayatnaniḥ pārdhaṃ tattatsarvam akāraṇam    \P
%sarvadyasahajasyaniṣkalādhyātmavedinā/      yadyatprayatnaniḥ pādya tat sarvamakāraṇāt/ \B
%sarvadyasahajasthasya niṣkalādhyātmavedinā//      yadyatprayatnaniḥ pādya tat sarvemikāraṇāt//2// \L
%sarvadā sahajasthasya niṣkalādhyātmavedina/      yadyatprayatnaniḥ pādyaṃ tattatsarvamakāraṇaṃ/ \N1
%\om \D
%sarvadā mahajarasya   niṣkalādhyātmavedanā  yadyataprayatra niṣyayiṃ tat sarvam akāraṇāṃ  \U1
%sarvadā sahajasthasya niṣkalādhyātmavedinaḥ//  yadyatprayatna niḥpādyaṃ tat sarvaṃ kāraṇa//  \U2
%-----------------------------
%
%-----------------------------
%vilāsinīnāṃ manohārigānaśravaṇāt/ \E
%vilāsinīnāṃ manohārigānaśravaṇāt  \P
%vilāsinīnāṃ manohārigānaśravaṇāt/ \B
%vilāsinīnāṃ manohārigānaśravaṇāt// \L
%vilāsinīnāṃ manohārigītaśravaṇāt/ \N1
%\om \D
%vilāsinīnāṃ manohārigītaśravaṇāt \U1
%vilāsinīnāṃ manohārigānaśravaṇāt// \U2
%-----------------------------
%
%-----------------------------
%atisauṃdaryakāminīnāṃ rūpadarśanāt/ \E
%atisuṃdaraṃ kāmināṃ rūpadarśanāt    \P
%atisauṃdarakāminināṃ  rūpadarśanāt/ \B
%atisuṃdarakāminināṃ  rūpadarśanāt// \L
%atisuṃdarakāminīnāṃ  rūpadarśanāt// \N1
%\om \D
%atisuṃdarakāminīnāṃ  rūpadarśanāt  \U1
%atisuṃdarakāminīnāṃ  rūpadarśanāt//  \U2
%-----------------------------
%
%-----------------------------
%kastūrīkar pūrayor gaṃdhagrahaṇāt/ \E
%kastūrīkar pūrayor gaṃdhagrahaṇāt \P
%kastūrīkar pūrayor gaṃdhagrahaṇāt/ \B
%kastūrīkar pūragaṃdhayor grahaṇāt/ \L  %%%S.15 Anfang!!
%\om \D
% ????????????????????????? \N1???
%kastūrikar puro    gaṃdhagrahaṇāt \U1
%kastūrīkar pūrayo gaṃdhagrahaṇāt// \U2
%-----------------------------
%
%-----------------------------
%manaḥ śaityakāri komalavastunaḥ sparśakāraṇāt/ \E
%manaḥ śaityakāri komalavastunaḥ sparśakāraṇāt \P
%manaḥ śaityakāri komalavastunaḥ saṃsparśakāṃ... \B
%manaḥ śaityakāri komalavastunaḥ/ saṃsparśakaṃ... \L
%manaḥ śītalakārī atikomalaparavastunaḥ sparśakaraṇāt// \N1
%\om \D
%manaḥ sīlakārī  atikomalaparavastutaḥ sparśakaraṇāt \U1
%manaḥ śaityakāri komalavastunaḥ sparśakāraṇāt// \U2
%-----------------------------
%
%-----------------------------
%atimādhuryaṃ citte karoti/ \E
%atimādhuryaṃ citte karoti \P
%atimādhuryaṃ citte laroti/ \B
%atimādhuryaṃ citte karoti// \L
%atimādhuryaṃ citte karoti/ \N1
%\om \D
%atimādhuryaṃ citte karoti \U1
%atimādhuryaṃ cikrī karoti/ \U2
%-----------------------------
%
%-----------------------------
%tādṛśaḥ svādanāt/ \E
%tādṛśaḥ svādanāt \P
%tādṛśaḥ svādanāt/ \B
%tādṛśaḥ svādanāt// \L
%tādṛśā  svādanāt/ \N1
%\om \D
%tādṛśā  svādanāt \U1
%tādṛśā  svādanāt// \U2
%-----------------------------
%
%-----------------------------
%anekadeśānāṃ sādhvasādhusthā na darśanāt/ \E
%anekadeśānāṃ sādhvasādhusthā na darśanāt  \P
%anekadeśānāṃ sādhvasādhusthā na darśanāt/ \B
%anekadeśānāṃ sādhvasādhusthā na darśanāt// \L
%anekadeśānāṃ sādhvasādhusthā na darśanāt/ \N1
%\om \D
%anekadeśānāṃ sādhūsthā na darśanāt \U1
%anekadeśānāṃ sādhvasādhusthā na darśanāt// \U2
%-----------------------------
%
%-----------------------------
%mitreṇa saha komalavacanāt/ \E
%maitreṇa saha komalavacanāt \P
%maitreṇa saha komalavacanāt/ \B
%maitreṇa saha komalavacanāt// \L
%maitreṇa saha komalavacanāt/ \N1
%\om \D
%maitreṇa saha komalavacanāt \U1
%maitreṇa saha komalavacanāt// \U2
%-----------------------------
%
%-----------------------------
%śatruṇā saha kaṭhinavacanāt/ \E
%śatruṇā sahakaṃ vinya vacanāt  \P
%śatruṇā saha kaṭhinya vacanāt/ \B
%śatruṇā saha kāṭhinya vacanāt// \L
%śatruṇāṃ saha kaṭhinya vacanān \N1
%\om \D
%śatṛṇā saha kāṭhinya vacanāt    \U1
%śatruṇāṃ saha kāṭhinya vacanāt// \U2
%-----------------------------
%
%-----------------------------
%yasya manasi harṣo vā dveṣo na bhavati sa puruṣa īśvaropadeśiko jñeyaḥ/ \E
%yasya manasi harṣo vā dveṣo na bhavati sa puruṣa īśvaropadeśako jñeyaḥ \P
% missing last folio \B
% yasya mana harṣo vā dveṣo bhavati sa puruṣa īśvaropade ko jñeyaḥ/ \L
% yasya manasi harṣo vā dveṣo na bhavati/ sa puruṣa īśvaropadeśako jñeyaḥ// \N1
%                                   vati// sa puruṣa īśvaropadeśako jñeyaḥ// \D
%yasya manasī harṣo vā dveṣo vā na bhavati sa puruṣa īśvaropadeśako jñeyaḥ \U1
%yasya manasī harṣo vā dveṣo na bhavati// pururṣo īśvaropadeśako jñeyaḥ// \U2
%-----------------------------
%
%-----------------------------
%[p.87] svalīlayā vadati calati bhāvābhāvayościttamudāsīnaṃ bhavati kasyāṃcidvārtāyāṃ harṣaviṣādaṃ na karoti yasya manaḥ sahajānaṃde magnaṃ bhavati/ \E
%       svalīlayā vadati calati va  bhāvābhāvayościttamudāsīnaṃ bhavati kasyāṃcidvārttāyāṃ haṭhaṃ na karoti yasya manaḥ sahajānaṃde magnaṃ bhavati \P
% missing last folio \B
% svalīlayā yā vadati calati ca bhāvābhāvayoś cittamudāsīna  bhavati// kasyāṃcid vārttāyāṃ haṭaṃ na karoti// yasya manaḥ// sahajānaṃdam añjaṃ bhavati/ \L
% svalīya yā vadati calati ca bhāvābhāvayoś cittamudāsīnaṃ  bhavati/ kasyāṃcid vārttāyāṃ haṭhaṃ na karoti yasya manaḥ sahajānaṃde magnaṃ bhavati/ \N1
%       svalīla yā vadati calati ca bhāvābhāvayoś cittamudāsīnaṃ  bhavati// kasyāṃcid vārttāyāṃ haṭhaṃ na karoti yasya manaḥ sahajānaṃde magnaṃ bhavati/ \D
%       svalīlayā vadati calati ca bhāvābhāvayoś cittamudāsīnaṃ  bhavati kasyāṃcid vārttāpāṃ haṭaṃ na karoti yasya manaḥ   sahajānaṃda  saṃjñaṃ bhavati \U1 %%%304.jpg
% svalīlayā yā vadati calatī ca// bhāvābhāvayoś cittamudāsīnaṃ  bhavati// kasyāṃcid vārttāyāṃ haṭhaṃ na karoti// yasya manaḥ  sahajānaṃdaṃde magnaṃ bhavati// \U2
%-----------------------------
%
%-----------------------------
%tena puruṣeṇa dṛṣṭiḥ sthirā karttavyā/ \E
%tena bhya puruṣeṇa dṛṣṭiḥ sthirā karttavyā \P
% missing last folio \B
% tena puruṣeṇa dṛṣṭiḥ sthirā karttavyā/ \L
% tena puruṣeṇa dṛṣṭiḥ sthirā karttavyaṃ// \N1
% tena svapuruṣeṇa dṛṣṭiḥ sthirā karttavyaṃ// \D
% tena puruṣeṇa dṛṣṭisthirā karttavyaḥ \U1
%tena puruṣeṇa dṛṣṭiḥ sthirā karttavyā// \U2
%-----------------------------
%
%-----------------------------
%āsanaṃ dṛḍhaṃ karttavyam/ \E
%āsanaṃ dṛḍhaṃ karttavyaṃ  \P
% missing last folio \B
% āsanaṃ dṛḍhaṃ karttavyaṃ/  \L
% āsanaṃ dṛḍhaṃ karttavyaṃ/  \N1
% āsanaṃ dṛḍhaṃ karttavyaṃ//  \D
% āsanadṛḍhaṃ karttavyaṃ  \U1
%āsanaṃ dṛḍhaṃ karttavyam// \U2
%-----------------------------
%
%-----------------------------
%pavanaḥ sthiraḥ karttavyaḥ/ \E
%pavanaḥ sthiraḥ karttavyaḥ \P
% missing last folio \B
% \om \L
% pavanaḥ sthiraḥ karttavyaḥ \N1
%pavanaḥ sthiraḥ karttavyaḥ// \D
%pavanaḥ sthiraḥ karttavyaḥ \U1
%pavanaḥ sthiraḥ karttavyaḥ// \U2
%-----------------------------
%
%-----------------------------
%etādṛśaḥ kaścinniyamaḥ/ \E
%etādṛśaḥ kaścinniyamaḥ   \P
% missing last folio \B
%etādṛśaḥ// kaścinniyamaḥ// \L
%etādṛśaḥ kaścinniyamaḥ/ \N1
%etādṛśaḥ kaścin niyamaḥ// \D
%etādṛśaḥ kaścinīyamaḥ \U1
%etādṛśaḥ kaścinnīyamaḥ// \U2
%-----------------------------
%
%-----------------------------
%siddhasya noktaḥ manaḥpavanābhyāṃ yadā sahajānaṃdasvasvarūpeṇa prakāśyate sa sahajayogaḥ kathyate/ \E
%siddhasya noktaḥ manaḥ pavanābhyāṃ yadā sahajānaṃdaḥ svasvarūpeṇa prakāśyate sa sahajayogaḥ kathyate \P
% missing last folio \B
%siddhasya noktaḥ// manapavanābhyāṃ yadā sahajānaṃdasvasvarūpeṇa prakāśyate sa sahajayogaḥ// kathyate... \L
%siddhasya noktaḥ/ manaḥ pavanābhyāṃ yadā sahajānaṃdaḥ/ svasvarūpeṇa prakāśyate sa sahajayoga kathyate/ \N1
%siddhasya noktaḥ// manaḥ pavanābhyāṃ yadā sahajānaṃdaḥ// svasvarūpeṇa prakāśyate sa sahajayoga kathyate// \D
%siddhasya noktaḥ  manaḥ pavanābhyāṃ padā sahajānaṃdaḥ svasvarūpeṇa prakāśate sa sahayogaḥ kathyate \U1
%siddhasya noktaḥ manaḥ pavanābhyāṃ yadā sahajānaṃdaḥ svasvarūpeṇa prakāśyate// sa sahajayogaḥ kathyate// \U2
%-----------------------------
%
%-----------------------------
%te madhye iti cakravarttikathanam// \E
%te madhye iti cakravarttikathanam// \P
% missing last folio \B
%rājayogamadhye... \L
%rājayogamadhye iti cakravarti kathyate
%rājayogamadhye iti cakravartti  nāma kathanaṃ// iti śrī paramarahasyāṃ śrīrāmacaṃdraviracitāyāṃ tatvayogabiṃdu samāptaḥ// //śrī svasti// //saṃvat 837   \N1  %%%% 1716 n. Chr.
%rājayogamadhye iti cakracarttī nāma kathanam // iti śrī paramarahasye śrīrāmacaṃdraviracitāyāṃ tatvayogabindu samāptam// //śubham// yadakṣarapadabhraṣṭaṃ mātrāhīnaṃcaya bhavet// tat sarvaṃ kṣamya ?? meca prasīdaparameśvara //1 // sūrye turaṅge navacandraghasre jyeṣṭhākhyakṛṣṇe bhṛguvārayuktam || tattvaprayogaḥ ṣaḍaharṣasaṇjñaṃ likhitaṃ suhetoḥ bhavatīha dehi || bhūyāt \N2
%rājayogamadhye iti cakravarttī  nāma kathanaṃ// iti paramarahasyāṃ śrīrāmacaṃdraviracitāyāṃ tatvayogabiṃdu samāptaḥ// śubhamastu/ saṃvat 1841// bhādau śudha 15tnīo vesarva śake rārāma rāma cha    \D
%rājayogamadhye iti cakravaktya?? nāma madhye iti cakravaktye? nāma madhye kathanaṃ  \U1
%rājayogasya madhye iti cakracarti kathyate// \U2
%-----------------------------
%
%-----------------------------
%iti śrīsarvaguṇasampannapaṃḍita-sukhānandamiśrasūrisūnupaṇḍita-jvālāprasādamiśrakṛtabhāṣāṭīkāsahito rājayoge binduyogaḥ samāptaḥ// śubhamastu//śrīrastu// \E
%iti śrīrāmacaṃdraparamahaṃsa viracitas tatvabinduyogasamāptaḥ saṃvat 1867 pauṣakṛṣṇaḥ 12 ravau śubham bhuyāt //??//\P
% missing last folio \B
% iti rājamacaṃdraparahaṃsa viracites tatvabiṃduyogasamāptaṃ// śrī kṛṣṇārpaṇamastu// cha// \L
% iti śrī paramarahasyāṃ śrīrāmacaṃdraviracitāyāṃ tatvayogabiṃdu samāptaḥ// //śrī svasti// //saṃvat 837   \N1  %%%% 1716 n. Chr.
% iti śrī pāramahaṃsyāṃ śrī rāmacaṃdraviracitāyāṃ tatvayogaviduḥ samāptaḥ śubhaṃ bhūyāt // // atarlakṣyaṃ bahi dṛḍhir nirmeṣomeṣa varjitaḥ saiṣāśāṃbhavīmudrā sarvata,n treṣugopitā 1 aṃtark ..... %see last 2 folios verses beyond text quote from fourth chapter of HP \U1
% iti śrī rāmacaṃdraparamahaṃsaviracitas tatvabiṃduyogasamāptaḥ// śrī śubhaṃ bhavatu// śrīsītārāmārpaṇamastuḥ// idaṃ pustakaṃ// śake 1805// vikramārka saṃmat// 1140// jayanām asaṃvatsare// udagayaṇe// griṣmaṛtau?// vaiśālhemāse// kṛṣṇapakṣe// tithau 23// bhānuvāsare// prathamayāmye// śrī kṣetra avaṃtikāyāṃ// śrī mahārudramahākālasaṃnidhāne na saṃpūrṇaṃ// lekhanaṃ ānaṃt? suta bābājoo rājadherakareṇa likhyate// yādṛśaṃ pustakaṃ dṛṣtvā tādṛsaṃ likhitaṃ mayā// yadi śuddhaṃ aśuddho cā mama doṣo na dīyate//1// śrīrāma// cha//    \U2
%-----------------------------
%
%-----------------------------
%%%%deciphering last folio margin note of %N1!!!  
%\begin{alignment}[
%    texts=edition[class="edition"];
%    translation[class="translation"],
%  ]
%\begin{edition}
% \ekddiv{type=ed}
%\begin{prose}homa\end{prose}
%\end{edition}
%\begin{translation}
%  \ekddiv{type=trans}
%  \begin{tlate}\end{tlate}
%   \end{translation}
% \end{alignment}
%\begin{otherlanguage}{english}
\chapter{Translation of the Yogatattvabindu}    
\ekddiv{type=trans}
\centerline{\textrm{\small{[Introduction]}}}
\bigskip
\begin{tlate}
Homage to Śrī Gaṇeśa. Now the methods of Rājayoga are laid down. This is the result of Rājayoga\footnote{This statement seems unconnected to the definition of rājayoga that follows.}: Rājayoga is that by which longterm durability of the body arises even amongst manifold royal pleasures even amongst the manifold royal entertainments and spectacle. This truly is Rājayoga. These are the varieties of this Rājayoga:
\noindent 1. Kriyāyoga, the Yoga of [mental] action; 2. Jñānayoga, the Yoga of knowledge; 3. Caryāyoga, the Yoga of wandering;\footnote{The first three Yogas allude to the four \textit{pāda}s of the Śaiva \textit{āgama}s; namely \textit{kriyā[pāda], caryā[pāda], yoga[padā]} and \textit{jñāna[pāda]}.\parencite[77]{nishvasa2015}.} 4. Haṭhayoga, the Yoga of force; 5. Karmayoga, the Yoga of deeds; 6. Layayoga, the Yoga of absorption; 7. Dhyānayoga, the Yoga of meditation, 8.Mantrayoga, the Yoga of Mantras; 9. Lakṣyayoga, the Yoga of fixation objects, 10. Vāsanāyoga, Yoga of mental residues; 11. Śivayoga, the Yoga of Śiva, 12. Brahmayoga, the Yoga of Brahman; 13. Advaitayoga, the Yoga of non-duality; 14. Siddhayoga, the Yoga of the Siddhas; 15. Rājayoga, the King of Yogas. These are the fifteen \textit{yoga}s.\footnote{At the current stage of research it is not clear if this list is a later addition by another scribe or, if indeed it originally stems from Rāmacandra. The list suggests a text following the order of yogas according to this list. However, the order of the yogas given in the list is not followed closely in the text.}
\end{tlate}
%%%%%%%%%%%%%
 \begin{tlate}
   \ekddiv{type=trans}
      \centerline{\textrm{\small{[Description of \textit{kriyāyoga}]}}}
      \bigskip
Now the characteristic of Kriyāyoga, the Yoga of [mental] action\footnote{In comparison to the Pātañjalean variant of Kriyāyoga, this variat consists of specific mental actions.} are described.
\paragraph{1.} This Yoga is liberation through [mental] action, it bestows success(\textit{siddhi}) in ones own body. Each wave the mind creates at the beginning of an action, of all those one shall withdraw oneself. Then Kriyāyoga ari
\paragraph{2.} Patience, discrimination, equanimity, peace, modesty, desireless: The Yogī who is endowed with these means is said to be a Kriyāyogī. 
\paragraph{3.} Envy, selfishness, cheating, violence, desire and intoxication, pride, lust, anger, fear, laziness, greed, error and impurity. 
\paragraph{4.} Attachment and aversion, indignation and idleness, impatience and dizzyness: Whoever doesn't experience these is called a Kriyāyogī.\footnote{The source of the four verses on Kriyāyoga is unknown.} \bigskip \bigskip

Patience, discrimination, equanimity, peace, contentment etc. are generated in his mind. He alone is called a Yogī of many actions (\textit{bahukriyāyogī})\footnote{The term \textit{bahukriyāyogī} seems to be unique in yoga literature.} Fraud, illusion, property,violence, craving, envy, ego, anger, anxiety, shame, greed, error, impurity, attachment, aversion, idleness, heterodoxy, false view, affection of the senses, sexual desire: He who diminishes these from day to day in is mind, he alone is called a Yogī of many actions (\textit{bahukriyāyogī}).
   \end{tlate}
   %%%%%%%%%%%%%%%%%%%%
   %%%%%%%%%%%%%%%%%%%%
   %%%%%%%%%%%%%%%%%%%%
   %%%%%%%%%%%%%%%%%%%%
  \ekddiv{type=trans}
        \bigskip
    \centerline{\textrm{\small{[Varieties of \textit{rājayoga}: Siddhakuṇḍalinīyoga and Mantrayoga]}}}
    \bigskip
    \begin{tlate}
      Now varieties of Rājayoga will be described. Which are these? One is Siddhakuṇḍalinīyoga\footnote{On the one hand it suprises that we find the term Siddhakuṇḍalinīyoga instead of Siddhayoga as given in the initial list, on the other hand it is suprising that this type of Yoga, given as the second last item in the Yoga taxnomy is introduced as the second type right after Kriyāyoga, which was the first item in the initial list as well as in the following material.What makes this term even more strange is the fact that \textit{kuṇḍaliṇī} is not mentioned at all in the following description of this type of Yoga.} [and one\footnote{It is not entirely clear if those are two different Yogas or one and the same type of Yoga. Just the pretty late witness U2 gives us a sort of description of Mantrayoga. Judging on the basis of U2 only one could translate ``One is Siddhakuṇḍalinīyoga being Mantrayoga.'' Judging by the contents given by the rest of the witnesses this passage leaves a big queastion mark.}] is Mantrayoga\footnote{It seems odd that Mantrayoga is mentioned in the same breath as Sidhdakuṇḍalinīyoga, even though it is not directly expressed in the following. Just the additional descriptions of witness U2, highlighted in a different colour than the main text, indirectly refers to a certain practice of Mantra which is \textit{japājapa} of the \textit{so 'haṃ} for a certain duration of the practioce of meditation that is presrcibed to be performed on every \textit{cakra}.}. These two Rājayogas are described [in the following]. At the location of the root-bulb exists one major vessel in the form of energy. This single vessel reaches to these openings which are \textit{iḍā}, \textit{piṅgalā} and \textit{suṣumnā}. On the left side is the \textit{iḍā}-channel, being a resemblence of the moon. On the right side exists the \textit{piṅgalā}-channel, being a resemblence of the sun. Within the middle path is a lotuspond being very subtle. [It is] made from a web of light [and it] shines like a thousand lightnings. She \extra{emerges as the central channel, assuming the form of benevolence (\textit{śiva}),} is the bestower of enjoyment and liberation. While abiding in (\textit{satyāṃ}) her (\textit{asyāṃ}) knowledge arises [to the point of which] the person becomes all-knowing.
    \end{tlate}
    %%%%%%%%%%%%%%%%%%%
    %%%%%%%%%%%%%%%%%%%
    %%%%%%%%%%%%%%%%%%%
    %%%%%%%%%%%%%%%%%%%
     \ekddiv{type=trans}
      \bigskip
    \centerline{\textrm{\small{[Description of the first Cakra]}}}
    \bigskip
    \begin{tlate}
      The means for the genesis of knowledge in the central channel will now be described. At the beginning\footnote{Supposedly at the beginning of the central channel.} exists the root \textit{cakra} having four petals. \extra{The first \textit{cakra} of support (\textit{ādhāra}) is at the anus [and] is red-colored. Gaṇeśa is the deity. He is success, intelligence and power. A rat is the mount. The Ṛṣi is Kūrma. The seal is contraction. The vitalwind is \textit{apāna}. The \textit{kalā} is the ``wave of consciousness'' (\textit{urmī}). The concentration is ``she who is powerful'' (\textit{ojasvinī}). In the four petals [of it resides] \textit{rajas}, \textit{sattva}, \textit{tamas} and the mind-faculties (\textit{manāṃsi}), [symbolized by the syllables or \textit{bīja}s] vaṃ śaṃ ṣaṃ and saṃ. A trident is situated in the middle of the triangle\footnote{This passage is odd since a triagle wasn't mentioned before.}.} In the middle is a trident, and \textit{kāmapīṭha}\footnote{Discuss the term \textit{kāmapīṭha}.} in the shape of a triangle. In the middle of this seat (\textit{pīṭha}) exists a single form in the shape of a flame. By meditating on this form the whole literature, all \textit{śāstra}s, all poems, dramas etc., everything [related to] elocution, appears in the mind of the person without learning. \extra{[Assigned to it] is external bliss\footnote{Discuss the four blisses.}, yogic bliss, heroic bliss [and] the bliss of coming to rest.}\footnote{It is noteworthy that only the first \textit{cakra} adds a detailled description of mounts, Ṛṣis, gods, seals and so forth among the current majority of witnesses at hand: E, P, L and U2. All other descriptions of the remaining eight \textit{cakra}s leave this out. The only exception is U2, a relatively late witness that adds similar descriptions for the other \textit{cakra}s as well. Since they are interesting for the history of the text I have added them to the edition's text. To indicate the extra status of those passages I have highlighted them in blue color.} An [over] hundredfold recitation of the non-recited [śataḥ = \ldots hundreds of?];  600 [repetitions for]; 9 \textit{ghaṭi}s [and] 40 \textit{palā}s.\footnote{Instructions for the duration of practice are found in all additions of U2 for each \textit{cakra}. It's not entirely clear if either the duration of meditation on the respective cakra, or the duration for the items in the list being visualised by the practitioner are meant here. However, to it seems to be done for the duration of 600 \textit{ajapājapa}, the ritualized repetition of the \textit{ajapā}, which is the voiceless uttering of the ``natural'' \textit{mantra} of the breath: \textit{so 'haṃ} - \textit{haṃ sa}. I suppose this means the practice is to be done for 600 in- and exhalations. The following part of the entry, namely ``\textit{ghaṭi} 9 \textit{palāni} 40'', probably refers to the exact time in which those 600 \textit{ajapājapa}s shall be performed. One \textit{ghaṭi} equals 1/60 of a day, which is 24 minutes. One \textit{pala} equals 1/60 of a \textit{ghaṭi} which is 24 seconds. This would equal 232 minutes or 3 hours and 52 minutes. Dividing the 600 \textit{ajapājapa}s by 232 minutes, this would result in a very slow frequence of breath of 2,586206897 in- and exhalations per minute.}
  \end{tlate}
    %%%%%%%%%%%%%%%%%%%
    %%%%%%%%%%%%%%%%%%%
    %%%%%%%%%%%%%%%%%%%
    %%%%%%%%%%%%%%%%%%%
  \ekddiv{type=trans}
      \bigskip
    \centerline{\textrm{\small{[Description of the second Cakra]}}}
    \bigskip
    \begin{tlate}
      Now the second, the six-petalled \textit{Svādhiṣṭānacakra} known as the seat of \textit{Uḍḍīyāṇa}\footnote{Discuss the term \textit{uḍḍīyāna}.}. \extra{The gender is the location. The color is yellow. The shine is yellow. \textit{Rajas} is the quality. The deity is Brahmā. The speech is \textit{vaikharī}\footnote{vaikharī f. in Kaśm. Śiv. °the 4. form of appearance of \textit{parā}, the empirical speech sound, Utpala's Ṭīkā to Śivadṛṣṭi 2, 7. [B.]― Schmidt p. 337. Welches Buch???} (\textit{vaikharī vāca}). The power is Sāvitrī. The mount is the goose. The \textit{Rṣi} is Vahaṇa. The appearance (\textit{prabhā} is the fire of love (\textit{kāmāgni}). The body is gross, The state is that of being awake. The Veda is Ṛg. The spiritual guide is the characteristic (\textit{liṅga}). The liberation is residing in the world of Brahma. The principle is pure level (\textit{śuddhabhūmikā}). The sphere is smell. The vitalwind is \textit{apāna}. The internal matrix [is]: vaṃ bhaṃ maṃ yaṃ raṃ laṃ. The external matrix: Kāmā ``she who is desire'', Kāmākhyā ``she who is the \textit{tīrtha} of \textit{Kāmākhyā}''\footnote{The Kāmākhyā is situated in Kāmarūpa on the Nīlakūṭa mountain in present day Assam. It's strange that it appears here, since Kāmarūpa appears already as the \textit{tīrtha} associated with the first \textit{cakra}.}, Tejasvinī ``she who is shining'', Ceṣṭikā ``she who is active'', Alasā ``she who is lazy'' [and] Mithunā ``she who is \textit{mithunā}''. A [more than] thousandfold recitation of the non-recited; 6000 [repetitions for]; 16 \textit{ghaṭi}s [and] 40 \textit{palā}s.\footnote{The practice is supposed to be done for the duration of 6000 \textit{ajapājapa}s divided into \textit{ghaṭi}s and 40 \textit{pala}s, resulting in 2320 minutes or 38,67 hours. Again this would result in a frequence of breath of 2,586206897 in- and exhalations per minute.}} In its middle exists extremely red glow. The adept becomes very handsome through meditation on it. \extra{He becomes one who is desired by young women.} The vital force increases from day to day.
    \end{tlate}
   %%%%%%%%%%%%%%%%%%%
    %%%%%%%%%%%%%%%%%%%
    %%%%%%%%%%%%%%%%%%%
    %%%%%%%%%%%%%%%%%%%
  \ekddiv{type=trans}
    \bigskip
    \centerline{\textrm{\small{[Description of the third Cakra]}}}
    \bigskip
    \begin{tlate}
      The third, a lotus with ten petals exists at the location of the navel. \extra{The colour is red (\textit{kapila}). Viṣṇu is the deity. Lakṣmī is the power. Vāyu is the Rṣi. Samāna is the vitalwind. The mount is Garuḍa. The deity is the suble body\footnote{Why another deity is given here?}. The state is sleep. The speech is the inaudible speech (\textit{madhyamāvāg})\footnote{<Śā, Ling>name of the speech which is inaudible and which is of the type of a thought without any definite presence of words making up the expression. Vkp I.143.<Abhyankar 1986: 300>}. The Veda is the Yajurveda. The [fire is the] southern fire. The liberation is ``proximity'' (\textit{samīpatā}).\footnote{What is this exactly?}. Viṣṇu is the characteristic of the teacher (\textit{guruliṅga}). The principle is water. The sphere is athmosphere (\textit{rajo viṣaya}). There are ten petals [and] ten matrices. [The] inner matrix: \textit{ḍaṃ ṭaṃ ṇaṃ taṃ thaṃ daṃ dhaṃ naṃ paṃ phaṃ}. The external matrix: Śānti ``she who peaceful'', Kṣamā ``she who is patient'', Medhā ``she who is insightful'', Tanayā ``the daughter'', Medhavinī ``she who is a learned teacher'', Puṣkarā ``she who is a lotus'', Haṃsagamanā ``she who moves like a swan'', Lakṣyā ``she who is the object aimed at'', Tanmayā ``she who is absorption'' and Amṛtā ``she who is immortality''. A [more than] thousandfold recitation of the non-recited; 6000 [repetitions for]; 16 \textit{ghaṭi}s [and] 40 \textit{palā}s.\footnote{Here we find the same instruction as in the previous description of the second \textit{cakra}. The practice is supposed to be done for the duration of 6000 \textit{ajapājapa}s divided into \textit{ghaṭi}s and 40 \textit{pala}s, resulting in 2320 minutes or 38,67 hours. Again this would result in a frequence of breath of 2,586206897 in- and exhalations per minute.}} In its middle exists a \textit{cakra} with five angles. In its middle is a single [divine] form. It's not possible to describe her shine with speech. Through the execution of meditation on this [divine] form the body of the person is going to be strong.
 \end{tlate}
   %%%%%%%%%%%%%%%%%%%
    %%%%%%%%%%%%%%%%%%%
    %%%%%%%%%%%%%%%%%%%
    %%%%%%%%%%%%%%%%%%%
         \ekddiv{type=trans}
       \bigskip
    \centerline{\textrm{\small{[Description of the fourth Cakra]}}}%%%%%See Jogpradipikaya Edition Page 163 
       \bigskip
         \begin{tlate}
           The fourth lotus having twelve-petals exists in the middle of the heart. \extra{[The] place of the Anāhatacakra is within the heart\footnote{This is redundant.}. The color is white. The quality is Tamas. The deity is Rudra. The power is Umā. The Ṛṣi is Hiraṇyagarbha. The mount is Nandi. The vitalwind is Prāṇa. The body is the cause of digits of light. The state is deep sleep. The speech is Paśyantī\footnote{Add footnote of entry in \textit{Tāntrikābhidhānakośa}.}. [The Veda is] Sāmaveda. The fire is the fire of the householder\footnote{Add explanation.}. The characteristic is Śiva. The level is the ability to attain everything on earth\footnote{Quote \textit{Tantrikābhidhānakośa}.}. The liberation is uniform [with the deity]. [There are] twelve petals, [and] twelve matrices: kaṃ khaṃ gaṃ ghaṃ ṇaṃ caṃ chaṃ jaṃ jhaṃ yaṃ taṃ [and] thaṃ. The external matrix: Rudrāṇī ``she who is Rudra's wife'', Tejasā ``she who is brilliant''\footnote{To be understood as \textit{tejasvinī}.}, Tāpinī ``she who is glow'', Sukhadā ``she who bestows happiness'', Caitanyā ``she who is consciousness'', Śivadā ``she who bestows grace'', Śānti ``she who is peaceful'', Umā ``she who is glorious'', Gaurī ``she who is beautiful'', Mātarā ``she who is bestowing the mother'', Jvalā ``she who is the flame'' [and] Prajvālinī ``she who is blazing''. A [more than] thousandfold recitation of the non-recited; 6000 [repetitions for]; 16 \textit{ghaṭi}s [and] 40 \textit{palā}s.\footnote{The \textit{ajapājapa} for this \textit{cakra} is to be performed 6000 times for a duration of 96 \textit{ghaṭi}s and 40 \textit{pala}s, resulting in 2320 minutes or 38.67 hours. Again this would result in a frequence of breath of 2,586206897 in- and exhalations per minute.}} Due to being made of [such an] intense light [the fourth lotus] is not in the range of sight. In its middle exists a lotus facing downward having eight petals. \extra{The mind resides in the \textit{cakra}. The mind is the deity. The power is external\footnote{n Muktabodha-Texte sehe ich 3 Belege für bahiśśakti Muktabodha/krīyakramādyotikā.html 2938 suṣirānte bahiśśaktiṃ vinyasedvyomarūpiṇīm | tasyā madhye tu Muktabodha/sakalāgamasārasaṅgraha.html 2186 suṣirāntabahiśśaktiṃ vyāpinīṃ cintayet tataḥ || Muktabodha/kriyakramadyotikavyākhyā.html 1846 tanmadhye ca bahiśśaktiṃ sudhābindu parisrutim}, [its] Ṛṣi is the self. In the middle of the navel exists a lotus. Its stalk measures ten \textit{aṅgula}s. The stalk of it is soft (\textit{komala}), pure [and] facing downwards. In its middle is [something] shining like a banana-flower. The mind has no determination of will, giving a firmer direction to man's thoughts for the moment by means of [conscious] submission. [It is] truly changeable in nature.} \extra{While the mind rests on the eastern petal [which is] white in colour clear intellekt arises, which is [endowed with] \textit{dharma}, fame and knowledge etc. While [the mind rests on] the south-east, [which is] reddish in color a mind that is weak due to sleep, laziness and illusion arises. While [the mind is situated] in the right south, [which is] black in color the generation of anger arises. While [the mind is situated] in the southwest, [which is] blue in color a mind of pride arises. While [the mind is situated] in the west, [which is] brown in color a mind that is longing for play, laughing, and celebration arises. While [the mind is situated] in the northwest, [which is] dark in color a mind which is restless by sorrow arises. While [the mind is situated] in the north, [which is] yellow in color a very happy mind with erotic and enjoyment arises. While [the mind is situated] in north-east [which is] whitish in color a mind of unity through knowledge arises.}
           
It's said that in its middle is the place of the \textit{prāṇa}-vitalwind [and] in the middle [of] the eight-petalled lotus is a pericarp (\textit{karṇikā}) in the form of a \textit{liṅga}. The technical designation of her is \textit{kalikā}. In the middle of this \textit{kalikā} exists a single thumbsized [divine] figurine (\textit{puttalikā}) being similiar to a ruby-gem in color. Her technical designation is embodied soul (\textit{jīva}). Not even with a thousand tongues it is possible to talk about her nature and her power. Here it is said [that]: ``Because of the exercise of meditation on this form the inhabitants of the universe [which are] Humans, Gandharvas, Kinnaras, Guhyakas, Vidyādharas and [their] females, in the heavenly world, underworld and open space are obedient to the will of the practicing person.''.
  \end{tlate}
    %%%%%%%%%%%%%%%%%%%
    %%%%%%%%%%%%%%%%%%%
    %%%%%%%%%%%%%%%%%%%
    %%%%%%%%%%%%%%%%%%%
  \ekddiv{type=trans}
      \bigskip
    \centerline{\textrm{\small{[Description of the fifth Cakra]}}}
    \bigskip
    \begin{tlate}
Now the fifth lotus having sixteen petals existing at the location of the throat. \extra{The colour is grey. The deity is the embodied soul (\textit{jīva}). The power is ignorance (\textit{avidyā}). The Ṛṣi is Virāṭ\footnote{Who is this?}. The mount is the wind (\textit{vāyu}). The vitalwind is \textit{udāna}. The digit (\textit{kalā}) is the flame. The binding (\textit{bandha}) is Jālandhara. The body is the primordial cause (\textit{mahākāraṇa}). The state is the fourth state (\textit{tūrya}). The speech is Parā\footnote{Im Kaśm. Śiv. °das ewige Wort, in welchem potentiell alle Begriffe und Worte ruhen; vgl. das śabdabrahma des Vyākaraṇa. [B.]― Schmidt S. 246}. [The Veda is the] Atharvaṇa Veda. The movable characteristic (\textit{jaṅgamaṃ liṅgaṃ}). The earth is Jīvaprāptā\footnote{What is this?}. The liberation is union with the deity (\textit{sāyujyatā}). [There are] sixteen petals [and] sixteen matrices. The internal matrix: aṃ āṃ iṃ īṃ u ūṃ ṛṃ ṝṃ ḷṃ ḹṃ eṃ aiṃ oṃ auṃ aṃ aṃḥ. The external matrix: Vidyā ``she who is knowledge'', Avidyā ``she who is ignorance'', Icchā ``she who is desire'', Śakti ``she who is power'', Jñānaśakti ``she who is the power of knowledge'', Śatalā ``she who is manifold'', Mahāvidyā ``she who is great knowledge'', Mahāmayā ``she who is great illusion'', Buddhi ``she who is intellect'', Tāmasī ``she who is darkness'', Maitrā ``she who is love'', Kumārī ``she who is a young girl'', Maitrāyaṇī ``she who is onb the path of benevolence'', Rudrā ``she who is howling'', Puṣṭā ``she who is abundance'', Siṃhanī ``she who is a lioness''. A thousandfold recitation of the non-recited; 1000 [repetitions for]; 2 \textit{ghaṭi}s, 46 \textit{palā}s. and 40 \textit{akṣara}s.\footnote{It is not entirely clear what kind of measure an \textit{akṣara} is. Maybe see Amanaska 1. Chapter second half in thesis of Jason to clear things up.}} In its middle exists a single person which shines like a thousand moons. Because of the exercise of meditation on this person all diseases which are (otherwise) not possible to be controlled vanish. The person lives up to 1001 years.
    \end{tlate}
   %%%%%%%%%%%%%%%%%%%
    %%%%%%%%%%%%%%%%%%%
    %%%%%%%%%%%%%%%%%%%
    %%%%%%%%%%%%%%%%%%%
\ekddiv{type=trans}
    \bigskip
    \centerline{\textrm{\small{[Description of the sixth Cakra]}}}
    \bigskip
    \begin{tlate}
      Now it exists a sixth \textit{cakra} named Ājñā. \extra{The deity is fire (\textit{agni}). The power is the godess of the centre (\textit{suṣumṇā}). The Ṛṣi is ``the violent'' (\textit{hiṃsa}). The mount is consciousness (\textit{caitanya}). The body is knowledge. The state is understanding. The speech is the ``incomparable'' (\textit{anupama}). The [Veda] is Sāmaveda. The \textit{liṅgaṃ} is intoxication (\textit{pramāda}). The half-matrix: the principle of ether. The gander is the living soul. The origin is the play of conciousness. Twofold matrix: haṃ kṣam is the inner matrix. The external matrix: Sthiti ``she who maintains'' [and] Prabhā ``she who is splendour''. A thousandfold recitation of the non-recited; 1000 [repetitions for]; 2 \textit{ghaṭi}s, 46 \textit{palā}s, and 40 \textit{akṣara}s.\footnote{It's not entirely clear what kind of measure is an \textit{akṣara}.}} This \textit{cakra} is located in the middle of the eyebrows and is two-petalled. In its middle exists a certain object being a form of blazing fire without parts, not being female not being male. Because of the exercise of meditation on it the body of the person becomes non-aging and immortal.
    \end{tlate}
   %%%%%%%%%%%%%%%%%%%
    %%%%%%%%%%%%%%%%%%%
    %%%%%%%%%%%%%%%%%%%
    %%%%%%%%%%%%%%%%%%%
 \ekddiv{type=trans}
    \bigskip
    \centerline{\textrm{\small{[Description of the seventh Cakra]}}}
    \bigskip
    \begin{tlate}
Now the seventh cakra having 64 petals and being full of nectar exists in the middle of the palate. \extra{The forehead is the Maṇḍala. The moon is the deity. The power is the nectar of immortality. The Rṣi is the supreme self. The seventeenth digit is the resident with the nectar of immortality. The wavy stream of nectar is great space. The uvula is the mother. The ornament/rhythm? (\textit{tālikā}) is a small bell. The own form of the body is the unspeakable Gāyatrī, [which has] the face of a crow, the eye of a human, the horn of a cow, a forehead that is Brahmapaṭhā?, a neck like a horse, the face of a peacock [and] limbs like a goose. [This is] the specific nature of the unspeakable Gayatrī.}    
  It is endowed with superabundant beauty. [It is] very bright. In its middle, red in color [is that which is] known as "uvula" (\textit{ghāṃṭikā}). [It] exists as a single pericarp. In its middle is a [certain] site. In the middle of it exists a hidden digit of the moon, being a stream of nectar like a river (\textit{amṛtādhārāsravantī}). Because of the exercise of meditation on this digit death does not come near him. Due to uninterrupted meditation, the stream (\textit{dhārā}) of nectar flows. Then the appearances of emaciation (\textit{kṣayaroga}), fever due to disordered bile (\textit{pittajvara}), heartburn (\textit{hṛdayadāha}), head-disease (\textit{śiroroga}) and tongue insensibility (\textit{jihvājaḍa}) vanish. Also eaten venom doesn't trouble him. If the mind is here, [it] becomes stable.     
  \end{tlate}
  %%%%%%%%%%%%%%%%%%%
    %%%%%%%%%%%%%%%%%%%
    %%%%%%%%%%%%%%%%%%%
    %%%%%%%%%%%%%%%%%%%
 \ekddiv{type=trans}
    \bigskip
    \centerline{\textrm{\small{[Description of the eighth Cakra]}}}
    \bigskip
          \begin{tlate}
Now exists the eigth \textit{cakra} having one hundred petals located at the aperture of Brahman. \extra{The teacher is the deity. Consciousness is the power. Virāṭ is the Ṛṣi, the witness above everything. Made of consciousness is that which is associated with (\textit{bhūta°}) the state beyond the fourth state. It has all colours. It has all matrices. It has all petals. The body is Virāṭ. The state is the standing still. The speech is wisdom.  The "I am that"-[expression] (\textit{sohaṃ}) is the Veda. The place is unsurpassed. A thousandfold recitation of the non-recited; 1000 [repetitions for]; 2 \textit{ghaṭi}s, 46 \textit{palā}s. and 40 \textit{akṣara}s.\footnote{It's not entirely clear what kind of measure is an \textit{akṣara}.} The count is all silent mutterings, [being] 21600. In this way it carries on day and night. He who knows the breath is a learned person. With the sound "sa" he exhales, with the sound "ha" he inhales again: "I'm he, he's I". Because of that the embodied soul constantly utters the Mantra.\footnote{Add intertextual evidence.}} ``The (divine) seat of  Jālaṃdhara'' is the designation of its lotus.\footnote{Find parallels where Jālandhara is situated on top of the head.} [It is] the place of the accomplished person. In its middle looking like a streak [and] having the form of smoke and fire, exists such a single [divine] form of the person (\textit{puruṣa}). Of her exists no end, nor a beginning. Due to the exercise of meditation on this [divine] form both coming and going of the person in space occurs. Affliction from the earth-element doesn't arise [anymore] even if one is situated in the middle of the earth. He constantly sees everything in front of his eyes and he becomes separated [from the material world]. The force of life increases eminently.    
     \end{tlate}
    %%%%%%%%%%%%%%%%%%%
    %%%%%%%%%%%%%%%%%%%
    %%%%%%%%%%%%%%%%%%%
    %%%%%%%%%%%%%%%%%%%
\ekddiv{type=trans}
       \bigskip
    \centerline{\textrm{\small{[Description of the ninth Cakra]}}}
    \bigskip
    \begin{tlate}
Now the divisions/differentiations of the ninth cakra are explained. The designation of it is ``the \textit{cakra} of the great void''. Above that there is no other. Therefore it is declared to be the \textit{cakra} of the great perfection. [Another] such name of it is ``(divine) seat of Pūrṇagiri''. In the middle of the \textit{mahāśūnyacakra} exists one lotus facing upward, very red in colour, with a thousand petals - an abode of brilliance and wholeness, whose fragrance is not in range of mind and speech. In the middle of this lotus exists one pericarp having the shape of a triangle. In the middle of the pericarp exists one seventeenth digit in the shape of a immaculé form. A light of the part exists shining like a thousand suns. [But] excessive heat is not arising. Shining like a thousand moons, excess of cold is not arising. \extra{Here at this location the ``I''(\textit{aham}) is the deity. The ``he is I'' (\textit{so 'ham}) is the power. This self is the Ṛṣi. The path is liberation. Brahma is the I above. ``I'm a circle''. In fire-area is the letter "sa". [There?] life arises, the living soul ascends and decends. The place is the hidden place of being. The colour is yellow. The light is the shine of ten million suns. The shine is always and visible. Śiva is the deity. The power is primordial illusion. The state is the dissolution of the self into Hara\footnote{Epiphet of Śiva.}. The transcendental sound has the nature of a sound with stable resonance. The seal is the ``fearless''. The illusion is the root. The body is the original matter. The range is speech and mind. Without delusion, without doubt, the unaffected and undefiled goal is dissolution, meditation [and] final absorption.} Above that is the place of infinite supreme bliss. There above is power (\textit{śakti}). Being designated as such she is one single digit. Due to the exercise of meditation on this part, the person manifests whatever he wishes for. He is furnished with royal pleasure and enjoyment. [Even] amusing oneself amongst women, and watching musical pleasures, the \textit{kāla} of the person grows daily like the \textit{kalā} of the moon in the bright half of the month. His body is not affected by merit and sin. Due to uninterrupted meditation the power of the light of the innate nature arises. He sees remotely located objects as if they'd be near.
\end{tlate}
    %%%%%%%%%%%%%%%%%%%
    %%%%%%%%%%%%%%%%%%%
    %%%%%%%%%%%%%%%%%%%
    %%%%%%%%%%%%%%%%%%%
  \ekddiv{type=trans}
     \bigskip
    \centerline{\textrm{\small{[Lakṣyayoga, the yoga of fixation]}}}
    \bigskip
 \begin{tlate}
   Now the yoga of fixation (\textit{lakṣyayoga}), which is easily accomplished is explained. Of this yoga of fixation there are five subdivisions:
   1. The upward directed fixation (\textit{ūrdhvalakṣya}),
   2. the downward directed fixation (\textit{adholakṣya}),
   3. the outer fixation (\textit{baḥyalakṣya}),
   4. the central fixation (\textit{madhyalakṣya}),
   5. the inner fixation (\textit{antaralakṣya}).
 \end{tlate}
   %%%%%%%%%%%%%%%%%%%
    %%%%%%%%%%%%%%%%%%%
    %%%%%%%%%%%%%%%%%%%
    %%%%%%%%%%%%%%%%%%%
  \ekddiv{type=trans}
     \bigskip
    \centerline{\textrm{\small{[1. Ūrdhvalakṣya - The upward directed fixation]}}}
    \bigskip    
  \begin{tlate}
At first the upward directed fixation (\textit{ūrdhvalakṣya}) is explained. The gaze (\textit{dṛṣṭi}) [should be] in the middle of the sky. And then having caused the mind to be directed upwards, it is caused to be fixed there. Due to the exercise of stabilizing of this fixation (\textit{lakṣya}) arises unity of the gazing point (\textit{dṛṣṭi}) with the light of the highest lord (\textit{parameśvara}). And then an indefinable invisible object arises in the middle of the sky. It arises in the range of sight of the practitioner. This is truly the upward directed fixation (\textit{ūrdhvalakṣya}).
  \end{tlate}
   %%%%%%%%%%%%%%%%%%%
    %%%%%%%%%%%%%%%%%%%
    %%%%%%%%%%%%%%%%%%%
    %%%%%%%%%%%%%%%%%%%
\ekddiv{type=trans}
   \bigskip
    \centerline{\textrm{\small{[2. Adholakṣya - The downward directed fixation]}}}
    \bigskip
  \begin{tlate}
    Now the downward directed fixation object (\textit{adholakṣya}). One should stabilize the gaze within the circumference (\textit{paryanta}) of twelve \textit{aṅgula}s beyond the nose. Or one should stabilize the gaze onto the tip of the nose. The fixation becomes stable due to firm exercise [on one] of the twofold aims [of fixation]. The breath becomes stable. Vitality increases.
\end{tlate}
 %%%%%%%%%%%%%%%%%%%
    %%%%%%%%%%%%%%%%%%%
    %%%%%%%%%%%%%%%%%%%
    %%%%%%%%%%%%%%%%%%%
\ekddiv{type=trans}
   \bigskip
    \centerline{\textrm{\small{[3. Bāhyalakṣya - The external fixation]}}}
    \bigskip
  \begin{tlate}
    Just as this [aim] is twofold, also the external fixation is said to be [like this]. Internally or externally the aim of fixation is to be done onto the heavenly emptiness. The fear of dying doesn't arise due to the exercise of meditation on the void at all places during ones life - while eating, moving and waking.\footnote{Note that the description of the five types of Lakṣyayoga stops here and the new topic about the body of the Rājayogin is introduced. However, the subject is resumed later on in the text. Even though all witnesses follow this specific and suprising order. Maybe a copist in the early stages of transmission of the text copied the text without noticing the folios of his template to be in the wrong order.}
  \end{tlate}
   %%%%%%%%%%%%%%%%%%%
    %%%%%%%%%%%%%%%%%%%
    %%%%%%%%%%%%%%%%%%%
    %%%%%%%%%%%%%%%%%%%
  \ekddiv{type=trans}
    \bigskip
    \centerline{\textrm{\small{[Description of the Rājayogin's Body]}}}
    \bigskip
      \begin{tlate}
Now it is said that this is the characteristic of the embodied person who is endowed with the royal yoga: Abundance arises at all times. No distance exists on earth. He dwells on earth having pervaded [it]. Birth and death both don't exist. Happiness does'nt exist. Suffering does'nt exist. Impediment does'nt exist. Habit doesn't exist. Place does'nt exist. The manifestation of permanent perception of the connection with god arises in the middle of the mind of this accomplished one. And he is shining - not cold, and not hot, not white [and] not yellow. Neither is there birth of him, nor does he have any attributes. And he is without parts, immacule and uncharacterized. His desire etc. doesn't arise in [situations of] lust [and] is not located within the duality of the result. He attains expanded enjoyment. However, his mind does not suffer attachment in this very state.     
    \end{tlate}
    %%%%%%%%%%%%%%%%%%%%%
    %%%%%%%%%%%%%%%%%%%
    %%%%%%%%%%%%%%%%%%%
    %%%%%%%%%%%%%%%%%%%
    %%%%%%%%%%%%%%%%%%%
 \ekddiv{type=trans}
    \bigskip
    \centerline{\textrm{\small{[Other Attributes]}}}
    \bigskip
  \begin{tlate}
    Another attribute of Rājayoga is described. Even ``of one who is in gain of a kingdom etc.'' [it is said that] perception of success does'nt arise. Even due to loss suffering does'nt arise in the mind. And then desire doesn't arise. And then with regards to an object that has been obtained for whatever reason towards ones object aversion does'nt arise. With regard to this object affection of the mind does'nt arise. Just this is said to be Rājayoga. And then his mind which knows the sacred speech is equal towards a person, friend and enemy. And a neutral view arises. In the mind of one who is entirely situated in the middle of the earth, the pride of authorship does't arise, because of death and rebirth, and because of happiness and enjoyment. Wile wandering the world he doesn't whish to know authorship. This is also said to be Rājayoga. New durable clothes made of silk, or however, old, worn [clothes] with holes smeared with sandalwood and musk, or smeared with mud. In whose mind joy and sorrow are not situated, just he is [in the state of] Rājayoga. Just he is in the state of Rājayoga for whom the mind is neither in abundance nor in lack, being located in a city, a forest, an uninhabited village or a village full of people.    
  \end{tlate}
    %%%%%%%%%%%%%%%%%%%%%
    %%%%%%%%%%%%%%%%%%%
    %%%%%%%%%%%%%%%%%%%
    %%%%%%%%%%%%%%%%%%%
    %%%%%%%%%%%%%%%%%%%
\ekddiv{type=trans}
      \bigskip
    \centerline{\textrm{\small{[Description of Caryāyoga]}}}
      \bigskip
     \begin{tlate}
        Now \textit{caryāyogaḥ}, the Yoga of wandering is explained. Shapeless, unchangeable, permanent [and] unsplitable. Such is the self. It is seen as such by the one whose mind abides in the self without moving. His self is not touched by sin and merit. Just as the leave of the lotus situated in the amidst water doesn't touch the water; likewise the self [is not touched by sin and merit]. Just as the wind wanders according to its own will in space, likewise the mind of one who is absorbed into the universal spirit [wanders according to its own will in space]. This is \textit{caryāyoga}.
      \end{tlate}
    %%%%%%%%%%%%%%%%%%%%%
    %%%%%%%%%%%%%%%%%%%
    %%%%%%%%%%%%%%%%%%%
    %%%%%%%%%%%%%%%%%%%
    %%%%%%%%%%%%%%%%%%%
  \ekddiv{type=trans}
       \bigskip
    \centerline{\textrm{\small{[Description of Haṭhayoga]}}}
      \bigskip
      \begin{tlate}
        Now \textit{haṭhayoga}, the forceful Yoga is explained. The practice of breath shall be done in this manner: "Exhalation, Inhalation [and] Retention etc. And then due to the six practices (\textit{ṣaṭkarma}), like \textit{dhauti} etc. the purification of the body arises. When the full breath abides in the middle of the sun-channel. Then the mind is unmovable. The form of bliss immediately shines through the motionless mind. Due to the execution of Haṭhayoga the mind becomes absorbed into emptiness. The time of death does not approach. Now, the second division of Haṭhayoga is explained. The shine of ten million suns in one's own body beginning from the feet to the top of head is contemplated in any color equal to white, yellow [or] red. Due to the execution of meditation in the entire body disease does'nt arise, fever doesn't arise and vitality grows.
      \end{tlate}
  %%%%%%%%%%%%%%%%%%%%%
    %%%%%%%%%%%%%%%%%%%
    %%%%%%%%%%%%%%%%%%%
    %%%%%%%%%%%%%%%%%%%
    %%%%%%%%%%%%%%%%%%%
  \ekddiv{type=trans}
    \bigskip
        \centerline{\textrm{\small{[Description of \textit{Jñānayoga}]}}}
          \bigskip
    \begin{tlate}
      Now the characteristic of \textit{jñānayoga} is explained.      
  \paragraph{1.} He shall see the world truly as being one, shining in all selves. By applying indistinctness he shall accomplish \textit{Jñānayoga}.
  \paragraph{2.} Wherever the world is established or made of omniscience, who knows thus by means of insight, he is a like an expert of knowledge.
  \paragraph{3.} He always attains the reality of \textit{śāmbhavī} - the goal of eternal non-duality. Just as the seed of the Nyagrodha\footnote{In rituals, the nyagrodha (Ficus indica or India fig or banyan tree) danda, or staff, is assigned to the kshatriya class, along with a mantra, intended to impart physical vitality or 'ojas'.27. Brian K. Smith. Reflections on Resemblance, Ritual, and Religion, Motilal Banarsidass Publishe, 1998} scattered onto the soil [always] becomes a tree.
  \paragraph{4.} The absolute unity (\textit{ekāntaṃ}), is seen as multibel (namely) made up of ten parts by oneself. The rolled up shoots of the branches are the sprouting stalks of the root shoot.
  \paragraph{5.} By virtue of its inherent nature, this branch with its branches, which is the fruit of the flower of love, is in the seed. Certainly, that is pure, eternal, unchanging and immaculate. 
  \paragraph{6.} One, not one and self-existing, existing in manifold ways through its own rule and work, [as] five principles (\textit{tattva}) which are: thinking mind (\textit{manas}), intellect (\textit{buddhi}), illusion (\textit{māya}), individuation (\textit{ahaṃkāra}) and modifications (\textit{vikriyā}).
  \paragraph{7.}In this way, the ten variations fully permeate the world and the non-world. Only one thing is and not something else: Whoever knows this is a connoisseur of reality.
  
   Transmigration is the appearance of the plant world, mountains, trees, earth etc. Transmigration is the appearance of living beings beginning with birds, horses, elephants and humans. And then whoever is one who is a [sense] object of sight is said to be visible. He who is not seen by sight is said to be invisible. In this way the view of separation of one's own self which is subjected to transmigration is to be removed by means of [applying the view of] unity. Only this is Jñānayoga. Because of the execution of it, time does'nt destroy the body.
\end{tlate}
  %%%%%%%%%%%%%%%%%%%%%
    %%%%%%%%%%%%%%%%%%%
    %%%%%%%%%%%%%%%%%%%
    %%%%%%%%%%%%%%%%%%%
    %%%%%%%%%%%%%%%%%%%
\ekddiv{type=trans}
      \bigskip
        \centerline{\textrm{\small{[The Division of the Inherent Nature]}}}
          \bigskip
\begin{tlate}    
  Now the division of the inherent nature is described.\footnote{This refers to the mention of \textit{svabhāva} in verse 5 of the description of Jñānayoga.} Just as the seed of the banyan tree ripens into the shape of the banyan tree, and by its own inherent nature attains such a tenfold division. [Namely]: "Root, shoot, bark, branch, twig, bud, the unfolding flower, flower, fruit and nectar." The division reaches [those] ten parts. In this way, the pure, unchanging, unblemished, attains such [division] precisely because of the inherent nature of the self. [Namely] the division "Earth, Water, Fire, Wind, Space, Mind, Intellect, Illusion, Transformations and Form". Because of the power of Jñānayoga, there arises the certainty that "The Self is verily one." As some particular soil (\textit{ekaika}) sometimes appears soft, sometimes beautiful, sometimes fragrant, sometimes unscented, sometimes golden, sometimes silver, is sometimes made of precious stone, sometimes appearing white, sometimes black, sometimes copper, sometimes yellow, sometimes mottled, sometimes like various fruit, sometimes like flowers, sometimes like the nectar of immortality, [and that only] because of its inherent nature. In this way, the self also takes the form of a human, a bird, a gazelle, an elephant, a vidyādhara, a gandharva, a centaur, great scholar or a great fool, a sick or healthy, an angry or or peaceful person, by virtue of its inherent nature. Because of Jñānayoga, transformation is recognized as formless, Just as the place of origin of the fruit is only one. But the transformation of the fruit is seen as manifold.
  
  One fruit falls onto the ground. It is getting bright. A bee drinks the flower juice of a fruit. The lover [bee] places itself on the flower wreath above the protuberant circular pistil. A bee drinks the juice of a fruit. The lover (bee) places herself on the flower wreath above the upstanding circular pistil. ne fruit throws the nectar over the flower. This is the inherent nature of the matter. In the same way also the one self enjoys the eight pleasures because of its own being.  
\\ \\
What are the eight enjoyments?  \hfill \break
A beautiful dwelling, good clothing, a good bed, a well-trained horse?, a nice place, food and drink.\footnote{The verse only gives 7 enjoyments!} Those are the eight enjoyments of the wise.
\\ \\
  1. Clothes made from silk; \hfill \break
  2. A site of the palace in which there are mainsions endowned with five or seven rooms.\hfill \break
  3. A huge, very soft and lovely bed; \hfill \break
  4. [on which] there is seated a lotus-like youthful, charming and virtuous wife;\hfill \break 
  5. An excellent throne;\hfill \break
  6. An exceptional valuable horse; \hfill \break
  7. Food that pleases the senses; \hfill \break
  8. Various drinks. \hfill \break 

Like the rays of the sun, the butter of milk, the burning of fire, the stupor of poison, the sesame oil from the sesame seed, the shade from the tree, the sweet odor from a fruit, the fire from a scabbard, the sweet sap of Śārkara\footnote{A liquor prepared from Dhātakī with sugar.} and so on, the cold of piles of snow, and so on is the inherent essence of things. In the same way, the course of the world is also in the center of the highest God's own form. And the Most High God is indivisible and all-filling.
\end{tlate}
  %%%%%%%%%%%%%%%%%%%%%
    %%%%%%%%%%%%%%%%%%%
    %%%%%%%%%%%%%%%%%%%
    %%%%%%%%%%%%%%%%%%%
    %%%%%%%%%%%%%%%%%%%
\begin{tlate}
  \ekddiv{type=trans}
   \bigskip
        \centerline{\textrm{\small{[Continuation of \textit{Lakṣyayoga} - Bāhyalakṣya]}}}
          \bigskip
Now the external fixation is taught. Beginning with a four finger wide distance from the tip of the nose, the space[-element] full of light whose appearance is blue shall be made the object of fixation. Or, a six finger wide distance from the tip of the nose, the wind-element whose appearance is greyish shall be made the object of fixation. Or, an eight finger wide distance from the tip of the nose, the very red fire[-element] shall be made the object of fixation. Or, a ten finger wide distance from the tip of the nose, the white water[-element] being fickle shall be made the object of fixation. Or, a twelve finger wide distance from the tip of the nose, the yellow-colored earth-element shall be made the object of fixation. Or beginning at the tip of the nose\footnote{Given the clear instructions of the respective distance of the exercise in the previous sentences, it is surprising that this instruction is lacking the mention of the distance.} the space-element full of fire shining like ten million suns shall be made the object of fixation. After having fixed the gaze on the space[-element?] or above the space[-element?], due to the execution of meditation he sees the sun without the group of thousand rays related to the sun. Or the mass of light situated seventeen fingers wide distance above the head shall be made the fixation object. Diseases of the limbs are removed without medical herbs. All enemies become friends while sleeping. The lifespan increases up to 1000 years. 
\end{tlate}
  %%%%%%%%%%%%%%%%%%%%%
    %%%%%%%%%%%%%%%%%%%
    %%%%%%%%%%%%%%%%%%%
    %%%%%%%%%%%%%%%%%%%
    %%%%%%%%%%%%%%%%%%%
\begin{tlate}
  \ekddiv{type=trans}
     \bigskip
        \centerline{\textrm{\small{[Continuation of \textit{Lakṣyayoga} - Antaralakṣya]}}}
          \bigskip
Now the inner fixation objects are taught. At the location of the root bulp rising from the staff of Brahma up to the aperture of Brahma exists the one white coloured Brahma channel. The interior of the Brahma channel, which equals a pale-red string shining like 10 million suns, goes upwards. A particular manifestation exists as such. Due to the execution of meditation on this manifestation, the eight great supernatural powers of humans beginning with \textit{aṇima} etc.\footnote{Write something about siddhis.} become established after one has entered into [the manufestation's] imminence. Or from the execution of meditation onto the bright light at the centre within the space at the forehead diseases related to the body beginning with leprosy vanish. Lifeforce increases. Or because of executing meditation on the middle of the eyebrows [of which there is] a very subtle and red colored light, he is one who is beloved among all royal people. Having seen this person, everybody's gaze is fixed onto him.
\end{tlate}
  %%%%%%%%%%%%%%%%%%%%%
    %%%%%%%%%%%%%%%%%%%
    %%%%%%%%%%%%%%%%%%%
    %%%%%%%%%%%%%%%%%%%
    %%%%%%%%%%%%%%%%%%%
\begin{tlate}
  \ekddiv{type=trans}
 \bigskip
 \centerline{\textrm{\small{[The Ten Main Bodily Channels]}}}
 \bigskip
 Now the divisions of channels within the body are explained. There are ten primary channels. Among them exists the pair of channels designated Idā and Piṅgalā at the entrance of the nose. The central channel leads from the palate to the door of Brahma. The Sarasvatī[-channel] exists at the centre of the face. The two rivers Gāṃdhārī and Hastjihvā exist within the centre of the two ears. The two rivers Pūṣā and Ālaṃbuṣā are situated at the center of the two eyes. The Śaṃkhinī channel strechtes from the the beginning of the opening of the penis through the Iḍā-channel. In such a way the channels are situated at the 10 openings. The other channels measured as 72000 are situated with a subtle form at the roots of the hairs.
\end{tlate}
  %%%%%%%%%%%%%%%%%%%%%
    %%%%%%%%%%%%%%%%%%%
    %%%%%%%%%%%%%%%%%%%
    %%%%%%%%%%%%%%%%%%%
    %%%%%%%%%%%%%%%%%%%
 \begin{tlate}
  \ekddiv{type=trans}
\bigskip
 \centerline{\textrm{\small{[The Ten Vitalwinds]}}}
 \bigskip
 Now [there are] ten vitalwinds are situated within the body. The Prāṇa vitalwind is located in the middle of the heart and causes inhalation and exhalation. The wish for eating an drinking exists. At the center of the anus the Apāna-Vitalwind exists. He does contraction and checking. At the center of the navel the Samāna[-vitalwind] exists. He causes to dry up all the channels. In this way the channels are caused to thrive, beauty is caused to be generated and the fire is caused to light up. Within the throat the Udāna-vitalwind is situated. This wind swallows food, [and] it drinks water. The Nāga-vitalwind exists in the entire body. Through the vitalwind the body is caused to move. The Kūrma-vitalwind exists within the eyes. It causes [the] opening and closing [of the eyes]. From the Kṛkala-vitalwind gagging arises. From the Devadatta-vitalwind jawning arises. From the Dhanaṃjaya-vitalwind speech arises.
\end{tlate}
  %%%%%%%%%%%%%%%%%%%%%
    %%%%%%%%%%%%%%%%%%%
    %%%%%%%%%%%%%%%%%%%
    %%%%%%%%%%%%%%%%%%%
    %%%%%%%%%%%%%%%%%%%
\begin{tlate}
  \ekddiv{type=trans}
\bigskip
 \centerline{\textrm{\small{[Continuation of \textit{Lakṣyayoga} - Madhyalakṣya]}}}
 \bigskip
Now the central fixation is taught. White-colored, or yellow-colored or red-coloured or smoke-coloured or blue-coloured, like the flame of fire, equal to a lightning, like the orb of the sun, like a half-moon, appearing like flaming space, measured according to ones own body, the fixation shall be directed onto the center of the glowing mind. While abiding in this fixation the burning of the impurity in the center of the mind arises. The Sattva-quality of the mind becomes revealed. After this has happend, the person abides supreme bliss.  
\end{tlate}
  %%%%%%%%%%%%%%%%%%%%%
    %%%%%%%%%%%%%%%%%%%
    %%%%%%%%%%%%%%%%%%%
    %%%%%%%%%%%%%%%%%%%
    %%%%%%%%%%%%%%%%%%%
\begin{tlate}
  \ekddiv{type=trans}
 \bigskip
 \centerline{\textrm{\small{[The Divisions of Space]}}}
 \bigskip
 Now the divisions of space are taught.The fixations of them are taught: Space, beyond space, great space, space of reality, the space of the sun. The fixation onto the pure and formless space \textit{akāśa} shall be done internally as well as externally. Moreover, the fixation of the beyond-space \textit{parākāśa} which is equal to dense darkness shall be done internally and externally. Moreover, the fixation of the great space (\textit{mahākāśa}) which is the plethora of the burning fire of the time of dissolution shall be done internally and externally. Moreover, for whom internally and externally the brightness of millions of blazing lights arises, he shall execute the fixation [directed onto] the reality-space (\textit{tattvakāśa}). After that the fixation of the sun-space (\textit{sūryakāśa}) which is associated with sundisk's appearance of light shall be done internally and externally. From the execution of these fixations contact of diseases does not arise within the body. Thus wrinkles and grey hair, sin or merit do not arise. The nine cakras, the sixteen Adhāras, the three lakṣyas and die five spaces. Who does not know [them?] within ones own body, he is only a Yogin by name.
\end{tlate}
  %%%%%%%%%%%%%%%%%%%%%
    %%%%%%%%%%%%%%%%%%%
    %%%%%%%%%%%%%%%%%%%
    %%%%%%%%%%%%%%%%%%%
    %%%%%%%%%%%%%%%%%%%
\begin{tlate}
  \ekddiv{type=trans}
 \bigskip
 \centerline{\textrm{\small{[The order of Cakras]}}}
 \bigskip
Now the practice of the cakras is explained. At the pelvic floor there is the Brahmacakra. Above the pelvic floor at the root of the gender is the Svadiṣṭhānacakra. At the navel there is the Maṇipūrakacakra. At the heart the Anāhatacakra. Situated within the throat is the Viśuddhicakra. The sixth is the cakra of the palate. In the center of the eyebrows is the Ājñācakra. At the opening of Brahma is the Kālacakra. The ninth is the Ākāśacakra. It is supreme emptiness.
\end{tlate}
  %%%%%%%%%%%%%%%%%%%%%
    %%%%%%%%%%%%%%%%%%%
    %%%%%%%%%%%%%%%%%%%
    %%%%%%%%%%%%%%%%%%%
    %%%%%%%%%%%%%%%%%%%
\begin{tlate}
  \ekddiv{type=trans}
 \bigskip
 \centerline{\textrm{\small{[The sixteen Container]}}}
 \bigskip
 Now the  divisions of the container-\textit{cakra}s are taught.
 From the execution of the fixation onto the light at the big toes of the feet stability of the gaze arises.
 The root-container is the second [one]. The heel of the backfoot is caused to be placed at the root of the big toe. As a result the fire is strengthened.
 One heel is caused to be placed at the Root-container. The heel of the other foot is caused to be placed at the root of the big toe of this foot.
 The fire of it is caused to be kindled.
 The third is the place of the anus-container. From the execution of expansion and contraction a stable vitalwind arises. Additionally death of the person does not arise. Additionally the person does not die.
 The fourth is the penis-container. Due to the execution of repeated practice of contracting the penis in the midst of therof, the adamantine channel appears in the middle of the staff of the back. From the repeated practice again [and again] the transition of both breath and mind into its center arises. Caused by the transition of them both into the center the trinity of knots breaks. There, from the breaking of that, the vitalwind after having filled up (the central channel?) resides in the center of the Brahma-lotus. Then virility and strength arises. The person becomes youthful forever.
 The fifth is Udyāna. From performing \textit{bandha} there, urine and faeces disappear.
 The sixth is the navel-container. From repeated practice of \textit{praṇava}, the unstruck sound arises by itself.
 The seventh is the container of the heart-form.
 The throat-support is the eighth. There the contraction of Jālaṃdhara is produced. While abiding therein the vitalwind in the Iḍā-and Piṅgalā-channels becomes stable.
 The ninth is the container of the uvula. There, the tip of the tongue becomes attached [to the uvula]. Then the nectar of immortality flows from the immortality-digit. From drinking the nectar of immortality diseases do not spread in the body.
 The tenth is the container of the palate. After the moving and milking has been done therein while abiding at the door of the uvula, the tongue resides inserted within the palate.
 The eleventh is the tongue-container at the surface of the tongue. Within it the tip of the tongue has to be churned. While doing [that] a sweet drink flows out. And in that manner the knowledge of areas like poetry, singing, metric and dance is generated.
 On top thereof is the twelfth, being the teeth-support, which is situated inbetween the teeth. At this place the tip of the tongue is to be positioned with force for the duration of one and a half \textit{ghāṭī}s (24+12 = 36 minutes). Abiding therein the diseases of the practitioner will entirely disappear!
 The thirteenth is the nose-container. While making it into the fixation object the mind becomes stable.
 The fourteenth is the container of breath at the root of the nose. From the execution of stabilizing of the gaze onto this, the light of one's own becomes perceptible within 60 months.
 He breaks the mundane prison through the perceptibility of the light.
 The fifteenth container is situated in the middle of the eyebrows. Due to stabilizing the gaze therein 10 million rays of light sparkle.
 The sixteenth is the eye-container. Without wavering, the gaze [ayam] is to be held at the tip of the finger without wavering. From practicing this on earth any energy exists [for him]. The light of everything that is arises as the object of sight. From that sight the person becomes omniscient. 
\end{tlate}
  %%%%%%%%%%%%%%%%%%%%%
    %%%%%%%%%%%%%%%%%%%
    %%%%%%%%%%%%%%%%%%%
    %%%%%%%%%%%%%%%%%%%
    %%%%%%%%%%%%%%%%%%%
\end{otherlanguage} 
%\section{Bibliography}
% \label{sec:bibli}
%\printshorthands[keyword=critEd]
%\printbibliography[title=Consulted Manuskcipts, keyword=codex]
%\printbibliography[title=Printed Editions, keyword=printsource]
%\printbibliography[title=Secondary Literature, keyword=seclit]
%\end{document}


