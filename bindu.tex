%Ultimatives Tool zur Datierung:
%https://www.cc.kyoto-su.ac.jp/~yanom/pancanga/

%skp = ignored in edition
%skm = ignored in xml

%%%---2-DO---%%%:
% - ENTER RDGs of N2 and make apparatus negative?!
% - ENTER RDGs of D2
% - add xml ids for cladistics
% - produce diplomatic transcripts for saktumiva
% - make extra layer in Apparatus for parallels in SVARODAYA, Siddhasiddhantapaddhati and Amanaska
% - check all daṇḍas!!! now I think that it's more likely that many of them were lost in copies. Lectio difficilior! Very unconventional style of the autor! 
% - read Sarvangayogapradipika, Maya Burger! 
% - maybe add second ciritical edition of yogasvarodaya?!
% - Korrekturlesen von \E!! 
% - Verspattern einbauen! 
% - präambel auslagern wie Jürgen
% - grep-search alle Verse!!!!
% - Mss spreadsheet
% - sort N1,D1,B2 zu N1,N2,D1
% - additions to U2: make footnotes for the bahir mātrā-s: explaining the inventions of female deities and tell that this is "schwer interpretierbar" 
%%%%%%%%%%%%%%%%%%%%%%%%%%%%%%%%%%%%%%%%%
% Don't forget
% Siddhasiddhantapaddhati Yogic Body descriptions are followed by Rāmacandra
% Quotes of the Yogasvarodaya in the Yoga Karṇikā
% Rāmacandra more a compiler than an author!!!
% Identify quotes of YTB in Haṭhasanketacandrikā 
%%%%%%%%%%%%%%%%%%%%%%%%%%%%%%%%%%%%%%%%%%%
%MSS notes
%
%--B: i and ī are not differenciated
%--P: no punctuation no daṇdas nothing
%--U1: dot . serves as daṇḍa 
%--\L and \U2 very similar
%--figure out for U2: // ajapājapaḥ sahasra // 6000 //gha 0 16 pa 0 40// \U2?!?!?!?!?!?
%%%%%%%%%%%%%%%%%%%%%%%%%%%%%%%%%%%%%%%%%%
%
%
%
%
%
%
%
%
%
%
%
%%%%%%%%%%%%%%%%%%%%%%%%%%%%%%%%%%%%%%%%%%%


\documentclass[10pt]{memoir}
\setstocksize{220mm}{155mm} 	        
\settrimmedsize{220mm}{155mm}{*}	
\settypeblocksize{170mm}{116mm}{*}	
\setlrmargins{18mm}{*}{*}
\setulmargins{*}{*}{1.2}
%\setlength{\headheight}{5pt}
\checkandfixthelayout[lines]
\linespread{1.16}

%%% more functions
\usepackage[dvipsnames]{xcolor}
%\usepackage[flushmargin]{footmisc}

%%% Hyphenation settings
\usepackage{hyphenat}
\hyphenation{he-lio-trope opos-sum}
\tracingparagraphs=1
%Hyphenation in Devanāgarī of the edition still missing? Probably this needs to be modified in babel-iast package? 

%%% babel
\usepackage[english]{babel}
\usepackage{babel-iast/babel-iast}
\babelfont[iast]{rm}[Renderer=Harfbuzz, Scale=1.3]{AdishilaSan}%AdishilaSan}
\babelfont[english]{rm}{Adobe Text Pro}


%%% more functionality
\PassOptionsToPackage{hyphens}{url}
\usepackage{hyperref}
\usepackage{cleveref}
\usepackage{url}
\usepackage{cleveref}
\usepackage{microtype}
\usepackage{lineno}
%\linenumbers
%\usepackage[type=lowerleftT]{fgruler}


%%%% test better pagebreaks
\def\fussy{%
  \emergencystretch\z@
  \tolerance 200%
  \hfuzz .1\p@
  \vfuzz\hfuzz}

\interfootnotelinepenalty=10000\relax

\usepackage[maxfloats=256]{morefloats}

\maxdeadcycles=500

\raggedbottomsectiontrue
\checkandfixthelayout

%%%
%\setlength{\parskip}{2cm plus0.5cm minus0.5cm}


%Index
\usepackage[backend=biber, bibstyle=verbose, citestyle=authoryear]{biblatex}

\DefineBibliographyStrings{german}{
  references = {Literaturverzeichnis},
  bibliography = {Bibliographie},
  shorthands = {Abkürzungen der Zeugen des kritischen Apparatus},
}

\makeatletter
\renewcommand*{\mkbibnamefamily}[1]{%
  \ifdefstring{\blx@delimcontext}{parencite}
    {\textsc{#1}}
    {#1}}
\makeatother

\DeclareFieldFormat{postnote}{#1}
\renewcommand{\postnotedelim}{:}
\addbibresource{bindu.bib}

%%% ekdosis
\usepackage[teiexport=tidy,parnotes=true]{ekdosis}% =tidy cleans up HTML and XML documents by fixing markup errors and upgrading legacy code to modern standards. parnotes=footnotes below or above critical apparatus

\SetLineation{lineation=page} %lineation=pagesets thenumbering to start afresh at the top of each page. modulo makes every fifth line numbered. 

\renewcommand{\linenumberfont}{\selectlanguage{english}\footnotesize} %sets language of lines to English

\SetTEIxmlExport{autopar=false} %autopar=falseinstructs ekdosis to ignore blank lines in the.tex sourcefile as markers for paragraph boundaries. As a result, each paragraph of the edition must be found within an environment associated with the xml <p> element

\SetHooks{
  lemmastyle=\bfseries,
  %refnumstyle=\selectlanguage{english}\bfseries,
  refnumstyle=\selectlanguage{english}\color{blue}\bfseries,
  appheight=0.8\textheight,
}

\newif\ifinapparatus
\DeclareApparatus{testium}[
%bhook=\inapparatustrue,
lang=english,
notelang=english,
% bhook=\selectlanguage{english},
bhook=\selectlanguage{english}\textbf{Testimonia:},
%maxentries=4, 
%ehook=.]
%sep={: },
]

\newif\ifinapparatus
\DeclareApparatus{source}[
%bhook=\inapparatustrue,
lang=english,
notelang=english,
% bhook=\selectlanguage{english},
bhook=\selectlanguage{english}\textbf{Sources:},%
%maxentries=4, 
%ehook=.]
%sep={: },
]

% Declare \ifinapparatus and set \inapparatustrue at the beginning of
% the apparatus criticus block. Also set the language.  
\newif\ifinapparatus
  \DeclareApparatus{default}[
  %bhook=\inapparatustrue, 
  lang=english,
  %maxentries=33,
  %bhook=\selectlanguage{english},
  sep = {] },
  delim=\hskip 0.75em,
  rule=\rule{0.7in}{0.4pt},
]

\newif\ifinapparatus
\DeclareApparatus{philcomm}[
%bhook=\inapparatustrue,
lang=english,
notelang=english,
bhook=\selectlanguage{english}\textbf{Philological Commentary:},
%bhook=\selectlanguage{english},
sep={: },
]

\ekdsetup{
showpagebreaks,
spbmk = \textcolor{blue}{spb},
hpbmk = \textcolor{red}{hpb}
}

% Macros and Definitions for the Print of Sigla
\def\acpc#1#2#3{{#1}\rlap{\textrm{\textsuperscript{#3}}}\textsubscript{\textrm{#2}}\space}
\def\sigl#1#2{{{#1}}\textsubscript{\textrm{#2}}}
\def\None{{\sigl{N}{1}}} \def\Noneac{\acpc{N}{1}{ac}\,} \def\Nonepc{\acpc{N}{1}{pc}\,}
\def\Ntwo{{\sigl{N}{2}}} \def\Noneac{\acpc{N}{2}{ac}\,} \def\Nonepc{\acpc{N}{2}{pc}\,}
\def\Done{{\sigl{D}{1}}} \def\Doneac{\acpc{D}{1}{ac}\,} \def\Donepc{\acpc{D}{1}{pc}\,}
\def\Dtwo{{\sigl{D}{2}}} \def\Dtwoac{\acpc{D}{2}{ac}\,} \def\Dtwopc{\acpc{D}{2}{pc}\,}
\def\Uone{{\sigl{U}{1}}} \def\Uoneac{\acpc{U}{1}{ac}\,} \def\Uonepc{\acpc{U}{1}{pc}\,}                 
\def\Utwo{{\sigl{U}{2}}} \def\Utwoac{\acpc{U}{2}{ac}\,} \def\Utwopc{\acpc{U}{2}{pc}\,}

%%%%%%%%%%%%%% Tattvabinduyoga - List of Witnesses   %%%%%%%%%%%%%%%%%%%
\DeclareWitness{ceteri}{\selectlanguage{english}cett.}{ceteri}[]   
\DeclareWitness{E}{\selectlanguage{english}E}{Printed Edition}[]    
\DeclareWitness{P}{\selectlanguage{english}P}{Pune BORI 664}[]  
\DeclareWitness{B}{\selectlanguage{english}B}{Bodleian 485}[]       
\DeclareWitness{N1}{\selectlanguage{english}N\textsubscript{1}}{NGMPP 38/31}[]
\DeclareWitness{N2}{\selectlanguage{english}N\textsubscript{2}}{NGMPP B 38/35}[]
\DeclareWitness{L}{\selectlanguage{english}L}{LALCHAND 5876}[]  
\DeclareWitness{D}{\selectlanguage{english}D}{IGNCA 30019}[] 
%\DeclareWitness{D2}{\selectlanguage{english}D\textsubscript{2}}{IGNCA 30020}[]  
\DeclareWitness{U1}{\selectlanguage{english}U\textsubscript{1}}{SORI 1574}[] 
\DeclareWitness{U2}{\selectlanguage{english}U\textsubscript{2}}{SORI 6082}[]
%%%%%%%%%%%%% Testimonia
\DeclareWitness{Ysv}{\selectlanguage{english}Ysv}{Yogasvarodaya}[] %%%add infos!  

%%%%%%%%%%%%%%%%%%%%%%%%%%%%%%%%%%%%%%%%%%%
% Macro for Editing Abbrevs.
\def\om{\textrm{\footnotesize \textit{om.}\ }} %prints om. for omitted in apparatus
\def\korr{\textrm{\footnotesize \textit{em.}\ }} %prints em. for emended in apparatus
\def\conj{\textrm{\footnotesize \textit{conj.}\ }} %prints conj. for conjectured in apparatus

% \supplied{text} EDITORIAL ADDITION -> Within \lem oder \rdg
% \surplus{text} EDITORIAL DELETION -> Within \lem oder \rdg
% \sic{text} CRUX
% \gap{text} LACUNAE -> [reason=??, unit=??, quantity=??, extent=??]


%%%%%%%%%%%%%%%%%%%%%%%%%%%%%%%%%%%%%%%%%%% All macros of this list can be used in 
% Macro for Editing Abbrevs.
\def\eyeskip{\textrm{{ab.\,oc. }}}
\def\aberratio{\textrm{{ab.\,oc. }}}
\def\ad{\textrm{{ad}}}
\def\add{\textrm{{add.\ }}}
\def\ann{\textrm{{ann.\ }}}
\def\ante{\textrm{{ante }}} 
\def\post{\textrm{{post }}}
%\def\ceteri{cett.\,}                   
\def\codd{\textrm{{codd.\ }}}

\def\coni{\textrm{{coni.\ }}}
\def\contin{\textrm{{contin.\ }}}
\def\corr{\textrm{{corr.\ }}}
\def\del{\textrm{{del.\ }}}
\def\dub{\textrm{{ dub.\ }}}

\def\expl{\textrm{{explic.\ }}} 
\def\explica t{\textrm{{explic.\ }}}
\def\fol{\textrm{{fol.\ }}}
\def\foll{\textrm{{foll.\ }}}
\def\gloss{\textrm{{glossa ad }}}
\def\ins{\textrm{{ins.\ }}}      
\def\inseruit{\textrm{{ins.\ }}} 
\def\im{{\kern-.7pt\lower-1ex\hbox{\textrm{\tiny{\emph{i.m.}}}\kern0pt}}} %\textrm{\scriptsize{i.m.\ }}}      
\def\inmargine{{\kern-.7pt\lower-.7ex\hbox{\textrm{\tiny{\emph{i.m.}}}\kern0pt}}}%\textrm{\scriptsize{i.m.\ }}}      
\def\intextu{{\kern-.7pt\lower-.95ex\hbox{\textrm{\tiny{\emph{i.t.}}}\kern0pt}}}%\textrm{\scriptsize{i.t.\ }}}           
\def\indist{\textrm{{indis.\ }}}  
\def\indis{\textrm{{indis.\ }}}
\def\iteravit{\textrm{{iter.\ }}} 
\def\iter{\textrm{{iter.\ }}}
\def\lectio{\textrm{{lect.\ }}}   
\def\lec{\textrm{{lect.\ }}}
\def\leginequit{\textrm{{l.n. }}} 
\def\legn{\textrm{{l.n. }}}
\def\illeg{\textrm{{l.n. }}}

\def\primman{\textrm{{pr.m.}}}
\def\prob{\textrm{{prob.}}}
\def\rep{\textrm{{repetitio }}}
\def\secundamanu{\textrm{\scriptsize{s.m.}}}            \def\secm{{\kern-.6pt\lower-.91ex\hbox{\textrm{\tiny{\emph{s.m.}}}\kern0pt}}}%   \textrm{\scriptsize{s.m.}}}
\def\sequentia{\textrm{{seq.\,inv.\ }}}  
\def\seqinv{\textrm{{seq.\,inv.\ }}}
\def\order{\textrm{{seq.\,inv.\ }}}
\def\supralineam{{\kern-.7pt\lower-.91ex\hbox{\textrm{\tiny{\emph{s.l.}}}\kern0pt}}} %\textrm{\scriptsize{s.l.}}}
\def\interlineam{{\kern-.7pt\lower-.91ex\hbox{\textrm{\tiny{\emph{s.l.}}}\kern0pt}}}   %\textrm{\scriptsize{s.l.}}}
\def\vl{\textrm{v.l.}}   \def\varlec{\textrm{v.l.}} \def\varialectio{\textrm{v.l.}}
\def\vide{\textrm{{cf.\ }}}
\def\cf{\textrm{{cf.\ }}} 
\def\videtur{\textrm{{vid.\,ut}}}
\def\crux{\textup{[\ldots]} }
\def\cruxx{\textup{[\ldots]}}
\def\unm{\textit{unm.}}
%%%%%%%%%%%%%%%%%%%%%%%%%%%%%%%%%%%%

% List of Scholars
\DeclareScholar{ego}{ego}[
forename=Nils Jacob,
surname=Liersch]

% Persons:14\DeclareScholar{ego}{ego}[15forename=Robert,16surname=Alessi]17% Useful shorthands:18\DeclareShorthand{codd}{codd.}{V,I,R,H}19\DeclareShorthand{edd}{edd.}{Lit,Erm,Sm}20\DeclareShorthand{egoscr}{\emph{scripsi}}{ego}

%Useful shorthands:
%\DeclareShorthand{codd}{codd.}{V,I,R,H}
%\DeclareShorthand{edd}{edd.}{Lit,Erm,Sm}
\DeclareShorthand{egoscr}{\emph{scripsi}}{ego}
\DeclareShorthand{egomute}{\unskip}{ego}

\usepackage{xparse}

%%% define environments and commands
\NewDocumentEnvironment{tlg}{O{}O{}}{\vspace{-1ex}\begin{verse}}{\hfill #1\\ \vspace{-1ex}\end{verse}} %verse environment
%\NewDocumentEnvironment{tlg}{O{}O{}}{\begin{verse}}{॥#1\hskip-4pt ॥\\ \end{verse}}
\NewDocumentCommand{\tl}{m}{{\selectlanguage{iast} #1}}

\NewDocumentCommand{\extra}{m}{{\textcolor{MidnightBlue}{#1}}} %command for additions to U2
\NewDocumentCommand{\crazy}{m}{{\textcolor{red}{#1}}} %totally corrupted passage 

\NewDocumentEnvironment{prose}{O{}}{\begin{otherlanguage}{iast}}{\end{otherlanguage}}
% \NewDocumentEnvironment{padd}{O{}}{\begin{otherlanguage}{iast}}{\end{otherlanguage}}
\NewDocumentEnvironment{tlate}{O{}}
%\NewDocumentEnvironment{tadd}{O{}}

%Define two commands: \skp ("sanskrit plus"), to be ignored by TeX in
%the edition text, but processed in the TEI output. Conversely, \skm
%("sanskrit minus") is to be processed in the edition text, but
%ignored if found in the apparatus criticus and in the TEI output:

\NewDocumentCommand{\skp}{m}{}
\TeXtoTEIPat{\skp {#1}}{#1}

%\NewDocumentCommand{\skpp}{m}{}
%\TeXtoTEIPat{\skpp {#1}}{#1}

\NewDocumentCommand{\skm}{m}{\unless\ifinapparatus#1-\fi}
\TeXtoTEIPat{\skm {#1}}{}

\NewDocumentCommand{\dd}{}{/\hskip-4pt/}
\TeXtoTEIPat{\dd {}}{//}


%%% modify environments and commands
%%% TEI mapping
\TeXtoTEIPat{\begin {tlg}}{<lg>} %lg=(Group of verse (s)) contains one or more verses or lines of verse that together form a formal unit (e.g. stanza, chorus).
\TeXtoTEIPat{\end {tlg}}{</lg>}

\TeXtoTEIPat{\begin {prose}}{<p>}
\TeXtoTEIPat{\end {prose}}{</p>}

\TeXtoTEIPat{\begin {tlate}}{<p>}
\TeXtoTEIPat{\end {tlate}}{</p>}

\TeXtoTEIPat{\\}{}
\TeXtoTEIPat{\linebreak}{<br/>}
\TeXtoTEIPat{\noindent}{}
%\TeXtoTEI{tl}{l}
\TeXtoTEI{emph}{hi}
\TeXtoTEI{bigskip}{}
\TeXtoTEI{None}{N1}
\TeXtoTEI{Ntwo}{N2}
\TeXtoTEI{Done}{D1}
\TeXtoTEI{Dtwo}{D2}
\TeXtoTEI{Uone}{U1}
\TeXtoTEI{Utwo}{U2}
%\TeXtoTEIPat{/}{ |}
%\TeXtoTEI{//}{ ||}
\TeXtoTEIPat{\korr}{em. }
\TeXtoTEIPat{\conj}{conj.}
\TeXtoTEIPat{\om}{om.}
\TeXtoTEIPat{english}{}
\TeXtoTEIPat{\hskip}{}
\TeXtoTEIPat{\hskip-4pt}{}
\TeXtoTEIPat{\hskip-2pt}{}
\TeXtoTEIPat{-}{ }
\TeXtoTEIPat{4pt}{}
\TeXtoTEIPat{2pt}{}
\TeXtoTEIPat{\textcolor {#1}{#2}}{<hi rend="#1">#2</hi>} 

% Nullify \selectlanguage in TEI as it has been used in
% \DeclareWitness but should be ignored in TEI.
\TeXtoTEI{selectlanguage}{}

\author{Nils Jacob Liersch}
\title{Yogatattvabindu of Rāmacandra\\ A Critical Edition and Annotated Translation}
\date{\today}

\parindent=15pt
\begin{document}
\maketitle
\clearpage
\tableofcontents
\addtocounter{page}{-1}
\thispagestyle{empty}
\clearpage


\chapter{The List of the 15 Yogas}
\begin{itemize}
\item It's not entirely clear if the list given at the beginning of the text codifying the fifteen \textit{yoga}s belongs to the original text or was a later addition by a another hand. One primary reason for this possibility is the structure of the \textit{yoga}s in the actual course of the text does not equal the list. The text begins with a description of \textit{kriyāyoga} and continues to describe \textit{siddhakuṇḍaliniyoga} and somewhat suprisingly mentions \textit{mantrayoga} in the same breath. One starts wondering why the structure of the text does not follow the codification. However the mention of \textit{jñānotpattav upāyaḥ} might be a clue why the second \textit{yoga} in the list might be \textit{jñānayoga}. So far it seems to me that there are three options or a combination of these to explain these apparent inconsistencies: 1. The text is highly corrupted. 2. The codification was a later addition of another hand. 3. The term \textit{jñānayoga} is listed due to the results of \textit{siddhakuṇḍalinīyoga}, which is the generation of knowledge due to the practice of a certain \textit{yoga} involving the central channel, as mentioned in this section of the text.
\end{itemize}

\chapter{Conventions in the Critical Apparatus}
\section{Sigla in the Critical Apparatus}

\begin{itemize}
\item E : Printed Edition
\item P : Pune BORI 664
\item L : Lalchand Research Library LRL5876
\item B : Bodleian Oxford D 4587
\item \None : NGMPP B 38-31
\item \Ntwo : NGMPP B 38-35 / A 1327-14
\item \Done : IGNCA 30019
\item \Dtwo : IGNCA 30020
\item \Uone : SORI 1574
\item \Utwo: SORI 6082
\end{itemize}

The order of the readings in the critical apparatus is arranged according to the quality of readings in decending order. The critical apparatus is positive. Gemitation is not recorded. 

\section{Punctuation}

The very inconsistent use of punctuation marks in the witnesses at hand makes standardization necessary. A close examination of the overall usage of punctuation suggest that in the course of the texts transmission punctuations have been dropped frequently or even have been added. Particularly in the lists given in the text the copists negliance or not properly dealing with punctuation resulted in various forms of those lists with and without punctuations. Due to missing punctuation in many instances copists either made up case endings, changed the text and combined the lists' items into compounds that weren't present in the assumed original text. Even though punctuation plays a role that should'nt be underestimated, the deviation of punctuation at the end of sentences, lists and verse-numbering will only be documented in the critical apparatus of the printed edition to meaningful extend. That means, for example that emendations of obvious mistakes in punctuation will not be recorded in the critical apparatus. However, the digital edition of this work provides a way more detailled documentation of deviations in punctuation in the form of diplomatic transcripts of each witness and even a function to display sentences cummulativly on top of each other.

In the printed edition of the \textit{Tattvayogabindu} the standard conventions of punctuation are followed:

In verse poetry, a \textit{daṇḍa} marks the end of a half verse, half of the \textit{śloka}, and the double \textit{daṇḍa} marks the end of a verse. A half verse is a \textit{pāda}, at least in some literary works, this is concluded by a \textit{daṇḍa} and the end of a \textit{śloka} by a double \textit{daṇḍa}. In prose the single \textit{daṇḍa} indicates the end of a sentence and the double \textit{daṇḍa} marks the end of a paragraph.

Variations in the usage of \textit{Avagraha} will be recorded. Items of lists will be separated by a single \textit{daṇḍa}. 

\section{Sandhi}

Among the witnesses we see deviating and inconsistent application of \textit{sandhi}. There is no clear evidence that originally \textit{sandhi} was intentionally not applied. This edition will therefore apply \textit{sandhi} consistently throughout the constituted text to provide a readable text sticking to contemporary conventions in Sanskrit. The variant readings concerning \textit{sandhi} are recorded consistently in the apparatus criticus. This is due to various textcritical problems arising from the inconsistent usage of punctuation which results in application or non-application of \textit{sandhi} wheter the respective witness applied a \textit{daṇḍa} or not. This is particularly the case within lists, which frequently occur in our compilation. Items were most likely originally separated by \textit{daṇḍa}. 

\section{Class Nasals}

Again, due to inconsistent use of class nasals among the witnesses \textit{anusvāra}s have been substituted with the respective class nasals throughout the edition.

\section{Lists}

Lists are very frequent in the \textit{Yogatattvabindu}. In fact, the text initially gives a list of 15 Yogas in the beginning and many more lists are have been utilized throughout the text. Many witness lost punctuation in the process of copying and as a consequence applied \textit{sandhi}, to arrive at a consistent and conveniently readable edition of the text, all list have been identified as such and normalized to the Nominativ Singular or Nominativ Plural form of the respective item. Items are separated by a double \textit{daṇḍa}. The differences in punctuation, as well as simple emendations regarding punctuation  won't be documented in the apparatus criticus. 
\clearpage

\chapter{Critical Edition of the Yogatattvabindu}
\begin{ekdosis}
\ekddiv{type=ed}
    \centerline{\textrm{\small{[Introduction]}}}
    \bigskip
    \begin{prose}
%--------------------------
% śrī gaṇeśāya namaḥ /                                                     rājayogāntargataḥ //  binduyogaḥ   \E 
% śrī gaṇeśāya namaḥ /                                                     atha tattvabiṃduyogaprāraṃbhaḥ     \L
% śrī ṇe ya maḥ /                                                          atha rājayoga         liṣyate      \P
% \om                                                                                                        \B      
% śrī gaṇeśāya namaḥ // śrī gurave namaḥ //                                atha rājayogaprakāro  likhyate //  \N1
% śrī gaṇeśāya namaḥ //                                                //  atha rājayogaprakāro  likhyate //  \N2
% śrī gaṇeśāya namaḥ // śrī sarasvatyai namaḥ // śrī nirañjanāya namaḥ //  atha rājayogaprakāro  likhyate //  \D1
% śrī gaṇeśāya namaḥ /  oṃ śrī niraṃjanāya //                              atha rājayogaprakāra  likhyate //  \U1
% śrī gaṇeśāya namaḥ /                                                     atha rājayoga         likhyate //  \U2
%--------------------------
%Homage to Śrī Gaṇeśa. Now the methods of rājayoga are laid down.
%--------------------------          
\app{\lem[wit={ceteri}]{śrī gaṇeśāya namaḥ}
        \rdg[wit={P}]{śrī ṇe ya maḥ}
        \rdg[wit={N1}]{śrī gaṇeśāya namaḥ || śrī gurave namaḥ ||}
        \rdg[wit={D1}]{śrī gaṇeśāya namaḥ || śrī sarasvatyai namaḥ || śrī nirañjanāya namaḥ}
        \rdg[wit={U1}]{śrī gaṇeśāya namaḥ || oṃ śrī niraṃjanāya}}\dd{}
\app{\lem[wit={N1,N2,D1}]{atha rājayogaprakāro likhyate}
        \rdg[wit={U1}]{atha rājayogaprakāra likhyate}
        \rdg[wit={E}]{rājayogāntargataḥ | binduyogaḥ}
        \rdg[wit={L}]{atha tattvabiṃduyogaprāraṃbhaḥ}
        \rdg[wit={P}]{atha rājayoga liṣyate}
        \rdg[wit={U2}]{atha rājayoga likhyate}}\dd{}\end{prose}
%-------------------------- 
% \om                        \E
% \om                        \L
% \om                        \B
% rājayogasyedaṃ phalaṃ      \P
% rājayogasya idaṃ phalaṃ    \N1
% rājayogasya idaṃ phalaṃ    \N2
% rājayogasya idaṃ phalaṃ // \D1
% rājayogasya idaṃ phalaṃ    \U1
% rājayogasyedaṃ phalaṃ /    \U2
% --------------------------
\ekddiv{type=ed}
\begin{prose}        
\app{\lem[wit={P,U2}]{rājayogasyedaṃ phalaṃ}
  \rdg[wit={N1,N2,D1}]{rājayogasya idaṃ phalaṃ}
  \rdg[wit={E,L}]{\om}}/
%--------------------------
%This is the result of \textit{rājayoga}:
%--------------------------
% \om                                                                                                                                                                       \E
% \om                                                                                                                                                                       \L
% \om                                                                                                                                                                       \B
% yena rājayogenānekarājyabhogasamaya   eva   anekapārthivavinodaprekṣaṇasamaya  eva   bahutarakālaṃ  śarīrasthitir  bhavati    sa eva  rājayogaḥ tasyaite     bhedāḥ      \P
% yena rājayogenānekarājyabhogasamaya   eva/  anekapārthivavinodaprekṣaṇasamaya  eva/  bahutarakālaṃ  śarīrasthitir  bhavati    sa eva  rājayogaḥ /  tasya ete bhedāḥ /    \N1
% yena rājayogena anekarājyabhogasamaya eva// anekapārthivavinodaprekṣaṇasamaya  eva   bahuttarakālaṃ śarīrasthitir  bhavati    sa eva  rājayogaḥ /  tasya ete bhedāḥ /    \N2
% yena rājayogena anekarājyabhogasamaya eva// anekapārthivavinodaprekṣaṇasamaya  eva// bahutarakālaṃ  śarīrasthitir  bhavati//  sa eva  rājayogaḥ // tasya ete bhedāḥ /    \D1
% yena rājayogena anekarājyabhogasamaya eva// anekapārthivavinodaprekṣaṇasamaya  eva// bahutarakālaṃ  śarīrasthitir  bhavati    sa evaṃ rājayogaḥ    tasya ete bhedāḥ //   \U1 
% yena rājayogena anekarājyabhogasamaya eva// anekapārthivavinodaprekṣyaṇasamaya eva// bahutarakālaṃ  śarīrasthitir  bhavati//  sa eva  rājayogastaisyaite     bhedāḥ //   \U2
% --------------------------
%\textit{Rājayoga} is that by which longterm durability of the body arises even amongst manifold royal pleasures even amongst the manifold royal entertainments and spectacle. This truly is \textit{rājayoga}. Of this [\textit{rājayoga}] these are the varieties: \end{tlate}
%--------------------------
yena \app{\lem[wit={P,N1}]{rājayogenānekarājyabhogasamaya}
  \rdg[wit={N2,D1,U1,U2}]{rājayogena anekarājyabhogasamaya}}
eva/ anekapārthivavinoda
      \app{\lem[wit={ceteri}]{prekṣaṇasamaya}
        \rdg[wit={U2}]{prekṣyaṇasamaya}}
      eva/ bahutarakālaṃ śarīrasthitir-bhavati/ sa
      \app{\lem[wit={ceteri}]{eva}
        \rdg[wit={U2}]{evaṃ}}
      \app{\lem[wit={ceteri}]{rājayogaḥ}
        \rdg[wit={U2}]{rājayogas}}/ 
      \app{\lem[wit={P,U2}]{tasyaite}
        \rdg[wit={ceteri}]{tasya ete}} bhedāḥ/
     \end{prose}
     \ekddiv{type=ed}
%-------------------------
%
% \om                                                                                                                                                                \E
% \om                                                                                                                                                                \L
% \om                                                                                                                                                                \B
% kriyāyogaḥ 1 jñānayogaḥ 2 caryāyogaḥ 3 haṭhayogaḥ 4 karmayogaḥ 5 layayogaḥ 6 dhyānayogaḥ 7 maṃtrayogaḥ 8 lakṣyayogaḥ 9 vāsanāyogaḥ 10 śivayogaḥ 11 brahmayogaḥ 12 advaitayogaḥ 13 siddhayogaḥ 14 rājayogaḥ 15 ete paṃcadaśayogāḥ \P
%
% kriyāyogaḥ / jñānayogaḥ / caryāyogaḥ / haṭhayogaḥ / karmayogaḥ / layayogaḥ / dhyānayogaḥ / maṃtrayogaḥ / lakṣyayogaḥ / vāsanāyogaḥ / śivayogaḥ / brahmayogaḥ / advaitayogaḥ / rājayogaḥ / siddhayogaḥ / ete paṃcadaśayogāḥ // \N1
%
% kriyāyogaḥ jñānayogaḥ caryāyogaḥ haṭhayogaḥ karmayogaḥ layayogaḥ dhyānayogaḥ maṃtrayogaḥ lakṣayogaḥ vāsanāyogaḥ śivayogaḥ brahmayogaḥ advaitayogaḥ rājayogaḥ siddhayogaḥ // ete paṃcadaśayogāḥ // \N2      
%      
% kriyāyogaḥ // jñānayogaḥ // caryāyogaḥ // haṭhayogaḥ // karmayogaḥ // layayogaḥ // dhyānayogaḥ // maṃtrayogaḥ // lakṣyayogaḥ // vāsanāyogaḥ // śivayogaḥ // brahmayogaḥ // advaitayogaḥ // rājayogaḥ // siddhayogaḥ // ete paṃcadaśayogāḥ // \D1
%
% kriyāyogaḥ // jñānayogaḥ // tvaryāyogaḥ // haṭhayogaḥ // karmayogaḥ // layayogaḥ // dhyānayogaḥ maṃtrayogaḥ  lakṣayogaḥ  vāsanāyogaḥ  śivayogaḥ  brahmayogaḥ  advaitayogaḥ  rājayogaḥ  siddhayogaḥ ete paṃcadaśayogāḥ  \U1
%
% kriyāyogaḥ // jñānayogaḥ // caryāyogaḥ // haṭhayogaḥ // karmayogaḥ // nayayogaḥ // dhyānayogaḥ // maṃtrayogaḥ // lakṣyayogaḥ // vāsanāyogaḥ // śivayogaḥ // brahmayogaḥ // advaitayogaḥ // siddhayogaḥ // rājayogaḥ // evaṃ paṃcadaśāyogā bhavaṃti // \U2
%-------------------------
       \begin{prose}
         kriyāyogaḥ 1\dd{}
         jñānayogaḥ 2\dd{}
         \app{\lem[wit={ceteri}]{caryāyogaḥ}
          \rdg[wit={U1}]{tvaryāyogaḥ}} 3\dd{}
        haṭhayogaḥ 4\dd{}
        karmayogaḥ 5\dd{}
        \app{\lem[wit={ceteri}]{layayogaḥ}
          \rdg[wit={U2}]{nayayogaḥ}} 6\dd{}
        dhyānayogaḥ 7\dd{}
        mantrayogaḥ 8\dd{}
        \app{\lem[wit={ceteri}]{lakṣyayogaḥ}
          \rdg[wit={U1}]{lakṣayogaḥ}} 9\dd{}
        vāsanāyogaḥ 10\dd{}
        śivayogaḥ 11\dd{} 
        brahmayogaḥ 12\dd{}
        advaitayogaḥ 13\dd{} 
        \app{\lem[wit={P,U2}]{siddhayogaḥ}
          \rdg[wit={N1,N2,D1,U1}]{rājayogaḥ}} 14\dd{}
        \app{\lem[wit={P,U2}]{rājayogaḥ}
          \rdg[wit={ceteri}]{siddhayogaḥ}} 15\dd{}     
        \note[type=philcomm, labelb=s4.z5a, lem={rājayoga}]{The initial codification of 15 \textit{yoga}s appears in N1,N2,P,D1,U1 and U2. It is ommitted in E and L. B can't be determined due to missing folios.}
        \app{\lem[wit={P,N1,D1,U1}]{ete pañcadaśayogāḥ}
          \rdg[wit={U2}]{evaṃ paṃcadaśāyogā bhavaṃti}}\dd{}
      \end{prose}
    \end{ekdosis}
    \ekdpb*{}
    %%%%%%%%%%%%
    %%%%%%%%%%%%
    %%%%%%%%%%%
    %%%%%%%%%%%%%
    %%%%%%%%%%%%
     \begin{ekdosis}
       \ekddiv{type=ed}
       \bigskip
        \centerline{\textrm{\small{[Description of Kriyāyoga]}}}
        \bigskip
%--------------------------        
% \om                                      \E
% \om                                      \L
% \om                                      \B
% idānīṃ kriyāyogasya lakṣaṇaṃ kathyate/   \P
% idānīṃ kriyāyogasya lakṣaṇaṃ kathyate/   \N1
% idānī  kriyāyogasya lakṣaṇaṃ kathyate//  \N2
% idānīṃ kriyāyogasya lakṣaṇaṃ kathayate/  \D1
% idānīṃ kriyāyogasya lakṣaṇaṃ kathyate/   \U1
% atha   kriyāyogas   lakṣaṇaṃ          // \U2
%--------------------------
%Now the characteristic of the Yoga of [mental] action (\textit{kriyāyoga}) described.
%--------------------------
 \begin{prose}
        \app{\lem[wit={ceteri}]{idānīṃ}
            \rdg[wit={N2}]{idānī}
            \rdg[wit={U2}]{atha}}
          \app{\lem[wit={ceteri}]{kriyāyogasya}
            \rdg[wit={U2}]{kriyāyogas}} lakṣaṇaṃ
          \app{\lem[wit={ceteri}]{kathyate}
            \rdg[wit={D1}]{kathayate}
            \rdg[wit={U2}]{\om}}/
\end{prose}
 \ekddiv{type=ed}
 \begin{tlg}
%--------------------------   
% \om                                                    \E
% \om                                                    \L
% \om                                                    \B
% kriyāmuktir    ayaṃ yogaḥ    svapiṇḍe siddhidāyakaḥ    \P
% kriyāmuktir    ayaṃ yogaḥ /  svapiṇḍe siddhidāyakaḥ /  \N1
% kriyāmukti    layaṃ yogaḥ    svapiṇḍe siddhidāyakaḥ /  \N2
% kriyāmuktir    ayaṃ yogaḥ    svapiṇḍe siddhidāyakaḥ /  \D1
% kriyāyuktir    ayaṃ yogaḥ /  svapiṇḍe siddhidāyakaḥ /  \U1
% kriyāmuktiḥ // ayaṃ yogaḥ    svapiṃ?  siddhidāyakaṃ // \U2 
%--------------------------
%This Yoga is liberation through [mental] action, it bestows success(\textit{siddhi}) in ones own body.
%-------------------------- 
   \tl{\note[type=testium, labelb=s10.z2, lem=kriyāmuktir]{Ysv:kriyāmuktimayo yogaḥ sapiṇḍisiddhidāyakaḥ | yatkāromīti saṅkalpaṃ kāryārambhe manaḥ sadā ||}
     \app{\lem[wit={ceteri}, alt={kriyāmuktir}]{kriyāmukti\skp{r-}}
    \rdg[wit={N2}]{kriyāmukti}
    \rdg[wit={U2}]{kriyāmuktiḥ ||}
}\app{\lem[wit={ceteri}, alt={ayaṃ}]{\skm{r-}ayaṃ}
  \rdg[wit={N2}]{layaṃ}}
yogaḥ svapiṇḍe
\app{\lem[wit={ceteri}]{siddhidāyakaḥ}
  \rdg[wit={U2}]{siddhidāyakaṃ}}/}\\
%-------------------------
% \om                                                    \E
% \om                                                    \L
% \om                                                    \B
% yaṃ yaṃ karoti kallolaṃ kāryāraṃbhe manaḥ sadā         \P
% yaṃ yaṃ karoti kallolaṃ kāryāraṃbhe manaḥ sadā/        \N1
% yaṃ yaṃ karoti kallolaṃ kāryāraṃbhe manaḥ sadā//1//    \N2
% yaṃ yaṃ karoti kallolaṃ kāryāraṃbhe manaḥ sadā/        \D1 
% yaṃ yaṃ karoti kallolaṃ kāryāraṃbhe manaḥ sadā/ 1      \U1
% yaṃ yaṃ karoti kallolaṃ kāryāraṃbhe manaḥ sadā/        \U2
%--------------------------
%Each wave the mind creates at the beginning of an action,
%-------------------------- 
\tl{yaṃ yaṃ karoti kallolaṃ kāryāraṃbhe manaḥ sadā/}\\
%--------------------------
% \om                                                        \E
% \om                                                        \L
% \om                                                        \B
% tattataḥ   kuñcanaṃ kurvan kriyāyogas tato bhavet            \P
% tattataḥ   kuñcanaṃ kurvan kriyāyogas ato bhava     //       \N1
% tattataḥ   kūrcanaṃ kurvan kriyāyogas ato bhava     //       \N2
% tattataḥ   kuñcanaṃ kurvan kriyāyogas ato bhava     //       \D1 
% taṃkṛ taṃ  kuñcanaṃ kurvan kriyāyogas ato ?va      //1//    \U1
% tatastataḥ kuṃcanaṃ kurvan kriyāyogas tato bhavet //1//    \U2
%--------------------------
%of all those one shall withdraw oneself. Then \textit{kriyāyoga} arises.
%--------------------------
\tl{\note[type=testium, labelb=s10.z4, lem=tattataḥ]{Ysv:tatsāṅgācaraṇaṃ kurvan kriyāyogar ato bhavet |}
  \app{\lem[wit={ceteri}]{tattataḥ}
    \rdg[wit={U2}]{tatastataḥ}
    \rdg[wit={U1}]{taṃkṛ taṃ}}
  \app{\lem[wit={ceteri}]{kuñcanaṃ}
    \rdg[wit={N2}]{kūrcanaṃ}} kurvan-kriyāyoga\skp{s}-\app{
    \lem[wit={P,U2}, alt={tato bhavet}]{\skm{s}tato bhavet}
    \rdg[wit={N1,N2,D1}]{ato bhava}
    \rdg[wit={U1}]{ato va}}\dd{}1\hskip-2pt\dd{}}
\end{tlg}
      \ekddiv{type=ed}
    \begin{tlg}
%--------------------------      
% \om                                                                                                 \B
% \om                                                                                                 \L
% kṣamā vivekaṃ vairāgyaṃ śāntiḥ santoṣaniṣpṛhā       etadyuktiyuto  yogī   kriyāyogī nigadyate       \E
% kṣamāvivekavairāgyaṃ    śāntiḥ santoṣanispṛhāḥ      etadyuktiyuto  yogī   kriyāyogī nigadyate       \P
% kṣamāvivekavairāgyaṃ    śāntiḥ santoṣanispṛhā       etat yuktiyuto yogī   kriyāyogī nigadyate       \N1
% kṣamāvivekavairāgyaṃ    śāntiḥ santoṣanispṛhā //2// etat yuktiyuto yo sau kriyāyogī nigadyate//     \N2
% kṣamāvivekavairāgyaṃ    śāntiḥ santoṣanispṛhaḥ      etat yuktiyuto yogī   kriyāyogī nigadyate       \D1
% kṣamāvivekavairāgya---- śāntisantoṣaniḥspṛhī        etadyuktiyuto  yo sau kriyāyogī nigadyate       \U1 
% kṣamā vivekaṃ vairāgyaṃ śāntisaṃtoṣaniṣpṛhāḥ //     etat muktiyuto yogī   kriyāyogī nigadyate //2// \U2
%--------------------------
%Patience, discrimination, equanimity, peace, modesty, desireless: The \textit{yogī} who is endowed with these means is said to be a \textit{kriyāyogī}.
%--------------------------
% The text of the Printed Edition starts here ---> 
%--------------------------
      \tl{\note[type=testium, labelb=s10.z5, lem=kṣamā°]{Ysv:kṣamāvivekavairāgyaśāntisantoṣanispṛhāḥ | etan muktiyuto yo'sau kriyāyogo nigadyate |}
        kṣamā
        \app{\lem[wit={ceteri}]{viveka}
          \rdg[wit={E,U2}]{vivekaṃ}}vairāgyaṃ \note[type=philcomm, labelb=s6.z6a, lem={°kṣamā}]{The printed edition E starts here.}
        śāntisantoṣa
        \app{\lem[wit={P}]{nispṛhāḥ}
          \rdg[wit={U2}]{°niṣpṛhāḥ}
          \rdg[wit={E,N1}]{°nispṛhā}
          \rdg[wit={D1}]{°nispṛhaḥ}
          \rdg[wit={U1}]{°niṣpṛhī}}/}\\
      \tl{\app{\lem[wit={E,P,U1}]{eta\skp{d}}
          \rdg[wit={N1,N2,D1,U2}]{etat}} 
        \app{\lem[wit={ceteri}, alt={yuktiyuto}]{\skm{d}-yuktiyuto}  %%%SANDHI
    \rdg[wit={U2}]{muktiyuto}}
  \app{
    \lem[wit={E,P,N1,D1,U2}]{yogī}    
    \rdg[wit={N2,U1}]{yo sau}}
  kriyāyogī nigadyate\dd{}2\hskip-2pt\dd{}}
\end{tlg}
       \ekddiv{type=ed}
     \begin{tlg}
%-----------------------
% \om                                                \B
% \om                                                \L
% mātsaryaṃ mamatā māyā hiṃsā ca   madagarvitā /     \E
% mātsarya  mamatā māyā hiṃsāśā    madagarvitāḥ      \P
% mātsarya  mamatā māyā hiṃsāḥ //  madagarvatā /     \N1    -> the hiṃsā---''ḥ//'' in \nepal looks like a śā -> indicator that the others copied from \nepal? 
% mātsarya  mamatā māyā hiṃsāśā    madagārvatā //3// \N2
% mātsarya  mamatā māyā hiṃsāśā    madagarvatā /     \D1
% mātsaryaṃ mamatā māyā hiṃsāśā    madagarvatā /     \U1
% mātsaryaṃ mamatā māyā hiṃsāśā    madagarvatā /     \U2
%-----------------------
%Envy, selfishness, cheating, violence, desire and intoxication, pride,
%-----------------------
       \tl{\note[type=testium, labelb=s10.z6, lem=mātsaryaṃ]{Ysv:mātsaryaṃ mamatā māyā hiṃsā ca madagarvitā | kāmaḥ krodho bhayaṃ lajjā lobho mohas tathā 'śuciḥ ||}
         \app{\lem[wit={E,U1,U2}]{mātsaryaṃ}
           \rdg[wit={P,N1,D1}]{mātsarya}}
         mamatā māyā
         \app{\lem[wit={E}]{hiṃsā ca}
           \rdg[wit={ceteri}]{hiṃsāśā}
           \rdg[wit={E}]{hiṃsā ca}
           \rdg[wit={N1}]{hiṃsāḥ}}
         madagarvatā/}\\
%-----------------------
% \om                                                   \B
% \om                                                   \L
% kāmakrodhabhayaṃ   lajjā lobhamohau tathā śuciḥ //      \E
% kāmakrodhabhayaṃ   lajjā lobhamohau tathā 'śuciḥ        \P
% kāmakrodhabhayaṃ   lajjā lobhamohau tathā 'śuciḥ /      \N1    -> the hiṃsā---''ḥ//'' in \nepal looks like a śā -> indicator that the others copied from \nepal? 
% kāmakrodho bhayaṃ  lajjā lobhamohau tathā śuciḥ //    \N2
% kāmakrodho bhayaṃ  lajjā lobhamohau tathā 'śuciḥ //    \D1
% kāmakrodhau bhayaṃ lajjā lobhamohau tathā 'śuciḥ      \U1
% kāmakrodhau bhayaṃ lajjā lobhamohau tathā śuciḥ //3// \U2
% -----------------------
% lust, anger, fear, laziness, greed, error and impurity.
%-----------------------
       \tl{kāma\app{\lem[wit={U1,U2}, alt={°krodhau}]{krodhau}
           \rdg[wit={E,P,N1}]{krodha°}
           \rdg[wit={D1}]{°krodho}}
         bhayaṃ lajjā lobhamohau tathā
         \app{\lem[wit={ceteri}]{'śuciḥ}
           \rdg[wit={E,N2,U2}]{śuciḥ}}\dd{}3\hskip-2pt\dd{}}    %%%AVAGRAHA
\end{tlg}
        \ekddiv{type=ed}
      \begin{tlg}
%-----------------------
%  \om                                                           \B
%  atha dveṣo ghṛṇālasyaṃ bhrāṃtir   daṃbho kṣamā bhramaḥ //     \L
%  rāgadveṣau ghṛṇālasyaṃ bhrāntitvaṃ     mokṣamā bhramaḥ /      \E
%  rāgadveṣau ghṛṇālasyaṃ bhrāṃtir   ddaṃbhokaṣmā bhramaḥ        \P
%  rāgadveṣau ghṛṇālasyaṃ bhrāṃtir   daṃbho kṣamā bhramaḥ //4//  \N1
%  rāgadveṣau ghṛnālasyaṃ bhrāṃtir   daṃbho kṣamā bhramaḥ //4    \N2
%  rāgadveṣau ghṛṇālasyaṃ bhrāṃtir   debho  kṣamā bhramaḥ //     \D1
%  rāgadoṣau  ghṛṇālasyaṃ bhrāṃti    daṃbha kṣamī bhramaḥ 4      \U1
%  rāgadveṣau ghṛṇālasyaṃ bhrāṃtir   daṃbho kṣamā bhramaḥ //     \U2
%-----------------------
%Attachment and aversion, indignation and idleness, impatience and dizzyness
%-----------------------
        \tl{\note[type=testium, labelb=s10.z6, lem=rāgadveṣau]{Ysv:rāgadveṣau ghṛṇālasyaśrāntidambhakṣamābhramāḥ | yasyaitāni na vidyante kriyāyogī sa ucyate ||}
          \app{\lem[wit={ceteri}]{rāgadveṣau}
            \rdg[wit={U1}]{rāgadoṣau}
            \rdg[wit={L}]{athadveṣo}}\note[type=philcomm, labelb=s7.z10, lem={rāga°}]{L starts here.}
          \app{\lem[wit={ceteri}]{ghṛṇālasyaṃ}
            \rdg[wit={N2}]{ghṛnā°}} 
          \app{
            \lem[wit={ceteri}, alt={bhraṃtir daṃbho}]{bhrantir-daṃbho}
            \rdg[wit={D1}]{bhrāṃtir debho}
            \rdg[wit={E}]{bhrāntitvaṃ}
            \rdg[wit={U1}]{bhrāṃti daṃbha}}
          \app{\lem[wit={ceteri}]{kṣamā bhramaḥ}
            \rdg[wit={E}]{mokṣam ābhramaḥ}
            \rdg[wit={U1}]{°kṣamī bhramaḥ}}/}\\
%-----------------------
%  \om                                               \B
%  yasyai tāni na vidyaṃte kriyāyogī sa ucyate //    \L
%  yasyai tāni ca vidyante kriyāyogī sa ucyate 3     \E
%  yasyai tāni na vidyaṃte kriyāyogī sa ucyate       \P1
%  yasyai tāni na vidyaṃte kriyāyogī sa ucyate //    \N1
%  yasyai tāni na vidyaṃte kriyāyogī sa ucyate //    \N2
%  yasyai tāni na vidyaṃte kriyāyogī sa ucyate //    \D1
%  yasyai tāni na vidyaṃte kriyāyogī sa ucyate       \U1
%  yasyai tāni na vidyaṃte kriyāyogī sa ucyate //4// \U2 
%  -----------------------
% Whoever doesn't experience these is called a \textit{kriyāyogī}. 
%  -----------------------        
        \tl{\note[type=philcomm, labelb=s10.z7, lem=yasyai]{Rāmacandra ends his quotes from the Ysv and continues the rest of his section on Kriyāyoga in prose. The Ysv ends the section on Kriyāyoga as follows: sa eva muktaḥ sa jñānī caṇḍināśena īśvaraḥ | kriyāmuktikaro yo'sau rājayogaḥ sa muktidaḥ ||}
          yasyai tāni \app{\lem[wit={ceteri}]{na}\rdg[wit={E}]{ca}} vidyante kriyāyogī sa ucyate\dd{}4\hskip-2pt\dd{}}\\
\end{tlg}
     \ekddiv{type=ed}
      \begin{prose}
%-----------------------
%  \om                                                                                          \B
%  yasyāntaḥkaraṇe kṣamāvivekavairāgyaśāntisantoṣādīny                         utpadyante //     \E
%  yasyāṃtaḥkaraṇe kṣamāvivekavairāgyaśāṃtisaṃtoṣa         ityādīny            utpādyaṃte        \P
%  tasyāṃtaḥkaraṇe kṣamāvivekavairāgyaśāṃtisaṃtoṣa         ityādīnotpādyaṃte                    \L
%  yasyāṃtaḥkaraṇe kṣamāḥ vivekavairāgya /    śāṃtisaṃtoṣa ityādīni            utpādyaṃte        \N1
%  yasyāṃtaḥkaraṇe kṣamā' vivekavairāgyā      śāṃtisaṃtoṣa ityādīni            utpādyaṃte /      \N2 %see Mss p3 recto vierte Zeile von unten  
%  yasyāṃtaḥkaraṇe kṣamā // vivekavairāgya // śāṃtisaṃtoṣa ityādīni            utpādyaṃte //     \D1
%  yasyāṃtaḥkaraṇe kṣamāvivekavairāgyaśāṃtisaṃtoṣa         ityādīna niraṃtaram utyaṃte        \U1
%  yasyāṃtaḥkaraṇe kṣamāvivekavairāgyaśāṃtisaṃtoṣa         ityādayo niraṃtaraṃ utpādyaṃte       \U2
%  -----------------------
%  Patience, discrimination, equanimity, peace, contentment etc. are generated in his mind.
%  -----------------------        
        yasyāntaḥkaraṇe
        \app{\lem[wit={ceteri},alt={kṣamā°}]{kṣamā}
          \rdg[wit={N1}]{kṣamāḥ}
          \rdg[wit={N2}]{kṣamā'}
        }\app{\lem[wit={ceteri}]{vivekavairāgyaśānti}
          \rdg[wit={N1}]{kṣamāḥ vivekavairāgya | śāṃti°}
          \rdg[wit={N2}]{°vairāgyāśānti°}
          \rdg[wit={D1}]{kṣamā || vivekavairāgya || śāṃti°}
        }\app{\lem[wit={ceteri}, alt={°santoṣa ityādīny}]{santoṣa ityādīny\skm{-u}} %the°-problem
          \rdg[wit={E}]{°santoṣādīny}
          \rdg[wit={L}]{°santoṣa ity ādīno°}
          \rdg[wit={U1}]{°santoṣa ity ādīna niraṃtaram}
          \rdg[wit={U2}]{°santoṣa ity ādayo niraṃtaraṃ}
        }\app{\lem[wit={ceteri}]{\skp{-}utpādyante}
          \rdg[wit={E}]{utpadyante}
          \rdg[wit={L}]{°tpādyaṃte}
          \rdg[wit={U1}]{utyaṃte}}/
%-----------------------
% \om \oxford
%  sa eva bahukriyāyogī kathyate /      \E
%  sa eva bahukriyāyogī kathyate        \P
%  sa eva bahukriyāyogī kathyate //     \L
%  sa eva bahukriyāyogī kathyate /      \N1
%  sa eva bahukriyāyogī sa kathyate /   \N2
%  sa eva bahukriyāyogā sa kathyate //  \D1
%  sa eva bahukriyāyogī kathyate /      \U1
%  sa eva bahukriyāyogī tkacyate /      \U2
%-----------------------
% He alone is called a \textit{yogī} of many actions (\textit{bahukriyāyogī}).
%-----------------------
        sa eva
        \app{\lem[wit={ceteri}]{bahukriyāyogī}
          \rdg[wit={D1}]{bahukriyāyogā}}
        \app{\lem[wit={ceteri}]{kathyate}
          \rdg[wit={D1,N2}]{sa kathyate}
          \rdg[wit={U2}]{tkacyate}}/
     \end{prose}
   \ekddiv{type=ed}
   \begin{prose}
%-----------------------
% \om \B
%               kāpaṭyaṃ      vittaṃ   hiṃsā    tṛṣṇā    mātsaryam    ahaṃkāraḥ    roṣaḥ kṣayaṃ   lajjā lobhamohā      aśucitvaṃ                       pākhaṃḍatvaṃ       bhrāntiḥ indriyavikāraḥ kāmaḥ          ete yasya manasi pratidinaṃ vyunā bhavanti /  \E
%               kāpaṭyaṃ      vittaṃ   hiṃsā    tṛṣṇā    mātsaryaṃ    ahaṃkāraḥ    roṣo bhayaṃ    lajjā lobhaḥ mohaḥ  aśucitvaṃ rāgaḥ dveṣaḥ   ālasyaṃ pākhaṃḍitvaṃ       bhrāṃtiḥ indriyaṃ vikāraḥ kāmaḥ        ete yasya manasi pratidinaṃ nyunā bhavanti   \P
%               kāpayaṃ     //vitaṃ // hiṃsā // tṛṣṇā // mātsaryaṃ // ahaṃkāraḥ // roṣo bhayaṃ // lajjā lobhaḥ // moha aśucitvaṃ // rājadveṣa  alasyaṃ // pākhaṃḍitvaṃ // bhrāṃtiḥ // itivikāraḥ // kāmaḥ        eta yasya manasi pratidinaṃ nyunā bhavaṃti//\L
%yasyāṃtakaraṇe kapatyaṃ māyā vitvaṃ   hiṃsā    tṛṣṇā    mātsaryaṃ    ahaṃkāraḥ    roṣo bhayaṃ     lajjā // lobhamohā  asucitvaṃ rāgadveṣaḥ // alasyaṃ pāṣaṃḍitvaṃ      bhraṃtiḥ / iṃdriyaivikāraḥ / kāmaḥ       ete yasya manasi pratidinaṃ nyunā bhavaīti/\N1
%               kāpaṭyaṃ māyā vitvaṃ   hiṃsā    tṛṣṇā    mātsaryaṃ    ahaṃkāraḥ    e?ṣo bhayaṃ    lajjā/ lobhamoha    asūcitvaṃ rāgadveṣaḥ    ālasyaṃ pārṣaḍitvaṃ        bhrāṃtiḥ iṃdriyavikāraḥ // kāma         ete yasya manasi pratidinaṃ nyunā bhavaṃti //  \N2      
%               kāpaṭyaṃ māya vitvaṃ   hiṃsā    tṛṣṇā    mātsarya     ahaṃkāraḥ    roṣo bhayaṃ     lajjā // lobhamohā  asucitvaṃ rāgadveṣaḥ // ālasyaṃ pāṣaṃḍitvaṃ        bhraṃtiḥ // iṃdriyavikāraḥ // kāmaḥ // ete yasya manasi pratidinaṃ nyunā bhavaṃti //  \D1
%               kāpachaṃ yāya vitvaṃ   hiṃsā    tṛṣṇā    mātsarya     ahaṃkāraḥ    roṣaḥ bhayaṃ   lajā lobhamohā      aśucitvaṃ rāgadveṣaḥ    ālasyaṃ pākhaṃḍitvaṃ       bhraṃtiḥ iṃdriyavīkāraḥ    kāmaḥ       rāte yasya manasi pratidinaṃ nyunā bhavaṃti //      \U1
%               kāpaṭyaṃ pāpā titaṃ    hiṃsā    tṛṣṇā    mātsaryaṃ // ahaṃkāraḥ    roṣo bhayaṃ     lajjā ----mohā      aśucitvaṃ rāgadveṣaḥ    ālasyaṃ pākhaṃḍitvaṃ //    bhraṃtiḥ iṃdriyavikāraḥ //-----        etate yasya manasi pratidinaṃ nyunā bhavaṃti // \U2
%-----------------------
%Fraud, illusion, property, violence, craving, envy, ego, anger, anxiety, shame, greed, error, impurity, attachment, aversion, idleness, heterodoxy, false view, affection of the senses, sexual desire: He who diminishes these from day to day in is mind,
%-----------------------              
      \app{\lem[wit={ceteri}]{kāpaṭyaṃ}
        \rdg[wit={N1}]{yasyāntaḥkaraṇe kapatyaṃ}
        \rdg[wit={L}]{kāpayaṃ}
        \rdg[wit={U1}]{kāpachaṃ}}\dd{}
      \app{\lem[wit={N1,N2}]{māyā}
        \rdg[wit={D1}]{māya}
        \rdg[wit={U1}]{yāya}
        \rdg[wit={U2}]{pāpa}
        \rdg[wit={E,P,L}]{\om}}\dd{}
        %\rdg[wit={E,P,L}]{\textbf{omitted in}}}
      \app{\lem[wit={E,P}]{vittaṃ}
        \rdg[wit={L}]{vitaṃ}
        \rdg[wit={N1,N2,D1,U1}]{vitvaṃ}
        \rdg[wit={U2}]{titaṃ}}\dd{}
      hiṃsā\dd{}
      tṛṣṇā\dd{}
      \app{\lem[wit={ceteri}]{mātsaryaṃ}
        \rdg[wit={E}]{mātsaryam}
        \rdg[wit={D1,U1}]{mātsarya}}\dd{}
      ahaṃkāraḥ\dd{}
      \app{\lem[wit={E,U1}]{roṣaḥ}
        \rdg[wit={ceteri}]{roṣo}
        \rdg[wit={N2}]{eṣo}}\dd{}
      \app{\lem[wit={ceteri}]{bhayaṃ}
        \rdg[wit={E}]{kṣayaṃ}}\dd{}
      \app{\lem[wit={ceteri}]{lajjā}
        \rdg[wit={U1}]{lajā}}\dd{}
      \app{\lem[wit={P,L}]{lobhaḥ}
        \rdg[wit={ceteri}]{lobha°}
        \rdg[wit={U2}]{\om}}\dd{}
      \app{\lem[wit={P}]{mohaḥ}
        \rdg[wit={L,N2}]{moha}
        \rdg[wit={ceteri}]{mohā}}\dd{}        
      \app{\lem[wit={ceteri}]{aśucitvaṃ}  %%%Frage: vor daṇḍa wird m zu ṃ??? 
        \rdg[wit={N1,D1}]{aśucitvaṃ}
        \rdg[wit={N2}]{aśūcitvaṃ}}\dd{}
      \app{\lem[wit={P}]{rāgaḥ}
        \rdg[wit={ceteri}]{rāga°}
        \rdg[wit={L}]{rāja°}
        \rdg[wit={E}]{\om}}\dd{}
      \app{\lem[wit={ceteri}]{dveṣaḥ}
        \rdg[wit={L}]{dveṣa}
        \rdg[wit={E}]{\om}}\dd{}
      \app{\lem[wit={ceteri}]{ālasyaṃ}
        \rdg[wit={E}]{\om}}\dd{}
      \app{\lem[wit={ceteri}]{pākhaṃḍitvaṃ}
        \rdg[wit={D1,N1}]{pāṣaṃḍitvaṃ}
        \rdg[wit={E}]{pākhaṃḍatvaṃ}
        \rdg[wit={N2}]{pārṣaḍitvaṃ}}\dd{}
     bhrāntiḥ\dd{}
     \app{\lem[wit={ceteri}, alt={indriyavikāraḥ}]{indriyavikāraḥ}
        \rdg[wit={U1}]{iṃdriyavīkāraḥ}
        \rdg[wit={P}]{iṃdriyaṃ vīkāraḥ}
        \rdg[wit={L}]{itivikāraḥ}}\dd{}
      \app{\lem[wit={ceteri}]{kāmaḥ}
        \rdg[wit={N2}]{kāma}
        \rdg[wit={U2}]{\om}}\dd{}
      \app{\lem[wit={ceteri}]{ete}
        \rdg[wit={L}]{eta}
        \rdg[wit={U1}]{rāte}
        \rdg[wit={U2}]{etate}}
      yasya manasi pradidinaṃ nyūna
      \app{\lem[wit={ceteri}]{bhavanti}
        \rdg[wit={N1}]{bhavaīti}}/ 
%-----------------------       
%sa eva bahukriyāyogī kathyate// \E
%sa eva bahukriyāyogī kathyate// \P
%sa eva bahukriyāyogī kathyate// \L
%sa eva bahukriyāyogī kathyate// \N1
%sa eva bahukriyāyogī kathyate// \N2
%sa eva bahukiyāyogī kathyate//  \D1
%sa eva bahukiyāyogī kathyaṃte// \U1
%sa eva bahukiyāyogī kathyaṃte// \U2
%-----------------------
%he alone is called a yogī of many actions (\textit{bahukriyāyogī})
%-----------------------
sa eva \app{\lem[wit={ceteri}]{bahukriyāyogī}
  \rdg[wit={D1,U1,U2}]{°kiyā°}}
      \app{\lem[wit={ceteri}]{kathyate}
        \rdg[wit={U1,U2}]{kathyaṃte}}/  
    \end{prose}
  \end{ekdosis}
  \ekdpb*{}
  %%%%%%%%%%%
  %%%%%%%%%%%%%
  %%%%%%%%%%%
  %%%%%%%%%%%%
  %%%%%%%%%%
 \begin{ekdosis}
    \ekddiv{type=ed}
    \bigskip
    \centerline{\textrm{\small{[Siddhakuṇḍalinīyoga and Mantrayoga]}}}
    \bigskip
    \begin{prose}
%-----------------------   
% \om                                   \B
%idānīṃ rājayogasya bhedāḥ kathyante // \E
%idānīṃ rājayogasya bhedāḥ kathyaṃte    \P
%idānīṃ rājayogasya bhedāḥ              \L
%idānīṃ rājayogasya bhedāḥ kathyaṃte    \N1
%idānīṃ rājayogasya bhedā kathyate//    \N2
%idānīṃ rājayogasya bhedāḥ kathyaṃte // \D1     
% \om                                   \U1
%idānīṃ rājayogasya bhedāḥ kathyaṃte // \U2
%-----------------------
%Now varieties of \textit{rājayoga} will be described.
%-----------------------
       idānīṃ rājayogasya
       \app{\lem[wit={ceteri}]{bhedāḥ}
         \rdg[wit={N2}]{bhedā}}
       \app{\lem[wit={ceteri}]{kathyante}
         \rdg[wit={N2}]{kathyate}
         \rdg[wit={L}]{\om}}/\note[type=philcomm, labelb=s8.z5a, lem={kathyante}]{The whole sentence is \om in U1.}     
%-----------------------
%te ke     \E
%te ke     \P
%te ke     \L
%ke te //  \D1
%ke te /   \N1
%kriyate// \N2       
%ke te     \U1
%te ke     \U2
%-----------------------
%Which are these?
%-----------------------       
\app{\lem[wit={D1,N1,U1}]{ke te}
         \rdg[wit={ceteri}]{te ke}
         \rdg[wit={N2}]{kriyate}}/ 
%-----------------------
%\om                                       \B
%ekaḥ siddhakuṇḍalinīyogaḥ / mantrayogaḥ / \E
%ekaḥ siddhakuṃḍaṃliṃ yogaḥ maṃtrayogaḥ    \P
%ekaḥ siddhakuṇḍalanīyoga /                \L 
%ekaḥ siddhakuṇḍalinīyogaḥ maṃtrayogaḥ /   \N1
%ekaḥ siddhakuṇḍalanīyogaḥ maṃtrayogaḥ //  \N2
%ekaḥ siddhakuṃḍalanīyogaḥ mantrayogaḥ //  \D1 
%ekaḥ siddhakuṇḍaliniyogaḥ mantrayogaḥ     \U1
%ekaḥ siddhakuṇḍalinīyoga // mantrayogaḥ   \U2
%-----------------------
%One is \textit{siddhakuṇḍalinīyoga} [and one] is \textit{mantrayoga}.       
%-----------------------
       ekaḥ
       \app{\lem[wit={E,N1}]{siddhakuṇḍalinīyogaḥ}
         \rdg[wit={U1}]{siddhakuṇḍalinīyogaḥ}
         \rdg[wit={U2}]{siddhakuṇḍalinīyoga}
         \rdg[wit={N2,D1}]{siddhakuṃḍalanīyogaḥ}
         \rdg[wit={P}]{siddhakuṃḍaṃliṃ yogaḥ}}
       \app{\lem[wit={ceteri}]{mantrayogaḥ}
         \rdg[wit={L}]{\om}}/ \note[type=philcomm, labelb=s8.z5aa, lem={mantrayogaḥ}]{The sudden appearance of \textit{mantrayoga} seems odd: This section of the text doesn't mention the practice of \textit{mantra} at all. It might me a mistake, or a later insertion. However, all witnesses preserve this reading except L. The sentence that follows confirms the reading of Mantrayoga by the usage of dual forms.}
%-----------------------
% \om                         \B
%astu rājayogaḥ kathyate /    \E
%amū rājayogau kathyete       \P
%amū rājayogau kathyate //    \L
%amū rājayogau kathyate       \N1
%amū rājayogau kathyate//     \N2  %%%p3verso
%amū rājayogau kathyate //    \D1 
%amū rājayogau kathyate       \U1
%amū rājayogau kathyaṃte //   \U2
%-----------------------
%These two rājayogas are described [in the following].
%-----------------------
       \app{\lem[wit={ceteri}]{amū}
         \rdg[wit={E}]{astu}}
       \app{\lem[wit={ceteri}]{rājayogau}
         \rdg[wit={E}]{rājayogaḥ}}
       \app{\lem[wit={P}]{kathyete}
         \rdg[wit={ceteri}]{kathyate}
         \rdg[wit={U2}]{kathyaṃte}}/
%-----------------------
% \om                                                              \B
%mūlakandasthāne    ekā tejorūpā    mahānāḍī varttate /            \E
%mūlaṃ kaṃdasthāne  ekā tejorūpā    mahānāḍī varttate              \P
%mūlakaṃdasthāne    ekā tejorūpā    mahānāḍī vartate               \L
%mūlakaṃdasthāne    eka tejorūpā    mahānāḍī varttate /            \N1
%mūlakaṃdasthāne    eka tejorūpā    mahānāḍī varttate /            \N2
%mūlakaṃdasthāne    ekā tejorūpā    mahānāḍī varttate //           \D1 
%mūlakaṃdasthāne    ekā tejorūpā    mahānāḍī vartate /             \U1
%mūlakaṃdasthāne // ekā tejorūpā // mahānāḍī pravarttate /         \U2
%-----------------------
%At the location of the root-bulb exists one major vessel in the form of energy.
%-----------------------       
       \app{\lem[wit={ceteri}]{mūlakandasthāne}
         \rdg[wit={P}]{mūlaṃ kaṃdasthāne}}
       \app{\lem[wit={ceteri}]{ekā}
         \rdg[wit={N1,N2}]{eka}}
       tejorūpā mahānāḍī
       \app{\lem[wit={ceteri}]{vartate}
         \rdg[wit={U2}]{pravartate}}/
%-----------------------
% \om                                                            \B
%iyam ekanāḍī /  iḍāpiṃgalāsuṣumṇā      etān bhedān prāpnoti /    \E
%iyaṃ ekanāḍī   iḍāpiṃgalāsuṣumṇā      etān bhedān prāpnoti      \P
%trayaṃ kā nāḍī iḍāpiṃgalāsuṣumnā //   etān bhedān prāpnoti      \L
%iyaṃ ekā nāḍī  iḍāpiṃgalāsuṣumnān /   ete  bhedān prāpnoti      \N1
%iyaṃ ekā nāḍī  iḍāpiṃgalāsuṣumnān//   ete  bhedān prāpnoti/     \N2
%iyaṃ ekā nāḍī  iḍāpiṃgalasuṣumnān //  ete  bhedān prāpnoti      \D1 
%iyaṃ ekā nāḍī  iḍāpiṃgalāsuṣumnā      etān bhedān prāpnoti      \U1
%iyaṃ eka nāḍī  iḍāpiṃgalāsuṣumṇā      etān bhegān prāpnoti      \U2
%-----------------------
%This single vessel reaches to these openings which are \textit{iḍā}, \textit{piṅgalā} and \textit{suṣumnā}.
%-----------------------       
   \app{
         \lem[wit={E},alt={iyam}]{iyam\skm{-e}}
         \rdg[wit={ceteri}]{iyaṃ}
         \rdg[wit={L}]{trayaṃ}}\app{
         \lem[wit={ceteri}, alt={ekā}]{\skp{-e}kā}
         \rdg[wit={E,P}]{eka}
         \rdg[wit={L}]{kā}}
       nāḍī iḍāpiṅgalā\app{\lem[wit={N1,N2,D1},alt={°suṣumṇān}]{suṣumṇān}
    \rdg[wit={ceteri}]{suṣumṇā}}
       \app{
         \lem[wit={ceteri}]{etān}
    \rdg[wit={N1,N2,D1}]{ete}}
  bhedān prāpnoti/
\end{prose}
\end{ekdosis}
%%%%%%%%%%%%%%%%%%%%%%%%
%%%%%%%%%%%%%%%%%%%%%
%%%%%%%%%%%%%%%%%%%%%
\begin{ekdosis}
      \ekddiv{type=ed}
      \begin{prose}
%-----------------------
%\om                                           \B
%vāmabhāge candrarūpā iḍā nāḍī varttate /      \E
%vāmabhāge caṃdrarūpā iḍā nāḍī varttate        \P
%vāmabhāge caṃdrarūpā iḍā nāḍī varttate //     \L
%vāmabhāge caṃdrarūpā iḍā nāḍī varttate /      \N1
%vāmabhāge caṃdrarūpā iḍā nāḍī varttate //     \N2
%vāmabhāge caṃdrarūpā iḍā nāḍī varttate /      \D1 
%vāmabhāge caṃdrarūpā iḍā nāḍī vartate         \U1
%vāmabhāge caṃdrarūpā     nāḍī pravarttate //  \U2
%-----------------------
%On the left side is the \textit{iḍā}-channel, being a resemblence of the moon.
%-----------------------        
        vāmabhāge candrarūpā
        \app{\lem[wit={ceteri}]{iḍā}
          \rdg[wit={U2}]{\om}} nāḍī
        \app{\lem[wit={ceteri}]{vartate}
          \rdg[wit={U2}]{pravarttate}}/
%-----------------------
% \om                                                \B
%dakṣiṇabhāge sūryarūpā piṅgalā  nāḍī    varttate /  \E
%dakṣiṇabhāge sūryarūpā piṃgalā  nāḍī    varttate    \P
%dakṣiṇabhāge sūryarūpā piṃgalā  nāḍī    varttate // \L
%dakṣiṇabhāge sūryarūpā piṃgalā  nāḍī    varttate // \N1
%dakṣiṇabhāge sūryarūpā piṃgalā  nāḍī    varttate/   \N2
%dakṣiṇabhāge sūryarūpā piṃgalā  nāḍī    varttate // \D1 
%dakṣiṇe bhāge sūryarūpā piṃgalā nāḍī    vartate     \U1
%dakṣiṇabhāge sūryarūpā piṃgalā  nāḍī pravartate //  \U2
%-----------------------
%On the right side exists the \textit{piṅgalā}-channel, being a resemblence of the sun.        
%-----------------------
        \app{\lem[wit={ceteri}]{dakṣiṇabhāge}
          \rdg[wit={U1}]{dakṣiṇe bhāge}}
        sūryarūpā piṅgalā nāḍī
        \app{\lem[wit={ceteri}]{vartate}
          \rdg[wit={U2}]{pravarttate}}/
%-----------------------
% \om                                                                   \B
%madhyamārge `tisūkṣmā padminī taṃtusamākārā  koṭividyutsamaprabhā      \E
%madhyamārge `tisūkṣmā padmanī taṃtusamākāra! koṭividyutsamaprabhā      \P
%madhyamārge `tisūkṣmā padmanī taṃtusamākārā  koṭividyutsamaprabhā      \L
%madhyamārge atisūkṣmā padmanī taṃtusamākārā  koṭividyutsamaprabhā //   \N1
%madhyamārge atisūkṣmā padmanī taṃtusamākārā  koṭividyutsamaprabhā //   \N2
%madhyarge   atisūkṣmā padminī taṃtusamākārā  koṭividyutsamaprabhā //   \D1 
%madhyamārge atisūkṣmā padminī taṃtusamākārā  koṭividyutsamaprabaḥ      \U1
%madhyamārge  tisūkṣmā padminī taṃtusamākārā  koṭividyutsamaprabhā //   \U2
%-----------------------
%Within the middle path is a lotuspond being very subtle. [It is] made from a web of light [and it] shines like a thousand lightnings.
%----------------------- 
        \app{\lem[wit={ceteri}]{madhyamārge}
          \rdg[wit={D1}]{madhyarge}}
        'tisūkṣmā
        \app{\lem[wit={ceteri}]{padminī}
          \rdg[wit={P,L,N1,N2}]{padmanī}}/
        \app{\lem[wit={ceteri}]{tantusamākārā}
          \rdg[wit={P}]{taṃtusamākāra°}}
        \app{\lem[wit={ceteri},alt={°prabhā}]{koṭividyutsamaprabhā}
          \rdg[wit={U1}]{°prabhaḥ}}/
      \end{prose}
    \end{ekdosis}
    %%%%%%%%%%%%%%%
    %%%%%%%%%%%%%%%%
    %%%%%%%%%%%%%%%
    %%%%%%%%%%%%
    %%%%%%%%%%%
    \begin{ekdosis}   
     \ekddiv{type=ed}
      \begin{prose}
%-----------------------
%\om                                                                                                                                                                 \B
%bhuktimuktipradā                                     'syā jñānotpattau satyaṃ puruṣaḥ sarvajño  bhavati      idānīṃ suṣumṇāyāṃ jñānotpattāv---upāyāḥ kathyante      \E
%bhuktimuktidā                                        asyā jñānotpattau satyāṃ puruṣaḥ sarvajño  bhavati      idānīṃ suṣumṇāyā  jñānotpattau   upāyāḥ kathyaṃte      \P
%bhuktimuktipradā //                                  asyā jñānotpattau satyāṃ puruṣaḥ sarvajño  bhavati   // idānīṃ suṣumnā    jñānotpattau   upāyaḥ kathyate //    \L
%bhuktimukti--------------------------------------------------dotpanne  sati---puruṣaḥ sarrvajño bhavati    / idānīṃ suṣumnāyāḥ jñanotpanno    'pāyāḥ kathyaṃte //   \N1
%bhuktimukti--------------------------------------------------dotpanne  sati---puruṣaḥ sarrvajño bhavati    / idānīṃ suṣumnāyāḥ jñanotpanno    upāyāḥ kathyaṃte //   \N2
%bhuktimukti--------------------------------------------------dotpanne  sati---puruṣaḥ sarrvajño bhavati    / idānīṃ suṣumnāyāḥ jñanotpattau   upāyāḥ kathyaṃte //   \D1 
%bhuktimukti--------------------------------------------------dotpanne  sati---puruṣaḥ sarrvajño bhavati    / idānīṃ suṣumnāya-jñanotpattau    upāyāḥ kathyaṃte //   \U1
%bhuktimuktidā śivarūpiṇī suṣumṇā nāḍī pravarttate // asyā jñānotpattau satyāṃ puruṣa--sarvajño  bhavati   // idānīṃ suṣumṇāyā  jñānotpattau   upāyā  kathyaṃte //   \U2
%-----------------------
  \app{\lem[wit={P,U2}]{bhuktimuktidā}
  \rdg[wit={ceteri}]{bhuktimuktido°}
  \rdg[wit={E,L}]{bhuktimuktipradā}}
   % \rdg[wit={U2}]{bhuktimuktidā śivarūpiṇī suṣumṇā nāḍī pravarttate}} %Lesart oder einfach zusätzliches Material? 
%\textcolor{red}{śivarūpiṇī suṣumṇā nāḍī pravarttate/}
\extra{śivarūpiṇī suṣumṇā nāḍī pravarttate/}
\app{\lem[resp=egoscr, type=emendation]{'syāṃ}
      \rdg[wit={E}]{\korr 'syā}
      \rdg[wit={P,L,U2}]{asyā}
      \rdg[wit={ceteri}]{\om}}
    \app{\lem[wit={E,P,L,U2}]{jñānotpattau}
      \rdg[wit={ceteri}]{°tpanne}}
    \app{\lem[wit={P,L,U2}]{satyāṃ}
      \rdg[wit={E}]{satyaṃ}
      \rdg[wit={ceteri}]{sati}}
    sarvajño bhavati/ idānīṃ
    \app{\lem[wit={E}]{suṣumṇāyāṃ}
      \rdg[wit={P,U2}]{suṣumṇāyā}
      \rdg[wit={U1}]{suṣumnāya°}
      \rdg[wit={N1,N2,D1}]{suṣumṇāyāḥ}
      \rdg[wit={L}]{suṣumnā°}}
    \app{\lem[wit={E}, alt={jñānotpattāv upāyāḥ}]{jñānotpattāv-upāyāḥ}
      \rdg[wit={ceteri}]{jñānotpattau upāyāḥ}
      \rdg[wit={U2}]{jñānotpattau upāyā}
      \rdg[wit={N1,N2}]{jñānotpanno 'pāyāḥ}}
    \app{\lem[wit={E,P,N1,N2,D1,U1,U2}]{kathyante}
      \rdg[wit={L}]{kathyate}}\dd{}
  \end{prose}
\end{ekdosis}
%%%%%%%%%%%%% 
%%%%%%%%%%%%%%
%%%%%%%%%%%%%%
%%%%%%%%%%%%%%%
\begin{ekdosis}
     \ekddiv{type=ed}
     \bigskip
    \centerline{\textrm{\small{[Description of the first Cakra]}}}
    \bigskip
    \begin{prose}
%-----------------------
%\om                                            \B
%ādau caturdalaṃ mūlaṃ cakraṃ varttate /        \E
%ādau caturddalaṃ mūlaṃ cakraṃ varttate /       \P
%ādau caturdalamūlacakraṃ varttate //           \L
%ādau caturdalaṃ mūlacakraṃ varttate            \N1
%ādau prathamacaturdalamūlacakraṃ pravarttate// \N2      
%ādau caturdalaṃ mūlacakraṃ varttate            \D1 
%ādau caturdalaṃ mūlaṃ cakraṃ vartate           \U1
%ādau caturdalaṃ mūlacakraṃ pravarttate //      \U2
%-----------------------
%At the beginning\footnote{Supposedly at the beginning of the central channel.} exists the root-cakra having four petals.     
%-----------------------      
      ādau \app{\lem[wit={N1,D1,U2}]{caturdalaṃ mūlacakraṃ}
        \rdg[wit={E,P,U1}]{caturdalaṃ mūlaṃ cakraṃ}
        \rdg[wit={L}]{caturdalamūlacakraṃ}
        \rdg[wit={N2}]{prathamacaturdalamūlacakraṃ}}
      \app{\lem[wit={ceteri}]{vartate}
        \rdg[wit={U2}]{pravartate}}/
    \end{prose}
   \end{ekdosis}
 %%%%%%%%%%%%%%%%%
 %%%%%%%%%%%%%%%%%
   %%%%%%%%%%%%%%%%%
   \begin{ekdosis}    
     \ekddiv{type=ed}
     \begin{prose}     
%-----------------------
%
%\om                                       \B
%prathamādhāracakraṃ varttate / gudāsthānaṃ    raktavarṇaṃ    gaṇeśadaivataṃ    siddhibuddhiśaktimuṣakavāhanam       kurmaṛṣiḥ /  ākuṃcamudrā /    apānavāyuḥ                                   caturdaleṣu     rajaḥsattvatamomanāṃsi /  vaṃ śaṃ ṣaṃ saṃ    madhyatrikoṇe triśikhāt    tanmadhye trikoṇākāraṃ kāmapīthaṃ varttate//    \E
%prathamaṃ ādhāracakraṃ         gudāsthānaṃ    raktavarṇaṃ    gaṇeśāṃ daivataṃ  siddhibuddhiśaktir mukhako vāhanam   kurmaṛṣiḥ    ākuṃcanamudrā    apānavāyuś-----------------------------------caturddaleṣu    rajaḥsattvatamomanāṃsi    vaṃ śaṃ ṣaṃ saṃ    madhyatrikoṇe triśikhā     tanmadhye trikoṇākāraṃ kāmapīthaṃ varttate //   \P
%prathamaṃ ādhāracakraṃ         gudāsthānaṃ    raktavarṇaṃ    gaṇeśadaivataṃ    siddhibuddhiśaktimuṣako vāhanaṃ //   kurmaṛṣiḥ    ākuṃcanamudrā    apānavāyuḥ                                   caturddaleṣu    rajaḥsattvatamomanāṃsi // vaṃ śaṃ ṣaṃ saṃ    madhyatrikoṇe triśikhā     tanmadhyatrikoṇākāraṃ kāmapīthaṃ vartate        \L
%---------------------------------------------------------------------------------------------------------------------------------------------------------------------------------------------------------------------------------------------------------------------------------------tanmadhyatrikoṇākāraṃ kāmapiṭhaṃ varttate /   \N1
%---------------------------------------------------------------------------------------------------------------------------------------------------------------------------------------------------------------------------------------------------------------------------------------tanmadhye trikoṇākāraṃ kāmapiṭhaṃ varttate /   \N2
%---------------------------------------------------------------------------------------------------------------------------------------------------------------------------------------------------------------------------------------------------------------------------------------tanmadhye trikoṇākāraṃ kāmapiṭhaṃ varttate /  \D1 
%---------------------------------------------------------------------------------------------------------------------------------------------------------------------------------------------------------------------------------------------------------------------------------------tanmadhye trikoṇākāraṃ kāmapiṭhaṃ varttate /   \U1
%prathamaṃ ādhāracakraṃ         gudāsthānaṃ // raktavarṇaṃ // gaṇeśadaivataṃ // siddhibuddhiśaktiḥ muṣako vāhanaṃ // kurmaṛṣiḥ // ākuṃcanamudrā // apānavāyu // urmīkalā // ojasvinīdhāraṇā // caturddaleṣu // rajaḥsattvatamomanāṃsi //  vaṃ śaṃ ṣaṃ saṃ // madhyatrikoṇe trirekhā //  tanmadhye trikoṇākāraṃ kāmapīthaṃ varttate //   \U2
%-----------------------
%The first cakra of support (\textit{ādhāra}) is at the anus [and] is red-colored. Gaṇeśa is the deity. He is success, intelligence and power. A rat is the mount. The Ṛṣi is Kūrma. The seal is contraction. The vitalwind is \textit{apāna}. The \textit{kalā} is the wave of consciousness (\textit{urmī}). The concentration is ``she who is powerful'' (\textit{ojasvinī})}. In the four petals [of it resides] \textit{rajas}, \textit{sattva}, \textit{tamas} and the mind-faculties (\textit{manāṃsi}), [symbolized by the syllables or \textit{bīja}s] vaṃ śaṃ ṣaṃ and saṃ. A trident is situated in the middle of the triangle\footnote{This passage is odd since a triagle wasn't mentioned before.}
%-----------------------
        \extra{
          \app{\lem[wit={P,L,U2}]{prathamaṃ ādhāracakraṃ}
                 \rdg[wit={E}]{prathamādhāracakraṃ vartate}}/
                 gudā sthānaṃ\dd{}
                 \app{\lem[type=emendation, resp=egoscr]{raktaṃ}
                   \rdg[wit={E,P,L,U2}]{\korr rakta°}}varṇaṃ\dd{}
            \app{\lem[type=emendation, resp=egoscr]{gaṇeśaṃ daivataṃ}
                 \rdg[wit={E,L,U2}]{\korr gaṇeśadaivataṃ}
                 \rdg[wit={P}]{gaṇeśāṃ daivataṃ}}\dd{}
            \app{\lem[type=emendation, resp=egoscr]{siddhibuddhiśaktiṃ muṣako vāhanaṃ} %Emendation!!!
                 \rdg[wit={E}]{\korr °śaktimuṣakavāhanam}
                 \rdg[wit={P}]{°śaktir mukhako vāhanam}
                 \rdg[wit={L}]{°śaktimuṣako vāhanaṃ}
                 \rdg[wit={U2}]{°śaktiḥ muṣako vāhanaṃ}}\dd{}
            \app{\lem[type=emendation, resp=egoscr]{kūrma} %%sandhi aḥ vor ṛ wird zu a + ṛ 
                 \rdg[wit={U2}]{\korr kurma}}ṛṣiḥ\dd{}
            \app{\lem[type=emendation, resp=egoscr]{ākuñcanaṃ mudrā}
                 \rdg[wit={P,L,U2}]{ākuñcana°}
                 \rdg[wit={E}]{\korr ākuṃca°}}mudrā\dd{}
            \app{\lem[type=emendation, resp=egoscr]{apānaḥ vāyuḥ}
                 \rdg[wit={E,L}]{\korr apānavāyuḥ}
                 \rdg[wit={P}]{°vāyuś}
                 \rdg[wit={U2}]{°vāyu}}\dd{}
               \extra{
                 \app{\lem[type=emendation, resp=egoscr]{ūrmī}
                   \rdg[wit={U2}]{\korr urmī}} kalā\dd{}
                 ojasvinī dhāraṇā\dd{}}
                 caturdaleṣu rajaḥsattvatamomanāṃsi\dd{}
                 vaṃ śaṃ ṣaṃ saṃ\dd{} madhyatrikoṇe
            \app{\lem[wit={P,L}]{triśikhā}
                 \rdg[wit={E}]{triśikhāt}
                 \rdg[wit={U2}]{trirekhā}}\dd{}}
           \end{prose}
         \end{ekdosis}
         \ekdpb*{}
        %%%%%%%%%%%%%%%%%
        %%%%%%%%%%%%%%%%%
        %%%%%%%%%%%%%%%%%
        %%%%%%%%%%%%%%%%%
        %%%%%%%%%%%%%%%%%
  \begin{ekdosis}
    \ekddiv{type=ed}
    \begin{prose}             
            \app{\lem[wit={ceteri}]{tanmadhye}
                 \rdg[wit={L,N1}]{tanmadhya}}
            trikoṇākāraṃ kāmapiṭhaṃ vartate/\note[type=philcomm, labelb=s10.zx, lem={prathamaṃ \ldots triśikhā}]{The whole section from \textit{prathamaṃ} to \textit{triśikhā} is missing in N1,N2,D1 and U1, but present in all other witnesses.}
%-----------------------
%\om                                                      \B
%tatpīṭhamadhye 'gniśikhākāraikā    mūrtir varttate /        \E
%tatpīṭhamadhye magniśikhākārā ekā  mūrtir varttate /      \P
%tatpīṭhamadhye   jniśikhāka!rāṇakā mūrti varttate //     \L
%tatpīṭhamadhye  agniśikhākārā ekā  mūrttir varttate //    \N1
%tatpīṭhamadhye  agniśikhākārā ekā  mūrttir varttate /     \N2
%tatpīṭhamadhye  agniśikhākārā ekā  mūrttir varttate //    \D1 
%tatpīṭhamadhye  agniśikhākārā ekā  mūrttir varttate //    \U1
%tatpīṭhamadhye  agniśikhākārā ekā  mūrttir asmi      //    \U2
%-----------------------
%In the middle of this seat (\textit{pīṭha}) exists a single form having the shape of a flame.             
%-----------------------
  tatpīṭhamadhye
\app{\lem[wit={E}]{'gniśikhākāraikā}
  \rdg[wit={ceteri}]{agniśikhākārā ekā}
  \rdg[wit={P}]{magniśikhākārā ekā}
  \rdg[wit={L}]{jñiśikhākarāṇakā}}
murti\skp{r-}\app{\lem[wit={E,P,L,N1,N2,D1,U1}, alt={vartate}]{\skm{r}vartate}
  \rdg[wit={U2}]{asmi}}/
%-----------------------%
%\om                                       \B
%tasyāḥ mūrtirdhyānakāraṇāt   sakalaśāstrakāvya-nāṭakādi-sakalavāṅmayaṃ vinābhyāsena puruṣasya manomadhye sphurati,     \E
%tasyā mūrter dhyānakaraṇāt   sakalaśāstrakāvya-nāṭakādi-sakalavāṅmayaṃ vinābhyāsena puruṣasya manomadhye sphurati      \P
%tasyā mūrtir dhyānakāraṇāt   sakalaśāstrakāvya-nāṭakādi //----vāṅmayaṃ vinābhyāsena puruṣasya manomadhye sphuraṃti!    \L
%tasyāḥ mūrter dhyānakaraṇāt  sakalaśāstrakāvya-nāṭakādi-sakalavāgmayaṃ vinābhyāsena puruṣasya manomadhye sphurati      \N1
%tasyā mūrtter dhyānakaraṇāt  sakalaśāstrakāvya-nāṭakādi-sakavāgmayaṃ   vinābhyāsena puruṣasya manomadhye sphurati//    \N2
%tasyāḥ mūrter dhyānakaraṇāt  sakalaśāstrakāvya-nāṭakādi-sakalavāgmayaṃ vinābhyāsena puruṣasya manomadhye sphurati      \D1 
%tasyā  mūrtair dhyānakaraṇāt sakalaśāstrakāvya-nāṭakādi-sakalavāgmayaṃ vinābhyāsena puruṣasya manomadhye sphurati      \U1
%tasyā          dhyānakaraṇāt sakalaśāstrakāvya-nāṭakādi-sakalavāṅmayaṃ vinābhyāsena puruṣasya manomadhye sphurati // asya bahir mānaṃdā // yogānaṃdā virānaṃdā // uparamānaṃdā // ajapājapa śat // 600 // ghaṭi 9 palāni 40 // \U2 %
%-----------------------
%Trough the practice of meditation on this form the whole literature, all \textit{śāstra}s, all poems, dramas etc., everything [related to] elocution, appears in the mind of the person without [prior] learning. \extra{[Assigned to it] is external bliss, yogic bliss, heroic bliss [and] the bliss of coming to rest.}
%-----------------------
\app{\lem[wit={ceteri}]{tasyā}
    \rdg[wit={E,N1,D1}]{tasyāḥ}}
\app{\lem[wit={ceteri}, alt={mūrter}]{mūrte\skp{r}\skm{r-dhyā}}
    \rdg[wit={E,L}]{mūrtir}
    \rdg[wit={U1}]{mūrtair}
    \rdg[wit={U2}]{\om}}\skp{-dhyā}nakaraṇāt sakalaśāstrakāvyanāṭakādi
\app{\lem[wit={ceteri}, alt={°sakala}]{sakala}
    \rdg[wit={L}]{\om}
    \rdg[wit={N2}]{saka°}}\app{\lem[wit={E,P,L,U2}]{vāṅmayaṃ}
    \rdg[wit={N1,N2,D1,U1}]{vāgmayaṃ}} vinābhyāsena puruṣasya manomadhye
\app{\lem[wit={ceteri}]{sphurati}
  \rdg[wit={L}]{sphuraṃti}}/
\end{prose}
  \end{ekdosis}
  %%%%%%%%%%%%%%%%%%%%
  %%%%%%%%%%%%%%%%%%%%
  %%%%%%%%%%%%%%%%%%%%
  %%%%%%%%%%%%%%%%%%%
  \begin{ekdosis}
    \ekddiv{type=ed}
    \begin{prose}
      \extra{asya
        \app{\lem[type=emendation, resp=egoscr, alt={bahir ānandā}]{bahir\skp{-}ānandā}
          \rdg[wit={U2}]{\korr bahir mānandā}}\dd{}
        yogānandā\dd{}
        \app{\lem[type=emendation, resp=egoscr]{vīrānandā}
          \rdg[wit={U2}]{\korr virā°}}\dd{}
        uparamānandā\dd{}
        \app{\lem[type=emendation, resp=egoscr]{ajapājapaḥ śataḥ}
          \rdg[wit={}]{\korr ajapājapaśat}}\dd{} 600\dd{} ghaṭi 9 palāni 40\dd{}} 
    \end{prose}
  \end{ekdosis}
   %%%%%%%%%%%%%%%%%
   %%%%%%%%%%%%%%%%%
   %%%%%%%%%%%%%%%%%
   %%%%%%%%%%%%%%%%%
   \begin{ekdosis}    
     \ekddiv{type=ed}
     \bigskip
    \centerline{\textrm{\small{[Description of the second Cakra]}}}
    \bigskip
    \begin{prose}
%-----------------------
% \om                                       \oxford
%idānīṃ dvitīyaṃ svādhiṣṭānacakraṃ   ṣaḍdalaṃ upāyanapīṭhasaṃjñakaṃ bhavati //  \E
%idānīṃ dvitīyaṃ svādhiṣṭānacakraṃ   ṣaṭdalaṃ uḍḍīyānapīṭhaṃ saṃjñakaṃ bhavati  \P
%idānīṃ dvitīyaṃ svādhiṣṭānacakraṃ   ṣaṭdalaṃ uḍḍīyān pīṭhaṃ saṃjñakaṃ bhavati  \L
%idānīṃ dvitīyaṃ svādhiṣṭānacakraṃ   ṣaṭdalaṃ uḍyānapīṭhasaṃjñakaṃ bhavati /    \N1
%idānī  dvitīyaṃ svādhinacakraṃ      ṣaḍḍalaṃ uḍyānapīṭhasaṃjñakaṃ bhavati      \N2
%idānīṃ dvitīyaṃ svādhiṣṭānacakraṃ   ṣaṭdalaṃ uḍyāṇāpīṭhasaṃjñikaṃ bhavati //   \D1 
%idānīṃ dvitīyaṃ svādhiṣṭhānacakraṃ  ṣaṭdalaṃ uḍāganapīṭasaṃjñakaṃ bhavati      \U1
%idānīṃ dvitīye svādhiṣṭānacakraṃ // ṣaṭdalaṃ // uḍḍīyāṇapīṭhasaṃjñakaṃ bhavati // liṃgasthānaṃ // pītavarṇaṃ // pītaprabhā // rajoguṇa // brahmādevatā // vaikharīvāca // sāvitrīśaktiḥ // haṃsavāhanaṃ // vahaṇaṛṣiḥ // kāmāgniprabhā //sthūladehā // jāgradavasthā // ṛgveda // ācāryaliṃgaṃ // braṃhmasalokatāmokṣaḥ // śuddhabhumikātatvaṃ // gaṃdho viṣayaḥ // apānavāyuḥ // aṃtarmātṛkā // vaṃ bhaṃ maṃ yaṃ raṃ laṃ // bahir mātrā // kāmā // kāmākhyā // tejasī // ceṣṭṛikā // alasā // mithunā // ajapājapaḥ sahasra // 6000 //gha 0 96 pa 0 40// \U2
%-----------------------
%Now the second, the six-petalled \textit{Svādhiṣṭānacakra} known as the seat of \textit{uḍḍīyāṇa}\footnote{Discuss the term \textit{uḍḍīyāna}.}. \extra{The gender is the location. The color is yellow. The shine is yellow. \textit{Rajas} is the quality. The deity is Brahmā. The speech is \textit{vaikharī}\footnote{vaikharī f. in Kaśm. Śiv. °the 4. form of appearacne of \textit{parā}, the empirical speech sound, Utpala's Ṭīkā to Śivadṛṣṭi 2, 7. [B.]― Schmidt p. 337. Welches Buch???} (\textit{vaikharīvāca}). The power is Sāvitrī. The mount is the goose. The \textit{Rṣi} is Vahaṇa. The appearance (\textit{prabhā} is the fire of love (\textit{kāmāgni}). The body is gross, The state is that of being awake. [The Veda associated with it is] the Ṛgveda. The spiritual guide is the \textit{liṅga}. The liberation is residing in the world of Brahma. The level is the pure earth (\textit{śuddhabhumikā}). The sphere is smell. The vitalwind is \textit{apāna}. The internal alphabet [is]: vaṃ bhaṃ maṃ yaṃ raṃ laṃ. The outer alphabet?: desire, the Tīrtha of \textit{Kāmākhyā}\footnote{The Kāmākhyā is situated in Kāmarūpa on the Nīlakūṭa mountain in present day Assam. It's strange that it appears here, since Kāmarūpa appears already as the Tīrtha associated with the first \textit{cakra}.}, beauty of both\footnote{Why dual here?}, \textit{ceṣṭṛikā} (what is that?), lazy [and] copulation.}
%-----------------------      
        \app{\lem[wit={ceteri}]{idānīṃ}
            \rdg[wit={N2}]{idānī}}
        \app{\lem[wit={ceteri}]{dvitīyaṃ}
            \rdg[wit={U2}]{dvitīye}}
        \app{\lem[wit={U1}]{svādhiṣṭhānacakraṃ}
            \rdg[wit={E,P,L,N1,D1,U2}]{svādhiṣṭānacakraṃ}
            \rdg[wit={N2}]{svādhinacakraṃ}}
        \app{\lem[wit={ceteri}]{ṣaṭdalaṃ}
            \rdg[wit={E}]{ṣaḍdalaṃ}
            \rdg[wit={N2}]{ṣaḍḍalaṃ}}
        \app{\lem[wit={U2},alt={uḍḍīyāṇapīṭha°}]{uḍḍīyāṇapīṭha}
            \rdg[wit={E}]{upāyanapīṭha°}
            \rdg[wit={L}]{uḍḍīyān pīṭhaṃ}
            \rdg[wit={N1,N2}]{uḍyānapīṭha°}
            \rdg[wit={D1}]{uḍyāṇāpīṭha°}
            \rdg[wit={U1}]{uḍāganapīṭa°}}saṃjñakaṃ bhavati/         
        \end{prose}
      \end{ekdosis}
      %%%%%%%%%%%%%%%%
      %%%%%%%%%%%%%%%
      %%%%%%%%%%%%%%%%
      %%%%%%%%%%%%%%%
      %%%%%%%%%%%%%%%
  \begin{ekdosis}
    \ekddiv{type=ed}
    \begin{prose}       
      \extra{\app{\lem[type=emendation, resp=egoscr]{liṅgaṃ}
          \rdg[wit={U2}]{\korr liṅga°}} sthānaṃ\dd{}
        \app{\lem[type=emendation, resp=egoscr]{pītaṃ}
          \rdg[wit={U2}]{\korr pīta°}} varṇaṃ\dd{}
        \app{\lem[type=emendation, resp=egoscr]{pītā}
          \rdg[wit={U2}]{\korr pīta°}} prabhā\dd{}
        rajo \app{\lem[type=emendation, resp=egoscr]{guṇaḥ}
          \rdg[wit={U2}]{\korr guṇa}}\dd{}
        brahmā devatā\dd{}
        vaikharī \app{\lem[type=emendation, resp=egoscr]{vāk}
          \rdg[wit={U2}]{\korr vāca}}\dd{}
        sāvitrī śaktiḥ\dd{}
        \app{\lem[type=emendation, resp=egoscr]{haṃso}
          \rdg[wit={U2}]{\korr haṃsa°}} vāhanaṃ\dd{}
        \app{\lem[type=emendation, resp=egoscr]{vahaṇo}
          \rdg[wit={U2}]{\korr vahaṇa}} ṛṣiḥ\dd{}
        \app{\lem[type=emendation, resp=egoscr, alt={kāmāgnir}]{kāmāgni\skp{r-}}
          \rdg[wit={U2}]{\korr kāmāgni°}}\skm{r-}prabhā\dd{}
        \app{\lem[type=emendation, resp=egoscr]{sthūlo dehaḥ}
          \rdg[wit={U2}]{\korr sthūladehā}}\dd{}
        jāgrad avasthā\dd{}
        \app{\lem[type=emendation, resp=egoscr]{ṛg vedaḥ}
          \rdg[wit={U2}]{\korr ṛg veda}}\dd{}
        \app{\lem[type=emendation, resp=egoscr]{ācāryaḥ}
          \rdg[wit={U2}]{\korr ācārya°}} liṅgaṃ\dd{}
        brahmasalokatā mokṣaḥ\dd{}
        \app{\lem[type=emendation, resp=egoscr]{śuddhabhumikā}
          \rdg[wit={U2}]{\korr śuddhabhumikā}} tattvaṃ\dd{}
        gaṃdho viṣayaḥ\dd{}
        \app{\lem[type=emendation, resp=egoscr]{apānaḥ}
          \rdg[wit={U2}]{apāna°}} vāyuḥ\dd{}
        aṃtar\skp{-}mātṛkā\dd{}
        vaṃ bhaṃ maṃ yaṃ raṃ laṃ\dd{}
        bahir-mātrā\dd{}
        kāmā\dd{}
        kāmākhyā\dd{}
        \app{\lem[type=emendation, resp=egoscr]{tejasvinī}
          \rdg[wit={U2}]{\korr tejasī}}\dd{}
        ceṣṭikā\dd{}
        alasā\dd{}
        mithunā\dd{}
        ajapājapaḥ \app{\lem[type=emendation, resp=egoscr]{sahasraḥ}
          \rdg[wit={U2}]{\korr sahasra}}\dd{} 6000\dd{} gha. 16 pa. 40\dd{}}
    \end{prose}
  \end{ekdosis}
  %%%%%%%%%%%%%%%
  %%%%%%%%%%%%%%%
  %%%%%%%%%%%%%%%
  %%%%%%%%%%%%%%%
  %%%%%%%%%%%%%%%
  \begin{ekdosis}
    \ekddiv{type=ed}
    \begin{prose}      
%-----------------------
%
% \om                                        \B
%tanmadhye atiraktavarṇaṃ tejo varttate /    \E
%tanmadhye 'tiraktavarṇaṃ tejo varttate      \P
%tanmadhye  tiraktavarṇaṃ tejo varttate //   \L
%tanmadhye  atiraktavarṇaṃ tejo varttate     \N1
%tanmadhye  atiraktavarṇatejo varttate      \N2
%tanmadhye  atiraktavarṇaṃ tejo varttate     \D1 
%tanmadhye  atiraktavarṇatejo varttate       \U1
%tanmadhye 'tiraktavarṇaṃ tejo vartate //    \U2
%-----------------------
%In its middle exists extremely red glow. The adept becomes very handsome by meditation on it.       
%-----------------------          
          tanmadhye \app{\lem[wit={P,U2}]{'tiraktavarṇaṃ}
            \rdg[wit={ceteri}]{atiraktavarṇaṃ}
         \rdg[wit={U1,N2}]{atiraktavarṇa°}}
       tejo vartate/
%-----------------------
% \om                                          \B
%tasya dhyānāt sādhako 'tisundaro bhavati /    \E
%tasya dhyānāt sādhako   tisuṃdaro bhavati      \P
%tasya dhyānāt sādhako   tisuṃdaro bhavati //   \L
%tasya dhyānāt sādhakaḥ  atisuṃdaro bhavati // \N1
%tasya dhyānāt sādhakaḥ  atisuṃdaro bhavati/   \N2
%tasya dhyānāt sādhakaḥ  atisuṃdaro bhavati // \D1 
%tasya dhyānāt sādhakaḥ  atisuṃdaro bhavati    \U1
%tasya dhyānāt sādhako  'tisundaro bhavati //   \U2
%-----------------------
%The adept becomes very handsome through meditation on it.
%-----------------------       
tasya dhyānāt \app{\lem[wit={E,P,L,U2}]{sādhako}
  \rdg[wit={ceteri}]{sādhakaḥ}}
\app{\lem[wit={E,P,L,U2}]{'tisundaro}
  \rdg[wit={ceteri}]{atisuṃdaro}} 
bhavati/ 
%-----------------------
% \om                                  \B
%                                pratidinam-āyur vardhate /             \E
%                                pratidinam-āyur vardhate               \P
%                                pratidinam-āyur vardhate //2//         \L
%                                dinaṃ dinaṃ prati āyurvarddhate // //  \N1
%yuvatīnāṃ ativallabho? bhavati dinadinaṃ prati āyur varddhate//        \N2  %%%3verso
%                                dinaṃ prati āyurvarddhate //2//        \D1 
%                                dinaṃ dinaṃ prati āyurvarddhate        \U1
%                                pratidinaṃ āyur varddhate //           \U2
%-----------------------
%\extra{He becomes one who is very desired by virgins.} The vital force increases from day to day. \end{tlate}
%-----------------------
\extra{yuvatīnāṃ ativallabho bhavati}/\note[type=philcomm, labelb=s10.z5aa, lem={yuvatīnāṃ\ldots bhavati}]{This additional sentence occurs in N2 only.}
\app{\lem[wit={ceteri}, alt={pratidinam}]{pratidinam\skm{-ā}}
  \rdg[wit={N1,U1}]{dinaṃ dinaṃ prati}
  \rdg[wit={N2}]{dinadinaṃ prati}
  \rdg[wit={D1}]{dinaṃ prati}}\skp{-ā}yur-vardhate\dd{}
    \end{prose}
  \end{ekdosis}
    %%%%%%%%%%%%%%
    %%%%%%%%%%%%%%%
    %%%%%%%%%%%%%%
    %%%%%%%%%%%%%%
    %%%%%%%%%%%%%%
\begin{ekdosis}
 \ekddiv{type=ed}
  \bigskip
    \centerline{\textrm{\small{[Description of the third Cakra]}}}
    \bigskip
 \begin{prose}
%-----------------------
% \om                                                 \B
%tṛtīye                      nābhisthāne    daśadalaṃ padmaṃ vartate      \E
%tṛtīyaṃ                     nābhisthāne    daśadalaṃ padmaṃ vartate     \P
%tṛtīyaṃ                     nābhisthāne // daśadalapadme vartate        \L
%tṛtīyaṃ                     nābhisthāne    daśadalaṃ padma varttate //  \N1
%tṛtīyacakraṃ                nābhisthāne    daśadalaṃ padma varttate /  \N2
%tṛtīyaṃ                     nābhisthāne    daśadalaṃ padma varttate //  \D1 
%tṛtīyaṃ                     nābhisthāne    daśadalakaṃ padmaṃ varttate   \U1
%atha tṛtīyaṃ maṇipūracakraṃ nābhisthāne // kapilavarṇaṃ // viṣṇudevatā // lakṣmīśaktiḥ // vāyuṛṣiḥ // samānavāyuḥ // garuḍavāhanaṃ // sūkṣmaliṃgadevatāha // svapnāvasthā // madhyamāvāk // yajurvedaḥ // dakṣināgniḥ // samipatāmokṣaḥ // guruliṃgaviṣṇuḥ // āpastatvaṃ // rajoviṣayaḥ daśadalāni // daśamātrāḥ // aṃtarmātrā // ḍaṃ ṭaṃ ṇaṃ taṃ thaṃ daṃ dhaṃ naṃ paṃ phaṃ // bahirmātrāḥ // śāṃtiḥ // kṣamā // medhā // tanyā // medhāvinī // puṣkarā // ahaṃsagamanā // lakṣyā //tanmayā // amṛtā // ajapājapa // 6000 gha 016 pa 040 //    \U2
%-----------------------
%\extra{The colour is red (\textit{kapila}). Viṣṇu is the deity. Lakṣmī is the power. Vāyu is the Rṣi. Samāna is the vitalwind. The mount is Garuḍa. The deity is the suble body\footnote{Why another deity is given here?}. The state is sleep. The speech is the inaudible speech (\textit{madhyamāvāg})\footnote{<Śā, Ling>name of the speech which is inaudible and which is of the type of a thought without any definite presence of words making up the expression. Vkp I.143.<Abhyankar 1986: 300>}. The Veda is the Yajurveda. The [fire is the] southern fire. The liberation is ``proximity'' (\textit{samīpatā}).\footnote{What is this exactly?}. Viṣṇu is the characteristic of the teacher (\textit{guruliṅga}). The principle is water. The sphere is athmosphere (\textit{rajo viṣaya}). There are ten petals [and] ten matrices. [The] inner matrix: \textit{ḍaṃ ṭaṃ ṇaṃ taṃ thaṃ daṃ dhaṃ naṃ paṃ phaṃ}. The external matrix : peace, patience, insight, the ``daughter''\textit{tanayā}, the ``learned teacher'', the ``lotus'', \textit{haṃsagamanā}, the ``fixation object'', absorption and immortality.} 
%-----------------------
    \app{\lem[wit={ceteri}]{tṛtīyaṃ}
      \rdg[wit={E}]{tṛtīye}
      \rdg[wit={U2}]{atha tṛtīyaṃ maṇipūracakraṃ}
      \rdg[wit={N2}]{tṛtīyacakraṃ}}
    nābhisthāne
    \app{\lem[wit={ceteri}]{daśadalaṃ}
      \rdg[wit={L}]{daśadala°}
      \rdg[wit={U1}]{daśadalakaṃ}
      \rdg[wit={U2}]{\om}}
    \app{\lem[wit={E,P,U1}]{padmaṃ}
      \rdg[wit={L}]{padme}
      \rdg[wit={N1,N2,D1}]{padma}
      \rdg[wit={U2}]{\om}}
    \app{\lem[wit={ceteri}]{vartate}
      \rdg[wit={U2}]{\om}}/
\end{prose}
\end{ekdosis}
\ekdpb*{}
      %%%%%%%%%%%%
      %%%%%%%%%%%%%%%
      %%%%%%%%%%%%%%
      %%%%%%%%%%%%%
      %%%%%%%%%%%%
  \begin{ekdosis}
    \ekddiv{type=ed}
    \begin{prose}    
      \extra{
        \app{\lem[type=emendation, resp=egoscr]{kapilaṃ}
          \rdg[wit={U2}]{\korr kapila°}} varṇaṃ\dd{}
        \app{\lem[type=emendation, resp=egoscr]{viṣṇur}
          \rdg[wit={U2}]{\korr viṣṇu}} devatā\dd{}
        lakṣmī śaktiḥ\dd{}
        \app{\lem[type=emendation, resp=egoscr, alt={vāyur}]{vāyu\skp{rṛ-}}
          \rdg[wit={U2}]{\korr vayu°}}\skm{-rṛ}ṣiḥ\dd{}
        \app{\lem[type=emendation, resp=egoscr]{samāno}
          \rdg[wit={U2}]{\korr samāna°}} vāyuḥ\dd{}
        \app{\lem[type=emendation, resp=egoscr]{garuḍo}
          \rdg[wit={U2}]{\korr garuḍa°}} vāhanaṃ\dd{}
      \app{\lem[type=emendation, resp=egoscr]{sūkṣmaliṅgaṃ devatā}
        \rdg[wit={U2}]{\korr sūkṣmaliṅgadevatāha}}\dd{}
      \app{\lem[type=emendation, resp=egoscr,alt={svapnā avasthā}]{svapnā-avasthā}
        \rdg[wit={U2}]{\korr svapnāvasthā}}\dd{}
      madhyamā vāk\dd{}
      yajur-vedaḥ\dd{}
      \app{\lem[type=emendation, resp=egoscr]{dakṣiṇo 'gniḥ}
        \rdg[wit={U2}]{\korr dakṣināgniḥ}}\dd{}
      \app{\lem[type=emendation, resp=egoscr]{samīpatā}
        \rdg[wit={U2}]{samipatā}} mokṣaḥ\dd{}
      \app{\lem[type=emendation, resp=egoscr]{guruliṅgo}
        \rdg[wit={U2}]{\korr guruliṅga°}} viṣṇuḥ\dd{}
      āpas tattvaṃ\dd{}
      rajo viṣayaḥ\dd{}
      daśadalāni\dd{}
      daśamātrāḥ\dd{}
      antar-mātrā\dd{}
      ḍaṃ ṭaṃ ṇaṃ taṃ thaṃ daṃ dhaṃ naṃ paṃ phaṃ\dd{}
      bahir-mātrāḥ\dd{}
      śāṃtiḥ\dd{}
      kṣamā\dd{}
      medhā\dd{}
      tanayā\dd{}
      medhāvinī\dd{}
      puṣkarā\dd{}
      \app{\lem[type=emendation, resp=egoscr]{haṃsagamanā}
        \rdg[wit={U2}]{\korr ahaṃsagamanā}}\dd{}
      lakṣyā\dd{}
      tanmayā\dd{}
      amṛtā\dd{}
      ajapājapaḥ \app{\lem[type=emendation, resp=egoscr]{sahasraḥ}
        \rdg[wit={U2}]{\korr sahasra}}\dd{} 6000\dd{} gha. 16 pa. 40\dd{}}   
\end{prose}
  \end{ekdosis}
  %%%%%%%%%%%%%%%%%%%
  %%%%%%%%%%%%%%%%%%%%
  %%%%%%%%%%%%%%%%%%
  %%%%%%%%%%%%%%%%%%
  %%%%%%%%%%%%%%%%%%
    \begin{ekdosis}
      \ekddiv{type=ed}
      \begin{prose}
%-----------------------
% \om                                       \B
%tanmadhye paṃcakoṇaṃ cakraṃ varttate//    \E
%tanmadhye paṃcakoṇaṃ cakraṃ varttate       \P
% \om  \L
%tanmadhye paṃcakoṇaṃ cakraṃ varttate//    \N1
%tanmadhye paṃcakoṇaṃ cakraṃ varttate/    \N2
%tanmadhye paṃcakoṇaṃ cakraṃ varttate//    \D1 
%tanmadhye paṃcakoṇaṃ cakraṃ varttate       \U1
%tanmadhye paṃcakoṇaṃ cakraṃ vartate//     \U2
%-----------------------
% In its middle exists a \textit{cakra} with five angles.
%-----------------------
tanmadhye pancakoṇaṃ cakraṃ vartate/ \note[type=philcomm, labelb=s14.z5, lem={tanmadhye ... cakraṃ vartate}]{This sentence is \om L.}
%-----------------------
% \om                                  \B
%tanmadhye ekā mūrtir vartate/        \E
%tanmadhye ekā mūrtir vartate          \P
%\om                                   \L
%tanmadhye ekā mūrttir varttate //     \N1
%tanmadhye ekā mūrttir varttate/     \N2
%tanmadhye ekā mūrttir varttate//     \D1 
%tanmadhye ekā mūrtir vartate          \U1
%tanmadhye ekā mūrtir asmi//          \U2
%-----------------------
%In its middle is a single (divine) form. 
%-----------------------
tanmadhye ekā mūrti\skp{r-}\app{\lem[wit={ceteri}, alt={vartate}]{\skm{r}vartate}
  \rdg[wit={U2}]{asmi}}/ \note[type=philcomm, labelb=s14.z6, lem={tanmadhye ... mūrtir vartate}]{This sentence \om in L.}
%-----------------------
% \om                                          \B
%tasyās tejo jihvayā kathayituṃ na śakyate /   \E
%tasyās tejo jihvayā kathayituṃ na śakyate     \P
%tasyās tejo jihvayā kathyituṃ na śakyate      \L
%tasyā tejo jihvayā kathayituṃ na śakyate //   \N1
%tasyā tejo jihvayā kathayituṃ na śakyate/   \N2
%tasyā tejo jihvayā kathayituṃ na śakyate //   \D1 
%tasyāstejo jihvayā kathatuṃ na śakyate        \U1
%tasyāstejo jihvayā vaktuṃ na śakyate //       \U2
%-----------------------
%It's not possible to describe her shine with speech (lit. with the tongue).
%-----------------------
 \app{\lem[wit={ceteri}, alt={tasyās}]{tasyā\skp{s-}}
    \rdg[wit={N1,N2,D1}]{tasyā}}\skm{s}tejo jihvayā
 \app{\lem[wit={ceteri}]{kathayituṃ}
    \rdg[wit={L}]{kathyituṃ}
    \rdg[wit={U1}]{kathatuṃ}
    \rdg[wit={U2}]{vaktuṃ}}
  na śakyate/
%-----------------------
% \om                                                                    \B
%tasyāḥ mūrter dhyānakāraṇāt    puruṣasya śarīraṃ sthiraṃ bhavati //     \E
%tasyā  mūrter dhyānakaraṇāt    -------------------------------------    \P
%tasyā  mūrtir dhyānakaraṇāt // puruṣasya śarīraṃ sthiram bhavati //     \L
%tasyāḥ mūrter dhyānakaraṇāt    puruṣasya śarīraṃ sthiraṃ bhavati /      \N1
%tasyāḥ mūrter dhyānakaraṇāt    puruṣasya śarīraṃ sthiraṃ bhavati//      \N2
%tasyāḥ mūrter dhyānakaraṇāt    puruṣasya śarīraṃ sthiraṃ bhavati /      \D1 
%tasyāḥ mūrter dhyānakaraṇāt    puruṣasya śarīraṃ sthiraṃ bhavati vā     \U1
%tasyāḥ dhyānakaraṇāt           puruṣasya śarīraṃ sthiraṃ bhavati //     \U2
%-----------------------
%Through the execution of meditation on this (divine) form the body of the person is going to be strong.   
%-----------------------
 tasyāḥ
  \app{\lem[wit={ceteri}, alt={mūrter}]{mūrte\skp{r-}}
      \rdg[wit={L}]{mūrtir}
      \rdg[wit={U2}]{\om}}\skm{r-}dhyāna\app{\lem[wit={ceteri}, alt={°karaṇāt}]{karaṇāt}
      \rdg[wit={E}]{°kāraṇāt}}
    \app{\lem[wit={ceteri}]{puruṣasya śarīraṃ sthiraṃ} %%%%Kolloquium: So ein langes Lemma sinnvoll? 
    \rdg[wit={P}]{\om}}
  \app{\lem[wit={ceteri}]{bhavati}
  \rdg[wit={U1}]{bhavati vā}
  \rdg[wit={P}]{\om}}\dd{}
 \end{prose}
\end{ekdosis}
%%%%%%%%%%%%%%%%
%%%%%%%%%%%%%%%
%%%%%%%%%%%%%%%
%%%%%%%%%%%%%%
%%%%%%%%%%%%%%%
\begin{ekdosis}
  \ekddiv{type=ed}
   \bigskip
    \centerline{\textrm{\small{[Description of the fourth Cakra]}}}
    \bigskip
  \begin{prose}
%-----------------------
% \om                                                   \B
%caturthaṃ hṛdayamadhye dvādaśadalaṃ kamalaṃ vartate/   \E
%caturthaṃ hṛdayamadhye dvadaśadalaṃ kamalaṃ varttate/  \P
%caturthaṃ hṛdayamadhye dvadaśadalaṃ kamalaṃ varttate/  \L
%caturthaṃ hṛdayamadhye dvadaśadalaṃ kamalaṃ varttate/  \N1
%caturthacakrakamalaṃ hṛdayamadhye dvadaśadalaṃ bhavati \N2    
%caturthaṃ hṛdayamadhye dvadaśadalaṃ kamalaṃ varttate   \D1 
%caturthaṃ hṛdayamadhye dvadaśadalaṃ kamalaṃ varttate/  \U1   
%caturthaṃ hṛdayamadhye dvadaśadalaṃ kamalam asti/      \U2
%
% anāhatacakraṃ hṛdayasthānaṃ // śvetavarṇaṃ tamoguṇaḥ // rudrodevatā // umāśaktiḥ // hiraṇyagarbhaṛṣiḥ // naṃdivāhanaṃ // prāṇavāyuḥ // jyotiḥ kalākāraṇaṃ dehe // suṣuptir avasthā // paśyaṃtivācā // sāmavedaḥ // gārhasyatyogniḥ? // śivaliṇgaṃ // prāptibhūmikā // sarūpatāmuktiḥ // dvādaśādalāni //dvādaśamātrā // kaṃ khaṃ gaṃ ghaṃ ṇaṃ caṃ chaṃ jaṃ jhaṃ yaṃ taṃ thaṃ // bahirmātrā // rudrāṇī // tejasā // tāpinī // spha?kadā // caitanyā // śivadā // śānti // umā // gaurī // mātara // jvālā // prajvālinī // ajapājapasahasra // 6000 gha. 96 pa. 40 // U2
%-----------------------
%The fourth lotus having twelve-petals exists in the middle at the heart. \extra{[The] Anāhatacakras place is within the heart\footnote{This seems to be redundant.}. The color is white. The quality is \textit{tamas}. The deity is Rudra. The power is Umā. The Ṛṣi is Hiraṇyagarbha. The mount is Nandi. The vitalwind is Prāṇa. In the body it is the light that causes parts (\textit{kalākaraṇa})\footnote{What is this?!}. The state is deep sleep. The speech is \textit{Paśyantī}\footnote{Add footnote of entry in \textit{Tāntrikābhidhānakośa}.}.The [Veda] is Sāmaveda. The fire is Gārhapatya\footnote{Add explanation.}. The Liṅgam is Śivaliṅga. The ability to attain everything on the earth [and] the uniform liberation [are attributed to this \textit{cakra}]. [There are] twelve petals, [and] twelve measures: kaṃ khaṃ gaṃ ghaṃ ṇaṃ caṃ chaṃ jaṃ jhaṃ yaṃ taṃ [and] thaṃ. The external measure: Rudra's wife, light (\textit{tejasā?}), glow, \textit{sphakadā}?, consciousness (\textit{caitanyā}), bestower of grace, peace, Umā, Gaurī, Mātara, the flame [and] Prajvālinī.}
%-----------------------
    \app{\lem[wit={ceteri}]{caturthaṃ}
       \rdg[wit={N2}]{caturthacakrakamalaṃ}} hṛdayamadhye dvādaśadalaṃ
    \app{\lem[wit={ceteri}]{kamalaṃ}
       \rdg[wit={N2}]{\om}} 
    \app{\lem[wit={ceteri}]{vartate}
       \rdg[wit={U2}]{asti}
       \rdg[wit={N2}]{bhavati}}/
%      \end{prose}
%       \end{ekdosis}
       %%%%%%%%%%%%%%%%%
       %%%%%%%%%%%%%%%%
       %%%%%%%%%%%%%%%%%%
       %%%%%%%%%%%%%%%%%
       %%%%%%%%%%%%%%%%
%  \begin{ekdosis}
%    \ekddiv{type=ed}
%    \begin{prose} %%%fastmark 
      \extra{anāhatacakraṃ hṛdayasthānaṃ\dd{}
        \app{\lem[type=emendation, resp=egoscr]{śvetaṃ}
          \rdg[wit={U2}]{\korr śveta°}} varṇaṃ\dd{}
        tamo guṇaḥ\dd{}
        rudro devatā\dd{}
        umā śaktiḥ\dd{}
        hiraṇyagarbha ṛṣiḥ\dd{}
        nandi vāhanaṃ\dd{}
        \app{\lem[type=emendation, resp=egoscr]{prāṇo}
          \rdg[wit={U2}]{\korr prāṇa°}} vāyuḥ\dd{}
        \app{\lem[type=emendation, resp=egoscr]{jyotiskalākāraṇaṃ deham}
          \rdg[wit={U2}]{\korr jyotiḥ kalākāraṇaṃ dehe}}\dd{}
        suṣuptir\skp{-}avasthā\dd{}
        \app{\lem[type=emendation, resp=egoscr]{paśyantī}
          \rdg[wit={U2}]{\korr paśyaṃti}} vācā\dd{}
        sāmavedaḥ\dd{}
        \app{\lem[type=emendation, resp=egoscr]{gārhapatyo 'gniḥ}
          \rdg[wit={U2}]{\korr gārhasyatyo gniḥ}}\dd{}
        \app{\lem[type=emendation, resp=egoscr]{śivo}
          \rdg[wit={U2}]{\korr śiva°}} liṅgaṃ\dd{}
        \app{\lem[type=emendation, resp=egoscr]{prāptiḥ}
          \rdg[wit={U2}]{\korr prāpti°}} bhūmikā\dd{}
        sarūpatā muktiḥ\dd{}
        dvādaśādalāni\dd{}
        dvādaśamātrā\dd{}
        kaṃ khaṃ gaṃ ghaṃ ṇaṃ caṃ chaṃ jaṃ jhaṃ yaṃ taṃ thaṃ\dd{}
        bahir\skp{-}mātrā\dd{}
        rudrāṇī\dd{}
        tejasā\dd{}
        tāpinī\dd{}
        sukhadā\dd{}
        caitanyā\dd{}
        śivadā\dd{}
        \app{\lem[type=emendation, resp=egoscr]{śāntiḥ}
          \rdg[wit={U2}]{\korr śānti}}\dd{}
        umā\dd{}
        gaurī\dd{}
        \app{\lem[type=emendation, resp=egoscr]{mātarā} %%%?????
          \rdg[wit={U2}]{\korr mātara}}\dd{}
        jvālā\dd{}
        prajvālinī\dd{}
        \app{\lem[type=emendation, resp=egoscr]{ajapājapaḥ}
          \rdg[wit={U2}]{\korr ajapājapaḥ}} \app{\lem[type=emendation, resp=egoscr]{sahasraḥ}
          \rdg[wit={U2}]{\korr sahasra}}\dd{} 6000\dd{} gha. 96 pa. 40\dd{}}
    \end{prose}
  \end{ekdosis}
  %%%%%%%%%%%%%
  %%%%%%%%%%%%%%
  %%%%%%%%%%%%%
  %%%%%%%%%%%%%%
  %%%%%%%%%%%%%%%
\begin{ekdosis}
    \ekddiv{type=ed}
    \begin{prose}
%-----------------------
% \om                                        \B
%atitejomayatvād   dṛṣṭigocaraṃ na bhavati   \E  
%atitejomayatvāt   dṛṣṭigocaraṃ na bhavati   \P
%atitejomayatvād   dṛṣṭigocaraṃ na bhavati// \L
%atitejomayatvāt / dṛṣṭigocaraṃ na bhavati/ \N1
%atitejomayatvāt   dṛṣṭigocaraṃ na bhavati/ \N2
%atitejomayatvāt / dṛṣṭigocaraṃ na bhavati/ \D1
%atitejomayatvāt / dṛṣṭigocaraṃ na bhavati/ \U1
%atitejomayatvād   dṛṣṭigocaratāṃ na yāti// \U2 
%-----------------------
%Due to being made of [such an] intense light [the fourth lotus] is not in the range of sight.
%-----------------------
atitejomayatvād-dṛṣṭi\app{\lem[wit={ceteri}, alt={°gocaraṃ}]{gocaraṃ}  %SANDHI einbauen?! 
                         \rdg[wit={U2}]{gocaratāṃ}}
na
    \app{\lem[wit={ceteri}]{bhavati}
      \rdg[wit={U2}]{yāti}}/   
%-----------------------
% \om                                               \B
%tanmadhye 'ṣṭadalam adhomukhaṃ kamalaṃ varttate // \E  
%tanmadhye 'ṣṭadale  mukhaṃ kamalaṃ varttate //     \P
%tanmadhye ṣṭadalaṃ  adhomukha--kamalaṃ vartate //  \L
%tanmadhye aṣṭadalaṃ adhomukhaṃ kamalaṃ vartate //  \N1
%tanmadhye aṣṭadalaṃ adhomukhaṃ kamalaṃ varttate//  \N2
%tanmadhye aṣṭadalaṃ adhomukhaṃ kamalaṃ vartate //  \D1
%tanmadhye aṣṭadalaṃ adhomukhaṃ kamalaṃ vartate /   \U1
%tanmadhye 'ṣṭadalaṃ adhomukhaṃ kamalaṃ asti / manaś-cakre// manodevatā// bahiśaktiḥ// ātmaṛṣih// nābhimadhye sthitaṃ padmaṃ nālaṃ tasya daśāgulaṃ/ komalaṃ tasya tan nālaṃ nirmalaṃ cāpy adhomukhaṃ/ kadalīpuṣpasaṃkāśaṃ tanmadhye ca pratiṣṭhitaṃ/ mana unnaty-asaṃkalpa/ vikalpātmakam-eva ca/ pūrvadale svetavarṇe yadā viśrāmate manaḥ// dharmakīrtividyādi sadbuddhir-bhavati/ agnikoṇe āraktavarṇe nidrā ālasyamāyāmandamatir-bhavati/ dakṣiṇe kṛṣṇavarṇeti tadā krodhotpattir bhavati/ naiṛtye nīlavarṇe mamatāmatir bhavati/ paścime kapilavarṇe/ krīḍāhāsotsavotsāhamatir bhavati/ vāyav ye śāmavarṇe cintodvegamatir bhavati/ uttare pītavarṇe bhogaśṛṇgāramahodayamatir bhavati/ īśāne gauravarṇe jñānasaṃdhāne matir bhavati/} \U2
%-----------------------
    tanmadhye \app{\lem[wit={ceteri},alt={'ṣṭadalam}]{'ṣṭadalam\skm{a}}
      \rdg[wit={P}]{'ṣṭadale}
      \rdg[wit={L}]{ṣṭadalaṃ}
      \rdg[wit={N1,N2,D1,U1}]{aṣṭadalaṃ}}\app{\lem[wit={ceteri},alt={adhomukhaṃ kamalaṃ}]{\skp{-a}dhomukhaṃ kamalaṃ}
        \rdg[wit={L}]{adhomukhakamalaṃ}
        \rdg[wit={P}]{mukhaṃ kamalaṃ}}
      \app{\lem[wit={ceteri}]{vartate}
        \rdg[wit={U2}]{asti}}/    
\end{prose}
\end{ekdosis}
%%%%%%%%%%%%%%%%
%%%%%%%%%%%%%%%
%%%%%%%%%%%%%%%
%%%%%%%%%%%%%%
%%%%%%%%%%%%%%%
  \begin{ekdosis}
     \ekddiv{type=ed}
\begin{prose}
  \extra{manaś-cakre\dd{}
    mano devatā\dd{}
        \app{\lem[type=conjecture, resp=egoscr]{bahiśśaktiḥ}
          \rdg[wit={U2}]{\conj bahiśaktiḥ}}\dd{}   
        \app{\lem[type=emendation, resp=egoscr]{ātmā}
          \rdg[wit={U2}]{\korr ātma°}} ṛṣiḥ\dd{}
        nābhimadhye sthitaṃ padmaṃ nālaṃ tasya
        \app{\lem[type=emendation, resp=egoscr]{daśāṅgulaṃ}
          \rdg[wit={U2}]{\korr daśāgulaṃ}}/ %In the middle of the navel [exists] a place, being a lotus, its tube measures ten \textit{aṅgula}s,
        komalaṃ tasya tan-nālaṃ nirmalaṃ cāpy-adhomukhaṃ/ %The fluid (\textit{komala}) of the tube is pure facing upwards.
        kadalīpuṣpasaṃkāśaṃ tanmadhye ca pratiṣṭhitaṃ/ % In its middle is a place shining like a banana-flower.
        mana \app{\lem[type=conjecture, resp=egoscr, alt={ānati}]{āna\skp{ty-a}}
          \rdg[wit={U2}]{\conj unnaty}}
      \app{\lem[type=emendation, resp=egoscr,alt={asaṃkalpam}]{\skm{ty-a}saṃkalpam}
          \rdg[wit={U2}]{\korr asaṃkalpa}}/  
        vikalpātmakam-eva ca/} %The mind isn't willing to rise up and is of changing nature.
\end{prose}
\end{ekdosis}
\ekdpb*{}
  %%%%%%%%%%
  %%%%%%%%%%
%%%%%%%%%%%%
%%%%%%%%%%%%%%
%%%%%%%%%%%%%%%
  \begin{ekdosis}
    \ekddiv{type=ed}
    \begin{prose}
        \extra{
          pūrvadale \app{\lem[type=emendation, resp=egoscr, alt={°śveta}]{śveta}
            \rdg[wit={U2}]{\korr sveta°}}varṇe yadā \app{\lem[type=emendation, resp=egoscr]{viśramate}
            \rdg[wit={U2}]{\korr viśrāmate}} manaḥ\dd{}
        dharmakīrtividyādisadbuddhir-bhavati/ %While the mind rests on the eastern petal [which is] white in colour clear intellekt arises, which is [endowed with]  \textit{dharma}, fame and knowledge etc. 
        %%%%%
        agnikoṇe āraktavarṇe \app{\lem[type=emendation, resp=egoscr, alt={nidrālasya}]{nidrālasya}
          \rdg[wit={U2}]{\korr nidrā ālasya°}}māyāmandamatir-bhavati/  %While [the mind rests on] the south-east, [which is] reddish in color a mind that is weak due to sleep, laziness and illusion arises.
        %%%%
        dakṣiṇe kṛṣṇavarṇeti tadā krodhotpattir-bhavati/ %While [the mind is situated] in the right south, [which is] black in color the generation of anger arises.
        %%%
        \app{\lem[type=emendation, resp=egoscr]{nairṛtye}
          \rdg[wit={U2}]{\korr naiṛtye}} nīlavarṇe mamatāmatir-bhavati/ %While [the mind is situated] in the southwest, [which is] blue in color a mind of pride arises.
        %%%
        paścime kapilavarṇe krīḍāhāsotsavotsāhamatir-bhavati/ %While [the mind is situated] in the west, [which is] brown in color a mind that is longing for play, laughing, and celebration arises.
        %%%
        vāyavye \app{\lem[type=emendation, resp=egoscr, alt={°śyāma}]{śyāma}
          \rdg[wit={}]{\korr śāma}}varṇe cintodvegamatir-bhavati/ %While [the mind is situated] in the northwest, [which is] dark in color a mind which is restless by sorrow arises.
        %%%
        uttare pītavarṇe bhogaśṛṅgāramahodayamatir-bhavati/ %While [the mind is situated] in the north, [which is] yellow in color a very happy mind with erotic and enjoyment arises.
        īśāne gauravarṇe
        \app{\lem[type=emendation, resp=egoscr, alt={jñānasaṃdhāna°}]{jñānasaṃdhāna}
          \rdg[wit={U2}]{\korr jñānasaṃdhāne}}
        matir-bhavati/} %While [the mind is situated] in north-east [which is] whitish in color a mind of unity arises through knowledge arises.
\end{prose}
\end{ekdosis}
  %%%%%%%%%%%%
  %%%%%%%%%%%%
  %%%%%%%%%%%%
  %%%%%%%%%%%%
  %%%%%%%%%%%%
    \begin{ekdosis}
      \ekddiv{type=ed}
      \begin{prose}
%-----------------------
% \om                                                     \B      
%tanmadhye prāṇavāyoḥ sthānam    aṣṭadalakamalamadhye liṃgākārā karṇikā  kathyate/  \E 
%tanmadhye prāṇavāyoḥ sthānam    aṣṭadalakamalamadhye liṃgākārā karṇikā  kathyate/  \P
%tanmadhye prāṇavāyoḥ sthānam    aṣṭadalakamalamadhye liṃgākārā karṇikā  kathyate// \L
%tanmadhye prāṇavāyoḥ sthānam    aṣṭadalakamalamadhye liṃgākārā karṇikā  kathyate// \N1
%tanmadhye prāṇavāyoḥ sthānam/   aṣṭadalakamalamadhye liṃgākārā karṇikā  kathyate// \N2
%tanmadhye prāṇavāyoḥ sthānam // aṣṭadalakamalamadhye liṃgākārā karṇi    kathyate// \D1
%tanmadhye prāṇavāyo  sthānam    aṣṭadalakamalamadhye liṃgākārā karṇikā  kathyate    \U1
%tanmadhye prāṇavāyo  sthānam // aṣṭadalakamalamadhye liṃgākārā karṇikā  kathyate    \U2
%-----------------------
%It's said that in its middle is the place of the \textit{prāṇa}-vitalwind [and] in the middle [of] the eight-petalled lotus is a pericarp (\textit{karṇikā}) in the form of a \textit{liṅga}.
%-----------------------
        tanmadhye prāṇavāyoḥ sthānam-aṣṭadalakamalamadhye liṃgākārā
        \app{\lem[wit={ceteri}]{karṇikā}
          \rdg[wit={U2}]{karṇi}}
        kathyate/   
%-----------------------
% \om                                                     \B
%tasyāḥ karṇiketi saṃjñā tatkarṇikāmadhye padmarāgasamānavarṇāṃ guṣṭhapramāṇaikā puttalikā varttate //          \E  
%tasyāḥ kaliketi saṃjñā tatkalikāmadhye   padmarāgaratnasamānavarṇāṃ aṃguṣṭhapramāṇā ekā puttalikā varttate     \P
%tasyāḥ kalikeli                 madhye   padmaratnasamānavarṇā // aṃguṣṭhapramāṇā // ekā puttalikā varttate // \L
%tasyāḥ kaliketi saṃjñā tatkalikāmadhye   padmarāgaratnasamānavarṇāṃ aṃguṣṭhapramāṇā ekā puttalikā varttate     \N1
%tasyāḥ kaliketi saṃjñā/tataḥ kalikāmadhye   padmarāgaratnasamānavarṇa aṃguṣṭhapramāṇā ekā putalikā varttate/   \N2 %%%p4recto
%tasyāḥ kaliketi saṃjñā tatkalikāmadhye   padmarāgaratnasamānavarṇā aṃguṣṭhapramāṇāt ekā puttalikā varttate /   \D1
%tasyāḥ kaliketi saṃjñā tatkalikāmadhye   padmarāgaratnasamānavarṇā aṃguṣṭhapramāṇāt ekā puttalikā varttate /   \U1
%tasyāḥ kaliketi saṃjñā tatkalikāmadhye   padmarāgaratnasamānavarṇā  // aṃguṣṭhapramāṇā ekā puttalikā varttate / \U2
%-----------------------
%The technical designation of her is kalikā. In the middle of this kalikā exists a single thumbsized (divine) figurine (puttalikā) being similiar to a ruby-gem in color.
%-----------------------        
tasyāḥ \app{\lem[wit={ceteri}]{kaliketi}
  \rdg[wit={L}]{kalikeli}
  \rdg[wit={E}]{karṇiketi}}
\app{\lem[wit={ceteri}]{saṃjñā}
  \rdg[wit={L}]{\om}}
\app{\lem[wit={ceteri}]{tatkalikāmadhye}
  \rdg[wit={N2}]{tataḥ}
  \rdg[wit={L}]{\om}}
\app{\lem[type=emendation, resp=egoscr]{padmarāgaratnasamānavarṇāṅguṣṭhapramāṇaikā}
  \rdg[wit={E}]{\korr padmarāgasamānavarṇāṃguṣṭhapramāṇaikā}
  \rdg[wit={P,N1}]{padmarāgaratnasamānavarṇāṃ || aṃguṣṭhapramāṇā || ekā}
  \rdg[wit={N2}]{padmarāgaratnasamānavarṇa aṃguṣṭhapramāṇā ekā}
  \rdg[wit={L}]{padmaratnasamānavarṇā aṃguṣṭhapramāṇā ekā}
  \rdg[wit={D1,U1}]{padmarāgaratnasamānavarṇā aṃguṣṭhapramāṇāt ekā}
  \rdg[wit={U2}]{padmarāgaratnasamānavarṇā || aṃguṣṭhapramāṇā ekā}} puttalikā vartate/   
%-----------------------
%
%tasyā  jīvasaṃjñā          tasyā  balamadhyasvarūpaṃ        koṭijihvābhir  vaktuṃ naiva śakyate // \E
%tasyā  jīvasaṃjñā          tasyā  balam atha svarūpaṃ       koṭijihvābhir  vaktuṃ naiva śakyate // \P 
%tasya                              bala sappa svarūpaṃ       koṭijihvāyābhi vaktuṃ na    śakyate // \L 
%tasyāḥ jīveti saṃjñāḥ      tasyāḥ balaṃ atha ca svarūpaṃ    koṭijihvābhir  vaktuṃ na    śakyate // \N1
%tasyāḥ jīveti saṃjñaḥ//    tasyā  balaṃ atha ca svarūpaṃ    koṭijihvābhir  vaktuṃ na    śakyate // \N2
%tasyāḥ jīveti saṃjña/      tasyāḥ balaṃ atha ca svarūpaṃ    koṭijihvābhir  vaktuṃ na    śakyate // \D1
%tasyāḥ jīveti saṃjñā       tasyāḥ balaṃ atha ca svarūpaṃ    koṭijihvābhir  vaktuṃ na    śakyate // \U1
%tasyā  jīvasaṃjñā//        tasya  balaṃ tasya atha svarūpaṃ koṭijihvābhir  vaktuṃ na    śakyate // \U2
%-----------------------  
%Her technical designation is embodied soul. Not even with a thousand tongues it is possible to talk about her nature and her power.
%-----------------------
\app{\lem[wit={E,P}]{tasyā}
     \rdg[wit={N1,N2,D1,U1}]{tasyāḥ}
     \rdg[wit={L}]{tasya}}
\app{\lem[wit={U2}]{jīveti saṃjñā}
     \rdg[wit={N1}]{jīveti saṃjñāḥ}
     \rdg[wit={N2}]{jīveti saṃjñaḥ}
     \rdg[wit={D1}]{jīveti saṃjña}
     \rdg[wit={E,P,U2}]{jīvasaṃjñā}
     \rdg[wit={L}]{\om}}
\app{\lem[wit={E,N2,P}]{tasyā}
     \rdg[wit={N1,D1,U1}]{tasyāḥ}
     \rdg[wit={U2}]{tasya}}
\app{\lem[wit={ceteri}]{balaṃ atha ca svarūpaṃ}
     \rdg[wit={P}]{balam atha svarūpaṃ}
     \rdg[wit={U2}]{balaṃ tasya atha svarūpaṃ}
     \rdg[wit={L}]{bala sappa svarūpaṃ}
     \rdg[wit={E}]{balamadhyasvarūpaṃ}}
\app{\lem[wit={ceteri}, alt={koṭijihvābhir}]{koṭijihvābhi\skp{r-}\skm{r-va}}
    \rdg[wit={L}]{koṭijihvāyābhi}}\skp{-va}ktuṃ
\app{\lem[wit={ceteri}]{na}
    \rdg[wit={E,P}]{naiva}}
  śakyate/
%-----------------------
%asyā  mūrter   dhyānakāraṇāt      svarga-pātāl--ākaśamanuṣyagandharvakinnaraguhyakavidyādharalokasambandhinyaḥ strīyo 'pi--------------------       vaśyā bhavanti / \E
%asyā  mūrter   dhyānakaraṇāt      svarga-pātāl--ākāśamanuṣyagandharvakiṃnaraguhyakavidyādharalokasaṃbaṃdhinyaḥ strīyo 'pi--------------------       vaśyā bhavanti / \P
%asyā  mūrtir   dhyānāt            svarga-pātāl--ākāśamanuṣyagaṃdharvakinnaraguhyakavidyādharalokasambandhinyaḥ strīyo 'pi--------------------       vaśyā bhavanti /L
%asyāḥ mūrter  dhyānakaraṇāt       svarga-pātāla ākāśamanuṣyagaṃdharvakinnaraguhyakavidyādharalokasaṃbaṃdhinyaḥ strīyaḥ sādhakasya puruṣasya         vaśyā bhavanti // \N1
%asyā  mūrttir dhyānakaraṇāt/      svarga-pātāla ākāśamanuṣya/ gaṃdharvakinnara/ guhyaka/vidyādhara/lokasaṃbaṃdhinyaḥ strīyaḥ sādhakasya puruṣasya   vaśyo bhavati/ \N2
%asyāḥ mūrter  dhyānakaraṇāt       svarga-pātāla ākāśamanuṣyagaṃdharvakiṃnaraguhyakavidyādharalokasaṃbaṃdhinyaḥ strīyaḥ sādhakasya puruṣasya         vaśyā bhavanti // \D1
%asyāḥ mūrter  dhyānakaraṇāt       svarga-pātāla ākāśamanuṣyagaṃdharvakiṃnaraguhyakavidyādharalokasaṃbaṃdhinyaḥ strīyaḥ sādhakasya puruṣasya         vaśyā bhavanti // \U1
%tasyāḥ mūrter dhyānaṃ karaṇāt //  svarga-pātāl--ākāśamanuṣyagandharvakinnaraguhyakavidyādharalokasaṃbadhinya---striyo  pi---------------------------vaśyā bhavaṃti // \U2
%-----------------------
%“Because of the exercise of meditation on this form the inhabitants of the universe (which are) Humans, Gandharvas, Kinnaras, Guhyakas, Vidyādharas and (their) females, in the heavenly world, underworld and open space are obedient to the will of the practicing person.”, is what is said here.  
%-----------------------
 \app{\lem[wit={ceteri}]{asyā}
    \rdg[wit={N1,D1,U1}]{asyāḥ}
    \rdg[wit={U2}]{tasyāḥ}}
 \app{\lem[wit={ceteri}, alt={mūrter}]{mūrte\skp{r-}}
    \rdg[wit={L,N2}]{mūrtir}}\app{\lem[wit={ceteri}, alt={dhyānakāraṇāt}]{\skm{r-}dhyānakāraṇāt}
    \rdg[wit={U2}]{dhyānaṃ karaṇāt}
    \rdg[wit={L}]{dhyānāt}}
  svargapātālākaśamanuṣyagandharvakinnaraguhyakavidyādharaloka\app{\lem[wit={ceteri}]{saṃbandhinyaḥ}\rdg[wit={U2}]{saṃdadhinya}}
 \app{\lem[wit={ceteri}]{strīyaḥ sādhakasya puruṣasya}
    \rdg[wit={E,P,L}]{strīyo 'pi}
    \rdg[wit={U2}]{striyo pi}}
 \app{\lem[wit={ceteri}]{vaśyā bhavanti}
   \rdg[wit={N2}]{vaśyo bhavati}}/
 %tanmadhye koṭicaṃdrasamaprabhaḥ ekaḥ puruṣo varttate  \N1bhavanti/\note[type=philcomm, labelb=s16, lem={bhavanti}]{\getsiglum{U1} adds a flawed phrase hereafter: \textit{pṛtvī lokasaṃbaṃdhanyo pi striyaḥ vaśyā bhavaṃti/}. I refrained to include it in the apparatus due to its redundance.}
%-----------------------
%ityatra kathyate// /E
%ityatra kathyate// \P
%ityatra kathyate// \L
%ityatra kiṃ kathyate // \N1
%ityatra kiṃ kathyate// \N2
%ityaṃtra kiṃ kathyate // \D1
%ityatra kiṃ kathyate vā \U1
%ityatra kathyate // \U2
%-----------------------
%is what is said here.  
%-----------------------  
ityatra \app{\lem[wit={ceteri}]{kiṃ}
  \rdg[wit={E,P,L,U2}]{\om}}
\app{\lem[wit={ceteri}]{kathyate}
  \rdg[wit={U1}]{kathyate vā}}\dd{}
  \end{prose}
\end{ekdosis}
%%%%%%%%%%%%%%%%%%
%%%%%%%%%%%%%%%%%%
%%%%%%%%%%%%%%%%%
%%%%%%%%%%%%%%%%%%
%%%%%%%%%%%%%%%%%%%
\begin{ekdosis}
  \ekddiv{type=ed}
  \bigskip
    \centerline{\textrm{\small{[Description of the fifth Cakra]}}}
    \bigskip  
    \begin{prose}
%-----------------------      
%-------pañcamaṃ kaṇṭhasthāne ṣoḍaśadalaṃ kamalaṃ      vartate //  \E
%-------paṃcamaṃ kaṃṭhasthāne ṣoḍaśadalaṃ kamalaṃ      vartate     \P
%-------paṃcamaṃ kaṃṭhasthāne ṣoḍaśadalaṃ kamalaṃ      vartate     \L
%idānīṃ paṃcamaṃ kamalaṃ      ṣodaśadalaṃ kaṃṭhasthāne varttate // \N1
%idānīṃ paṃcamaṃ kamalaṣodaśadalaṃ kaṃṭhasthāne varttate // \N2
%idānīṃ paṃcamaṃ kamalaṃ      ṣodaśadalaṃ kaṃṭhasthāne varttate // \D1 --------> Was in diesem Falle machen?
%idānīṃ paṃcamaṃ kamalaṃ      ṣodaśadalaṃ kaṃṭhasthāne varttate // \U1
%-------paṃcamaṃ viśuddhacakraṃ           kaṃṭhastāne              \U2     
%-----------------------
%Now (follows the description of) the fifth lotus having sixteen petals (which) exists at the location of the throat.
%-----------------------
%U2 continues: dhūmra?varṇe jīvodevatā// avidyāśaktiḥ// virāṭrṣiḥ// vāyurvāhanaṃ// udānavāyuḥ// jvālākalā jālaṃdharobaṃdhaḥ mahākāraṇadeha// tūryāvasthā// parāvācā// atharvaṇavedaḥ// jaṃgamaliṅgaṃ jīvaprāptābhūmikā// sāyujyatāmokṣaḥ// ṣoḍaśadalāni// ṣoḍaśamātrāḥ// atarmātrār-carāḥ// aṃ āṃ iṃ īṃ u ūṃ ṛṃ ṝṃ ḷṃ ḹṃ eṃ aiṃ oṃ auṃ aṃ aṃḥ// bahirmātrā vidyā// avidyā// ichā// śakti// jñānaśaktiḥ// śatalā// mahāvidyā// mahāmāyā// buddhiḥ// tamasī// maitrā?// kumārī// maitrāyaṇī// rudrā// puṣṭa// siṃhanī// ajapājapasahasra/ 1000 gha. 2 pa. 46 akṣara 40//
%-----------------------     
%The colour is smoke-colour. The deity is the embodied soul (\textit{jīva}). The power is ignorance (\textit{avidyā}). The Ṛṣi is Virāṭ\footnote{Who is this?}. The mount is the vitalwind (\textit{vāyu}). The vitalwind is \textit{udāna}. Its Kalā is the flame. The \textit{bandha} is Jālandhara. The body supra-causal (\textit{mahākāraṇa}). The state is the fourth state (\textit{tūrya}). The speech is Parā\footnote{Im Kaśm. Śiv. °das ewige Wort, in welchem potentiell alle Begriffe und Worte ruhen; vgl. das śabdabrahma des Vyākaraṇa. [B.]― Schmidt S. 246}. The [Veda is] Atharvaṇa Veda. The \textit{liṅga} is the living. The level is Jīvaprāptā\footnote{What is this?}. The liberation is absorption into the divine essence (\textit{sāyujyatā}). [There are] sixteen petals [and] sixteen matrices. The internal matrix: aṃ āṃ iṃ īṃ u ūṃ ṛṃ ṝṃ ḷṃ ḹṃ eṃ aiṃ oṃ auṃ aṃ aṃḥ. The external matrix: Vidyā ``she who is knowledge'', Avidyā ``she who is ignorance'', Icchā ``she who is desire'', Śakti ``she who is power'', Jñānaśakti ``she who is the power of knowledge'', Śatalā ``she who is manifold'', Mahāvidyā ``she who is great knowledge'', Mahāmayā ``she who is great illusion'', Buddhi ``she who is intellect'', Tamasī ``she who is darkness'', Maitrā ``she who is love'', Kumārī ``she who is a young girl'', Maitrāyaṇī ``she who is???'', Rudrā ``she who is howling'', Puṣṭā ``she who is abundance'', Siṃhanī ``she who is a lioness''. A thousandfold recitation of the non-recited; 1000 [repetitions for]; 2 \textit{ghaṭi}s, 46 \textit{palā}s. and 40 \textit{akṣara}s.
%-----------------------  
  \app{\lem[wit={N1,N2,D1,U1}]{idānīṃ}
\rdg[wit={ceteri}]{\om}}
pañcamaṃ
\app{\lem[wit={N1,D1,U1}]{kamalaṃ ṣodaśadalaṃ kaṇṭhasthāne}
  \rdg[wit={N2}]{kamalaṣodaśadalaṃ kaṇṭhasthāne}
  \rdg[wit={E,P,L}]{kaṇṭhasthāne ṣoḍaśadalaṃ kamalaṃ}
  \rdg[wit={U2}]{viśuddhacakraṃ kaṃṭhastāne}}
\app{\lem[wit={ceteri}]{vartate}
  \rdg[wit={U2}]{\om}}/
  \end{prose}
  \end{ekdosis}
  %%%%%%%%%%%
  %%%%%%%%%%%
  %%%%%%%%%%%
  %%%%%%%%%%%
  %%%%%%%%%%%
  \begin{ekdosis}
    \ekddiv{type=ed}
    \begin{prose}
      \extra{\app{\lem[type=emendation, resp=egoscr]{dhūmraṃ varṇaṃ}
          \rdg[wit={U2}]{\korr dhūmravarṇe}}\dd{}
        jīvo devatā\dd{}
        avidyā śaktiḥ\dd{}
        \app{\lem[type=emendation, resp=egoscr]{virāṭ}
          \rdg[wit={U2}]{\korr virāṭha}} ṛṣiḥ\dd{}
        vāyur\skp{-}vāhanaṃ\dd{}
        \app{\lem[type=emendation, resp=egoscr]{udāno}
          \rdg[wit={U2}]{\korr udāna°}} vāyuḥ\dd{}
        jvālā kalā\dd{}
        jālaṃdharo bandhaḥ\dd{}
        \app{\lem[type=emendation, resp=egoscr]{mahākāraṇaḥ dehaḥ}
          \rdg[wit={U2}]{\korr mahākāraṇadeha}}\dd{}
        \app{\lem[type=emendation, resp=egoscr]{tūrya āvasthā}
          \rdg[wit={U2}]{\korr tūryāvasthā}}\dd{}
        parā vācā\dd{}
        \app{\lem[type=emendation, resp=egoscr]{atharvaṇo}
          \rdg[wit={U2}]{\korr atharvaṇa}} vedaḥ\dd{}
        \app{\lem[type=emendation, resp=egoscr]{jaṅgamaṃ}
          \rdg[wit={U2}]{\korr jaṃgama°}} liṅgaṃ\dd{}
        jīvaprāptā bhūmikā\dd{}
        sāyujyatā mokṣaḥ\dd{}
        ṣoḍaśadalāni\dd{}
        ṣoḍaśamātrāḥ\dd{}
        \app{\lem[type=emendation, resp=egoscr]{antarmātrā}
          \rdg[wit={U2}]{\korr antarmātrār carāḥ}}\dd{}  %%%what does carā here mean? I emend to the formulation found for the U2 additions in the previous cakra 
        aṃ āṃ iṃ īṃ u ūṃ ṛṃ ṝṃ ḷṃ ḹṃ eṃ aiṃ oṃ auṃ aṃ aṃḥ\dd{}
        bahir-mātrā\dd{}
        vidyā\dd{}
        avidyā\dd{}
        \app{\lem[type=emendation, resp=egoscr]{icchā}
          \rdg[wit={U2}]{\korr ichā}}\dd{}
        \app{\lem[type=emendation, resp=egoscr]{śaktiḥ}
          \rdg[wit={U2}]{\korr śakti}}\dd{}
        jñānaśaktiḥ\dd{}
        śatalā\dd{}
        mahāvidyā\dd{}
        mahāmāyā\dd{}
        buddhiḥ\dd{}
        \app{\lem[type=emendation, resp=egoscr]{tāmasī}
          \rdg[wit={U2}]{\korr tamasī}}\dd{} %%%She who is darkness????
        maitrā\dd{}
        kumārī\dd{}
        maitrāyaṇī\dd{} %%%what's this??? 
        rudrā\dd{}
        \app{\lem[type=emendation, resp=egoscr]{puṣṭā}
          \rdg[wit={U2}]{\korr puṣṭa°}}\dd{}
        siṃhanī\dd{}
        \app{\lem[type=emendation, resp=egoscr]{ajapājapaḥ sahasraḥ}
          \rdg[wit={U2}]{\korr ajapājapasahasra}}\dd{} 1000\dd{} gha. 2 pa. 46 akṣara 40\dd{}}%%%%%Kolloquium besprechen! Was is akṣara? 
    \end{prose}
  \end{ekdosis}
  \ekdpb*{}
  %%%%%%%%%%%
  %%%%%%%%%%%
  %%%%%%%%%%%
  %%%%%%%%%%%
  %%%%%%%%%%%
  \begin{ekdosis}
      \ekddiv{type=ed}
      \begin{prose}
%----------------------- 
%tanmadhye koṭisūryasamāna       ekaḥ puruṣo vartate / \E
%tanmadhye koṭicaṃdrasamaprabhaḥ ekaḥ puruṣo vartate   \P
%tanmadhye koṭicaṃdrasamaprabhā  ekaḥ puruṣo vartate   \L
%tanmadhye koṭicaṃdrasamaprabhaḥ ekaḥ puruṣo varttate  \N1
%tanmadhye koṭicaṃdrasamaprabhaḥ ekaḥ puruṣo varttate  \N2
%tanmadhye koṭicaṃdrasamaprabhā  eka--puruṣo varttate  \D1
%tanmadhye koṭicaṃdrasamaprabhaḥ ekaḥ puruṣo varttate  \U1
%tanmadhye koṭicaṃdrasamaprabhaḥ // eka pumān varttate // \U2
%----------------------- 
%In its  middle exists a single person which shines like a thousand moons.
%----------------------- 
tanmadhye
\app{\lem[wit={ceteri}]{koṭicandrasamaprabhaḥ}
  \rdg[wit={L,D1}]{°prabhā}
  \rdg[wit={E}]{koṭisūryasamāna}}
\app{\lem[wit={ceteri}]{ekaḥ puruṣo}
  \rdg[wit=D1]{ekapuruṣo}
  \rdg[wit={U2}]{eka pumān}}
vartate/
%----------------------- 
%tasya puruṣasya dhyānakāraṇād--- asādhyarogā naśyanti // \E
%tasya puruṣasya dhyānakāraṇād--- asādhyarogā naśyanti // \L
%tasya puruṣasya dhyānakāraṇād--- asādhyarogā naśyaṃti // \P
%tasya puruṣasya dhyānakaraṇāt--  asādhyarogā naśyaṃti // \N1
%tasya puruṣasya dhyānakaraṇāt    asādhyarogā naśyaṃti    \N2
%tasya puruṣasya dhyānakaraṇāt /  asādhyarogā naśyaṃti // \D1
%tasya puruṣasya dhyānakaraṇāt /  asādhyarogā naśyaṃti    \U1
%tasya puṃsaḥ    dhyānakaraṇāt // asādhyarogā naśyaṃti // \U2
%----------------------- 
%Because of the exercise of meditation on this person all diseases which are (otherwise) not possible to be controlled vanish.
%----------------------- 
tasya
\app{\lem[wit={ceteri}]{puruṣasya}
  \rdg[wit={U2}]{puṃsaḥ}}
dhyānakaraṇād-asādhyarogā naśyanti/
%----------------------- 
%ekasahasravarṣaparyaṃtaṃ sa puruṣo jīvatīdānīṃ     \E
%ekasahasravarṣaparyaṃtaṃ sa puruṣo jīvati          \P
%ekasahasravarṣa             puruṣo jīvati //       \L
%ekasahasravarṣaparyaṃtaṃ    puruṣo jīvati /        \N1
%ekasahasravarṣaparyaṃta     puruṣo jīvati /        \N2
%ekasahasravarṣaparyaṃtaṃ    puruṣo jīvati /        \D1
%ekasahasravarṣaparyaṃtaṃ    puruṣo jīvati          \U1
%ekasahasravarṣaparyaṃtaṃ    puruṣo jīvati //       \U2
%----------------------- 
%The person lives up to 1001 years.
%----------------------- 
ekasahasravarṣa\app{\lem[wit={ceteri},alt={°paryantaṃ}]{paryantaṃ}
  \rdg[wit={N2}]{°paryaṃta}
  \rdg[wit={L}]{\om}}
\app{\lem[wit={ceteri}]{puruṣo}
\rdg[wit={E,P}]{sa puruṣo}}
  jīvati\dd{}
    \end{prose}
  \end{ekdosis}
%%%%%%%%%%%%%%%%
%%%%%%%%%%%%%%%%
%%%%%%%%%%%%%%%%
%%%%%%%%%%%%%%%
%%%%%%%%%%%%%%%%
\begin{ekdosis}
  \ekddiv{type=ed}
   \bigskip
    \centerline{\textrm{\small{[Description of the sixth Cakra]}}}
    \bigskip
 \begin{prose}
%----------------------- 
%īdānīṃ ṣaṣṭhaṃ bhrūmadhye ājñācakraṃ                vartate//   \E
%īdānīṃ ṣaṣṭhaṃ bhrūmadhye ājñācakraṃ                vartate//   \P
%īdānīṃ ṣaṣṭhaḥ bhrūmadhye ājñācakraṃ                vartate//   \L
%idānīṃ ṣaṣṭhacakraṃ       ajñānāmakaṃ               varttate // \N1
%idānīṃ ṣaṣṭhacakraṃ       ajñānāmaka                varttate    \N2
%idānīṃ ṣaṣṭhacakraṃ       ajñānāmakaṃ               varttate // \D1
%idānīṃ ṣaṣṭhacakraṃ       ājñānāmakaṃ               vartate     \U1
%idānīṃ ṣaṣṭa   bhrūmadhye ājñācakraṃ raktavarṇaṃ //             \U2
%-----------------------
%āgnirdevatā suṣumṇāśaktiḥ// hiṃsaṛṣiḥ// caitanyavāhanaṃ// jñānadehī// vijñānāvathā// anupamavācā// sāmadevaḥ// pramādaliṃgaṃ// ardhamātrā// ākāśātatvaṃ// jīvahiṃsa// caitanyalīlraṃbhaḥ// dvemātrā// hiṃkṣaṃ// aṃtarmātrā// bahirmātrā//sthiti//prabhā?// ajapājapasahasra// 1000 gha. 2 pa. 46 akṣara 40// \U2
%-----------------------
%The deity is fire. The power is the godess of the centre (\textit{suṣumṇā}). The Ṛṣi is ``the violent'' (\textit{hiṃsa}). The mount is consciousness (\textit{caitanya}. The body is knowledge. The state is understanding. The speech is the ``incomparable'' (\textit{anupama}). The [Veda] is Sāmaveda.The \textit{liṅgaṃ} is intoxication (\textit{pramāda}). The half-measure: the reality of ether, ``the violence of living'' (\textit{jīvahiṃsa}) [and] the origin of the play of Conciousness. Two measures: haṃ kṣam. The inner measure is external measure: maintenance of life (\textit{sthiti}) [and] splendour (\textit{prabhā}).
%-----------------------
   idānīṃ
    \app{\lem[wit={ceteri}]{ṣaṣṭhacakraṃ}
       \rdg[wit={E,P}]{ṣaṣṭhaṃ bhrūmadhye}
       \rdg[wit={L}]{ṣaṣṭhaḥ bhrūmadhye}
       \rdg[wit={U2}]{ṣaṣṭa bhrūmadhye}}
    \app{\lem[wit={ceteri}]{ājñā}
      \rdg[wit={N1,N2,D1}]{ajñā}
    }\app{\lem[wit={U1,D1,N1}]{nāmakaṃ}
       \rdg[wit={E,P,L}]{cakraṃ}
       \rdg[wit={U2}]{cakraṃ raktavarṇaṃ}
       \rdg[wit={N2}]{nāmaka}}
   \app{\lem[wit={ceteri}]{vartate}
     \rdg[wit={U2}]{\om}}/
\end{prose}
  \end{ekdosis}
  %%%%%%%%%%%%%%%
  %%%%%%%%%%%%%%
  %%%%%%%%%%%%%%
  %%%%%%%%%%%%%%
  %%%%%%%%%%%%%%
  \begin{ekdosis}
    \ekddiv{type=ed}
    \begin{prose}
     \extra{\app{\lem[type=emendation, resp=egoscr, alt={agnir}]{agni\skp{r-de}}
         \rdg[wit={U2}]{\korr āgnir}
       }\skm{r-de}vatā\dd{}
       suṣumṇā śaktiḥ\dd{}
       \app{\lem[type=emendation, resp=egoscr]{hiṃso}
         \rdg[wit={U2}]{\korr hiṃsa°}} ṛṣiḥ\dd{}
       \app{\lem[type=emendation, resp=egoscr]{caitanyaṃ}
         \rdg[wit={U2}]{\korr caitanya°}} vāhanaṃ\dd{}
       \app{\lem[type=emendation, resp=egoscr]{jñāno dehaḥ}
         \rdg[wit={U2}]{\korr jñānadehī}}\dd{}
       vijñānāvasthā\dd{}
       \app{\lem[type=emendation, resp=egoscr]{anupamā}
         \rdg[wit={U2}]{\korr anupama°}} vācā\dd{}
       sāmavedaḥ\dd{}
       \app{\lem[type=emendation, resp=egoscr]{pramādaḥ}
         \rdg[wit={U2}]{\korr pramāda°}} liṃgaṃ\dd{}
       \app{\lem[type=emendation, resp=egoscr]{ardhā mātrā}
         \rdg[wit={U2}]{\korr ardhamātrā}}\dd{}
       \app{\lem[type=emendation, resp=egoscr]{ākāśaṃ}
         \rdg[wit={U2}]{\korr ākāśā}} tattvaṃ\dd{}
       \app{\lem[type=emendation, resp=egoscr]{jīvo haṃsaḥ}
         \rdg[wit={U2}]{\korr jīvahiṃsa}}\dd{}
       caitanya\app{\lem[type=emendation, resp=egoscr, alt={°līlā}]{līlā āraṃbhaḥ}
         \rdg[wit={U2}]{\korr °līlāraṃbhaḥ}}\dd{}
       dve mātrā\dd{}
       haṃ kṣaṃ\dd{}
       aṃtar-mātrā\dd{}
       bahir-mātrā\dd{}
       \app{\lem[type=emendation, resp=egoscr]{sthitiḥ}
         \rdg[wit={U2}]{\korr sthiti}}\dd{}
       prabhā\dd{}
       \app{\lem[type=emendation, resp=egoscr]{ajapājapaḥ sahasraḥ}
         \rdg[wit={U2}]{\korr ajapājapasahasra}}\dd{} 1000\dd{} gha. 2 pa. 46 akṣara 40\dd{}}
\end{prose}
  \end{ekdosis}
  %%%%%%%%
  %%% %%%%%%%%%%
  %%%%%%%%%%%
  %%%%%%%%%%%
  %%%%%%%%%%%%
    \begin{ekdosis}
      \ekddiv{type=ed}
      \begin{prose}     
%----------------------- 
                                       %dvidalaṃ tanmadhye  'gnijvālākārakamalaṃ     kiṃcid vastu vartate/    \E
                                       %dvidalaṃ tanmadhye  agnijvālākārakamalaṃ     kiṃcid vastu vartate/    \P
                                       %dvidalaṃ tanmadhye  agnijvālākārakamalaṃ     kiṃcid vastu vartate/    \L
%                                                           agnijvālākārakamalaṃ     kiṃcid vastu vartate/    \B
%tac cakraṃ bhruvor madhye dvidalakaṃ sthitaṃ // tanmadhye  agnijvālākāraṃ akalaṃ    kiṃcid vastu varttate/   \N1
%tac-cakraṃ bhruvor-madhye dvidalakaṃ sthitaṃ /  tanmadhye  agnijvālākāraṃ akalaṃ    kiṃcid-vastu vartate/    \N2
%tac cakraṃ bhruvor madhye dvidalakaṃ sthitaṃ // tanmadhye  agnijvālākāraṃ akalaṃ    kiṃcid vastu varttate/   \D1
%tac-cakraṃ bhruvor-madhye dvidalakaṃ sthitaṃ    tanmadhye  agnijvālākāraṃ akala     kiṃcit vastu vartate/    \U1  
%                                                tanmadhye  agnijvālākārakamalaṃ //  kiṃcid-vastu varttate/ \U2   
%-----------------------    
 \app{\lem[wit={ceteri}, alt={tac cakraṃ bhruvor madhye dvidalakaṃ sthitaṃ}]{tac-cakraṃ bhruvor-madhye dvidalakaṃ sthitaṃ}
     \rdg[wit={E,P,L}]{dvidalaṃ}
     \rdg[wit={U2}]{\om}}
   tanmadhye
   \app{\lem[wit={N1,N2,D1}]{'gnijvālākāraṃ akalaṃ}
     \rdg[wit={ceteri}]{agnijvālākāraṃ akalaṃ}
     \rdg[wit={U1}]{agnijvālākāraṃ akala}}
   \note[type=philcomm, labelb=s20.z11a, lem={agnijvālākāra°}]{Witness B starts here.}
   kiṃcid-vastu vartate/
%-----------------------  
%na strī pumān     / tasya dhyānakāraṇāt  puruṣasya  śarīraṃ  ajarāmaraṃ bhavati /     \E
%na strī pumān    // tasyā dhyānakaraṇāt  puruṣasya  śarīraṃ  ajarāmaro  bhavati /     \B
%na strī pumān    // tasyā dhyānakaraṇāt  puruṣasya  śarīraṃ  ajarāmaro  bhavati /     \L
%na strī na pumān // tasyā dhyānakaraṇāt  puruṣasya  śarīraṃ  ajarāmaro  bhavati /     \P
%na strī na pumān /  tasya dhyānakaraṇāt  puruṣasya  śarīraṃ  ajarāmaraṃ bhavati      \N1
%na strī na pumān /  tasya dhyānakaraṇāt  puruṣasya  śarīraṃ  ajarāmaraṃ bhavati //   \N2
%na strī na pumān /  tasya dhyānakaraṇāt  puruṣasya  śarīraṃ  ajarāmaraṃ bhavati      \D1
%na strī na pumān    tasya dhyānakaraṇāt  puruṣasya  śarīraṃ  ajarāmaraṃ bhavati vā   \U1
%na strī na pumān /  tasya dhyānakāraṇāt/ puruṣasya--śarīram--ajarāmaraṃ bhavati /    \U2   
%-----------------------
   na strī
   \app{\lem[wit={ceteri}]{na pumān}
     \rdg[wit={E,B,L}]{pumān}}/
   puruṣasya \app{\lem[wit={ceteri}, alt={°ajarāmaraṃ}]{śarīramajarāmaraṃ}
     \rdg[wit={B,L,P}]{°ajarāmaro}}
   \app{\lem[wit={ceteri}]{bhavati}
     \rdg[wit={U2}]{bhavati vā}}\dd{}   
 \end{prose}
\end{ekdosis}
%%%%%%%%%%%%%%%
%%%%%%%%%%%%%%%
%%%%%%%%%%%%%%%
%%%%%%%%%%%%%%%
%%%%%%%%%%%%%%%
\begin{ekdosis}
  \ekddiv{type=ed}
   \bigskip
    \centerline{\textrm{\small{[Description of the seventh Cakra]}}}
    \bigskip
    \begin{prose}
%-----------------------
% idānīṃ saptamaṃ  tālumadhye catuḥṣaṣṭidalaṃ              amṛtapūrṇaṃ vartate / \E
% idānīṃ saptamaṃ  tālumadhye catuḥṣaṣṭhidalaṃ             amṛtapūrṇaṃ vartate / \P
% idānīṃ saptamaṃ  // tāludeśe madhye catuḥṣaṣṭhidala      amṛtapūrṇaṃ vartate / \L
% idānīṃ saptamaṃ  // tāludeśe madhye catuḥṣaṣṭhidala      amṛtapūrṇaṃ vartate / \B
% idānīṃ saptamaṃ  cakraṃ     catuḥṣaṣṭhidalaṃ tālumadhye  amṛtapūrṇaṃ varttate // \N1
% idānīṃ saptamaṃ  cakraṃ     catuṣaṣṭhidalaṃ tālumadhye   amṛtapūrṇa  varttate // \N2      
% idānīṃ saptamaṃ  cakraṃ     catuḥṣaṣṭhidalaṃ tālumadhye  amṛtapūrṇaṃ varttate // \D1
% idānīṃ saptamaṃ  cakraṃ     catuḥṣaṣṭhidalaṃ tālumadhye  amṛtapūrṇaṃ varttate // \U1
% idānīṃ saptamaṃ  tālumadhye catuḥṣaṣṭidalaṃ //           amṛtapūrṇaṃ vartate / \U2      
%-----------------------
% Now the seventh cakra having 64 petals and being full of nectar exists in the middle of the palate.
%-----------------------
%U2: \extra{lalāṭa maṃḍalaṃ// caṃdro devatā// amṛtā śaktiḥ// paramātmā ṛṣiḥ// amṛtavāsinīkalāsaptadaśī amṛtakallolanadī// mahākāśa// aṃbikā// laṃbikā// ghaṃṭikā// tālikā// ajapāgāyatrīdehasvarūpaṃ// kākamukhī// naranetrāgośṛṃgālalāṭabrahmapaṭhāhayagrīvā// mayūramukhā// haṃsavadaṃgāni// ajapāgāyatrīsvarūpaṃ// 
%-----------------------
%  
%-----------------------
%ich fange mal so an: bei der Angabe der Elemente eines
%tantrischen Mantras finden sich in den
%Ritualhandbüchern (paddhati) Stellen wie die folgende:
%
%  śrīmahāgaṇapatimantrasya brahmā ṛṣiḥ gāyatraṃ chandaḥ
%  śrīmahāgaṇapatirdevatā gaṃ bījaṃ hrīṃ śaktiḥ namaḥ kīlakaṃ mama
%  śrīmahāgaṇapatiprasādātsarvasiddhyarthe śrījape viniyogaḥ/
%
%"Für dieses śrīmahāgaṇapatimantra ist Brahmā der Ṛṣi, gāyatraṃ das Metrum, Gaṇeśa dies Gottheit ...".
%Bis hier ist die Angabe nicht anders als für vedische Mantras, wo
%auch der Ṛṣi (also Autor), das Metrum und die Gottheit genannt
%werden müssen, sonst wirkt die Rezitation nicht.
%
%Dann kommen die tantrischen Elemente:
%
%gaṃ ist die Keimsilbe (im Mantra der Gottheit, nämlich: oṃ gaṃ mahāgaṇapataye namah. etc)
%hrīṃ = Śakti usw. Hier folgen dann noch beliebig viele tantrische Elemente.
%Am Ende kommt dann noch die Anwendung des Mantra.
%
%Dein Text scheint diese Struktur nachzubilden, aber merkwürdigerweise in der Beschreibung
%eines cakra. Man muß also vielleicht lesen:
%
%lalāṭa(ṃ) maṃḍalaṃ    Die Stirn ist das Maṇḍala
%caṃdro devatā        Mond die Gottheit
%amṛtā śaktiḥ
%paramātmā ṛṣiḥ
%amṛtavāsinī kalā saptadaśī
%amṛtakallolanadī mahākāśa
%aṃbikā laṃbikā
%ghaṃṭikā tālikā
%
%ajapāgāyatrīdehasvarūpaṃ
%kākamukhī//
%naranetrā
%gośṛṃgā
%lalāṭa brahmapaṭhā
%hayagrīvā//
%mayūra mukhā//
%haṃsavad aṃgāni//
%ajapāgāyatrī svarūpaṃ
%
%Die ajapā gāyatrī ist das mantra, welches der Atem ganztätig als so 'ham = haṃsa vollzieht.
%Steht auch in meiner Sahib Kaul-Paddhati. Diesem Mantra wird nun ein Körper zugeschrieben,
%der genauer beschrieben wird, mit Gesicht, Augen, Hörnern (?).
%
%Aber klar ist mir das auch nicht, jedenfalls wird hier ein cakra wie das mantra einer Gottheit
%behandelt. In jedem Fall interessant.
%
%Liebe Grüße
%Jürgen   
%-----------------------
idānīṃ saptamaṃ
      \app{\lem[wit={N1,D1,U1}]{cakraṃ catuḥṣaṣṭhidalaṃ tālumadhye}
        \rdg[wit={N2}]{cakraṃ catuṣaṣṭhidalaṃ tālumadhye}
        \rdg[wit={E,P,U2}]{tālumadhye catuḥṣaṣṭidalaṃ}
        \rdg[wit={L,B}]{tāludeśe madhye catuḥṣaṣṭhidala}}
      \app{\lem[type=emendation, resp=egoscr]{'mṛtapūrṇaṃ}
        \rdg[wit={ceteri}]{\korr amṛtapūrṇaṃ}
        \rdg[wit={N2}]{amṛtapūrṇa}}
      vartate/
\end{prose}
  \end{ekdosis}
  %%%%%%%%%%%%%%
  %%%%%%%%%%%%%%
  %%%%%%%%%%%%%%
  %%%%%%%%%%%%%%
  %%%%%%%%%%%%%%
  \begin{ekdosis}
    \ekddiv{type=ed}
    \begin{prose}
      \extra{\app{\lem[type=emendation, resp=egoscr]{lalāṭaṃ}
          \rdg[wit={U2}]{\korr lalāṭa°}} maṇḍalaṃ\dd{}
        caṃdro devatā\dd{}
        amṛtā śaktiḥ\dd{}
        paramātmā ṛṣiḥ\dd{}
        amṛtavāsinī kalāsaptadaśī\dd{}
        amṛtakallolanadī \app{\lem[type=emendation, resp=egoscr]{mahākāśā}
          \rdg[wit={U2}]{\korr mahākāśa}}\dd{}
        aṃbikā laṃbikā\dd{}
        ghaṃṭikā tālikā\dd{}
        ajapāgāyatrī dehasvarūpaṃ\dd{}
        kākamukhī\dd{}
        naranetrā\dd{}
        gośṛṃgā\dd{}
        lalāṭabrahmapaṭhā\dd{}
        hayagrīvā\dd{}
        mayūramukhā\dd{}
        haṃsavad-aṃgāni\dd{}  
        ajapāgāyatrī svarūpaṃ\dd{}\note[type=philcomm, labelb=s17.z1, lem={lalāṭaṃ maṇḍalaṃ}]{This additional passage is found in U2 only. Suprisingly after the additions to this \textit{cakra}, the scribe/author of these additions does'nt add instructions for the duration of practice as before.}}
        \end{prose}
      \end{ekdosis}
      \ekdpb*{}
        %%%%%%%%%%%%%%%
        %%%%%%%%%%%%%%%
        %%%%%%%%%%%%%%%
        %%%%%%%%%%%%%%%
        %%%%%%%%%%%%%%%
  \begin{ekdosis}
    \ekddiv{type=ed}
    \begin{prose}
%-----------------------
%adhikaśobhāyuktam-----atiśvetaṃ       tanmadhye       raktavarṇaṃ ghāṃṭikāsaṃjñaikā      karṇikā varttate / \E 
%adhikataraśobhayuktaṃ atiśvetaṃ       tanmadhye       raktavarṇaṃ ghaṭikāsaṃjñā ekā      karṇikā varttate / \P
%adhikataraśobhayuktaṃ // atiśvetaṃ // tanmadhye       raktavarṇaṃ ghaṇikāsaṃjñā ekā ekā  karṇikā varttate / \L
%adhikataraśobhayuktaṃ // atiśvetaṃ // tanmadhye       raktavarṇaṃ ghaṃṭikāsaṃjñā ekā ekā karṇikā varttate / \B
%adhikataraśobhayuktaṃ atiśvetaṃ       tanmadhye       raktavarṇaṃ ghaṃṭikāsaṃjñā ekā     karṇikā varttate / \N1
%adhikataraśobhāyuktaṃ  atiśvetaṃ      tanmadhye       raktavarṇa--ghaṇṭikāsaṃjñā ekā     karṇikā vartate /  \N2
%adhikataraśobhayuktaṃ atiśvetaṃ       tanmadhye       raktavarṇaṃ ghaṃṭikāsaṃjñā ekā     karṇikā varttate / \D1
%adhikataraśobhayuktaṃ atiśvetaṃ       tanmadhye       raktavarṇaṃ ghaṃṭikāsaṃjñā ekā     karṇikā varttate / \U1      
%adhikataraprabhāmuktaṃ // atiśvetaṃ // tanmadhye       raktavarṇaṃ ghaṃṭikāsaṃjñā// ekā   karṇikā varttate / \U2   
%-----------------------
%[It is] endowed with superabundant beauty. [It is] very bright. In its middle, red in color [is that] known as "uvula" (\textit{ghāṃṭikā}). [It] exists as a single pericarp.  
%-----------------------      
      adhi\app{\lem[wit={ceteri},alt={°kataraśobhayuktaṃ}]{kataraśobhayuktaṃ}
        \rdg[wit={N2}]{°kataraśobhāyuktaṃ}
        \rdg[wit={E}]{°kaśobhāyuktam}
        \rdg[wit={U2}]{°kataraprabhāmuktaṃ}}\dd{}
        atiśvetaṃ\dd{}
        tanmadhye
        \app{\lem[wit={ceteri}]{raktavarṇaṃ}
          \rdg[wit={N2}]{raktavarṇa°}}
        \app{\lem[wit={ceteri},alt={ghaṇṭikā°}]{ghaṇṭikā}
          \rdg[wit={E}]{ghāṃṭikā°}
          \rdg[wit={P}]{ghaṭikā°}
          \rdg[wit={L}]{ghaṇikā°}}saṃjñā/
        \app{\lem[wit={ceteri}]{ekā}
          \rdg[wit={L,B}]{ekā ekā}}
          karṇikā vartate/
%-----------------------          
%tanmadhye bhūmiḥ / \E
%tanmadhye bhūmiḥ / \P
%tanmadhye bhūmiḥ / \L
%tanmadhye bhūmiḥ / \B
%tanmadhye bhūmiḥ / \N1
%tanmadhye bhūmiḥ / \N2
%tanmadhye bhūmiḥ / \D1
%tanmadhye bhūmis- / \U1
%tanmadhye bhūmi   / \U2         
%-----------------------
%In its middle is a place. 
%-----------------------        
       tanmadhye
       \app{\lem[wit={ceteri}]{bhūmiḥ}
         \rdg[wit={U1}]{bhūmis°}
         \rdg[wit={U2}]{bhūmi}}/
%-----------------------  
%tanmadhye prakaṭacandrakalā 'mṛtādhārā bhavati         / \E
%tanmadhye prakaṭacandrakalā 'mṛtādhārā sravati         / \P
%tanmadhye prakaṭacandrakalā 'mṛtādhārā sravaṃti        / \L
%tanmadhye prakaṭacandrakalā 'mṛtādhārā sravaṃti        / \B
%tanmadhye prakaṭacandrakalā amṛtādhārāsravaṃtī varttate/ \N1
%tanmadhye prakaṭacaṃdrakalā amṛtādhārāsravaṃtī varttate/ \N2
%tanmadhye prakaṭacandrakalā 'mṛtādhārāsravaṃtī varttate/ \D1 %sravantī f. Fluss Nom Sg
%tanmadhye pragaṭacaṃdrakalā amṛtadhārāsravaṃtī varttate  \U1
%tanmadhye-ṃdrakaṭaṃ caṃdrakalā amṛtadhārā sravati       /\U2       
%-----------------------
%In its middle exists a hidden digit of the moon, being a stream of nectar like a river (\textit{amṛtādhārāsravantī}. 
%-----------------------
       tanmadhye
       \app{\lem[wit={ceteri},alt={prakaṭa°}]{'prakaṭa}
         \rdg[wit={U1}]{pragaṭa}
         \rdg[wit={U2}]{°ṃdrakaṭaṃ}}candrakalā
       \app{\lem[wit={ceteri}]{amṛtadhārāsravantī}
         \rdg[wit={L,B}]{'mṛtādhārā sravaṃti}
         \rdg[wit={P,U2}]{'mṛtādhārā sravati}
         \rdg[wit={E}]{'mṛtādhārā bhavati}}
       \app{\lem[wit={N1,N2,D1,U1}]{vartate}
         \rdg[wit={ceteri}]{\om}}/
\end{prose}
  \end{ekdosis}
  %%%%%%%%%%%%%%%%
  %%%%%%%%%%%%%%%
  %%%%%%%%%%%%%%
  %%%%%%%%%%%%%%
  %%%%%%%%%%%%%%%
  \begin{ekdosis}
    \ekddiv{type=ed}
    \begin{prose}
%-----------------------
%tasyāḥ kalāyā     dhyānakāraṇāt tasya samīpe maraṇaṃ nāyāti/     \E -> does not come near to death -> na-ā-yāti
%tasyāḥ kalāyā     dhyānakaraṇāt tasya samīpe maraṇaṃ nāyāti/     \P
%tasyāḥ karṇikāyā  dhyānakaraṇāt tasya samīpe maraṇaṃ na yāti     \L
%tasyāḥ karṇikāyā  dhyānakaraṇāt tasya samīpe maraṇaṃ na yāti     \B
%tasyāḥ kalāyāḥ    dhyānakaraṇāt tasya samīpe maraṇaṃ nāyāti      \N1
%tasyāḥ kalāyāḥ    dhyānakaraṇāt tasya samīpe maraṇaṃ nāyāti/     \N2       
%tasyāḥ kalāyāḥ    dhyānakaraṇāt tasya samīpe maraṇaṃ nāyāti      \D1
%tasyāḥ kalāyā     dhyānakaraṇāt tasya samīpe maraṇaṃ nāyāti/     \U1
%tasyāḥ kalāyā     dhyānakāraṇāt// tasya samīpe maraṇaṃ na yāti/  \U2
%-----------------------
%Because of the exercise of meditation on this digit death does not come near him. 
%-----------------------
       tasyāḥ
       \app{\lem[wit={ceteri}]{kalāyā}
         \rdg[wit={N1,N2,U1}]{kalāyāḥ} %Sandhi-mistake in apparatus in this case?
         \rdg[wit={L,B}]{karṇikāyā}}
       dhyānakaraṇāt tasya samīpe maraṇaṃ
       \app{\lem[wit={ceteri}]{nāyāti}
         \rdg[wit={L,B,U2}]{na yāti}}/    
%-----------------------
%nirantaradhyānād        -amṛtadhārāyāḥ sajīvo bhavati /  \E
%niraṃtaradhyānāt---------amṛtadhārā plāvanaṃ   bhavati /  \P
%niraṃtaradhyānakaraṇād   amṛtadhārā           sravati /  \L
%niraṃtaradhyānakaraṇād   amṛtadhārā           sravati /  \B
%niraṃtaradhyānakaraṇāt / amṛtadhārā           sravaṃti / \N1
%niraṃtaradhyānakaraṇāt   amṛtadhārā            sravaṃti    \N2
%niraṃtaradhyānakaraṇāt / amṛtadhārā           sravaṃti / \D1
%niraṃtaradhyānakaraṇāt   amṛtadhārā             sravati /  \U1
%niraṃtaradhyānakaraṇāt / amṛtadhārā plavanaṃ  bhavati / \U2
%-----------------------
%Due to uninterrupted meditation the stream (\textit{dhārā}) of nectar flows. 
%-----------------------
       nirantara\app{\lem[wit={ceteri},alt={°dhyānakaraṇād}]{dhyānakaraṇād\skm{a}}
         \rdg[wit={E,P}]{°dhyānād}}
       \app{\lem[wit={ceteri}, alt={amṛtadhārā}]{\skp{-a}mṛtadhārā}
         \rdg[wit={E}]{amṛtadhārāyāḥ sajīvo}
         \rdg[wit={P}]{amṛtadhārā plāvanaṃ}
         \rdg[wit={U2}]{amṛtadhārā plavanaṃ}}
       \app{\lem[wit={L,B,U1}]{sravati}
         \rdg[wit={N1,N2,D1}]{sravaṃti}
         \rdg[wit={E,P,U2}]{bhavati}}/       
%-----------------------
%tadā  yakṣam-aroga----pittajvarahṛdayadāha-śiroroga-jihvā--jaḍa-bhāvā           naśyanti / \E
%tadā     kṣayaroga----pittajvarahṛdayadāha-śiroroga-jihvā--jaḍa-bhāvān          naśyanti / \P
%tadā     kṣayaroga----pittajvarahṛdayadāha-----roga-jihvāyājaḍa-bhāvān          naśyanti / \L
%tadā     kṣayaroga----pittajvarahṛdayadāha-----roga-jihvāyājaḍa-vān             naśyanti / \B
%         kṣayarogaṃ   pittajvarahṛdayadāha-śiroroga-jihvāyājaḍa-bhāvā           naśyanti / \N1 %besser kṣayarogaṃ emendieren zu vollem Kompositum?
%         kṣayarogaṃ   pittajvarahṛdayadāha-śiroroga-jihvāyājaḍa-bhāvātā         naśyanti / \N2
%         kṣayaṃ rogaṃ pittajvarahṛdayadāha-śiroroga-jihvāyājaḍa-bhāvā           naśyanti / \D1
%         kṣayaroga----pittajvarahṛdayadāha-śiroroga-jihvāyājaḍa-bhāvā           naśyanti / \U1  
%tadā     kṣayarogo----ptatti// jvara hṛdayadāha// śiroroga// jihvājaḍatā// dayo naśyanti / \U2       
%-----------------------
%Then the appearances of emaciation (\textit{kṣayaroga}), fever due to disordered bile (\textit{pittajvara), heartburn (\textit{hṛdayadāha}), head-disease (\textit{śiroroga}) and tongue insensibility (\textit{jihvājaḍa}) vanish. %!!!Krankheiten in Ayurvedabuch checken! medizinische Identifikationen!
%-----------------------
       \app{\lem[wit={E,P,L,B,U2}]{tadā}
         \rdg[wit={ceteri}]{\om}}
       \app{\lem[type=emendation, resp=egoscr]{kṣayarogapittajvarahṛdayadāhaśirorogajihvājaḍabhāvā}
         \rdg[wit={E}]{\korr yakṣamarogapittajvarahṛdayadāhaśirorogajihvājaḍabhāvā}
         \rdg[wit={P}]{kṣayarogapittajvarahṛdayadāhaśirorogajihvājaḍabhāvān}
         \rdg[wit={L}]{kṣayarogapittajvarahṛdayadāharogajihvāyājaḍabhāvān}
         \rdg[wit={B}]{kṣayarogapittajvarahṛdayadāharogajihvāyājaḍavān}
         \rdg[wit={N1}]{kṣayarogaṃ pittajvarahṛdayadāhaśirorogajihvāyājaḍabhāvā}
         \rdg[wit={N2}]{kṣayarogaṃ pittajvarahṛdayadāhaśirorogajihvāyājaḍabhāvātā}
         \rdg[wit={D1}]{kṣayaṃ rogaṃ pittajvarahṛdayadāhaśirorogajihvāyājaḍabhāvā}
         \rdg[wit={U1}]{kṣayarogapittajvarahṛdayadāhaśirorogajihvāyājaḍabhāvā}
         \rdg[wit={U2}]{kṣayarogoptatti || jvara hṛdayadāha || śiroroga || jihvājaḍatā || dayo}}
         naśyanti/
%-----------------------       
%bhakṣitam--api   viṣan    na bādhate / \E
%bhakṣitam--api   viṃṣa    na bādhate / \P
%bhākṣitam--api   viṣaṃ    na bādhyate / \L
%bhākṣitamār pi   viṣaṃ    na bādhyate / \B
%bhakṣitam        viṣamapi na bādhyate / \N1
%bhakṣitaṃ        viṣamapi na bādhate / \N2
%bhakṣitāṃ        viṣamapi na bādhyate / \D1
%bhakṣitaṃ        viṣamapi na bādhyate   \U1
%bhakṣitam--api   viṣaṃ    na bādhyate / \U2       
%-----------------------       
%Also eaten venom doesn't trouble him. 
%-----------------------
         \app{\lem[wit={N2,U1}]{bhakṣitaṃ}
           \rdg[wit={N1}]{bhakṣitam}
           \rdg[wit={D1}]{bhakṣitāṃ}
           \rdg[wit={E,P,L,U2}]{bhakṣitam api}
           \rdg[wit={B}]{bhākṣitamār pi}}
         \app{\lem[wit={N1,N2,D1,U1}, alt={viṣam api}]{viṣam-api}
           \rdg[wit={L,B,U2}]{viṣaṃ}
           \rdg[wit={E}]{viṣan}
           \rdg[wit={P}]{viṃṣa}}
         na
         \app{\lem[wit={E,P,N2}]{bādhate}
           \rdg[wit={ceteri}]{bādhyate}}/       
%-----------------------
%yady-atra manaḥ sthiraṃ   bhavati /  \E
%yady-atra manaḥ sthiraṃ   bhavati /  \P
%yady-atramapi manasthiraṃ bhavati /  \L              %VARIANTE UNSICHER!!!WAS MEINT JÜRGEn??
%yady-atramapi manasthiraṃ bhavati /  \B
%yady-atra     manasthiraṃ bhavati /  \N1
%yadyanna      manasthiraṃ bhavati // \N2
%yadyanna      manasthiraṃ bhavati /  \D1
%yadyatra      manasthiraṃ bhavati    \U1
%yadyatra      manasthiraṃ bhavati//  \U2       
%-----------------------
%If here the mind becomes stable.       
%-----------------------
         \app{\lem[wit={ceteri}]{yadyatra}
           \rdg[wit={L,B}]{yadyatram api}
           \rdg[wit={N1,D1}]{yadyanna}}
         \app{\lem[wit={E,P}]{manaḥ sthiraṃ}
           \rdg[wit={ceteri}]{manasthiraṃ}}
         bhavati\dd{}         
    \end{prose}
  \end{ekdosis}
%%%%%%%%%%%%%%% 
%%%%%%%%%%%%%%%
%%%%%%%%%%%%%%%
%%%%%%%%%%%%%%%
%%%%%%%%%%%%%%%
\begin{ekdosis}
  \ekddiv{type=ed}
   \bigskip
    \centerline{\textrm{\small{[Description of the eighth Cakra]}}}
    \bigskip
    \begin{prose}
%-----------------------
%idānīṃ brahmarandhrasthāne 'ṣṭamaṃ śatadalaṃ cakraṃ varttate / \E
%idānīṃ brahmaraṃdhrasthāne 'ṣṭamaṃ śatadalaṃ cakraṃ vartate / \P
%idānīṃ brahmaraṃdhrasthāne aṣṭamaṃ śatadalaṃ cakraṃ vartate / \L
%idānīṃ brahmaraṃdhrasthāne aṣṭamaṃ śatadalaṃ cakraṃ vartate / \B
%idānīṃ aṣṭamacakraṃ brahmaraṃdhrasthāne śatadalaṃ   vartate / \N1
%idānīṃ aṣṭamacakraṃ brahmaraṃdhrasthāne śatadalaṃ   vartate  \N2
%idānīṃ aṣṭamacakraṃ brahmaraṃdhrasthāne śatadalaṃ   vartate / \D1
%idānīṃ aṣṭamaṃ cakraṃ brahmaraṃdhrasthāne śatadalaṃ   vartate . \U1
%idānīṃ brahmaraṃdhrasthāne 'ṣṭamaṃ śatadalaṃ cakraṃ varttate // \U2
%-----------------------
%guru devatā// caitanya śaktiḥ// virāṭ ṛṣiḥ// sarvotkṛṣṭasākṣiḥ// bhūtaturyātītacaitanyātmakaṃ// sarvavarṇāḥ// sarvamātrāḥ// sarvadalāni virāṭdeha sthitāvasthā prajñāvācā sohaṃ veda anupamasthānaṃ// ajapājapasahasra/ 1000 gha 02 pa 046 akṣara 40// sarvajapasaṃkhyā// 21600// ekaviṃśatisahasrāṇiṣaṭśatāni// tathaivaca niśāhevahate// prāṇaḥ yojānātisapaṃḍitaḥ// sakāreṇa bahiryātihakāreṇaviśotpunaḥ// haṃsaḥ sohaṃ// tato maṃtraṃ jīvojapati sarvadā//    
%-----------------------
%Now exists the eigth \textit{cakra} having one hundred petals located at the aperture of Brahman.
%-----------------------
idānīṃ
\app{\lem[wit={N1,N2,D1}]{aṣṭamacakraṃ brahmaraṃdhrasthāne śatadalaṃ}
    \rdg[wit={E,P,U2}]{brahmarandhrasthāne 'ṣṭamaṃ śatadalaṃ cakraṃ}
    \rdg[wit={L,B}]{brahmaraṃdhrasthāne aṣṭamaṃ śatadalaṃ cakraṃ}
    \rdg[wit={U1}]{cakraṃ brahmaraṃdhrasthāne śatadalaṃ}}
vartate/
%          \end{prose}
%        \end{ekdosis}
        %%%%%%%%%%%%
        %%%%%%%%%%%%
        %%%%%%%%%%%%
        %%%%%%%%%%%%
        %%%%%%%%%%%%
%  \begin{ekdosis}
%    \ekddiv{type=ed}
%    \begin{prose}
\extra{\app{\lem[type=emendation, resp=egoscr, alt={gurur}]{guru\skp{r-de}}
          \rdg[wit={U2}]{\korr guru°}}\skm{r-de}vatā\dd{}
        \app{\lem[type=emendation, resp=egoscr]{caitanyaḥ}
          \rdg[wit={U2}]{\korr caitanya°}} śaktiḥ\dd{}
        virāṭ ṛṣiḥ sarvotkṛṣṭasākṣiḥ\dd{}
        \app{\lem[type=emendation, resp=egoscr]{bhūtaturyātītaṃ}
          \rdg[wit={U2}]{\korr bhūtaturyātīta°}} caitanyātmakaṃ\dd{}
        sarvavarṇāḥ\dd{}
        sarvamātrāḥ\dd{}
        sarvadalāni\dd{}
        virāṭ \app{\lem[type=emendation, resp=egoscr]{dehaḥ}
          \rdg[wit={U2}]{\korr deha°}}
        sthitāvasthā\dd{} 
        prajñā vācā\dd{}
        sohaṃ \app{\lem[type=emendation, resp=egoscr]{vedaḥ}
          \rdg[wit={U2}]{\korr veda}}\dd{}
        \app{\lem[type=emendation, resp=egoscr]{anupamaṃ}
          \rdg[wit={U2}]{\korr anupama°}} sthānaṃ\dd{}
         \app{\lem[type=emendation, resp=egoscr]{ajapājapaḥ sahasraḥ}
          \rdg[wit={U2}]{\korr ajapājapasahasra}}\dd{} 1000 ghaṭi 2 palā 46 akṣara 40\dd{}
        \app{\lem[type=emendation, resp=egoscr]{sarvajapaḥ}
          \rdg[wit={U2}]{\korr sarvajapa°}} saṃkhyā\dd{}
        21600\dd{}
        ekaviṃśatisahasrāṇiṣaṭśatāni\dd{}
        tathaiva ca niśāhe vahate\dd{}
        prāṇaḥ yo jānāti sa paṃḍitaḥ\dd{} %%prāṇaḥ = m nom pl
        sakāreṇa bahir-yāti hakāreṇa viśet punaḥ\dd{}  
        haṃsaḥ sohaṃ\dd{}
        tato mantraṃ jīvo japati sarvadā\dd{}}
    \end{prose}
  \end{ekdosis}
%The teacher is the deity. Consciousness is the power. Virāṭ is the Ṛṣi, the witness above everything. Made of consciousness is that which is associated with (\textit{bhūta°) the state beyond the fourth state. It has all colours. It has all matrices. It has all petals. The body is Virāṭ. The state is the standing still. The speech is wisdom.  The "I am that"-[expression] (\textit{sohaṃ}) is the Veda. The place is unsurpassed. A thousandfold recitation of the non-recited; 1000 [repetitions for]; 2 \textit{ghaṭi}s, 46 \textit{palā}s. and 40 \textit{akṣara}s.\footnote{It's not entirely clear what kind of measure is an \textit{akṣara}.} The count is all silent mutterings, [being] 21600. Day and night in this way it carries on. He who knows the breath is a learned person. With the sound of "sa" he exhales, with the sound of "ha" he inhales again: "I'm he, he's I". Because of that the embodied soul constantly utters the Mantra.\footnote{Add intertextual evidence.}
  %%%%%%%%%%%%%%%
  %%%%%%%%%%%%%%%
  %%%%%%%%%%%%%%%
  %%%%%%%%%%%%%%
  %%%%%%%%%%%%%%%
  \begin{ekdosis}
    \ekddiv{type=ed}
    \begin{prose}
%----------------------
%tasya kamala----jātyadharaṇīpīṭha iti saṃjñā / \E
%tasya kamalasya jālaṃdharapīṭha iti saṃjñā / \P
%tasya kamalasya jālaṃdharapīṭha iti saṃjñā ...  \L
%tasya kamalasya jālaṃdharapīṭhasaṃjñā ...  \B
%tasya kamalasya jālaṃdharapīṭha iti saṃjñā ...  \N1
%tasya kamalasya jālaṃdharapīṭha iti saṃjñā ...  \N2
%tasya kamalasya jālaṃdharapīṭha iti saṃjñā ...  \D1
%tasya kamalasya jālaṃdharapīṭha iti saṃjñā ...  \U1      
%tasya kamalasya jālaṃdharapīṭha iti saṃjñā //   \U2
%----------------------
%``The (divine) seat of  Jālaṃdhara'' is the designation of the lotus of it. 
%----------------------      
tasya \app{\lem[wit={ceteri}]{kamalasya}
        \rdg[wit={E}]{kamala°}}
      \app{\lem[wit={ceteri}]{jālandharapīṭha}
        \rdg[wit={B}]{jālandharapīṭha°}
        \rdg[wit={E}]{jātyadharaṇīpīṭha}}
      \app{\lem[wit={ceteri}]{iti}
        \rdg[wit={B}]{\om}}
      \app{\lem[wit={ceteri}]{saṃjñā}
        \rdg[wit={B}]{°saṃjñā}}/
%---------------------- 
%siddhapuruṣasya sthānam / \E
%siddhapuruṣasya sthānam / \P
%siddhapuruṣasya sthānam mūrti vartate // \L                         %%% schwerer Satz -> wie soll ich hier entscheiden?! 
%siddhapuruṣasya sthānam mūrti vartate // \B %Zeilensprung
%siddhapuruṣasya sthānam // \N1
%siddhapuruṣasya sthānam // \N2
%siddhapuruṣasya sthānam // \D1
%siddhapuruṣasya sthānam    \U1
%siddhapuruṣasya sthānaṃ   \U2
%----------------------      
%[It is] the place of the accomplished person.
%----------------------
siddhapuruṣasya
      \app{\lem[wit={ceteri}]{sthānaṃ}
        \rdg[wit={L,B}]{sthānam mūrti vartate}}/
       \end{prose}
     \end{ekdosis}
     \ekdpb*{}
  %%%%%%%%%%%%%%%
  %%%%%%%%%%%%%%%
  %%%%%%%%%%%%%%%
  %%%%%%%%%%%%%%
  %%%%%%%%%%%%%%%
  \begin{ekdosis}
    \ekddiv{type=ed}
    \begin{prose}
%----------------------
%tanmadhye    'gnidhūmākārarekhā     yādṛśy    ādṛśy ekā  puruṣasya mūrttir varttate / \E
%tanmadhye    'gnidhūmākārarekhā     yādṛśī   tādṛśy ekā  puruṣasya mūrttir varttate / \P
%tanmadhye    'gnidhūmākārārekhā     yādṛśī   tādṛśy ekā  puruṣasya mūrttir varttate / \L               
%tanmadhye    'gnidhūmākārārekhā     yādṛśī   tādṛśy ekā  puruṣasya mūrttir varttate / \B
      
%tanmadhye    'gnidhūmākārāreṣā      yādṛśī   tādṛśī ekā  puruṣasya mūrttir varttate / \N1
%tanmadhye    agnidhūmrākārarekhā    yādṛśī / tādṛśī ekā  puruṣasya mūrttir varttate / \N2
%tanmadhye    agnidhūmākārāreṣā      yādṛśī   tādṛśī ekā  puruṣasya mūrttir varttate / \D1
%tanmadhye    agnidhūmrākārārekhā    yādṛśī   tādṛśī ekā  puruṣasya mūrtir  vartate    \U1
%tanmadhye    'gnidhūmrākārārekhāyāḥ  etādṛśī         ekā  puruṣasya mūrtir  vartate // \U2
%----------------------      
%In its middle [is] something like a streak having the form of smoke and fire. Such a single [divine] form of the person (\textit{puruṣa}) exists [there].        
%---------------------      
      tanmadhye
      \app{\lem[wit={E,P,L,B}]{'gnidhūmākārarekhā}
        \rdg[wit={N1,D1}]{'gnidhūmākārāreṣā}
        \rdg[wit={N2,U1}]{agnidhūmrākārarekhā}
        \rdg[wit={U2}]{'gnidhūmrākārārekhāyāḥ}}
      \app{\lem[wit={ceteri}]{yādṛśī}
        \rdg[wit={E}]{yādṛśy°}
        \rdg[wit={U2}]{etādṛśī}}/
      \app{\lem[wit={P,L,B}]{yādṛśy}
        \rdg[wit={E}]{ādṛsy}
        \rdg[wit={N1,N2,D1,U1}]{yādṛśī}
        \rdg[wit={U2}]{\om}}ekā puruṣasya mūrtir-vartate/
%---------------------
%tasyā  nādir nāṃto 'sti / \E
%tasyā  nādināṃ 'to sti / \P
%tasyā  nādir nāṃto sti / \L -> vor dem bei allen anderen vorigen Satz!?!?!?! 
%tasyā  nādir nāṃto sti / \B -> vor dem bei allen anderen vorigen Satz!?!?!?! 
%tasyāḥ nāstyaṃtaḥ ādir-api nāsti / \N1????
%tasyāḥ nāstyaṃtaḥ ādir-api nāsti / \N2
%tasyāḥ nāstyaṃtaḥ ādir api nāsti / \D1 
%tasyāḥ nāstyaṃtaḥ ādir-api nāsti    \U1
%tasyā  nādir naṃto sti              \U2
%---------------------
% Of her exists no end, nor a beginning.
%---------------------      
      \app{\lem[wit={E,P,L,B}]{tasyā} %Sandhi-difference included! 
        \rdg[wit={ceteri}]{tasyāḥ}}
      \app{\lem[alt={nādir nānto 'sti}, wit={ceteri}]{nādir-nānto 'sti}
        \rdg[wit={N1,N2,D1,U1}]{nāstyaṃtaḥ ādir api nāsti}
        \rdg[wit={P}]{nādināṃ 'to sti}}/
\end{prose}
\end{ekdosis}
  %%%%%%%%%%%%%
  %%%%%%%%%%%%%%
  %%%%%%%%%%%%%
  %%%%%%%%%%%%%
  %%%%%%%%%%%%%
  \begin{ekdosis}
    \ekddiv{type=ed}
    \begin{prose}   
%---------------------    
%tasyā  mūrtter dhyānakāraṇāt pratyakṣaṃ niraṃtaraṃ  puruṣasyākāśe   gamāgamau   bhavataḥ / \E
%tasyā  mūrtter dhyānakaraṇāt pratyakṣaniraṃtaraṃ    puruṣasyākāśe   gamāgamau   bhavataḥ / \P
%tasyā  mūrtir  dhyānakaraṇāt pratyakṣaniraṃtaraṃ    puruṣasyākāśe   gamāgamau   bhavataḥ / \L         
%tasyā  mūrtir  dhyānakaraṇāt pratyakṣaṃ niraṃtaraṃ  puruṣasyākāśe   gamāgamau   bhavataḥ / \B
%tasyāḥ mūrttir dhyānakaraṇāt pratyakṣaniraṃtaraṃ    puruṣasya ākāśe gamāgamau   bhavataḥ / \N1
%tasyāḥ mūrttir dhyānakaraṇāt pratyakṣaniraṃtaraṃ    puruṣa ākāśe    gamāgame    bhavataḥ / \N2
%tasyāḥ mūrtir  dhyānakaraṇāt pratyakṣaniraṃtaraṃ    puruṣasya ākāśe gamāgamau   bhavataḥ / \D1
%tasyāḥ mūrter  dhyānakaraṇāt/ pratyakṣaniraṃtaraṃ   puruṣasya ākāśi gamāmamo   bhavataḥ   \U1
%tasyāḥ mūrter  dhyānakaraṇāt pratyakṣaniraṃtaraṃ    puruṣasyākāśa---gamāgamau bhavata //      \U2
%---------------------    
%BEDEUTUNG DES SATZES BIS JETZT UNKLAR! Idee: Zeilensprung aus übernächstem Satz! Streiche pratyakṣaṃ niraṃtaraṃ und der Satz ergibt Sinn!  
%gamāgamau nom.  dual = coming and going ; bhavataḥ = 3p du ind pres von bhū
%Due to the exercise of meditation on this (divine) form both coming and going of the person in space occurs. 
%Kolloquium: Meinung zu Kompositum pratyakṣaniraṃtaraṃ = macht wenig Sinn oder?
%{\englishnote{\small Even though every single witness at hand transmits the latter reading right after \textit{°karaṇāt}, several considerations make it reasonable to conject that the original sentence is corrupted and was written without it. The main consideration to assume the corruption is that \textit{pratyakṣaṃ nirantaraṃ} is ungrammatical. The second is that the sentence is way more meaningful without it. The third that two sentences later we get the phrase in a meaningful context. Due to the last consideration my best guess is an interlace at an early stage of transmission.}}
%---------------------
      tasyā \app{\lem[alt={mūrter},wit={E,P,U1,U2}]{mūrte\skp{r-}}
        \rdg[wit={ceteri}]{mūrtir}
      }\app{\lem[alt={dhyānakaraṇāt},type=conjecture, resp=egoscr]{\skm{-r}-dhyānakaraṇāt}
        \rdg[wit={E,B}]{\conj dhyānakāraṇāt pratyakṣaṃ niraṃtaraṃ}
        \rdg[wit={ceteri}]{dhyānakaraṇāt pratyakṣaniraṃtaraṃ}}
      \note[type=philcomm, labelb=s22.z4, lem={°kāraṇāt pratyakṣaṃ niraṃtaraṃ}]{Even though every single witness at hand transmits the latter reading right after °\textit{karaṇāt}, several considerations make it reasonable to conject that the original sentence is corrupted and was written without it. The main consideration to assume the corruption is that the syntactical units \textit{pratyakṣaṃ nirantaraṃ} is ungrammatical in this construction. The second is that the sentence is way more meaningful without it. The third that two sentences later we get the phrase in a meaningful context. Due to the last consideration my best guess is an interlace at an early stage of transmission.}
      \app{\lem[wit={ceteri}]{puruṣasyākāśe}
        \rdg[wit={N2}]{puruṣa ākāśe}
        \rdg[wit={U2}]{puruṣasyākāśa°}
        \rdg[wit={U1}]{puruṣasya ākāśi}}
      \app{\lem[wit={ceteri}]{gamāgamau}
        \rdg[wit={U1}]{°gamo}
        \rdg[wit={N2}]{°game}}
        \app{\lem[wit={ceteri}]{bhavataḥ}
          \rdg[wit={U2}]{bhavata}}/
%---------------------     
%pṛthvīmadhye  sthitasyāpi    pṛthvī-------bādho   na bhavati / \E
%pṛthvīmadhye  sthitasyāpi    pṛthaka                 bhavati   \P %Zeilenspringer führt zu Verlust von Zeile in Pune
%pṛthvīmadhye  sthitasyāpi    pṛthvī-------bādho   na bhavati / \L
%pṛthivīmadhye sthitasyāpi // pṛtvī--------bādho   na bhavati // \B
%pṛthvīmadhye  sthitāv-api    pṛthvī kṣato bādho   na bhavati // \N1
%pṛthvīmadhye  sthitāv-api    pṛthvī kṣato bādho   na bhavati // \N2      
%pṛthvīmadhye  sthitāv-api    pṛthvī kṣato bādho   na bhavati // \D1
%pṛthvīmadhye  sthitāv-api    pṛthvī kṣato bādho   na bhavati     \U1
%pṛthīvīmadhye sthitasyāpi    pṛthvī       bādhoko na bhati     \U2
%---------------------
%Affliction from the earth-element does not arise [anymore] even if one is situated in the middle of the earth.        
%---------------------
        \app{\lem[wit={ceteri}]{pṛthvīmadhye}
          \rdg[wit={B,U2}]{pṛtivīmadhye}}
        \app{\lem[wit={ceteri}]{sthitasyāpi}     
          \rdg[wit={N1,N2,D1,U1}]{sthitāv-api}}
        \app{\lem[wit={E,L}]{pṛthvībādho}
          \rdg[wit={B}]{pṛtvībādho}
          \rdg[wit={N1,N2,D1,U1}]{kṣato bādho}
          \rdg[wit={P}]{pṛthaka}
          \rdg[wit={U2}]{pṛthvī bādhoko}}
        \app{\lem[wit={ceteri}]{na bhavati}
          \rdg[wit={P}]{bhavati}
          \rdg[wit={U2}]{na bhati}}/
%---------------------
%sakalān pratyakṣaṃ niraṃtaraṃ paśyati ca pṛthagbhavati / \E
% \om                                                       \P      
%sakalāḥ pratyakṣaṃ niraṃtara paśyatī  ca pṛthak bhavati // \B
%sakalāḥ pratyakṣaṃ niraṃtara paśyatī  ca pṛthak bhavati / \L
%sakalāpratyakṣaniraṃtaraṃ    paśyati  ca pṛthak ca bhavati // \N1
%sakalapratyakṣaniraṃtaraṃ    paśyati  ca pṛthak ca bhavati    \N2      
%sakalāpratyakṣaniraṃtaraṃ    paśyati  ca pṛthak pṛthak bhavati \D1
%sakalāpratyakṣaniraṃtaraṃ    paśyati  ca/ pṛthak ca bhavati // \U1
%\om                                                     \U2
%---------------------
%He constantly sees everything in front of his eyes and he becomes separated (from the material world).
%---------------------
        \app{\lem[type=emendation, resp=egoscr]{sakalaṃ pratyakṣaṃ nirantaraṃ}
          \rdg[wit={N1,N2,D1,U1}]{\korr sakalāpratyakṣaṃ nirantaraṃ}
          \rdg[wit={B,L}]{sakalāḥ pratyakṣaṃ niraṃtara}
          \rdg[wit={E}]{sakalān pratyakṣaṃ niraṃtaraṃ}
          \rdg[wit={P,U2}]{\om}}
        \app{\lem[wit={ceteri}]{paśyati}
          \rdg[wit={L,B}]{paśyatī}
          \rdg[wit={P,U2}]{\om}}
        \app{\lem[wit={E}]{pṛthagbhavati}
          \rdg[wit={B,L}]{ca pṛthak bhavati}
          \rdg[wit={N1,N2,U1}]{ca pṛthak ca bhavati}
          \rdg[wit={P,U2}]{\om}}/  
%---------------------
%atiśayenāyur vardhate /   \E
%atiśayenāyur vardhate     \P      
%atīśayanāyur vardhayate / \B
%atīśayanāyur vardhayate // \L
%atiśayena āyur varddhate // \N1
%atiśayena āyur varddhate // \N2     
%atiśayena āyur varddhate // \D1
%atiśayena āyur varddhate // \U1
%\om                         \U2
%---------------------
% The force of life increases eminently. 
%---------------------
        \app{\lem[alt={atiśayenāyur},wit={E,P}]{atiśayenāyu\skp{r-}}
          \rdg[wit={B,L}]{atīśayanāyur}
          \rdg[wit={N1,N2,D1,U1}]{atiśayena āyur}
          \rdg[wit={U2}]{\om}}\app{\lem[alt={vardhate},wit={ceteri}]{\skm{r-}vardhate}
          \rdg[wit={B,L}]{vardhayate}}\dd{}        
    \end{prose}
  \end{ekdosis}
%%%%%%%%%%%%%%%
%%%%%%%%%%%%%%
%%%%%%%%%%%%%
%%%%%%%%%%%%%
%%%%%%%%%%%%%
\begin{ekdosis}
 \ekddiv{type=ed}
   \bigskip
    \centerline{\textrm{\small{[Description of the ninth Cakra]}}}
    \bigskip
 \begin{prose}
%---------------------
%idānīṃ navamacakrasya   bhedāḥ kathyante /  \E
%idānīṃ navamacakrasya   bhedāḥ kathyante /  \P
%idānīṃ navamacakrasya   bhedāḥ kathyate     \L
%idānīṃ navamaṃ cakrasya bhedāḥ kathyate //  \B
%idānīṃ navamacakrasya   bhedāḥ kathyaṃte // \N1
%idānīṃ navamacakrasya   bheda  kathyate  // \N2
%idānīṃ navamacakrasya   bhedāḥ kathyaṃte // \D1
%idānīṃ navamaś cakrasya bhedāḥ kathyaṃte    \U1   
%idānīṃ navamacakrasya   bhedaḥ kathyate /   \U2
%---------------------
%Now the divisions/differentiations of the ninth cakra are explained.
%---------------------
idānīṃ
\app{\lem[wit={ceteri},alt={°navama}]{navama}
  \rdg[wit={B}]{navamaṃ}
  \rdg[wit={U1}]{navamaś°}}cakrasya
\app{\lem[wit={ceteri}]{bhedāḥ}
  \rdg[wit={N2}]{bheda}}
\app{\lem[wit={ceteri}]{kathyante}
  \rdg[wit={L,B,N2,U2}]{kathyate}}/
%------------------------------
%tasya mahāśūnyacakram    iti  saṃjñā /  \E
%tasya mahāśūnyacakram    iti  saṃjñā /  \P
%tasya mahāśūnye cakram   iti  saṃjñā    \L
%tasye mahāśūnye cakram   iti  saṃjñā    \B
%tasya mahāśūnye cakreti       saṃjñā // \N1
%tasya mahāśūnyacakreti        saṃjñā // \N2
%tasya mahāśūnyacakreti        saṃjñā // \D1
%tasya mahāśūnyacakreti        saṃjñā /  \U1
%\om /                                   \U2
%---------------------
%The designation of it is ``the \textit{cakra} of the great void (\textit{mahāśūnyacakra})''.
%------------------------------
tasya \app{\lem[wit={ceteri}, alt={mahāśūnya°}]{mahāśūnya}
  \rdg[wit={L,B,N1}]{mahāśūnye}
  \rdg[wit={U2}]{\om}
}\app{\lem[wit={ceteri},alt={°cakreti}]{cakreti}
  \rdg[wit={E,P}]{°cakram iti}
  \rdg[wit={L,B}]{cakram iti}
  \rdg[wit={U2}]{\om}}
\app{\lem[wit={ceteri}]{saṃjñā}
  \rdg[wit={U2}]{\om}}/
%------------------------------
%tadupary aparaṃ kimapi nāsti / \E
%tadupary aparaṃ kimapi nāsti \P
%tadupary        kimapi nāsti \B ??-> auch mögliche Lesart
%tadupari        kimapi nāsti \L
%tadupari aparaṃ kiṃapi nāsti / \N1
%tadupari aparaṃ kiṃapi nāsti / \N2
%tadupari aparaṃ kiṃapi nāsti / \D1
%tadupari aparaṃ kiṃapi nāsti   \U1
% \om                           \U2
%---------------------
%kim api: somewhat, to a considerable extent, rather, much more, still, further. Śa
%---------------------
%Above that there is no other. 
%---------------------
\app{\lem[wit={E,P,B},alt={tadupary}]{tad\skp{-}upary\skm{a-}}
  \rdg[wit={ceteri}]{tad upari}
  \rdg[wit={U2}]{\om}}\app{\lem[wit={ceteri}]{\skp{-a}paraṃ}
  \rdg[wit={B,L,U2}]{\om}}
\app{\lem[wit={ceteri}]{kimapi}
  \rdg[wit={N1,N2,D1,U1}]{kiṃ api}
  \rdg[wit={U2}]{\om}} nāsti/
%------------------------------
%tadeva-mahāsiddhacakraṃ kathyate // \E
%tadeva-mahāsiddhacakraṃ kathyate    \P 
%tadeva-mahāsiddhacakraṃ kathyate // \B
%tadeva-mahāsiddhacakraṃ kathyate // \L
%tadeva-mahāsiddhacakraṃ kathyate // \N1
%tadeva-mahāsiddhacakraṃ kathyate // \N2
%tadeva-mahāsiddhacakraṃ kathyate // \D1
%tadeva-mahāsiddhacakraṃ kathyate /  \U1
% \om                                \U2
%---------------------
%Therefore it is declared to be the \textit{cakra} of the great perfection (\textit{mahāsiddhacakra}).
%---------------------
tad-eva mahāsiddhacakraṃ kathyate/
%------------------------------
%       tasya           pūrṇagiripīṭha               etadṛśaṃ nāma /  \E 
%       tasya           pūrṇagiripīṭham-iti          etādṛśaṃ nāma    \P
%       tasya           pūrṇagiripīṭham-iti saṃjñā   etādṛsaṃ nāma    \B ->!!! 
%       tasya           pūrṇagiripīṭham-iti saṃjñā   etādṛsaṃ nāma    \L
%       tasya cakrasya  pūrṇagiri                    etādṛśaṃ nāma /  \N1
%       tasya cakrasya  pūrṇagiri                    etādṛśaṃ nāma /  \N2
%       tasya cakrasya  pūrṇagiri                    etādṛśaṃ nāma /  \D1
%       tasya cakrasya  pūrṇagire                    etādṛśaṃ nāmaḥ   \U1
%madhye tasya           pūrṇagiripīṭham-iti          ekādaśaṃ nāma // \U2   
%-----------------------------
%Such a name of it is ``(divine) seat of Pūrṇagiri''.   
%------------------------------
\app{\lem[wit={ceteri}]{tasya}
  \rdg[wit={N1,N2,D1,U1}]{tasya cakrasya}
  \rdg[wit={U2}]{madhye tasya}}
\app{\lem[wit={E,P,B,L,U2},alt={pūrṇagiri°}]{pūrṇagiri}
  \rdg[wit={N1,N2,D1}]{pūrṇagiri}
  \rdg[wit={U1}]{pūrṇagire}}\app{\lem[wit={P,B,L,U2}, alt={pīṭham°}]{pīṭham\skm{i-}}
  \rdg[wit={E}]{pīṭha}
  \rdg[wit={ceteri}]{\om}}\app{\lem[wit={P,U2},alt={iti}]{\skp{-i}ti}
  \rdg[wit={B,L}]{iti saṃjñā}
  \rdg[wit={ceteri}]{\om}}
\app{\lem[wit={ceteri}]{etādṛśaṃ}
  \rdg[wit={E}]{etadṛśaṃ}
  \rdg[wit={U2}]{ekādaśaṃ}}
\app{\lem[wit={ceteri}]{nāma}
  \rdg[wit={U1}]{nāmaḥ}}/
\end{prose}
\end{ekdosis}
  %%%%%%%%%%%%%%
  %%%%%%%%%%%%%%%
  %%%%%%%%%%%%%% 
  %%%%%%%%%%%%%%
  %%%%%%%%%%%%%%
   \begin{ekdosis}
     \ekddiv{type=ed}
     \begin{prose}
%------------------------------
%tasya mahāśūnyacakrasya madhye ūrdhvamukham iti raktavarṇaṃ sakalaśobhāspadam    \E
%tasya mahāśūnyacakrasya madhye ūrdhvamukham iti raktavarṇa--sakalaśobhāspadaṃ     \P
%tasya mahāśūnyacakrasya madhye ūrdhvamukhem iti raktavarṇaṃ sakalaśobhāspadaṃ // \B    
%tasya mahāśūnyacakrasya madhye ūrdhvamukham iti raktavarṇaṃ sakalaśobhāspadaṃ // \L
%tasya mahāśūnyacakramadhye     ūrdhvamukhaṃ atiraktavarṇaṃ  sakalaśobhāspadaṃ /   \N1 ->!!!
%tasya mahāśūnyacakramadhye     ūrdhvamukhaṃ atiraktavarṇaṃ  sakalaśobhāspadaṃ     \N2
%tasya mahāśūnyacakramadhye     ūrdhvamukhaṃ atiraktavarṇaṃ  sakalaśobhāspadaṃ /   \D1
%tasya mahāśūnyacakramadhye     ūrdhvamukhaṃ atiraktavarṇaṃ  sakalaśobhāspadaṃ     \U1
%tasya mahāśūnyacakrasya        urdhvamukham-ativarṇaṃ       sakalaśobhanāsyadaṃ / \U2                                             
%------------------------------
%anekakalyāṇapūrṇaṃ sahasradalan      ekaṃ kamalaṃ  varttate / \E
%anekakalyāṇapūrṇaṃ sahasradalaṃ      ekaṃ kamalaṃ  vartate    \P
%anekakalyāṇapūrṇa--sahasradalaṃ      ekaṃ kamalaṃ  vartato    \B
%anekakalyāṇapūrṇaṃ sahasradalaṃ      ekaṃ kamalaṃ  vartate    \L
%anekakalyāṇapūrṇaṃ sahasradalaṃ      eka--kamalaṃ  varttate   \D1
%anekakalyāṇapūrṇaṃ sahasradalaṃ      ekaṃ kamalaṃ  vartate    \N1
%anekakalyāṇapūrṇa--sahasradalaṃ      ekaṃ kamalaṃ  varttate    \N2
%anekakalyāṇapūrṇaṃ sahasradalaṃ           kamalaṃ  vartate /   \U1
%anekakalyāṇapūrṇaṃ // sahasradalaṃ   ekaṃ kamalaṃ  vartate / \U2
%Fragezeichen in |nepal ... schreiber Einfügung? 
%------------------------------
%In the middle of the \textit{mahāśūnyacakra} exists one lotus facing upward, very red in color with a thousand petals - an abode of brilliance and wholeness.
%------------------------------
tasya mahāśūnya\app{\lem[wit={ceteri},alt={°cakramadhye}]{cakramadhye}
  \rdg[wit={E,P,B,L}]{°cakrasya madhye}
  \rdg[wit={U2}]{°cakrasya}}
\app{\lem[wit={ceteri},alt={°mukhaṃ}]{ūrdhvamukhaṃ}
  \rdg[wit={E,P,L}]{ūrdhmukham}
  \rdg[wit={U2}]{urdhvamukham}
  \rdg[wit={B}]{ūrdhvamukhem}}
\app{\lem[wit={ceteri}]{\skp{-}atiraktavarṇaṃ}
  \rdg[wit={E,L,B}]{iti raktavarṇaṃ}
  \rdg[wit={P}]{iti raktavarṇa°}
  \rdg[wit={U2}]{ativarṇaṃ}}
sakala\app{\lem[wit={ceteri},alt={°śobhāspadaṃ}]{śobhāspadaṃ}
  \rdg[wit={E}]{°śobhāspadam}
  \rdg[wit={U2}]{°śobhanāsyadaṃ}}
\app{\lem[wit={ceteri}]{anekakalyāṇapūrṇaṃ}
  \rdg[wit={B,N2}]{°pūrṇa°}}
sahasradalaṃ
\app{\lem[wit={ceteri}]{ekaṃ}
  \rdg[wit={D1}]{eka°}
  \rdg[wit={U1}]{\om}}
kamalaṃ
\app{\lem[wit={ceteri}]{vartate}
  \rdg[wit={B}]{vartato}}/
%---------------------
%yasya           parimalo manaso vacaso na gocaraḥ // \E
%yasya           parimalo manasā vacasā na gocaraḥ /  \P
%yasya           parimalo manasā vacasā    gocaraḥ /  \L
%yasya           parimalo manasā vacasā na gocaraḥ /  \B
%yasya           parimalo manasā vacasā na gocaraḥ /  \N1
%yasya           parimalo manasā vacasā na gocara /   \N2
%yasya           parimalo manasā vacasā na gocaraḥ /  \D1
%yasya           parimalo vacasā manasā na gocaraḥ    \U1
%yasya kamalasya parimalo manasā vācā   na gocara ..  \U2
%---------------------
%Whose fragrance is not in range by mind and speech. 
%Dessen Duft ist nicht in Reichweite von Geist und Sprache. 
%---------------------
\app{\lem[wit={ceteri}]{yasya}
  \rdg[wit={U2}]{yasya kamalasya}}
parimalo
\app{\lem[wit={E}]{manaso vacaso}
  \rdg[wit={P,L,B,N1,N2,D1}]{manasā vacasā}
  \rdg[wit={U1}]{vacasā manasā}
  \rdg[wit={U2}]{manasā vācā}
}
\note[type=philcomm, labelb=s22.z4, lem={°manaso vacaso}]{All manuscripts at hand share this usage of the instrumentals. Only the printed edition conjectures the forms into the exspected genitiv. I adopted the variant of the printed edition to arrive at a grammatically correct text.}
\app{\lem[wit={ceteri}]{na}
  \rdg[wit={L}]{\om}
}
\app{\lem[wit={ceteri}]{gocaraḥ}
  \rdg[wit={N2,U2}]{gocara}}/
\ekdpb*{}
%---------------------
%tasya kamalasya madhye trikoṇarūpa-ikā karṇikā varttate /    \E
%tasya kamala----madhye trikoṇārūpā ekā karṇikā varttate/ \P
%tasya kamalasya madhye trikoṇarūpā ekā karṇikā varttate/     \L
%tasya kamalasya madhye trikoṇarūpā ekā karṇikā varttate/     \B
%tasya kamalasya madhye trikoṇarūpā eka karṇikā varttate/     \N1
%tasya kamalasya madhye trikoṇarūpā eka karṇikā varttate/     \N2
%tasya kamalasya madhye trikoṇarūpā ekā karṇikā varttate/     \D1
%tasya kamalasya madhye trikoṇarūpā ekā karṇikā vartate       \U1
%tasya kamalasya madhye trikoṇarūpā ekā karṇikā vartate //    \U2
%---------------------
%In the middle of this lotus exists one pericarp having the shape of a triangle. 
%------------------------------
tasya
\app{\lem[wit={ceteri}]{kamalasya}
  \rdg[wit={P}]{kamala°}}
madhye
\app{\lem[wit={E}]{trikoṇarūpaikā}
  \rdg[wit={ceteri}]{trikoṇārūpā ekā}
  \rdg[wit={N1,N2}]{trikoṇārūpā eka}}
karṇikā vartate\dd{}
%------------------------------
%tatkarṇikāmadhye saptadaśī         niraṃjanarūpā kalā varttate/ \E
%tatkarṇikāmadhye saptadaśireṇa ekā niraṃjanarūpā kalā vartate// \L
%tatkarṇikāmadhye saptadaśireṇa ekā niraṃjanarūpā kalā vartate// \B
%tatkarṇikāmadhye saptadaśī     ekā niraṃjanarūpā kalā vartate// \P
%tatkarṇikāmadhye saptadaśī     ekā niraṃjanarūpā kalā vartate// \N1
%tatkarṇikāmadhye saptadaśī     ekā niraṃjanarūpā kalā vartate/  \N2
%tatkarṇikāmadhye saptadaśī     ekā niraṃjanarūpā kalā vartate// \D1
%tatkarṇikāmadhye saptadaśī     ekā niraṃjanarūpā kalā vartate  \U1
%tatkarṇikāmadhye saptadaśī     eka niraṃjanarūpā kalā varttate/ \U2
%---------------------
%In the middle of the pericarp exists one seventeenth digit in the shape of a immaculé form.
%---------------------
tatkarṇikāmadhye
\app{\lem[wit={ceteri}]{saptadaśī}
  \rdg[wit={L,B}]{saptadaśireṇa}}\note[type=philcomm, labelb=s22.z4, lem={saptadaśī}]{A \textit{saptadaśī kalā} appears frequently in Śaiva literature. References need to be added here.}
\app{\lem[wit={ceteri}]{ekā}
  \rdg[wit={E}]{\om}}
nirañjanarūpā kalā varttate/
%---------------------
%koṭisūryasamaprabhaṃ kalāyās tejo vartate /    \E
%koṭisūryasamaprabhā kalāyās tejo vartate /     \L
%koṭisūryasamaprabhā kalāyās tejo vartate /     \B
%koṭisūryasamaprabha kalāyās tejo vartate /     \P
%koṭisūryasamaprabhaṃ kalāyās tejo vartate /    \N1
%koṭisūryasamaprabhaṃ kalāyā  tejo varttate //  \N2
%koṭisūryasamaprabhaṃ kalāyās tejo vartate /    \D1
%koṭisūryasadṛṣaprabhaṃ kalāyās tejo vartate /  \U1
%koṭisūryasamaprabhā // kalāyās tejo varttate / \U2
%---------------------
%A light of the part exists shining like a thousand suns. 
%------------------------------
koṭisūrya\app{\lem[alt={°samaprabhaṃ}, wit={ceteri}]{samaprabhaṃ}
  \rdg[wit={L,B,U2}]{samaprabhā}
  \rdg[wit={P}]{samaprabha}
  \rdg[wit={U1}]{sadṛṣaprabhaṃ}}
kalāyās-tejo vartate/
%------------------------------
%param udbhavo nāsti /     \E
%parim uṣṇabhavo nāsti /   \P
%parim uṣṇabhavo nāsti /   \L
%parim uṣṇabhavo nāsti /   \B
%parim uṣṇabhāvo nāsti /   \N1
%para  uṣṇabhāvo nāsti     \N2
%parim auṣṇabhāvo nāsti /  \D1
%paraṃ uṣṇabhāvo nāsti     \U1
%param uṣṇabhāvo nāsti /   \U2
%---------------------
%[But] excessive heat is not arising. 
%------------------------------
\app{\lem[alt={param},wit={E,U1,U2}]{param\skm{-u}}
  \rdg[wit={U1}]{paraṃ}
  \rdg[wit={N2}]{para}
  \rdg[wit={ceteri}]{parim}
}\app{\lem[wit={ceteri}, alt={uṣṇabhāvo}]{\skp{-u}ṣṇabhāvo}
  \rdg[wit={P,L,B}]{uṣṇabhavo}
  \rdg[wit={D1}]{auṣṇabhāvo}
  \rdg[wit={E}]{udbhavo}
}
nāsti/
%------------------------------
%koṭicandrasamaprabhā    śītalaṃ paraṃ   śītabhāvo   nāsti / \E
%koṭicandrasamaprabhā    śītalaṃ paraṃ   śītabhavo   nāsti / \P
%\om /                                                      \L
%koṭicandrasamaprabhā    śītalaṃ paraṃ   śītabhavo   nāsti / \B
%koṭicandrasamaprabhaṃ   śītalaparaṃ         bhavo   nāsti / \N1
%koṭicandrasamaprabhaṃ   śītalapara----------bhavo   nāsti // \N2
%koṭicaṃdrasamaprabhaṃ   śītalaparaṃ         bhavo   nāsti / \D1
%koṭicaṃdrasamaṃ prabhaṃ śītalaṃ paraṃ       bhavo   nāsti / \U1
%koṭicaṃdrasamaprabhā    śītalaṃ paraṃ śītalabhāvo   nāsti / \U2
%---------------------
%Shining like a thousand moons, excess of cold is not arising.
%---------------------
koṭicandra\app{\lem[alt={°samaprabhaṃ},wit={N1,N2,D1}]{samaprabhaṃ}
  \rdg[wit={E,P,B,U2}]{°samaprabhā}
  \rdg[wit={U1}]{°samaṃ prabhaṃ}
  \rdg[wit={L}]{\om}}
\app{\lem[wit={N1,D1}]{śītalaparaṃ}
  \rdg[wit={ceteri}]{śītalaṃ paraṃ}
  \rdg[wit={N2}]{śītalapara}
  \rdg[wit={L}]{\om}}
\app{\lem[wit={ceteri}]{bhāvo} 
  \rdg[wit={E,P,B}]{śītabhāvo}
  \rdg[wit={U2}]{śītalabhāvo}
  \rdg[wit={L}]{\om}}
nāsti/
% \end{prose}
%\end{ekdosis}
%%%%%%%%%%%%%
%%%%%%%%%%%%
%%%%%%%%%%%%
%%%%%%%%%%%%
%%%%%%%%%%%%
% \begin{ekdosis}
% \ekddiv{type=ed}
% \begin{prose} 
%------------------------------
%asyāḥ kalāyā   dhyānayogāt    sādhakasya manasi duḥkhaṃ na bhavati / \E
%asyāḥ kalādhyānayogāt         sādhakasya manasi duḥkhaṃ na bhavati / \P

%asyāḥ kalāyāḥ  dhyānakaraṇāt  sādhakasya manasi duḥkhaṃ na bhavati / N1
%asyā kalāyā    dhyānakaraṇāt  sādhaka----manasi duḥkhaṃ na bhavati / N2
%asyāḥ kalāyāḥ  dhyānakaraṇāt  sādhakasya manasi duḥkhaṃ na bhavati / D1
%
%asyāḥ kalāyā   dhyānayogāt    sādhakasya manasi duḥkhaṃ bhavati /B
%asyāḥ kalāyā   dhyānayogāt    sādhakasya manasi duḥkhaṃ bhavati /L
%asyāḥ kalāyā   dhyānakaraṇāt/ sādhakasya manasi duḥkhaṃ na bhavati / U1
%asyā  kalāyāḥ  dhyānayogāt//  sādhakasya manasi duḥkhaṃ na bhavati // \U2
%atrastāne 'haṃ devatā// sohaṃ śaktiḥ// ātmāṛṣiḥ// mokṣamārhaḥ// haṃbhrahmordhaṃ// haṃcakra iti// agnicakre sakaro bhavatī// prāṇīrūḍho bhave jjīva ārohaty avarohati bhavaguhāsthānaṃ pitavarṇaṃ// koṭisūryapratikāśaṃ tejaḥ sadoditaprabhā śīvodevatā// mūlamāyā śaktiḥ// hara ātmālayāvsthā dhvanisthirānādātmako khaṃḍa 'dhvani// adhorāmudrā// mūlamāyā// prakṛtidehaḥ// vāṅmanogocaraḥ// niḥprapaṃcaḥ// niḥsaṃśayaḥ// nistaraṃganirlepalakṣaṃ laya// dhyānasamādhi 
%---------------------
%asyāḥ kalāyā dhyānakaraṇāt\varc{\emend kalāyāḥ dhyānakaraṇāt \nepal \dehlia}{kalāyā dhyānayogāt \nepal \dehlia kalādhyānayogāt \pune} sādhakasya manasi duḥkhaṃ na\varc{na \edprint \pune \nepal \dehlia}{\om \oxford \lalchand} bhavati /
%Due to the exercise of meditation upon the digit suffering does not arise in the mind of the practitioner (anymore). 
%------------------------------
\app{\lem[wit={ceteri}]{asyāḥ}
  \rdg[wit={N2,U2}]{asyā}}
\app{\lem[wit={N2,U1}]{kalāyā dhyānakaraṇāt}
  \rdg[wit={N1,D1}]{kalāyāḥ dhyānakaraṇāt}
  \rdg[wit={E,B,L}]{kalāyā dhyānayogāt}
  \rdg[wit={U2}]{kalāyāḥ dhyānayogāt}
  \rdg[wit={P}]{kalādhyānayogāt}}
\app{\lem[wit={ceteri}]{sādhakasya}
  \rdg[wit={N2}]{sādhaka°}}
duḥkhaṃ
\app{\lem[wit={ceteri}]{na}
  \rdg[wit={B,L}]{\om}}
bhavati/
% \end{prose}
%\end{ekdosis}
%%%%%%%%%%%%%
%%%%%%%%%%%%
%%%%%%%%%%%%
%%%%%%%%%%%%
%%%%%%%%%%%%
% \begin{ekdosis}
% \ekddiv{type=ed}
% \begin{prose} 
\extra{atra
   \app{\lem[type=emendation, resp=egoscr]{sthāne}
    \rdg[wit={U2}]{\korr stāne}} 'haṃ devatā\dd{}
  sohaṃ śaktiḥ\dd{}
  ātmāṛṣiḥ\dd{}
  \app{\lem[type=emendation, resp=egoscr]{mokṣo}
    \rdg[wit={U2}]{\korr mokṣa°}} mārgaḥ\dd{}
   \app{\lem[type=emendation, resp=egoscr]{ahaṃ brahmordhvaṃ}
    \rdg[wit={U2}]{\korr haṃ brahmordhaṃ}}\dd{}
   \app{\lem[type=emendation, resp=egoscr]{ahaṃ cakra iti}
     \rdg[wit={U2}]{\korr haṃcakra iti}}\dd{}
   agnicakre
   \app{\lem[type=emendation, resp=egoscr]{sakāro}
     \rdg[wit={U2}]{\korr sakaro}}
   \app{\lem[type=emendation, resp=egoscr]{bhavati}
     \rdg[wit={U2}]{\korr bhavatī}}\dd{}
   prāṇī rūḍho bhavej-jīva ārohaty-avarohati\note[type=philcomm, labelb=sX.zX, lem={prāṇī}]{Find parallels of hemistich.}\dd{}
bhavaguhā sthānaṃ\dd{}
   \app{\lem[type=emendation, resp=egoscr]{pitaṃ}
     \rdg[wit={U2}]{\korr pita°}} varṇaṃ\dd{}
   koṭisūryapratikāśaṃ tejaḥ\dd{}
   \app{\lem[type=emendation, resp=egoscr]{sadoditā}
     \rdg[wit={U2}]{\korr sadodita°}} prabhā\dd{}
   \app{\lem[type=emendation, resp=egoscr]{śivo}
     \rdg[wit={U2}]{\korr śīvo}} 
   devatā\dd{}
   mūlamāyā śaktiḥ\dd{}
   \app{\lem[type=emendation, resp=egoscr]{harātmālayāvasthā}
     \rdg[wit={U2}]{\korr hara ātmālayāvasthā}}\dd{}
   dhvanisthirānādātmako \app{\lem[type=emendation, resp=egoscr]{'khaṇḍadvaniḥ}
     \rdg[wit={U2}]{\korr khaṃḍadhvani}}\dd{} 
   aghorā mudrā\dd{}
   \app{\lem[type=emendation, resp=egoscr]{mūlā} %macht diese emdendation wirklich Sinn? 
     \rdg[wit={U2}]{\korr mūla°}} māyā\dd{}
   \app{\lem[type=emendation, resp=egoscr,alt={prakṛtir}]{prakṛti\skp{r-}}
     \rdg[wit={U2}]{\korr prakṛti°}}\skm{r-}dehaḥ\dd{}
   vāṅmano 'gocaraḥ\dd{} %%
   niḥprapañcaḥ\dd{}
   niḥsaṃśayaḥ\dd{}
   nistaraṃganirlepalakṣaṃ %%%see pw Vol. 3, S. 229 for nistaranga
  \app{\lem[type=emendation, resp=egoscr]{layo}
     \rdg[wit={U2}]{\korr laya}}
   \app{\lem[type=emendation, resp=egoscr]{dhyānaḥ samādhiḥ}
     \rdg[wit={U2}]{\korr dhyānasamādhi}}\dd{}}
 %\extra{Here at this location the ``I''(\textit{aham}) is the deity. The ``he is I'' (\textit{so 'ham}) is the power. This self is the Ṛṣi. The path is liberation. Brahma is the I above. ``I'm a circle''. In the circle of fire is the letter "sa". [There?] life arises, the living soul ascends and decends. The place is the hidden place of being. The colour is yellow. The light is the shine of ten million suns. The shine is always and visible. Śiva is the deity. The power is primordial illusion. The state is the dissolution of the self into Hara\footnote{Epiphet of Śiva.}. The transcendental sound has the nature of a sound with stable resonance. The seal is the ``fearless''. The illusion is the root. The body is the original matter. It is not within reach of speech and mind. It is without delusion. It is without doubt. The unaffected and undefiled goal is dissolution, meditation [and] final absorption.}
%\end{prose}
%\end{ekdosis}
   %%%%%%%%%%%%%%
   %%%%%%%%%%%%%%%
   %%%%%%%%%%%%%%%
   %%%%%%%%%%%%%%
   %%%%%%%%%%%%%%
%  \begin{ekdosis}
%    \ekddiv{type=ed}
%    \begin{prose}
%---------------------
%tadupari anaṃtaparamānandasya sthānam / \E
%tadupari anaṃtaparamānandasya sthānaṃ   \P
%tadupari anantaparamānaṃdasya sthānam / \N1
%tadupari anantaparamānaṃdasya sthānam / \N2
%tadupari anantaparamānaṃdasya sthānaṃ / \D1
%tadupari anantaparamānaṃdasya sthānam vartate/ \B
%tadupari anaṃtaparamānaṃdasya sthānam vartate/ \L
%tadupari alakṣaparamānaṃdasya sthānam   \U1
%tadupari anaṃtaparamānaṃdasya sthānaṃ// U2
%---------------------
%Above that is the place of infinite supreme bliss.
%---------------------
tadupari
\app{\lem[wit={ceteri}, alt={ananta°}]{ananta}
  \rdg[wit={U1}]{alakṣa°}}paramānaṃdasya
\app{\lem[wit={ceteri}]{sthānam}
  \rdg[wit={D1,U2}]{stānaṃ}
  \rdg[wit={B,L}]{sthānam vartate}}/
%---------------------
%tatrordhvaśaktiḥ / \E
%tatordhvaśaktiḥ \P
%rdhaśakti ardhaśakti \B
%rdhaśakti ardhaśakti \L
%tatrordhvaśaktiḥ / \N1
%tatra ūrdhva śaktiḥ / \D1
%tatra ūrdhva śakti / \N2
%urdhvaśaktir         \U1
%tatrordhvaśaktiḥ// \U2
%---------------------
%There above is \textit{śakti},
%------------------------------
\app{\lem[wit={E,N1,U2}]{tatrordhvaśaktiḥ}
  \rdg[wit={P}]{tatordhvaśaktiḥ}
  \rdg[wit={U1}]{urdhvaśaktir}
  \rdg[wit={D1}]{tatra ūrdhva śaktiḥ}
  \rdg[wit={N2}]{tatra ūrdhva śakti}
  \rdg[wit={B,L}]{rdhaśakti ardhaśakti}}/
%------------------------------
%etādṛśī  saṃjñā   ekā kalā vartate / \E
%ekādaśā  saṃjñā   ekā kalā vartate   \P
%etādṛśī  saṃjñā   ekā kalā vartate /  \N1
%etādṛśī  saṃjñā   ekā kalā varttate / \N2
%etādṛsaṃ saṃjñā   ekā kalā vartate / \D1
%ekādaśā  saṃjñā   ekā kalā vartate / \B
%ekādaśā  saṃjñā   ekā kalā vartate / \L
%etādṛśī  saṃjñakā ekā kalā vartate /  \U1
%etādṛśā  saṃjñā   ekā kalā vartate/ \U2 
%---------------------
%Being designated as such she is one single digit. 
%------------------------------
\app{\lem[wit={ceteri}]{etādṛśī}
  \rdg[wit={U2}]{etādṛśā}
  \rdg[wit={D1}]{etādṛsaṃ}
  \rdg[wit={P,B,L}]{ekādaśā}}
\app{\lem[wit={ceteri}]{saṃjñā}
  \rdg[wit={U1}]{saṃjñakā}}
ekā kalā vartate/ 
%------------------------------
%asyāḥ  kalāyā   dhyānakāraṇāt     puruṣo yadicchati / \E
%asyāḥ  kalāyā   dhyānakāraṇāt     puruṣo yadicchati ?Zeichen? \P
%asyāḥ  kalāyā   dhyānakāraṇāt     puruṣo yadicchati  tad bhavati \N1
%tasyāḥ kalāyāḥ  dhyānakāraṇāt     puruṣo yadicchati  tad bhavati \N2
%asyāḥ  kalāyā   dhyānakāraṇā      puruṣo yadicchati  tad bhavati \D1
%asyāḥ  kalāyā   dhyānakāraṇāt /   puruṣo yadicchati / \B
%asyāḥ  kalāyā   dhyānakāraṇāt /   puruṣo yadicchati / \L
%asyā   kalāyā   dhyānakāraṇāt     puruṣo yadicchati tad bhavati vā \U1
%asyāḥ  kalāyāḥ  dhyānakāraṇāt //  puruṣo yadicchati // \U2
%---------------------
%Due to the exercise of meditation on this part the person manifests whatever he wishes for.
%------------------------------
\app{\lem[wit={ceteri}]{asyāḥ}
  \rdg[wit={U1}]{asyā}
  \rdg[wit={N2}]{tasyāḥ}}
\app{\lem[wit={ceteri}]{kalāyā}
  \rdg[wit={N2,U2}]{kalāyāḥ}}
\app{\lem[wit={ceteri}]{dhyānakāraṇāt}
  \rdg[wit={D1}]{dhyānakāraṇā}}
puruṣo yad-icchati
\app{\lem[wit={N1,N2,D1}, alt={tad bhavati}]{tad-bhavati}
  \rdg[wit={U1}]{tad bhavati vā}
  \rdg[wit={ceteri}]{\om}}/ 
%------------------------------
%tasya sukhabhogavataḥ / \E
%tasya sukhabhogavataḥ \P
%rājya-sukhabhogavataḥ \N1
%rājya-sukhabhogavataḥ \N2
%rājya-sukhabhogavṛtaḥ \D1 !!!
%tasya-khaṃ bhogavataṃ / \B
%tasya-sukhaṃ bhogavaṃtaṃ / \L
%rājya-sukhabhogavataḥ \U1
%tasya-sukhabhogavataḥ / \U2
%---------------------
%He is furnished with royal pleasure and enjoyment. 
%------------------------------
\app{\lem[wit={D1}]{rājyasukhabhogavṛtaḥ}
  \rdg[wit={N1,N2,U1}]{rājyasukhabhogavataḥ}
  \rdg[wit={E,P,U2}]{tasya sukhabhogavataḥ}
  \rdg[wit={B}]{tasya-khaṃ bhogavataṃ}
  \rdg[wit={L}]{tasya-sukhaṃ bhogavaṃtaṃ}}/
%------------------------------
%strīmadhye     vilāsavataḥ    saṃgītavilāsavataḥ vinodaprekṣāvataḥ        puruṣasya pratidinaṃ śuklapakṣe candrakalāvat   kalā     vardhate/   \E
%strīmadhye     vilāsavataḥ    saṃgītavinodaprekṣāvataḥ              eva   puruṣasya pratidinaṃ śuklapakṣe candrakalāvat   kalā     vardhate /  \P
%strīmadhye     vilāsavaṃtaṃ   saṃgītaṃ prekṣāvatāḥ //               evaṃ  puruṣasya pratidinaṃ śuklapakṣe caṃdrakalāvat / kalā     vartate /   \L
%strīmadhye     vilāsavaṃtaṃ   saṃgītaṃ vinodavaṃtaṃ prekṣāvaṃtāḥ // eva   puruṣasya pratidinaṃ śuklapakṣe caṃdrakalāvat / kalā     vartate /   \B
%strīmadhye     vilāsavataḥ    saṃgītavinodaprekṣyāvataḥ             evaṃ  puruṣasya pratidinaṃ śuklapakṣe candrakalā vṛddhivato?   vardhate / \N1
%śrī strīmadhye vilāsavataḥ    saṃgītavinodaprekṣāvataḥ              evaṃ  puruṣasya pratidinaṃ śuklapakṣa candrakalā vṛddhi vaṃto  varttate /  \N2
%strīmadhye     vilāsavataḥ // saṃgītavinodaprekṣyāvataḥ //          evaṃ  puruṣasya pratidinaṃ śuklapakṣe candrakalā vṛddhivato    vardhate / \D1
%strīmadhye     vilāśavataḥ    saṃgītavinodaprekṣyāvataḥ             eka   puruṣasya pratidinaṃ śuklapakṣe caṃdrakalā vṛddhir       varddhate / \U1
%strīmadhye     vilāsavata     saṃgītavinodaprekṣāvata//             evaṃ  puruṣasya pratidinaṃ śuklapakṣe candrakalāvat   kalā     varttate/   \U2
%---------------------
%(Selbst) bei einem Menschen, der sich inmitten von Frauen vergnügt, (und) ein Musikvergnügen
%ansieht, wächst täglich die Kraft (kalā = śakti?) wie die "kalā" (Phase) des Mondes in der hellen Monatshälfte.
%The \textit{kalā} of a person grows daily, like the \textit{kalā} of the moon in the bright half of the month, even amusing oneself amongst women and watching a musical pleasure.
%(Even) amusing oneself amongst women, and watching musical pleasures, the \textit{kāla} of the person grows daily like the \textit{kalā} of the moon in the bright half of the month. 
%------------------------------
\app{\lem[wit={ceteri}]{strīmadhye}
  \rdg[wit={N2}]{śrī strīmadhye}}
\app{\lem[wit={ceteri}]{vilāsavataḥ}
  \rdg[wit={U2}]{vilāsavata°}
  \rdg[wit={L,B}]{vilāsavaṃtaṃ}} 
\app{\lem[wit={N1,D1,U1}]{saṃgītavinodaprekṣyāvataḥ}
  \rdg[wit={P,N2}]{saṃgītavinodaprekṣāvataḥ}
  \rdg[wit={U2}]{saṃgītavinodaprekṣāvata}
  \rdg[wit={B}]{saṃgītaṃ vinodavaṃtaṃ prekṣāvaṃtāḥ}
  \rdg[wit={E}]{saṃgītavilāsavataḥ vinodaprekṣāvataḥ}
  \rdg[wit={L}]{saṃgītaṃ prekṣāvatāḥ}}
 \app{\lem[wit={P,B}]{eva}
  \rdg[wit={ceteri}]{evaṃ}
  \rdg[wit={U1}]{eka}}
puruṣasya pratidinaṃ śuklapakṣe
candrakalā\app{\lem[wit={E,P,L,B,U2},alt={°vat kalā}]{vat kalā}
  \rdg[wit={N1,D1}]{vṛddhivato}
  \rdg[wit={N2}]{vṛddhi vaṃto}
  \rdg[wit={U1}]{vṛddhir}}
\app{\lem[wit={E,P,N1,D1,U1}]{vardhate}
  \rdg[wit={ceteri}]{vartate}}/  
%------------------------------
%puṇyapāpe  'sya śarīraṃ   na spṛśataḥ /    \E
%\om                                     \P
%puṇyapāpe  asya śarīrena     spṛśataḥ /      \N1
%puṇyapāpe  asya śarīrena     spṛśataḥ /      \N2
%puṇyapāpe  asya śarīrena     spṛśataḥ /      \D1
%puṇyapāpe  asya śarīrasya na spṛśataḥ // \B
%puṇyapāpe  asya śarīrasya na spṛśataḥ // \L
%puṇyapāpau asya śarīrena     spṛśāt         \U1
%puṇyapāpe  asya śarīraṃ   na spṛśataḥ // \U2
%---------------------
%puṇyapāpe\varc{puṇyapāpe \edprint \lalchand \oxford \nepal \dehlia}{\om \pune} 'sya\varc{'sya \edprint}{asya \nepal \dehlia \oxford \lalchand \om \pune} śarīrasya\varc{śarīrasya \lalchand \oxford}{śarīraṃ \edprint śarīrena \nepal \dehlia \om \pune} na\varc{na \edprint \oxford \lalchand}{\om \nepal \dehlia \pune} spṛśataḥ\varc{spṛśataḥ \edprint \lalchand \oxford \nepal \dehlia}{\om \pune} /
%---------------------
%His body is not affected by merit and sin. 
%------------------------------
\app{\lem[wit={ceteri}]{puṇyapāpe}
  \rdg[wit={U1}]{puṇyapāpau}
\rdg[wit={P}]{\om}}
\app{\lem[wit={E}]{'sya}
  \rdg[wit={P}]{\om}
  \rdg[wit={ceteri}]{asya}}  
\app{\lem[wit={B,L}]{śarīrasya}
  \rdg[wit={N1,N2,D1,U1}]{śarīrena}
  \rdg[wit={E,U2}]{śarīraṃ}
  \rdg[wit={P}]{\om}}
\app{\lem[wit={E,B,L,U2}]{na}
  \rdg[wit={N1,N2,D1,U1,P}]{\om}}
\app{\lem[wit={ceteri}]{spṛśataḥ}
  \rdg[wit={U1}]{spṛśāt}}/
%------------------------------
%                          nirantaradhyānakaraṇāt     nijasvarūpaṃ prakāśanasāmarthyaṃ bhavati / \E
%                          \om until .....            nijasvarūpaprakāśasāmarthyaṃ     bhavati / \P
%                          niraṃtaraṃ dhyānakaraṇāt   nijasvarūpaprakāśasāmarthyaṃ     bhavati / \B
%                          niraṃtaraṃ dhyānakaraṇāt// nijasvarūpaprakāśasāmarthyaṃ     bhavati / \L
%                          nirantaradhyānakaraṇāt /   nijasvarūpaprakāśasāmarthyaṃ     bhavati / \N1 <-----
%                          niraṃtaradhyānakaraṇāt /   nijasvarūpaprakāśasāmarthyaṃ     bhavati // \N2
%                          nirantaradhyānakaraṇāt /   nijasvarūpaprakāśasāmarthyaṃ     bhavati / \D1
%                          nirantaradhyānakaraṇāt /   nijasvarūpaprakāśasāmarthyaṃ     bhavati    \U1
%evaṃ puruṣasya pratidinaṃ niraṃtaraṃ dhyānakaraṇāt   nijasvarūpaṃ prakāśanasāmarthyaṃ bhavati// \U2 
%---------------------
%Due to uninterrupted meditation the power of the light of the innate nature arises. 
%------------------------------
\app{\lem[wit={ceteri}]{nirantaradhyānakaraṇāt}
  \rdg[wit={B,L}]{niraṃtaraṃ dhyānakaraṇāt}
  \rdg[wit={U2}]{evaṃ puruṣasya pratidinaṃ niraṃtaraṃ dhyānakaraṇāt}
  \rdg[wit={P}]{\om}}
\app{\lem[wit={ceteri}]{nijasvarūpaprakāśasāmarthyaṃ}
  \rdg[wit={E,U2}]{nijasvarūpaṃ prakāśanasāmarthyaṃ}}
bhavati/
%------------------------------
%dūrasthopi ca dūrasthavastu                   samīpa iva   paśyati // \E
%dūrasthamapi                                  samīpam iva  paśyati // \N1
%dūrasthamapi                                  samīpaṃ iva  paśyati // \N2
%dūrasthamapy-arthaṃ                           samīpa iva   paśyati // \D1
%dūrasthamapi padārthaṃ                        samīpa iva   paśyati // \B
%dūrasthamapi parārthaṃ                        samīpa iva   paśyati // \L
%dūrasthamapi padārthaṃ                        samīpa iva   paśyati // \P
%dūrasthamapy-arthaṃ                           samīpam eva   paśyati // \U1
%dūrasthamapi bhavati //dūrasthamapi padārthaṃ samīpa iva   paśyati// \U2
%------------------------------
%dūrasthamapyarthaṃ\varc{dūrasthamapyarthaṃ \dehlia}{dūrasthamapi padārthaṃ \oxford \pune durasthamapi parārthaṃ \lalchand sūrastamapi \nepal ca dūrasthavastu \edprint} samīpa\varc{samīpa \dehlia \edprint \lalchand \oxford \pune}{samīpam \nepal} iva paśyati //
%------------------------------
%He sees remotely located objects as if they'd be near.
%------------------------------
\app{\lem[wit={D1,U1},alt={dūrastham apy arthaṃ}]{dūrastham-apy-arthaṃ}
  \rdg[wit={B,P}]{dūrastham api padārthaṃ}
  \rdg[wit={L}]{dūrastham api parārthaṃ}
  \rdg[wit={E}]{dūrasthopi ca dūrasthavastu}
  \rdg[wit={N1,N2}]{dūrastham api}
  \rdg[wit={U2}]{dūrastham api bhavati || dūrastham api padārthaṃ}}
\app{\lem[wit={ceteri}]{samīpa iva}
  \rdg[wit={N1}]{samīpam iva}
  \rdg[wit={N2}]{samīpaṃ iva}
  \rdg[wit={U1}]{samīpam eva}}
paśyati\dd{}
%------------------------------
\end{prose}
\end{ekdosis}
\ekdpb*{}
%%%%%%%%%%%%%%%%
%%%%%%%%%%%%%%%%
%%%%%%%%%%%%%%%%
%%%%%%%%%%%%%%%%
%%%%%%%%%%%%%%%
\begin{ekdosis}
 \ekddiv{type=ed}
   \bigskip
    \centerline{\textrm{\small{[Lakṣyayoga, the yoga of fixation]}}}
    \bigskip
    \begin{prose}
%------------------------------
%idānīṃ sukhasādhyo lakṣyayogaḥ kathyate / \E
%idānīṃ sukhasādho  lakṣyayogaḥ kathyate / \P
%idānīṃ sukhasādho  lakṣayogaḥ  kathyate / \B
%idānīṃ sukhasādhe  lakṣayogaḥ  kathyate // \L
%idānīṃ sukhasādhyo lakṣyayogaḥ kathyate / \N1
%idānīṃ sukhasādhya lakṣanayogaḥ kathyate / \N2
%idānīṃ sukhasādhyo lakṣyayogaḥ kathyate / \D1
%idānīṃ sukhasādhyopalakṣayogaḥ kathyate / \U1
%idānīṃ sukhasādhyo lakṣyayogaḥ kathyate / \U2
%------------------------------
%Now the yoga of fixation{\textit{lakṣyayoga}}, which is easily accomplished is explained. 
%------------------------------
      idānīṃ
      \app{\lem[wit={ceteri}]{sukhasādhyo}
        \rdg[wit={N2}]{°sādhya}
        \rdg[wit={P,B}]{°sādho}
        \rdg[wit={L}]{°sādhe}
        \rdg[wit={U1}]{°sādhyopa°}}
    \app{\lem[wit={ceteri}]{lakṣyayogaḥ}
        \rdg[wit={B,L}]{lakṣayogaḥ}
        \rdg[wit={U1}]{°lakṣayogaḥ}
        \rdg[wit={N2}]{lakṣanayogaḥ}}
      kathyate/
%------------------------------      
%asya lakṣyayogasya  paṃcabhedā     bhavanti   ūrdhvalakṣyam / adholakṣyam / lakṣyam /      bāhyalakṣyam /  aṃtaralakṣyam /  \E
%asya lakṣyayogasya  paṃcabhedā     bhavanti   ūrdhvalakṣyam   adholakṣyam / madhyalakṣyam  bāhyalakṣyam    aṃtaralakṣyam /  \P
%asya lakṣayogasya   paṃce bhedāḥ   bhavaṃtī   ūrdhvalakṣam//  adholakṣam// bāhyakṣam//                     aṃtaralakṣam //  \B
%asya lakṣayogasya   paṃcabhedāḥ    bhavaṃti   ūrdhvalakṣam    adholakṣam// madhyalakṣam//  bāhyakṣam//     aṃtaralakṣam //  \L
%     lakṣyayogasya  paṃcabhedā     bhavaṃti// urdhvalakṣya    adholakṣya   bāhyalakṣya     madhyalakṣya    antaralakṣya //  \N1
%     lakṣanayogasya paṃcabhedā     bhavati//  urdhvalakṣa     adholakṣa    bāhyalakṣa      madhyalakṣa     antaralakṣa //   \N2
%     lakṣyayogasya  paṃcabhedā     bhavaṃti// urdhvalakṣya    adholakṣya   bāhyalakṣya     madhyalakṣya    antaralakṣya //  \D1
%a----lakṣayogasya   paṃcabhedā     bhavati    urdhvalakṣa                  bāhyalakya      madhyalakṣa     aṃtaralakṣya     \U1
%asya lakṣayogasya   paṃcabhedā     bhavaṃti// ūrdhvalakṣam//  adholakṣam/  bāhyalakṣyam /  madhyalakṣaṃ/   sarvalakṣyam /   \U2
%------------------------------
%Of this yoga of fixation (\textit{lakṣyayoga}) there are five subdivisions: 1. The upward directed fixation {\textit{ūrdhvalakṣya}), 2. the downward directed fixation (\textit{adholakṣya}),3. the central fixation (\textit{madhyalakṣya}) 4. the outer fixation (\textit{baḥyalakṣya}), 5. the inner fixation (\textit{antaralakṣya}).
%------------------------------
      \app{\lem[wit={E,P,B,L,U2}]{asya}
        \rdg[wit={ceteri}]{\om}}
      \app{\lem[wit={ceteri}]{lakṣyayogasya}
        \rdg[wit={B,L,U2}]{lakṣayogasya}
        \rdg[wit={U1}]{alakṣayogasya}
        \rdg[wit={N2}]{lakṣanayogasya}}
      \note[type=philcomm, labelb=s28.z2, lem={lakṣyayogasya}]{The designation of this type of yoga is transmitted in various variants. Given the list of the 15 yogas at the beginning of the text it is very likely that the correct name of the yoga is \textit{lakṣyayoga} and not \textit{lakṣayoga} or \textit{lakṣanayoga}.}
      \app{\lem[wit={ceteri}]{pañcabhedā}
        \rdg[wit={L}]{paṃcabhedāḥ}
        \rdg[wit={B}]{paṃce bhedāḥ}}
     \app{\lem[wit={ceteri}]{bhavanti}
       \rdg[wit={B}]{bhavaṃtī}
       \rdg[wit={N2,U1}]{bhavati}}/
    1 \app{\lem[wit={E,P}]{ūrdhvalakṣyam}
       \rdg[wit={L,B,N2}]{ūrdhvalakṣam}
       \rdg[wit={N1,D1}]{urdhvalakṣya}
       \rdg[wit={N2,U1}]{urdhvalakṣa}}/
    2 \app{\lem[wit={E,P}]{adholakṣyam}
       \rdg[wit={B,L,U2}]{adholakṣam}
       \rdg[wit={N1,D1}]{adholakṣya}
       \rdg[wit={N2}]{adholakṣa}
       \rdg[wit={U1}]{\om}}/
    3 \app{\lem[wit={U2}]{bāhyalakṣyam}
       \rdg[wit={N1,D1}]{bāhyalakṣya}
       \rdg[wit={N2}]{bāhyalakṣa}
       \rdg[wit={U1}]{bāhyalakya}
       \rdg[wit={B}]{bāhyakṣam}
       \rdg[wit={E}]{lakṣyam}
       \rdg[wit={P}]{madhyalakṣyam}
       \rdg[wit={L}]{madhyalakṣam}}/
    4 \app{\lem[type={emendation}, resp={egoscr}]{madhyalakṣyam}
       \rdg[wit={N1,D1}]{\korr madhyalakṣya}
       \rdg[wit={N2,U1}]{madhyalakṣa}
       \rdg[wit={U2}]{madhyalakṣaṃ}
       \rdg[wit={E,P}]{bāhyalakṣyam}
       \rdg[wit={L}]{bāhyakṣam}
       \rdg[wit={B}]{\om}}/
    5 \app{\lem[wit={E,P}]{antaralakṣyam}
       \rdg[wit={N1,D1,U1}]{antaralakṣya}
       \rdg[wit={B,L}]{aṃtaralakṣam}
       \rdg[wit={N2}]{antaralakṣa}
       \rdg[wit={U2}]{sarvalakṣyam}}/
\end{prose}
\end{ekdosis}
%%%%%%%%%%%%%
%%%%%%%%%%%%%
%%%%%%%%%%%%%
%%%%%%%%%%%%%
%%%%%%%%%%%%%
 \begin{ekdosis}
   \ekddiv{type=ed}
   \bigskip
     \centerline{\textrm{\small{[1. Ūrdhvalakṣya - The upward directed fixation]}}}
     \bigskip
\begin{prose}    
%------------------------------      
%prathamam ūrdhvalakṣyaṃ kathyate/  \E
%prathamam ūrdhvalakṣyaḥ kathyate/  \P
%atha      ūrdhvalakṣaṃ          // \L
%athama    urdhalakṣaṃ           // \B
%prathamaṃ urdhvalakṣaḥ  kathyate/  \N1
%prathamaṃ urdhvalakṣaḥ  kathyate/  \N2
%prathamaṃ urdhvalakṣaḥ  kathyate/  \D1
%prathamaṃ urdhvalakṣya/ kathyate/  \U1
%prathamaṃ urdhvalakṣaṃ  kathyate/  \U2
%------------------------------
%At first the upward directed fixation{\textit{adholakṣya} is explained. 
%------------------------------
     \app{\lem[wit={E,P},alt={prathamam}]{prathama\skp{mū}}
       \rdg[wit={N1,N2,D1,U1,U2}]{prathamaṃ}
       \rdg[wit={L}]{atha}
       \rdg[wit={B}]{athama}}\app{\lem[wit={E},alt={ūrdhvalakṣyaṃ}]{\skm{mū}rdhvalakṣyaṃ}
       \rdg[wit={P}]{ūrdhvalakṣyaḥ}
       \rdg[wit={U1}]{urdhvalakṣya}
       \rdg[wit={L}]{ūrdhvalakṣaṃ}
       \rdg[wit={U2}]{urdhvalakṣaṃ}
       \rdg[wit={N1,N2,D1}]{urdhvalakṣaḥ}
       \rdg[wit={B}]{urdhalakṣaṃ}}
     \app{\lem[wit={ceteri}]{kathyate}
       \rdg[wit={L,B}]{\om}}/ 
%------------------------------     
%ākāśamadhye dṛṣṭiḥ / \E
% \om                 \P
%ākāśamadhye dṛṣṭiḥ / \L
%ākāśamadhye dṛṣṭi    \B
%ākāśamadhye dṛṣṭiḥ / \N1
%ākāśamadhye dṛṣṭiḥ / \N2
%ākāśamadhye dṛṣṭiḥ / \D1
%ākāśamadhye dṛṣṭiḥ / \U1
%ākāśamadhye dṛṣṭiḥ / \U2
%------------------------------
%The gaze (\textit{dṛṣṭi)) [should be] in the middle of the sky. 
%------------------------------
  \app{\lem[wit={ceteri}]{ākāśamadhye}
    \rdg[wit={P}]{\om}}
  \app{\lem[wit={ceteri}]{dṛṣṭiḥ}
    \rdg[wit={B}]{dṛṣṭi}
    \rdg[wit={P}]{\om}}/
%------------------------------     
%kadā ca    mana    ūrdhvaṃ      kṛtvā sthāpayati /     \E x
%atha ca    mana    ūrdhvaṃ      kṛtvā sthāpyate /      \P x
%atha vā            ūrdhvaṃ mana kṛtvā sthāpyate        \L
%atha vā            ūrdhvamana   kṛtvā sthāpyate        \B
%atha ca // mana    urdhvaṃ      kṛtvā sthāpyate /      \N1 x
%atha ca mana       ūrdhvaṃ      kṛtvā sthāpyate /      \N2 x
%atha vā mana       ūrdhaṃ       kṛtvā sthāpyate        \D1 x
%atha ca maner------ddhvaṃ       kṛtvā sthāpyate        \U1
%atha    mana       urdhvaṃ      kṛtvā sthāpyate//      \U2 x
%------------------------------
%And then having caused the mind to be directed upwards, it is caused to be fixed there. 
%------------------------------
  \app{\lem[wit={P,N1,N2,U1}]{atha ca}
    \rdg[wit={L,B,D1}]{atha vā}
    \rdg[wit={U2}]{atha}
    \rdg[wit={E}]{kadā ca}}
  \app{\lem[wit={E,P,N2}]{mana ūrdhvaṃ}
    \rdg[wit={N1,U2}]{mana urdhvaṃ}
    \rdg[wit={D1}]{mana ūrdhaṃ}
    \rdg[wit={U1}]{manerddhvaṃ}
    \rdg[wit={L}]{ūrdhvaṃ mana}
    \rdg[wit={B}]{ūrdhvamana}}
  kṛtvā
  \app{\lem[wit={ceteri}]{sthāpyate}
    \rdg[wit={E}]{sthāpayati}}/
\end{prose}
\end{ekdosis}
%%%%%%%%%%%%
%%%%%%%%%%%%
%%%%%%%%%%%%
%%%%%%%%%%%%
%%%%%%%%%%%%
  \begin{ekdosis}
    \ekddiv{type=ed}
    \begin{prose} 
%------------------------------ 
%etasya lakṣyasya  dṛḍhakaraṇāt   parameśvarasya tejasā saha dṛṣṭer-aikyaṃ  bhavati /  \E
%etasya lakṣyasya  dṛḍhakaraṇāt   parameśvarasya tejasā saha dṛṣṭer-aikyaṃ  bhavati /  \P
%etasya lakṣasya   dṛḍhīkṛtvā//   parameśvarasya teja---saha dṛṣṭair-aikā   bhavati //  \L
%etasya lakṣasya   dṛḍhīkṛtvā//   parameśvarasya teja---saha dṛṣṭair-aikā   bhavati //  \B
%etasya lakṣyasya  dṛḍhīkaraṇāt / parameśvarasya tejasā saha dṛṣteḥ aikyaṃ  bhavati /  \N1
%etasya lakṣaṇasya dṛḍhīkaraṇāt   parameśvarasya tejasā saha dṛṣteḥ ekaṃ    bhavati //  \N2
%etasya lakṣasya   dṛḍhīkaraṇāt// parameśvarasya tejasā saha dṛṣṭeḥ aikyaṃ  bhavati // \D1
%etasya lakṣasya   dṛḍhīkaraṇāt/  parameśvarasya tejasā saha dṛṣṭer-aikyaṃ  bhavati/ \U1
%etasya lakṣasya   dṛḍhīkaraṇāt   parameśvarasya tenasā saha dṛṣṭer-aikyaṃ  bhavati // \U2
%------------------------------
%Due to the exercise of stabilizing of this fixation (\textit{lakṣya}) arises unity of the gazing point (\textit{dṛṣṭi}) with the light of the highest lord (\textit{parameśvara}). 
%------------------------------
etasya
  \app{\lem[wit={E,P,N1}]{lakṣyasya}
    \rdg[wit={ceteri}]{lakṣasya}
    \rdg[wit={N2}]{lakṣaṇasya}}
  \app{\lem[wit={ceteri}]{dṛḍhīkaraṇāt}
    \rdg[wit={E,P}]{dṛḍhakaraṇāt}
    \rdg[wit={L,B}]{dṛḍhīkṛtvā}}
  parameśvarasya
\app{\lem[wit={ceteri}]{tejasā}
  \rdg[wit={U2}]{tenasā}
  \rdg[wit={L,B}]{teja°}}
saha
\app{\lem[wit={E,P,U1,U2}]{dṛṣṭer-aikyaṃ}
  \rdg[wit={N1,D1}]{dṛṣṭeḥ aikyaṃ}
  \rdg[wit={N2}]{dṛṣteḥ ekaṃ}
  \rdg[wit={L,B}]{ dṛṣṭair aikā}}
bhavati/  
%------------------------------
%atha cākāśa----madhye    yaḥ kaścidadṛṣṭaḥ   padārtho bhavati /  \E x
%atha cākāśa----madhye    yaḥ kaścidadṛṣṭaḥ   padārtho bhavati /  \P  x
%atha vākāśa----madhye    yaḥ kacciddṛṣṭaḥ    padārtho bhavati    \L  x
%athā cākāśa----madhye    yaḥ kaccit dṛṣṭaḥ   padārtho bhavati    \B   x
%atha ca ākāśa--madhye    yaḥ kaścitadṛṣtaḥ   padārthe bhavati /  \N1   x
%atha// ākāśa---madhye    yaḥ kaścita adṛṣtaḥ padārtha bhavati /  \N2  x 
%atha ca ākāśa--madhye    yaḥ kaścitadṛsṭaḥ   padārtho bhavati /  \D1    x
%atha ca/ ākāśa-madhye    yaḥ kaścidadṛsṭaḥ   padārtho bhavati    \U1    x
%atha cākāśa----madhye    yaḥ kaściddṛsṭa-----padārtho bhavati /  \U2
%------------------------------
%And then an indefinable invisible object arises in the middle of the sky.
%------------------------------
\app{\lem[wit={ceteri}]{atha}
  \rdg[wit={B}]{athā}}
\app{\lem[wit={E,P,B,U2},alt={cākāśa°}]{cākāśa}
  \rdg[wit={N1,D1,U1}]{ca ākāśa°}
  \rdg[wit={L}]{vākāśa°}
  \rdg[wit={N2}]{ākāśa°}}madhye
yaḥ
\app{\lem[wit={ceteri},alt={kaścid adṛṣṭaḥ}]{kaścid\skp{-}adṛṣṭaḥ}
  \rdg[wit={L}]{kaccid dṛṣṭaḥ}
  \rdg[wit={B}]{kaccit dṛṣṭaḥ}
  \rdg[wit={N2}]{kaścita adṛṣtaḥ}
  \rdg[wit={U2}]{kaścid dṛsṭa°}}
\app{\lem[wit={ceteri}]{padārtho}
  \rdg[wit={N1}]{padārthe}
  \rdg[wit={N2}]{padārtha}}
bhavati/ 
%------------------------------
%sa sādhakasya dṛṣṭigocaro bhavati//  \E
%sa sādhakasya dṛṣṭigocaro bhavati//  \P
%   sādhakasya dṛṣṭigocaro bhavati//  \L
%   sādhakasya dṛṣṭigocaro bhavatī    \B
%sa sādhakasya dṛṣṭigocare bhavati // \D1  saḥ-Sonderregel -> ḥ fällt aus vor allen Konsonanten
%sa sādhakasya dṛṣṭigocare bhavati // \N1
%   sādhakasya dṛṣṭigocarā bhavati // \N2
%sa sādhakasya dṛṣṭigocaro bhavati    \U1
%   sādhakasya dṛṣṭigocare bhavati // \U2
%------------------------------
%It arises in the range of sight of the practitioner.  
%------------------------------
\app{\lem[wit={ceteri}]{sa}
  \rdg[wit={L,B,N2,U2}]{\om}}
sādhakasya
\app{\lem[wit={D1,N1,U2}]{dṛṣṭigocare}
  \rdg[wit={ceteri}]{dṛṣṭigocaro}
  \rdg[wit={N2}]{dṛṣṭigocarā}}
\app{\lem[wit={ceteri}]{bhavati}
  \rdg[wit={B}]{bhavatī}}/
%------------------------------
%ayam evordhvalakṣyaḥ      \E
%ayam evordhvalakṣyaḥ      \P
%ayam evordhvalakṣaḥ  //   \L
%ayam evordhalakṣaḥ  //    \B
%ayam evordhvalakṣya  //   \N1
%ayam eva vodhalakṣaṇam // \N2
%ayam evordhvalakṣyaḥ //   \D1
%ayam evordhvalakṣyaḥ      \U1
%ayam evordhvalakṣya //    \U2
%------------------------------
%This is truly the upward directed fixation (\textit{ūrdhvalakṣya}).
%------------------------------
aya\skp{m-e}\app{\lem[wit={E,P,D1,U1},alt={evordhvalakṣyaḥ}]{\skm{m-e}vordhvalakṣayaḥ}
  \rdg[wit={L}]{°lakṣaḥ}
  \rdg[wit={B}]{evordhalakṣaḥ}
  \rdg[wit={N1,U2}]{°lakṣya}
  \rdg[wit={N2}]{eva vodhalakṣaṇam}}/
\end{prose}
\end{ekdosis}
%%%%%%%%%%%%%%%%
%%%%%%%%%%%%%%%
%%%%%%%%%%%%%%%%
%%%%%%%%%%%%%%%%
%%%%%%%%%%%%%%%%%
\begin{ekdosis}
  \ekddiv{type=ed}
   \bigskip
    \centerline{\textrm{\small{[2. Adholakṣya - The downward directed fixation]}}}
    \bigskip
 \begin{prose}
%------------------------------
%                            nāsikāyāḥ  upari     dvādaśāṃgulamūlaparyantaṃ dṛṣṭiḥ sthirā karttavyā /   \E
%       athādholakṣaḥ        nāsikāyā   upari     dvādaśāṃgulaparyantaṃ     dṛṣṭiḥ sthirā karttavyā /   \P
%       athādholakṣaḥ //     nāsikāyā   upari     dvādaśāṃgulaparyaṃtaṃ     dṛṣṭiḥ sthirā karttavyā     \L
%       athādholakṣa //      nāsikāyā   upari     dvādaśāṃgulaparyaṃtaṃ     dṛṣṭiḥ sthirā karttavyā     \B
%       atha adholakṣyaḥ //  nāsikāyā   upari     dvādaśaṃgulaparyaṃtaṃ     dṛṣṭiḥ sthirā karttavyā //  \N1
%       atha adholakṣanaḥ // nāsikāyā   upari     dvādaśāṃgulaparyaṃtaṃ     dṛṣṭiḥ sthirā karttavyā //  \N2
%       atha adholakṣaḥ //   nāsikāyā   upari     dvādaśaṃgulaparyaṃtaṃ     dṛṣṭiḥ sthirā karttavyā //  \D1
%       atha adholakṣa       nāsikāyā   upari     dvādaśaṃgulaparyaṃtaṃ     dṛṣṭi--sthirā karttavyā     \U1
%                            nāsikāyāḥ  upariṣṭāt    daśāṃgulaparyaṃtaṃ     dṛṣṭiḥ sthirā karttavyā //  \U2
%------------------------------
%Now the downward directed fixation object (\textit{adholakṣya}). One should stabilize the gaze within the circumference (\textit{paryanta}) of twelve \textit{aṅgula}s beyond the nose.
%------------------------------
\app{\lem[type=emendation, resp=egoscr]{athādholakṣyaḥ}
  \rdg[wit={N1}]{\korr atha adholakṣyaḥ}
  \rdg[wit={P,L}]{athādholakṣaḥ}
  \rdg[wit={B}]{athādholakṣa}
  \rdg[wit={N2}]{atha adholakṣanaḥ}
  \rdg[wit={D1}]{atha adholakṣaḥ}
  \rdg[wit={U1}]{atha adholakṣa}
  \rdg[wit={E,U2}]{\om}}/
\app{\lem[wit={ceteri}]{nāsikāyā}
  \rdg[wit={E,U2}]{nāsikāyāḥ}}
\app{\lem[wit={ceteri}]{upari}
  \rdg[wit={U2}]{upariṣṭāt}}
\app{\lem[wit={ceteri}]{dvādaśāṃgulaparyantaṃ}
  \rdg[wit={E}]{°mūlaparyantaṃ}
  \rdg[wit={U2}]{daśāṃgulaparyaṃtaṃ}}
\app{\lem[wit={ceteri}]{dṛṣṭiḥ}
  \rdg[wit={U1}]{dṛṣṭi°}}
sthirā karttavyā/
\ekdpb*{}
%------------------------------
%atha vā nāsikāyā agre dṛṣṭiḥ sthirā karttavyā / \E
%atha vā nāsikāyā agre dṛṣṭiḥ sthirā karttavyā / \P
%\om / \L
%\om / \B
%atha vā nāsikāyā  agre dṛṣṭiḥ sthirā karttavyā // \N1
%atha vā nāsikā    agre dṛṣṭi-sthirā karttavyā      \N2
%atha vā nāsikāyā  agre dṛṣṭiḥ sthirā karttavyā // \D1
%atha vā nāśikāyāḥ/ agre dṛṣṭiḥ/ sthirā karttavyā / \U1
%atha vā nāsikāyā  agre dṛṣṭiḥ sthirā karttavyā // \U2
%------------------------------
%Or one should stabilize the gaze onto the tip of the nose.
%------------------------------
\app{\lem[wit={ceteri}]{atha vā}
  \rdg[wit={L,B}]{\om}}
\app{\lem[wit={ceteri}]{nāsikāyā}
  \rdg[wit={U1}]{nāsikāyāḥ}
  \rdg[wit={N2}]{nāsika}}
\app{\lem[wit={ceteri}]{agre}
  \rdg[wit={L,B}]{\om}}
\app{\lem[wit={ceteri}]{dṛṣṭiḥ}
  \rdg[wit={N2}]{dṛṣṭi°}}
\app{\lem[wit={ceteri}]{sthirā}
  \rdg[wit={L,B}]{\om}}
\app{\lem[wit={ceteri}]{karttavyā}
  \rdg[wit={L,B}]{\om}}/ 
%------------------------------
%lakṣadūyasya  dṛḍhīkaraṇāt / dṛṣṭiḥ sthirā bhavati / \E
%lakṣadvayasya dṛṣṭīkaraṇāt / dṛṣṭiḥ sthirā bhavati / \P
%lakṣadvayasya dṛḍhīkaraṇāt   dṛṣṭi--sthiro bhavati / \L
%lakṣadvayasya dṛḍhīkaraṇān---dṛṣṭiḥ sthiro bhavatī   \B
%lakṣadvayasya dṛdhīkaraṇāt   dṛṣṭiḥ sthirā bhavati / \N1
%lakṣadvayasya dṛḍhīkaraṇād---dṛṣṭi--sthirā bhavati / \N2
%lakṣadvayasya dṛḍhīkaraṇāt   dṛṣṭiḥ sthirā bhavati / \D1
%lakṣadvayasya dṛḍhīkaraṇāt   dṛṣṭiḥ sthirā bhavati / \U1
%lakṣadvayasya dṛḍhīkaraṇāt   dṛṣṭi--sthirā bhavati // \U2
%------------------------------
%The fixation becomes stable due to firm exercise [on one] of the twofold aims [of fixation]. 
%------------------------------
\app{\lem[wit={ceteri}]{lakṣadvayasya}   %emend to lakṣyadvayasya??? 
  \rdg[wit={E}]{lakṣadūyasya}} 
\app{\lem[wit={N2}, alt={dṛḍhīkaraṇād}]{dṛḍhīkaraṇād\skm{-dṛ}}
  \rdg[wit={E,L,N1,D1,U1,U2}]{dṛḍhīkaraṇāt}
  \rdg[wit={P}]{dṛṣṭīkaraṇāt}
  \rdg[wit={B}]{dṛḍhīkaraṇān}
}\app{\lem[wit={ceteri}, alt={dṛṣṭiḥ}]{\skp{-dṛ}ṣṭiḥ}
  \rdg[wit={L,N2,U2}]{dṛṣṭi°}}
\app{\lem[wit={ceteri}]{sthirā}  
  \rdg[wit={B}]{sthiro}
  \rdg[wit={L}]{°sthiro}}
\app{\lem[wit={ceteri}]{bhavati}
  \rdg[wit={B}]{bhavatī}}/
%\end{prose}
%\end{ekdosis}
%%%%%%%%%%%%%%%
%%%%%%%%%%%%% 
%%%%%%%%%%%%%%
%%%%%%%%%%%%%%
%%%%%%%%%%%%%%%
%\begin{ekdosis}
%    \ekddiv{type=ed}
%    \begin{prose}
%------------------------------
%pavanaḥ sthiro bhavati / \E
%pavanaḥ sthiro bhavati / \P
%\om                    / \L
%\om                    / \B
%pavanaḥ sthiro bhavati / \N1
%pavana--sthiro bhavati /   \N2
%pavanaḥ sthiro bhavati / \D1
%pavana--sthiro bhavati  / \U1
%pavana--sthiro bhavati  / \U2
%------------------------------
%The breath becomes stable. 
%------------------------------
\app{\lem[wit={E,P,N1,D1}]{pavanaḥ}
  \rdg[wit={N2,U1,U2}]{pavana°}
  \rdg[wit={L,B}]{\om}}
\app{\lem[wit={ceteri}]{sthiro}
  \rdg[wit={L,B}]{\om}}
\app{\lem[wit={ceteri}]{bhavati}
  \rdg[wit={L,B}]{\om}}/
%------------------------------
%āyurvarddhate / \E
%āyurvarddhate / \P
%āyurvarddhate / \L
%āyurvardhate /  \B
%āyurvardhate /  \N1
%āyurvardhate /  \N2
%āyurvardhate /  \D1
%āyurvarddhate   \U1
%āyurvarddhate //  \U2
%------------------------------
%Vitality increases. 
%------------------------------
āyur-varddhate/
%------------------------------
%etad dūyam       api bāhyalakṣyam eva  bhavati      bāhyāṃtara       ākāśe         śūnyalakṣyaṃ    karttavyaḥ / \E
%etad dvayam      api bāhyalakṣyam eva  bhavati      bāhyābhyaṃtare   ākāśe cet     śūnyalakṣyaṃ    karttavyaḥ / \P
%etad dvayam      api bāhyalakṣam  eva  bhavati//    bāhyābhyaṃtare   ākāśacen      śūnyaṃ lakṣaṃ   karttavyā // \L
%etad dvayadvayam api bāhyalakṣam  eva  bhavatī//    bāhyābhyaṃtare   ākāśacvat     śūnyaṃ lakṣaṃ   karttavyā // \B
%etat advayam     eva bāhyalakṣam  api  kathyate //  bāhyo bhyaṃtaraṃ ākāśavat------śūnyalakṣyaḥ    karttavyaḥ / \N1
%etad dvayam      eva bāhyalakṣam  api  kathyate //  bāhyābhyaṃtaram--ākāśavat------śūnyalakṣaḥ     karttavyaḥ   \N2
%etat advayam     eva bāhyalakṣam  api  kathyate //  bāhyo bhyaṃtaraṃ ākāśavat //   śūnyalakṣyaḥ    karttavyaḥ / \D1
%etat dvayam      eva bāhyalakṣyam api  kathyate/    bāhyābhyaṃtare   ākāśavat------śūnyalakṣyaḥ    karttavyaḥ  \U1
%etat dvayam      api bāhyalakṣyam eva  bhavati//    bāhyābhyaṃtare   ākāśe cet     śūnyalakṣyaṃ    karttavyaḥ / \U2
%------------------------------
%Just as this [aim] is twofold, also the external fixation is said to be [like this]. Internally or externally the aim of fixation is to be done onto the heavenly void.  
%------------------------------
\app{\lem[wit={P,L,N2},alt={etad dvayam}]{etad-dvayam\skm{-e}}
  \rdg[wit={E}]{etad dūyam}
  \rdg[wit={B}]{etad dvayadvaya}
  \rdg[wit={N2,D1}]{etat advayam}
  \rdg[wit={U1,U2}]{etat dvayam}}\app{\lem[wit={N1,N2,D1,U1}, alt={eva}]{\skp{-e}va}
  \rdg[wit={ceteri}]{api}} 
\app{\lem[wit={E,P,U1,U2},alt={bāhyalakṣyam}]{bāhyalakṣyam\skm{-a}}
  \rdg[wit={ceteri}]{°lakṣam}}\app{\lem[wit={N1,N2,D1,U1},alt={api}]{\skp{-a}pi}
  \rdg[wit={E,P,L,B,U2}]{eva}}
\app{\lem[wit={N1,N2,D1,U1}]{kathyate}
  \rdg[wit={E,P,L,U2}]{bhavati}
  \rdg[wit={B}]{bhavatī}}/
\app{\lem[wit={N2},alt={bāhyābhyantaram}]{bāhyābhyantaram\skm{-ā}}                %Übersetzung nochmal überdenken! 
  \rdg[wit={N1,D1}]{bāhyo bhyaṃtaraṃ}
  \rdg[wit={P,L,B,U1,U2}]{bāhyābhyaṃtare}
  \rdg[wit={E}]{bāhyāṃtara}}\app{\lem[wit={N1,N2,D1,U1},alt={ākāśavat}]{\skp{-ā}kāśavat}
  \rdg[wit={B}]{ākāśacvat}
  \rdg[wit={L}]{ākāśacen}
  \rdg[wit={P,U2}]{ākāśe cet}
  \rdg[wit={E}]{ākāśe}}
\app{\lem[wit={N1,D1,U1}]{śūnyalakṣyaḥ}
  \rdg[wit={E,P,U2}]{śūnyalakṣyaṃ}
  \rdg[wit={N2}]{śūnyalakṣaḥ}
  \rdg[wit={L,B}]{śūnyaṃ lakṣaṃ}}
\app{\lem[wit={ceteri}]{karttavyaḥ}
  \rdg[wit={L,B}]{karttavyā}}/ 
%------------------------------
%jāgraddaśāyāṃ    calanadaśāyāṃ   bhojanadaśāyāṃ   sthitikāle sarvasthāne   śūnyasya dhyānakāraṇāt //                              \E
%jāgraddaśāyāṃ    calanadaśāyāṃ   bhojanaṃ daśāyāṃ sthitikāle sarvasthāne   śūnyasya dhyānakāraṇāt //                              \P
%jāgradādidaśāyāṃ calanadaśāyāṃ// bhojanadaśāyāṃ   sthitikāle sarvasthāneṣu śūnyasya dhyānakāraṇāt //                              \L
%jāgradādidaśāyāṃ calanadaśāyāṃ// bhojanadaśāyāṃ   sthitikāle sarvasthāneṣu śūnyasya dhyānakaraṇāt //                              \B
%jāgraddaśāyāṃ    cakabadaśāyāṃ   bhojanadaśāyāṃ   sthitikāle sarvvasthāne  śūnyasya dhyānakaraṇāt  maraṇatrāso na bhavati//       \N1
%jāyadaśāyāṃ      calanadaśāyāṃ/  bhojanadaśāyāṃ   sthitikāle sarvasthāne   śūnyasya dhyānakaraṇāt  maraṇatrāśo na bhavati//       \N2
%jāgraddaśāyāṃ    calanadaśāyāṃ   bhojanadaśāyāṃ   sthitikāle sarvvasthāne  śūnyasya dhyānakaraṇāt  maraṇatrāso na bhavati// śūnya \D1
%jāgraddaśāyāṃ    calanadaśāyāṃ                    sthitikāle sarvasthāne   śūnyasya dhyānakaraṇāt/ maraṇasautrāṃ na bhavati vā    \U1
%jāgṛaddaśāyāṃ    calanadaśāyāṃ   bhojanadaśāyāṃ   sthitikāle sarvasthāne   śūnyasya dhyānakaraṇāt//                               \U2
%------------------------------
%The fear of dying does not arise due to the exercise of meditation on the void at all places during ones life - while eating, moving and waking. 
%------------------------------
\app{\lem[wit={ceteri}]{jāgraddaśāyāṃ}
    \rdg[wit={N2}]{jāgṛaddaśāyāṃ}
    \rdg[wit={N2}]{jāyadaśāyāṃ}
    \rdg[wit={L,B}]{jāgradādidaśāyāṃ}}
\app{\lem[wit={ceteri}]{calanadaśāyāṃ}
    \rdg[wit={N1}]{cakabadaśāyāṃ}}
\app{\lem[wit={ceteri}]{bhojanadaśāyāṃ}
    \rdg[wit={P}]{bhojanaṃ daśāyāṃ}
    \rdg[wit={U1}]{\om}}
  sthitikāle
\app{\lem[wit={ceteri}]{sarvasthāne}
    \rdg[wit={L,B}]{sarvasthāneṣu}}
  śūnyasya dhyānakāraṇāt
\app{\lem[wit={N1,D1}]{maraṇatrāso}
    \rdg[wit={N2}]{maraṇatrāśo}
    \rdg[wit={U1}]{maraṇasautrāṃ}
    \rdg[wit={E,P,L,B,U2}]{\om}}
\app{\lem[wit={ceteri}]{na}
    \rdg[wit={E,P,B,U2}]{\om}}
\app{\lem[wit={N1,N2}]{bhavati}
    \rdg[wit={D1}]{bhavati || śūnya}
    \rdg[wit={U1}]{bhavati vā}
    \rdg[wit={ceteri}]{\om}}/
 \end{prose}
\end{ekdosis}
%%%%%%%%%%%%%%%%%
%%%%%%%%%%%%%%%
%%%%%%%%%%%%%%%%
%%%%%%%%%%%%%%%%
%%%%%%%%%%%%%%%%%
\begin{ekdosis}
  \ekddiv{type=ed}
       \bigskip
    \centerline{\textrm{\small{[The Rājayogin's Body]}}}
    \bigskip
 \begin{prose}
%------------------------------  
%idānīṃ rājayogayuktasya           śarīre yaccihnaṃ  tat    kathyate / \E
%idānīṃ rājayogayuktasya puruṣasya yaccharīracihnaṃ         kathyate / \P
%idānīṃ rājayogayuktasya puruṣasya          cinhnaṃ         kathyate / \L
%idānīṃ rājayogayuktasya puruṣasya          cinhnaṃ         kathyate // \B
%idānīṃ rājayogayuktasya puruṣasya yaccarīracihnaṃ   tat    kathyate / \N1
%idānīṃ rājayogayuktasya puruṣasya yaccharīracihūṃ   tat    kathyate// \N2
%idānīṃ rājayogayuktasya puruṣasya yaccarīracihnaṃ   tat    kathyate / \D1
%idānīṃ rājayogayuktasya puruṣasya yaccharīre cinhaṃ tata   kathyate \U1
%idānīṃ rājayogayuktasya puruṣasya yat śarīracinhaṃ         kathyate / \U2
%------------------------------
%Now it is said that this is the characteristic of the embodied person who is endowed with the royal yoga:
%------------------------------
  idānīṃ rājayogayuktasya
  \app{\lem[wit={ceteri}]{puruṣasya}
    \rdg[wit={E}]{\om}}
  \app{\lem[wit={N1,D1,P},alt={yac carīracihnaṃ}]{yac-carīracihnaṃ}
    \rdg[wit={U2}]{yat śarīracinhaṃ}
    \rdg[wit={E}]{śarīre yac cihnaṃ}
    \rdg[wit={U1}]{yac charīre cinhaṃ}
    \rdg[wit={N2}]{yac charīracihūṃ}
    \rdg[wit={L,B}]{cinhnaṃ}}
  \app{\lem[wit={E,N1,N2,D1}]{tat}
    \rdg[wit={U1}]{tata}
    \rdg[wit={ceteri}]{\om}}
  kathyate/
\end{prose}
    \end{ekdosis}
    %%%%%%%%%%%%%%%%%
    %%%%%%%%%%%%%%%%%
    %%%%%%%%%%%%%%%%%
    %%%%%%%%%%%%%%%%%%
    %%%%%%%%%%%%%%%%
  \begin{ekdosis}
    \ekddiv{type=ed}
    \begin{prose}
%------------------------------  
%tatsarvatra pūrṇo bhavati / \E
%tatsarvatra pūrṇā bhavati / \P
%tatsarvatra pūrṇo bhavati / \L
%tatsarvatra pūrṇo bhavatī / \B
%  sarvvatra pūrṇo bhavati / \N1
%  sarvvatra pūrṇā bhavati  \N2
%  sarvvatra pūrṇo bhavati  \D1
%  sarvvatra pūrṇo bhavati   \U1
%tatsarvatra pūrṇo bhavati// \U2
%------------------------------
%Abundance arises at all times. %Alternative=permanent Abundance arises because of that.   
%------------------------------
\app{\lem[wit={N1,N2,D1,U1},alt={sarvatra°}]{sarvatra}
  \rdg[wit={ceteri}]{tatsarvatra°}}
\app{\lem[wit={ceteri}, alt={°pūrṇo}]{pūrṇo}
  \rdg[wit={P,N2}]{pūrṇā}}
\app{\lem[wit={ceteri}]{bhavati}
  \rdg[wit={B}]{bhavatī}}/
%------------------------------  
%pṛthivyāḥ dūre tiṣṭhati / \E
%pṛthivyāḥ hare tiṣṭhati / \P
%\om                      \L
%\om                      \B
%pṛthivyāḥ dūre  tiṣṭhati / \N1
%pṛthivyāḥ dūra  tiṣṭhati / \N2
%pṛthivyāḥ dūre  tiṣṭhati / \D1
%pṛthivyāḥ ddūre tiṣṭhati / \U1 %emend to na tiṣṭhati? 
%pṛthivyā dūraṃ  tiṣṭhati // \U2 !!dūraṃ
%------------------------------
%No distances exist on earth.
%------------------------------
\app{\lem[type=conjecture, resp=egoscr]{pṛthivyāṃ}
  \rdg[wit={ceteri}]{\conj pṛthivyāḥ}
  \rdg[wit={U2}]{pṛthivyā}
  \rdg[wit={L,B}]{\om}} 
\app{\lem[wit={U2}]{dūraṃ}
  \rdg[wit={E,N1,D1}]{dūre}
  \rdg[wit={U1}]{ddūre}
  \rdg[wit={N2}]{dūra}
  \rdg[wit={L,B}]{\om}}
\app{\lem[type=conjecture, resp=egoscr]{na tiṣṭhati}
  \rdg[wit={ceteri}]{\conj tiṣṭhati}
  \rdg[wit={L,B}]{\om}}/
%------------------------------
%pṛthivyāṃ vyāpya tiṣṭhati / \E
%pṛthi-----vyāpya tiṣṭhati / \P
%\om                         \L
%\om                         \B
%pṛthvāṃ vyāpya   tiṣṭhati /   \N1
%pṛthvīṃ vyāpya   tiṣṭhati /   \N2
%pṛthvīṃ vyāpya   tiṣṭhati /   \D1  %geht auch pṛthu für Erde? 
%\om   \U1
%pṛthivyā vyāti   tiṣṭhati     \U2
%------------------------------
%He dwells on earth having pervaded [it]. 
%------------------------------
\app{\lem[type=emendation, resp=egoscr]{pṛthivīṃ}
  \rdg[wit={E}]{pṛthivyāṃ}
  \rdg[wit={P}]{pṛthi°}
  \rdg[wit={N1}]{pṛthvāṃ}
  \rdg[wit={N2,D1}]{pṛthvīṃ}
  \rdg[wit={U2}]{pṛthivyā}
  \rdg[wit={L,B,U2}]{\om}}
\app{\lem[wit={ceteri}]{vyāpya}
  \rdg[wit={U2}]{vyāti}
  \rdg[wit={L,B,U1}]{\om}} 
\app{\lem[wit={ceteri}]{tiṣṭhati}
  \rdg[wit={L,B,U2}]{\om}}/
%------------------------------
% yasya janmamaraṇe  na staḥ sukhaṃ na bhavati /  \E
% yasya janmamaraṇe  na staḥ sukhaṃ na bhavati /  \P
% \om                                            \L
% \om                                            \B
% yasya janmamaraṇe  na staḥ sukhaṃ na bhavati /  \N1
% yasya janmamaraṇe  na staḥ sukhaṃ na bhavati /  \N2
% yasya janmamaraṇe  na staḥ sukhaṃ na bhavati /  \D1
% \om                                            \U1
% yasya jananamaraṇe na staḥ sukhaṃ na bhavati /  \U2 maraṇe nom/acc dual! staḥ von as 3. dual 
%------------------------------
% Birth and death both do not exist. Happiness does not exist. 
% ------------------------------
\app{\lem[wit={ceteri}]{yasya}
  \rdg[wit={L,B,U1}]{\om}}
\app{\lem[wit={ceteri}]{janmamaraṇe}
  \rdg[wit={U2}]{jananamaraṇe}
  \rdg[wit={L,B,U1}]{\om}}
\app{\lem[wit={ceteri}]{na}
  \rdg[wit={L,B,U1}]{\om}}
\app{\lem[wit={ceteri}]{staḥ}
  \rdg[wit={L,B,U1}]{\om}}/
\app{\lem[wit={ceteri}]{sukhaṃ}
  \rdg[wit={L,B,U1}]{\om}}
\app{\lem[wit={ceteri}]{na}
  \rdg[wit={L,B,U1}]{\om}}
\app{\lem[wit={ceteri}]{bhavati}
  \rdg[wit={L,B,U1}]{\om}}/
% \om                 \E
% \om                 \P
% \om                 \L
% \om                  \B
% duḥkhaṃ na bhavati / \N1
% duḥkhaṃ na bhavati / \N2
% duḥkham na bhavati / \D1
% \om                  \U1
% \om                  \U2
% ------------------------------
%Suffering does not exist. 
%------------------------------
\app{\lem[wit={N1,N2,D1}]{duḥkhaṃ}
  \rdg[wit={ceteri}]{\om}} 
\app{\lem[wit={N1,N2,D1}]{na}
  \rdg[wit={ceteri}]{\om}} 
\app{\lem[wit={N1,N2,D1}]{bhavati}
  \rdg[wit={ceteri}]{\om}}/
%------------------------------
% \om               \E
% kalaṃ na bhavati  \L
% kulaṃ na bhavatī// \B
% kūlaṃ na bhavati / \P
% kūlaṃ na bhavati / \N1
% kūlaṃ na bhavati / \N2
% kūlaṃ na bhavati / \D1
% \om               \U1
% kulaṃ na bhavatī// \U2
%------------------------------
%Impediment does not exist.
%------------------------------
\app{\lem[wit={P,N1,N2,D1}]{kūlaṃ}
  \rdg[wit={B,U2}]{kulaṃ}
  \rdg[wit={L}]{kalaṃ}
  \rdg[wit={E,U1}]{\om}}
\app{\lem[wit={ceteri}]{na}
  \rdg[wit={E,U1}]{\om}}
\app{\lem[wit={ceteri}]{bhavati}
  \rdg[wit={B,U2}]{bhavatī}
  \rdg[wit={E,U1}]{\om}}/
%------------------------------
% \om                  \E
% śītalaṃ na bhavati / \P
% \om                  \L
% \om                  \B
% śīlaṃ na bhavati /   \N1
% śīlaṃ na bhavati /   \N2
% śīlaṃ na bhavati /   \D1
% śīlaṃ na bhavati /   \U1
% śīlaṃ na bhavati /   \U2
%------------------------------
% Habit doesn't exist. 
% ------------------------------
\app{\lem[wit={ceteri}]{śīlaṃ}
  \rdg[wit={P}]{śītalaṃ}
  \rdg[wit={E,L,B}]{\om}}
\app{\lem[wit={ceteri}]{na}
  \rdg[wit={E,L,B}]{\om}}
\app{\lem[wit={ceteri}]{bhavati}
  \rdg[wit={E,L,B}]{\om}}/
%------------------------------
% \om                 \E
% sthānaṃ na bhavati / \P
% \om                  \L
% \om                  \B
% sthānaṃ na bhavati / \N1
% sthānaṃ na bhavati / \N2
% sthānaṃ na bhavati / \D1
% sthānaṃ na bhavati / \U1
% sthānaṃ na bhavati / \U2
%------------------------------
% Place does not exist. 
%------------------------------
\app{\lem[wit={ceteri}]{sthānaṃ}
  \rdg[wit={E,L,B}]{\om}}
\app{\lem[wit={ceteri}]{na}
  \rdg[wit={E,L,B}]{\om}}
\app{\lem[wit={ceteri}]{bhavati}
  \rdg[wit={E,L,B}]{\om}}/
% \end{prose}
%\end{ekdosis}
  %%%%%%%%%%%%%%%
  %%%%%%%%%%%%%%%%
  %%%%%%%%%%%%%%%
  %%%%%%%%%%%%%
  %%%%%%%%%%%%%%%
%    \begin{ekdosis}
%      \ekddiv{type=ed}
%      \begin{prose}
%------------------------------
% \om                                                                             \E
%asya siddhasya manomadhye īśvarasaṃbaṃdhī prakāśo niraṃtaraṃ     pratyakṣo bhavati  \P
%asya siddhasya manomadhye īśvarasaṃbaṃdhi prakāśo  niraṃtaraṃ    pratyakṣo bhavati  \L
%asya siddhasya manomadhye īśvaraṃ saṃbaṃdhī prakāśo  niraṃtaraṃ  pratyakṣo bhavatī//  \B
%asya siddhasya manomadhye īśvarasaṃbaṃdhī prakāśaḥ niraṃtaraṃ    pratyakṣa bhavati  \N1
%asya siddhasya manomadhye īśvarasaṃbaṃdhī prakāśaḥ niraṃtaraṃ    pratyakṣa bhavati/  \N2
%asya siddhasya manomadhye īśvarasaṃbaṃdhi prakāśaḥ niraṃtaraṃ    pratyakṣo bhavati  \D1
%asya siddhasyaṃ pṛthivī vyāpya tiṣṭhati yasya yanma maraṇai na saḥ sukhaṃ na bhati kulaṃ na bhavati śīlaṃ na bhavati sthānaṃ na bhavati ..... asya siddhasya manomadhye īśvarasaṃbaṃdhī prakāśaḥ niraṃtaraṃ pratyakṣo bhavati  \U1
%asya siddhasya manomadhye īśvarasaṃbaṃdhī prakāśo nirattaraṃ  pratyakṣo bhavati//  \U2
%------------------------------
%The manifestation of permanent perception of the connection with god arises in the middle of the mind of this accomplished one. 
%------------------------------
\app{\lem[wit={ceteri}]{asya}
  \rdg[wit={E}]{\om}}
\app{\lem[wit={ceteri}]{siddhasya}
  \rdg[wit={U1}]{siddhasyaṃ pṛthivī vyāpya tiṣṭhati yasya yanma maraṇai na saḥ sukhaṃ na bhati kulaṃ na bhavati śīlaṃ na bhavati sthānaṃ na bhavati asya siddhasya}
  \rdg[wit={E}]{\om}}
\note[type=philcomm, labelb=s34.z3, lem={asya siddhasyaṃ}]{U1 repeats the whole section from pṛthivī to ... sthānaṃ na bhavati due to an eyeskip in the process of copying.}
\app{\lem[wit={ceteri}]{manomadhye}
  \rdg[wit={E}]{\om}}
\app{\lem[wit={ceteri}]{īśvarasaṃbandhī}
  \rdg[wit={B}]{īśvaraṃ saṃbaṃdhī}
  \rdg[wit={E}]{\om}}
\app{\lem[wit={ceteri}]{prakāśo}
  \rdg[wit={N1,N2,D1,U1}]{prakāśaḥ}
  \rdg[wit={E}]{\om}}
\app{\lem[wit={ceteri}]{nirantaraṃ}
  \rdg[wit={U2}]{nirattaraṃ}
  \rdg[wit={E}]{\om}}
\app{\lem[wit={ceteri}]{pratyakṣo}
  \rdg[wit={N1}]{prakyakṣa}
  \rdg[wit={E}]{\om}}
\app{\lem[wit={ceteri}]{bhavati}
  \rdg[wit={B}]{bhavatī}
  \rdg[wit={E}]{\om}}/
\ekdpb*{}
%------------------------------
%sa ca prakāśo na śīto na coṣṇo na śveto na pīto bhavati/ \E
%sa ca prakāśo na śīto na coṣṇo na śveto na pīto bhavati/ \P
%sa ca prakāśo na śīto na coṣṇo na śveto na pīto bhavatī// \L
%sa ca prakāśo na śīto na coṣṇo na śveto na pīto bhavatī// \B
%sa ca prakāśo na śīto na coṣṇo na śveto na pīto bhavati/ \N1
%sa ca prakāśo na śīto na coṣṇo na śveto na pīto bhavati    \D1
%sa ca prakāśo na śīto na coṣṇo na kheto na pīto bhavati/ \N2
%sa ca prakāśo na śīto na ?hbho?na kheto na pīto bhavati // \U1
%sa ca prakāśo// na śīto na coṣṇo na śveto pīto na bhavati // \U2
%------------------------------
%And he is shining - not cold, and not hot, not white [and] not yellow. 
%------------------------------
sa ca prakāśo na śīto na
\app{\lem[wit={ceteri}]{coṣṇo}
  \rdg[wit={U1}]{...o}}
na
\app{\lem[wit={ceteri}]{śveto}
  \rdg[wit={N2,U1}]{kheto}}
\app{\lem[wit={ceteri}]{na pīto}
  \rdg[wit={U2}]{pīto na}}
\app{\lem[wit={ceteri}]{bhavati}
  \rdg[wit={L,B}]{bhavatī}}/
%------------------------------
%tasya na jātir na kiñciccihnam  \E
%tasya na jātir na kiñciccihnaṃ  \P
%tasya na jātir na kiṃciccinhaṃ  \L
%tasya na jātir na kiṃciccinhaṃ  \B
%tasya na jātir na kiṃciccihūṃ  \N1
%tasya na jāti na kiṃciccihūṃ//  \D1
%tasya na jāti na  kiṃciccihūṃ  \N2
%tasya na jātir na kiṃcit khecha cinhaṃ  \U1
%tasya na jānāti na kiṃcit cinhaṃ //  \U2
%------------------------------
%Neither is there birth of him, nor does he have any attributes.
%------------------------------
tasya na
\app{\lem[wit={ceteri}, alt={jātir}]{jātir\skm{-na}}
  \rdg[wit={D1,N2}]{jāti}
  \rdg[wit={U2}]{jānāti}
}\skp{-na}
\app{\lem[wit={ceteri}, alt={kiñcic cihnaṃ}]{kiñcic\skp{-}cihnaṃ}
  \rdg[wit={E}]{°cihnam}
  \rdg[wit={D1,N1,N2}]{°cihūṃ}
  \rdg[wit={U1}]{kiṃcit khecha cinhaṃ}
  \rdg[wit={U2}]{na kiṃcit cinhaṃ}}/
%------------------------------
%ayaṃ   ca niṣkalo   niraṃjanaḥ   alakṣyaś ca bhavati \E
%ayaṃ   ca niṣkalo   niraṃjanaḥ   alakṣyaś ca bhavati \P
%vyayaṃ ca niṣkalo   niraṃjanaṃ// alakṣaś  ca bhavati// \L
%vyayaṃ ca nīṣkalo   niraṃjanaṃ// alakṣaś  ca bhavatī// \B
%ayaṃ   ca niṣkalo   niraṃjanaḥ// alakṣyaś ca bhavati// \D1
%ayaṃ   ca nīṣkalo   niraṃjanaḥ   alakṣaś  ca bhavati// \N1
%ayaṃ   ca niṣkalo   niraṃjanaḥ   alakṣaś  ca bhavati// \N2
%ayaṃ   ca niḥkalo   niraṃjanaḥ   alakṣyaḥ    bhavati/ \U1
%ayaṃ   ca nīṣkalo   niraṃjanaḥ// alakṣyaḥ    bhavati// \U2
%------------------------------
%And he is without parts, immacule and uncharacterized.  
%------------------------------
\app{\lem[wit={ceteri}]{ayaṃ}
  \rdg[wit={L,B}]{vyayaṃ}}
ca
\app{\lem[wit={ceteri}]{niṣkalo}
  \rdg[wit={B,U2}]{nīṣkalo}
  \rdg[wit={U1}]{niḥkalo}}
nirañjanaḥ/
\app{\lem[wit={ceteri}, alt={alakṣyaś}]{alakṣyaś\skm{-ca}}
  \rdg[wit={U1,U2}]{alakṣyaḥ}
  \rdg[wit={L,B,N1,N2}]{alakṣaś}
}\app{\lem[wit={ceteri}, alt={ca}]{\skp{ca}}
  \rdg[wit={U1,U2}]{\om}}
\app{\lem[wit={ceteri}]{bhavati}
  \rdg[wit={B}]{bhavati}}/
%------------------------------
%atha ca phaladvaṃde  na         kāminy āder   yasyecchā         na bhavati // \E
%atha ca phalacaṃda   na         kāminy āder   yasyochā          na bhavati    \P
%atha ca phalavaṃda   na         kāminy ādir   yasya             na bhavati    \L
%atha ca phalaṃ jaṃda na         kāminy ādar   yasye             na bhavatī    \B
%atha ca phalacaṃdra  na         kāminy āder   yasya  yasyeccha     bhavati/   \N1
%atha ca phalacaṃda   na         kāminy āde    yasya  yasyechā      bhavati//  \D1
%atha ca phalaṃ/caṃdra           kāminy āder   yasya  yasyeccha     bhavati/   \N2
%atha ca phalaṃ caṃda na         kāminy āder   yasya  yaṃ           bhavati    \U1
%atha ca phalacaṃda   na         kāminy āder   yasyechā             bhavati//  \U2
%------------------------------
%And his desire etc. doesn't arise in [situations of] lust [and] is not located within the duality of the result.  
%------------------------------
atha ca
\app{\lem[wit={E}]{phaladvande}
     \rdg[wit={P,D1,U2}]{phalacaṃda}
     \rdg[wit={U1}]{phalaṃ caṃda}
     \rdg[wit={L}]{phalavaṃda}
     \rdg[wit={B}]{phalaṃ jaṃda}
     \rdg[wit={N1}]{phalacaṃdra}
     \rdg[wit={N2}]{phalaṃ/ caṃdra}}
\app{\lem[wit={ceteri}]{na}
     \rdg[wit={N2}]{\om}}
\skp{kāminy-}\app{\lem[wit={ceteri}, alt={āder}]{\skm{kāminy}āde\skp{r}}
     \rdg[wit={D1}]{āde}
     \rdg[wit={B}]{ādar}
     \rdg[wit={L}]{ādir}
}\app{\lem[wit={E},alt={yasyecchā}]{\skm{r}-yasyecchā}
     \rdg[wit={P}]{yasyochā}
     \rdg[wit={L}]{yasya}
     \rdg[wit={B}]{yasye}
     \rdg[wit={N1,N2}]{yasya yasyeccha}
     \rdg[wit={D1}]{yasya yasyechā}
     \rdg[wit={U1}]{yasya yaṃ}
     \rdg[wit={U2}]{yasye chā}}
\app{\lem[wit={E,P,L,B}]{na}
     \rdg[wit={ceteri}]{\om}}
\app{\lem[wit={ceteri}]{bhavati}
  \rdg[wit={B}]{bhavatī}}/
% \end{prose}
%  \end{ekdosis}
  %%%%%%%%%%%%%%
  %%%%%%%%%%%%%%
  %%%%%%%%%%%%%%%
  %%%%%%%%%%%%%%
  %%%%%%%%%%%%%%%
%  \begin{ekdosis}
%    \ekddiv{type=ed}
%    \begin{prose}
%------------------------------
% \om                      \E
% \om                      \P
% \om                      \L
% \om                      \B
%taṃ taṃ bhogaṃ prāpnoti   \D1
%taṃ taṃ bhogaṃ prāpnoti   \N1
%taṃ taṃ bhogaṃ prāpnoti// \N2
%tataṃ bhogaṃ prāpnoti     \U1
% \om                      \U2
%------------------------------
%He attains widespread enjoyment. 
%------------------------------
\app{\lem[wit={D1,N1,N2}]{taṃ taṃ}
  \rdg[wit={U1}]{tataṃ}
\rdg[wit={ceteri}]{\om}}
\app{\lem[wit={D1,N1,N2,U1}]{bhogaṃ prāpnoti}
  \rdg[wit={ceteri}]{\om}}/ 
%------------------------------
% \om                      \P
% \om                      \L
% \om                      \B
%atha vā yasya    mana    eva   sthāne 'nurāgaṃ        na prāpnoti// \D1
%atha vāsya/vātya mana   eva   sthāne 'nurāgaṃ        na prāpnoti/ \N1
%atha vā syamana         eva   sthāne 'nurāgaṃ        na prāpnoti/ \N2
%atha vā svāmana         etata sthāne  nurāgaṃ         na prāpnoti/ \U1
% \om                      \U2
%------------------------------
%However, his mind does not suffer attachment in this very state.  
%------------------------------
\app{\lem[wit={D1,N1,N2,U1}]{atha}
  \rdg[wit={ceteri}]{\om}} 
\app{\lem[wit={D1}]{vā yasya}
  \rdg[wit={N1}]{vāsya}
  \rdg[wit={N2}]{vā syamana}
  \rdg[wit={U1}]{vā svāmana}
  \rdg[wit={ceteri}]{\om}}
\app{\lem[wit={D1,N1,N2,U1}]{mana}
  \rdg[wit={ceteri}]{\om}}
\app{\lem[wit={D1,N1,N2,U1}]{eva}
  \rdg[wit={U1}]{etata}
  \rdg[wit={ceteri}]{\om}}
\app{\lem[wit={D1,N1,N2,U1}]{sthāne}
  \rdg[wit={ceteri}]{\om}}
\app{\lem[wit={D1,N1,N2}]{'nurāgaṃ}
  \rdg[wit={U1}]{nurāgaṃ}
  \rdg[wit={ceteri}]{om}}
\app{\lem[wit={D1,N1,N2,U1}]{na prāpnoti}
  \rdg[wit={ceteri}]{\om}}/ 
\end{prose}
\end{ekdosis}
%%%%%%%%%%%%%%%%
%%%%%%%%%%%%%%%
%%%%%%%%%%%%%%%
%%%%%%%%%%%%%%%%
%%%%%%%%%%%%%%%
  \begin{ekdosis}
 \ekddiv{type=ed}
   \bigskip
    \centerline{\textrm{\small{[Other Attributes]}}}
    \bigskip
 \begin{prose}
%------------------------------
%anyad  rājayogasya cihnaṃ kathyate   \E
% \om                                 \P
%anyata rājayogasya cinhaṃ kathyate// \L
%anyata rājayogasya cinhaṃ kathyate// \B
%anyat  rājayogasya cinhaṃ kathyate// \N1 yasyecchā bhavati??? taṃ taṃ bhogaṃ prāpnoti/ atha vāsya mana eva sthāne 'nu rāgaṃ na prāpnoti/ anyat rājayogasya cinhaṃ kathate//
%anyat  rājayogasya cihuṃ  kathyate// \D1
%anyad  rājayogasya ciṃhuṃ kathyate// \N2
%anyat  rājayogacinhaṃ     kathyate/  \U1
%anyat  rājayogasya cinhaṃ kathyate// \U2
%------------------------------
%[Now] another attribute of Rājayoga is described. 
%------------------------------
\app{\lem[wit={E,N2},alt={anyad}]{anyad\skm{-rā}}
  \rdg[wit={N1,D1,U1,U2}]{anyat}
  \rdg[wit={L,B}]{anyate}
  \rdg[wit={P}]{\om}
}\app{\lem[wit={ceteri},alt={rājayogasya}]{\skp{-rā}jayogasya}
  \rdg[wit={U1}]{rājayoga°}
  \rdg[wit={P}]{\om}}
\app{\lem[wit={E}]{cihnaṃ}
  \rdg[wit={L,B,N1,U2}]{cinhaṃ} %????
  \rdg[wit={N2}]{ciṃhuṃ}
  \rdg[wit={D1}]{cihuṃ}
  \rdg[wit={P}]{\om}}
\app{\lem[wit={ceteri}]{kathyate}
  \rdg[wit={P}]{\om}}/
\end{prose}
  \end{ekdosis}
  %%%%%%%%%%%%%%
  %%%%%%%%%%%%%%
  %%%%%%%%%%%%%%
  %%%%%%%%%%%%%%
  %%%%%%%%%%%%%
    \begin{ekdosis}
      \ekddiv{type=ed}
      \begin{prose}
%------------------------------
%yasya rājyādilābhe 'pi    phalalābho na bhavati/ \E
% \om                                            \P
%yasya rājādilābhe   ty     aphalalābho       na bhavatī \L
%yasya rājādilābhe   ty     aphalalābho       na bhavatī \B
%yasya rājyādilābhe  pi     phalalābho       ba bhavati/ \N1
%yasya rājyādilābhe  pi     phalalābho       na bhavati// \D1
%yasya rājyādilobhe  pi ca  phalalābho       na bhavati// \N2
%yasya rājyādilābe  'pi ca  palalābho        na bhavati/ \U1
%yasya rājyādilābho                          na bhavati/ \U2
%------------------------------
%Even ``of one who is in gain of a kingdom etc.'' [it is said that] perception of success does'nt arise.
%------------------------------
\app{\lem[wit={ceteri}]{yasya}
  \rdg[wit={P}]{\om}} 
\app{\lem[wit={E,N1,D1}]{rājyādilābhe}
  \rdg[wit={L,B}]{rājā°}
  \rdg[wit={N2}]{°lobhe}
  \rdg[wit={U1}]{°lābe}
  \rdg[wit={U2}]{°lābho}
  \rdg[wit={P}]{\om}}
\app{\lem[wit={E,N1,D1}]{'pi}
  \rdg[wit={N2,U1}]{'pi ca}
  \rdg[wit={L,B}]{ty}
  \rdg[wit={P,U2}]{\om}}
\app{\lem[wit={E,N1,D1,N2}]{phalalābho}
  \rdg[wit={U1}]{pala°}
  \rdg[wit={L,B}]{aphala°}
  \rdg[wit={P,U2}]{\om}}
\app{\lem[wit={E,D1,N2,U1,U2}]{na bhavati}
  \rdg[wit={L,B}]{na bhavatī}
  \rdg[wit={N1}]{ba bhavati}
  \rdg[wit={P}]{\om}}/
%------------------------------
%hānāv api manomadhye duḥkhaṃ na bhavati/ \E
% \om                                      \P
%hananād pi mānomadhye duḥkahṃ na bhavatī/ \L
%hananād pi mānomadhye duḥkahṃ na bhavatī/ \B
%hānāv api manomadhye duḥkhaṃ na bhavati/ \N1 %emend to hānau loc. sg. of hāni -> abandonment
%hānāv api manomadhye duḥkhaṃ na bhavati// \D1
%hānāv  pi manomadhye duḥkhaṃ na bhavati// \N2
%hānāv api manomadhye duḥkhaṃ na bhavati/  \U1
%hānād api manomadhye duḥkhaṃ na bhavati// \U2
%------------------------------
%Even due to loss suffering does'nt arise in the mind.  
%------------------------------
\app{\lem[wit={ceteri},alt={hānāv}]{hānā\skp{v-}}
  \rdg[wit={U2}]{hānād}
  \rdg[wit={P,L}]{nahanād}
  \rdg[wit={P}]{\om}
}\app{\lem[wit={ceteri},alt={api}]{\skm{v-}api}
  \rdg[wit={L,B,N2}]{pi}
  \rdg[wit={P}]{\om}}
manomadhye duḥkhaṃ na
\app{\lem[wit={ceteri}]{bhavati}
  \rdg[wit={L,B}]{bhavatī}}/
%------------------------------
%atha ca tṛṣṇā na bhavati/ \E
% \om                      \P
%atha ca tṛṣṇā na bhavati/ \L
%atha ca tṛṣṇā na bhavatī/ \B
%atha ca tṛṣṇā na bhavati/ \N1
%atha ca tṛṣṇā na bhavati  \D1
%atha ca tṛṣṇā na bhavati/ \N2
%atha ca tṛṣṇā na bhavati/ \U1
%atha ca tṛṣṇā na bhavati/ \U2
%------------------------------
%And then desire doesn't arise. 
%------------------------------
\app{\lem[wit={ceteri}]{atha ca}
  \rdg[wit={P}]{\om}}
\app{\lem[wit={ceteri}]{tṛṣṇā}
  \rdg[wit={P}]{\om}}
\app{\lem[wit={ceteri}]{na}
  \rdg[wit={P}]{\om}}
\app{\lem[wit={ceteri}]{bhavati}
  \rdg[wit={B}]{bhavatī}
  \rdg[wit={P}]{\om}}/
%------------------------------
%atha ca kasmin                                  padārthasyopary   anicchā na bhavati/ \E
% \om                                                                                       \P
%atha ca kasmin na    padārtho   prāpte kasyāpi  padārthasyopari   ānīcha  na  bhavati//    \L
%atha ca kasmin na    padārthau  prāpte kasyāpi  padārthāsyopari   ānīchā  ni  bhavati//    \B
%atha ca kasminn pi   padārthe   prāpta kasyāpi  padārthasya upari anusthā na  bhavaṃti//   \N1 
%atha ca kasminn api  padārthe   prāpte kasyāpi  padārthasya upari anichā      bhavaṃti     \D1
%atha ca kasminn pi   padārthe   prāpte kasyāpi  padārthasya upari anisthā na  bhavati//    \N2
%atha ca kasminn api  padārthe   prātpe kasyāpi  padārthasya upari aniṣṭā  na  bhavati      \U1
%atha ca kasmin   adhipadārtha   prāpte kābhyādi padārthopari      anicha  na  bhavati//    \U2 %%%407.jpg
%------------------------------
%And then with regards to some object that has been obtained for whatever reason towards ones object aversion does'nt arise.   
%------------------------------
\app{\lem[wit={ceteri}]{atha ca}
  \rdg[wit={P}]{\om}}
kasminn\skm{-a}\app{\lem[wit={D1,U1},alt={api}]{\skp{-a}pi}
  \rdg[wit={L,B}]{na}
  \rdg[wit={N1,N2}]{pi}
  \rdg[wit={U2}]{adhi}
  \rdg[wit={E,P}]{\om}}
\app{\lem[wit={ceteri}]{padārthe}
  \rdg[wit={L}]{padārtho}
  \rdg[wit={B}]{padārthau}
  \rdg[wit={U2}]{padārtha°}
  \rdg[wit={E,P}]{\om}}
\app{\lem[wit={ceteri}]{prāpte}
  \rdg[wit={N1}]{prāpta}
  \rdg[wit={E,P}]{\om}}
\app{\lem[wit={ceteri}]{kasyāpi}
  \rdg[wit={U2}]{kābhyādi}
  \rdg[wit={E,P}]{\om}}
\app{\lem[wit={E},alt={padārthasyopary}]{padārthasyopa\skp{ry-}}
  \rdg[wit={L,B}]{padārthasyopari}
  \rdg[wit={U2}]{padārthopari}
  \rdg[wit={ceteri}]{padārthasya upari}
  \rdg[wit={P}]{\om}}\app{\lem[wit={E},alt={anicchā}]{\skm{rya-}\skp{a}nicchā}
  \rdg[wit={L}]{ānīcha}
  \rdg[wit={B}]{ānīchā}
  \rdg[wit={N1}]{anusthā}
  \rdg[wit={D1}]{anichā}
  \rdg[wit={N2}]{anisthā}
  \rdg[wit={U1}]{aniṣṭā}
  \rdg[wit={U2}]{anicha}}
\app{\lem[wit={ceteri}]{na}
  \rdg[wit={B}]{ni}
  \rdg[wit={P,D1}]{\om}}
\app{\lem[wit={ceteri}]{bhavati}
  \rdg[wit={N1,D1}]{bhavaṃti}
  \rdg[wit={P}]{\om}}/
%------------------------------
%kasmin    padārthe manaso   nurāgo na bhavati/    \E
%asminnapi padārthe manaso   nurāgo na bhavati... ayam api padārthe manasonurāgo na bhavati... \P
%asminn    padārthe manaso   nurāgo na bhavatī/    \L
%asminn    padārthe manaso   nurāgo na bhavatī/    \B
%asminnapi padārthe manasaḥ anurāgo    bhavati/    \N1
%asminnapi padārthe manasaḥ anurāgo    bhavati//   \D1
%asminnapi padārthe manasaḥ anurāgo    bhavati/    \N2
%asminnapi padārthe manasa  anurāgo    bhavati     \U1 
%kasminnpi padārthe         anurāgo na bhavati// ayam api padārthe anurāgo na bhavati//  \U2
%------------------------------
%With regard to this object also affection of the mind does'nt arise. 
%------------------------------
\app{\lem[wit={ceteri},alt={asminn}]{asmi\skp{nn-}}
  \rdg[wit={E,U2}]{kasmin}
}\app{\lem[wit={ceteri},alt={api}]{\skm{nn-}api}
  \rdg[wit={E,L,B}]{\om}} 
padārthe
\app{\lem[wit={E,P,L,B}]{manaso}
  \rdg[wit={N1,D1,N2,U1}]{manasaḥ}
  \rdg[wit={U1}]{manasa}
  \rdg[wit={U2}]{\om}}
\app{\lem[wit={E,P,L,B}]{'nurāgo}
  \rdg[wit={ceteri}]{anurāgo}}
\app{\lem[wit={E,P,U2}]{na bhavati}
  \rdg[wit={L,B}]{na bhavatī}
  \rdg[wit={ceteri}]{bhavati}}/\note[type=philcomm, labelb=s33.z2, lem={na bhavati}]{P and U2 add \textit{ayam api padārthe anurāgo na bhavati ||} after this sentence, which is clearly a corruption.}
%------------------------------
%ayam  api rājayogaḥ kathyate/  \E
%atham api rājayogaḥ kathyate   \P
%atha  samarājayogaḥ kathyate/  \L
%ayam  api rājayogaḥ kathyate/  \B
%ayam  api rājayogaḥ kathyate/  \N1
%ayam  api rājayogaḥ kathyate// \D1
%ayam  api rājayoga  kathyate// \N2
%ayam  api rājayogaḥ kathyate/  \U1
%ayam  api rājayoga  kathyate// \U2
%------------------------------
%Just this is said to be Rājayoga. 
%------------------------------
\app{\lem[wit={ceteri},alt={ayam}]{aya\skp{m-}}
  \rdg[wit={P}]{atham}
  \rdg[wit={L}]{atha}
}\app{\lem[wit={ceteri},alt={api}]{\skm{-m}api}
  \rdg[wit={L}]{sama}}
\app{\lem[wit={ceteri}]{rājayogaḥ}
  \rdg[wit={N2,U2}]{rājayoga}}
kathyate/
\end{prose}
\end{ekdosis}
  %%%%%%%%%%%%%%
  %%%%%%%%%%%%%%%
  %%%%%%%%%%%%%%%
  %%%%%%%%%%%%%%%
  %%%%%%%%%%%%%%
    \begin{ekdosis}
      \ekddiv{type=ed}
      \begin{prose}
%------------------------------ %%%%split in stemma?! maitre mitre!!!
%atha caḥ yasya manaḥ   munividvat  puruṣeṣu maitre        ca samaṃ bhavati/ \E
%atha ca  yasya manaḥ   śunividvat  puruṣe   maitre śatrau ca samaṃ bhavati \P
%atha ca  yasya manaḥ   bhunividvat puruṣe   maitre śatrau ca samaṃ bhavati/ \L
%atha ca  yasya manaḥ   śrunividvat puruṣe   maitre śatro  ca samaṃ bhavatī/ \B
%atha ca  yasya manaḥ/  śrutividyut puruṣe   mitre  śatrau ca samaṃ bhavati/ \N1
%atha ca  yamanaḥ       śrutividyut puruṣe   mitre  śatrau ca samaṃ bhavati// \D1
%atha ca  yasya manaḥ   śrutividyut puruṣe   mitre  śatrau ca samaṃ bhavati/ \N2
%atha ca  yasya mana    śrunividvat puruṣe   mitre  śatrau ca samaṃ bhavati \U1
%atha ca  yasya manaḥ   śuciviśuddhapuruṣe   mitre  śatrau ca samaṃ bhavati// \U2
%------------------------------
%And then his mind which knows the sacred speech is equal towards a person, friend and enemy.  
%------------------------------
atha
\app{\lem[wit={ceteri}]{ca}
  \rdg[wit={E}]{caḥ}}
\app{\lem[wit={ceteri}]{yasya}
  \rdg[wit={D1}]{ya}}
manaḥ
\app{\lem[resp=egoscr, type=emendation]{śrutividvat\skp{-}}
  \rdg[wit={E}]{munividvat}
  \rdg[wit={P}]{śunividvat}
  \rdg[wit={L}]{bhunividvat}
  \rdg[wit={B,U1}]{śrunividvat}
  \rdg[wit={N1,N2,D1}]{śrutividyut}
  \rdg[wit={U2}]{śuciviśuddha°}
}\app{\lem[wit={ceteri}]{puruṣe}
  \rdg[wit={E}]{puruṣeṣu}}
\app{\lem[wit={ceteri}]{mitre}
  \rdg[wit={E,P,L,B}]{maitre}}
 \app{\lem[wit={ceteri}]{śatrau}
   \rdg[wit={B}]{śatro}
   \rdg[wit={E}]{\om}}
 ca samaṃ bhavati/
 \ekdpb*{}
%------------------------------
%dṛṣṭiś ca samā bhavati/   \E
%dṛṣṭiś ca namnā bhavati   \P
% \om                      \L
% \om                      \B
%dṛṣṭiś ca samā bhavati//  \N1
%dṛṣṭiś ca samā bhavati//  \D1
%dṛṣṭiś ca samā bhavati//  \N2
%dṛṣṭiś ca samā bhavati/   \U1
%dṛṣṭiś ca samā bhavati/   \U2
%------------------------------
%And a neutral view arises. 
%------------------------------
\app{\lem[wit={ceteri},alt={dṛṣṭiś}]{dṛṣṭi\skm{ś-ca}}
  \rdg[wit={L,B}]{\om}
}\app{\lem[wit={ceteri}]{\skp{ś-}ca}
  \rdg[wit={L,B}]{\om}}
\app{\lem[wit={ceteri}]{samā}
  \rdg[wit={P}]{namnā}
  \rdg[wit={L,B}]{\om}}
\app{\lem[wit={ceteri}]{bhavati}
  \rdg[wit={L,B}]{\om}}/
%------------------------------
%sakalapṛthvīmadhye gamanavataḥ       sukhabhogavataḥ      yasya manasi karttṛtvābhimāno   nāsti/ \E
%sakalapṛthvīmadhye gamanāgamanavataḥ sukhabhogavataḥ      yasya manasi kartṛtvābhimāno    nāsti/ \P
%sakalapṛtvīmadhye  gamanāgamanataḥ   sukhabogho bhavataḥ  yasya manasi kartu tvābhimano   nāsti/ \L
%sakalapṛthvīmadhye gamanāgamanataḥ   sukhabogho bhavataḥ  yasya manasi kartutvābhimano    nāsti// \B
%sakalapṛthvīmadhye gamanavataḥ//     sukhabhogavataḥ/     yasya manasi kartṛtvādyabhimāno nāsti/  \N1
%sakalapṛthvīmadhye gamanaṃvataḥ//    sukhabhogavataḥ      yasya manasi kartṛtvādyabhimāno nāsti// \D1
%sakalapṛthvīmadhye gamavataḥ         sukhabhogavataḥ      yasya manasi kartṛtvādyabhimāno nāsti// \N2
%sakalapṛthvīmadhye gamanavataḥ       sukho bhogavataḥ     yasya manasi kartṛtvābhimāno    nāsti   \U1
%sakalapṛthvīmadhye gamanāgamanavat// sukhabhogavat        yasya manasi kartṛtvābhimāno    nāsti// \U2
%------------------------------
%In the mind of one who is situated in the centre of the entire earth, the pride of authorship does't arise, because of death and rebirth, and because of happiness and enjoyment.  %%%check translation think about the Sanskrit 
%------------------------------
\app{\lem[wit={ceteri}]{sakalapṛthvīmadhye}
  \rdg[wit={L}]{°pṛtvī°}}
\app{\lem[wit={P}]{gamanāgamanavataḥ}
  \rdg[wit={U2}]{gamanāgamanavat}
  \rdg[wit={L,B}]{gamanāgamanataḥ}
  \rdg[wit={E,N1,U1}]{gamanavataḥ}
  \rdg[wit={D1}]{gamanaṃvataḥ}
  \rdg[wit={U1}]{gamavataḥ}}
\app{\lem[wit={ceteri}]{sukhabhogavataḥ}
  \rdg[wit={L,B}]{sukhabogho bhavataḥ}
  \rdg[wit={U1}]{sukho bhogavataḥ}
  \rdg[wit={U2}]{sukhabhogavat}}
yasya manasi
\app{\lem[wit={E,P,U1,U2}]{kartṛtvābhimāno}
  \rdg[wit={B}]{kartutvābhimano}
  \rdg[wit={L}]{kartu tvābhimano}
  \rdg[wit={N1,N2,D1}]{kartṛtvādyabhimāno}}
nāsti/
%------------------------------
%atha ca lokamadhye gamanavataḥ sukhabhogavataḥ yasya manasi karttṛtvābhimāno nāsti/....
%atha ca lokamadhye kartṛtvaṃ na jñāpayati/ \E
%anucalokamadhye    kartṛtvaṃ na jñāpayati/ \P
%anucaralokamadhya  kartṛtvābhimano nāsti \L
%anucaralokamadhya--kartṛtvābhimano nāsti// \B
%anucalokamadhye    kartṛtvaṃ    jñāpayati// \N1
%anucalokamadhye    kartṛtvaṃ na jñātvā payati/ \D1
%anucalokamadhye    kartṛtvaṃ na jñāpayati/ \N2
%anucalokamadhye    kartṛtvaṃ    jñātva payati \U1
%anucalokamadhye    kartṛtvaṃ na jñāpayati \U2
%------------------------------
%while wandering the world he doesn't whish to know authorship. 
%------------------------------
\app{\lem[wit={L,B}]{anucara}
  \rdg[wit={N1,N2,D1,U1,U2,P}]{anuca°}
  \rdg[wit={L,B}]{anucara°}
  \rdg[wit={E}]{atha ca}
}\app{\lem[wit={ceteri}]{lokamadhye}
  \rdg[wit={L,B}]{°madhya}}
\app{\lem[wit={E,P,D1,N2,U2}]{kartṛtvaṃ na}
  \rdg[wit={L,B}]{kartṛtvābhimano}
  \rdg[wit={N1,U1}]{kartṛtvaṃ}}
\app{\lem[wit={E,P,N1,N2,U2}]{jñāpayati}
  \rdg[wit={D1,U1}]{jñātva payati}
  \rdg[wit={L,B}]{nāsti}}/
%------------------------------
%so  pi  rājayogaḥ kathyate// \E
%so  pi  rājayogaḥ kathyate   \P
%so  pi  rājayoga  kathyate/   \L
%so  pi  rājayoga  kathyate/   \B
%so  pi  rājayogaḥ kathyate//  \N1
%so  pi  rājayoga  kathyate//  \D1
%so 'pi  rājayoga  kathyate// \N2
%so  pi  rājayoga  kathyate/   \U1
%so  pi  rājayoga  kathyate    \U2
%------------------------------
%This is also said to be Rājayoga. 
%------------------------------
so 'pi
\app{\lem[wit={E,P,N1}]{rājayogaḥ}
  \rdg[wit={ceteri}]{rājayoga}}
kathyate/
\end{prose}
  \end{ekdosis}
  %%%%%%%%%%%%%%
  %%%%%%%%%%%%%
  %%%%%%%%%%%%%%
  %%%%%%%%%%%%%%
  %%%%%%%%%%%%%
  \begin{ekdosis}
    \ekddiv{type=ed}
    \begin{prose}
%------------------------------
%navīnāni         paṭṭasūtramaya     dhṛtāni vastrāṇi   \E
%navīnāni         paṭasūtramayāni    dhṛtāni vastrāṇi   \P
%navinīnīśpī      paṭṭasūtramayāni   dhṛtāni vastrāṇi// \L
%navinīnīr api    paṭṭasūtramayāni   dhṛtāni vastrāṇi// \B
%navīnāni         paṭasūtramayāni    dhṛtāni vastrāṇi/  \N1
%navīnāni         paṭasūtramayāni    dhṛtāni vastrāṇi// \D1
%navīnāni         paṭasūtramayāni    dhṛtāni vastrāṇi/  \N2
%navīnāni         padasūtramayāni       tāni vastrāṇi   \U1
%navīnāni      paṭ(h)asūtramayāni dhṛtāni            \U2
%------------------------------
%New durable clothes made of silk,  
%------------------------------
\app{\lem[wit={ceteri}]{navīnāni}
  \rdg[wit={L}]{navīnīnīś pī}
  \rdg[wit={B}]{navinīnīr api}}
\app{\lem[wit={E,L,B}, alt={paṭṭa°}]{paṭṭa}
  \rdg[wit={P,N1,D1,N2,U2}]{paṭa°}
  \rdg[wit={U1}]{pada}
}sūtra\app{\lem[wit={ceteri}]{mayāni}
  \rdg[wit={E}]{maya}}
\app{\lem[wit={ceteri}]{dhṛtāni}
  \rdg[wit={U1}]{tāni}}
\app{\lem[wit={ceteri}]{vastrāṇi}
  \rdg[wit={U2}]{\om}}
%------------------------------ %%%%KOLLOQUIUM: was hier tun? kastūrī/kasturikā = gleichwertig 
%atha vā jīrṇāni chidrāṇi    dhṛtāni    kastūrīcandanalepair   vā  kardamalepena   yasya manasi harṣaśokau  na staḥ/ \E
%atha vā jīrṇāni sachadrāṇi  dhūtāni    kastūrīcaṃdanalepo     vā  karddamalepo vā yasya manasi harṣaśokau na staḥ/ \P
%atha vā jīrṇāni svachidrāṇi dhṛtāni    kasturīcaṃdanalepo     cā  kardamalepo  vā yasya manasi harṣaśokau na sthaḥ// \L
%atha vā jīrṇāni svachidrāṇi dhṛtāni    kastūrīcaṃdanalepo     vā  kardamalepo  vā yasya manasi harṣaśokau na sthaḥ// \B
%atha vā jīrṇāni sacchidrāṇi dhṛtāni/   kasturikā caṃdanalepo vā/ kardamalepo  vā yasya manasi harṣaśoko  na sthaḥ  \N1
%atha vā jīrṇāni sacchidrāṇi dhṛtāni//  kasturikā caṃdanalepo vā/ kardamalepo  vā yasya manasi harṣaśoko  na sthaḥ  \D1
%atha vā jīrṇāni sacchidrāṇi dhṛtāni // kasturikā caṃdanalepo vā/ kardamalepo  vā yasya manasi harṣaśoka  na sthāḥ \N2
%atha vā jīrṇāni sachidrāṇi  dhvatāni   kasturikā caṃdanalepo vā  kardamalepo  vā yasya manasi harṣaśokau na sthāḥ \U1 %%272.jg
%                                       kastūrīcaṃdanalepo     vā                  yasya manasi harṣaśoko  na sta// \U2
%------------------------------
%or however, old, worn (clothes) with holes smeared with sandalwood and musk, or smeared with mud. In whose mind joy and sorrow are not situated,
%------------------------------
atha vā jīrṇāni
\app{\lem[wit={N1,N2,D1}]{sacchidrāṇi}
  \rdg[wit={U2}]{sachidrāṇi}
  \rdg[wit={P}]{sachadrāṇi}
  \rdg[wit={L,B}]{svachidrāṇi}
  \rdg[wit={E}]{chidrāṇi}}
\app{\lem[wit={ceteri}]{dhṛtāni}
  \rdg[wit={U2}]{dhvātāni}
  \rdg[wit={P}]{dhūtāni}}
\app{\lem[wit={E,P,B,U2}]{kastūrī}
  \rdg[wit={L}]{kasturī}
  \rdg[wit={N1,N2,D1,U1}]{kasturikā}
}\app{\lem[wit={E},alt={candana°}]{candana}
  \rdg[wit={ceteri}]{caṃdana°}
}\app{\lem[wit={E},alt={lepair}]{lepai\skm{r-vā}}
  \rdg[wit={ceteri}]{lepo}} 
\app{\lem[wit={ceteri},alt={vā}]{\skp{r-}vā}
  \rdg[wit={L}]{cā}}
\app{\lem[wit={E}]{kardamalepena}
  \rdg[wit={ceteri}]{kardamalepo}}
\app{\lem[wit={ceteri}]{vā}
  \rdg[wit={E}]{\om}}
yasya manasi
harṣa\app{\lem[wit={ceteri},alt={°śokau}]{śokau}
  \rdg[wit={N1,D1,U2}]{°śoko}
  \rdg[wit={N2}]{°śoka}}
na
\app{\lem[type=emendation, resp=egoscr]{sthau}
  \rdg[wit={ceteri}]{\korr sthaḥ}
  \rdg[wit={N2,U1}]{sthā}
  \rdg[wit={U2}]{sta}}
%------------------------------
%sa evātra tiṣṭhati/         \E
%sa eva rājayogaḥ            \P
%sa eva rājayogaḥ// idānīṃ// \L
%sa eva rājayogaḥ// idānīṃ// \B
%sa eva rājayogaḥ//          \N1
%sa eva rājayogaḥ//          \D1
%sa eva rājayogaḥ//          \N2
%sa eva rājayogaḥ            \U1
%sa eva rājayoga             \U2
%------------------------------
%just he is in the state of Rājayoga. 
%------------------------------
%yasya janmamaraṇe na staḥ sukhaṃ na bhavati/ kulaṃ na bhavati śīlaṃ na bhavati/ sthānaṃ na bhavati/ \E
%\om \P
%\om \L
%\om \B
%\om \N1
%\om \D1
%\om \N2
%\om \U1
%\om \U2
%------------------------------
%One who is not situated in birth and death has no happiness, has no family, and cold does not arise, place does not arise.?!?!!?
%----------------------------
\app{\lem[wit={ceteri}]{sa eva}
  \rdg[wit={E}]{sa evātra}}
\app{\lem[wit={ceteri}]{rājayogaḥ}
  \rdg[wit={U2}]{rājayoga}
  \rdg[wit={L,B}]{rājayogaḥ || idānīṃ ||}
  \rdg[wit={E}]{tiṣṭhati}}/\note[type=philcomm, labelb=s29.z7, lem={°tiṣṭhati}]{E adds \textit{yasya janmamaraṇe na staḥ sukhaṃ na bhavati | kulaṃ na bhavati śīlaṃ na bhavati | sthānaṃ na bhavati |} here, which seems to be a dittography of previous sentences.}
%----------------------------
%rājayogaḥ naramadhye      atha ca vanamadhye             yuddhe saṃgrāmamadhye                        vā yasya manaḥ        bhayapūrṇaṃ vā  na bhavati/  so pi rājayogaḥ kathyate// \E
%          nagaramadhye    'tha ca vanamadhye                  utasaṃgrāmamadhye                       vā yasya mana      ūnaṃ    pūrṇaṃ vāṃ na bhavati   so pi rājayogaḥ            \P
%          nagaramadhye     tha ca vanamadhye                 udvastagrāmamadhye                       vā yasya manaḥ     unaṃ    pūrṇaṃ vā  na bhavati   so pi rājayogaḥ//          \L
%          nagaramadhye  (')tha ca vanamadhye                udvastagrāmaṃmadhye                       vā yasya manaḥ     unaṃ    pūrṇaṃ vā  na bhavatī   so pi rājayogaḥ//          \B
%          nagaramadhye    atha ca vanamadhye/                 udvesūgrāmamadhye .. ..pūrṇagrāmamadhye vā yasya manaḥ     ūnaṃ na pūrṇaṃ vā  na bhavati// so pi rājayogaḥ//          \N1
%          ṣagaramadhye    atha ca vanamadhye//                udvesūgrāmamadhye svetapūrṇagrāmamadhye vā yasya manaḥ     ūnan na pūrṇaṃ vā  na bhavati/  so pi rājayogaḥ//          \D1
%          nagaramadhye    atha ca vanamadhye//                udvesūgrāmamadhye svetapūrṇagrāmamadhye vā yasya manaḥ     ūnan na pūrṇaṃ vā  na bhavati/  so pi rājayogaḥ//          \N2
%       vā nagaramadhye    atha ca vanamadhye                 udassaṃgrāmamadhye  lokapūrṇagrāmamadhye vā yasya manaḥ     unaṃ    pūrṇaṃ     na bhavati   so pi rājayogaḥ            \U1
%          nagaramadhye    'tha vā vanamadhye                  udvasagrāmamadhye                       vā yasya mana      ūnaṃ    pūrṇaṃ vāṃ na bhavati   so pi rājayogaḥ            \U2
%------------------------------
%Just he is in the state of Rājayoga for whom the mind is neither in abundance nor in lack, being located in a city, a forest, an uninhabited village or a village full of people. 
%----------------------------
\app{\lem[wit={ceteri}]{nagaramadhye}
  \rdg[wit={E}]{rājayogaḥ nagaramadhye}
  \rdg[wit={D1}]{ṣagaramadhye}
  \rdg[wit={U1}]{vā nagaramadhye}}
\app{\lem[wit={P,L,B,U2}]{'tha ca}
  \rdg[wit={E,N1,N2,D1,U1}]{atha ca}}
vanamadhye
\app{\lem[wit={U2},alt={udvasa°}]{udvasa}
  \rdg[wit={E}]{yuddhe saṃ°}
  \rdg[wit={P}]{utasaṃ°}
  \rdg[wit={L,B}]{udvasta°}
  \rdg[wit={N1,N2,D1}]{udvesū°}
  \rdg[wit={U1}]{udassaṃ°}
}\app{\lem[wit={ceteri}]{grāmamadhye}
  \rdg[wit={B}]{grāmaṃ madhye}}
\app{\lem[wit={U1}]{lokapūrṇagrāmamadhye}
  \rdg[wit={N1}]{....pūrṇagrāmamadhye}
  \rdg[wit={D1,N2}]{svetapūrṇagrāmamadhye}}
vā yasya
\app{\lem[wit={P,U2}]{mana}
  \rdg[wit={ceteri}]{manaḥ}}
\app{\lem[wit={P,N1,N2,U2}]{ūnaṃ}
  \rdg[wit={D1,N2}]{ūnan}
  \rdg[wit={L,B,U1}]{unaṃ}
  \rdg[wit={E}]{bhaya°}}
\app{\lem[wit={N1,N2,D1}]{na}
  \rdg[wit={ceteri}]{\om}}
pūrṇaṃ
\app{\lem[wit={ceteri}]{vā}
  \rdg[wit={P,U2}]{vāṃ}
  \rdg[wit={U1}]{\om}}
na bhavati/ so
\app{\lem[type=emendation, resp=egoscr]{'pi}
  \rdg[wit={ceteri}]{\korr pi}}
\app{\lem[wit={ceteri}]{rājayogaḥ}
  \rdg[wit={E}]{rājayogaḥ kathyate}}/ 
\end{prose}
\end{ekdosis}
%%%%%%%%%%%%%%
%%%%%%%%%%%%%%
%%%%%%%%%%%%%%
%%%%%%%%%%%%%
%%%%%%%%%%%%%%% 
\begin{ekdosis}
  \ekddiv{type=ed}
  \bigskip
   \centerline{\textrm{\small{[Description of Caryāyoga]}}}
      \bigskip
 \begin{prose}
%----------------------------
%idānīṃ      yogaḥ  kathyate/ \E
%idānīṃ caryāyogaḥ  kathyate   \P
%idānīṃ caryāyogaḥ  kathyate// \L
%idānīṃ caryāyogaḥ  kathyate// \B
%idānīṃ caryāyoga   kathyate// \N1
%idānīṃ caryāyogaḥ  kathyate// \D1 [S.7, Z.7]
%idānīṃ caryāyoga   kathyate// \N2
%idānīṃ tvaryāyogaḥ kathyate \U1
%idānīṃ caryāyoga   kathyate// \U2
%------------------------------
%Now \textit{caryāyogaḥ}, the Yoga of wandering is explained.
%----------------------------
idānīṃ
\app{\lem[wit={ceteri}]{caryāyogaḥ}
     \rdg[wit={U1}]{tvaryāyogaḥ}
     \rdg[wit={E}]{yogaḥ}} kathyate/
% \end{prose}
%  \end{ekdosis}
  %%%%%%%%%%%
  %%%%%%%%%%%
%%%%%%%%%%%%
  %%%%%%%%%%%%
  %%%%%%%%%%%
 %  \begin{ekdosis}
 %    \ekddiv{type=ed}
%     \begin{prose}
%----------------------------
%nirākāro         nityo 'bhedyaḥ    sa etādṛśaḥ ātmani                  mano   yasya  niścalaṃ tiṣṭhati/  \E
%nirākāro  'calo  nityo  bhedhyaḥ   sa etādṛa   ātmā    etādṛśo  ātmani mano   yasya  niścala  tiṣṭhati   \P %%%7639.jpg
%nirākāro  calo   nityo  bhedhyaḥ   sa etādṛa   ātmā sa etādṛśe  ātmani               niścala  tiṣṭhati/  \L     %daṇḍa nach ātmā besser -> emend? oder in weiteren Hss?
%nirākāro  calo   nityo  bhedhyaḥ   sa etādṛa   ātmā sa etādṛśye ātmani               niścalaṃ tiṣṭhati/  \B
%nirākālo  nityo   calo 'bhedhyaḥ/  sa etādṛśaḥ ātmā    etādṛśe  ātmani manaḥ  yasya  niścalaṃ tiṣṭhati   \N1
%nirākālo  nityo   calo 'bhedhyaḥ// sa etādṛśaḥ ātmā    etādṛśe  ātmani manaḥ  yasya  niścalaṃ tiṣṭhati   \D1
%nirākālo  nityo   calo 'bhedhyaḥ   sa etādṛśaḥ ātmā    etādṛśa  ātmani manaḥ  yasya  niścala  tiṣṭhati/  \N2
%nirākāro  nityo   calo abhedhyaḥ   sa etādṛśaḥ ātmā    etādṛśo  ātmani mano   yasya  niścalaṃ bhavati    \U1
%nirvikāro  'calo nityo 'bhedhya    sa etādṛśā  ātmani                  mano   yasya  niścalaṃ tiṣṭhati// \U2
%------------------------------
%Shapeless, unchangeable, permanent [and] unsplitable. Such is the self. It is seen as such by the one whose mind abides in the self without moving. 
%------------------------------
\app{\lem[wit={E,P,L,B,U1}]{nirākāro}
  \rdg[wit={N1,N2,D1}]{nirākālo}
  \rdg[wit={U2}]{nirvikāro}}
\app{\lem[wit={P,U2}]{'calo}
  \rdg[wit={L,B}]{calo}
  \rdg[wit={N1,N2,D1,U1}]{nityo}
  \rdg[wit={E}]{\om}}
\app{\lem[wit={E,P,L,B,U2}]{nityo}
  \rdg[wit={ceteri}]{calo}}
\app{\lem[wit={E,N1,N2,D1}]{'bhedyaḥ}
  \rdg[wit={P,L,B}]{bhedhyaḥ}
  \rdg[wit={U1}]{abhedhyaḥ}
  \rdg[wit={U2}]{'bhedyha}}
   sa
\app{\lem[wit={P,L,B}]{etādṛśa}
  \rdg[wit={E,N1,N2,D1,U1}]{etādṛśaḥ}
  \rdg[wit={U2}]{etādṛśā}}
\app{\lem[wit={ceteri}]{ātmā}
  \rdg[wit={E,U2}]{ātmani}}
\app{\lem[wit={L,B}]{sa}
  \rdg[wit={ceteri}]{\om}}
\app{\lem[wit={N2}]{etādṛśa}
  \rdg[wit={P,U1}]{etādṛśo}
  \rdg[wit={L,N1,D1}]{etādṛśe}
  \rdg[wit={B}]{etādṛśye}
  \rdg[wit={E,U2}]{\om}}
\app{\lem[wit={ceteri}]{ātmani}
  \rdg[wit={E,U2}]{\om}}
\app{\lem[wit={E,P,U1,U2}]{mano}
  \rdg[wit={N1,N2,D1}]{manaḥ}
  \rdg[wit={L,B}]{\om}}
\app{\lem[wit={ceteri}]{yasya}
  \rdg[wit={L,B}]{\om}}
\app{\lem[wit={ceteri}]{niścalaṃ}
  \rdg[wit={P,L,N2}]{niścala}}
\app{\lem[wit={ceteri}]{tiṣṭhati}
  \rdg[wit={U1}]{bhavati}}/
\ekdpb*{}
%------------------------------
%tasyātmanaḥ puṇyapāpasparśo na bhavati/ \E
%tasyātmanaḥ puṇyapāpasparśo na bhavati  \P
%tasyātmanaḥ puṇyapāpasparśo na bhavati/ \L
%tasyātmanaḥ puṇyapāpasparśo na bhavatī/ \B
%tasyātmanaḥ punyapāpasparśo na bhavati/  \N1
%tasyātmanaḥ punyapāpasparśo na bhavati// \D1
%tasyātmanaḥ puṇyapāpasparśo na bhavati/ \N2
%tasya ātmanaḥ puṇyapāsya sparśo na bhavati  \U1
%tasya ātmanaḥ puṇyapāsya sparśo na bhavati//  \U2
%------------------------------
%His self is not touched by sin and merit. 
%------------------------------
\app{\lem[wit={ceteri}]{tasyātmanaḥ}
  \rdg[wit={U1,U2}]{tasya ātmanaḥ}}
\app{\lem[wit={ceteri}]{puṇyapāpasparśo}
  \rdg[wit={U1,U2}]{puṇyapāsya sparśo}}
na bhavati/
%------------------------------
%udakamadhye sthitasya padmapatre       yathodakasya sparśo    bhavati/  tathaivātmani   \E
%udakamadhye sthitasya padmanī patrasya yathodakasya sparśo na bhavati   tathaivātmani   \P
%udakamadhye sthitasya padmanī patrasya yathodakasya sparśo na bhavati/  tathaivātmani   \L
%udakamadhye sthitasya padmanī patrasya yathodakasya sparśā na bhavatī/  tathaivātmani   \B
%udakamadhye sthitasya padminī patrasya yathā/ udakasparśo  na bhavati/  tathaivātmani   \N1
%udakamadhye sthitasya padminī patrasya yathā  udakasparśo  na bhavati// tathaivātmani   \D1
%udakamadhye sthitasya padminī patrasya yathā  udakasparśo  na bhavati/  tathaivātmani   \N2
%udakamadhye sthitasya padminī patrasya yathā  udakasparśo  na bhavati   tathaivātmani   \U1
%udakamadhye sthitasya padminī patrasya yathodakasparśo     na bhavati// tathaivātmani   \U2
%------------------------------
%Just as the leave of the lotus situated in the amidst water doesn't touch the water; likewise the self [is not touched by sin and merit].
%------------------------------
udakamadhye sthitasya
\app{\lem[wit={ceteri}]{padminī patrasya}
  \rdg[wit={P,L,B}]{padmanī patrasya}
  \rdg[wit={E}]{padmapatre}}
\app{\lem[wit={E,P,L}]{yathodakasya sparśo}
  \rdg[wit={B}]{yathodakasya sparśā}
  \rdg[wit={N1,N2,D1,U1}]{yathā udakasparśo}
  \rdg[wit={U2}]{yathodakasparśo}}
na
\app{\lem[wit={ceteri}]{bhavati}
  \rdg[wit={B}]{bhavatī}}
tathaivātmani/
%------------------------------
%yathākāśamadhye   pavanaḥ svecchayā bhramati/ \E
%yathākāśamadhye   pavanaḥ svechayā  bhramati \P
%yathā ākāśamadhye pavanaḥ svechayā  bhramati/ \L
%yathā ākāśamadhye pavanaḥ svechayā  bhramatī/ \B
%yathā ākāśamadhye pavanasvachayā    bhramati/ \N1
%yathā ākāśamadhye pavanasvachayā    bhramati \D1
%yathā ākāśamadhye pavanasvachayā    bhramati/ \N2
%yathā ākāśamadhye pavanaḥ svechayā  bhramayati \U1
%yathā 'kāśamadhye pavanaḥ svechayā  bhramati// \U2
%------------------------------
%Just as the wind wanders according to its own will in space,...  
%------------------------------
yathā\app{\lem[wit={E,P}]{kāśamadhye}
  \rdg[wit={U2}]{'kāśamadhye}
  \rdg[wit={ceteri}]{ākāśamadhye}}
\app{\lem[wit={ceteri}]{pavanaḥ svechayā}
  \rdg[wit={N1,N2,D1}]{pavanasvachayā}}
\app{\lem[wit={ceteri}]{bhramati}
  \rdg[wit={U1}]{brahmayati}}
%------------------------------
%tathā yasya manaḥ nirākāramadhye līnaṃ bhavati/  sa eva caryāyogaḥ// \E
%tathā yasya manaḥ nirākāramadhye līnaṃ bhavati   sa eva caryāyogaḥ   \P
%tathā yasya manaḥ nirākāramadhye līnaṃ bhavati   sa eva caryāyogaḥ// \L
%tathā yasya manaḥ nirākāramadhye līnaṃ bhavatī   sa eva caryāyogaḥ// \B
%tathā yamanaḥ     nirākāramadhye līnaṃ bhavati/  sa eva kriyāyogaḥ// \N1
%tathā yasya manaḥ nirākāramadhye līnaṃ bhavati/  sa eva kriyāyogaḥ// \D1 !!!!!Stemma point!!!!!!
%tathā       pavananirākāramadhye līnaṃ bhavati/  sa eva kriyāyogaḥ// \N2
%tathā yasya manaḥ nirākāramadhye līnaṃ bhavati   sa eva kriyāyogaḥ   \U1 
%tathā yasya manaḥ nirākāramadhye līnaṃ bhavati// sa eva caryāyogaḥ// \U2
%------------------------------
%Likewise is the mind of whom is absorbed into the universal spirit [wanders according to its own will in space]. This is \textit{\caryāyoga}.  
%------------------------------
tathā
\app{\lem[wit={ceteri}]{yasya manaḥ}
  \rdg[wit={D1}]{yamanaḥ}
  \rdg[wit={N2}]{pavana°}}
nirākāramadhye līnaṃ
\app{\lem[wit={ceteri}]{bhavati}
  \rdg[wit={B}]{bhavatī}}/
sa eva
\app{\lem[wit={ceteri}]{caryāyogaḥ}
  \rdg[wit={N1,N2,D1,U1}]{kriyāyogaḥ}}\dd{}
\end{prose}
\end{ekdosis}
 %%%%%%%%%%%%%%%%%%%%%%%%%%%%%
 %%%%%%%%%%%%%%%%%%%%%%%%%%%%%
 %%%%%%%%%%%%%%%%%%%%%%%%%%%%%
 %%%%%%%%%%%%%%%%%%%%%%%%%%%%%
 %%%%%%%%%%%%%%%%%%%%%%%%%%%%%
\begin{ekdosis}
  \ekddiv{type=ed}
  \bigskip
    \centerline{\textrm{\small{[Description of Haṭhayoga]}}}
      \bigskip
      \begin{prose}
%------------------------------
%idānīṃ grahayogaḥ kathyate/  \E %[p.23]
%idānīṃ haṭhayogaḥ kathyate   \P
%idānīṃ haṭhayogaḥ kathyate/  \L
%idānīṃ haṭayoga   kathyate/  \B
%idānīṃ haṭhayogaḥ kathyate//  \N1
%idānīṃ haṭhayogaḥ kathyate/  \D1
%idānīṃ haṭhayoga  kathyate// \N2
%idānīṃ haṭhayogaḥ kathyate   \U1
%idānīṃ haṭhayoga  kathyate   \U2
%------------------------------
%Now \textit{haṭhayoga} is explained. 
%------------------------------
        idānīṃ
        \app{\lem[wit={P,L,N1,D1,U1}]{haṭhayogaḥ}
          \rdg[wit={U2}]{haṭhayoga}
          \rdg[wit={B}]{haṭayoga}
          \rdg[wit={E}]{grahayogaḥ}}
        kathyate/ \\
\end{prose}
  \end{ekdosis}
  %%%%%%%%%%%%
  %%%%%%%%%%%%
  %%%%%%%%%%%%
  %%%%%%%%%%%%
  %%%%%%%%%%%
  \begin{ekdosis}
    \ekddiv{type=ed}
    \begin{prose}
%------------------------------
%recakapūrakakumbhaka  ityādiprakāreṇa   pavanasādhanaṃ     kartavyam/ \E
%recakapūrakakuṃbhaka  ityādiprakāreṇa   pavanasādhanaṃ     karttavyaṃ \P
%recakapūrakakumbhaka  ityādiprakāreṇa   pavanasya sādhanaṃ kartavyam// \L
%recakapūrakakuṃbhaka  ityādiprakāreṇa// pavanasya sādhanaṃ kartavyam \B
%recakapūrakakuṃbhaka/ ityādiprakāreṇa   pavanasya sādhanaṃ kartavyaṃ/ \N1
%recakapūrakakuṃbhaka  ityādiprakāreṇa   pavanasya sādhanaṃ kartavyaṃ// \D1
%recakapūrakakuṃbhaka  ityādhiprakāreṇa  pavanasya sādhanaṃ kartavyaṃ// \N2
%recakapūrakakuṃbhaka  ityādiprakāreṇa   pavanasya sādhanaṃ kartavyaṃ \U1
%recakapūrakakuṃbhaka  ityādiprakāreṇa   pavanasya sādhanaṃ kartavyaṃ// \U2
%------------------------------
%The practice of breath shall be done in this manner: "Exhalation, Inhalation [and] Retention etc.
%------------------------------        
        recakapūrakakuṃbhaka
        \app{\lem[wit={ceteri}, alt={ityādi}]{ityādi}
          \rdg[wit={N2}]{ityādhi°}
        }prakāreṇa
        \app{\lem[wit={ceteri}]{pavanasya sādhanaṃ}
          \rdg[wit={E,P}]{pavanasādhanaṃ}}
 \app{\lem[wit={E,L,B}]{kartavyam}
   \rdg[wit={ceteri}]{kartavyaṃ}}/
%------------------------------
%atha ca dhautyādiṣaṭkarmakāraṇāt   śarīrasya śuddhir bhavati/ \E
%atha ca dhautyādiṣaṭkarmakāraṇāt   śarīrasya śuddhir bhavati \P
%atha ca dhautyādiṣaṭkarmakāraṇāt// śarīrasya śuddhir bhavati \L
%atha ca  dhotyādiṣaṭkarmakaraṇāt// śarīrasya śuddhir bhavatī \B
%atha ca dhautyādiṣaṭkarmakaraṇāt/  śarīrasya śuddhir bhavati/ \N1
%atha ca dhautyādiṣaṭkarmakaraṇāt   śarīrasya śuddhir bhavati// \D1
%atha ca dhautyādiṣaṭkarmakaraṇāt// śarīrasya śuddhir bhavati// \N2
%atha   vidhotyādiṣaṭkarmakaraṇāt   śarīrasya śuddhir bhavati/ \U1
%atha ca dhautyādiṣaṭkarmakaraṇāt// śarīrasya śuddhir bhavati// \U2 %%%408.jpg 
%------------------------------
%And then due to the six practices(\textit{ṣaṭkarma}), like \textit{dhauti} etc. the purification of the body arises. 
%------------------------------        
 atha
 \app{\lem[wit={ceteri}]{ca}
   \rdg[wit={U1}]{\om}}
 \app{\lem[wit={ceteri}]{dhautyādi}
   \rdg[wit={B}]{dhotyādi}
   \rdg[wit={U1}]{vidhotyādi}
 }ṣaṭkarmakāraṇāt śarīrasya śuddhir\skp{-}bhavati/
 %------------------------------
%sūryanāḍīmadhye       pavanaḥ pūrṇo yadā tiṣṭati/   \E %!
%sūryanāḍīmadhye       pavanaḥ pūrṇo yadā tiṣṭati    \P
%sūryanāḍīmadhye       pavanapūrṇo   yadāti/         \L
%sarvasūryanāḍīmadhye  pavanapūrṇo   yadāti/         \B
%sūryanāḍīmadhye       pavanaḥ pūrṇo yadā tiṣṭhati/  \N1
%sūryanāḍīmadhye       pavanaḥ pūrṇo yadā tiṣṭhati   \D1
%sūryanāḍīmadhye       pvanaḥ  pūrṇo yadā tiṣṭhati/  \N2
%sūryanāḍīmadhye       pavanaḥ pūrṇo yadā tiṣṭhati/  \U1
%sūryanāḍīmadhye       pavanaḥ sūryo yadā tiṣṭhati// \U2
%------------------------------
%When the full breath abides in the middle of the sun-channel, ... 
%------------------------------
 \app{\lem[wit={ceteri}]{sūryanāḍīmadhye}
   \rdg[wit={B}]{sarvasūryanāḍīmadhye}}
 \app{\lem[wit={ceteri}]{pavanaḥ pūrṇo}
   \rdg[wit={L,B}]{pavanapūrṇo}
   \rdg[wit={N2}]{pvanaḥ pūrṇo}}
 \app{\lem[wit={ceteri}]{yadā tiṣṭhati}
   \rdg[wit={L,B}]{yadāti}}/
%------------------------------
%tadā mano  niścalaṃ bhavati/  \E
%tadā mano  niścalo  bhavati   \P
%tadā mano  niścalo  bhavati/  \L
%tadā mano  niścalo  bhavatī// \B
%tadā manaḥ niścalaṃ bhavati/  \N1
%tadā manaḥ niścalaṃ bhavati   \D1
%tadā manaḥ niścalaṃ bhavati   \N2
%tadā manaḥ niścalaṃ bhavati   \U1
%tadā mano  niścalaṃ bhavati// \U2
%------------------------------
%Then the mind is unmovable. 
%------------------------------
 tadā
 \app{\lem[wit={ceteri}]{mano}
   \rdg[wit={N1,N2,D1,U1}]{manaḥ}}
\app{\lem[wit={ceteri}]{niścalaṃ}
  \rdg[wit={P,L,B}]{niścalo}}
bhavati/
%------------------------------
%manaso  niścalatvena ānandarūpaṃ      pratyakṣaṃ bhāsate/  \E
%manaso  niścalatve   ānandaṃ svarūpa--pratyakṣaṃ bhāsate   \P %%%%7640.jpg
%manaso  niścalatve   ānandaṃ svarūpaṃ pratyakṣaṃ bhāsate/  \L
%manaso  niścalatve   ānaṃdaṃ svarūpaṃ pratyakṣaṃ bhāsate// \B
%manasaḥ niścalatve   ānaṃdasvarūpaṃ   pratyakṣaṃ bhāsate/  \N1
%manasaḥ niścalatve   ānaṃdasvarūpaṃ   pratyakṣaṃ bhāsate/  \D1
%manasaḥ niścalatve   ānaṃdasvarūpaṃ   pratyakṣaṃ bhāṣate/  \N2
%manasaḥ niścalatve   ānaṃdasvarūpaṃ   pratyakṣaṃ bhāṣate/  \U1 %%%273.jpg
%manaso  niścalatve   ānaṃdasvarūpaṃ   pratyakṣaṃ bhāsate// \U2
%------------------------------
%The form of bliss immediately shines through the motionless mind.  
%------------------------------
\app{\lem[wit={ceteri}]{manaso}
  \rdg[wit={N1,N1,D1,U1}]{manasaḥ}}
\app{\lem[wit={ceteri}]{niścalatve}
  \rdg[wit={E}]{niścalatvena}}
\app{\lem[wit={ceteri}]{ānandasvarūpaṃ}
  \rdg[wit={L,B}]{ānaṃdaṃ svarūpaṃ}
  \rdg[wit={P}]{ānandaṃ svarūpa°}
  \rdg[wit={E}]{ānandarūpaṃ}}
pratyakṣaṃ
\app{\lem[wit={ceteri}]{bhāsate}
  \rdg[wit={N2,U1}]{bhāṣate}}/
%------------------------------
%haṭhayogakāraṇāt  manaḥ   śūnyamadhye līnaṃ   bhavati/  kālaḥ samīpe   nāgacchati/  \E
%haṭhayogakāraṇāt  manaḥ   śūnyamadhye līnaṃ   bhavati   kālaḥ samīpe   nāgacchati   \P %%%%7640.jpg
%haṭhayogakāraṇāt  manaḥ   śūnyamadhye līnaṃ   bhavati/  kālaḥ samīpe   nāgacchati// \L
%haṭayogākāraṇāt   manaḥ// śūnyamadhye līnaṃ   bhavatī/  kālāsamīpe nāma gacchati//  \B
%haṭhayogakaraṇāt  manaḥ   śūnyamadhye līnaṃ   bhavati/  kālaḥ samīpe   nāgachati//  \N1
%haṭhayogakaraṇāt  manaḥ   śūnyamadhye līnaṃ   bhavati// kālaḥ samīpe   nāgachaṃti// \D1
%haṭhayogakaraṇāt  mana----śūnyamadhye līnaṃ   bhavati/  kālasamīpe     nāgachati//  \N2
%haṭhayogakaraṇāt/ manaḥ   śūnyamadhye līnaṃ   bhavati/  kālasamīpe ti  nāgachati    \U1 %%%273.jpg
%haṭhayogakaraṇāt  manaḥ   śūnyamadhye sthānaṃ bhavati// kāsaḥ samīpe   nāgachati//  \U2
%------------------------------
%Due to the execution of haṭhayoga the mind becomes absorbed into emptiness. The time of death does not approach.
%------------------------------
\app{\lem[wit={ceteri}, alt={haṭha°}]{haṭha}
  \rdg[wit={B}]{haṭa}
}\app{\lem[wit={ceteri},alt={yoga°}]{yoga}
  \rdg[wit={B}]{yogā°}
}\app{\lem[wit={ceteri}]{karaṇāt}
  \rdg[wit={E,P,L,B}]{kāraṇāt}}
\app{\lem[wit={ceteri}]{manaḥ}
  \rdg[wit={N2}]{mana}}
śūnyamadhye
\app{\lem[wit={ceteri}]{līnaṃ}
  \rdg[wit={U2}]{sthānaṃ}}
bhavati/
\app{\lem[wit={ceteri}]{kālaḥ}
  \rdg[wit={B}]{kālā°}
  \rdg[wit={N2,U1}]{kāla°}
  \rdg[wit={U2}]{kāsaḥ}}
samīpe
\app{\lem[wit={ceteri}]{nāgacchati}
  \rdg[wit={B}]{nāma gacchati}
  \rdg[wit={D1}]{nāgachaṃti}
  \rdg[wit={U1}]{ti nāgachati}}/
\end{prose}
  \end{ekdosis}
%%%%%%%%%%
%%%%%%%%%%%%
  %%%%%%%%%%%%%%
  %%%%%%%%%%%%%%%
  %%%%%%%%%%%%%
  \begin{ekdosis}
    \ekddiv{type=ed}
    \begin{prose}
%------------------------------
%idānīṃ haṭhayogasya dvitīyo  bhedaḥ kathyate/   \E
%idānīṃ haṭhayoga----dvitīya--bhedaḥ kathyate    \P
%idānīṃ haṭhayogasya dvitīya--bhedāḥ kathyante/  \L
%idānīṃ haṭayogasya  dvitīyaṃ bhedāḥ kathyaṃte// \B
%idānīṃ haṭhayogasya dvitīyo  bhedaḥ kathyate//  \N1
%idānīṃ haṭhayogasya dvitīya--bhedaḥ kathyate    \D1
%idānīṃ haṭayogasya  dvitīyo  bhedaḥ kathyate    \U1
%idānīṃ haṭhayogasya dvitīyo  bhedaḥ kathyate//  \U2 
%------------------------------
%Now, the second division of haṭhayoga is explained.
%------------------------------
idānīṃ
\app{\lem[wit={ceteri}]{haṭhayogasya}
  \rdg[wit={B,U1}]{haṭayogasya}
  \rdg[wit={P}]{haṭhayoga°}}
\app{\lem[wit={ceteri}]{dvitīyo}
  \rdg[wit={P,L,D1}]{dvitīya°}
  \rdg[wit={B}]{dvitīyaṃ}}
\app{\lem[wit={ceteri}]{bhedaḥ}
  \rdg[wit={L,B}]{bhedāḥ}}
\app{\lem[wit={ceteri}]{kathyate}
  \rdg[wit={L,B}]{kathyante}}/
     \end{prose}
  \end{ekdosis}
  %%%%%%%%%%%%
  %%%%%%%%%%%%
  %%%%%%%%%%%
  %%%%%%%%%%%%
  %%%%%%%%%%%%%
  \begin{ekdosis}
    \ekddiv{type=ed}
    \begin{prose}
a%------------------------------
%pādādārabhya śiraḥ paryaṃtaṃ    svaśarīre  koṭisūryatejaḥ   samānaṃ śvetaṃ pītaṃ       raktaṃ kiṃcidvarṇaṃ ciṃtyate/  \E
%pādādārabhya śiraḥ paryaṃtaṃ    svaśarīre  koṭisūryatejaḥ   samānaṃ śvetaṃ pītaṃ nīlaṃ raktaṃ kiṃdrupaṃ    cityate    \P
%pādādārabhya śira--paryaṃtaṃ    svaśarīre  koṭisūryatejaḥ   samānaśvetaṃ nīlaṃ         raktaṃ tiṃdrupaṃ    ciṃtate/   \L
%pādādārabhya śira--paryaṃtaṃ    svaśarīre  koṭisūryatejaḥ// samānaśvetanīlaṃ           raktaṃ kiṃdrupaṃ    ciṃtate//  \B
%pādādārabhyā śiraḥ paryentaṃ    svaśarīre  koṭisūryatejaḥ   samānaṃ śvetaṃ pītaṃ nīlaṃ laktaṃ kiṃcidrūpaṃ  ciṃtyate   \N1 
%pādādārabhyā śiraḥ paryaṃtaṃ    svaśarīre  koṭisūryatejaḥ   samānaṃ śvetaṃ pītaṃ nīlaṃ raktaṃ kiṃcidrūpaṃ  ciṃtyate   \D1
%pādādārabhya śiraḥ pariyataṃ    svaśarīraṃ koṭisūryatejaḥ   samānaṃ śvetaṃ pītaṃ nīlaṃ raktaṃ ciṃrūpaṃ     ciṃtyate   \U1
%pādādārabhya śiro  paryaṃtaṃ    svaśarīre  koṭisūryye tejaḥ samānaṃ śvetaṃ pītaṃ nīlaṃ raktaṃ kiṃcidrūpaṃ  ciṃtyate// \U2
%------------------------------
%The shine of ten million suns in one's own body beginning from the feet to the top of head is contemplated in any color equal to white, yellow [or] red.
%------------------------------
\app{\lem[wit={ceteri}]{pādādārabhya}
  \rdg[wit={N1,D1}]{pādādārabhyā}}
\app{\lem[wit={ceteri}]{śiraḥ}
  \rdg[wit={L,B}]{śira°}
  \rdg[wit={U2}]{śiro}}
\app{\lem[wit={ceteri}]{paryantaṃ}
  \rdg[wit={N1}]{paryentaṃ}
  \rdg[wit={U1}]{pariyataṃ}}
\app{\lem[wit={ceteri}]{svaśarīre}
  \rdg[wit={U1}]{svaśarīraṃ}}
\app{\lem[wit={ceteri}]{koṭisūryatejaḥ}
  \rdg[wit={U2}]{koṭisūryye tejaḥ}}
\app{\lem[wit={ceteri}]{samānaṃ}
  \rdg[wit={L,B}]{samāna°}
  \rdg[wit={ceteri}]{śvetaṃ}
  \rdg[wit={B}]{śveta°}}
\app{\lem[wit={ceteri}]{pītaṃ}
  \rdg[wit={L,B}]{\om}}
nīlaṃ
\app{\lem[wit={ceteri}]{raktaṃ}
  \rdg[wit={N1}]{laktaṃ}}
\app{\lem[wit={N1,D1,U2}]{kiṃcidrūpaṃ}
  \rdg[wit={P,B}]{kiṃdrupaṃ}
  \rdg[wit={L}]{tiṃdrupaṃ}
  \rdg[wit={U1}]{ciṃrūpaṃ}
  \rdg[wit={E}]{kiṃcidvarṇaṃ}}
\app{\lem[wit={ceteri}]{cintyate}
  \rdg[wit={P}]{cityate}
  \rdg[wit={L,B}]{ciṃtate}}/
%------------------------------
%ttad  dhyānakāraṇāt     sakalaṃ   rogajvalanaṃ     bhavati/                      āyur          vardhate/          \E
%tad   dhyānakāraṇāt     sakalāṃge rogajvalanaṃ  na bhavati                       āyur vṛddhir  bhavati   \P
%tad   dhyānakāraṇāt     sakalaṃge rogajvalanaṃ  na bhavati/                      āyur          vardhate/          \L
%tat   dhyānakāraṇāt     sakalaṃge rogajvalanaṃ  na bhavati/                      āyur vṛddhir  bhavatī/  \B
%na    dhyānaṃ kāraṇāt/  sakalāṃge roga          na bhavati/  jvalanaṃ na bhavati āyur vṛddhir  bhavati/  \N1
%ta    dhyānaṃ karaṇāt// sakalāṃge rogajvalanaṃ  na bhavati//                                             \D1
%tad---dhyānaṃ karaṇāt / sakalāṃge roga          na bhavati   jvaranaṃ na bhavati āyu--vṛddhir  bhavati// \N2
%ta    dhyānaṃ karaṇāt   sakalāṃge roga kṣataṃ?  na bhavati                       āyur vṛddhir  bhavati   \U1
%tat   dhyānakāraṇāt     sakalāṃge rogajvalanaṃ     bhavati//                     āyur vṛddhir  bhavati// \U2
%------------------------------
%aDue to the execution of meditation in the entire body disease does'nt arise, fever doesn't arise and vitality grows.  
%------------------------------
\app{\lem[wit={E,P,L,N2},alt={tad}]{tad\skp{-d}}
  \rdg[wit={B,U2}]{tat}
  \rdg[wit={D1,U1}]{ta}
  \rdg[wit={N1}]{na}
}\app{\lem[wit={ceteri},alt={dhyānakāraṇāt}]{\skp{-d}dhyānakāraṇāt}
  \rdg[wit={N1,N2,D1,U1}]{dhyānaṃ karaṇāt}}
\app{\lem[wit={P,N1,D1,N2,U1,U2}]{sakalāṃge}
  \rdg[wit={L,B}]{sakalaṃge}
  \rdg[wit={E}]{sakalaṃ}}
\app{\lem[type=emendation, resp=egoscr]{rogaḥ}
\rdg[wit={N1,N2}]{\korr roga}
\rdg[wit={E,P,L,B,D1,U2}]{rogajvalanaṃ}
\rdg[wit={U1}]{roga kṣataṃ}}
\app{\lem[wit={ceteri}]{na}
  \rdg[wit={E,U2}]{\om}}
bhavati/
\app{\lem[wit={N2}]{jvaranaṃ na bhavati}
  \rdg[wit={N1}]{jvalanaṃ na bhavati}
  \rdg[wit={ceteri}]{\om}}/
\app{\lem[wit={ceteri}, alt={āyur}]{āyu\skp{r-vṛ}}
  \rdg[wit={N2}]{āyu°}
  \rdg[wit={D1}]{\om}
}\app{\lem[wit={ceteri},alt={vṛddhir}]{\skm{r-vṛ}ddhi\skp{r-bha}}
  \rdg[wit={E,L,D1}]{\om}
}\app{\lem[wit={ceteri},alt={bhavati}]{\skm{r-bha}vati}
  \rdg[wit={B}]{bhavatī}
  \rdg[wit={E,L}]{vardhate}
  \rdg[wit={D1}]{\om}}\dd{}
\end{prose}
\end{ekdosis}
 %%%%%%%%%%%%%%%%%%%%%%%%%%%%%
 %%%%%%%%%%%%%%%%%%%%%%%%%%%%%
 %%%%%%%%%%%%%%%%%%%%%%%%%%%%%
 %%%%%%%%%%%%%%%%%%%%%%%%%%%%%
 %%%%%%%%%%%%%%%%%%%%%%%%%%%%%
\begin{ekdosis}
  \ekddiv{type=ed}
          \bigskip
    \centerline{\textrm{\small{[Description of Jñānayoga]}}}
          \bigskip
          \begin{prose}
%------------------------------
%idānīṃ jñānayogasya lakṣaṇaṃ kathyate/ \E
%idānīṃ jñānayogasya lakṣaṇaṃ kathyate \P
%idānīṃ jñānayogasya lakṣaṇaṃ// \L 5976_0011.jpg 
%idānīṃ jñānayogasya lakṣaṇaṃ// \B
%idānīṃ jñānayogasya lakṣaṇaṃ// \N1 %%%%p.6 verso 
%idānīṃ jñānayogasya lakṣaṇaṃ// \D1
%idānīṃ jñānayogasya lakṣaṇaṃ kathyate// \N2
%idānī  jñānayogasya lakṣaṇaṃ kathyate   \U1
%idānīṃ jñānayogasya lakṣaṇaṃ kathyate// \U2
%------------------------------
%Now the characteristic of jñānayoga is explained. 
%-----------------------------
\app{\lem[wit={ceteri}]{idānīṃ}
  \rdg[wit={U1}]{idānī}}
jñānayogasya lakṣaṇaṃ
\app{\lem[wit={E,P,N2,U1,U2}]{kathyate}
  \rdg[wit={L,B,N1,D1}]{\om}}/
\end{prose}
\end{ekdosis}
\ekdpb*{}
  %%%%%%%%%%%%%%
  %%%%%%%%%%%%%%
  %%%%%%%%%%%%%%%
  %%%%%%%%%%%%%%%
  %%%%%%%%%%%%%%%
  \begin{ekdosis}
    \ekddiv{type=ed}
%--------------------------------------
%ekam eva jagat paśyed viśvāvasu vibhāsvaram/
%avikalpatayā yuktyā jñānayogaṃ samācaret//1// \E
%
%ekam eva cayat paśyed viśvātmāsu vibhāsvaram       
%avikalpatayā yuktyā jñānayogaṃ samācaret 1 \P
%
%ekam evā jagat paśyed viśvātmāsu vibhāsvaraṃ//
%avikalpatayā yuktā jñānayogaṃ samācaret// \L
%
%ekam evā jagat paśyad visvātmāsu vibhāsvaraṃ//
%avikalpatayā yuktā jñānayogaṃ samācaret// \B
%
%ekam eva jagat paśyed viśvātmā viśvabhāvanaḥ/
%iti kṛtvā tu vai yukto jñānayogaṃ samācaret// SVARODAYA
%
%ekam eva jagat paśyed dviśvātmāsu vibhāsvaraṃ/
%avikalpatayā yuktyā jñānayogaṃ samācaret//1// \N1
%
%ekam eva jagat paśyed dviśvātmāsu vibhāsvaraṃ//
%avikalpatayā yuktyā jñānayogaṃ samācaret//1// \D1
%
%ekam eva jagat paśyed dviśvātmāsu vibhāsvaraṃ//
%avikalpatayā yuktyā jñānayogaṃ samācaret//1// \N2
%
%ekam eva jagataḥ paśyed dviśvātmāsu vibhāsvaraṃ
%āvikalpatayā yuktyā jñānayogaṃ samācaret//1// \U1
%
%ekam eva jagataḥ paśyed dviśvātmāsu vibhāsvaraṃ
%āvikalpatayā yuktyā jñānayogaṃ samācaret// \U2
%------------------------------
%He shall see the world truly as being one, shining in all selves. 
%By applying indistinctness he shall accomplish \textit{jñānayoga}.   
%------------------------------
\begin{tlg}
  \tl{\note[type=testium, labelb=s45.z2, lem=ekam eva]{Ysv: ekam eva jagat paśyed viśvātmā viśvabhāvanaḥ | iti kṛtvā tu vai yukto jñānayogaṃ samācaret ||}
eka\skp{-m}\app{\lem[wit={ceteri}, alt={eva}]{\skm{-m}eva}
  \rdg[wit={L,B}]{evā}}
\app{\lem[wit={ceteri},alt={jagat}]{jaga\skp{t-}}
  \rdg[wit={P}]{cayat}
}\app{\lem[wit={ceteri},alt={paśyed}]{\skm{t-}paśye\skp{-d}}
  \rdg[wit={B}]{paśyad}
}\app{\lem[wit={P,L,B},alt={viśvātmāsu}]{\skm{d-}viśvātmāsu}
  \rdg[wit={E}]{viśvāvasu}
  \rdg[wit={N1,D1,N2,U1,U2}]{dviśvātmāsu}}
vibhāsvaraṃ/}\\
\tl{\app{\lem[wit={ceteri}]{avikalpatayā}
  \rdg[wit={U1,U2}]{āvikalpatayā}}
\app{\lem[wit={ceteri}]{yuktyā}
  \rdg[wit={L,B}]{yuktā}} 
jñānayogaṃ samācaret\dd{}1\hskip-2pt\dd{}}
\end{tlg}
\end{ekdosis}
%%%%%%%%%%%%%
%%%%%%%%%%%%%
%%%%%%%%%%%%%
%%%%%%%%%%%%% 
%%%%%%%%%%%%%
    \begin{ekdosis}
      \ekddiv{type=ed}
%------------------------------
%yatra yatra sthito vāpi sarvajñānamayaṃ jagat/ 
%sa evaṃ vetti bodhena so pi jñānādhikāraṇāt//2// \E 
%
%yatra yatra sthito vāpi sarvajñānamayaṃ jagat  
%ya evaṃ vetti bodhena so pi jñānādhikāravān \P
%
%yatra yatra sthito vāpi sarvajñānamayaṃ jagat//  
%ya evaṃ vetti bodhena so pi jñānādhikāravān// \L
%
%yatra yatra sthito vāpi sarvajñānamayaṃ jagat//  
%ya evaṃ ve bodhena so pi jñānādhikāravān// \B
%
%yatra tatra sthito vāpi sarvajñānamayaṃ jagat/
%ya evam asti bodhena so'pi jñānādhikāravān/ \SVARODAYA
%
%yatra yatra sthito vāpi sarvajñānamayaṃ jagat/
%ya evaṃ vetti bodhena so pi jñānādhikāravān//2//\N1
%
%yatra yatra sthito vāpi sarvajñānamayaṃ jagat//
%ya evaṃ vetti bodhena so pi jñānādhikāravān//2//\D1
%
%yatra yatra sthito vāpi sarvajñānamayaṃ jagat//
%ya evaṃ vetti bodhena so pi jñānādhikāravān//2//\N2
%
%yatra yatra sthito vāpi sarvajñānamayaṃ jagat  %%%273.jpg
%evaṃ vette na bodhena so pi jñānādhikāravān 2    \U1
%
%yatra yatra sthito hiṃsa sarvajñānamayaṃ jagat//  
%evaṃ vetti bodhena so pi jñānādhikāravān// 2    \U2
%------------------------------
%Wherever the world is established or made of omniscience,
%who knows thus by means of insight, he is a like an expert of knowledge.      
%------------------------------
\begin{tlg}
  \tl{\note[type=testium, labelb=s45.z4, lem=yatra yatra]{Ysv: yatra tatra sthito vāpi sarvajñānamayaṃ jagat | ya evam asti bodhena so'pi jñānādhikāravān ||}
    yatra tatra sthito \app{\lem[wit={ceteri}]{vāpi}
      \rdg[wit={U2}]{hiṃsa°}} sarvajñānamayaṃ jagat/}\\
  \tl{\app{\lem[wit={ceteri}]{ya evaṃ}
      \rdg[wit={U1,U2}]{evaṃ}}
    \app{\lem[wit={ceteri}]{vetti}
      \rdg[wit={U1}]{vette na}
      \rdg[wit={B}]{ve}} bodhena so pi
    \app{\lem[wit={ceteri}]{jñānādhikāravān}
      \rdg[wit={E}]{jñānādhikāraṇāt}}\dd{}2\hskip-2pt\dd{}}
\end{tlg}
   \end{ekdosis}
   %%%%%%%%%%%%%%%
   %%%%%%%%%%%%%%%
   %%%%%%%%%%%%%%
   %%%%%%%%%%%%%% 
   %%%%%%%%%%%%%%%
  \begin{ekdosis}
    \ekddiv{type=ed}
%------------------------------
%
%\om!!!!!                                                                                                        \E
%
%prāpnoti śāmbhavīmantrān  sadā nityaparāyaṇaḥ/   yathā nyagrodhavījaṃ hi kṣitau   vaptur drumāyate/               \SVARODAYA  
%prāpnoti śāmbhavīṃ sattāṃ sadāṃdvaitaparāyaṇaḥ   yathā nyagrodhabījaṃ hi kṣitāv   uptaṃ drumāyate likāṃ pa..vāḥ 4 \P  7640.jpg last line check word!!!
%prāpnoti śāmbhavīṃ sattān sadādvaitaparāyaṇaḥ//  yathā nyagrodhavīja  hi kṣitāv   utpadyate yathā//               \L
%prāpnoti śāmbhaviṃ sattāṃ sadādvaitaparāyaṇaḥ//  yathā nyagrodhabījāṃ hi kṣitī    utpadyate//                      \B
%prāpnoti sāṃbhavīṃ satta  sadādvaitaparāyaṇaḥ//  yathā nyagrodhavījaṃ hi kṣitāv   uptaṃ drumāyate 3//              \N1
%prāpnoti sāṃbhavīsattāṃ   sadādvaitaparāyaṇaḥ//  yathā nyagrodhavījaṃ hi kṣitāv   uptaṃ drumāyate//                \D1
%prāpnoti sāṃbhavīsattā    sadādvaitaparāyaṇaḥ//  yathā nyagrodhavījaṃ hi kṣitāv   uptaṃ drumāyate//                \N2 %drumaayate=denom. wie ein beim  sein 
%prāpnoti sāṃbhavīsattāṃ   sadādvaitaparāyaṇaḥ    yathā nyagrodhabījaṃ hi kṣitāptā ukta drumāyate 3              \U1
%prāpnoti sāṃbhavīsattāṃ   yadādvaitaparāyaṇaḥ//  yathā nyagrodhabījaṃ hi kṣitāv   uptaṃ drumāyate//               \U2
%------------------------------
%He always attains the reality of śāmbhavī - the goal of eternal non-duality.  
%Just as the seed of the Nyagrodha scattered onto the soil [always] becomes a tree.
%------------------------------
\begin{tlg}
  \tl{\note[type=testium, labelb=s45.z6, lem=prāpnoti]{Ysv: prāpnoti śāmbhavīmantrān sadā nityaparāyaṇaḥ | yathā nyagrodhavījaṃ hi kṣitau vaptur drumāyate ||}
    \app{\lem[wit={ceteri}]{prāpnoti}
      \rdg[wit={E}]{\om}}
    \app{\lem[wit={D1,U1,U2}]{sāṃbhavīsattāṃ}
      \rdg[wit={P,B}]{śāmbhavīṃ sattāṃ}
      \rdg[wit={L}]{śāmbhavīṃ sattān}
      \rdg[wit={N1}]{sāṃbhavīṃ satta}
      \rdg[wit={N2}]{sāṃbhavīsattā}
      \rdg[wit={E}]{\om}}
    \app{\lem[wit={ceteri}]{sadādvaitaparāyaṇaḥ}
      \rdg[wit={U1}]{sadāṃdvaita°}
      \rdg[wit={E}]{\om}}/}\\
  \tl{\app{\lem[wit={ceteri}]{yathā}
      \rdg[wit={E}]{\om}}
    \app{\lem[wit={ceteri}]{nyagrodhabījaṃ}
      \rdg[wit={N1,N2,D1}]{°vījaṃ}
      \rdg[wit={L}]{°vīja}
      \rdg[wit={E}]{\om}}
    \app{\lem[wit={ceteri}]{hi}
      \rdg[wit={E}]{\om}}
    \app{\lem[wit={ceteri},alt={kṣitāv}]{kṣitāv\skm{-u}}
      \rdg[wit={B}]{kṣitī}
      \rdg[wit={U1}]{kṣitāptā}
      \rdg[wit={E}]{\om}
 }\app{\lem[wit={ceteri},alt={uptaṃ drumāyate}]{\skp{-u}ptaṃ drumāyate}
      \rdg[wit={P}]{uptaṃ drumāyate likāṃ pa..vāḥ}
      \rdg[wit={L}]{utpadyate yathā}
      \rdg[wit={B}]{utpadyate}
      \rdg[wit={U1}]{ukta drumāyate}
      \rdg[wit={E}]{\om}}\dd{}3\hskip-2pt\dd{}}
\end{tlg}
\end{ekdosis}
    %%%%%%%%%%%%
    %%%%%%%%%%%%
    %%%%%%%%%%%%
    %%%%%%%%%%%
    %%%%%%%%%%%%
  \begin{ekdosis}
    \ekddiv{type=ed}
%------------------------------
%ekāntaṃ  naikadā  svena   dṛśyate  daśadhā  kṛtaḥ/  mūlāṅkurasya  coddaṇḍāḥ śākhākuṇḍalapallavāḥ//3//   \E    cod?von cud? Wurzel in guṇa + daṇḍa? !!! em. zu śaśvadhā = immer wieder, jederzeit 
% \om                                                                                                    \P
%ekāṃte   nekadhā  svena   dṛśyaṃte daśadhāt kṛp?tā/ mūlāṃkurutva kudaṃḍaḥ  śākhākilekālapallavā        \B
%ekāṃte   nekadhā  svena   dṛśyaṃte daśadhāt kṛtaḥ/  mūlāṃkurutva kudaṃḍa   śākhākalikālapallavā        \L
%ekāṃtaṃ  naikadhā śveta   dṛśyate  daśadhā  kṛtā//  mūlāṃkurutva codaṃḍaḥ  śāvārakumbhalapallavaḥ//4// \N1   
%ekāṃtaṃ  naikadhā śvetana dṛśyate  daśadhā  kṛtā//  mūlāṃkurutva codarāṭaḥ śālavākumapadṛtravā//4//    \D1
%ekāṃtaṃ  naikadhā śvetana dṛśyet   śadhā    kṛtā//  mūlāṃkurutva codarāṭaḥ śākhākumbhalapallavā//4//   \N2
%yekāṃtaṃ naikadhā svena   dṛśyate  śadhā    kṛtā    mūlāṃkurutva codaṃḍa   śākhākumbhalapallavaḥ       \U1
%ekāṃtaṃ  naikadhā svetana dṛśyate  daśadhā  kṛtiḥ// mūlāṃkurutva codaṃḍaḥ  śākhākusumapallavāḥ//       \U2
%------------------------------
%Nur eines, nicht zusammen mit dem Ich wird das zehnfach gemachte gesehen. 
%Die aufgerollten Sprossen der Äste, welche die austreibendem Stöcke sind vom Spross der Wurzel. 
%------------------------------
%Die absoluten Einheit (ekāntaṃ), wird als multibel (nämlich) aus zehn Teilen gemacht von einen selbst gesehen. !!!!!
%Die aufgerollen Sprösslinge der Zweige sind austreibende Stengel des Wurzeltriebes.  
%------------------------------
%The absolute unity (ekāntaṃ), is seen as a multiple (namely) made up of ten parts by oneself.
%The rolled up shoots of the branches are the sprouting stalks of the root shoot.  
%------------------------------
%The absolute unity (ekāntaṃ), is seen as manifoldly created again and again by oneself.
%The rolled up shoots of the branches are the sprouting stalks of the root shoot.  
%------------------------------
\begin{tlg}
  \tl{\note[type=philcomm, labelb=s45.z6, lem=ektāntaṃ]{The remaining verses of the \textit{jñānayoga}-section are not found in the Ysv.}
\app{\lem[wit={ceteri}]{ekāntaṃ}
  \rdg[wit={B,L}]{ekānte}
  \rdg[wit={U1}]{yekāṃtaṃ}
  \rdg[wit={P}]{\om}}
\app{\lem[wit={ceteri}]{naikadhā}
  \rdg[wit={E}]{naikadā}
  \rdg[wit={B,L}]{nekadhā}
  \rdg[wit={P}]{\om}}
\app{\lem[wit={ceteri}]{svena}
  \rdg[wit={N1}]{śveta}
  \rdg[wit={D1,N1}]{śvetana}
  \rdg[wit={P}]{\om}}
\app{\lem[wit={ceteri}]{dṛśyate}
  \rdg[wit={B,L}]{dṛśyaṃte}
  \rdg[wit={N2}]{dṛśyet}
  \rdg[wit={P}]{\om}}
\app{\lem[wit={E,N1,N2}]{daśadhā}    %%[type=conjecture, resp=egoscr]{śaśvadhā}????
  \rdg[wit={E,N1,N2}]{\conj daśadhā}
  \rdg[wit={B,L}]{daśadhāt}
  \rdg[wit={N2,U1}]{śadhā}
  \rdg[wit={P}]{\om}}
\app{\lem[type=emendation, resp=egoscr]{kṛtaṃ}
  \rdg[wit={E,L}]{\korr kṛtaḥ}
  \rdg[wit={N1,N2,D1,U1}]{kṛtā}
  \rdg[wit={B}]{kṛptā}
  \rdg[wit={U2}]{kṛtiḥ}
  \rdg[wit={P}]{\om}}/}\\
 \tl{\app{\lem[wit={E}]{mūlāṅkurasya}
  \rdg[wit={ceteri}]{mūlāṃkurutva}
  \rdg[wit={P}]{\om}}
\app{\lem[wit={E,N1,U2}]{coddaṇḍāḥ}
  \rdg[wit={D1,N2}]{codarāṭaḥ}
  \rdg[wit={B}]{kudaṃjaḥ}
  \rdg[wit={L}]{kudaṃḍa}
  \rdg[wit={P}]{\om}}
\app{\lem[wit={E}]{śākhākuṇḍalapallavāḥ}
  \rdg[wit={B,L}]{śākhākilekālapallavā}
  \rdg[wit={N1,U1}]{śāvārakumbhalapallavaḥ}
  \rdg[wit={N2}]{śākhākumbhalapallavā}
  \rdg[wit={D1}]{śālavākumapadṛtravā}
  \rdg[wit={U2}]{śākhākusumapallavāḥ}
  \rdg[wit={P}]{\om}}\dd{}4\hskip-2pt\dd{}}
\end{tlg}
\end{ekdosis}
    %%%%%%%%%%%%%%%
    %%%%%%%%%%%%%%%
    %%%%%%%%%%%%%%
    %%%%%%%%%%%%%%
    %%%%%%%%%%%%%%%   
   \begin{ekdosis}
     \ekddiv{type=ed}
%------------------------------
%srehapuṇyaphalaṃ   bīje vistaro yaṃ svabhāvataḥ/  tathāsau   nirmalo  nityo nirvikāro niraṃjanaḥ//4// \E
%snehapuṣpaphalaṃ   bīje vistāro yaṃ svabhāvataḥ   tāthāpasau nirmalau nityo nirvikāro niraṃjanaḥ     \P   %%7641.jpg Z.1
%snehe puṣpaphala---bīja-vistāro ya  svabhāvatāḥ   yāthāsau   nirmalo  nityo nirvikāro niraṃjanaḥ//    \B
%snehe puṣpaphala---bīja-vistāro ya  svabhāvatāḥ// tāthāsau   nirmalo  nityo nirvikāro niraṃjanaḥ//    \L
%snehapuṣpaphalaṃ   bīje vistārā yaṃ svabhāvataḥ/  tathāsau   nirmalo  nityo nirvikāro niraṃjanaḥ//5// \N1
%snehapuṣpaphalaṃ   bīje vistārā yasya  bhāvataḥ// tathāsau   nirmalo  nityo nirvikāro niraṃjanaḥ//5// \D1
%snehapuṣpaphalaṃ   vīje vistāro yaṃ svabhāvataḥ// tathāsau   nirmalo  nityo nirvikāro niraṃjanaḥ//5// \N2
%snehapuṣpaṃ phalaṃ bīje vistāro yaḥ svabhāvataḥ   tathāsau   nirmalo  nityo nirvikāro niraṃjanaḥ 5    \U1  %%%%274.jpg
%snehapuṣpaphalaṃ   bīje vistāro yaṃ svabhāvataḥ// tathāsau   nirmalo  nityo nirvikāro niraṃjanaḥ// 5  \U2 %%%first Śloka in this series that is numbered in U2 
%------------------------------
%Aufgrund seines inhärenten Wesens ist dieser Ast mit seinen Zweigen, welcher die Frucht der Blüte der Liebe ist, im Samen.
%Gewiss, ist jenes rein, ewig, unveränderlich und makellos. 
%------------------------------
%By virtue of its inherent nature, this branch with its branches, which is the fruit of the flower of love, is in the seed.
%Certainly, that is pure, eternal, unchanging and immaculate.
%------------------------------
\begin{tlg}
  \tl{
\app{\lem[wit={P,N1,N2,D1,U2}]{snehapuṣpaphalaṃ}
  \rdg[wit={B,L}]{snehe puṣpaphala°}
  \rdg[wit={U1}]{snehapuṣpaṃ phala}
  \rdg[wit={E}]{srehapuṇyaphalaṃ}}
\app{\lem[wit={ceteri}]{bīje}
  \rdg[wit={B,L}]{bīja}
  \rdg[wit={N2}]{vīje}}
\app{\lem[wit={ceteri}]{vistāro}
  \rdg[wit={N1,D1}]{vistārā}}
\app{\lem[wit={E,P,N1,N2,U2}]{'yaṃ}
  \rdg[wit={B,L}]{ya}
  \rdg[wit={U1}]{yaḥ}
  \rdg[wit={D1}]{yasya}}
\app{\lem[wit={ceteri}]{svabhāvataḥ}
  \rdg[wit={B,L}]{svabhāvatāḥ}
  \rdg[wit={D1}]{bhāvataḥ}}/}\\
\tl{\app{\lem[wit={ceteri}]{tathāsau}
    \rdg[wit={B}]{yathāsau}
    \rdg[wit={P}]{tathāpasau}}
  \app{\lem[wit={ceteri}]{nirmalo}
    \rdg[wit={P}]{nirmalau}}
nityo nirvikāro niraṃjanaḥ\dd{}5\hskip-2pt\dd{}}
\end{tlg}
\end{ekdosis}
%%%%%%%%%%%%%%%
%%%%%%%%%%%%%%%
%%%%%%%%%%%%%%%%
%%%%%%%%%%%%%%%%
%%%%%%%%%%%%%%
\begin{ekdosis}
%------------------------------
%eko  nekaḥ  svayaṃbhūś ca dhāmnā ca    bahudhā sthitaḥ/   paṃcatattvamanobuddhi-māyāhaṃkāravikriyāḥ //5//   \E
%eko  nekaḥ  svayaṃbhūś ca svadhāmnā    bahudhā sthitāḥ    paṃcatatvamanobuddhir māyāhaṃkāravikriyāḥ   6     \P
%eko  neka   svayaṃbhūś ca dhāmnāya     bahudhā sthitaḥ//  paṃcatatvamanobuddhi--māyāhaṃkāravikriyā  //      \B
%eko  nekaḥ  svayaṃbhūś ca svadhābhāva  bahudhā sthitāḥ//  paṃcatatvamanobuddhi--māyāhaṃkāravikriyā  //      \L
%eko  nekaḥ  svayaṃbhuś ca svayāṃmnā    bahudhā sthitaḥ/   paṃcatatvamanobuddhir māyāhaṃkāravikriyā  //6//   \N1
%eko  nekaḥ  svayaṃbhaś ca svadhā...ṣ   bahudhā sthitāḥ//  paṃcatatvamanobuddhir māyāhaṃkāravikriyā  //6//   \D1
%eko  neka   svayaṃbhūś ca svadhāmnāva  bahudhā sthitaḥ//  paṃcatatvamanobuddhir māyāhaṃkāravikriyā  //6//   \N2
%yeko naika/ svayaṃbhūtyā  svabhāvā     bahudhā sthitaḥ    paṃcatatvamanobuddhir māyāhaṃkāravikriyāḥ   6     \U1
%eko  naiko  svayaṃbhūś ca svadhāmnā    bahudhā sthitaḥ//  paṃcatatvamanobuddhir māyāhaṃkāravikriyā  //6//   \U2
%------------------------------
%Eins, nicht eins und aus sich selbst heraus seiend durch das eigene Walten und Wirken mannigfach existierend,
%[als] fünf Prinzipien (\textit{tattva}), welche da sind: denkender Verstand (\textit{manas}), Intellekt (\textit{buddhi}), Illusion (\textit{māya}), Individuation (\textit{ahaṃkāra}) und Modifikationen (\textit{vikriyā}). 
%------------------------------
%One, not one and self-existing, existing in manifold ways through its own rule and work,
%[as] five principles (\textit{tattva}) which are: thinking mind (\textit{manas}), intellect (\textit{buddhi}), illusion (\textit{māya}), individuation (\textit{ahaṃkāra}) and modifications ( \textit{vikriya}).
%------------------------------
  \ekddiv{type=ed}
  \begin{tlg}
  \tl{
\app{\lem[wit={ceteri}]{eko}
  \rdg[wit={U1}]{yeko}}
\app{\lem[type=emendation, resp=egoscr]{naikaḥ}
  \rdg[wit={U1}]{\korr naika}
  \rdg[wit={U2}]{naiko}
  \rdg[wit={ceteri}]{nekaḥ}
  \rdg[wit={B,N2}]{neka}}
\app{\lem[wit={ceteri}]{svayaṃbhūś\skp{-}ca}
  \rdg[wit={U1}]{svayaṃbhūtyā}}
\app{\lem[wit={P,U2}]{svadhāmnā}
  \rdg[wit={E}]{dhāmnā ca}
  \rdg[wit={B}]{dhāmnāya}
  \rdg[wit={L}]{svadhābhāva}
  \rdg[wit={N1}]{svayāṃmnā}
  \rdg[wit={D1}]{svadhā..ṣa}
  \rdg[wit={N2}]{svadhāmnāva}
  \rdg[wit={U1}]{svabhāvā}}
bahudhā
\app{\lem[wit={P,L,D1}]{sthitāḥ}
  \rdg[wit={ceteri}]{sthitaḥ}}/}\\
\tl{\app{\lem[wit={E}]{paṃcatattvamanobuddhimāyāhaṃkāravikriyāḥ}
    \rdg[wit={P,U1}]{°buddhir}
    \rdg[wit={B,L}]{°kriyā°}
    \rdg[wit={N1,N2,D1,U2}]{°buddhir māyāhaṃkāravikriyā}}\dd{}6\hskip-2pt\dd{}}
\end{tlg}
\end{ekdosis}
\ekdpb*{}
%%%%%%%%%%%%%%%
%%%%%%%%%%%%%%%%
%%%%%%%%%%%%%%%
%%%%%%%%%%%%%%%
%%%%%%%%%%%%%%
\begin{ekdosis}
%------------------------------ 
%evaṃ daśavidhaṃ viśvaṃ lokālokasavistaram/   eka  eva na cānyo sti yo jānāti sa tattvavit//6// \E
%evaṃ daśavidhaṃ viśvaṃ lokālokasavistaraṃ    eka  eva na cānyo sti yo jānāti sa tatvavit 6 \P
%evaṃ daśavidhā  viśvaṃ lokālokasavistaraṃ//  eka  eva na cānyā sti yo jānāti sa tatvavit// \B
%evaṃ daśavidhā  viśvaṃ lokālokasavistaraṃ//  eka  eva na cānyo sti yo jānāti sa tatvavit// \L
%evaṃ daśavidhaṃ viśvaṃ lokālokasavistarāṃ/   eka  eva na cānyo sti yo nānāti sa tatvavit//7// \N1
%evaṃ daśavidhaṃ viśvaṃ lokālokasavistaraṃ//  eka  eva na cānyo sti yo jānāti sa tatvavit//7// \D1
%evaṃ daśavidhā  viśvaṃ lokālokasavistaraṃ//  eka  eva na cānyo sti yo jānāti sa tatvavit//7// \N2
%evaṃ daśavidha--viśvaṃ lokālokasavistaraṃ    eka yeva na cānyo sti yo jānāti sa tatvavit 7 \U1
%evaṃ daśavidhaṃ viśvaṃ lokāloke savistaraṃ// ekam eva na cānyo sti yo jānāti sa tatvavit//7// \U2 %%%409.jpg 
%------------------------------
%Auf diese Weise durchdringen die zehn Variationen die Welt und die Nicht-Welt im vollen Umfang.  
%Nur das Eine ist und nicht etwas anderes: Wer das weiß ist ein Kenner der Realität.  
%------------------------------
%In this way, the ten variations fully permeate the world and the non-world.
%Only one thing is and not something else: Whoever knows this is a connoisseur of reality.
 %------------------------------
  \ekddiv{type=ed}
  \begin{tlg}
   \tl{
     evaṃ
     \app{\lem[wit={B,L,N2}]{daśavidhā viśvaṃ}
       \rdg[wit={E,P,N1,D1,U2}]{daśavidhaṃ viśvaṃ}
       \rdg[wit={U1}]{daśavidhaviśvaṃ}}
     \app{\lem[wit={ceteri}]{lokālokasavistaram}  
       \rdg[wit={N1}]{°savistarāṃ}
       \rdg[wit={U2}]{°loke savistaraṃ}}/}\\
   \tl{\app{\lem[wit={ceteri}]{eka}
       \rdg[wit={U2}]{ekam}}
       \app{\lem[wit={ceteri}]{eva}
         \rdg[wit={U1}]{yeva}}
       na cānyo sti yo jānāti sa tattvavit/dd{}7\hskip-2pt\dd{}}\\
   \end{tlg}
 \end{ekdosis}
  %%%%%%%%%%%%%%%%%%%%%%%%%%%%%
 %%%%%%%%%%%%%%%%%%%%%%%%%%%%%
 %%%%%%%%%%%%%%%%%%%%%%%%%%%%%
 %%%%%%%%%%%%%%%%%%%%%%%%%%%%%
 %%%%%%%%%%%%%%%%%%%%%%%%%%%%%
  \begin{ekdosis}
    \ekddiv{type=ed}
    \begin{prose}     
%------------------------------
%pṛthvīvanaspatiparvatādisthārarūpaḥ         saṃsāra---manuṣyahastyaśvapakṣītyādiko    jaṃgamarūpaḥ   saṃsāraḥ// \E
%pṛthvīvanaśpatiparvatādisthāvararūpaḥ       saṃsāraḥ  manuṣyahastyaś ca pakṣītyādiko  jaṃgamarūpaḥ   saṃsāraḥ \P
%pṛthvīvanaspatīparvatādisthāvararūpā        saṃsāraḥ/ manuṣyahasteśvapakṣītyādiko     jaṃgamarūpaḥ   saṃsāraḥ// \B
%pṛthvīvanaspatiparvatādisthāvararūpā        saṃsāraḥ  manuṣyahasteśvapakṣītyādiko     jaṃgamarūpā    saṃsāraḥ// \L
%pṛthvīvanaspatīparvvate tyādisthāvararūpaḥ  saṃsāraḥ  manuṣyahastīaśvapakṣītyādiko    jaṃgamarūpaḥ   saṃsāraḥ// \N1
%pṛthvīvanaspatīparvato tyādisthāṃvararūpaḥ  saṃsāraḥ  manuṣyahastīaśvapakṣītyādiko    jaṃgamaḥ rūpaḥ saṃsāraḥ// \D1
%pṛthvīvanaspatiparvate 'thyādisthāvararūpa  saṃsāraḥ  manuṣyahastipakṣītyādiko        jaṃgamarūpaḥ   saṃsāraḥ// \N2
%pṛthivīvanaspatīparvate iyādisthāvararūpaḥ  saṃsāra---manuṣyahastiasvapakṣītyādiko    jagadrūpaḥ     saṃsāro \U1
%pṛthvīvanaspatiparvatādisthāvararūpaḥ       saṃsāraḥ//manuṣyahasttyaś ca pakṣītyādiko jaṃgamarūpaḥ   saṃsāraḥ//8// \U2
%------------------------------
%Der Geburtenkreislauf ist die Erscheinung der Pflanzenwelt, der Berge, der Bäume, der Erde etc. Der Geburtenkreislauf ist die Erscheinung der Lebewesen beginnend mit Vögeln, Pferden, Elefanten und Menschen. 
%------------------------------
%The cycle of birth is the appearance of the plant world, mountains, trees, earth etc. The cycle of birth is the appearance of living beings beginning with birds, horses, elephants and humans. 
%------------------------------
\app{\lem[wit={ceteri},alt={pṛthvī°}]{pṛthvī}
        \rdg[wit={U1}]{pṛthivī°}
      }\app{\lem[wit={E,N2,U2},alt={°vanaspati°}]{vanaspati}
        \rdg[wit={P}]{vanaś°}
        \rdg[wit={B,L,N1,D1,U1}]{°patī°}
      }\app{\lem[wit={P,B,L,U2}, alt={°parvatādisthāra°}]{parvatādisthāvara}
        \rdg[wit={E}]{°parvatādisthāra°}
        \rdg[wit={N1}]{°parvvate tyādisthāvara°}
        \rdg[wit={N2}]{°parvate 'thyādisthāvara°}
        \rdg[wit={D1}]{°parvato tyādisthāṃvara°}
        \rdg[wit={N2}]{°parvate 'thyādisthāvara°}
        \rdg[wit={U1}]{°parvate iyādisthāvara°}
      }\app{\lem[wit={ceteri}]{rūpaḥ} \rdg[wit={L,B}]{rūpā}
        \rdg[wit={N2}]{rūpa}} \app{\lem[wit={ceteri}]{saṃsāraḥ}
        \rdg[wit={E,U1}]{saṃsāra°}}/
      manuṣya\app{\lem[wit={B,L},alt={°hasteśvapakṣītyādiko}]{hasteśvapakṣītyādiko}
          \rdg[wit={E}]{°hastyaśvapakṣītyādiko}
          \rdg[wit={N1,D1}]{°hastīaśvapakṣītyādiko}
          \rdg[wit={N2}]{°hastipakṣītyādiko}
          \rdg[wit={U1}]{°hastiasvapakṣītyādiko}
          \rdg[wit={U2}]{°hasttyaś ca pakṣītyādiko}}
        \app{\lem[wit={ceteri}]{jaṃgamarūpaḥ}
          \rdg[wit={L}]{°rūpā}
          \rdg[wit={D1}]{jaṃgamaḥ rūpaḥ}
          \rdg[wit={U1}]{jagad°}}
        \app{\lem[wit={ceteri}]{saṃsāraḥ}
          \rdg[wit={U1}]{saṃsāro}}/
%   \end{prose}
% \end{ekdosis}
%%%%%%%%%%%%%%%
%%%%%%%%%%%%%%%
%%%%%%%%%%%%%%%%
%%%%%%%%%%%%%%%%
%%%%%%%%%%%%%%%%
% \begin{ekdosis}
%  \ekddiv{type=ed}
%  \begin{prose}
%------------------------------
%atha ca   yo  dṛṣṭiviṣayaḥ  sa dṛśya  ucyate/  yo dṛṣṭyā na vīkṣyate sa adṛśya ity  ucyate/ \E
%atha ca   yo  dṛṣṭiviṣayaḥ  sa dṛśya  ucyate   yo dṛṣṭyā na vīkṣyate sa adṛśya ity  ucyate  %%%7641.jog
%atha ca// yo  daṣṭiviṣayaḥ  sa dṛśya  ucyate// yo dṛṣṭyā na vīkṣyate sa adṛśya ty   ucyate// \B
%atha ca   yo ddṛṣṭiviṣayaḥ  sa dṛśya  ucyate// yo dṛṣṭyā na vīkṣyate sa adṛśye ty   ucyate... \L
%atha ca   ya ddṛṣṭiviṣayaḥ  sa dṛśyad ucyate   yo dṛṣṭyā na vīkṣyate sa adṛśya ity  ucyate// \N1
%atha vā   ya dārṣṭiviṣayaḥ  sa dṛśya  ucyate/  yo dṛṣṭyā na vīkṣyate sa adṛśya ity  ucyate// \D1
%atha ca   ya  drṣṭiviṣayaḥ  sa dṛśya  ucyate/  yo dyā    na vīkṣyate sa adṛśya śaty ucyate/ \N2
%atha ca   yaḥ drṣṭiviṣayaḥ  sa dṛśy---ucyate   yo dṛṣṭvā na vīkṣyate sa adṛśya ity  ucyate \U1
%atha ca   yo  dṛṣṭiviṣayaḥ  sa dṛśya  ucyate// yo dṛṣṭyā na vīkṣyate sa adṛśya ity  ucyate// \U2
%------------------------------
%Und dann, wer einer ist, der ein [Sinnes]objekt des Sehens ist, der wird gesagt, ist sichtbar. Wer nicht durch das Sehen gesehen wird, der wird gesagt ist unsichtbar. 
%------------------------------
atha
      \app{\lem[wit={ceteri}]{ca}
        \rdg[wit={D1}]{vā}}
      \app{\lem[wit={ceteri}]{yo}
        \rdg[wit={U1}]{yaḥ}
        \rdg[wit={N1,N2,D1}]{ya}}
      \app{\lem[wit={ceteri}]{dṛṣṭi}
        \rdg[wit={L,N1}]{ddṛṣṭi}
        \rdg[wit={B}]{daṣṭi}
        \rdg[wit={D1}]{dārṣṭi}
}viṣayaḥ sa
\app{\lem[wit={ceteri}]{dṛśya}
  \rdg[wit={N1}]{dṛśyad}
  \rdg[wit={U1}]{dṛṣy°}}
ucyate/
yo
\app{\lem[wit={ceteri}]{dṛṣṭyā}
  \rdg[wit={N2}]{dyā}}
na vīkṣyate sa adṛṣya
\app{\lem[wit={ceteri},alt={ity°}]{i\skp{ty-u}}
  \rdg[wit={L,B}]{ty°}
  \rdg[wit={N2}]{śaty°}
}\skm{ty-u}cyate/
%    \end{prose}
%  \end{ekdosis}
%%%%%%%%%%%%%%%
%%%%%%%%%%%%%%
%%%%%%%%%%%%%%
%%%%%%%%%%%%%%%
%%%%%%%%%%%%%%%
% \begin{ekdosis}
%     \ekddiv{type=ed}
%     \begin{prose}
%------------------------------
%evaṃ saṃsārasya svātmano  bhedaṃ dūrīkṛty---aikam eva darśanaṃ sa eva jñānayogaḥ/   \E
%evaṃ saṃsāra----svātmano  bhedaṃ dūrīkṛtya  aikyena   darśanaṃ        jñānayogaḥ    \P
%evaṃ saṃsārasya svātmano  bheda--dūrīkṛtya  aikyona   darśanaṃ        jñānayogaḥ/   \B
%evaṃ saṃsāra----svātmano  bhedaṃ dūrīkṛtya  aikyona   darśanaṃ        jñānayogaḥ... \L
%evaṃ saṃsārasya svātmanaḥ bhedāṃ dūrīkṛtya  ekyena    darśanaṃ        jñānayogaḥ//  \N1
%evaṃ saṃsārasya svātmanaḥ bhedāṃ dūrīkṛtya  ekyena    darśanaṃ        jñānayogaḥ/   \D1
%evaṃ saṃsārasya svātmanaḥ bhedaṃ dūrīkṛtya  ekena     darśanaṃ        jñānayogaḥ/   \N2
%evaṃ saṃsārasya svātmanaḥ bhedaṃ dūrīkṛtya  ekānta    darśanaṃ        jñānayogaḥ    \U1
%evaṃ saṃsāra....svātmanoḥ bhedaṃ dūrīkṛtyaṃ ekye?     darśanaṃ        jñānayoga     \U2
%------------------------------
%In this way the view of separation of one's own self which is subjected to transmigration is to be removed by means of [applying the view of] unity. Only this is Jñānayoga.  
%------------------------------
evaṃ
       \app{\lem[wit={ceteri}]{saṃsārasya}
         \rdg[wit={P,L,U2}]{saṃsāra°}}
       \app{\lem[wit={E,P,B,L}]{svātmano}
         \rdg[wit={N1,D1,N2,U1}]{svātmanaḥ}
         \rdg[wit={U2}]{svātmanoḥ}}
       \app{\lem[wit={ceteri}]{bhedaṃ}
         \rdg[wit={B}]{bheda}
         \rdg[wit={D1,N1}]{bhedāṃ}}
\app{\lem[wit={U2}]{dūrīkṛtyaṃ}
  \rdg[wit={ceteri}]{°kṛtya}
  \rdg[wit={E}]{°kṛty}}
\app{\lem[wit={P}]{aikyena}
  \rdg[wit={E}]{aikam eva}
  \rdg[wit={P,B,L}]{aikyona}
  \rdg[wit={N1,D1}]{ekyena}
  \rdg[wit={N2}]{ekena}
  \rdg[wit={U1}]{ekānta}
  \rdg[wit={U2}]{ekye}}
darśanaṃ
\app{\lem[wit={E}]{sa eva}
  \rdg[wit={ceteri}]{\om}}
\app{\lem[wit={ceteri}]{jñānayogaḥ}
  \rdg[wit={U2}]{jñānayoga}}/ 
 %  \end{prose}
 %\end{ekdosis}
 %%%%%%%%%%%%%%
 %%%%%%%%%%%%%%
 %%%%%%%%%%%%%%
 %%%%%%%%%%%%%%%
 %%%%%%%%%%%%%%%
 % \begin{ekdosis}
 %   \ekddiv{type=ed}
 %   \begin{prose}
%------------------------------
%tasya         kāraṇāt kālaḥ śarīranāśaṃ na karoti/ \E
%tasya         kāraṇāt kālaḥ śarīranāśaṃ na karoti/ \P
%tasya         karaṇāt kālaḥ śarīranāśaṃ na karoti// \B
%tasya         karaṇāt kālaḥ śarīranāśaṃ na karoti... \L
%tasya         karaṇāt kālaḥ śarīranāśaṃ na karoti// \N1
%tasya         karaṇāt kālaḥ śarīranāśaṃ na karoti// \D1
%tasya         karaṇāt kālaḥ śarīranāśaṃ    karoti/ \N2
%gatasya dhyānakaraṇāt kālaḥ śarīranāśaṃ na karoti 8 \U1
%tasya         karaṇāt kālaśarīranāśanaṃ    karoti// \U2
%------------------------------
%Because of the execution of this time does not destroy the body. 
%------------------------------
\app{\lem[wit={ceteri}]{tasya}
  \rdg[wit={U1}]{gatasya}}
\app{\lem[wit={ceteri}]{kāraṇāt}
  \rdg[wit={U1}]{dhyānakaraṇāt}}
\app{\lem[wit={ceteri}]{kālaḥ}
  \rdg[wit={U1}]{kāla°}}
śarīranāśaṃ
\app{\lem[wit={ceteri}]{na}
  \rdg[wit={N2,U2}]{\om}}
karoti/
    \end{prose}
  \end{ekdosis}
 %%%%%%%%%%%%%%%%%%%%%%%%%%%%%
 %%%%%%%%%%%%%%%%%%%%%%%%%%%%%
 %%%%%%%%%%%%%%%%%%%%%%%%%%%%%
 %%%%%%%%%%%%%%%%%%%%%%%%%%%%%
 %%%%%%%%%%%%%%%%%%%%%%%%%%%%%
  \begin{ekdosis}
    \ekddiv{type=ed}
    \begin{prose}
%------------------------------
%idānīṃ tasyabhedaḥ    kathyate/   \E
%idānīṃ svabhāvabhedaḥ kathyate    \P
%idānī  svābhāvabhedaḥ kathyate//  \B
%idānīṃ svābhāvabhedaḥ kathyate//  \L
%idānīṃ svabhāvabhedaṃ kathyate//  \N1
%idānīṃ svabhāvabhedaṃ kathyate//  \D1
%idānīṃ svabhāvabheda  kathyate//  \N2
%idānīṃ svabhāvabhedāḥ kathyate    \U1
%idānīṃ svabhāvabhedaḥ kathyate//  \U2
%------------------------------
%Now the division of the inherent nature is described. 
%------------------------------  
\app{\lem[wit={ceteri}]{idānīṃ}
  \rdg[wit={B}]{idānī}}
\app{\lem[wit={P,U2}]{svabhāvabhedaḥ}
  \rdg[wit={B,L}]{svā°}
  \rdg[wit={E}]{tasyabhedaḥ}
  \rdg[wit={N1,D1}]{°bhedaṃ}
  \rdg[wit={N2}]{°bheda}
  \rdg[wit={U1}]{°bhedāḥ}}
kathyate/
    \end{prose}
\end{ekdosis}
  %%%%%%%%%%%%
  %%%%%%%%%%%%%
  %%%%%%%%%%%%%%
  %%%%%%%%%%%%%%
  %%%%%%%%%%%%%%
\begin{ekdosis}\ekddiv{type=ed}\begin{prose}
%------------------------------
%yathā vaṭabījam/ vaṭarūpeṇa pariṇataṃ    sat    daśadhā    bhedaṃ svabhāvata eva prāpnoti/  \E %%%[P.27]
%yathā vaṭabījaṃ  vaṭarūpeṇa pariṇāte     sat    dṛśadhā    bhedaṃ svabhāvata eva prāpnoti   \P
%yathā vaṭabījena rūpeṇa     pariṇamate/  śata   daśadhā    bhedaṃ svābhāva   eva prāpnotī// \B
%yathā vaṭabījena rūpeṇa     pariṇamate   śata   daśadhā    bhedaṃ svābhāva   eva prāpnotī// \L
%yathā vaṭabījaṃ  vaṭarūpeṇa pariṇataṃ//  satṛ   daśadhā    bhedaṃ svabhāvata eva prāpnoti/  \N1
%yathā vaṭabījaṃ  vaṭarūpeṇa pariṇataṃ/   sa     daśadhā    bhedaṃ svabhāvata eva prāpnoti// \D1
%yathā vathabījaṃ vaṭarūpeṇa pariṇataṃ/   sa tu  daśadhā    bhedaṃ svabhāvata eva prāpnoti/  \N2
%yathā vaṭabījaṃ  vaṭarūpeṇa pariṇataṃ    sa tat daśadhā    bhedaṃ svabhāvata eva prāpnotī   \U1
%yathā vaṭabīja---vaṭarūpeṇa pariṇamate// sa     dasat                            prāpnoti// \U2
%------------------------------
%Wie der Samen des Banyan-Baumes zur Gestalt des Banyan-Baumes heranreift und er sich aufgrund seiner eigenen inhärenten Natur so eine zehnfachen Auftheilung erreicht. [Nämlich]: 
%------------------------------
%Just as the seed of the banyan tree ripens into the shape of the banyan tree, and by its own inherent nature attains such a tenfold division. [Namely]:
%------------------------------
yathā
      \app{\lem[wit={P,N1,D1,U1}]{vaṭabījaṃ}
        \rdg[wit={E}]{°bījam}
        \rdg[wit={U2}]{°bīja°}
        \rdg[wit={B,L}]{°bījena}
        \rdg[wit={N2}]{vatha°}}
      \app{\lem[wit={ceteri}]{vaṭarūpeṇa}
        \rdg[wit={L,B}]{rūpeṇa}}
      \app{\lem[wit={B,L,U2}]{pariṇamate}
        \rdg[wit={P}]{pariṇāte}
        \rdg[wit={ceteri}]{pariṇataṃ}}
      \app{\lem[wit={U1}]{sa tat}
        \rdg[wit={N2}]{sa tu}
        \rdg[wit={N1}]{satṛ}
        \rdg[wit={E,P}]{sat}
        \rdg[wit={B,L}]{śata}
        \rdg[wit={D1,U2}]{sa}}
      \app{\lem[wit={ceteri}]{daśadhā}
        \rdg[wit={P}]{dṛśadhā}
        \rdg[wit={U2}]{dasat}}
      \app{\lem[wit={ceteri}]{bhedaṃ}
        \rdg[wit={U2}]{\om}}
      \app{\lem[wit={ceteri}]{svabhāvata}
        \rdg[wit={B,L}]{svabhāva}
        \rdg[wit={U2}]{\om}}
      \app{\lem[wit={ceteri}]{eva}
        \rdg[wit={U2}]{\om}}
      \app{\lem[wit={ceteri}]{prāpnoti}
        \rdg[wit={B,L,U1}]{prāpnotī}}/
%      \end{prose}\end{ekdosis}
%%%%%%%%%%%%%%%%%
%%%%%%%%%%%%%%%%%
%%%%%%%%%%%%%%%%
%%%%%%%%%%%%%%%
%%%%%%%%%%%%%%%%
%\begin{ekdosis}
%      \ekddiv{type=ed}
%      \begin{prose}
%------------------------------ %%%%STEMMA POINT!!!!
%mūlāṃkuratvagdaṇḍaśākhākalikāpallavapuṣpaphalasnehā                    iti daśabhedān    prāpnoti// \E
%mūla aṃkuratvakdaṃdaśākhākilpikāpallavā puṣpaphalasneha                iti daśabhedān    prāpnotīti \P  %%%7642.jpg
%mūlaṃ aṃkuratvakdaṃdaśākhākilakālapallavā// vistāroyaṃ svābhāvataḥ     iti daśabhedān    prāpnoti// \B DSCN7160 Z. 4
%mūlaṃ aṃkuratvakdaṃdaśākhākilāpallavā// vistāroyaṃ svābhāvataḥ//       iti daśabhedān    prāpnoti... \L
%mūlāṃ aṃkuratvakdaṃḍaśākhāṃ kalikāpallavapuṣpaphalasneha//             iti bhedo daśadhā prāpnoti// \N1
%mūlāṃkuratvakdaṇdaśākhāṃ kalikāpallavapuṣpaphalasnehaṃ                 iti bhedo daśadhā prāpnoti// \D1
%mūlāṃkuratvakdaṇdaśākhāṃ kalikāpallavapuṣpaphalasneha/                 iti bhedo daśadhā prāpnoti// \N2
%mūlāṃaṃkuratvakdaṇdaśākhākalikāpallavapuṣpaphalasneha                  iti bhedo daśadhā prāpnoti \U1
% \om                                                                                \U2
%------------------------------
%"Wurzel, Spross, Rinde, Ast, Zweig, Knospe, die sich entfaltende Blüte, Blüte, Frucht und Nektar." Die Auftheilung erreicht [diese] zehn Teile. 
%------------------------------
%"Root, shoot, bark, branch, twig, bud, the unfolding flower, flower, fruit and nectar." The division reaches [those] ten parts.
%------------------------------
\app{\lem[wit={E}]{mūlāṃkuratvagdaṇḍaśākhākalikāpallavapuṣpaphalasnehā}
          \rdg[wit={P}]{mūla aṃkuratvakdaṃdaśākhākilpikāpallavā puṣpaphalasneha}
          \rdg[wit={B}]{mūlaṃ aṃkuratvakdaṃdaśākhākilakālapallavā || vistāroyaṃ svābhāvataḥ}
          \rdg[wit={L}]{mūlaṃ aṃkuratvakdaṃdaśākhākilāpallavā || vistāroyaṃ svābhāvataḥ ||}
          \rdg[wit={N1}]{mūlāṃ aṃkuratvakdaṃḍaśākhāṃ kalikāpallavapuṣpaphalasneha ||}
          \rdg[wit={N2}]{mūlāṃkuratvakdaṇdaśākhāṃ kalikāpallavapuṣpaphalasneha|}
          \rdg[wit={D1}]{mūlāṃkuratvakdaṇdaśākhāṃ kalikāpallavapuṣpaphalasnehaṃ}
          \rdg[wit={U1}]{mūlāṃaṃkuratvakdaṇdaśākhākalikāpallavapuṣpaphalasneha}
          \rdg[wit={U2}]{\om}}
        \app{\lem[wit={ceteri}]{iti}
          \rdg[wit={U2}]{\om}}
        \app{\lem[wit={N1,D1,N2,U1}]{bhedo daśadhā}
          \rdg[wit={E,P,L,B}]{daśabhedān}
          \rdg[wit={U2}]{\om}}
        \app{\lem[wit={ceteri}]{prāpnoti}
          \rdg[wit={P}]{prāpnotīti}
          \rdg[wit={U2}]{\om}}/ 
%      \end{prose}
%    \end{ekdosis}
%%%%%%%%%%%%%
%%%%%%%%%%%%% 
%%%%%%%%%%%%%%
%%%%%%%%%%%%%%%
%%%%%%%%%%%%%%%
%      \begin{ekdosis}
%        \ekddiv{type=ed}
%        \begin{prose}
%------------------------------
%yathā nirmalo  nirvikāraḥ niraṃjana   eka  etādṛśa  ātmā svabhāvād eva/ pṛthivyaptejovāyvākāśamanobuddhimāyāvikārarūpabhedān    prāpnoti/ \E
%tathā nirmalaḥ nirvikāraḥ niraṃjanaḥ  eka  etādṛśa  ātmasvabhāvād eva   pṛthvyetetejo vādvyākāśamanobuddhimāyāvikārarūpabhedāt  prāpnoti \P
%tathā nirmalo  nirvikāraḥ niraṃjanaḥ  eka  etādṛśa  ātmasvabhāvād eva   pṛthvyāpatejovādvyākāśamanobuddhimāyāvikārarūpabhedāna  prāpnoti// \B
%tathā nirmalo  nirvikāraḥ niraṃjanaḥ/ eka  etādṛśa  ātmasvabhāvād eva   pṛthvyāpatejovāybākāśamanobuddhimāyāvikārarūpābhedāna   prāpnoti  \L
%tathā nirmalaḥ nirvikāraḥ niraṃjanaḥ  ekaḥ etādṛśaḥ ātmasvabhāvād eva   pṛthvyāpatejovāybākāśamanobuddhimāyāvikārarūpābhedān    prāpnoti/ \N1
%tathā nirmalaḥ nirvikāraḥ niraṃjanaḥ  eka  etādṛśaḥ ātmasvabhāvād eva   pṛthvīpate/ jīvīkāśamanobuddhir māyāvikārarūpabhedāt    prāpnoti \D1
%tathā nirmalaḥ nirvikāraḥ niraṃjanaḥ  ekaḥ etādṛśaḥ ātmasvabhāvād eva   pṛthvīpate/ jīvīkāśamanobuddhir māyāvikārarūpabhedāt    prāpnoti/ \N2
%tathā nirmalaḥ nirvikāraḥ niraṃjanaḥ  ekaḥ etādṛśaḥ ātmascabhāvād eva   pṛthakte jīvāyuvākāśamanobuddhir māyāyāvikārarūpabhedāt prāpnoti \U1 %%%275.jpg
%yathā nirmalaḥ nirvikāraḥ niraṃjanaḥ  eka  etādṛśa  ātmasvabhāvād eva// pṛthvyaptejovāyyākāśa// manobuddhimayāvikārarūpabhedān  prāpnoti/ \U2
%------------------------------
%In dieser Weise erreicht auch das reine, unveränderliche, makellose, eine solche [Auftheilung] eben aufgrund der inhärenten Natur des Selbst. [Nämlich] die Aufteilung "Erde, Wasser, Feuer, Wind, Raum, Geist, Intellektekt, Illusion, Umwandlungen und Gestalt".
%------------------------------
%In this way, the pure, unchanging, unblemished, attains such [division] precisely because of the inherent nature of the self. [Namely] the division "Earth, Water, Fire, Wind, Space, Mind, Intellect, Illusion, Transformations and Form". %------------------------------
          \app{\lem[wit={ceteri}]{tathā}
            \rdg[wit={E,U2}]{yathā}}
          \app{\lem[wit={E,B,L}]{nirmalo}
            \rdg[wit={ceteri}]{nirmalaḥ}}
          nirvikāraḥ
          \app{\lem[wit={E}]{niraṃjana}
            \rdg[wit={ceteri}]{niraṃjanaḥ}}
          \app{\lem[wit={ceteri}]{eka}
            \rdg[wit={N1,N2,U1}]{ekaḥ}}
          \app{\lem[wit={E}]{etādṛśa}
            \rdg[wit={N1,N2,U1}]{etādṛśaḥ}}
          \app{\lem[wit={ceteri}]{ātmasvabhāvād}
            \rdg[wit={E}]{ātmā°}}
          eva
          \app{\lem[wit={N1}]{pṛthvyāpatejovāybākāśamanobuddhimāyāvikārarūpābhedān}
            \rdg[wit={E}]{pṛthivyap°}
            \rdg[wit={B,L}]{°bhedāna}
            \rdg[wit={P}]{pṛthvyetetejovādvyākāśa°}
            \rdg[wit={D1,N2}]{pṛthvīpate | jīvīkāśamanobuddhir māyāvikārarūpabhedāt}
            \rdg[wit={U1}]{pṛthakte jīvāyuvākāśamanobuddhir māyāyāvikārarūpabhedāt}
            \rdg[wit={U2}]{pṛthvyaptejovāyyākāśa || manobuddhimayāvikārarūpabhedā}}
          prāpnoti/
        \end{prose}
      \end{ekdosis}
      \ekdpb*{}
      %%%%%%%%%%%%%%%%
      %%%%%%%%%%%%%%
      %%%%%%%%%%%%%%%%
      %%%%%%%%%%%%%%%%
      %%%%%%%%%%%%%%%
  \begin{ekdosis}
    \ekddiv{type=ed}
    \begin{prose}
%------------------------------
%jñānayogaprabhāvād     eka eva  ātmā iti niścayo bhavati// \E
%jñānayogaḥ prabhāvād   eka eka  ātmā iti niścayo bhavati \P
%jñānayogaḥ// prabhāvād eka eka  ātmā iti niścayā bhavatī// \B
%jñānayogaḥ// prabhāvād eka eka  ātmā iti niścayo bhavati// \L
%jñānayogaprabhāvāt     eka eva  ātmā iti niścayo bhavati// \N1
%jñānayogaprabhāvāt     eka eva  ātmā iti niścayo bhavati// \D1
%jñānayogaprabhāvāt     eka eva  ātmā iti niścayo bhavati// \N2
%jñānayogaprabhāvāt tu  eka yeva ātmā iti niścayo bhavati \U1
%jñānayogaprabhāvād     eka eva  ātmā iti niścayo bhavati// \U2
%------------------------------
%Aufgrund der Macht von Jñānayoga entsteht so die Gewissheit "Das Selbst ist wahrlich eins".
%------------------------------
%Because of the power of jñānayoga, there arises the certainty that "The Self is verily one."    
%------------------------------
\app{\lem[wit={E,U2}, alt={jñānayogaprabhāvād}]{jñānayogaprabhāvā\skp{d-}}
  \rdg[wit={N1,D1,N2,U1}]{°bhavāt}
  \rdg[wit={L,B}]{jñānayogaḥ// prabhāvād°}
  \rdg[wit={P}]{jñānayogaḥ prabhāvād}
}\skm{d-}eka
\app{\lem[wit={ceteri}]{eva}
  \rdg[wit={P,B,L}]{eka}
  \rdg[wit={U1}]{yeva}}
ātmā iti niścayo bhavati/
    \end{prose}
  \end{ekdosis}
 %%%%%%%%%%%%%%%%%%%%%%%%%%%%%
 %%%%%%%%%%%%%%%%%%%%%%%%%%%%%
 %%%%%%%%%%%%%%%%%%%%%%%%%%%%%
 %%%%%%%%%%%%%%%%%%%%%%%%%%%%%
 %%%%%%%%%%%%%%%%%%%%%%%%%%%%%
    \begin{ekdosis}
    \ekddiv{type=ed}
    \begin{prose}
%------------------------------
%yathaikaiva   pṛthvī  kvacit komalarūpā                                                   kvacit parimalarūparahitā kvacit suvarṇarūpā   kvacid raupyarūpā    \E %%%p.28 
%yathā ekaika  pṛthvī  kvacit komalarūpā                                                                                                                       \P   
%yathā ekaika  pṛthvī  kvacit komalarūpā// kvacit manohararūpā//  kvacit parimalarūpayuktā// kvacit parimalarohitā// kvacit suvarṇarūpa                        \B
%yathā ekaika  pṛthvī  kvacit komalarūpā   kvacit manohararūpāḥ// kvacit parimalarūpayuktā// kvacit parimalarahitā// kvacit suvarṇarūpā                        \L
%yathā ekaiva  pṛthivī kvacit komalarūpa/  kvacit manoharā/       kvacit parimalarūpāyuktā// kvacit parimalarahitā/  kvacit suvarṇarūpā/  kvacit rūpyarūpā/    \N1
%yathā ekaiva  pṛthivī kvacit komalarūpa   kvacit manoharā//      kvacit parimalarūpāyuktā/  kvacit parimalarohitā   kvacit suvarṇarūpa// kvacit rūpyarūpa//   \D1
%yathā ekaṃ ca pṛthivī kvacit komalarūpa   kvacit manoha?rā       kvacit parimalarūpāyuktaḥ/ kvacit parimalarohitā   kvacit suvarṇarūpā   kvacit rūpyarūpa     \N2
%yathā ekai ca pṛthivī kvacit                                                                                              khavarṇakupā   kvacit rūpyarūpā     \U1
%yathā ekaika  pṛthvī  kvacit komalarūpā// kvacit manohararūpa//  kvacit parimalarūpāyuktā/  kvacit parimalarohitā// kvacit suvarṇarūpā// kvacit rajatarūpā//  \U2
%------------------------------
%Wie irgendein bestimmter Erdboden (\textit{ekaika}) manchmal weich erscheint, manchmal schön erscheint, manchmal mit Wohlgeruch versehen ist, manchmal ohne Wohlgeruch ist, manchmal golden erscheint, manchmal silbern erscheint, ...
%
%As some particular soil (\textit{ekaika}) sometimes appears soft, sometimes appears beautiful, sometimes fragrant, sometimes unscented, sometimes golden, sometimes silver,... 
%------------------------------
\app{\lem[type=emendation, resp=egoscr]{yathaikaikaḥ}
  \rdg[wit={E}]{\korr yathaikaiva}
  \rdg[wit={P,B,L,U2}]{yathā ekaika}
  \rdg[wit={N1,D1}]{yathā ekaiva}
  \rdg[wit={N2}]{yathā ekaṃ ca}
  \rdg[wit={U1}]{yathā ekai ca}}
\app{\lem[wit={E,P,B,L,U2}]{pṛthvī}
  \rdg[wit={ceteri}]{pṛthivī}}
kvacit
komala\app{\lem[wit={E,P,B,L,U2},alt={°rūpā}]{rūpā}
    \rdg[wit={ceteri}]{°rūpa}}\dd{}
\app{\lem[wit={ceteri}]{kvacit}
  \rdg[wit={E,P,U1}]{\om}}
\app{\lem[wit={B}]{manohararūpā}
  \rdg[wit={L}]{°rūpāḥ}
  \rdg[wit={U2}]{°rūpa}
  \rdg[wit={N1,N2,D1}]{manoharā}
  \rdg[wit={E,P,U1}]{\om}}\dd{}
\app{\lem[wit={ceteri}]{kvacit}
  \rdg[wit={E,P,U1}]{\om}}
\app{\lem[wit={ceteri},alt={°parimala}]{parimala}
  \rdg[wit={E,P,U1}]{\om}
}\app{\lem[wit={B,L},alt={°rūpayuktā}]{rūpayuktā}
  \rdg[wit={N1,D1}]{°rūpā°}
  \rdg[wit={N2}]{°rūpāyuktaḥ}
  \rdg[wit={E,U1}]{\om}}\dd{}
\app{\lem[wit={ceteri}]{kvacit}
  \rdg[wit={P,U1}]{\om}}
\app{\lem[wit={ceteri},alt={°parimala}]{parimala}
  \rdg[wit={E}]{°parimalarūpa°}
  \rdg[wit={P,U1}]{\om}
}\app{\lem[wit={E,L,N1},alt={°rahitā}]{rahitā}
  \rdg[wit={B,N2,U2}]{°rohitā}
  \rdg[wit={ceteri}]{\om}}\dd{}
\app{\lem[wit={ceteri}]{kvacit}
  \rdg[wit={P,U1}]{\om}}
\app{\lem[wit={E,L,N2,U2}]{suvarṇarūpā}
  \rdg[wit={B,D1}]{°rūpa}
  \rdg[wit={U1}]{khavarṇakupā}
  \rdg[wit={P}]{\om}}\dd{}
\app{\lem[wit={ceteri}]{kvacit}
  \rdg[wit={P,B,L}]{\om}}
\app{\lem[wit={E}]{raupyarūpā}
  \rdg[wit={N1,U1}]{rūpyarūpā}
  \rdg[wit={D1,N2}]{rūpyarūpa}
  \rdg[wit={U2}]{rajatarūpā}
  \rdg[wit={P,B,L}]{\om}}\dd{}
%------------------------------
%kvacid ratnamayī   kvacic ca śvetā                                kvacidraktā   kvacitpītā    \E %%%p.28 
%                                                                                             \P   
%kvacid ratnamaī//  kvacit śverūpā// kvacitkṛṣṇā//                 kvacidraktā/  kvacitpītā//  \B
%kvacid ratnamaī//  kvacit śvetarūpā kvacitkṛṣṇā//                 kvacidraktā// kvacitpītā//  \L
%kvacid ratnamayī/  kvacit śveta/    kvacitkṛṣṇa??/                kvacidrakta/  kvacitpītā/   \N1
%kvacid ratnamayī// kvacit śvetā//   kvacitkṛṣṇā [S8., Z.7]        kvacidrakta   kvacitpītā//  \D1
%kvacid ratnamayī   kvacit śveta     kvacitkṛṣṇā// [S6. verso]     kvacidrakta   kvacitpītā    \N2
%kvacid ratnamayī   kvacit śveta     kvacitkṛṣṇā                   kvacidrakta   kvacitpītā    \U1
%kvacid ratnamayī// kvacit śvetā//   kvacitkṛṣṇā//                 kvacidraktā// kvacitpītā//  \U2
%------------------------------
% ... manchmal aus Edelstein gemacht ist, manchmal weiß erscheint, manchmal schwarz, manchmal kupfern, manchmal gelb,
%
%... is sometimes made of precious stone, sometimes appearing white, sometimes black, sometimes copper, sometimes yellow, 
%------------------------------
\app{\lem[wit={ceteri},alt={°kvacid}]{kvaci\skp{d-ra}}
  \rdg[wit={P}]{\om}
}\app{\lem[wit={ceteri},alt={ratnamayī}]{\skm{d-ra}tnamayī}
  \rdg[wit={B,L}]{°maī}
  \rdg[wit={P}]{\om}}\dd{}
\app{\lem[wit={ceteri}]{kvacit}
  \rdg[wit={E}]{kvacic ca}
  \rdg[wit={P}]{\om}}
\app{\lem[wit={E,D1,U2}]{śvetā}
  \rdg[wit={N1,N2,U1}]{śveta}
  \rdg[wit={L}]{śvetarūpā}
  \rdg[wit={B}]{śverūpā}
  \rdg[wit={P}]{\om}}\dd{}
\app{\lem[wit={ceteri}]{kvacit kṛṣṇā}
  \rdg[wit={N1}]{kṛṣṇa}
  \rdg[wit={E,P}]{\om}}\dd{}
\app{\lem[wit={ceteri},alt={°kvacid}]{kvaci\skp{d-ra}}
  \rdg[wit={P}]{\om}
}\app{\lem[wit={E,B,L,U2},alt={raktā}]{\skm{d-ra}ktā}
  \rdg[wit={ceteri}]{°rakta}}\dd{}
kvacit pītā\dd{}
%------------------------------
%kvacitkarburā   kvacin nānāvidharūpā        kvacid viṣarūpā    kvacid amṛtarūpamayī svabhāvata eva bhavati//  \E  %%%p.28
%                                                               kvacid amṛtamayī     svabhāvata eva bhavati    \P  %%%rest is \om
%kvacitkarburā// kvacin nānāvidhaphalarūpā   kvacit viṣarūpā//  kvacid amṛtamaī/     svabhāvata eva bhavataḥ// \B
%kvacitkarburā// kvacin nānāvidhāphalarūpā   kvacit viṣarūpā//  kvacid amṛtamaī//    svabhāvata eva bhavataḥ// \L
%kvacitkarburā,  kvacin nānāvidhaphalarūpā/  kvacid puṣparūpā,  kvacid amṛtamayī     svabhāvata eva bhavati/   \N1
%kvacitkarburā   kvacin nānāvidhaphalarūpā// kvacid puṣparūpā// kvacid amṛtamayī/    svabhāvata eva bhavati//  \D1
%kvacitkarburā   kvacin nānāvidhaphalarūpā                      kvacid amṛtamayī/    svabhāvata eva bhavati//  \N2
%kvacitkarpurā   kvacin nānāvidhophalarūpā   kvacid ....[rest omitted]                                         \U1
%kvacitkarburā// kvacit nānāvidhaphalarūpā// kvacir viśarūpā//  kvacit amṛtamayī//   svabhāvata eva bhavati//  \U2
%------------------------------
%machmal gesprenkelt, machmal wie verschiedenartige Frucht erscheint, manchmal wie Blumen erscheint, machmal wie der Nektar der Unsterblichkeit erscheint, [und das nur] nur aufgrund seiner inhärenten Natur.
%
%sometimes mottled, sometimes appearing like various fruit, sometimes appearing like flowers, sometimes appearing like the nectar of immortality, [and that only] only because of its inherent nature. 
%------------------------------
kvavit
\app{\lem[wit={ceteri}]{karburā}
  \rdg[wit={U1}]{karpurā}}\dd{}
\app{\lem[wit={ceteri}]{kvaci\skp{n-nā}}
  \rdg[wit={U2}]{kvacit}
  \rdg[wit={P}]{\om}
}\app{\lem[wit={ceteri}]{\skm{n-nā}nāvidhaphalarūpā}
  \rdg[wit={E}]{°vidharūpā}
  \rdg[wit={P}]{\om}}\dd{}
\app{\lem[wit={ceteri},alt={kvacid}]{kvaci\skp{d-pu}}
  \rdg[wit={B,L}]{kvacit}
  \rdg[wit={U2}]{kvacir}
  \rdg[wit={P,N2}]{\om}
}\app{\lem[wit={N1,D1},alt={puṣparūpā}]{\skm{d-pu}ṣparūpā}
\rdg[wit={E,B,L}]{viṣarūpā}
\rdg[wit={U2}]{vśarūpā}
\rdg[wit={U1}]{\om}}\dd{}
\app{\lem[wit={ceteri}]{kvaci\skp{d-a}}
  \rdg[wit={U2}]{kvacit}
  \rdg[wit={U1}]{\om}
}\app{\lem[wit={ceteri}]{\skm{d-a}mṛtamayī}
  \rdg[wit={E}]{amṛtarūpamayī}
  \rdg[wit={B,L}]{°maī}
  \rdg[wit={U1}]{\om}}\dd{}
\app{\lem[wit={ceteri}]{svabhāvata}
  \rdg[wit={U1}]{\om}}
\app{\lem[wit={ceteri}]{eva}
  \rdg[wit={U1}]{\om}}
\app{\lem[wit={ceteri}]{bhavati}
  \rdg[wit={B,L}]{bhavataḥ}
  \rdg[wit={U1}]{\om}}\dd{}
%     \end{prose}
%  \end{ekdosis}
%%%%%%%%%%%%%%
%%%%%%%%%%%%%%%
%%%%%%%%%%%%%%
%%%%%%%%%%%%%%
%%%%%%%%%%%%%%%%
%\begin{ekdosis}
%\ekddiv{type=ed}
%\begin{prose}
%------------------------------
%tathaivātmā   manuṣyapakṣihariṇahastividyādharagandharvakinnaramahāpaṃḍitamahāmūrkharogyarogikrodhiśāṃtarūpaḥ           svabhāvād eva bhavati/ \E
%tathaivātmā   manuṣyapakṣihariṇāhastividyādharagaṃdharvakinnaramahāṃpiṃḍitamahārmūkharogī    krodhiśāṃtarūpāḥ           svabhāvād eva bhavati \P
%tathaivātmā// manuṣyapakṣihariṇahastividyādharagaṃdharvakinnaramahāpiṃḍatamahāmūrkharogī     krodhadhiśāṃtarūpaḥ        svabhāvād eva bhavatī/ \B
%tathaivātmā   manuṣyapakṣihariṇahastividyādharagaṃdharvakinnaramahāpaṃḍitamahāmūrkharogī     krodhadhīśāṃtarūpāḥ        svabhāvād eva bhavatī/ \L
%tathātmā//    manuṣya,pakṣi,hariṇa,hastī,vidyādhara,gandharvakiṃnara/mahāpaṃḍitamahāmūrva,rogī, arogī/krodhī,śāntarūpa,svabhāvād eva bhati/ \N1 %%%%%%%CRAZY SWITCH BETWEEN DAṆḌA AND COMMA
%tathātmā//    manuṣyapakṣi// hariṇahastīvidyādharagandharvakinnaramahāpaṃḍitamahāmūrvarogī arogīkrodhīśāṃtarūpa---------svabhāvād eva bhavati/ \D1
%tathātmā//    manuṣyapakṣihariṇahastividyādharagandharvakinnaramahāpaṇḍitamahāmūrkharogīarogīkrodhīśāṃtarūpa------------svabhāvād eva bhavati/ \N2
%                                     vidyādharagaṃdharvakinnaramahāpaṇḍitamahāmūrṣarogīarogīkrodhīśāṃtarūpa        evaṃ svabhāvaṃ dharati  \U1
%tathaivātmā   manuṣyapakṣihariṇahastividyādharagaṃdharvakinnaramahāpaṃḍitamahāmūrkharogī arogī krodhiśāṃtarūpaḥ          svabhāvād eva bhavati// \U2 %%%410.jpg
%------------------------------
%Auf diese Weise nimmt auch das Selbst aufgrund seiner inhärenten Natur die Form eines Menschen, Vogels, einer Gazelle, eines Elefants, eines Vidyādharas, eines Gandharvas, Zentauren, eines großen Gelehrten oder großen Dummkopfes, eines Kranken oder Gesunden, eines Zornigen oder Friedlichen an. 
%------------------------------
%------------------------------
%tathaivātmā   manuṣyapakṣihariṇahastividyādharagandharvakinnaramahāpaṃḍitamahāmūrkha  rogyarogikrodhi---śāṃtarūpaḥ      svabhāvād eva bhavati/ \E
%tathaivātmā   manuṣyapakṣihariṇāhastividyādharagaṃdharvakinnaramahāpiṃḍitamahārmūkha  rogī-----krodhi---śāṃtarūpāḥ      svabhāvād eva bhavati \P
%tathaivātmā// manuṣyapakṣihariṇahastividyādharagaṃdharvakinnaramahāpiṃḍatamahāmūrkha  rogī-----krodhadhiśāṃtarūpaḥ      svabhāvād eva bhavatī/ \B
%tathaivātmā   manuṣyapakṣihariṇahastividyādharagaṃdharvakinnaramahāpaṃḍitamahāmūrkha  rogī-----krodhadhīśāṃtarūpāḥ      svabhāvād eva bhavatī/ \L
%tathātmā//    manuṣyapakṣihariṇahastīvidyādharagandharvakiṃnaramahāpaṃḍitamahāmūrva   rogīarogīkrodhī---śāntarūpa-------svabhāvād eva bhati/ \N1 %%%%%%%CRAZY SWITCH BETWEEN DAṆḌA AND COMMA
%tathātmā//    manuṣyapakṣihariṇahastīvidyādharagandharvakinnaramahāpaṃḍitamahāmūrva   rogīarogīkrodhī---śāṃtarūpa-------svabhāvād eva bhavati/ \D1
%tathātmā//    manuṣyapakṣihariṇahastividyādharagandharvakinnaramahāpaṇḍitamahāmūrkha  rogīarogīkrodhī---śāṃtarūpa-------svabhāvād eva bhavati/ \N2
%                                     vidyādharagaṃdharvakinnaramahāpaṇḍitamahāmūrṣa   rogīarogīkrodhī---śāṃtarūpa       evaṃ svabhāvaṃ dharati  \U1
%tathaivātmā   manuṣyapakṣihariṇahastividyādharagaṃdharvakinnaramahāpaṃḍitamahāmūrkha  rogīarogīkrodhi---śāṃtarūpaḥ      svabhāvād eva bhavati// \U2 %%%410.jpg
%------------------------------
%Auf diese Weise nimmt auch das Selbst aufgrund seiner inhärenten Natur die Form eines Menschen, Vogels, einer Gazelle, eines Elefants, eines Vidyādharas, eines Gandharvas, Zentauren, eines großen Gelehrten oder großen Dummkopfes, eines Kranken oder Gesunden, eines Zornigen oder Friedlichen an.
%
%In this way, the self also takes the form of a human, a bird, a gazelle, an elephant, a vidyādhara, a gandharva, a centaur, great scholar or a great fool, a sick or healthy, an angry or or peaceful person, by virtue of its inherent nature.       
%------------------------------      
\app{\lem[wit={E,P,B,L,U2}]{tathaivātmā}
  \rdg[wit={ceteri}]{tathātmā}}
\app{\lem[wit={ceteri},alt={manuṣya°}]{manuṣya}
  \rdg[wit={U1}]{\om}
}\app{\lem[wit={ceteri},alt={°pakṣi°}]{pakṣi}
  \rdg[wit={U1}]{\om}
}\app{\lem[wit={ceteri},alt={°hariṇa°}]{hariṇa}
  \rdg[wit={P}]{°hariṇā°}
  \rdg[wit={U1}]{\om}
}\app{\lem[wit={N1,D1},alt={°hastī°}]{hastī}
  \rdg[wit={ceteri}]{hasti}
  \rdg[wit={U1}]{\om}
}vidyādharagaṃdharvakinnaramahā\app{\lem[wit={ceteri},alt={°paṇḍita°}]{paṇḍita}
  \rdg[wit={B}]{piṃḍata}
}mahā\app{\lem[wit={ceteri},alt={°mūrkha°}]{mūrkha}
  \rdg[wit={P}]{°rmūkha°}
  \rdg[wit={N1,D1}]{°mūrva°}
  \rdg[wit={U1}]{°mūrṣa°}
}\app{\lem[type=emendation, resp=egoscr]{rogyarogī}
  \rdg[wit={E}]{\korr °rogyarogi}
  \rdg[wit={N1,N2,D1,U1,U2}]{°rogī arogī}
  \rdg[wit={P,B,L}]{°rogī}
}\app{\lem[wit={ceteri},alt={°krodhī°}]{krodhī}
  \rdg[wit={E,P}]{°krodhi°}
  \rdg[wit={B,L}]{°krodha°}
}\app{\lem[wit={ceteri},alt={°śānta°}]{śānta}
  \rdg[wit={B,L}]{°dhiśānta°}
}\app{\lem[wit={ceteri},alt={°rūpaḥ}]{rūpaḥ}
  \rdg[wit={P,L}]{°rūpāḥ}
  \rdg[wit={N1,N2,D1,U1}]{°rūpa}}
\app{\lem[wit={ceteri},alt={svabhāvād eva}]{svabhāvād-eva}
  \rdg[wit={U1}]{evaṃ svabhāvaṃ}}
\app{\lem[wit={ceteri}]{bhavati}
  \rdg[wit={B,L}]{bhavatī}
  \rdg[wit={N1}]{bhati}
  \rdg[wit={D1}]{dharati}}\dd{}
%\end{prose}
%\end{ekdosis}
%%%%%%%%%%%%
%%%%%%%%%%%%
%%%%%%%%%%%%
%%%%%%%%%%%%%
%%%%%%%%%%%%%
% \begin{ekdosis}
%    \ekddiv{type=ed}
%    \begin{prose}
%------------------------------      
%jñānayogādhikāra rūparahito  jñāyate/  yathā plakṣasyotpattiḥ/ sthānam eva bhavati// \E
%jñānayogādhikāra rūparahito  jñāyate   yathā phalasyotpattisthānam ekam eva bhavati \P  %%%7643.jpg                          
%jñānayogādhikāra rūparahito  jñāyate// yathā phalasyotpattisthānam ekam eva bhavatī// \B
%jñānayogādhikāra rūparahito  jñāyate// yathā phalasyotpattisthānam ekam eva bhavati// \L
%jñānayogād vikāra rūparahito jñāyate/  yathā phalasyotpattisthānam ekam eva bhavati/ \N1
%jñānayogādhikāra rūparahito  jñāyate// yathā phalasyotpattisthānam ekaseva  bhavati// \D1
%jñānayogadhikāra rūparahito  jñāyate// yathā phalasyotpattisthānam eva kameva bhavati// \N2
%jñānayogāt vikāra rūparahito jñāyate   yathā phalasyotpattisthāna  ekam eva ti \U1
%jñānayogādhikāra rūparahito  jāyate//  yathā phalasyotpattisthānam ekam eva bhavati// \U2
%------------------------------
%em. zu jñānayogādhikāriṇā? em. zu vikāraṃ rūparahitaṃ -> By the man of Jñānayoga the modifications are known as formless.?! Just as the place of origin of the fruit is only one.
%Aufgrund von Jñānayoga wird die Umwandlung als Formlos erkannt.
%Because of jñānayoga, transformation is recognized as formless. 
%------------------------------
\app{\lem[wit={N1,U1}]{jñānayogād-vikāra}
  \rdg[wit={ceteri}]{jñānayogadhikāra}}
rūparahito
\app{\lem[wit={ceteri}]{jñāyate}
  \rdg[wit={U2}]{jāyate}}\dd{}
yathā
\app{\lem[wit={ceteri}]{phalasyotpatti}
  \rdg[wit={E}]{plakṣasyotpattiḥ}
}\app{\lem[wit={ceteri},alt={°sthānam}]{sthāna\skp{m-e}}
  \rdg[wit={E}]{sthānam}
  \rdg[wit={U1}]{°sthāna}
}\app{\lem[wit={ceteri},alt={ekam}]{\skm{m-e}ka\skp{m-e}}
  \rdg[wit={D1}]{ekas}
  \rdg[wit={N2}]{eva}
  \rdg[wit={E}]{\om}
}\app{\lem[wit={ceteri},alt={eva}]{\skm{m-e}va}
  \rdg[wit={N2}]{kam eva}}
\app{\lem[wit={ceteri}]{bhavati}
  \rdg[wit={B}]{bhavatī}
  \rdg[wit={U1}]{ti}}/
%------------------------------
%atha ca phalasya gatir bahudhā dṛśyate/ \E
%atha ca phalasya gati  bahudhā dṛśyate    \P
%atha ca phalasya gatir bahudhā dṛśyate// \B
%atha ca phalasya gatir bahudhā dṛśyate// \L
%atha ca phalasya gatir bahudhā dṛśyate/ \N1
%atha ca phalasya gatir bahudhā dṛśyate// \D1
%atha ca phalasya gati  bahudhā dṛśyate/ \N2
%atra ca phalasya gati  bahudhā dṛśyate \U1
%atha ca phalasya gatir bahudhā dṛśyate// \U2
%------------------------------
%But the path of the fruit is seen manifold. 
%------------------------------
atha ca phalasya \app{\lem[wit={ceteri},alt={gatir}]{gati\skp{r-ba}}
  \rdg[wit={P,N2,U1}]{gati}
}\skm{r-ba}hudhā dṛśyate\dd{}
\end{prose}
  \end{ekdosis}
%%%%%%%%%%%%%%%%%
%%%%%%%%%%%%%%%%%
%%%%%%%%%%%%%%%%%%
%%%%%%%%%%%%%%%%%%
%%%%%%%%%%%%%%%%%%%
\begin{ekdosis}
    \ekddiv{type=ed}
    \begin{prose}
%------------------------------ %%%STEMMAPOINT śuklaṃ//śuṣkaṃ
% ekaṃ phalaṃ pṛthvīmadhye  patati/  śuklaṃ bhavati/   \E
% ekaṃ phalaṃ pṛthvīmadhye  patati   śuklaṃ bhavati    \P
% ekaṃ phalaṃ pṛthvīmadhye  patiśuklaṃ      bhavatī//  \B
% ekaṃ phalaṃ pṛthvīmadhye  patati   śuṣkaṃ bhavatī    \L
% ekaṃ phala--pṛthvīmadhye  patati/  śuklaṃ bhavati/   \N1 %%%p.7 recto letzte Zeile 
% ekaṃ phala--pṛthvīmadhye  patati// śuklaṃ bhavati//  \D1
% eva  phala--pṛthvīmadhye  patati   śuklaṃ bhavati//  \N2
% ekaṃ phalaṃ pṛthivīmadhye patati   śuṣkaṃ bhavati    \U1
% ekaphalaṃ   pṛthvīmadhye  patati// śuṣkaṃ bhavati//  \U2
%------------------------------
%One fruit falls onto the ground. It is getting bright.  Dürre entsteht// Licht entsteht// Es wird hell!!!. 
%------------------------------
\app{\lem[wit={ceteri}]{ekaṃ}
  \rdg[wit={U2}]{eka°}
  \rdg[wit={N2}]{eva}}
\app{\lem[wit={ceteri}]{phalaṃ}
  \rdg[wit={N1,N2,D1}]{phala°}}
\app{\lem[wit={ceteri},alt={pṛthvī°}]{pṛthvī}
  \rdg[wit={U1}]{pṛthivī°}
}madhye patati/
\app{\lem[wit={ceteri}]{śuklaṃ}
  \rdg[wit={L,U1,U2}]{śuṣkaṃ}}
\app{\lem[wit={ceteri}]{bhavati}
  \rdg[wit={B}]{bhavatī}}/
 %   \end{prose}
 % \end{ekdosis}
%%%%%%%%%%%%%%%
%%%%%%%%%%%%%%
%%%%%%%%%%%%%
%%%%%%%%%%%%%%%
%%%%%%%%%%%%%%
%  \begin{ekdosis}
%    \ekddiv{type=ed}
%    \begin{prose}
%------------------------------
% ekasya phalasya makaraṃdaṃ bhramaraḥ  pibati/  \E
% ekasya phalasya makaraṃdaṃ bhramaraḥ  pibaṃti  \P
% ekasya            karaṃdaṃ bhramaraṃ  pibatī/  \B
% ekasya          makaraṃdaṃ bhramaraṃ  pibati   \L
% ekasya phalasya makaraṃdabhramaraḥ    pibati/  \N1 %%%p.7 recto letzte Zeile 
% ekasya phalasya makaraṃdabhramaraḥ    pibati/  \D1
% ekasya phalasya makaraṃdaṃ bhramara   pibati/  \N2
% ekasya phalasya makaraṃdaṃ bhramanaḥ  pibati   \U1
% ekasya phalasya makaraṃdaṃ bhramaraḥ  pibati// \U2
% ------------------------------
% Eine Biene trinkt den Blumensaft der einen Frucht.
% A bee drinks the flower juice of a fruit.     
%------------------------------
ekasya
\app{\lem[wit={ceteri}]{phalasya}
  \rdg[wit={P,L}]{\om}} 
\app{\lem[wit={E,P,L,N2,U1,U2}]{makaraṃdaṃ}
  \rdg[wit={L,N1}]{makaraṃda°}
  \rdg[wit={B}]{karaṃdaṃ}}
\app{\lem[wit={ceteri}]{bhramaraḥ}
  \rdg[wit={B,L}]{bhramaraṃ}
  \rdg[wit={N2}]{bhramara}}
\app{\lem[wit={ceteri}]{pibati}
  \rdg[wit={P}]{pibaṃti}
  \rdg[wit={B}]{pibatī}}/
%------------------------------
% ekasya phalasya  mālāṃ kāminī tuṃgakucamaṃḍalopari dadhāti/ \E
% ekasya phalasya  mālāṃ kāminī tuṃgakucamaṃḍalopari dadhāti \P
% ekasya phalasya  mālāṃ kāminī tuṃgakucamaṃḍalopari dadhātī// \B
% ekasya phalasya  mālāṃ kāminī tuṃgakucamaṃḍalopari dadhāti// \L
% ekasya phalasya  mālāṃ kāminī tuṃgakucamaṃḍalopari dadhāvati/ \N1 %%%p.7 recto letzte Zeile 
% ekasya phalasya  mālāṃ kāmibī tuṃgakucamaṇḍalopari dadhāti// \D1
% ekasya phalasyaṃ mālākāminī   tuṃgakucamaṇḍalopari dadhovati// \N2
% ekasya phalasya  mālāṃ kāmini tuṃ  kucamaṃḍalopari dadhāti \U1
% ekasya phalasya  mālāṃ kāminī tuṃgakucamaṃḍalopari dadhāti// \U2
%------------------------------
% of the one fruit Blütenkranz/Girlande die Verliebte (biene) führt ein unmittelbar über dem Kreis des Blütenstempels der wie eine Brust ist ein.  %tu.mga = hervorstehend 
%Die Verliebte (Biene) platziert sich auf dem Blütenkranz über dem emportstehenden Kreisförmigen Blütenstempel.    
%------------------------------
ekasya
\app{\lem[wit={ceteri}]{phalasya}
  \rdg[wit={N2}]{phalasyaṃ}}
\app{\lem[wit={ceteri}]{mālāṃ}
  \rdg[wit={N2}]{mālā°}}
\app{\lem[wit={ceteri}]{kāminī}
  \rdg[wit={D1}]{kāmibī}}
\app{\lem[wit={ceteri},alt={tuṅga°}]{tuṅga}
  \rdg[wit={U1}]{tuṃ°}
}kucamaṃḍalopari
\app{\lem[wit={ceteri}]{dadhāti}
  \rdg[wit={N1}]{dadhāvati}
  \rdg[wit={N2}]{dadhovati}}/
%------------------------------ 
%ekaṃ phalaṃ mṛtamanuṣyopari   kṣipyate/  ayaṃ vastunaḥ svabhāvaḥ/  tathā eka evātmā   svīyabhāvād evāṣṭau    bhogān  bhunakti/ \E
%ekaṃ phalaṃ mṛtamanuṣyopari   kṣipyate   ayaṃ vastunaḥ svabhāvaḥ   tathā eka evātmā   svīyabhāvād evāṣṭau    bhogān  bhunakti \P
%ekaṃ phalaṃ mṛtamanuṣyopari   kṣapyate// ayaṃ vastunaḥ svabhāvaḥ/  tathā eka evātmā   svabhāvād   evāṣṭau    bhogān  bhunakte// \B
%ekaṃ phalaṃ mṛtam anuṣyopari  kṣipyate// ayaṃ vastunaḥ svabhāvaḥ   tathā eka evātmā   svabhāvād   evāṣṭau    bhogān  bhunakte// \L
%ekaphalaṃ   mṛtamanuṣyopari   kṣipyate// ayaṃ vastunaḥ svabhāvaḥ/  tathā eka evātmā   svīyabhāvād evāṣṭau  bhogānā bhunakti/ \N1
%ekaphalaṃ   mṛtam anuṣyopari  kṣipyate// ayaṃ vastunaḥ svabhāvaḥ// tathā eka evātmā   svīyabhāvād evāṣṭau  bhogān  bhunakti// \D1
%ekaphalaṃ   mṛtamanuṣyopari   kṣipyate/  ayaṃ castunaḥ svabhāvaḥ/  tathā ekaevātmā    svīyabhāvād evāstau  bhogāt  bhunakti/ \N2
%ekaphalaṃ   mṛtamanuṣyopari   kṣipyate/  ayaṃ castunaḥ svabhāvaḥ/  tathā ekaevātmā    svīyabhāvād evāstau bhogāt  bhunakti/ \U1 %%%276.jpg
%ekaṃ phalaṃ mṛtamanuṣyopari   kṣipyate// ayaṃ vastunaḥ svabhāvaḥ// tathā ekameva ātmā svīyabhāvād evāṣṭabhogān    bhunakti// \U2
%------------------------------
%Die eine Frucht schleudert den Nektar über die Blüte. (em zu anu.s.a = Blüthe?) Dies ist die inhärente Natur der Sache. So genießt auch das eine Selbst genießt aufgrund des eigenen Seins die acht Genüsse. 
%------------------------------
\app{\lem[type=emendation, resp=egoscr,alt={ekaṃ phalam}]{ekaṃ phala\skp{m-a}}
  \rdg[wit={E,P,B,L}]{\korr ekaṃ phalaṃ}
  \rdg[wit={N1,N2,D1,U1}]{eka°}}
\app{\lem[type=emendation, resp=egoscr]{\skm{m-a}mṛta\skp{m-a}}
  \rdg[wit={ceteri}]{\korr mṛta°}}
\app{\lem[type=emendation, resp=egoscr]{anuṣṇopari}
  \rdg[wit={}]{\korr manuṣyopari}}
\app{\lem[wit={ceteri}]{kṣipyate}
  \rdg[wit={B}]{kṣapyate}}/
%\end{prose}
%\end{ekdosis}
%------------------------------
%ke te ṣṭau  bhogāḥ – suvāsaś ca   suvastrañ ca  suśayyā    sunitaṃbinī/       susthānañ cānnapānāni    aṣṭau bhogāś ca dhīmatām/       padṛsūtramayāni vasrāṇi// \E
%ke te ṣṭau  bhogauḥ  suvāsaś ca   suvāsaś   ca  suyyā      sunitāṃbinīḥ//     susthānāś cānpanānp------aṣṭau bhogāś ca dhīmatāṃ 1      padasūtramayāni vastrāṇi?? \P %%%7643.jpg
%      aṣṭau bhogāḥ   suvāsac ca   suvasaś   ca  suśayyāḥ   sūnitaṃbinī/       susthānaś vānnapānāny----aṣṭau bhogāś cā sudhīmatām//1// paṭasūtrāmayāni vasrāṇi// \B
%      aṣṭau bhogāḥ   suvāsaś ca   suvāsaś   ca  suśayyāḥ   sūnitaṃbinī//      susthānāś cānnapānāny----aṣṭau bhogāś cā sudhīmatāṃ//1// paṭasūtrāmayāni vastrāṇi// \L
%ke te ṣṭau  bhogāḥ// suvāyaś ca/                suśayyā    sunitaṃbinī/       susthātāś cātmapanasyā----ṣṭau bhogāḥ    sudhipaṇa----------padṛsūtrayāni   vasrāṇi \N1
%ke te ṣṭau  bhogāḥ// suvāyaś ca//               suśayyā    sunittaṃbinī//     susthātāś cānmanasyā------ṣṭau bhogāḥ    sudhiṣaṇa----------padṛsūtrayāni   vasrāṇi   \D1
%ke te ṣṭau  bhogāḥ   suvāyaś ca                 suśayya    sunitaṃbinī/       susthānāś cānmanasyā------ṣṭau bhogāḥ    sudhiyane          padṛsūtrayāni   vasrāṇi \N2
%ke te ṣṭe   bhogā –  suvāsaś ca                 suśayyā ca sunītavinīta       susthātāś cānnapānaḥ syādaṣṭau bhogāḥ    sudhiṣaṇāṃ         padṛsūtramayāni vasrāṇi \U1
%ke te aṣṭau bhogā // suvāsaś ca// suvaṃśaś ca// suśayyā//  sunitaṃbinī//      sudehaṃ// sukhasaṃtānaṃ// abhayādicāṣṭakaṃ//              paṭasūtramayāni vasrāṇi \U2
%------------------------------
%What are the eight enjoyments? A beautiful dwelling, good clothing, a good bed, a well-trained horse?, a nice place, food & drink. Tose are the eight enjoyments of the wise. Clothes made from silk.
%------------------------------
%\begin{ekdosis}
%  \ekddiv{type=ed}
%  \begin{prose}
\app{\lem[wit={ceteri}]{ke te}
  \rdg[wit={L,B}]{\om}}
\app{\lem[wit={ceteri}]{'ṣṭau}
  \rdg[wit={L,B}]{aṣṭau}
  \rdg[wit={U1}]{ṣṭe}}
\app{\lem[wit={ceteri}]{bhogāḥ}
  \rdg[wit={P}]{bhobauḥ}
  \rdg[wit={U1,U2}]{bhogā}}
\end{prose}
\end{ekdosis}
\ekdpb*{}
\begin{ekdosis}
  \ekddiv{type=ed}
  \begin{tlg}
\tl{\app{\lem[wit={ceteri},alt={suvāsaś ca}]{suvāsaś-ca}
  \rdg[wit={B}]{suvāsac ca}}
\app{\lem[wit={E},alt={suvastrañ ca}]{suvastrañ-ca}
  \rdg[wit={U2}]{suvaṃśaś ca}}
\app{\lem[wit={ceteri}]{suśayyā}
  \rdg[wit={U1}]{suśayyā ca}
  \rdg[wit={L,B}]{suśayyāḥ}
  \rdg[wit={P}]{suyyā}}
\app{\lem[wit={ceteri}]{sunitaṃbinī}
  \rdg[wit={P}]{sunitāṃbinīḥ}
  \rdg[wit={U1}]{sunītavinīta}}/}\\
\tl{\app{\lem[wit={E},alt={susthānañ}]{susthāna\skp{ñ-cā}}
  \rdg[wit={P,L,N2}]{susthānāś}
  \rdg[wit={N1,D1,U1}]{susthātāś}
  \rdg[wit={U2}]{sudehaṃ}}
\app{\lem[wit={L}]{\skm{ñ-cā}nnapānān\skp{y-a}}
  \rdg[wit={B}]{°vānna°}
  \rdg[wit={E}]{°pānāni}
  \rdg[wit={P}]{cānpanānp°}
  \rdg[wit={N1}]{cātmapanasyā°}
  \rdg[wit={N2,D1}]{cānmanasyā°}
  \rdg[wit={U1}]{cānnapānaḥ syād°}
  \rdg[wit={U2}]{sukhasaṃtānaṃ}}
\app{\lem[wit={E,P},alt={aṣṭau bhogāś ca}]{aṣṭau bhogāś-ca dhīmatām}
  \rdg[wit={B,L}]{aṣṭau bhogāś cā sudhīmatām}
  \rdg[wit={N1}]{ṣṭau bhogāḥ sudhipaṇa°}
  \rdg[wit={D1}]{ṣṭau bhogāḥ sudhiṣaṇa°}
  \rdg[wit={U1}]{aṣṭau bhogāḥ sudhiṣaṇāṃ}
  \rdg[wit={U2}]{abhayādicāṣṭakaṃ}
  \rdg[wit={N1,N2,D1,U1}]{aṣṭau bhogāḥ}
  \rdg[wit={U2}]{abhayādicāṣṭakaṃ}}/dd{}1\hskip-2pt\dd{}}
\end{tlg}
\end{ekdosis}
%------------------------------
%padṛsūtramayāni vasrāṇi//  \E
%padasūtramayāni vastrāṇi?? \P %%%7643.jpg
%paṭasūtrāmayāni vasrāṇi//  \B
%paṭasūtrāmayāni vastrāṇi// \L
%padṛsūtrayāni   vasrāṇi    \N1
%padṛsūtrayāni   vasrāṇi    \D1
%padṛsūtrayāni   vasrāṇi    \N2
%padṛsūtramayāni vasrāṇi    \U1
%paṭasūtramayāni vasrāṇi    \U2
%------------------------------
%Clothes made from silk,...
%------------------------------
%paṃcasaptā dṛālikā         yuktāni harmyāṇi teṣu vāsaḥ    ativipulā  mṛdutarasukhāsuśayyā/     \E
%paṃcasaptā dadhikā         yuktāni harmyāṇi teṣu cāsaḥ 2  ativipulā  mṛduttarachadavatīśayyā 2  \P
%paṃcasatyā dātikā          yuktāni harmyāṇi teṣu vāstu    ativipulā  mṛdutaralāśayyā//2//        \B
%paṃcasatyā dātikā          yuktāni harmyāṇi teṣu vāstu    ativipulā  mṛdutaralāśayyā//3//        \L
%paṃca vā sapta vā dṛālikā  yuktāni harmyāṇi/              ativapulā  mṛdu/uttaracchaṃdavatīśayyā/  \N1
%paṃca vā sapta vā dṛāṃlikā yuktāni harmyāṇi               ativapulā  mṛduuttarachaṃdavatīśayyā/     \D1
%paṃca vā sapta vā tālikā---yuktāni harmyāni               ativipulā  mṛduuttarachaṃdavatīśayyā    \N2
%paṃca vā sapta vā dālikā---yuktāni harmyāṇi               ativipulāṃ mṛduuttarachadavatiśaiyyā     \U1
%--------------------------saudhāni harmyāṇi vāsāya kecit// aṣṭau bhogān āha// sugrahaṃ// suvastraṃ// suśayā sustrī//  \U2
%--------------------------------------------
%,a site of the palace in which there are mainsions endowned with five or seven rooms. A huge, very soft and lovely bed.  
%-------------------------------------------
\begin{ekdosis}
  \ekddiv{type=ed}
  \begin{prose}
\app{\lem[type=emendation, resp=egoscr,alt={paṭṭa°}]{paṭṭa}
  \rdg[wit={E,N1,D1,N2,U1}]{\korr padṛ°}
  \rdg[wit={P}]{pada°}
  \rdg[wit={B,L,U2}]{paṭa°}
}\app{\lem[wit={ceteri},alt={sūtra°}]{sūtra}
  \rdg[wit={B,L}]{sūtrā}
}\app{\lem[wit={ceteri}]{mayāni}
  \rdg[wit={N1,N2,D1}]{yāni}}
\app{\lem[wit={P,L}]{vastrāṇi}
  \rdg[wit={ceteri}]{vasrāṇi}} 1\dd{}
\app{\lem[wit={N1,N2,D1,U1}]{paṃca vā sapta vā}
  \rdg[wit={E,P}]{paṃcasaptā}
  \rdg[wit={L,B}]{paṃcasatyā}}
\app{\lem[type=emendation, resp=egoscr]{śālikā}
  \rdg[wit={E,N1}]{\korr dṛālikā}
  \rdg[wit={D1}]{dṛāṃlikā}
  \rdg[wit={P}]{dadhikā}
  \rdg[wit={B,L}]{dātikā}
  \rdg[wit={N2}]{tālikā}
  \rdg[wit={U1}]{dālikā}
}\app{\lem[wit={ceteri}]{yuktāni}
  \rdg[wit={U2}]{saudhāni}}
harmyāṇi
\app{\lem[wit={L,B}]{teṣu vāstu}
  \rdg[wit={E}]{teṣu vāsaḥ}
  \rdg[wit={P}]{teṣu cāsaḥ}
  \rdg[wit={U2}]{vāsāya kecit}
  \rdg[wit={ceteri}]{\om}} 2\dd{}
\app{\lem[wit={ceteri}]{ativipulā}
  \rdg[wit={N1,D1}]{ativapulā}
  \rdg[wit={U1}]{ativipulāṃ}
  \rdg[wit={U2}]{aṣṭau bhogān āha ||}}
\app{\lem[type=emendation, resp=egoscr,alt={mṛdūttara}]{mṛdūttara}
  \rdg[wit={E,P,L,B}]{\korr mṛdutara°}
  \rdg[wit={N1,N2,D1,U1}]{mṛdu | uttara°}
  \rdg[wit={U2}]{sugrahaṃ ||}
}\app{\lem[wit={N1,N2,D1},alt={°chandavatī°}]{chandavatī}
  \rdg[wit={P}]{°chadavatī°}
  \rdg[wit={U1}]{°chadavati°}
  \rdg[wit={U2}]{suvastraṃ ||}
}\app{\lem[wit={ceteri}]{śayyā}
  \rdg[wit={U2}]{suśayā sustrī}} 3\dd{}
%------------------------------
%padminī tārūṇyavatī  manoharā guṇavatī  tatropaviṣṭā kāṃtā/      \E
%padminī tārūṇyavatī  manoharā guṇavatī  tatopaviṣṭā  kāṃtā 4     \P
%padminī tārūnyavatī  manoharā guṇavatī//tatrāpavistā kāṃtā 4     \B
%padminī tārūnyavatī  manoharā guṇavatī//tatropavistā kāṃtā// 4// \L
%padmanī tārūṇyavatī  manoharā guṇavatī  tatropavistā//           \N1
%padminī tārūrāyavatī manoharā guṇavatī  tatropavistā//           \D1
%padminī tārūnyavatī  manoharā guṇavatī  tatropavistā             \N2
%padminī tārūnyavati  manoharā guṇavati  tatropavistā             \U1
%                                                                 \U2
%--------------------------------------------
%[On which] there is situated [tatropaviṣṭā] a lotus-like [em. zu tāruṇyavatī] youthful, charming and virtuous wife. 
%-------------------------------------------
\app{\lem[wit={ceteri}]{padminī}
  \rdg[wit={N1}]{padmanī}
  \rdg[wit={U2}]{\om}}
\app{\lem[type=emendation, resp=egoscr]{tāruṇyavatī}
  \rdg[wit={ceteri}]{\korr tārūṇyavatī}
  \rdg[wit={N2}]{tārūrāyavatī}
  \rdg[wit={U2}]{\om}}
\app{\lem[wit={ceteri}]{manoharā guṇavatī}
   \rdg[wit={ceteri}]{tatropavistā}
  \rdg[wit={P}]{tato°}
  \rdg[wit={B}]{tatrā°}
  \rdg[wit={U2}]{\om}}
\app{\lem[wit={E,P,B,L}]{kāntā}
  \rdg[wit={ceteri}]{\om}} 4\dd{}
%------------------------------
%sādhu āśanam/      atimūlyañ ca/         manoramam annaṃ।       tathā vidhaṃ pānam/   \E
%sādhu āsanaṃ 5     atimūlo 'śvaḥ 6       manoramam annaṃ    7   tathā vidhaṃ pānaṃ 8  \P
%sādhu āsanaṃ 5     atimūlyo asvaṃ//6     manoramyam attaṃ //7   tathā vidhapānaṃ//8   \B
%sādhu āsanaṃ// 5// atimūlyo aśvaṃ//6//   manoramyam annaṃ //7// tathā vidhapānaṃ//8// \L
%sādhyāsanaṃ//      amūlyo svaś ca//      manoramam attaṃ        tathā vidhaṃ pānaṃ/   \N1
%sādhyāsanaṃ//      amūlyo svaś ca//      manoramam attaṃ        tathā vidhaṃ pānaṃ//  \D1
%sādhyāsanaṃ        amūlyo svaś ca        manotamam annaṃ        tathā vidhapānaṃ//    \N2
%sādhyāsanaṃ        amolyo svaś ca        manoramam annaṃ        tathā vidhaṃ pānaṃ    \U1
%sādhu āsanaṃ//           suśvaḥ//        suṣṭu annaṃ//          tathā vidhayānaṃ//    \U2
%--------------------------------------------
%good throne/seat; atimūlyo (überaus wertvolles) 'śvaṃ (Pferd), manorama ( die Sinne erfreuendes) Essen, verschiedenes Trinken. 
%-------------------------------------------
\app{\lem[type=emendation, resp=egoscr]{sādhvāsanaṃ}
  \rdg[wit={E}]{\korr sādhu āśanam}
  \rdg[wit={P,B,L,U2}]{sādhu āsanaṃ}
  \rdg[wit={N1,N2,D1}]{sādhyāsanaṃ}} 5\dd{}
\app{\lem[type=emendation, resp=egoscr]{atimūlyo 'śvaḥ}
  \rdg[wit={E}]{\korr atimūlyañ ca}
  \rdg[wit={P}]{atimūlo 'śvaḥ}
  \rdg[wit={L,B}]{atimūlyo asvaṃ}
  \rdg[wit={N1,N2,D1,U1}]{amūlyo svaś ca}
  \rdg[wit={U2}]{suśvaḥ}} 6\dd{}
\app{\lem[wit={ceteri},alt={manoramam annaṃ}]{manoramam-annaṃ}
  \rdg[wit={B}]{manoramyam attaṃ}
  \rdg[wit={L}]{manoramyam annaṃ}
  \rdg[wit={N1,D1}]{manoramam attaṃ}
  \rdg[wit={U2}]{suṣṭu annaṃ}} 7\dd{}
tathā
\app{\lem[wit={ceteri}]{vidhaṃ pānaṃ}
  \rdg[wit={L,B,N2}]{vidhapānaṃ}
  \rdg[wit={U2}]{vidhayānaṃ}} 8\dd{}
\end{prose}
\end{ekdosis}
%------------------------------
%ete   ṣṭau bhogāḥ   kathitāḥ/   eke  duḥkhaṃ   bhajante/  bhikṣāṃ  yācante// kiñca \E
%ete   ṣṭau bhogāḥ   kathitā 9   eke  duḥkha    bhajaṃte   bhikṣāṃ  yāṃcaṃte ca  \P
%ete   ṣṭau bhogāḥ//             eka  duḥkhā    bhajaṃte/  bhikṣā   yāṃcate ca//  \B
%ete   ṣṭau bhogāḥ//             eka  duḥkhā    bhajaṃte// bhikṣā   yāṃcate ca//  \L
%ete  aṣṭau bhogā    kathyate/   eke  duḥkhaṃ   bhajaṃte/  bhikṣyāṃ yācate ca/   \N1
%ete  aṣṭau bhogāḥ   kathyaṃte// ete  duḥkhaṃ   bhajaṃte/  bhikṣyāṃ yācaṃte ca// \D1
%ete  aṣṭau ghogā    kathyate//  ete  duḥkhataṃ bhajate    bhikṣāṃ  yācate ca//  \N2
%rāte aṣṭau bhogāḥ   kathyate    ete  duḥkhaṃ   bhajate    bhikṣāṃ  pācate ca    \U1
%ete  ṣṭau  bhogāḥ// kathitāḥ//  ekaṃ duḥkhaṃ   bhajaṃte// bhikṣā   yācaṃte ca// \U2
%------------------------------
%die acht genüsse wurden erzählt. Sie bringen Leid und die Bet. 
%------------------------------
\begin{ekdosis}
  \ekddiv{type=ed}
  \begin{prose}
\app{\lem[wit={ceteri}]{ete}
  \rdg[wit={U1}]{rāte}}
\app{\lem[wit={ceteri}]{'ṣṭau}
  \rdg[wit={N1,N2,D1,U1}]{aṣṭau}}
\app{\lem[wit={ceteri}]{bhogāḥ}
  \rdg[wit={N1,N2}]{bhogā}
  \rdg[wit={U1}]{ghogā}}
\app{\lem[wit={E,U2}]{kathitāḥ}
  \rdg[wit={P}]{kathitā}
  \rdg[wit={N1,N2,U1}]{kathyate}
  \rdg[wit={D1}]{kathyaṃte}
  \rdg[wit={L,B}]{\om}}\dd{}
  \end{prose}
\end{ekdosis}
%------------------------------
%      yathā sūryasya tejaḥ   dugdhasya    ghṛtam   agner jvalanaṃ viṣān mūrchā   tilāttailam/    vṛkṣāc-chāyā/  phalāt parimalaḥ       kāṣṭhād agniḥ    arkarādibhyo   madhuro rasaḥ/ \E
%      yathā sūryasys tejaḥ   dugdhasya    ghṛtaḥ   agne dvāhaḥ    viṣān mūrchāti tilāttailaṃ     vṛkṣāt-chāyā   phalāsarimalaḥ         kāṣṭād  agniḥ    śarkvarādibhyo madhuro rasaḥ  \P
%      yathā sūryasye tejāḥ   dugdha-------ghṛtaḥ   agne dvāhaḥ//  viṣān mūrchā   tilāttailaṃ//   vṛkṣā--chāyā   phalāsarimalaḥ         kaṣṭād  agniḥ    śarkadībhyo    madhuro  \B
%      yathā sūryasya tejāḥ   dugdha-------ghṛtaḥ   agne dvāhaḥ//  viṣān mūrchā   tilātailaṃ//    vṛkṣā--chāyā   phalāt parimalaḥ       kaṣṭād  agniḥ    śarkadībhyo    madhuro  \L
%      yathā sūryasya tejaḥ/  dugdhasya    ghṛtaṃ/  agne dahiḥ??   viṣān mūrchā   tilāttailaṃ,    vṛkṣāc-chāyā/  phalāt parimalaḥ/      kāṣṭhād āgniḥ/   śarkkarādibhyo madhuro rasaḥ/ \N1
%      yathā sūryasya tejaḥ// dugdhasya    ghṛtaṃ// agne dadhiḥ    viṣān mūrchā   tilāttailaṃ//   vṛkṣā--chāyā// phalāt palātparimalaḥ//kāṣṭhād āgniḥ//  śarkarādibhyo  madhuro rasaḥ/ \D1
%      yathā sūryasya tejaḥ   dusya        ghṛtaṃ   agne dadhi     viṣān mūrchā   tilatailaṃ      vṛkṣā--chāyā   phalāt parimalaḥ       kāṣṭhād āgniḥ    śarkarādibhyo  madhuro rasaḥ/ \N2
%      yathā sūryaśca tejaḥ   dugdhasy     ghṛttaṃ  agne dārhaṃ    viṣāt mūrchā   tilātailaṃ      vrakṣā-chāyā   phalāt parimalaḥ       kāṣṭhād āgniḥ    śarkarādibhyo  madhuro rasaḥ \U1
%      yathā sūryasya tejaḥ// dugdhasya    ghṛtaṃ// agne dāhiḥ//   viṣān mūrchā   tilātailaṃ//    vṛkṣā--chāyā// phalāt parimalaḥ//     kāṣṭād  agniḥ    śarkarādibhyo  madhuro rasaḥ// \U2
%------------------------------
%Gleichwie die Strahlen der Sonne, die Butter der Milch, das Brennen des Feuers, die Betäubung aufgrund von Gift, das Sesamöl aus dem Sesamkorn, der Schatten vom Baum, der Wohlgeruch von einer Frucht, das Feuer von einem Holzscheid, der Süße Saft [em. zu śārkara] a liquor prepared from Dhātakī with sugar] und so weiter,   
%------------------------------
%Like the rays of the sun, the butter of milk, the burning of fire, the stupor of poison, the sesame oil from the sesame seed, the shade from the tree, the sweet odor from a fruit, the fire from a scabbard, the sweet sap [em . to śārkara] a liquor prepared from Dhātakī with sugar] and so on,
%------------------------------
\begin{ekdosis}
\ekddiv{type=ed}
\begin{prose}
yathā
\app{\lem[wit={ceteri}]{sūryasya}
  \rdg[wit={U1}]{sūryaś ca}}
\app{\lem[wit={ceteri}]{tejaḥ}
  \rdg[wit={L,B}]{tejāḥ}}\dd{}
\app{\lem[wit={E,P,N1,D1,U2}]{dugdhasya}
  \rdg[wit={L,B}]{dugdha°}
  \rdg[wit={N2}]{dusya}
  \rdg[wit={U1}]{dugdhasy}}
\app{\lem[wit={ceteri}]{ghṛtaṃ}
  \rdg[wit={P,L,B}]{ghṛtaḥ}}\dd{}
\app{\lem[wit={E}, alt={agner}]{agne\skp{r-dā}}
  \rdg[wit={ceteri}]{agne}
}\app{\lem[type=emendation, resp=egoscr, alt={dāhaḥ}]{\skm{r-dā}haḥ}
  \rdg[wit={P,L,B}]{\korr dvāhaḥ}
  \rdg[wit={N1}]{dahiḥ}
  \rdg[wit={N2}]{dadhi}
  \rdg[wit={D1}]{dadhiḥ}
  \rdg[wit={U1}]{dārhaṃ}
  \rdg[wit={U2}]{dāhiḥ}
  \rdg[wit={E}]{jvalanaṃ}}\dd{}
\app{\lem[wit={ceteri}]{viṣā\skp{n-mū}}
  \rdg[wit={U1}]{viṣāt}
}\skm{n-mū}rchā\dd{}
\app{\lem[wit={ceteri},alt={tilāt}]{tilā\skp{t-tai}}
  \rdg[wit={P}]{titilāt}
  \rdg[wit={N2}]{tila}
  \rdg[wit={U1}]{tilā}
}\skm{t-tai}laṃ\dd{}
\app{\lem[wit={E,N1}, alt={vṛkṣāt}]{vṛkṣā\skp{c-chā}}
  \rdg[wit={P}]{vṛkṣāt}
  \rdg[wit={L,B,N2,D1,U2}]{vṛkṣā}
  \rdg[wit={U1}]{vrakṣā}
}\skm{c-chā}yā\dd{}
\app{\lem[wit={ceteri},alt={phalāt}]{phalā\skp{t-pa}}
  \rdg[wit={L,B}]{phalā}
}\app{\lem[wit={ceteri},alt={parimalaḥ}]{\skm{t-pa}rimalaḥ}
  \rdg[wit={L,B}]{sarimalaḥ}
  \rdg[wit={D1}]{palāt parimalaḥ}}\dd{}\note[type=philcomm, labelb=s27.z10, lem={parimalaḥ}]{Clarification: Witness \getsiglum{D1} reads \textit{phalāt palāt parimala}.}
\app{\lem[wit={ceteri}, alt={kāṣṭhād}]{kāṣṭhā\skp{d-a}}
  \rdg[wit={P,U2}]{kāṣṭād}
  \rdg[wit={B,L}]{kaṣṭād}
}\app{\lem[wit={ceteri}, alt={agniḥ}]{\skm{d-a}gniḥ}
  \rdg[wit={N1,N2,D1,U1}]{āgniḥ}}\dd{}
\app{\lem[type=emendation, resp=egoscr]{śārkarādibhyo}
  \rdg[wit={E}]{\korr arkarādibhyo}
  \rdg[wit={P}]{śarkvarādibhyo}
  \rdg[wit={L,B}]{śarkadībhyo}}
madhuro
\app{\lem[wit={ceteri}]{rasaḥ}
  \rdg[wit={L,B}]{\om}}\dd{}
\ekdpb*{}
%------------------------------
%himānībhyaḥ   śītam      ityādipadārthānāṃ svabhāvaḥ         tathā    saṃsāro'pi parameśvarasvarūpamadhye      tiṣṭhati/ \E
%himānībhyaḥ   śītaṃ      ityādipadārthasvabhāva        eva   tathā    saṃsāro'pi parameśvarasvarūpamadhye      tiṣṭhati \P
%sahīmānībhyaḥ śītaḥ/     ityādipadārthāsvabhāvataḥ// eva     tathā    saṃsāro pi paremesvara svarūpasya madhye tiṣṭhatī/ \B
%sahimānibhyaḥ śītaḥ//    ityādiphadārthāḥ svabhāvataḥ// eva  tathā    saṃsāro pi paremesvara svarūpasya madhye tiṣṭhati// \L
%himānibhyaḥ   śaityāṃ    ityādipadārthasvabhāva evā/         tathā    saṃsāro pi parameśvarasvarūpamadhye      tiṣṭhati// \N1
%himānibhyaḥ   śaityaṃ // ityādipadārthasvabhāva eva//        tathā    saṃsāro pi parameśvarasvarūpamadhye      tiṣṭhati// \D1 
%himānitpa     śaityāś    atyādipadārtharthasvabhāva eva//    tathā    saṃsāro pi parameśvarasvarūpamadhye      tiṣṭhati \N2
%himānībhyaḥ   śaityaṃ    ityādipadārthasvabhāvaḥ ravaḥ?      tathā vā saṃsāro pi parameśvararūpamadhye         tiṣṭhati/ \U1
%himānībhyaḥ   śītyaṃ//   ityādipadārthāsvabhāva eva//        tathā    saṃsāro pi parameśvarasvarūpamadhye      tiṣṭhaṃti// \U2
%------------------------------
%die Kälte von Schneehaufen, und so weiter ist das inhärente Wesen der Dinge. IN gleicher Weise befindet sich auch der Weltengang im Zentrum der eigenen Gestalt von höchsten Gott.
%the cold of piles of snow, and so on is the inherent essence of things. In the same way, the course of the world is also in the center of the highest God's own form. 
%------------------------------
\app{\lem[wit={ceteri}]{himānībhyaḥ}
  \rdg[wit={L,B}]{sahimānibhyaḥ}
  \rdg[wit={N2}]{himānitpa}}
\app{\lem[wit={D1,U1}]{śaityaṃ}
  \rdg[wit={N1}]{śaityāṃ}
  \rdg[wit={U2}]{śītyaṃ}
  \rdg[wit={N2}]{śaityāś}
  \rdg[wit={E,P}]{śītaṃ}
  \rdg[wit={L,B}]{śītaḥ}}\dd{}
\app{\lem[wit={N1,D1,P}]{ityādipadārthasvabhāva}
  \rdg[wit={U2}]{°padārthā°}
  \rdg[wit={B}]{ityādipadārthāsvabhāvataḥ}
  \rdg[wit={N2}]{atyādipadārtharthasvabhāva}
  \rdg[wit={U1}]{°svabhāvaḥ}
  \rdg[wit={L}]{ityādiphadārthāḥ svabhāvataḥ}
  \rdg[wit={E}]{ityādipadārthānāṃ svabhāvaḥ}}
\app{\lem[wit={ceteri}]{eva}
  \rdg[wit={N1}]{evā}
  \rdg[wit={U1}]{ravaḥ}
  \rdg[wit={E}]{\om}}\dd{}
\app{\lem[wit={ceteri}]{tathā}
  \rdg[wit={U1}]{tathā vā}}
saṃsāro 'pi
\app{\lem[wit={ceteri}]{parameśvarasvarūpamadhye}
  \rdg[wit={L,B}]{paremesvara svarūpasya madhye}
  \rdg[wit={U1}]{parameśvararūpamadhye}}
\app{\lem[wit={ceteri}]{tiṣṭhati}
  \rdg[wit={B}]{tiṣṭhatī}
  \rdg[wit={U2}]{tiṣṭhaṃti}}\dd{}
%------------------------------
%parameśvaro 'khaṇḍa--paripūrṇaḥ/  \E
%parameśvaro khaṃḍa---paripūrṇaś ca    \P
%parameśvaro khaṃḍa---paripūrṇaś ca// \B
%parameśvaro khaṃḍa---paripūrṇaś ca//  \L
%parameśvaro 'ṣaṃḍa---paripūrṇaś ca//  \N1
%parameśvaro  ṣaṃḍa---paripūrṇaś ca//  \D1 %%%S.9 verso
%parameśvaro yarāṇḍa--paripūrṇaś ca//  \N2
%parameśvaro khaṃḍaḥ  paripūrṇaś ca   \U1 %%%277.jpg
%parameśvaro 'khaṃḍa--paripūrṇaś ca//   \U2
%------------------------------
%Und der höchste Gott ist unteilbar und das All erfüllend.
%And the Most High God is indivisible and all-filling.
%------------------------------
parameśvaro
\app{\lem[wit={ceteri}, alt={'khaṇḍa}]{'khaṇda}
  \rdg[wit={N1,D1}]{'ṣaṃḍa°}
  \rdg[wit={N2}]{yarānda°}
  \rdg[wit={U1}]{khaṃḍaḥ}
}\app{\lem[wit={ceteri}]{paripūrṇaś\skp{-}ca}
  \rdg[wit={E}]{paripūrṇaḥ}}\dd{} 
  \end{prose}
\end{ekdosis}
%------------------------------
%idānīṃ lakṣyaṃ kathyate/ \E
%idānīṃ bāhyalakṣyaṃ kathyate \P
%idānīṃ ṣāhyalakṣa kathyate// \B
%idānīṃ bāhyalakṣa kathyate// \L
%idānīṃ bāhyalakṣaṃ kathyate// \N1
%idānīṃ bāhyalakṣaṇa kathyate// \D1 %%%S.9 verso
%idānīṃ bāhyalakṣaṇa kathyate/ \N2
%idānīṃ bāhyalakṣyaḥ kathyate \U1 %%%277.jpg
%idānīṃ bāhyalakṣaṇaṃ kathyate// \U2
%------------------------------
%
%------------------------------
%nāsāgrādārabhyāṃgulacatuṣṭayapramāṇaṃ nīlākāraṃ tejaḥ pūrṇamākāśaṃ lakṣyaṃ karttavyam/ \E
%nāsāgrādārabhyāṃgulacatuṣṭayapramāṇaṃ nilākāraṃ tejaḥ pūrṇam ākāśaṃ lakṣyaṃ kartavyaṃ  \P
%nāsāgrādārabhyāṃgulacatuṣṭayaṃ pramāṇaṃ nilākāraṃjaḥ pūrṇam ākāśalakṣaṃ kartavyaṃ//    \B
%nāsāgrād ārabhyāṃ gulacatuṣṭayaṃ pramāṇaṃ nilākāraṃ tejaḥ// pūrṇam ākāśaṃ lakṣaṃ kartavyaṃ// \L
%nāsāgrādārabhyāṃgulacatuṣṭayapramāṇaṃ nīlākāraṃ tejapūrṇam ākāśalakṣaṃ karttavyaṃ \N1
%nāsāgrādārabhyāṃgulacatuṣṭayapramāṇaṃ nīlākāraṃ tejapūrṇamākāśalakṣaṃ karttavyaṃ \D1
%nāsāgrādārabhyāṃgulacatuṣṭayapramāṇaṃ nirākāraṃ tejapūrṇamākāśalakṣaṇaṃ karttavyaṃ// \N2
%nāsāgrādārabhyāṃgulacatuṣṭayapramāṇaṃ nīlākāraṃ tejaḥ pūrṇam ākāśaṃ lakṣyaṃ karttavyam \U1
%nāsāgrādārabhyāṃgulacatuṣṭayapramāṇaṃ nīlākāraṃ tejaḥ pūrṇakāmākāśalakṣyaṃ karttavyaṃ \U2 %%%411.jpg
%
%------------------------------
%
%------------------------------
%athavā nāsāgrād ārabhyāṃgulacatuṣṭayapramāṇaṃ nīlākāraṃ tejaḥ pūrṇamākāśaṃ  lakṣyaṃ karttavyam/ \E
%\om                                                                                             \P
%athavā nāsāgrād ārabhyaṣaḍaṃgulaṃ pramāṇaṃ ?bi?ṣi?īnāvarṇaṃ .. .. .. ..??.  lakṣyaṃ kartavyam/  \B
%athavā nāsāgrād ārabhya dvādaśāṃgulapramāṇaṃ pītavarṇaṃ pṛthvītatvaṃ     .  lakṣaṃ  kartavyaṃ/  \L
%\om                                                                                             \N1
%\om                                                                                             \D1
%\om                                                                                             \N2
%\om                                                                                             \U1
%\om                                                                                             \U2
%------------------------------
%
%------------------------------
%athavā nāsāgrādārabhya ṣaḍaṃgulapramāṇaṃ pavanatattvaṃ dhūmrākāraṃ lakṣyaṃ karttavyam// \E
%athavā nāsāgrād ārabhya ṣaḍaṃgulapramāṇaṃ pavanatatvaṃ dhūmrākāraṃ lakṣyaṃ karttavyam \P
%\om \B?
%\om \L
%athavā nāsāgrādābhya   ṣadaṃgulapramāṇaṃ pavanatatvaṃ dhūmrākāraṃ lakṣaṃ karttavyaṃ/  \N1
%athavā nāsāgrādābhya   ṣadaṃgulapramāṇaṃ pavanatatvaṃ dhūmrākāraṃ lakṣaṃ karttavyaṃ// \D1
%athavā nāsāgrārabhya   ṣadaṃgulapramāṇaṃ pavanatatvaṃ dhūmrākāraṃ lakṣaṇaṃ karttavyaṃ// \N2
%athavā nāsāgrādārabhyadvadaśaṃgulapramāṇaṃ pavanatatvaṃ dhūmrākāraṃ lakṣyaṃ karttavyaṃ \U1
%atha vā nāsāgrād ārabhyaṣaḍaṃgulapramāṇaṃ pavanatatvaṃ dhūmrākāraṃ lakṣaṃ karttavyaṃ// \U2
%------------------------------
%
%------------------------------
%atha vā nāsāgrādārabhya tattvaṃ dādaśāṃgulapramāṇaṃ pītavarṇaṃ pṛthvītattvaṃ lakṣyaṃ karttavyam/ \E
%atha vā nāsāgrād ārabhya dvādaśāṃgulapramāṇaṃ pītavarṇaṃ pṛthvītatvaṃ lakṣyaṃ karttavyaṃ \P
%athavā nāsāgrādārabhya dvadaśāṃgulapramāṇaṃ pītavarṇaṃ pṛthvītatvaṃ lakṣaṃ kartavyaṃ// \B
% \om                                                                                    \L
%atha vā nāsāgrādārabhyā  ṣaḍaṃgulapramāṇā mati raktaṃ tejo lakṣaṃ karttavyaṃ \N1
%atha vā nāsāgrādārabhya  ṣaḍaṃgulapramāṇā mati rattaṃ tejo lakṣaṃ karttavyaṃ// \D1
%atha vā nāsāgrādārabhyaṃ ṣṭāṃgulapramāṇamati rakṭaṃ tejo lakṣaṇaṃ kartavyaṃ// \N2
%atha nāsāgrādārabhyāṣṭaṃgulapramāṇamitiriktaṃ tejo lakṣyaṃ karttavyaṃ/ \U1
%atha vā nāsāgrādārabhyaṃ ṣṭagulapramāṇaṃ mati raktaṃ teja lakṣyaṃ karttavyaṃ// \U2 
%------------------------------
%
%------------------------------
%atha vā nāsāgrādārabhya koṭisūryasamaprabhaṃ tejaḥ/ \E
%atha vā nāsāgrādārabhya koṭisūryasamaprabhaṃ tejaḥ   \P
%atha vā nāsāgrādārabhya koṭisūryasamaprabhaṃ tejaḥ/  \B
%\om \L
%atha vā nāsāgrādārabhya daśāṃgulapramāṇaṃ śuklaṃcaṃcalamudakaṃ lakṣya karttavyaṃ/ \N1
%atha vā nāsāgrādārabhya daśāṃgulapramāṇaṃ śuklaṃcaṃ?calamudakaṃ lakṣya karttavyaṃ// \D1
%atha vā nāsāgrādārabhya daśāṃgulapramāṇaṃ śuklaṃcaṃdrākāramudakaṃ lakṣyaṃ kartavyaṃ \U1
%atha vā nāsāgrādārabhya daśāṃgulapramāṇaṃ śuklaṃ caṃcalamudakaṃ lakṣaṇaṃ kartavyaṃ// \N2 [S.7 Verso, Zeile 1]
%%atha vā nāsāgrādārabhya daśāṃgulapramāṇaṃ śuklaṃ caṃcalam udakaṃ lakṣaṃ kartavyaṃ// \U2
%------------------------------
%
%------------------------------
%\om                                                                                          \E
%\om                                                                                          \P
%\om                                                                                          \B
%\om                                                                                          \L
%atha vā nāsāgrād ārabhya dvadaśāṃgulapramāṇaṃ pītavarṇṇaṃ prthvītatvaṃ lakṣaṃ karttavyaṃ/   \N1
%\om                                                                                          \D1???????????????? CHECK!
%atha vā nāsāgrād ārabhya dvadaśāṃgulapramāṇaṃ pītavarṇaṃ prthvītatvaṃ lakṣaṇaṃ karttavyaṃ//  \N2
%atha vā nāsāgrād ārabhya dvādaśā aṃgulapramāṇaṃ pītavarṇaṃ prthvītatvaṃ lakṣyaṃ karttavyaṃ   \U1
%atha vā nāsāgrād ārabhya dvādaśāṃgulapramāṇaṃ pītavarṇaṃ pṛthvītatvaṃ lakṣaṃ karttavyaṃ//    \U2
%------------------------------
%
%------------------------------
%\om                                                                 \E
%\om                                                                 \P
%\om                                                                 \B  DSCN7161.JPG letzte 3 Zeilen!
%atha vā nāsāgrād ārabhya koṭisūryasamaprabhāṃ tejaḥ                   \L
%atha vā nāsāgrād ārabhya koṭisūryasamaprabhaṃ tejaḥ                    \N1
%atha vā nāsāgrād ārabhya koṭisūryasamaprabhaṃ tejaḥ                    \D1
%atha vā nāsāgrād ārabhya koṭisūryasamaprabhaṃ tejaḥ                    \N2
%atha vā nāsāgrād ārabhya koṭisūryasamaprabhaṃ tejaḥ                    \U1
%atha vā nāsāgrād ārabhya koṭisūryasamaprabhaṃ tejaḥ                    \U2
%------------------------------
%
%------------------------------
%pūrṇam ākāśatattvaṃ lakṣyaṃ karttavyam/ ākāśamadhye ākāśopari    dṛṣṭiṃ kṛtvā dhyānakāraṇāt// sūryaṃ vinā sūryasambandhinī sahasrakiraṇapaṅktīḥ paśyati/ \E
%pūrṇam ākāśatatvaṃ lakṣyaṃ karttavyaṃ               ākāśopari    dṛṣtiṃ kṛtvā dhyānakaraṇāt sūryaṃ vinā sūryasaṃbaṃdhīnīṃ sahasrakiraṇāvalīṃ pati   \P
%pūrṇam ākāśatatvaṃ lakṣaṃ kartavyaṃ//               ākāśopari    dṛṣti  kṛtvā ākāśamadhyedhyānakaraṇāt// sūryaṃ vinā sūryasaṃbaṃdhīnī sahasrakiraṇāvali paśyatī//\B
%pūrṇam ākāśatatvaṃ lakṣaṃ karttavyaṃ//  ākāśamadhye dhyānakaraṇāt//                                      sūryaṃ vinā sūryasaṃbaṃdhinī sahasrakiraṇāvali paśyati// \L
%pūrṇaṃ ākāśatatvaṃ lakṣyaṃ karttavyaṃ/  ākāśamadhye  ākāśoparī vā dṛṣṭiṃ kṛtvā dhyānakaraṇāt sūryaṃ vinā sūryasaṃbaṃdhinī sahasrāṇy api kīraṇāṇi paśyatī/     \N1
%pūrṇaṃ ākāśatatvaṃ lakṣyaṃ karttavyaṃ// ākāśamadhye  ākāśopari vā dṛṣṭiṃ kṛtvā dhyānakaraṇāt sūryaṃ vinā sūryasaṃbaṃdhinī sahasrāṇapi kīraṇāṇi paśyatī//     \D1
%pūrṇa  ākāśatatvaṃ lakṣaṇaṃ karttavyaṃ//ākāśamadhye  ākāśopari vā dṛṣṭiṃ kṛtvā dhyānakaraṇāt sūryavināsūryasaṃbaṃdhinī sahasrāṇapi kiraṇāṇi paśyate//     \N2
%pūrṇaṃ ākāśatatvaṃ lakṣyaṃ karttavyaṃ   ākāśamadhye  ākāśopari vā dṛṣṭiṃ kṛtvā dhyānakaraṇāt sūryaṃ vināsūryasaṃbaṃdhinī sahasrāṇyapi kiraṇāṇi paśyaṃti     \U1
%pūrṇaṃ ākāśatatvaṃ lakṣaṃ karttavyaṃ//  ākāśamadhye  ākāśopari vā dṛṣṭiṃ kṛtvā dhyānakaraṇāt// sūrya vinā sūryasaṃbaṃdhinī sahasrakiraṇāvaliṃ paśyati//     \U2
%------------------------------
%
%-----------------------------
%atha vā śivopari vṛddhaṃ saptadaśāṃgulapramāṇaṃ tejaḥ puṃjalakṣyaṃ karttavyam/ \E
%\om \P
%atha vā śiroparir urdhvaṃ saptadaśāṃgulapramāṇaṃ tejaḥ pūṃjaṃlakṣaṇaṃ kartavyaṃ/ \B
%atha vā śiropari ūrdhvasaptadaśāṃgulapramāṇaṃ tejaḥ pūṃjaṃlakṣaṃ      kartavyaṃ/ \L
%atha kā śiropari ūrddhvaṃ saptadaśāṃgulapramāṇaṃ tejā puṃjalakṣaṃ karttavyaṃ/ \N1
%atha vā śiropari ūrddhvaṃ saptadaśāṃgulapramāṇaṃ tejā puṃjalakṣyaṃ karttavyaṃ// \D1
%atha vā śiropari ūrddhvaṃ saptadaśāṃgulaṃ parāṇaṃ tejaḥ puṃjalakṣaṇaṃ kartavyaṃ// \N2
%atha vā śiropari ūrddhaṃ saptadaśāṃgulapramāṇaṃ tejaḥ puṃjakaṃ lakṣyaṃ kartavyaṃ \U1 %%%278.jpg
%atha vā śiropari ūrddhaṃ saptadaśāṃgulapramāṇatejaḥ puṃjaṃ lakṣyaṃ karttavyaṃ// \U2
%-----------------------------
%
%-----------------------------
%atha vā dṛṣṭer agre tatparaṃ svarṇākāraṃ pṛthvītattvaṃ lakṣyaṃ kartavyam/ \E
%atha vā dṛṣṭer agne taptasvarṇavarṇakāraṃ pṛthvītatvaṃ lakṣyaṃ \P
%atha vā dṛṣṭeragne taptasuvarṇavarṇapṛthivītatvaṃ lakṣaṃ kartavyaṃ/ \B
%atha vā dṛṣṭer agne taptasuvarṇavarṇapṛthītatvaṃ lakṣaṃ kartavyaṃ/ \L
%atha vā dṛṣṭer ag..? taptavarṇākāraṃ pṛthvītatvaṃ lakṣaṃ karttavyaṃ/ \N1
%atha vā dṛṣṭer agre taptavarṇākāraṃ pṛthvītatvaṃ lakṣaṃ karttavyaṃ// \D1 %%%p.10 beginning
%atha vā dṛṣṭer agre taptavarṇākāraṃ pṛthvītatvaṃ lakṣaṇaṃ karttavyaṃ/ \N2
%atha vā dṛṣṭer agre taptavarṇākāraṃ pṛthvītatvaṃ lakṣyaṃ karttavyaṃ \U1
%atha vā dṛṣṭer agre taptasvarṇavarṇākāraṃ pṛthvīṃ tatvaṃ lakṣaṃ karttavyaṃ// \U2
%-----------------------------
%
%-----------------------------
%uktānāṃ lakṣyāṇāṃ madhye yasya kasyāpyekasya lakṣyakaraṇāt valita palitā dūre bhavanti/ \E
%uktānāṃ lakṣaṇānāṃ madhye yasya kasyāppyekasya lakṣyakaraṇāt  valitpalitādi dūre bhavati \P
%uktānāṃ lakṣaṇaṃ madhye yasya kasyāpi kasya lakṣakaraṇāt// valitaṃ palitādi dūre bhavatī/ \B
%uktānāṃ lakṣaṇaṃ madhye yasya kasyāpi kasya lakṣakaraṇāt// valitaṃ palitādi dūre bhavati// \L
%uktānāṃ lakṣyaṇāṃ madhye yasya kasyāpyekasya lakṣasya karaṇāt valitapalitādi dūre bhavati \N1
%uktānāṃ lakṣyaṇaṃ madhye yasya kasyāpekasya lakṣasya karaṇāt valitapalitādi dūre bhavati// \D1
%uktānāṃ lakṣāṇāmadhye yasya lasyāpyelasya lakṣaṇasya karaṇāt /va/oder??/dva/lita palitādidūre bhavati/ \N2
%uktānāṃ lakṣyaṇāṃ madhye yasya kasyāpi kasya lakṣyasya karaṇā valitapalitādidūre bhavati \U1
%uktānāṃ lakṣāṃ madhye yasya kasyāpye kasya lakṣyakaraṇāt valitapalitādidūre bhavaṃti// \U2
%-----------------------------
%
%-----------------------------
%aṃgarogāḥ vinauṣadhaṃ dūrībhavanti/ samagrāḥ śatravaḥ svapnepyamitrannāyāṃti/ sahasravarṣamāyur bhavati/ \E
%aṃgirogā vinauṣadhaṃ dūre bhavati samagrāḥ śatravaḥ  svapnevyamitratāmayāṃti  sahasravarṣam āyur varddhate \P  %%%7646.jpg Z.1 
%aṃgirogādi vinauṣadhaṃ dūro bhavatī samagrāḥ śatrave svapnevyamitratāmayāṃti// sahasravarṣam āyur vardhate// \B
%aṃgirogādi vinauṣadhaṃ dūro bhavati samagrāḥ śatravo svapnevyamitratāmayāṃtī sahasravarṣam āyur vardhate// \L
%aṃgarogā vinauṣadhaṃ dūre bhavaṃti/ samagrāḥ śatravaḥ svapinevam ityaṃ nāyāti/ sahasravarṣaparyaṃtam āyuṣaṃ varddhate/ \N1
%aṃgarogā vinauṣadhaṃ dūre bhavaṃti// samagrāḥ śatravaḥ svacanevam ityaṃ nāyāti// sahasravarṣaparyaṃtam āyuṣaṃ varddhate// \D1
%aṃgarogā vinauṣadhaṃ dūre bhavati// samagrā śatravaḥ svapinevan nityaṃ nāyāti// sahasravarṣaparyaṃtamāyuṣaṃ vardhate// \N2
%aṃgarogā vinauṣadhaṃ dūre bhavati  samagrāḥ śatravaḥ svapinevam itevaṃ naiyati sahasravarṣaparyaṃtamāyuṣyaṃ varddhate \U1
%aṃgarogā vinauṣadhaṃ dūre bhavaṃti//  samagra śatravaḥ svapnepi mitratāmāyāṃti// sahasravarṣam āyur varddhate// \U2
%-----------------------------
%
%-----------------------------
%apaṭhitaṃ śāstraṃ jihvāgreṇoccarati/  etādṛśaṃ phalaṃ bahutaraṃ bhavati// \E
%apaṭhitaṃ śāstraṃ jihvāgreṇoccarati   etādṛśaṃ mitratāmāyāṃti sahasravarṣamāyur varddhate apaṭhitaṃ śāstraṃ jihvāgreṇoccarati etādṛśaṃ phalaṃ bahutaraṃ bhavati \P
%apaṭhitaṃ śāstraṃ jihvāgreṇoccaratī/  etādṛśaṃ phalaṃ bahutaraṃ bhavatī// \B
%apaṭhitaṃ śāstraṃ jihvāgreṇoccarati   etādṛśaṃ phalaṃ bahutaraṃ bhavaṃtī// \L
%apaṭhitaṃ śāstraṃ jihvāgreṇoccarate// etādṛśaṃ bahutaraṃ phalaṃ bhavati// \N1
%apaṭhitaṃ śāstraṃ jihvāgreṇoccarate// etādṛśaṃ bahutaraṃ phalaṃ bhavati// \D1
%apathitaṃ śāstraṃ jihvāgreṇoccarate// etādṛśaṃ bahutaraṃ phalaṃ bhavati// \N2
%apathitaṃ śāstraṃ jihvāgreṇoccarate   etādṛśyaṃ bahutaraṃ phalaṃ bhavati \U1
%apathitaṃ śāstraṃ jihvāgreṇoccarati// etādṛśaṃ phalaṃ bahutaraṃ phalaṃ bhavati// \U2
%-----------------------------
%
%-----------------------------
%idānīm anyataraṃ lakṣyaṃ kathyate/ \E
%idānīṃ aṃtaraṃ lakṣyaṃ kathyate \P
%idānīṃ antaralakṣaṃ kartavyaṃ// \B
%idānīṃ aṃtaralakṣaṃ kartavyaṃ// \L
%idānīṃ antaralakṣyakaṃ kathyate// \N1
%idānīṃ antaralakṣyaṃ kathyate// \D1
%idānīṃ aṇtaralakṣyaṇaṃ kathyate// \N2
%idānīṃ aṇtaralakṣyaṇaṃ kathyate \U1
%idānīm ataraṃ lakṣyaṃ  kathyate// \U2
%-----------------------------
%
%-----------------------------
%mūlakandasthāne brahmadaṇḍotpannā nāḍī śvetavarṇā brahmadaṇḍaparyantam ekā brahmanāḍī varttate/ \E
%mūlakaṃdasthāne brahmānaṃḍādutpannā śvetavarṇā brahmaraṃdhraparyaṃtaṃ ekā brahmanāḍī varttate   \P
%mūlakaṃ sthāne  brahmānaṃḍādutpannā śvetāvarṇā brahmaraṃdhraparyaṃtaṃ ekā nāḍī vartate/     \B
%mūlakaṃdasthāne  brahmānaṃdādutpannā śvetāvarṇā brahmaraṃdhraparyaṃtaṃ ekanāḍī vartate/     \L
%mūlakaṃdasthāne brahmadaṃḍa ityannā śvetavarṇā brahmaraṃdhraparyaṃtaṃ ekā brahmanāḍī varttate/ \N1
%mūlakaṃdasthāne brahmadaṃḍādutpannā śvetavarṇā// brahmaraṃdhraparyaṃtaṃ ekā brahmanāḍī varttate// \D1
%mūlakaṃdasthāne brahmadaṇḍadūtpannā śvetavarṇā brahmaraṃdhraparyaṃtaṃ ekā brahmanāḍī varttate/ \N2
%mūlakaṃdasthāne brahmadaṇādūtpannaḥ śvetavarṇāṃ brahmaraṃdhraparyaṃtaṃ ekā brahmanāḍī varttate \U1
%mūlakaṃdasthāne brahmadaṇḍādutpannā śvetavarṇā brahmaraṃdhryaparyaṃtaṃ ekā brahmanāḍī varttate// \U2
%-----------------------------
%
%-----------------------------
%brahmanāḍīmadhye kamalatantusamānākārā koṭisūryavidyutsamaprabhā ūrdhvaṃ calati/ \E
%brahmanāḍīmadhye kamalataṃ samānākārā koṭisūryavidyutsamaprabhā ūrdhvaṃ calati \P
%brahmanāḍīmadhye kamalataṃtusamānākārā koṭisūryavidyutsabhāprabhā ūrdhvaṃ calati/ \B
%brahmanāḍīmadhye kamalataṃtusamānākārā koṭisūryavidyutsabhāprabhā ūrdhvaṃ calati/ \L
%brahmanāḍīmadhye kamalatantusamānākārā koṭisūryavidyutsamaprabhā ūrdhvaṃ calati/ \N1
%brahmanāḍīmadhye kamalataṃtusamānākārā koṭisūryavidyutsamaprabhā ūrdhvaṃ calati// \D1
%\om                                                                              \N2
%brahmanāḍīmadhye kamalatantusamānākārā koṭisūryavidyutsamaprabhā rdhvaṃ ccalati  \U1
%brahmanāḍīmadhye kamalataṃtusamānākārā koṭisūryavidyutsamaprabhā// urdhvaṃ calati  \U2 %%%412.jpg 
%-----------------------------
%
%-----------------------------
%etādṛśyekā mūrttir varttate/ tanmūrterdhyānakāraṇāt         aṣṭamahāsiddhayo'ṇimādayastasya puruṣasya samīpamāgatya tiṣṭhanti// \E
%etādṛśyekā mūrttir vartate  tasyā mūrter dhyānakaraṇāt      aṣṭamahāsiddhayo ṇimādyāḥ aṇimāmahimāladhīmāgirimāduredīyavā dure stutvā parakāyapraveśītā puruṣasya samī māgatya tiṣṭhaṃti \P
%etādṛśyekā mūrttir varttate/ tasyāmūrtedhyānakaraṇāt//      aṣṭamahāsiddhayo// aṇimādyāḥ// aṇimāmahimāladhimāgirimādurevāyadivā yadi vā dure śrutvā parakāyāpraveśitā// puruṣasya samīpe māgatya tiṣṭhati// \B
%etādṛśyekā mūrttir varttate/ tasyāmūrter dhyānakaraṇāt//    aṣṭamahāsiddhayo aṇimādyāḥ// aṇimāmahimāladhimāgarimādure vāyadi vā ddure śrutvā parakāyāpraveśitā puruṣasya samīpam āgatya tiṣṭhati// \L
%etādṛśī ekā mūrttir varttate/ tasyāḥ mūrtter dhyānakāraṇāt/ aṇimādīsiddhiḥ                  puruṣasya samīpe? āgatya tiṣṭhanti// \N1
%etādṛśī ekā mūrttir varttate// tasyā mūrtter dhyānakāraṇāt// aṇimādyaṣṭasiddhiḥ puruṣasya samīpe āgatya tiṣṭhati// \D1
%\om \N2
% aṇimādyaṣṭasiddhiḥ puruṣasya sāmīpe āgatya tiṣṭhati \U1
%etādṛśyekā mūrttir varttate// tasyā mūrter dhyānakaraṇāt//  aṣṭamahāsiddhayo aṇimādyāḥ// puruṣasya samīpam āgamya tiṣṭhati// \U2
%-----------------------------
%
%-----------------------------
%atha vā lalāṭoparyākāśamadhye   śuklasadṛśasya tejaso dhyānakāraṇāt       śarīrasambandhinaḥ  kuṣṭhādayo rogā  naśyanti/  āyur vṛddhir bhavati/  \E
%atha vā lalāṭopari ākāśamadhye śuklasadṛśasya tejaso dhyānakāraṇāt       śarīrasaṃbaṃdhinaḥ  kuṣṭhādayo rogāḥ  naśyaṃtī   āyur vṛddhir bhavati   \P  %%%7647.jpg
%atha vā lalāṭopari ākāśamadhye śuklasadṛśasya tejaso dhyānakāraṇāt//     charīrasambandhinaḥ kuṣṭhādayo rogā  naśyaṃtī//  āyur vṛddhir bhavatī   \B
%atha vā lalāṭopari ākāśamadhye śuklasadṛśasya tejaso dhyānakāraṇāt       charīrasambandhinaḥ kuṣṭhādayo rogā  naśyaṃti//  āyur vṛddhir bhavati// \L
%atha vā lalāṭopari ākāśamadhye śuklasadṛśasya tejaso dhyānakāraṇāt       śarīrasambandhī     kuṣṭhādayo rogāḥ naśyaṃti/   āyur vṛddhir bhavati/  \N1
%atha vā lalāṭopari ākāśamadhye śuklasadṛśasya tejaso dhyānakāraṇāt       śarīrasaṃbaṃdhī     kuṣṭādayo rogāḥ naśyaṃti//   āyur vṛddhir bhavati//  \D1
%                                        tasyā mūrtter dhyānakaraṇāc       charīrasaṃbaṃdhi----kuṣṭadayo  rogāḥ naśyaṃti   āyur vṛddi   bhavati/  \N2
%atha vā lalāṭoparī ākāśamadhye śuklasadṛśasya tejo   dhyānakāraṇāt       śarīrasambaṃdhī     kuṣṭhādayo rogā naśyaṃti     āyur vṛddhir bhavati   \U1 %%%279.jpg
%atha vā lalāṭoparī ākāśamadhye śuklasadṛśasya tejaso dhyānakāraṇāt//     śarīrasambaṃdhinaḥ  kuṣṭhādayo rogā naśyaṃti//   āyur vṛddhir bhavati//  \U2
%-----------------------------
%
%-----------------------------
%          bhruvor madhyetiriktavarṇasyātisthūlasya tejaso dhyānakāraṇādbahulānāṃ pārthivānāṃ tatpuruṣāṇāṃ ca vallabho bhavati/     \E
%atha vā   bhruvor madhye  tiraktavarṇasyātisthūlasya tejaso dhyānakaraṇāt   sakalānāṃ pārthivapuruṣāṇāṃ vallabho bhavati          \P
%atha vā// bhruvor madhye 'tiraktavarṇasyātisthūlasya tejaso dhyānaṃ karaṇāt-sakalānāṃ pārthivapuruṣāṇāṃ vallabho bhavati/         \B DSCN7163.jpg Z.1
%atha vā// bhruvor madhye 'tiraktavarṇasyātisthūlasya tejaso dhyānakaraṇāt   sakalānāṃ pārthivapuruṣāṇāṃ vallabho bhavati/         \L
%atha vā   bhruvor madhye 'tiraktavarṇasyātisthūlasya tejaso dhyānakaraṇāt-sakalānāṃ pārthivapuruṣāṇāṃ vallabho bhavati/           \N1
%atha vā   bhruvor madhye 'tiraktavarṇasyātisthūlasya tejaso dhyānakaraṇāt-sakalānā pārthivapuruṣāṇāṃ vallabho bhavati             \D1 %%%p.10 verso
%atha vā   bhruvor madhyetiraktavarṇasyātisthūlasya tejaso dhyānakaraṇāt-sakālānāṃ pārthivapuruṣāṇāṃ vallabho bhavati/             \N2
%atha vā   bhruvor madhye ti raktavarṇasyātī sthalasya tejaso dhyānakaraṇāt sakalānāṃ pārthivapuruṣāṇāṃ vallabho bhavati/          \U1
%atha vā   bṛvor madhye ati raktavarṇasya 'ti sthūlasyaḥ tejāso dhyānakaraṇāt sakalānāṃ pārthivapuruṣāṇāṃ vallabho bhavati         \U2
%-----------------------------
%Or also/ Because of executing meditation on the middle of the eyebrows [of which there is] a very subtle and red colored light he is one who is beloved among all royal people.    
%-----------------------------
%jagadvallabho pi bhavati/ asya puruṣasyāvalokanena sarveṣāṃ dṛṣṭiḥ sthirā bhavati// \E
%taṃ puruṣaṃ pratisarveṣāṃ dṛṣṭiḥ sthirā bhavati  \P
%taṃ puruṣaṃ vrati sarveṣāṃ dṛṣṭisthirā bhavatī// \B
%taṃ puruṣaṃ pratisarveṣāṃ dṛṣṭisthirā bhavati// \L
%taṃ puruṣaṃ dṛṣṭvā sarveṣāṃ dṛṣṭisthirā bhavati// \N1
%taṃ puruṣaṃ dṛṣṭvā sarveṣāṃ dṛṣṭisthirā bhavati// \D1
%taṃ puruṣaṃ dṛṣṭā       sarveṣāṃ dṛṣṭisthirā bhavati// \N2
%taṃ puruṣaṃ dṛṣṭvā       sarveṣāṃ dṛṣṭisthirā bhavati \U1
%taṃ puruṣaṃ pratisarveṣāṃ dṛṣṭisthirā bhavati// \U2
%-----------------------------
%
%-----------------------------
%idānīṃ śarīramadhye nāḍīnāṃ bhedāḥ kathyante daśa mukhyanāḍyaḥ/ \E
%idānīṃ  śarīramadhye nāḍīnāṃ bhedāḥ kathyaṃte daśamukhyā nāḍyaḥ \P
%idānī  śarīramadhye nāḍībhedaḥ kathyate// daśamukhye nāḍyā \B
%idānī  śarīramadhye nāḍībhedaḥ kathyate// daśamukhyā nāḍayas... \L
%idānīṃ śarīramadhye nāḍīnāmaparo bhedaḥ kathyate// daśa mukhyanādhyaḥ/ \N1
%idānīṃ śarīramadhye nāḍīnāmaparo bhedaḥ// kathyaṃte daśa mukhyānādhyaḥ// \D1
%idānī  śarīramadhye nāḍīnāmaparo bhedāḥ kathyate// daśa mukhyanāḍyaḥ// \N2
%idānīṃ śarīramadhye nāḍīnāmaparo bhedāḥ kathyate daśa mukhyanāḍyas \U1
%idānīṃ śarīramadhye nāḍīnaṃ bhedaḥ kathyate daśa mukhya nāḍyaḥ// \U2
%-----------------------------
%Now the divisions of channels within the body are explained. There are ten channels belonging to the face. 
%-----------------------------
%tanmadhye dvayamijāpiṃgalāsaṃjñakaṃ nāsā dvāre tiṣṭhati/ \E
%tanmadhye nāḍīdvayaṃ idāṃ piṃgalā saṃjñakaṃ nāsādvāre tiṣṭhati  \P
%tanmadhye nāḍīdvayaṃ /idāpiṃgalā saṃjñīkāḥ nāsādvāre tiṣṭhati//  \B
%tanmadhye nāḍīdvayaṃ idāpiṃgalā   saṃjñīkāḥ nāsādvāre tiṣṭhati//  \L
%tanmadhye nāḍīdvayam/ idāpiṃgalā saṃjñakaṃ nāsādvāre tiṣṭhati//  \N1
%tanmadhye nāḍīdvayaṃ  idāpiṃgalā saṃjñakaṃ nāsānāsādvāre tiṣṭhati//  \D1
%tanmadhye nāḍīdvayam/ idānīṃ piṃgalā saṃjñakaṃ nāsādvāre tiṣṭhati//  \N2
%tanmadhye nāḍīdvayaṃ  idāpiṃgalā saṃjñākaṃ nāsādvāre tiṣṭhati  \U1
%tanmadhye nāḍidvayaṃ// idā// piṃgalā// saṃjñākaṃ// nāsādvāre tiṣṭhati//  \U2
%-----------------------------
%
%-----------------------------
%suṣumṇā    tālumārge brahmadvāraparyantaṃ vahati tiṣṭhati/ \E
%suṣumṇā    nālumārgeṇa brahmaraṃdhraparyantavahati tiṣṭhati... \P
%suṣumṇā    tālumārge brahmaraṃdhraparyantaṃ vahatī tiṣṭhati... \B
%suṣumṇā    tālumārge brahmaraṃdhraparyantaṃ vahati tiṣṭhati... \L
%suṣumṇā tu tālumārgeṇa brahmadvāraparyantaṃ vahatī tiṣṭhati... \N1
%suṣumṇā tu tālumārgeṇa brahmadvāraparyantaṃ vahatī tiṣṭhati    \D1
%suṣumṇā tu tālumārge brahmadvāraparyantaṃ vahatī tiṣṭhati// \N2
%suṣumṇā tu tālumārgeṇa brahmadvāraparyantaṃ vahati tiṣṭhati \U1
%suṣumṇā    tālumārgeṇa brahmadvāraparyantaṃ vahati// \U2
%-----------------------------
%The wind is located at the end of the door of Brahma by the path of the palate.? 
%-----------------------------
%        sarasvatī mukhamadhye tiṣṭhati/ \E
%        sarasvatī mukhamadhye tiṣṭhati  \P
%        sarasvatī mukhamadhye tiṣṭhatī/ \B
%        sarasvatī mukhamadhye tiṣṭhati/ \L
%        sarasvatī mukhamadhye varttate/ \N1
%        sarasvatī mukhamadhye varttate// \D1
%        sarasvatī mukhamadhye varttate/ \N2
%        sarasvatī mukhamadhye varttate \U1
%ti sraḥ sarasvati mukhamadhye tiṣṭhati// \U2
%-----------------------------
%
%-----------------------------
%gāṃdhārī hyasti jihvākarṇayor madhye vahalyau tiṣṭhataḥ/    \E
%gāṃdhārī hastinī jihve karṇayor madhye vahatyau tiṣṭhataḥ   \P
%gāṃdhārī hastī jihvē karṇavahatyo tiṣṭhati//                \B
%gāṃdhārī hasti jihvē karṇavahatyo tiṣṭhati...               \L
%gāṃdhārī hastinī jihvē karṇayor madhye vahatyau tiṣṭhataḥ// \N1
%gāṃdhārī hastinī jihvē karṇayor madhye vahatyau tiṣṭhataḥ// \D1
%gāṃdhārī hastinī jihvē karṇayor madhye vahatyau tiṣṭhataḥ// \N2
%gāṃdhādī harratī jihvakarṇayormadhye tiṣṭhataḥ              \U1
%gāṃdhārī// hastinī// jihvē// netrayor madhye vahaṃtyaḥ//    \U2
%-----------------------------
%
%-----------------------------
%pūṣā lambusemā netrayormadhyervahalyā tiṣṭhataḥ/ \E
%pūṣā laṃbuse   netrayor madhye vahatyau tiṣṭataḥ \P
%pūṣo ḍalabuṣenetramadhye vahatyo tiṣṭhati/ \B
%pūṣo  ulabuso netramadhye vahatyo tiṣṭhaṃti// \L
%pūṣāṃ alaṃbuṣenetrayor madhye vahatyo tiṣṭhataḥ/ \N1
%pūṣāṃ alaṃbuṣenetrayor madhye vahatyau tiṣṭhataḥ// \D1
%pūṣāṃ alaṃbuṣenetayor madhye vahatyo tiṣṭhataḥ/ \N2
%pūṣā alaṃbuṣenetayor madhye vahatyau tiṣṭhataḥ \U1
%pūṣāya śakhinī// karṇayor madhye vahatyo tiṣṭhata// alaṃbuṣā// bhu?madhye vaṃhatyo tiṣṭhati// \U2
%-----------------------------
%
%-----------------------------
%śaṃkhinī liṃgadvārād ārabhyeḍāmārgeṇa brahmasthānaparyaṃtaṃ tiṣṭhatīti/     \E
%śaṃkhinī liṃgadvārād ārabhya iḍāmārgeṇa brahmasthānaparyaṃtaṃ tiṣṭhati      \P   %%%%%%%7648.jpg
%śaṃkhinī liṃgadvārād ārabhya iḍāmārgeṇa brahmasthānaparyaṃtaṃ tiṣṭhati/     \B
%śaṃkhinī liṃgadvārād ārabhya iḍāmārgeṇa brahmasthānaparyaṃtaṃ tiṣṭhati//    \L 
%śaṃkhanī liṃgadvārād ārabhya iḍāmārgeṇa brahmasthānaparyaṃtaṃ tiṣṭhati/     \N1
%śaṃkhinī liṃgadvārād ārabhya iḍāmārgeṇa brahmasthānaparyaṃtaṃ tiṣṭhati//     \D1
%śaṃkhinī liṃgadvārād ārabhya iḍānīṃ mārgeṇa brahmasthānaparyaṃtaṃ tiṣṭhati/ \N2
%śaṃkhinī liṃgadvārārabhya    iḍāmārgeṇa brahmasthānaparyaṃtaṃ tiṣṭhati      \U1
%kuhū     liṃgadvārād ārabhya iḍāmārgeṇa brahmasthānaparyaṃtaṃ tiṣṭhati// śāṃkhinī mūladvārād arabhya piṃgalamargeṇa brahmasthānaparyaṃtaṃ tiṣṭhati// \U2
%-----------------------------
%
%-----------------------------
%etādṛśanāḍyo  daśasu dvāreṣu tiṣṭhanti/    \E
%etādṛṣānāḍyo daśasu dvāreṣu tiṣṭhaṃti      \P
%etādṛṣyānāḍyo daśasu dvāreṣu tiṣṭhaṃti/    \B
%etādṛṣyānāḍyo daśadvāreṣu    tiṣṭhaṃti/    \L 5876_15.jpg
%etādaśanāḍyo  daśasu dvāreṣu tiṣṭhaṃti/    \N1
%etādaśanāḍyo  daśasu dvāreṣu tiṣṭhaṃti//   \D1
%etādaśasudvāreṣu tiṣṭhaṃti/                \N2
%etādṛśa nāḍyo daśa svadhāreṣu  tiṣṭhati    \U1
%etādaśa nāḍyo daśaśoṣu dvāreṣu tiṣṭhaṃti// \U2 %%%413.jpg
%-----------------------------
%
%-----------------------------
%anyā dvisaptatisahasraparimitā nāḍayo lomnāṃ mūleṣu sūkṣmarūpeṇa tiṣṭanti// \E
%anyā dvisaptatisahasraparimitā nāḍyo lomnāmūleṣu sūkṣmarūpeṇa tiṣṭaṃti      \P
%anyā dvisaptatīsahasraparimitā nāḍyo lomnāmūleṣu sūkṣmarūpeṇa tiṣṭaṃti// \B
%anyā dvisaptatisahasraparimitā nāḍyo lomnāmūleṣu sūkṣmarūpeṇa tiṣṭaṃti// \L
%anyā dvisaptatisahasraparamitā nāḍyā lomnāṃ mūleṣu sūkṣmarūpeṇa tiṣṭaṃti// \N1
%anyā dvisaptatisahasraparamitā nāḍyā lomnāṃ mūleṣu sūkṣmarūpeṇa tiṣṭaṃti// \D1
%anyā dvisaptatrisahasraparimitā nāḍyā lomnāṃ mūleṣu sūkṣmarūpeṇa tiṣṭaṃti// \N2
%anyā dvisaptatisahasraparimitāgryo lomnā mūleṣu sūkṣmarūpeṇa tiṣṭaṃti \U1
%anyā hidaśonā dvisatyati sahasraḥ//71110// parimitā nādhyo lomnāṃ mūleṣu sūkṣmarūpeṇa tiṣṭaṃti// \U2
%-----------------------------
%
%-----------------------------
%[p.36]
%idānīṃ śarīramadhye vāyavo daśa tiṣṭhanti/ \E
%idānīṃ śarīramadhye vāyavo daśa tiṣṭhaṃti  \P
%idānīṃ śarīramadhye .....\om               \B
%idānīṃ śarīramadhye .....\om               \L
%idānīṃ śarīramadhye vāyavas tiṣṭhaṃti/     \N1
%idānīṃ śarīramadhye vāyavas tiṣṭhaṃti//    \D1
%idānīṃ śarīramadhye vāyavas tiṣṭhaṃti/     \N2
%idānīṃ śarīramadhye vāyavas tiṣṭhaṃti      \U1
%idānīṃ śarīramadhye vāyavo daśa ṣṭaṃti//   \U2
%-----------------------------
%
%-----------------------------
%teṣāṃ nāmāni kāryāṇi kathyante/ \E
%teṣāṃ nāmāni kārmāṇi kathyante/ \P
%\om \B
%\om \L
%teṣāṃ kāryāṇi kathyante/ \N1
%teṣāṃ kāryāṇi kathyaṃte/ \D1
%teṣāṃ kāryāṇi kathyate/ \N2
%teṣāṃ kāryāṇi kathyate \U1 %%%280.jpg
%teṣāṃ kāryāṇi kathyate \U2
%-----------------------------
%their obligations are taught. 
%-----------------------------
%prāṇavāyurtdṛdayamadhye śvāsocchāsaṃ karoti/ \E
%prāṇavāyur hṛdayamadhye śvāsochāsaṃ karoti       \P
%------------------------śvāsośvaroti/               \B
%                        śvāsośvareti...             \L 
%                        prāṇavāyuhṛdayamadhye utsvāsaprasvāsasaṃ karoti//   \N1
%prāṇavāyuhṛdayamadhye utsvāsaprasvāsaṃ karotī   \D1
%prāṇavāyuhṛdayamadhye ūrdhvaśvāsapraśvāsaṃ karoti// \N2
%prāṇavāyuhṛdayamadhye ūdhvasaprasase karoti \U1
%prāṇavāyuhṛdayamadhye  svāsochvāsaṃ karoti \U2
%-----------------------------
% The Prāṇa vitalwind is located in the middle of the heart and causes inhalation and exhalation. 
%-----------------------------
%aśanapānecchā bhavati/   gudamadhye samāno vāyur vartate/                                                             sapta samagrā nāḍīḥ śoṣayati/  \E
%aśanapānechā bhavati     gudamadhye 'pānāvāyus tiṣṭhati   sa āṃkucanastaṃbhanaṃ karoti    nābhīmadhye samāno varttate   sa sama?grā  nāḍīḥ  śoṣayati   \P
%aśanapānechā  bhavati//   gudamadhye appānāvāyor tiṣṭhatī sa āṃkucanastaṃbhanaṃ karotī/   nābhīmadhye smānā  vartate/   sa samagrā  nāḍī    śoṣayati// \B
%aśanapāne cha bhavati//  gudamadhye apānāvāyo tiṣṭhati    sa āṃkucanastaṃbhanaṃ karotī/   nābhīmadhye samānā vartate   sa samagrā  nāḍī    śoṣayatī// \L
%asitapittecha bhavati/   guḍamadhye apānavāyu tiṣṭhati    sa ākuṃcanasthaṃbhanaṃ karoti// nābhimadhye samāno varttate/ sa samāgraṃ nādhyaṃ śoṣayati/  \N2
%aśitapiteccha bhavati/   gudamadhye apānavāyu tiṣthati/   sa ākuṃcanaṃ staṃbhanaṃ karoti/ nābhimadhye samāno varttate/ sa samagraṃ nādhyaṃ śoṣayati//  \N1
%aśitapiteccha bhavati//   gudamadhye apānavāyus tiṣthati/ sa ākuṃcanaṃ staṃbhanaṃ karoti// nābhimadhye samāno varttate// sa samagraṃ nādhyaṃ śoṣayati// \D1
%asīte pitechā bhavati    gudamadhye apānavāyu tiṣthati    sa ākuṃcanaṃ staṃbhanaṃ karoti  nābhimadhye samāno varttate  sa samagrāṃ nāḍīṃ śoṣayati \U1
%aśanapānechā  bhavati//  gudamadhye apānāvāyo tiṣṭhati//   āṃkucanastabhanaṃ karoti/    nābhipadmamadhye samāno vartate// sa samagrā nāḍī śoṣayati \U2
%-----------------------------
%
%-----------------------------
%tathā nāḍīśoṣaṇāt rucimutpādayati/ vahniṃ dīpayati/ \E
%tathā nāḍīḥ  pośayati rucim utpādayati vahniṃ dīpayatī \P
%tathā   pośayatī/ tathā poṣayatī// rucirutpādayatī vahnī dīpayatī/ \B
%tathā pośayatī                     rucim utpādayatī vahnī dīpayatī... \L
%tathā nāḍīṃ pośayati/ kvacit-utpādayati/ āgniṃ dīpayati \N1
%tathā nāḍīṃ pośayati// kvacit-utpādayati// āgniṃ dīpayati \D1 %%%p. 11 recto
%tathā nāḍīṃ pośayati/ kvacit-utpādayati/ āgniṃ dīpayati \N2
%tathā nāḍīṃ pośa iti  rucim utpādayati agnīṃ dīpayati \U1
%            ṣoṣayati  rucim utpādayati// vahniṃ dīpayati// \U2
%-----------------------------
%
%-----------------------------
%tālumadhye udāno vāyus-tiṣṭhati/ sa vāyuḥ ratnaṃ līlati/ pānīyaṃ pibati/ nāgavāyuḥ sarvaśarīre varttate/ tasmād-vāyoḥ śarīraṃ cālayati/ śokamāpnoti// vivilaḥ \E
%tālumadhye udāno vāyus-tiṣṭhati  sa vāyu  ratnaṃ gilati  pānīyaṃ pībati  nāgavāyuḥ sakale śarīre varttate tasmād-vāyo śarīraṃ calayati śophamāpnoti vikṛtaḥ  \P %%%7649.jpg
%tālumadhye udānavāyus-tiṣṭhati/ sa  vāyur annaṃ galayatī/ pānīyaṃ pibatī/ nāgavāyuḥ sakalaśarīre varttate tasmād-vāyoḥ// śarīre cālatī/ śokamāpnoti vi??kru??taḥ// \B DSCN7163.JPG Z.11
%tālumadhye udānavāyus tiṣṭhati// sa vāyur annaṃ galayati// pānīyaṃ pibatī// nāgavāyusakalaśarīre vartate tasmād vāyoḥ// śarīre cālayatī śokamāpnoti vikutaḥ... \L
%tālumadhye udānavāyus-tiṣṭhati/ sa  vāyuḥ ratnaṃ śilati/ pānīyaṃ pibati/ nāgavāyuḥ sakale śarīre varttate// tasmād-vāyoḥ śarīraṃ calati/                 \N1
%tālumadhye udāno vāyus-tiṣṭhati// sa vāyur annaṃ gilati/ pānīyaṃ pibati nānāgavāyuḥ sakale śarīre varttate// tasmād-vāyoḥ śarīraṃ calati//               \D1
%tālumadhye udānāni vāyus-tiṣṭhati/ sa vāyur-annaṃ gīlati/ pānīyaṃ pibati/ nāgavāyuḥ sakale śarīre varttate// tasmād-vāyoḥ śarīraṃ calati/                \N2
%tālumadhye udānavāyus-tiṣṭhati  sa vāyur-annaṃ gilati pānīyaṃ pibati nāgavāyu sakale śarīre varttate  tasmād-vāyoḥ śarīraṃ calati                        \U1
%tālumadhye udānovāyus-tiṣṭhati// sa vāyur annaṃ gilati// pānīyaṃ pibati// nāgavāyuḥ sakale śarīre varttate// tasmād-vāyoḥ śarīraṃ calayati śokam āpnoti vikṛtaḥ// \U2
%-----------------------------
%
%-----------------------------
%kūrmavāyur netramadhye tiṣṭhati/ nimeṣonmeṣaṃ karoti/ \E
%kūrmavāyur netramadhye nimeṣonmeṣaṃ karoti \P
%kūrmavāyoḥ netramadhye nimeṣonmeṣaṃ karotī/ \B
%kūrmavāyoḥ netramadhye nimiṣonmeṣaṃ karotī... \L
%kūrmo vāyunetramadhye tiṣṭhati/ unmeṣaṃ nimeṣaṃ karoti/ \N1
%kūrmo vāyunetramadhye tiṣṭhati/ unmeṣaṃ nimeṣaṃ ca karoti// \D1
%kūrmo vāyunetramadhye tiṣṭhati/ unmeṣaṃ nimeṣaṃ karoti/ \N2
%\om                                                     \U1
%kūrmavāyur netramadhye nimiṣonmeṣaṃ karoti//            \U2
%-----------------------------
%
%-----------------------------
%kṛkalakartā vāyurudgāraṃ karoti devadattavāyoḥ  jṛmbhaṇaṃ bhavati/ dhanaṃjayavāyoḥ śabda utpadyate// \E
%kṛkalavāyur udhāraṃ karoti      devadattavāyor  jumbhā bhavati dhanaṃjayavāyo  śabdāḥ utpadyete  \P
%kṛkalavāyur udhāraṃ karotī      devadattavāyor  jumbhābhavaṃtī dhanaṃjayavāyoḥ śabda utpadyate// \B
%kṛkalavāyur uhāraṃ karotī       devadattavāyor  jṛṃbhā bhavatī dhanaṃjayavāyoḥ śabdaḥ utpadyate// \L
%kṛkalavāyor-ūdgāro bhavati//    devadattavāyor  jṛṃbha utpadyate// dhanaṃjayavāyo śabda utpadyate// \N1
%kṛkalavāyor-ūdgāto bhavati/     devadattavāyor  jṛṃbha utpadyate// dhanaṃjayavāyo śabda utpadyate// \D1
%kṛkaravāyor-ūdgāro bhavati/     devadattavāyo   jṛṃbhotpadyate/ dhanaṃjayavāyo śabdotpadyate// \N2
%                                devadattavāyor  jaṃbhā utpadyate dhanaṃjayavāyoḥ sabta utpadyate \U1
%puṣkaravāyur udgāraṃ karoti//   devadattavāyo   jṛṃbhā bhavati// dhanaṃjayavāyoḥ śabda utpadyate// \U2
%-----------------------------
%
%-----------------------------
%\om                               \E
%idānī  madhyalakṣaṃ kathyate      \P
%idānīṃ madhyalakṣaṇaṃ kathyate//  \B DSCN7164 Z.1
%idānīṃ madhye lakṣaṃ kathyate//   \L
%idānīṃ madhyalakṣyaṃ kathyate//   \N1
%idānīṃ madhyalakṣyaṃ kathyate//   \D1
%idānīṃ madhyalakṣaṇaṃ kathyate//  \N2
%idānīṃ madhyalakṣyaṃ kathyate     \U1
%idānīṃ madhye lakṣyaṃ kathyate//  \U2
%-----------------------------
%
%-----------------------------
%aṃtha ca pītavarṇaṃ raktavarṇaṃ vā dhūmrākāraṃ yannīlavarṇaṃ vā agniśikhāsadṛśaṃ vidyutsamānaṃ sūryamaṇḍalasadṛśaṃ arddhacandrasadṛśaṃ jvaladākāśasamākāraṃ svaśarīraparimitaṃ tejomanomadhye tathyaṃ kartavyam// \E
%śvetavarṇaṃ atha pītavarṇaṃ  raktaṃ vā dhūmrākāraṃ yannīlavarṇaṃ vā 'gniśikhāsadṛśaṃ vidyutsamānaṃ  sūryamaṇdalasadṛśaṃ  arddhacaṃdrasadṛśaṃ jvaladākāśasamākāraṃ   svaśam ākāśaṃ svaśarīraparimitaṃ tejomanomadhye lakṣyaṃ karttavyaṃ\P
%śvetavaraṃ atha pītavarṇaṃ// rakta vā dhūmrākāraṃ yannīlavarṇaṃ vā// agniśikhāsadṛśaṃ vidyutsamānaṃ sūryamaṇdalasadṛśaṃ/ ūrdhvacaṃdrasadṛśaṃ jvaladākāśasamākāraṃ// svaśarīraparimitaṃ tejomanomadhye lakṣaṃ kartavyaṃ//  \B
%śvetavarṇaṃ atha pītavarṇaṃ  raktaṃ vā dhūmrākāraṃ yannīlavarṇaṃ vā// agniśikhāsadṛśaṃ vidyutsamāne sūryamaṇdalasadṛśaṃ// ardhacaṃdrasadṛśaṃ jvaladākāśasamākāra    svaśarīraparimitaṃ tejomanomadhye lakṣaṃ kartavyaṃ//  \L
%śvetavarṇā/ athavā pītavarṇaṃ raktaṃ vā dhūmārava nīlavarṇaṃ vā agniśikhāsadṛśaṃ vidyutsamānaṃ sūryamaṇdalaṃ sadṛśaṃ/ ūrdhvacaṃdrasadṛśaṃ jvaladākāśasamānakāraṃ// svaśarīraparimitaṃ tejomanomadhye lakṣyaṃ karttavyaṃ//  \N1
%śvetavarṇaṃ// athavā pītavarṇaṃ raktaṃ vā dhūmākāro vā nīlavarṇaṃ vā agniśikhāsadṛśaṃ vidyutsamānaṃ// sūryamaṇdalaṃ sadṛśaṃ// ūrdhvacaṃdrasadṛśaṃ jvalad ākāśasamānakāraṃ// svaśarīraparimitaṃ tejomanomadhye lakṣyaṃ karttavyaṃ// \D1
%śvetavarṇā athavā pītavarṇa raktavarṇa dhūmravarṇa nīlavarṇaṃ vā agniśikhāsadṛśaṃ vidyutsamānaṃ sūryamaṇdalasadṛśaṃ ūrdhvacaṃdrasadṛśaṃ jvaladākāśasamānakāraṃ// svaśarīraparimitaṃ tejomanomadhye lakṣaṇaṃ karttavyaṃ//  \N2
%svetavarṇaṃ atha vā pītavarṇaṃ raktaṃ vā dhūmrākāra vannīlavarṇaṃ vā agniśikhāsadṛśaṃ vidyutsamānaṃ sūryamaṇdalasadṛśaṃ ārdhacaṃdrasadṛśaṃ jaladāsamānākāraṃ svaśarīraparimanomittaṃ tejomadhye lakṣyaṃ karttavyaṃ  \U1
%svatavarṇaṃ atha vā pītavarṇaṃ// raktaṃ vā dhūmrākāraṃ yan nīlavarṇaṃ vā agniśikhāsadṛśaṃ vidyutsamānaṃ sūryamaṇdalasadṛśaṃ arddhacaṃdrasadṛśaṃ jvaladākāraṃsamākāraṃ svaśarīraparimitaṃ tejo manomadhye lakṣaṃ kartavyaṃ//  \U2
%-----------------------------
%
%-----------------------------
%ekasmin lakṣye kṛte sati manomadhye sthitasya malasya[p.38]%dāho bhavati/ \E
%etasmillakṣye kṛte sati manomadhye sthitasya dāho bhavati \P
%etasmin lakṣe kṛte satī manomadhye sthitasya malasya dāho bhavati \B
%etasmillakṣe kṛte satī manomadhye sthitasya malasya dāho bhavati... \L
%ekasmin lakṣye kṛte sati manomadhye sthitasya malasya dāho bhavati/ \N1
%ekasmin lakṣye kṛte sati manomadhye sthitasya malasya dāho bhavati// \D1
%ekasmin lakṣaṇo kṛte sati manomadhye sthitasya malasya dāho bhavati/ \N2
%etasmin na lakṣye kṛte satī manomadhye sthitasya malasya dāho bhavati \U1 %%%281.jpg
%etasmil lakṣe kṛte satī manomadhye sthitasya malasya dāho bhavati// \U2
%
%-----------------------------
%
%-----------------------------
%manasaḥ   sattvaguṇaprakāśo bhavati/ puruṣa ānandamayo bhūtvā tiṣṭhati//   \E 
%manasaḥ   sattvaguṇaḥ prakaṭo bhavati puruṣa ānandamayo bhūtvā tiṣṭhati    \P   %%%7650.jpg
%manasaḥ// sattvaguṇo prakaṭo bhavati// puruṣa ānandamayo bhūtvā tiṣṭhati// \B
%manasaḥ// sattvaguṇaḥ prakaṭo bhavati puruṣa ānandamayo bhūtvā tiṣṭhati//  \L
%manasaḥ sattvaguṇe prakaṭo bhavati/ puruṣa ānandamayo bhūtvā tiṣṭhati//    \N1
%manaḥ saḥ sattvaguṇo prakaṭo bhavati// puruṣa ānandamayo bhūtvā tiṣṭhati// \D1
%manasaḥ sattvaguṇo prakaṭo bhavati/ puruṣa ānandamayo bhūtvā tiṣṭhati//    \N2
%manasaḥ sattvaguṇo prakaṭo bhavati puruṣa ānandamayo bhūtvā tiṣṭhati       \U1
%manasaḥ satvaguṇa prakaśo bhavati// puruṣa ānandamayo bhūtvā tiṣṭhati//    \U2 %%414.jpg
%-----------------------------
%
%-----------------------------
%idānīm-ākāśabhedāḥ kathyante/ \E
%idānīmākaśabhedāḥ kathyaṃte   \P
%idānīṃ ākaśabhedāḥ kathyaṃte/ \B
%idānīṃ ākaśabhedāḥ kathyate/  \L
%idānīṃ ākaśabhedāḥ kathyaṃte/ \N1
%idānīṃ ākaśabhedāḥ kathyaṃte// \D1
%idānīṃ ākāśabhedāḥ kathyate/  \N2
%idānīṃ ākāśabhedāḥ kathyaṃte  \U1
%idānīm ākāśabhedāḥ kathyate// \U2
%-----------------------------
%
%-----------------------------
%te ākāśaḥ paramākāśaḥ mahākāśaḥ tattvākāśaḥ sūryākāśaḥ/ bāhyābhyantare nirmalaṃ nirākāramākāśalakṣyaṃ karttavyam/ \E
%teṣāṃ lakṣyāni ca kathyaṃte ākāśaḥ parākāśaḥ mahākāśaḥ tatvākāśaḥ sūryakāśaḥ  bāhyābhyaṃtare nirmalaṃ nirākāramākāśaṃ lakṣyaṃ karttavyaṃ  \P
%ākāśaḥ paramākāśaḥ// mahākāśa// tattvākāśaḥ sūryākāśa// bāhyābhyaṃtaro nirmalaṃ nirākāramākāśaṃ lakṣaṃ kartavyaṃ// \B
%ākāśaḥ paramākāśaḥ// mahākāśaḥ tattvākāśaḥ sūryākāśaḥ  bāhyābhyaṃtare nirmalaṃ nirākāramākāśaṃ lakṣaṃ kartavyaṃ// \L
%teṣāṃ lakṣyāni kathyate//  ākāśa,parākāśa,mahākāśa,tatvākāśa,sūryakāśa// bāhyābhyaṃtare nirmalaṃ nirākāraṃ ākāśalakṣyaṃ kartavyaṃ// \N1
%teṣāṃ lakṣyāṇi kathyaṃte//  ākāśaparākāśamahākāśatatvākāśasūryakāśa bāhyābhyaṃtare nirmalaṃ nirākāraṃ ākāśalakṣyaṃ karttavyaṃ// \D1   %%%p.11 verso
%teṣāṃ lakṣaṇāni kathyate// ākāśaparākāśamahākāśatatvākāśasūryakāśaḥ bāhyābhyaṃtare nirmalaṃ nirākāraṃ ākāśalakṣaṇaṃ kartavyaṃ// \N2
%ṣāṃ lakṣyāṇi kathyaṃte ākāśaparākāśamahākāśatatvākāśasūryakāśabāhyābhyaṃtare nirmalaṃ nirākāraṃ ākāśalakṣyaṃ karttavyaṃ  \U1
%teṣāṃ lakṣyāni kathyaṃte// ākāśaḥ parākāśa// mahākāśaḥ// tatvākāśaḥ// sūryakāśaḥ// bāhyābhyaṃtare nirmalaṃ nirākāraṃ mākāśaṃ lakṣyaṃ karttavyaṃ// \U2
%
%-----------------------------
%
%-----------------------------
%tataḥ paraṃ bāhyābhyantareṣvanandhakārasadṛśaṃ parākāśaikyaṃ lakṣyaṃ karttavyam// \E
%tataḥ paraṃ bāhyābhyantarai ghanāṃ dhakāraṃ sadṛśaparākāśasya lakṣyaṃ karttavyam \P
%tataḥ paraṃ bāhyābhyaṃtare ghanāṃ ghakārasadṛśaḥ parākāśalakṣaṃ kartavyaṃ// \B
%tataḥ paraṃ bāhyābhyaṃtare        dhakārasadṛśaḥ parākāśalakṣaṃ kartavyaṃ... \L
%tataḥ paraṃ bāhyābhyantare ghanāṃ dhakārasadṛśaparākāśasya lakṣyaṃ kattavyam// \N1
%tataḥ paraṃ bāhyābhyantare ghanāṃ dhakārasadṛśaparākāśasya lakṣyaṃ kattavyaṃ// \D1
%tataḥ paraṃ bāhyābhyantare ghanāṃ dhakārasadṛśaparākāśasya lakṣaṇaṃ karttavyam// \N2
%tataḥ paraṃ bāhyābhyantare ghanāṃ dhakārasadṛśaparākāśasya lakṣyaṃ karttavyaṃ \U1
%tataḥ       bāhyābhyantare ghanāṃ dhakārasadṛśaṃ parākāśasya lakṣaṃ karttavyaṃ// \U2
%-----------------------------
%
%-----------------------------
%tataḥ paraṃ pralayakālīnajvaladdāvānalapūrṇaṃ bāhyābhyantare, mahākāśalakṣyaṃ karttavyam/ \E
%tataḥ paraṃ pralayakālīnajaladvaḍavānalapūrṇaṃ bāhyābhyaṃtare mahākāśaṃ lakṣyaṃ karttavyaṃ \P
%tataḥ paraṃ pralayakālīnaḥ jaladvaḍavānalapūrṇaṃ bāhyābhyaṃtare mahākāśalakṣaṃ kartavyaṃ// \B
%tataḥ paraṃ pralayakālīnaḥ jvaladvaḍavānalapūrṇaṃ bāhyābhyaṃtare mahākāśalakṣaṃ kartavyaṃ// \L
%tataḥ paraṃ pralayakālīnajvaladvṛddha?[S.9 verso letzte Zeile] nalapūrṇa bāhyābhyaṃtare mahākāśalakṣyaṃ karttavyaṃ// \N1
%tataḥ paraṃ pralayakālīnajvaladdāvānalapūrṇaṃ bāhyābhyaṃtare mahākāśaṃ lakṣaṃ karttavyaṃ// \D1
%tataḥ paraṃ pralayakālīnajvaladvṛ? nalapūrṇa  bāhyābhyaṃtare  mahākāśalakṣaṃ karttavyaṃ// \N2
%tataḥ paraṃ pralayakālītajjalavaḍavānalapūrṇaṃ bāhyābhyaṃtare mahākāśaṃ lakṣyaṃ kartavyaṃ \U1
%tataḥ       pralayakālīnajvaladvaḍavānalapūrṇabāhyābhyaṃtare ghanāṃ dhakārasadṛśaṃ mahākāśasya lakṣaṃ karttavyaṃ \U2
%-----------------------------
%
%-----------------------------
%\om                                                                                                                  tataḥ bāhyābhyantare prakāśamānayarsūsahitaṃ sūryākāśaṃ lakṣyaṃ[p.39]karttavyam/ \E
%tataḥ paraṃ bāhyābhyaṃtare koṭidīpānāṃ prakāśaprāptau  yādṛśamau jvalyaṃ bhavati tādṛśaṃ tatvākāśaṃ lakṣyaṃ karttavyaṃ  tataḥ paścādbāhyābhyaṃtare prakāśamāgasūryaṃ biṃbasahitaṃ sūryākāśalakṣyaṃ karttavyaṃ ... \P
%tataḥ paraṃ bāhyābhyaṃtare koṭidīpānāṃ prakāśaprāpto   yādṛśamau jvalaṃ bhavatī/ tādṛśaṃ tatvāśalakṣaṃ kartavyaṃ// paccā bāhyābhyaṃtare prakāśamān sūryabiṃbasahitasūryakāśalakṣaṃ kartavyaṃ mataḥ ... \B
%tataḥ paraṃ bāhyābhyaṃtare koṭidīpānāṃ prakāśaprāpto   yādṛśamu  jvalaṃ bhavatī/ tādṛśaṃ tatvāśalakṣaṃ kartavyaṃ paccā bāhyābhyaṃtare prakāśamān sūryabiṃbasahitasūryakāśalakṣaṃ kartavyaṃ mataḥ ... \L 
%tataḥ paraṃ bāhyābhyaṃtare koṭidīpānāṃ prakāśaprāptau  yādṛśamau jvalyaṃ bhavati/ tādṛśaṃ tatvākāśaṃ lakṣyaṃ kartavyaṃ// tataḥ paścāt bāhyābhyaṃtare prakāśamānasūryabiṃbasahitaṃ sūryakāśaṃ lakṣyaṃ karttavyaṃ// \N1
%tataḥ paraṃ bāhyābhyaṃtare koṭidīpānāṃ prakāśaprāptau  yādṛśamau jvalyaṃ bhavati// tādṛśaṃ tatvākāśaṃ lakṣaṃ kartavyaṃ// tataḥ paścāt bāhyābhyaṃtare prakāśamānasūryabiṃbasahitaṃ sūryakāśaṃ lakṣyaṃ karttavyaṃ// \D1
%tataḥ paraṃ bāhyābhyaṃtare koṭidīpānāṃ prakāśaprāptau  yādṛśamau jvala bhavati/ tādṛśaṃ tatvākāśaṃ lakṣaṃ kartavyaṃ// tataḥ paścādābhyaṃtare prakāśamānasūryabiṃbasahitaṃ sūryakāśaṃ lakṣaṃ karttavyaṃ// \N2
%tataḥ paraṃ bāhyābhyaṃtare koṭidīpānāṃ prakāśaprāptau  yādṛśamau jvalaṃ bhavati tādṛśaṃ tatvākāśaṃ lakṣyaṃ kartavyaṃ tataḥ paścāt bāhyabhyaṃttare prakāśamānasūryabiṃbasāhitaṃ sūryakāśaṃ lakṣyaṃ karttavyaṃ \U1
%tataḥ paraṃ bāhyābhyaṃtare koṭidīpānāṃ prakāśaprāptau  yādṛśemau jvalyaṃ bhavati tādṛśaṃ// tatvākāśaṃ lakṣaṃ karttavyaṃ// tataḥ paścād bāhyābhyaṃtare prakāśamānasūryabiṃbasāhitaṃ sūryākāśaṃ lakṣyaṃ karttavyaṃ// \U2
%-----------------------------
%
%-----------------------------
%eteṣāṃ lakṣyāṇāṃ kāraṇāt śarīraṃ rogāsaṃsargi bhavati// \E
%eteṣāṃ lakṣāṇāṃ karaṇāt  śarīre  rogasaṃsargo na bhavati \P %%%7651.jpg
%eteṣāṃ lakṣaṇaṃ karaṇāt// śarīre rogasaṃsargo na bhavatī/ \B
%eteṣāṃ lakṣaṃ karaṇāt śarīre rogasaṃsargo na bhavati... \L
%eteṣāṃ lakṣyaṇāṃ karaṇāt śarīrarohasaṃsarge na bhavati/ \N1
%eteṣāṃ lakṣyāṇāṃ karaṇāt śarīrarohasaṃsargo na bhavati// \D1
%eteṣāṃ lakṣāṇākāraṇāccharīrarogāsaṃsargo na bhavati// \N2
%eteṣāṃ lakṣyāṇāṃ karaṇāt śarīrarogāsaṃsargo na bhavati \U1
%eteṣāṃ lakṣyāṇāṃ karaṇāt// śarīre rogāsaṃsargo na bhavati \U2
%-----------------------------
%
%-----------------------------
%tathā valitapalitaṃ   puṇyaṃ pāpaṃ na bhavati//    \E
%tathā valitapalitaṃ   puṇyāṃ pāpaṃ ca na bhavati   \P
%tathā// valitapalitaṃ puṇyāṃ pāpaṃ ca na bhavatī// \B
%tathā valitaṃ palitaṃ puṇyāṃ pāpaṃ ca na bhavatī// \L
%tathā valitaṃ palitaṃ puṇyaṃ pāpaṃ ca na bhavati// \N1
%tathā valitaṃ palitaṃ puṇyaṃ pāpaṃ ca na bhavati// \D1
%tathā valitaṃ palitaṃ puṇyaṃ pāpaṃ ca na bhavati// \N2
%tathā valitaṃ palitaṃ puṇyaṃ pāpaṃ ca na bhati \U1
%tathā valīpalitaṃ puṇyaṃ pāpaṃ ca na bhavati \U2
%-----------------------------
%
%-----------------------------
%navacakraṃ kalādhāraṃ trilakṣyaṃ vyomapaṃcakam/ \E
%navacakraṃ kalādhāraṃ trilakṣyaṃ vyomapaṃcakaṃ  \P
%śloka navacakraṃ kalādhāraṃ trilakṣaṃ vyomapaṃcakam/ \B
%//śloka// navacakraṃ kalādhāraṃ trilakṣaṃ vyomapaṃcakam... \L %%%%%%%%%%%%GREP THIS%%%%%%%%%%%%% SSP 2.31!!!
%navacakrakalādhāraṃ trilakṣyaṃ vyomapaṃcakaṃ/ \N1
%navacakrakalādhāraṃ trilakṣyaṃ vyomapaṃcakaṃ// \D1
%navacakrakalādhāraṃ trilakṣaṃ vyomapaṃcakaṃ/ \N2
%navacakraṃ kalādhāraṃ trilakṣyaṃ vyomapaṃcakaṃ \U1 %%%282.jpg
%navacakraṃ kalādhāraṃ trilakṣyaṃ vyomapaṃcakaṃ// \U2
%-----------------------------
%
%-----------------------------
%svadehe yo na jānāti sa yogī nāmadhārakaḥ//       \E
%svadehe yo na jānāti sa yogī nāmadhārakaḥ 1       \P
%svadehe yo na jānāti sa yogī nāmadhārakaḥ//1//    \B
%svadehe yo na jānāti sa yogī nāmadhārakaḥ//1//   \L
%samakriyānajānāti sa yogī nāmadhāraka//           \N1
%samakriyānajānāti sa yogī nāmadhārakaḥ//           \D1
%samakriyānajānāti sa yogī nāmadhāraka//           \N2
%samakriyānajānāti sa yogī nāmadhārakaḥ            \U1
%svadehe yo na jānāti sa yogī nāmadhārakaḥ        \U2
%-----------------------------
%
%-----------------------------
%idānīṃ cakrāṇām anukramaḥ kathyate/ \E
%idānīṃ cakrāṇām anukramaḥ kathyate \P
%idānīṃ cakrāṇām anukramaḥ// \B
%idānīṃ cakrāṇām anukramaḥ// \L 19.jpg 
%idānīṃ cakrāṇām anukrama  kathyaṃte/ \N1
%idānīṃ cakrāṇām anukramā  kathyaṃte// \D1
%idānīṃ cakrānām-anukramā  kathyaṃte/ \N2
%idānīṃ cakrānāmanukramaḥ  kathyate \U1
%idānīṃ cakrānām anukramaḥ  kathyate// \U2
%-----------------------------
%Now the practice of the cakras is explained. 
%-----------------------------
%ādhāre brahmacakram/ ādhāropari liṃgamūle sbādhiṣṭhānacakram/ nābhau maṇipūrakacakram/ hṛdaye anāhatacakram/ kaṇṭhasthāne viśuddhicakram/ ṣaṣṭhaṃ tālucakram/ bhruvor madhye ājñācakram/ brahmasthāne kālacakram/ navamamākāśacakram/ etatparaṃ śūnyam/ \E
%
%ādhāre brahmacakraṃ 1 ādhāropari liṃgamūle svādhiṣṭhānacakram 2 nābhau maṇipūrakacakraṃ hṛdaye 'nāhatacakraṃ 4 kaṃṭhasthāne viśuddhicakraṃ 5 ṣaṣṭhaṃ tālucakraṃ 6 bhruvor madhye agnejacakraṃ 7 brahmasthāne kālacakraṃ 8 navamaṃ ākāśacakraṃ 8 tataḥ paraṃ śūnyaṃ \P
%
%ādhāro brahmacakram/ ādhāropari liṃgamūle svādhiṣṭhānacakraṃ//2// nābhau maṇipūrakacakram//3  hṛdaye anāhatacakram// 4 kaṇṭhasthāne viśuddhicakraṃ// ṣaṣṭhaṃ tālucakre/6 bhruvormadhye ājñāyacakraṃ/ brahmasthāne kālacakraṃ// 8 navamaṃ ākāśacakraṃ/9 tatparaṃ śūnyam/ \B
%
%ādhāro brahmacakram// ādhāropari liṃgamūle svādhiṣṭhānacakraṃ//2// nābhau maṇipūrakacakram//3// hṛdaye anāhatacakram//4// kaṇṭhasthāne viśuddhacakraṃ// ṣaṣṭha tālucakre//6// bhruvormadhye āgneyacakraṃ//7// brahmasthāne kālacakraṃ//8// navamaṃ ākāśacakraṃ//9// tatparaṃ śūnyam// \L
%
%ādhāre brahmacakraṃ liṃge svādhiṣṭhānacakram/ nābhau maṇipūrakacakram/ hṛdaye viśuddhacakraṃ/ kaṇṭhasthāne anāhatacakraṃ/ ṣaṣṭhaṃ tālucakram/ bhruvor madhye ājñācakram brahmaraṃdhrasthāne kālacakraṃ/ navamaṃ ākāśacakram/ tatparamaśūnyaṃ/ \N1
%
%ādhāre brahmacakraṃ liṃge svādhiṣṭhānacakram// nābhau maṇipūrakacakraṃ// hṛdaye viśuddhacakraṃ// kaṃṭhasthāne anāhatacakraṃ// ṣaṣṭhaṃ tālucakraṃ// bhruvor madhye ājñācakraṃ// brahmaraṃdhrasthāne kālacakraṃ// navamaṃ ākāśacakram/ tatparaṃ// tatparamaśūnyaṃ// \D1
%
%ādhāre brahmacakraṃ liṃge svādhiṣṭhānacakram// nābhau maṇipūrakacakram/ hṛdaye viśuddhacakraṃ/ kaṇṭhasthāne anāhatacakraṃ ṣaṣṭhaṃ tālucakram/ bhruvor madhye ājñācakram brahmaraṃdhrasthāne kālacakraṃ/ navama ākāśacakram tata paraśūnyaṃ/ \N2
%
%ādhāre brahmacakraṃ liṃge svādhiṣṭhānacakraṃ nābhau maṇipūrakacakraṃ hṛdaye viśuddhacakraṃ kaṇṭhasthāne anāhatacakraṃ ṣaṣṭhaṃ tālucakraṃ bhruvor madhye ājñācakram brahmaraṃdhrasthāne brahmacakraṃ navamaṃ rattu?! ākāśacakram tatparaśūnyaṃ \U1
%
%ādhāre brahmacakraṃ//1//ādhāro pariliṃgamūle svādhiṣṭhānacakraṃ//2// nābhau maṇipūrakacakraṃ//3// hṛdaye anāhatacakraṃ//4// kaṇṭhasthāne viśuddhacakraṃ//5// tālucakra //6// bhruvor madhye ājñācakram//7// brahmaraṃdhrasthāne kalācakraṃ//8// ākāśacakram ūrdhvaṃ tatparaṃ śūnyaṃ//9// \U2
%-----------------------------
%
%-----------------------------
%idānīm ādhāracakrasya bhedāḥ kathyanta/   \E
%idānīm ādhāracakrasya bhedaḥ kathyate     \P
%idānīm ādhāracakrasya bhedā kathyaṃte/    \B DSCN7165.jpg Z.3
%idānīm ādhāracakrasya bhedā kathyaṃte//   \L
%idānīm ādhāracakrasya bhedaḥ kathyate/    \N1
%idānīṃ ādhāracakrasya bhedaḥ kathyate//   \D1
%idānī  ādhāracakrasya bhedaḥ kathyaṃte/   \N2
%idānīṃ ādhāracakrasya bhedāḥ kathyaṃte    \U1
%idānīṃ ādhāracakrasya bhedāḥ kathyaṃte // \U2
%-----------------------------
%
%-----------------------------
%pādayor aṃguṣṭhe tejaso lakṣyakāraṇāt dṛṣṭiḥ sthirā bhavati/ \E
%pādayor aṃguṣṭhe tejaso lakṣyakaraṇāt dṛṣṭiḥ sthirā bhavati  \P
%pādayor aṃguṣṭhai tejasaṃ lakṣaṃ kartavyaṃ kāraṇāt// dṛṣṭiḥ sthirā bhavati/ \B
%pādayor aṃguṣṭhe tejasaṃ lakṣaṃ karttavyaṃ kāraṇāt dṛṣṭiḥ sthirā bhavatī/ \L
%pādayor aṃguṣṭhe tejaso lakṣyakāraṇāt dṛṣṭisthirā bhavati/ \N1
%pādayor aṃguṣṭhe tejaso lakṣyakāraṇāt dṛṣṭiḥ sthirā bhavati \D1
%pādayor aṃguṣṭhe tejaso lakṣakāraṇāt dṛṣṭisthirā bhavati/ \N2
%pādayor aṃguṣṭhe tejaso lakṣyakāraṇāt dṛṣṭisthirā bhavati \U1
%pādayor aṃguṣṭhe tejaso lakṣyakāraṇāt dṛṣṭisthirā bhavati// \U2 %%%415.jpg
%-----------------------------
%
%-----------------------------
%dvitīyo mūlādhāraḥ/  pādāṃguṣṭhasya mūle parapādasya pārṣṇiḥ sthāpyate tadāgniḥ prabalo bhavati/ \E
%dvitīyo mūlādhāraḥ   pādāṃguṣṭhasya mūle 'parapādasya dhāraḥ pādāṃduṣṭhasya mūleḥ paradādasya pārṣṇiḥ sthāpyate tadāgniḥ prabalo bhavati \P
%dvitīyo mūlādhāraḥ/  pādāṃguṣṭhasya mūle aparasya pādapārṣṇiḥ syāpyate tadāgniḥ prabalo bhavatī/ \B
%dvitīyo mūlādhāraḥ   pādāṃguṣṭhasya mūle aparasya pādapārṣṇīḥ syāpyate tadāgniḥ prabalo bhavatī/ \L
%dvitīyo mūlādhāraḥ/  pādāṃguṣṭhasya mūle aparapādasya pārṣṇiḥ sthāpyate agniḥ prabalo bhavati/   \N1
%dvitīyo mūlādhāraḥ// pādāṃguṣṭhasya mūle aparapādasya pārṣṇiḥ sthāpyate agniprabalo bhavati//   \D1  %%%p.12 recto
%dvitīyo mūlādhāraḥ   pādāṃguṣṭhasya mūle aparapādasya pārṣṇiḥ sthāpyate/ \om                     \N2
%dvitīyo mūlādharaḥ   pādāṃguṣṭhasya mūle aparapādasya pārṣṇiḥ sthāpyate agniṃ prabalo bhavati    \U1
%dvitīyo mūlādhare    pādāṃguṣṭhasya mūle 'parapādasya pārṣṇiḥ sthāyyaṃte//                       \U2
%-----------------------------
%
%-----------------------------
%ekaḥ pārṣṇir ādau mūlādhāre sthāpyate/     tasya pādasyāṃguṣṭhamūle parasya pādasya pārṣṇiḥ sthāpyate// tadagniḥ pradīpyate// \E [P.41]
%ekā pārṣṇir ādau  mūlādhāre sthāpyate      tasya pādasyāṃguṣṭhamūle 'parasya pādasya pārṣṇiḥ sthāpyate   tadagnīḥ pradipyate \P
%ekā pārṣṇir ādau  mūlādhāra sthāpyate      tasya pādasyāṃguṣṭhamūle aparasya pādasya pārṣṇiḥ sthāpyate// tadagnīḥ pradipyate// \B
%ekā pārṣṇir ādau  mūlādhārā sthāpyate      tasya pādasyāṃguṣṭhamūle aparasya pādasya pārṣṇiḥ sthāpyate// tadāgnīḥ pradivyate// \L
%ekā pārṣṇiḥ mūladdhāre sthāpyate/          tasya pādasya aṃguṣṭhamūlaṃ/ aparasya pādasya pārṣṇiḥ sthāpyaṃ agnir dāpyate?!/ \N1
%ekā pārṣṇiḥ mūlādhārai sthāpyate//         tasya pādasyāṃguṣṭhamūle// aparasya pādasya pārṣṇiḥ sthāpyaṃ// agnir dīpyate// \D1
% \om ------------------------------------- tasya pādasyāṃguṣṭhamūle// aparasya pādasya pārṇisthāpyaṃ agni dīpate// \N2
% ekāṃ pārṣṇir mūlādhāra sthāpyate           tasya pādasya aṃguṣṭhamūlaṃ aparasya pārṣṇo sthāpyate agni dīpyate  \U1
% \om                                                                                                         tadagnīḥ pradipyate// \U2
%-----------------------------
%One heel is to be placed at the Root-container. 
%-----------------------------
%tṛtīyaṃ gudādhārasthānaṃ  tanmadhye saṃkocavikāsākuṃcanakāraṇāt pavanaḥ sthiro bhavati// \E
%tṛtīyaṃ gudādhārasthānaṃ  tanmadhye saṃkocavikāśākuṃcanakāraṇāt pavanaḥ sthiro bhavati   \P
%tṛtīyaṃ gudādhārasthāne   tanmadhye saṃkocavikāśākuṃcanakāraṇāt pavanaḥ sthiro bhavati// \B
%tṛtīyaṃ gudādhārasthānaṃ  tanmadhye saṃkocavikāśa ākuṃcanakāraṇāt pavanasthiro bhavatī// \L
%tṛtīyaṃ gudādhārasthānaṃ  tanmadhye saṃkocavikāśākuṃcanakāraṇāt pavanaḥ sthiro bhavati// \N1
%tṛtīyaṃ gudādhārasthānaṃ  tanmadhye saṃkocavikāśākuṃcanaṃ kāraṇāt pavanasthiro bhavati// \D1
%tṛtīyaṃ gudādhārasthānaṃ  taṃmadhye saṃkocavikāśākuṃcanaṃ kāraṇāt pavanasthiro bhavati// \N2
%tṛtīyaṃ gudādhārasthānaṃ  taṃmadhye saṃkocavikāśā akuṃcanakāraṇāt pavanasthiro bhavati \U1
%tṛtīya  gudādhārasthānaṃ// tanmadhye saṃkocavikāśā kuṃcanakāraṇāt pavanasthiro bhavati// \U2
%-----------------------------
%
%-----------------------------
%anyacca/ puruṣasya maraṇaṃ na bhavati/ \E
%anuca puruṣasya maraṇaṃ bhavati  \P
%anucarapuruṣasya maraṇaṃ bhavatī/ \B
%anucakrapuruṣasya maraṇaṃ bhavatī/ \L
%anūca puruṣasya maraṇaṃ na bhavati ve?/ \N1
%anuca puruṣasya maraṇaṃ na bhavati// \D1
%anūca puruṣasya maraṇaṃ na bhavati// \N2
%anuca puruṣasya maraṇaṃ na bhavati  \U1
%anuca puruṣasya maraṇaṃ na bhavati//  \U2
%-----------------------------
%
%-----------------------------
%caturthaṃ liṃgādhāraṃ tanmadhye/ liṃgasaṃkocanābhyāsāt  paścimadaṇḍamadhye prajñā nāḍī bhavati/ tanmadhye punarabhyāsakaraṇānmanaḥ pavanayoḥ saṃcāro bhavati/ \E
%caturthaṃ liṃgādhāraṃ tanmadhye liṃgasaṃkocanābhyāsāt   paścīmadaṇḍamadhye vajñā nāḍī bhavati   tanmadhye punar abhyāsakaraṇānmanaḥ pavanayoḥ saṃcāro bhavati \P
%caturthaliṃgādhāraṃ tanmadhye liṃgasaṃkocanābhyāsāt paścīmadaṇḍamadhye vajñā nāḍī bhavatī/ tanmadhye punar abhyāsakaraṇāt punaḥ pavanayo saṃcāro bhavatī/     \B
%caturthaliṃgādhāraṃ// tanmadhye liṃgasaṃkocanābhyāsāt   paścamadaṇḍamadhye vajñā nāḍī bhavatī// tanmadhye punar abhyāsakaraṇāt punaḥ pavanayo saṃcāro bhavatī//     \L %%%%%%%%%%%20.jpg
%caturthaṃ liṃgādhāraṃ tanmadhye/ liṃgasaṃkocanābhyāsāt/ paścimadaṇḍamadhye vajranāḍī bhavati/ tanmadhye punaḥ abhyāsakaraṇāt manaḥ pavanayoḥ saṃcāro bhavati/ \N1
%caturthaliṃgādhāraṃ// tanmadhye/ liṃgasaṃkocanābhyāsāt// paścimadaṇḍamadhye vajrānāḍī bhavati// tanmadhye punaḥ abhyāsakaraṇāt manaḥ pavanayoḥ saṃcoro bhavati// \D1
%caturthaṃ liṃgādhāraṃ tanmadhye  liṃgasakoṇābhyāsāt//   paścimadaṇḍamadhye vajranāḍī bhavati/ tanmadhye punarābhyāsakaraṇāt manaḥ pavanayoḥ saṃcāro bhavati// \N2
%caturthaṃ liṃgādhāraṃ tanmadhye  liṃgasaṃkocanābhyāsāt  paścimadaṇḍamadhye vajranāḍī bhavati tanmadhye punarābhyāsakaraṇāt manaḥ pavanayoḥ saṃcāro bhavati    \U1    %%%283.jpg
%caturthaṃ liṃgādhāraṃ tanmadhye  liṃgasaṃkocanābhyāsāt  paścimadaṇḍamadhye vajranāḍī bhavati tanmadhye punarābhyāsakaraṇān manaḥ pavanayoḥ saṃcāro bhavati//   \U2
%-----------------------------
%
%-----------------------------
%tayoḥ saṃcārān madhye granthitrayaṃ truṭyati/ tatroṭanāt pavano brahmakamalamadhye pūrṇo bhūtvā tiṣṭhati/ tato vīryastambho bhavati/ puruṣaḥ sadaiva yuvā bhavati/ \E
%tayoḥ saṃcārān madhye graṃthitrayaṃ truṭyati  tato vīryastaṃbho bhavati puruṣaḥ saṃdaivaṃ yuve prabhavati  \P
%tayo saṃcārān madhye granthitrayaṃ truṭyatī/ tatroṭanāt pavano brahmakamadhye pūrṇā bhūtvā tiṣṭhati// tato vīryastambho bhavatī// puruṣaḥ sadaiva yuvai bhavatī/ \B
%tayoḥ saṃcārān madhye graṃthitrayaṃ truṭayatī tatroṭanāt pavano brahmakamadhye pūrṇā bhūtvā tiṣṭhati// tato vīryastaṃbho bhavati  puruṣaḥ sadaiva yuvaiva bhavati// \L
%tayoḥ saṃcārān madhye granthitrayaṃ truṭyati/ tat troṭanāt pavanaḥ brahmakamalamadhye pūrṇo bhūtvā tiṣṭhati/ tato vīryastambho bhavati/ puruṣaḥ sadaiva yuvā/e va bhavati// \N1 %truṭyati="zerbrechen"
%tayoḥ saṃcārāt madhye graṃthitrayaṃ truṭyati// tata troṭanāt pavanaḥ brahmakamalamadhye pūrṇo bhūtvā tiṣṭhati// tato vīryastambho bhavati// puruṣaḥ sadaiva yuvaiva bhavati// \D1 
%tayoḥ saṃcārān madhye granthitrayaṃ ... ..ti/ tata troṭanāt pavanaḥ brahmakamalamadhye pūrṇo bhūtvā tiṣṭhati/ tato vīryastambho bhavati/ puruṣa sadaiva  yurvaiva bhavati// \N2
%tayoḥ saṃccārāt madhye graṃthitrayaṃ trudyati tatroṭaṇāt pavanaḥ brahmakamalamadhye pūrṇo bhūtvā tiṣṭhati tato vīryastaṃbho bhavati/ puruṣaḥ sadaiva  yuvaivaṃ bhavati \U1
%tayoḥ saṃccārān madhye graṃthitrayaṃ truṭyati// tattroṭaṇāt pavanaḥ brahmakamalamadhye pūrṇo bhūtvā tiṣṭhati// tato vīryastaṃbho bhavati puruṣaḥ sadaiva vaibhavo bhavati// \U2
%-----------------------------
%
%-----------------------------
%paṃcama udgīryāṇāṃ svādhiṣṭhānaṃ tatra bandhanānmalamūtrayornāśo bhavati/ ṣaṣṭho nābhyādhāraḥ/ tasmin sthāne prāṇavāyornirodhāt ṣaḍapi kamalānyūrdhvamukhāni vikasaṃti// \E
%paṃcamaṃ uḍḍīyāṇāṃ svādhiṣṭhānaṃ tatra baṃdhadānān malamūtrayor nāśo bhavati ṣaṣṭho nābhyādhāraḥ tatra prāṇavābhyāsād nāhato nāraḥ svayam utpadyate / \P
%paṃcama uḍḍiyānāṃ svādhiṣṭhānaṃ tatra baṃdha dīyate/ malamūtrayor nāśo bhavatī// ṣaṣṭho nābhyādhāraḥ tatra prāṇavābhyāsād anāhato nādaḥ// svayam utpadyate// \B
%paṃcamaṃ uḍḍiyānāṃ svādhiṣṭhānaṃ tatra baṃdha dīyate/ mūlamūcayor nāśo bhavati// ṣaṣṭho nābhyādhāraḥ tatra prāṇavābhyāsād anāhato nādaḥ// svayam utpadyate... \L 
%paṃcamaṃ udyānaṃ tatra baṃdhanāt malamūtrayornāśe/o[s.10, verso, z4] bhavati// ṣaṣṭho nābhyādhāraḥ/ tatra praṇavābhyāsāt anāhato nādaḥ svayamūtpadyate/  \N1
%paṃcamaṃ udyāṇāṃ tatra vaṃdhanāt malamūtrayor nāśo bhavati// ṣaṣṭho nābhyādhāraḥ// tatra prāṇavābhyāsāt anāhato nādaḥ// svayam utpadyate// \D1
%paṃcamodyānaṃ tatra baṃdhanāt malamūtrayor nāśo bhavati/                       ṣaṣṭho nābhyādhāraḥ  tatra praṇavābhyāsāt anāhato tādaḥ svayaṃ utpadyate/ \N2
%paṃcamaṃ uddyānaṃ tatra baṃdhadānāt malamūtrayornāśo bhavati ṣaṣṭho nābhyādhārastatra praṇavābhyāṃ sadānāhato nadaḥ svayam utpadyate   \U1
%paṃcamaṃ uḍḍīyāṇaṃ svādhiṣṭhānaṃ tatra badhadānān malamūtrayor nāśo bhavati// ṣaṣṭho nābhyādhāre// tatra prāṇavābhyāsād anohato nādaḥ svayam utpadyate// \U2
%-----------------------------
%
%-----------------------------
%\om                                                                         \E
%saptamo hṛdayarūpadhāraḥ   tasmin sthāne prāṇavāyor nirodhāt ṣadapi kamalānyūrdhvamukhāni  vikasaṃti  \P  %%%7653.jpg 
%                           tasmin sthāne prāṇavāyo nirodhāt/ yaḍapi kamalāny ūrdhvamukhāni vikasaṃti// \B
%saptamo hṛdayarūpadhāraḥ// tasmin sthāne prāṇavāyor nirodhāt ṣadapi kamalānyūrdhvamukhāni  vikasaṃti// \L
%saptamo hṛdayarūpa ādhāraḥ tasmin sthāne prāṇacā?yor nirūṃdhanāt/ ṣadapi kamalānyūrdhvamukhaṃ vikasaṃti// \N1
%saptamo hṛdayarūpa ādhāraḥ// tasmin sthāne prāṇavāyor nir???ūṃ???dhanāt// ṣadapi kamalānyūrdhvamukhaṃ vikasaṃti// \D1
%saptamo hṛdayarūpādhāraḥ   tasmin sthāne prāṇavāyor nirūṃdhanāt/ ṣadapi kamalānyūrdhvemukhaṃ [S.9, recto, z.4] vikasaṃti// \N2
%saptamo hṛdayarūpādhāraḥ   tasmin sthāne prāṇavāyor nirūṃdhanāt  ṣadapi kamalān yūrusyordha mukhaṃ bhavati vikasaṃti  \U1
%saptamo hṛdayādhāraḥ       tasmin sthāne prāṇavāyor nirodhāt//  ṣadapi kamalān yūrddhvamukhāni vikasaṃti//  \U2
%-----------------------------
%
%-----------------------------
%aṣṭamaṃ kaṇṭhādhāraḥ/ tatra jālaṃdharo bandho dīyate/ tasmin satīḍāyāṃ piṃgalāyāṃ[p.43]pavanaḥ sthiro bhavati/  \E
%aṣṭamaḥ kaṃṭhādhāraḥ  tatra jālaṃdharo baṃdho dīyate  tasmin satīḍāyāṃ piṃgalāyāṃ[p.43]pavanaḥ sthiro bhavataḥ  \P
%aṣṭame kaṇṭhādhāraḥ/ tatra jalaṃ baṃdho dīyate tasmin satīyāṃ piṃgalāyāṃ pavanaḥ sthiro bhavatī/ \B  %%%%DSCN7166.jpg Z.3
%aṣṭame kaṇṭhādhāraḥ/ tatra jalaṃ baṃdho dīyate tasmin satīyāṃ piṃgalāyāṃ pavanaḥ sthiro bhavatī// \L
%aṣṭamaḥ kaṇṭhādhāraḥ/ tatra jālaṃdharo baṃdho dīyate/ tasmin sati iḍāyāṃ piṃgalāyāṃ pavanaḥ sthiro bhavati/ \N1
%aṣṭamaḥ kaṃṭhādhāraḥ// tatraḥ jālaṃdharo baṃdho dīyate// tasmin sati iḍāyāṃ piṃgalāyāṃ pavanasthiro bhavati// \D1  %%%p.12 verso
%aṣṭamakaṇṭhādhāraḥ/ tatra jālaṃdharabandho dīyate// tasminsatiśadāyāṃ piṃgalāyāṃ pavanaḥ sthiro bhavati/ \N2
%aṣṭamaḥ kaṇṭhādhāraḥ tatra jālaṃdharo bandho dīpyate tasmin sati iḍāyāṃ piṃgalāyāṃ pavanaḥ sthiro bhavati \U1
%aṣṭamaḥ kaṇṭhādhāraḥ tatra jālaṃdharo bandho dīyate tasmin sati piḍāyā piṃgalāyāṃ pavanaḥ sthiro bhavati// \U2
%-----------------------------
%
%-----------------------------
%navamo ghaṃṭikādhāraḥ/ tatra jihvāgraṃ lagnaṃ bhavati/   tatomṛtakalāyā  amṛtaṃ sravati/ tadamṛtapānāt śarīramadhye rogasaṃcāro na bhavati/ \E
%navamo ghaṭikādhāraḥ   tatra jihvāgraṃ lagnaṃ bhavati    tatomṛtakakalāyā amṛta sravati  tadamṛtapānāccharīramadhye rogasaṃcāro na bhavati  \P
%navo ghaṃṭikādhāraḥ//  tatra jihvāgraṃ lagnaṃ bhavatī/   tatomṛtakalāyā  amṛtaṃ sravati/ tadamṛtakalāyāṃ amṛtapānīcharīramadhye rogasaṃcāro bhavatī/ \B
%navamo ghaṃṭādhāraḥ//  tatra jihvāgraṃ lagnaṃ bhavati//  tatomṛtakalāyāṃ                                 amṛtapānācharīramadhye rogasaṃcāro bhavati// \L %eyeskip in line.. :(
%navamo ghaṃṭikādhāraḥ/ tatra jihvāgraṃ lagnaṃ bhavati/   tatomṛtakalāyā  amṛtaṃ sravati/ tadamṛtapānāt śarīramadhye rogasaṃcāro na bhavati/ \N1
%navamo ghaṃṭikādhāraḥ// tatra jihvāyāgraṃ lagnaṃ bhavati//   tataḥ amṛtakalāyāḥ  amṛtaṃ sravati// tad amṛtapānāc charīramadhye rogasaṃcāro na bhavati// \D1
%navamo ghaṃṭikādhāraḥ/ tatra jihvāgraṃ lagnaṃ bhavati/   tatomṛtakalāyā  amṛtaṃ sravati/ tadamṛtapānāt śarīramadhye rogasaṃcāro na bhavati/ \N2
%navamo ghaṃṭikādhāras  tatra juhvāyāṃ lagnaṃ bhavati vā  tataḥ amṛtakalāyāḥ  amṛtaṃ sravati tadamṛtapānāt charīramadhye rogasaṃcāro na bhavati \U1
%navamo ghaṃṭikādhāraḥ  tatra jihvāgraṃ lagnaṃ bhavati//  tato mṛtakalāyāḥ  amṛtaṃ sravati// tadamṛtapānā  charīramadhye rogasaṃcāro na bhavati// \U2
%-----------------------------
%
%-----------------------------
%daśamaṃ tālvādhāraḥ/  tanmadhye vānaṃdollahanaṃ ca kṛtvā laṃbikāpraveśe sati tālunimagnā jihvā tiṣṭhati/ \E
%daśamas tālvādhāraḥ   tanmadhye cālanaṃ dohanaṃ cakratvā laṃbikāpraveśe śe sati tālumagnā jihvā tiṣṭhati  \P %%%7654.jpg
%daśamaṃ stālvādhāraḥ/ tanmadhye cālanaṃ dohanaṃ cakratvā laṃbikāpraveśe sati tālumagnā jihvā tiṣṭhati/ \B
%daśamastālvādhāraḥ//  tanmadhye cālanaṃ dohanaṃ ca kṛtvā laṃbikāpraveśe sati tālumagnā jihvā tiṣṭhati ... \L
%daśamatālvādhāraḥ//   tanmadhye cānanaṃ dohanaṃ ca kṛtvā laṃbikāpraveśe grati tālunimagnā jihvā tiṣṭhati/ \N1
%daśamas tālvādhāraḥ   tanmadhye cānanaṃ dohanaṃ ca kṛtvā laṃbikāpravese grati tālunimagnā jihvā tiṣṭhati// \D1
%daśamatālvādhāraḥ     tanmadhye cālanaṃ dohanaṃ ca kṛtvā laṃbikāpraveśe gratitālūnimagnā                    \N2
%daśamastālvādhāraḥ staṃnmadhye cālanaṃ dohanaṃ ca sva/sca? kṛtvā cālaṃ vikā praveśe sati tālūnilagnā juhvā tiṣṭhati \U1 %%%284.jpg
%daśamas tālvādhāraḥ   tanmadhye cālanaṃ dohanaṃ chedanaṃ ca kṛtvā laṃbikāpraveśe sati tālunimagnā jihvā tiṣṭhati// \U2 %%416.jpg
%-----------------------------
%
%-----------------------------
%ekādaśo jihvādhāraḥ/ tasmin jihvāgreṇa manthanaṃ kriyate tasmin kṛtetimadhuraṃ pānīyaṃ sravati/ tadā ca kavitvacchandonāṭakādiviṣayajñānamutpadyate/ \E
%ekādaśo jihvātale jihvādhāraḥ tasmin jihvāgreṇa mathanaṃ kriyate tasmin kṛte timadhuraṃ pānīyaṃ sravati tathā ca kavit vachaṃdonāṭakādiviṣayajñānam utpadyate  \P
%ekādaśo jihvātale jihvādhāraḥ// tasmin jihvāgreṇa manthanaṃ kṛtvā// tasmiṃ kṛte satimadhuraṃ pānīyaṃ sravatī// tathā kvacit vacchaṃdonāṭakādiviṣayapānam utpadyaṃte/ \B
%ekādaśo jihvātale jihvādhāraḥ// tasmin jihvāgreṇa mathanaṃ kṛtvā// tasmiṃ kṛte satimadhuraṃ pānīyaṃ sravati// tathā kvacit vachaṃdonāṭakādiviṣayajñānam utpadyate// \L
%ekādaśo jihvādhāraḥ/ tasmin jihvāgreṇa manthanaṃ kriyate/ tasmin kṛte atimadhuraṃ pānīyaṃ sravati/ tathā ca kavitvagītacchaṃdanāṭakādiviṣaye jñānam utpadyate/ \N1
%ekādaśo jihvātale jihvādhāraḥ// tasmin jihvāgreṇa mathanaṃ kriyate// tasmiṃ kṛte satimadhuraṃ pānīyaṃ sravati// tathā ca kvacit tachaṃdanāṭakādiviṣayajñānam utpadyate \D1 
% jihvāgreṇa manthanaṃ kriyate// tasmin kṛte atimadhuraṃ pānīyaṃ sravati// kaminnāsikā phatkāravat// tathā ca kavitvagītachaṃdanāṭakādiviṣaye jñānam utpadyate/ \N2
% ekādaśā jihvātale jihvādhāraḥ tasminna jihvāgreṇa manthanaṃ kriyate tasminn kṛte ti madhuraṃ pānīyaṃ sravati tathā ca kavitva gītachaṃdavacchaṃdanāḍīviṣayaṃ jñānānam utpadyate \U1
%ekādaśo jihvātale jihvādhāraḥ tasmin jihvāgreṇa manthanaṃ kriyate// tasminn kṛte 'ti madhuraṃ pānīyaṃ sravati// tathā ca kavitvaṃ chaṃdonāṭakādiviṣayajñānam utpadyate// \U2
%-----------------------------
%
%----------------------------
%tadupari dvādaśadantayomadhye dantādhāraḥ/ tasmin sthāne jihvāyā agraṃ ghaṭīmātraṃ valātkāreṇa sthāpyate/ tasmin sati sādhakasya samagrā rogā naśyanti// \E %%%[p.44]
%tadupari dvādaśo daṃtayor madhye  daṃtādhāraḥ    tasmin sthāne jihvāyā agraṃ ghaṭīmātram ārghaghaṭīmātraṃ bālātkāreṇa sthāpyate tasmin satisādhakasya samagrā rogā naśyaṃti \P
%tadupari dvādaśo daṃtayor madhye// daṃtādhāraḥ// tasmin sthāne jihvāyā 'agnaṃ ghaṭīmātram ārghaghaṭimātraṃ bālākāreṇa sthāpyate// tasmiṃ sādhakasya samagrā rogā naśyaṃtī// \B
%tadupari dvādaśo daṃtayor madhye// daṃtādhāraḥ tasmin sthāne jihvāyā agnaṃ ghaṭīmātram ārddhaghaṭimātraṃ bālākāreṇa sthāpyate// tasmiṃ sādhakasya samagrā rogā naśyaṃti... \L
%tadupari dvādaśayor madhye daṃtādhāraḥ/ tasmin sthāne jihvāyā agraṃ ghaṭīmātraṃ arddhaghaṭimātraṃ valātkāreṇa sthāpyate// tasmin sati sādhakasya samagrā rogā naśyaṃti// \N1
%tadupari dvādaśayor madhye daṃtādhāraḥ// tasmin sthāne jihvāyā agraṃ ghaṭīmātraṃ arddhaghaṭimātraṃ valātkāreṇa sthāpyate// tasmin sati sādhakasya samagrā rogā naśyaṃti// \D1
%tadupari dvādaśayor madhye daṃtādhāraḥ// tasmin sthāne jihvāyāgraṃ ghaṭīmātraṃ arddha?ghaṭimātraṃ valātkāreṇa sthāpyate// tasmin sati sādhakasya samagrā rogā naśyanti \N2
%tadupari dvādaśo daṃtayor madhye   daṃtādhāraḥ tasmin sthāne jihvāyāṃ agraṃ ghaṭīmātram ārdhaghaṭikāmātraṃ bālāt kāreṇa sthāpyate tasminn sati sādhakasya samagra rogā naśyaṃti \U1
%tadupari dvādaśor daṃtayo madhye  daṃtādhāraḥ    tasmin sthāne jihvāyā agraṃ ghaṭīmātram ārghaghaṭīmātraṃ bālātkāreṇa sthāpyate// tasmin satisādhakasya samagrā rogā naśyaṃti// \U2
%-----------------------------
%
%----------------------------
%trayodaśo nāsikāgrādhāraḥ/ tasmin lakṣye kṛte sati manaḥ sthiraṃ bhavati/ \E
%trayodaśo nāsikāgrādhāraḥ tasmiṃ lakṣye kṛte sati manaḥ sthiraṃ bhavati \P
%trayodaso nāsikādhāraḥ/ tasmin ḍraṣṭe kṛte minasthire bhavati/ \B
%trayodaso nāśikādhāraḥ  tasmin ḍraṣṭe kṛte manaḥ sthiro bhavati/ \L
%trayodaśo nāsikādhāraḥ/ tasmin lakṣe kṛte sati manasthiraṃ bhavati/ \N1
%trayodaśo nāsikādhāraḥ// tasmin lakṣe kṛte sati manasthiraṃ bhavati \D1
%trayodaśo nāsikādhāraḥ/ tasmin lakṣe kṛte sati manasthiraṃ bhavati/ \N2
%trayodaśo nāsikādhāraḥ tasmiṃ lakṣye kṛte sati manasthiraṃ bhavati \U1
%trayodaśo nāsikādhāraḥ tasmil lakṣe kṛte sati manasthiraṃ bhavati// \U2
%-----------------------------
%
%----------------------------
%caturdaśo nāsāmūlādhāraḥ/ tasmin dṛṣṭeḥ sthairyakāraṇātṣaṣṭhe māsi svīyantejaḥ pratyakṣaṃ bhavati/ tejasaḥ pratyakṣatve pārthivaṃ sakalaṃ bandhanaṃ tuṭyati/ \E
%caturdaśo nāsā mūlādhāro tasmin dṛṣṭeḥ sthairyakāraṇāt ṣaṣṭhe māsi svīyaṃ tejaḥ pratyakṣaṃ bhavati tejasaḥ pratyakṣatve pārthivaṃ sakalaṃ baṃdhanaṃ truṭy?ati/ \P %%%7654.jpg vorletzte Zeile
%caturdaśo nāso mūlādhāraḥ// tasmin llakṣe krute satī sthairyakāraṇāt// ṣaṣṭhe māse svayaṃ tejaḥ pratyakṣaṃ bhavati// tejasaḥ pratyakṣatve pārthivaṃ sakalaṃ baṃdhanaṃ truṭayati/ \B
%caturdaśo nāso mūlādhāraḥ  tasmin lakṣe kṛte satī sthairyakāraṇāt      ṣaṣṭhe māse svayaṃ tejaḥ pratyakṣaṃ bhavati// tejasaḥ pratyakṣatve pārthivaṃ sakalaṃ baṃdhanaṃ truṭayati/ \L
%caturdaśo nāsāmūle vāyvā? bāybā? dhāraḥ/ tasmin dṛṣṭeḥ sthairyakāraṇāt ṣaṣṭhe māsi svīyaṃ tejaḥ pratyakṣaṃ bhavati/ tejasaḥ pratyakṣatve pārthivaṃ sakalaṃ baṃdhanaṃ trudyati/ \N1
%caturdaśo nāsāmūle vāyvādhāraḥ// tasmin dṛṣṭeḥ sthairyakāraṇāt ṣaṣṭhe māsi svīyaṃ tejaḥ pratyakṣaṃ bhavati// tejasaḥ pratyakṣatve pārthivaṃ sakalaṃ baṃdhanaṃ trudyati// \D1  %%%p.13 recto 
%caturdaśo nāsāmūle vāyvādhāraḥ??/        tasmin dṛṣṭeḥ sthairyakāraṇāt ṣaṣṭhe māsi svayaṃ tejaḥ pratyakṣaṃ bhavati tejasaḥ pratyakṣatve pārthiva sakalaṃ bandhanaṃ trudyati// \N2
%caturdaśo nāsā mūlevādhāraḥ tasmiṃ na dṛṣṭeḥ sthairyakāraṇāt ṣaṣṭhe māse svīyaṃ tejaḥ pratyakṣaṃ bhavati tejasaḥ pratyakṣatve pārthivaṃ sakalaṃ baṃdhanaṃ tru??ati \U1
%caturdaśo nāsā mūlādhāraḥ  tasmin laṣṭhe? sthairyakāraṇāt    ṣaṣṭhe māsi svayaṃ tejaḥ pratyakṣaṃ bhavati// tejasaḥ pratyakṣatve pārthivaṃ sakalaṃ baṃdhanaṃ truṭyati// \U2
%-----------------------------
%
%----------------------------
%pañcadaśo bhruvormadhyādhārastasmin dṛṣṭeḥ sthirīkaraṇāt koṭikiraṇāḥ sphuraṃti/ \E
%paṃcadaśo bhruvormadhyādhāraḥ tasmin ḍṛṣṭeḥ sthirīkaraṇāt koṭikiraṇāḥ sphuraṃti  \P  %%%7655.jpg
%paṃcadaśo bhruvormadhye dhāraḥ// tasmin ḍṛṣṭeḥ sthirikaraṇāt// koṭikiriṇā sphuraṃti// \B
%paṃcadaśo bhruvormadhye dhāraḥ// tasmin ḍṛṣṭe sthirīkaraṇāt// koṭikiriṇā sphuraṃti// \L
%pañcadaśo bhruvormadhye ādhāraḥ/ asmin dṛṣṭeḥ sthirīkaraṇāt koṭikiraṇāni sphuraṃti/ \N1
%pañcadaśo bhruvormadhye ājñādhāraḥ// ..smin dṛṣṭeḥ sthirīkaraṇāt koṭikiraṇāni sphuraṃti// \D1

%pañcadaśo bhruvormadhye ādhāraḥ tasmin dṛṣṭeḥ sthirīkaraṇāt koṭikiraṇāni sphuraṃti/ \N2 [S.9]
%pañcadaśo bhruvormadhye ādhāra asin na dṛṣṭeḥ sthirīkaraṇāt koṭikiraṇāni sphuraṃti \U1
%pañcadaśo bhruvormadhyādhāra tasmin dṛṣṭisthirīkaraṇāt koṭikiraṇaḥ sphuraṃti// \U2
%-----------------------------
%
%----------------------------
%ṣoḍaśo  netrādhāraḥ/ ayam aṃgulyagreṇa cālyate/  tadabhyāsāt/ pṛthvīmadhye  yatkiṃcin tejo  varttate/  \E   %%%p.45
%ṣoḍaśo  netrādhāraḥ  ayam aṃgulyagreṇa cālyate   tadabhyāsāt  pṛthvīmadhye  yatkiṃcit tejo  vartate... \P
%ṣoḍaśo  netrā//      ayam aṃgulyagreṇa cālyate// tadabhyāsāt  pṛthivīmadhye yatkiṃcit tejo  vartate//  \B %%%%%%%%%%%%%%%%DSCN7167.jpg Z. 1
%ṣoḍaśo  netrā//      ayam aṃgulyagreṇa cālyate// tadabhyāsāt  pṛthivīmadhye yatkiṃcit tejo  vartate... \L
%ṣoḍaśaḥ netrādhāraḥ/ ayaṃ agulyagreṇa  cālyate/  tadabhyāsāt  pṛthvīmadhye  yatkiṃcit tejaḥ varttate/  \N1
%ṣoḍaśaḥ netrādhāraḥ// ayaṃ agulyagreṇa  cālyate//  tadabhyāsāt  pṛthvīmadhye  yatkiṃcit tejaḥ varttate \D1
%ṣoḍaśaḥ netrādhāraḥ/ ayaṃ aṃgugreṇa    cālyate/  tadabhyāsāt  pṛthvīmadhye  yatkiṃcit tejaḥ varttate/  \N2
%ṣoḍaśo  netrādhāraḥ  ayaṃ aṃgulyagreṇa cālyate   tadābhyāsāt  pṛthvīmadhye  yatkiṃcit       vatate     \U1 %%%%%%%%%%%%%%%%%%285.jpg
%ṣoḍaśo  netrādhāraḥ  ayamaṃgulyagreṇa cālyate//  tadabhyāsāt  pṛthivīmadhye yatkiṃcit// tejo  vartate//  \U2
%-----------------------------
%
%----------------------------
%tatsarvaṃ tejo dṛṣṭiviṣayaṃ bhavati/ taddarśanāt puruṣaḥ sarvajño bhavati// \E
%tatsarvaṃ tejo dṛṣṭiviṣayaṃ bhavati  tadarśanāt puruṣaḥ sarvajño bhavati     \P
%tatsarvaṃ tejo dṛṣṭiviṣayaṃ bhavatī// taddarśanāt puruṣaḥ sarvajño bhavatī// \B
%tatsarvaṃ tejo dṛṣṭiviṣayaṃ bhavati// taddarśanāt puruṣaḥ sarvajño bhavati// \L
%tatsarvvatejo dṛṣṭiviṣayaṃ bhavati taddarśanāt puruṣaḥ sarvvajño bhavati// \N1
%tatsarvatejo dṛṣṭiviṣayaṃ bhavati taddarśanāt puruṣaḥ sarvvajño bhavati// \D1
%tatsarvatejo dṛṣṭiviṣayaṃ bhavati taddarśanāt puruṣaḥ sarvajño bhavati// \N2
%tatsarvaṃ tejo dṛṣṭīviṣayaṃ bhavati tat darśa ?? puruṣaḥ sarvajño bhavati \U1
%tatsarvaṃ tajaso dṛṣṭiviṣayaṃ bhavati// tad darśanāt puruṣaḥ sarvajño bhavati// \U2
%
%
%-----------------------------
%

%----------------------------
%Note: Rāmacandra does not adopt the yāmas and niyāmas from the Yogasvarodaya! 
%----------------------------
%idānīm aṣṭāṃgayogavicāraḥ kathyate/ yamaniyamāsanaprāṇāyāmapratyāhāradhyānadhāraṇāsamādhir iti/ eteṣāṃ lakṣaṇāni kathyante/     \E
%idānīm aṣṭāṃgayogasya vicāraḥ kathyate  yamaniyamāsanaprāṇāyāmapratyāhāradhyānadhāraṇāsamādhir iti  eteṣāṃ lakṣaṇāni kathyaṃte  \P
%idānīm aṣṭāṃgayogasya vicāraḥ kathyate/ yamaniyamāsanaprāṇāyāmapratyāhāradhāraṇādhyānasamādhir iti/ eteṣāṃ lakṣaṇāni kathyaṃte/ \B
%idānīm aṣṭāṃgayogasya vicāraḥ kathyate/ yamaniyamāsanaprāṇāyāmapratyāhāradhāraṇādhyānasamādhir iti/ eteṣāṃ lakṣaṇāni kathyaṃte/ \L
%idānīm aṣṭāṃgayogasya vicāraḥ kathyate// yamaniyamāsanaprāṇāyāmapratyāhāradhyānadhāraṇāsamādhiyaḥ eteṣāṃ lakṣaṇāni kathyaṃte/   \N1
%idānīm aṣṭāṃgayogasya vicāraḥ kathyate// yamaniyamāsanaprāṇāyāmapratyāhāradhyānadhāraṇāsamādhi// eteṣāṃ lakṣaṇāni kathyaṃte//   \D1
%idānīṃ aṣṭāṃgayogasya vicāraḥ kathyate// yamaniyamāsanaprāṇāyāmapratyāhāradhyānadhāraṇāsamādhiyaḥ eteṣāṃ lakṣaṇāni kathyaṃte/   \N2
%idānīṃ aṣṭāṅgayogasya vicāraḥ kathyate// yamaniyamāsanaprāṇāyāmapratyāhāradhyānadhāraṇāsamādhi eteṣāṃ lakṣaṇāni kathyate   \U1
%idānīṃ aṣṭāṅgayogasya vicāra  kathyate// yamaniyamāsanaprāṇāyāmapratyāhāradhyānadhāraṇāsamādhir iti// eteṣāṃ lakṣaṇāni kathyaṃte//   \U2
%-----------------------------
%Now the procedure of the eightfold yoga (\textit{aṣṭāṅgayoga})is explained: "Yama, niyama, āsana, prāṇāyāma, pratyāhāra, dhyāna, dhāraṇā and samādhi." Their characteristics will be explained.   
%----------------------------
%śāntiḥ/ ṣaṇṇāmindriyāṇāṃ jayaḥ/ svalpāhāraḥ/ nidrājayaḥ/ śītoṣṇajayaḥ/ ete yamāḥ/ \E
%śāṃtiḥ ṣaṇāṃ iṃdriyāṇāṃ jayaḥ ahāraḥ svalpaḥ nidrājayaḥ śaityajayaḥ uṣṇa?jayaḥ ete yamāniyamāḥ ...\P
%śāntiḥ ṣaṇāṃ iṃdriṇāṃ jayaḥ// ahāraḥ svalpaḥ nidrāyā jayaḥ// śaityajayaḥ/ uṣṇājayaḥ// ya te yamaḥ// \B
%śāntiḥ ṣaṇṇāṃ iṃdriyāṇāṃ jayaḥ// ahāraḥ// svalpaḥ// nidrāyāḥ jayaḥ/ śaityajayaḥ uṣṇajayaḥ ya te yamaḥ... \L
%śāntiṣaṇṇāṃ indriyāṇāṃ jayaḥ/ svalpāḥ nidrājayaḥ/ śītyajayaḥ/ uṣṇajayaḥ/ ete yamāḥ/ \N1
%śāṃtiṣasmāṃ? indriyāṇāṃ jayaḥ// āhāraḥ svalpāḥ nidrājayaḥ// śaityajayaḥ// uṣṇajayaḥ/ ete yamāḥ \D1
%śāntiṣaṇṇāṃ indriyāṇāṃ jayaḥ/ ahāraḥ svalpāḥ nidrājayaḥ/ śaityajayaḥ uṣṇajayaḥ/ ete yamāḥ/ \N2
%śāntiḥ ṣaṇṇām iṃdriyāṇāṃ jayaḥ āhāraḥ sajayaḥ nidrājayaḥ śaityajayaḥ auṣṇājayaḥ ete yamāḥ \U1
%śānti  śaṇa iṃdriyāṇāṃ jayaḥ// āhāraḥ svalpaḥ// nidrāyāḥ jayaḥ// śaityajayaḥ// uṣṇājayaḥ// ete yamāḥ// \U2 %%%417.jpg 
%----------------------------
%These are the Yāmas: Peace, conquer of the senses, little food, conquer of sleep, conquer of cold and heat.
%----------------------------
%niyamāḥ khalu cāpalabhāvānnivāryasthairye sthāpyate/ ekāṃte sevanam/ prāṇimātre [P.46] samā buddhiḥ/ audāsīnyaṃ kasyāpi vastuna icchā na karttavyā yathālābhasaṃtoṣaḥ/ parameśvaranāma na vismaraṇīyam/ manomadhye dainyaṃ karttavyam/ iti niyamāḥ// \E
%
%svalu cāpalābhāvānnirvārya sthairye sthāpyate ekāṃta sevānaṃ prāṇimātre samābuddhiḥ udāsīnyaṃ kasyāpi vastuna icchā na kartavyā yathālābhasaṃtoṣaḥ parameśvaranāma na vismaraṇīyaṃ manomadhye dainyaṃ kartavyaṃ iti niyamāḥ\P %%%7656.jpg
%
%svalu cāpalabhāvānnirvārya sthāpyate// ekāṃta sevānāṃ prāṇimātre samābuddhiḥ udāsīnyaṃ kasyāpi vastunaḥ// icchā na kartavyā yathālābhasaṃtoṣaḥ/ parameśvaranāma na vismaraṇīyaṃ manomadhye dainyaṃ kartavyaṃ// iti niyamaḥ// \B
%
%
%svalu cāpalabhāvānnirvārya sthāpyate// ekāṃtasevānāṃ prāṇimātre samābuddhiḥ/ udāsīnyaṃ kasyāpi vastunaḥ/ icchā na kartavyā yathā lābhasaṃtoṣaḥ parameśvaranāma na vismaraṇīyaṃ manomadhye dainyaṃ karttavyaṃ/ iti niyamaḥ// \L
%
%niyamaḥ khalu capalabhāvānnivāryasthairye sthāpyate/ ekāṃte sevanam/ prāṇimātre samā buddhiḥ/ udāsīnya/ kasyāpi vastunaḥ icchā na karttavyā// yathā lābhasaṃtoṣaḥ/ parameśvaranāma vismaraṇīyam/ manomadhye dainyaṃ na karttavyam/ //[S.11] \N1
%
%niyamaḥ khalu manaḥ capalabhāvān nivārye sthāpyate// ekāṃtasevanaṃ// prāṇimātre samā buddhiḥ// udāsīnya// kasyāpi vastunaḥ icchā na karttavyā// yathā lābhasaṃtoṣaḥ// parameśvaranāma vismaraṇīyaṃ// manomadhye dainyaṃ na karttavyaṃ// \D1
%
%niyamaḥ khalū manaḥ capalabhāvānnivāryasthairye sthāpyate ekāṃtasevanam/ prāṇimātre [P.46] samā buddhiḥ udāsīnya kasyāpi vastunaḥ icchā na karttavyā/ yathā lābhasaṃtoṣaḥ parameśvaranāmavismanīyam/ manomadhye dainyaṃ na karttavyam// // \N2  \em zu vismāra
%
%niyamaḥ khalū manaḥ capalabhāvānnivārayasthairye sthāpyate ekāṃtasevanaṃ prāṇimātre samā buddhi udāsīnyāṃ kasyāpi vastunaḥ icchā na karttavyaṃ yathā lābhasaṃtoṣaḥ parameśvaraḥ nāma na vismaraṇīyaṃ mano dainyaṃ na karttavyaṃ  \U1
%niyamaḥ// khalū cāpalābhāvānnivārya sthāpyate// ekāṃtasevanaṃ// prāṇimātre samābuddhi// udāsīnyaṃ// kasyāpi vastuna icchā na karttavyaṃ// yathā lābhasaṃtoṣaḥ// parameśvaraḥ nāma na vismaraṇaṃ// yaṃ mano madhye dainyaṃ na karttavyaṃ iti niyamaḥ//  \U2
%----------------------------
%Niyamās are truly: Keeping the mind from the unsteady state [and] ground it in calmness, retreating to a lonely place, refraining from contact to animals, unchanging intellect, keeping equanimous one shall not crave for things, as well as being contend with what is given, never forgetting the name of the highest lord, one shall not bring the mind into depression. 
%----------------------------
%āsanalakṣaṇaṃ bahuṣu grantheṣu nirūpitamasti tenātra na nirūpyate/ \E
%āsanalakṣaṇaṃ bahuṣu graṃtheṣu nirūpitam asti tenātra na nirūpyate \P
%āsanaṃ lakṣaṇāṃ bahūgraṃtheṣu nirūpyamasti tenātranirūpyate/       \B
%āsanalakṣaṇāṃ bahūgraṃtheṣu   nirūpyam asti tenātranirūpyate//     \L
%āsanasya lakṣaṇaṃ bahūgraṃthe nirūpitam/ ataḥ atrāyaṃ nirūpyate/   \N1
%āsanasya lakṣaṇaṃ bahūgraṃthe nirūpitaṃ// ataḥ atratyaṃ nirūpyate// \D1 %%%p. 13 verso
%āsanasya lakṣaṇaṃ bahugraṃthe nirūpitam// ataḥ atrāyaṃ nirūpyate/  \N2
%āsanasya lakṣaṇaṃ bahugraṃthe nirūpitam tan attaḥ atra na nirūpyate  \U1
%āsanalakṣaṇaṃ tu bahugraṃtheṣu nirūpitam asti// tenātra nirūpyate// \U2
%----------------------------
%The characteristic of posture has been discussed in many works and will not be discussed here.  
%----------------------------
%prāṇāyāmas tu sukumāreṇa sādhituṃ na śakyate atastasya nāmamātraṃ kathyate/ \E
%prāṇāyāmas tu sukumāreṇa sādhituṃ na śakyate atastasya nāmamātraṃ kathyate  \P
%prāṇāyāmas tu kumāreṇa sādhituṃ na śakyate// ataḥ nāma kathyate/ \B
%prāṇāyāmas tu kumāreṇa sādhituṃ na śakyate// ataḥ nāma kathyate// \L
%prāṇāyāmas tu kūmāreṇa puruṣeṇa sādhituṃ na śakyate/ ataḥ tasya nāmamātraṃ kathitaṃ/ \N1
%prāṇāyāmas tu kūmāreṇa puruṣeṇa sādhituṃ na śakyate// ataḥ tasya nāmamātre kathitaṃ// \D1
%prāṇāyāmas tu kūmāreṇa puruṣeṇa sādhituṃ na śakyate// ata tasya nāmamātre kathitaṃ/ \N2
%prāṇāyāmas tu kūmāreṇa puruṣeṇa sādhituṃ na śakyate atas tasya nāmamātre kathitaṃ \U1
%prāṇāyāmas tu kūmāreṇa          sādhituṃ na śakyate// atā tasya nāmamātraṃ kathyate// \U2
%----------------------------
%Breath-control can't be practiced by young persons. That's why it is just mentioned by name. 
%----------------------------
%pratyāhāraḥ pratyato manaḥ saṃsārānnivartyātmani sthāpyate// manomadhye [P.47] ye vikārā utpadyante/ tepi nivāraṇīyāḥ/ anekacamatkāriṇī buddhirutpadyate/ sāṃgopāṃgaṃ dhyānaṃ ca bahutaraṃ prāg uktam/ tenātra nocyate// \E XX! this one?
%pratyāhāraḥ kathyate manaḥ saṃsārān nivṛtyātmanī sthāpyate manomadhye ye vikāraḥ utpadyaṃte tepi nivāraṇīyāḥ anekacamatkāriṇi buddhir utpadyataraṃ prāg uktam tenātra nocyate  \P
%pratyāhāraḥ kathyate// manaḥ saṃsārān nivṛtyātmanī sthāpyate// manomadhye ye vikārā utpadyaṃte tepi nivāraṇīyā anekacamatkāriṇī buddhir utpadyate/ sāgopyā// dhyānaṃ ca bahutaraṃ prāg uktam tenātra nocyate// \B 
%pratyāhāraḥ kathyate   manaḥ saṃsārān nivṛttyātmanī sthāpyate// manomadhye ye vikārā utpadyaṃte tepi nivāraṇīyā anekacamatkāriṇi buddhir utpadyate sāgopyā// dhyānaṃ ca bahutaraṃ prāg uktam tenātra nocyate// \L %%%%0023.jpg
%pratyāhāraḥ kathyate// manaḥ saṃsārān nivṛtya ātmani sthāpyate/ manomadhye ye vikārā utpadyante/ tepi nivāraṇīyāḥ/ anekacamatkārakarakāraṇī buddhi utpadyate sāṃgopyāḥ/ dhyānaṃ ca bahutaraṃ uktam tena atra nocyate/ \N1
%pratyāhāraḥ kathyate// manaḥ saṃsārān nivṛtya ātmani sthāpyate// manomadhye ye vikārāḥ utpadyaṃte// tepi nivāraṇīyāḥ// anekacamatkārakāraṇī buddhi utpadyate// sāṃgopyāḥ// dhyānaṃ ca bahutaraṃ uktaṃ tena atra nocyate// \D1
%pratyāhāraḥ kathyate// manaḥ saṃsārānnivṛtya ātmani vāraṇīyāḥ// anekacamatkārakarakāraṇī buddhi utpadyate sāgopyāḥ/ dhyānaṃ ca bahuttaraṃ uktam tenātra nocyate// \N2
%pratyāhāraḥ kathyate   manaḥ saṃsārān nivṛtyātmanī sthāpyate   manomadhye ye vikārā utpadyaṃte tepi nivāraṇīyaḥ anekacamatkāriṇī buddhir utpadyate sāgaupyā dhyānaṃ bahutaraṃ uktaṃ tena atra na ucyate \U1 %%%286.jpg
%pratyāhāraḥ kathyate// manaḥ saṃsārān nivṛtyātmanī sthāpyate// manomadhye ye vikārā utpadyaṃte tepi nivāraṇīyaḥ// anekacamatkāriṇī buddhir utpadyate// sāgopyā// dhyānaṃ bahutaraṃ prāg uktaṃ tenātra nocyate// \U2
%-----------------------------
%Pratyāhāra is [when] the mind is intend on escaping from Saṃsāra and caused to remain in the self. Changes within the mind arise, but they are [supposed to be] kept off. Not just one miracle arises in the buddhi. They are secret. Dhyāna has been taught many times. Because of that is not discussed here.     
%----------------------------
%idānīṃ piṃḍabrahmāṃḍayoraikyamasti tasmāt brahmāṇḍamadhye ye padārthāstepi piṃḍamadhye santīti kathyante/ padastale talaṃ varttate/ pādopari talātalaṃ varttate/ gulphayormahātalaṃ varttate/ jaṃghāmadhye sutalaṃ varttate/ jānumadhye vitalaṃ varttate/ ūrvormadhye'talaṃ varttate// \E %[P.48]
%
%idānīṃ piṃḍabrahmāṃḍayor aikyam asti tasmād brahmāṃḍamadhye ye padārthāste piṃḍamadhye saṃti kathyate pādayos tele talaṃ varttate pādopari talātalaṃ vartate pādopari talaṃ vartate gulphayor mahātalaṃ varttate jānumadhye vitalaṃ ūrvormadhye atalaṃ \P
%
%idānīṃ piṃḍabrahmāṃḍayor ekyam asti// tasmā brahmāṃḍamadhye ye padārthāste piṃḍamadhye sati kathyate// pādayas talās talaṃ vartate// pādopari talātalaṃ vartate/ gulphayor mahātalaṃ vartate// jaṃghāmadhye stutalaṃ vartate// jānubhyāṃvitalaṃ vartate// ūrvo madhye atalaṃ vartate// \B %%%%%%%%%%%%DSCN7168.jpg Z.2
%
%idānīṃ piṃḍabrahmāṃḍayor aikyam asti// tasmāt brahmāṃḍamadhye ye padārthāste piṃḍamadhye saṃ kathyaṃte// pādayo stalās talaṃ vartate// pādopari talātalaṃ vartate// gulphayor mahātalaṃ vartate// jaṃghāmadhye sutalaṃ varttate// jānubhyāṃ vitalaṃ vartate// ūrvor madhye atalaṃ vartate// \L
%
%idānīṃ piḍabrahmāḍayoḥ aikyam asti// tasmāt brahmāṇḍamadhye ye padārthāḥ tepi piṃḍamadhye saṃti// te kathyante// padayoraṃguṣṭale talaṃ varttate/ nādupari talātalaṃ varttate/ gulpho parimahātalaṃ varttate/ jaṃghāmadhye sutalaṃ/ jānvomadhye vitalaṃ/ ūrvormadhye atalaṃ// \N1
%
%idānīṃ piḍabrahmāḍayoḥ aikyam asti// tasmāt brahmāṇḍamadhye ye padārthāḥ tepi piṃḍamadhye saṃti/ te kathyaṃte// padayoraṃguṣṭale talaṃ varttate/ tādupari talātalaṃ varttate// gulpho parimahātalaṃ varttate// jaṃghāmadhye sutalaṃ// jānvormadhye vitalaṃ// ūrvormadhye atalaṃ// \D1
%
%idānīṃ piṇḍabrahmāḍayoḥ ekamasti/ tasmānte brahmāṇḍamadhye ye padārthā tepi piṇḍamadhye saṃti te kathyante// padayoraṃguṣṭale talaṃ varttate tādupari talātalaṃ varttate/ gulpho parimahātalaṃ varttate jaṃghāmadhye sutalaṃ/ jānvomadhye vitalaṃ/ ūrvormadhye atalaṃ// \N2
%
%idānīṃ piṇḍabrahmāḍayor aikam asti tasmāt brahmāṇḍamadhye ye padārthā sarvepi piṇḍamadhye saṃti kathyate pādayor aṃguṣṭatale talaṃ ca vartate tad uparī talātalaṃ varttate gulpho parimahātalaṃ varttate jaṃghāmadhye sutalaṃ jānvormadhye vitalaṃ ūrvormadhye atalaṃ \U1
%
%idānīṃ piṇḍabrahmāḍayor aikam asti// tasmād brahmāṇḍamadhye ye padārthās tanmadhye piṇḍamadhye sati kathyaṃte// pādayoṃguṣṭatale mūlaṃ rasātalāt vartate// pādopari talātalaṃ varttate// gulphayor mahātalaṃ varttate// jaghāmadhye sutalaṃ vartate// jānumadhye vitalaṃ// ūrvormadhye atalaṃ// \U2
%-----------------------------
%
%----------------------------
%idānīṃ śarīramadhye lokatrayaṃ kathyate/ mūlādhāre bhūr lokaḥ/ liṃgāgre bhuvar lokaḥ/ liṃgamadhye svarlokaḥ// \E
%idānīṃ piṃḍamadhye lokatrayaṃ kathyate mūlādhāre bhūr lokaḥ liṃgāgre bhuvarlokaḥ liṃgamūle svarlokaḥ \P
%idānīṃ piḍopiri lokatrayaṃ kathyate// mūlādhāre bhūr lokaḥ liṃgāgre bhuvar lokaliṃgamadhye svarlokaḥ// \B
%idānīṃ piṃḍopari lokatrayaṃ kathyate// mūlādhāre bhūr lokaḥ// liṃgāgre bhuvarloka liṃgamadhye svarlokaḥ// \L
%idānīṃ piṃḍamadhye lokatrayaṃ kathyate/ mūlādhāre bhūr lokaḥ/ liṃgamūle svarlokaḥ   \N1
%idānīṃ piṃḍamadhye lokatrayaṃ kathyate// mūlādhāre bhūr lokaḥ// liṃgāgre bhuvar lokaḥ// liṃgamadhye svarlokaḥ// \D1
%idānīṃ piṃḍamadhye lokatrayaṃ kathyate/ mūlādhāre bhūr lokaḥ liṃgmūle svargalokaḥ// \N2
%idānīṃ upari tataṃ lokaṃ piṃḍamadhye lokatrayaṃ kathyate mūlādhāre bhūrlokaḥ liṃgāgre bhuvarlokaḥ liṃgamūle svaravarlokaḥ \U1
%idānīṃ piṃḍamadhye lokatrayaṃ kathyate// mūlādhāre bhūrlokaḥ// liṃgāgre bhuvarlokaḥ// liṃgamūle svarlokaḥ// \U2
%-----------------------------
%
%----------------------------
%idānīm uparitanaṃ lokacatuṣka kathyate/ pṛṣṭhadaṃḍāṃkure maharlokaḥ/ daṇḍacchidramadhye janalokaḥ/ taddaṇḍanāḍīmadhye tapolokaḥ/ daṇḍamalamadhye satyalokaḥ/ \E
%idānīm uparitanuloka catuṣkaṃ kathyate pṛṣṭhadaṃḍākūre maharlokaḥ    daṃḍaschidramadhye janalokaḥ  taddaṃḍanālimadhye tapolokaḥ daṃḍakamalamadhye satyalokaḥ \P
%idānīm uparitanulokaḥ catuṣṭayaṃ kathyate// daṃḍaṣṭaṭheṃ skure maharlokā/ daṇḍachidramadhye janaloka taddaṃḍanālikāmadhye ..polokaḥ daṇḍakamalamadhye satyalokaḥ// \B
%idānīm uparitanalokaḥ catuṣṭayaṃ kathyate// daṃḍaṣṭaṭheṃ kure maharlokaḥ/ daṇḍachidramadhye janaloka taddaṃḍatālikāmadhye tapolokaḥ daṇḍakamalamadhye satyalokaḥ// \L
%idānīṃ uparijanaṃ                  lokacatuṣkaṃ kathyate/pṛṣṭhadaṃḍāṃ kure maharllokaḥ/ daṇḍacchidramadhye janalokaḥ/ taddaṇḍanālī \om \N1 !!!!!!!!!!!!!!!!!!!!!!important omission stemmapoint S.11 verso
%idānīṃ// uparitanaṃ lokacatuṣkaṃ kathyate// pṛṣṭhadaṃḍāṃkure maharlokaḥ daṇḍachidramadhye janalokaḥ// taddaṇḍanālamadhye tapolokaḥ// daṇḍakamalamadhye satyalokaḥ// \D1
%idānīṃ uparijanaṃ                  lokacatuṣkaṃ kathyate// pṛṣṭhadaṃḍākūle maharllokaḥ/ uchidramadhye janalokaḥ/ taddaṇḍanālī - - - - - [text indicates lacunae in Vorlage] \om \N2 !!!!!!!!!!!!!!!!!!!!!!important omission stemmapoint
%idānīṃ uparitanaṃ lokaṃ catuṣkaṃ kathyate pṛṣṭhadaṃḍāṃkure maharlokaḥ daṃḍasthitamadhye janalokaḥ taddaṇḍanāḍīmadhye tapolokaḥ daṇḍamalamadhye satyalokaḥ \U1
%idānīṃ uparitanalokacatuṣkaṃ kathyate// pṛṣṭhadaṃḍāṃkure maharlokaḥ// daṃḍachidramadhye janalokaḥ// daṇḍanālimadhye tapolokaḥ// daṇḍakamalamadhye satyalokaḥ// \U2
%-----------------------------
%
%
%
%
%
%
%
%
%
%
%
%
%
%
%----------------------------
%atha brahmāṇḍamadhye caturdaśalokāni sthānāni tānyapi piṃḍe varttante// \E
%atha brahmāṇḍamadhye caturdaśalokāsthānāni tānyapi piḍe  varttate  \P
%atha brahmāṇḍamadhye caturdaśalokasthānānī tānyapi piṃḍo vartate... \B
%atha brahmāṇḍamadhye caturdaśalokasthānānī tānyapi piṃḍe vartate... \L
%\om                                                                 \N1
%atha brahmāṇḍamadhye catvāro lokasvāminaḥ// te pi piṃḍamadhye varttate \D1 %%%p. 14 recto
%\om                                                                 \N2
%atha brahmāṇḍamadhye catvāro lokāḥ svāminaḥ te pi piṃḍamadhye vartate \U1
%atha brahmāṇḍamadhye caturddaśalokāḥ stānāni// tānyapi piṃḍe vartate// \U2 %%418.jpg
%-----------------------------
%
%----------------------------
%śarīramadhye dvau kukṣī dve sakthinī vakṣaḥsthalaṃ kaṃṭhamūlaṃ kaṃṭhamadhyaṃ laṃbikāmūlaṃ tāludvāraṃ tālumadhyaṃ lalāṭamadhye śrṛṃgāṭikā kapolamadhye kamalinīmadhye brahmaraṃdhra kamalinyastrikūṭasthānam/ \E
%śarīramadhye   dvau kukṣī  dve sakṭhi??nī vakṣaḥ schalaṃ kaṃṭhamūlaṃ kaṃṭhamadhyaḥ laṃbikāmūlaṃ tāludvāraṃ tālumadhye lalāṭamadhyaṃ śṛṃgāṭikā kapolamadhye kamalinīmadhye brahmaraṃdhraṃ ūrddhvaṃ kamalinyā strikūṭasthānam \P %%%7658.jpg
%śarīramadhye// dvau kukṣau dve sakṭhinī vakṣaḥsthalaṃ kaṃṭhamūlaṃ kamardhye laṃbikāmūlaṃ tāludvāraṃ tālumadhyaṃ lalāṭamadhyaṃ//śṛṃgāṭikā// kapolamadhye// kamalinīmadhyaṃ brahmaraṃdhraṃ kamalīnyāṃ strikūṭasthānam// \B
%śarīramadhye   dvau kukṣau dve sakthinī vakṣasthalaṃ  kaṃṭhamūle  kaṃṭhamadhyaṃ laṃbikāmūlaṃ tāludvāraṃ tālamadhyaṃ lalāṭamadhyaṃ śṛṃgāṭikā karālamadhye kamalinīmadhyaṃ brahmaraṃdhraṃ ūrdhvaṃ kamalīnyā trikūṭasthānam... \L
%\om                                                                 \N1
%śarīramadhye dvau kukṣīnau vartatte// vakṣasthale// kaṃṭhasya mūle kaṃṭhamadhye laṃbikāyāmūle/ tāludvāre tālumadhye lalāṭe// śṛṃgāṭikāyāṃ kapolamadhye// kamalinīmadhye// brahmaraṃdhre// ūrdhvakamalinyaḥ// trikūṭasthāne// saptapātāle//\D1
%\om                                                                 \N2
%śarīramadhye   dvau kṣīṇau varttate vakṣasthale kaṃṭhasya mūle kaṃṭhamadhye laṃbikāyāmūle tāludvāre tālumadhye lalāṭe śṛṃgāṭikāyāṃ kapolamadhye kamalinīmadhye brahmaraṃdhre urdhvakamalinyaḥ trikūṭasthāne... \U1  %%%%287.jpg
%śarīramadhye dvau kukṣī dve sakthinī// vakṣas sthalaṃ// kaṃṭhamūle// kaṃṭhamadhyaḥ// laṃbikāmūlaṃ// tāludvāraṃ// tālumadhyaṃ// lalāṭamadhyaṃ// śrṛṃgāṭikā kapolamadhye// kamalinīmadhye// brahmaraṃdhraṃ// urdhvakamalinyā strikūṭasthānam// \U2
%-----------------------------
%
%----------------------------
%evam ekaviṃśati sthāne ekaviṃśatibrahmāṃḍāni vasaṃti// \E
%evam ekaviṃśasthānedv?ekaviṃśabrahmāni vasaṃti \P
%ekam ekaṃviṃśasthāne kekaviṃśabrahmāḍānī vasaṃtī// \B
%ekam ekaṃviṃśasthāne ??ṣce??kaviṃśabrahmāḍānī vasaṃtī// \L %%%0024.jpg
%\om                                                                 \N1
%evaṃ ekaviṃśatisthāne ekaviṃśatibrahmāṃḍāni vasaṃti// \D1
%\om                                                                 \N2
% ekāviṃśati brahmāṃḍāni vasaṃti \U1
%evam ekaviṃśasthānekaviṃśabrahmāṃḍāni vasaṃti// \U2
%-----------------------------
%
%----------------------------
%idānīṃ saptadvīpāni piṃḍamadhye kathyante// \E
%idānīṃ saptadvīpāni piṃḍamadhye kathyaṃte \P
%idānī satyadvīpāni piṃḍamadhye kathyate// \B
%idānīṃ saptadvīpāni piṃḍamadhye kathyate \L
%\om                                                                 \N1
%idānīṃ saptadvīpāni piṃḍamadhye kathyaṃte// \D1
%\om                                                                 \N2
%idānīṃ saptadvīpāni piṃḍamadhye kathyaṃte \U1
%idānīṃ saptadvīpāni piṃḍamadhye kathyaṃte// \U2
%-----------------------------
%
%----------------------------
%majjāmadhye jaṃbudvīpaḥ/ asthimadhye śākadvīpaḥ śirāmadhye śālmalidvīpaḥ/    \E
%majjāmadhye jaṃbūdvīpaḥ/ asthīmadhye śākadvīpaḥ śirāmadhye śālmalidvīpaḥ     \P
%majjāmadhye jaṃbudvīpaḥ/ astimadhye śākaladvīpaḥ// śirāmadhye śākaladvīpaḥ// \B
%majjāmadhye jaṃbudvīpaḥ  astimadhye śākaladvīpaḥ śarīramadhye śākadvīpaḥ...  \L
%\om                                                                 \N1
%majjāmadhye jaṃbudvīpaḥ// asthimadhye śākadvīpaḥ śiromadhye śālmalidvīpaḥ//  \D1
%\om                                                                 \N2
%majjāmadhye jaṃbudvīpaḥ  astimadhye śāktidvīpaḥ śīromadhye śālmalidvīpaḥ     \U1
%majjāmadhye jaṃbudvīpaḥ// astimadhye śākadvīpaḥ// śīromadhye śālmalīdvīpaḥ// \U2
%-----------------------------
%
%----------------------------
%māṃsamadhye kuśadvīpaḥ/ tvacāmadhye krauṃcadvīpaḥ/ śarīrasthalomamadhye gomedadvīpaḥ/ nakhamadhye puṣkaradvīpaḥ// etāni dvīpāni madhye tiṣṭhanti// \E [p.50]
%māṃsamadhye kuśadvīpaḥ tvacāmadhye krauṃcadvīpaḥ   śarīrasya lomamadhye gomedadvīpaḥ nakhamadhye puṣkaradvīpaḥ etāni dvīpāni guptānimadhye tiṣṭhaṃti \P
%māṃsamadhye kuśadvīpaḥ tvacāmadhye krauṃcadvīpaḥ// śarīrasya lomamadhye gomedadvīpaḥ// nakhamadhye puṣkaradvīpaḥ// etāni dvīpāni guptānimadhye tiṣṭhaṃti// \B
%māṃsamadhye kuśadvīpaḥ tvacāmadhye krauṃcadvīpaḥ   śarīrasya lomamadhye gomedadvīpaḥ  taravamadhye puṣkaradvīpaḥ etāni dvīpāni guptānimadhye tiṣṭhaṃti// \L
%\om                                                                 \N1
%māṃsamadhye kuśadvīpaḥ// tvacāmadhye krauṃcadvīpaḥ śarīrasya lomamadhye gomayadvīpaḥ/ nakhamadhye śvetadvīpaḥ/  etāni rūpaṇi guptamadhye tiṣṭhaṃti// \D1
%\om                                                                 \N2
%māṃsamadhye kuśadvīpaḥ tvacāmadhye krauṃcadvīpaḥ   śarīrasya lomadhye   gomayadvīpaḥ  taravamadhye svetadvīpaḥ etāni rūpāṇī guptamadhye tiṣṭhaṃti \U1
%māṃsamadhye kuśadvīpaḥ// tvacāmadhye krauṃcadvīpaḥ// śarīrasya lomadhye  gomedadvīpaḥ// nakhamadhye puṣkaradvīpaḥ// etāni dvīpāni guptānimadhye tiṣṭhaṃti// \U2
%-----------------------------
%
%----------------------------
%idānīṃ piṃḍamadhye saptasamudrāḥ kathyante// prasvedamadhye kṣārasamudraḥ/ lalāṭamadhye kṣīraḥ samudraḥ/ vāṅmadhye madhusamudraḥ/ kaphamadhye dadhisamudraḥ/ medomadhye ghṛtasamudraḥ/ rasamadhye ikṣurasasamudraḥ// vīryamadhye svādusamudraḥ/ pādamadhye kūrmasthānam// \E
%
%idānīṃ piṃḍamadhye saptasamudrāḥ kathyaṃte prasvedamadhye kṣārasamudraḥ lālāmadhye kṣīrasamudraḥ vasāmadhye madhusamudraḥ kaphamadhye dadhisamudraḥ medomadhye ghṛtasamudraḥ raktaamadhye ikṣurasasamudraḥ vīryamadhye svādudakasamudraḥ pādamadhye kūrmasthānam \P
%
%idānīṃ piṃḍamadhye samudrāḥ kathyate// prasvedamadhye kṣārasamudraḥ// lalāṭamadhye kṣīrasamudraḥ// raktamadhye vasāmadhye madasamudraḥ kaphamadhye dadhisamudraḥ// medomadhye ghṛtasamudraḥ// ikṣusamudraḥ/ vīryamadhye svādukasamudraḥ/karmasthāna pādasamadhye/ \B
%
%idānīṃ piṃḍamadhye samudrāḥ kathyaṃte// prasvedamadhye sārasasamudraḥ// lalāṭamadhye kṣīrasamudraḥ// raktamadhye vasāmadhye madyasamudraḥ// kaphamadhye dadhisamudraḥ// medamadhye ghṛtasamudraḥ// ikṣusamudraḥ// vīryamadhye svādukasamudraḥ// karmasthāna pādamadhye/ \L
%
%\om                                                                 \N1
%idānīṃ piṃḍamadhye saptasamudrāḥ kathyete// prasvedamadhye kṣārasasamudralalāṭamadhye kṣīrasamudraḥ/ vasāmadhye dadhisamudraḥ// medamadhye ghṛtasamudraḥ// vasāmadhye madhusamudraḥ// raktamadhye ikṣusamudraḥ// vīryamadhye amṛtasamudraḥ/ pādamtale kūrmasthānaṃ/ \D1
%\om                                                                 \N2
%
%idānīṃ piṃḍamadhye saptasamudrāḥ kathyaṃte svedamadhye kṣārasasamudraḥ lalāṭamadhye kṣīrasamudraḥ vasāmadhye dadhisamudraḥ medamadhye ghṛtasamudraḥ vasāmadhye madhusamudraḥ raktamadhye ???samudraḥ vīryamadhye mṛtasamudraḥ pādamadhye kūrmasthānaṃ \U1 %%%288.jpg
%
%idānīṃ piṃḍamadhye saptasamudrāḥ kathyaṃte// prasvedamadhye kṣārasāgaraḥ// la???madhye kṣīrasamudraḥ// vīryamadhye svāduḥ samudraḥ// majjāmadhye madhusamūdraḥ// kaphamadhye dadhisamudraḥ// medamadhye ghṛtasamudraḥ// raktamadhye ikṣurasamudraḥ// \U2
%-----------------------------
%
%----------------------------
%idānīṃ navadvāreṣu nāsikayoḥ kinnarakhaṃḍanaraharikhaṃḍauḥ netrayoḥ ketumāla bhadrāśvau/ karṇayoḥ hiraṇmayakhaṃḍaramyakakhaṃḍau/ gude kurukhaṃḍaḥ liṃge ilāvṛtakhaṇḍaḥ// \E  [p.51]
%idānīṃ navadvāreṣu navakhaṃjani? kathyaṃte mukhe bharatakhaṃḍaḥ 1 nāsikayoḥ kinarakhaṃḍe 3 netrayoḥ ketumāla bhadrāśve 4 karṇayor hiraṇmayaramyaka khaṃdaḥ 5 gude kurukhaṃḍaḥ 6 liṃge ilāvṛtaḥ 7  \P 7659.jpg!!!
%idānīṃ navakhaṃḍāni kathyaṃte/ mukhe bharatakhaṃḍaḥ nāsikayor madhye kināraharikhaṃḍā/ netrayo ketumāla bhadrāsve/ karṇayor hiraṇyamayaramyakhaṃḍaḥ/ gude kurukhaṃḍāḥ/ liṃge iḍṛttaṃ??/ \B DSCN7169.JPG Z.4
%idānīṃ navakhaṃḍāni kathyaṃte// mukhe bharatakhaṃḍaḥ// nāsikayor madhye kinārasiṃhakhaṃḍā netrayo ketumālabhadrāsve// karṇayor hiraṇyamayaramyakhaṃḍaḥ gude kurukhaṃḍāḥ// liṃge ilāvṛtaṃ// \L   0025.jpg
%\om                                                                 \N1
%idānīṃ navadvāramadhye navaṣaṃḍāḥ kathyaṃte// bharataṣaṃḍaḥ/ kāśmīraṣaṃḍaḥ/ strīmaṃḍalaṣṃḍaḥ/ dvijakṣaṃḍaḥ/ ekapādaṣaṃḍaḥ/ rākṣasaṣaṃḍaḥ ghāṃdhāraṣaṃḍaḥ// kaivarttaṣaṃḍaḥ// garbhaṣaṃḍhaḥ// \D1 %%%p.14 verso
%\om                                                                 \N2
%idānīṃ navadvāramadhye navakhaṃḍāḥ kathyate bharatakhaṃḍaḥ kāsmīrakhaṃḍaḥ strīmaṃḍalakhaṃḍaḥ ???dvīttakhaṃḍaḥ yekapādakhaṃḍaḥ rākṣasakhaṃḍaḥ ghaṃdhārakhaṃḍaḥ kaivartakhaṃḍaḥ garbhakaṃḍhaḥ \U1
%idānīṃ navadvāreṣu  navakhaṃḍāṇi  kathyaṃte// pādamadhye kūrmasthānaṃ// mukhaṃ bhāratakhaṃḍaṃ// nāsikayoḥ// kinnara// harikhaṃḍa// netrayoḥ// ketumāla// bhadraśvekarṇayoḥ// hiraṇmaya// ramyakakaṃḍe// gudekurukhaṃḍaṃ// liṃge ulāvṛtaṃ// evaṃ navakhaṃḍāḥ//    \U2
%-----------------------------
%
%----------------------------
%idānīmaṣṭamakulaparvatāḥ kathyante/ \E
%idānīm aṣṭakulaparvatāḥ kathyaṃte   \P
%idānīm aṣṭamakulaparvatāḥ kathyaṃte// \B
%idānīm aṣṭamakulaparvatāḥ kathyaṃte// \L
%\om                                                                 \N1
%idānīṃ piṃḍamadhye aṣṭakulaparvatāḥ kathyaṃte// \D1
%\om                                                                 \N2
%idānīṃ piṃḍamadhye aṣṭakulaparvatāḥ kathyaṃte \U1
%idānīm aṣṭakulaparvatā kathyaṃte// \U2
%-----------------------------
%
%----------------------------
%merudaṇḍamadhye merumaṃdaraḥ/ brahmakapāṭamadhye kailāsaḥ/ \E
%merudaṇḍamadhye merumaṃdaraḥ  brahmakapāṭamadhye kailāsaḥ  \P
%merudaṇḍamadhye merumaṃdaraḥ/ brahmakapāṭamadhye kailāsaḥ/ \B
%merudaṇḍamadhye merumaṃdaraḥ/ brahmakapāṭamadhye kailāsaḥ/ \L
%\om                                                                 \N1
%merudaṃḍamadhye merumparvataḥ// brahmakapāṭamadhye kailāsaparvataḥ \D1
%\om                                                                 \N2
%merudaṇḍamadhye merumparvattaḥ brahmakapāṭamadhye kailāsaparvataḥ \U1
%merudaṇḍamadhye merumaṃdaraḥ// brahmakapāṭhamadhye kailāsaḥ// \U2 %%%419.jpg 
%-----------------------------
%
%----------------------------
%pṛṣṭhamadhye himācalaḥ/ vāmaskandhe malayācalaḥ/ dakṣiṇaskandhe mandarācalaḥ/ dakṣiṇakarṇe vindhyācalaḥ/  \E
%pṛṣṭhaṃ madhye himācalaḥ vāmaskaṃdhe malayācalaḥ dakṣiṇaskaṃdhe maṃdarācalaḥ dakṣiṇakarṇe vindhyācalaḥ    \P
%pṛthvīamadhye himācalaḥ/ vāmaskaṃdhe malayācalaḥ/ dakṣiṇaskaṃdhe maṃdarācalaḥ/ dakṣiṇakarṇe vindhyācalaḥ/ \B
%pṛthvīmadhye himācalaḥ/ vāmaskaṃdhe malayācalaḥ/ dakṣiṇaskaṃdhe maṃdarācalaḥ/ dakṣiṇakarṇe viṃdhyācalaḥ/  \L
%\om                                                                 \N1
%paiṭimadhye himācalaḥ//    parvataḥ vāmaskaṃdhe malayācalaḥ dakṣaṇaskaṃdhe maṃdarācalaḥ dakṣaṇakarṇe viṃdhyācalaḥ  \D1
%\om                                                                 \N2
%paiṭhamadhye himācalaparvataḥ vāmaskaṃdhe malayācalaḥ dakṣaṇaskaṃdhe maṃdarācalaḥ dakṣaṇakarṇe viṃdhyācalaḥ  \U1
%pṛṣṭhamadhye himācalaḥ// vāmaskandhe malayācalaḥ// dakṣiṇaskandhe mandarācalaḥ// dakṣiṇakarṇe vindhyācalaḥ//  \U2
%-----------------------------
%
%----------------------------
%vāmakarṇe mainākaḥ/ lalāṭamadhye śrīśailaḥ/ apare śailāḥ hastayoḥ pādayor aṃgulīnāṃ mūleṣu varttaṃte// \E
%vāmakarṇe mainākaḥ  lalāṭamadhye śrīśailaḥ  apare śailā hastayoḥ  pādayor aṃgulīnāṃ mūleṣu varttaṃte  \P
%vāmakarṇe mainākaḥ/ lalāṭamadhye śrīśailāsaḥ/ apare śailā hastayoḥ/ pādayor aṃgulimūleṣu vartate// \B
%vāmakarṇe mainākaḥ/ lalāṭamadhye śrīśailaḥ/ apare śailā hastayoḥ/ pādayor aṃgulīmūleṣu vartate/ \L
%\om                                                                 \N1
%vāmakarṇe mainākaḥ// lalāṭamadhye śrīśailaḥ// apare parvatāḥ hastayoḥ pādayor aṃgulīnāṃ madhye vartatte// \D1
%\om                                                                 \N2
%vāmakarṇe mainākaḥ  lalāṭamadhye śrīśailaḥ  apare parvatāḥ hastayoḥ pādayor aṃgulībhyāṃ madhye parvate \U1
%vāmakarṇe mainākaḥ lalāṭamadhye śrīśailaḥ// apare śailāḥ// hastayoḥ pādayor aṃgulīnāṃ mūleṣu vartaṃte// \U2
%-----------------------------
%
%----------------------------
%idānīṃ śarīramadhye nava nāḍyastiṣṭhanti tanmadhye navanadīnāṃ sthānāni varttante/ \E
%idānīṃ śarīre navanaḍyas tiṣṭhaṃti tanmadhye navāṃnā nadīnāṃ sthānāni vartaṃte \P
%idānīṃ śarīre navanaḍyas tiṣṭhanti// tanmadhye navānāṃ nadīnāṃ sthānāni vartate/ \B
%idānīṃ śarīre navanaḍyaḥ tiṣṭhaṃti/ tanmadhye navānāṃ nadīnāṃ sthānāni vartaṃte/ \L
%\om                                                                 \N1
%idānīṃ śarīre ṇavānāḍyas tiṣṭhati// tanmadhye navānāṃ nadīnāṃ sthānāni vartraṃte//  \D1
%\om                                                                 \N2
%idānīṃ śarīre ṇavānaḍyaḥ stiṣṭhaṃti tanmadhye navānāṃ nadīnāṃ sthānāni vartaṃte  \U1
%idānīṃ śarīramadhye nava nāḍyas tiṣṭhati// tanmadhye navānāṃ nadīnāṃ nivarttaṃte// \U2
%-----------------------------
%
%----------------------------
%gaṃgāyamune vitastā candrabhāgā sarasvatī vipāśā śatahradā irāvatī narmadā/   \E [p.52]
%gaṃgāyamunā vitastā caṃdrabhāgā sarasvatī vipāśā śātahṛdā irāvati narmmadā    \P
%gaṃgāyamunā vitastā caṃdrabhāgā sarasvatī vipāśā śāśatahṛdā irāvati narmadā/  \B
%gaṃgāyamunā vitastā caṃdrabhāgā sarasvati vipāśā śatat hṛda irāvati narmadā// \L
%\om                                                                 \N1
%gaṃgāyamunā vitastā caṃdrabhāgā sarasvatī/ vaipaśā śata hṛdā// irāvatī/ narmadā/ \D1
%\om                                                                 \N2
%gaṃgāyamunā vitastā caṃdrabhāgā sarasvatī vaipaśā śata hṛdā airāvati narmadā \U1
%gaṃgāyamunā vitastā candrabhāgā sarasvatī vipāśā śātadrumā// narmadā   \U2
%-----------------------------
%
%-----------------------------
%aparā nadyo nadāni srotāṃsi taṭākāni vāpīkūpādi saptatisahasranāḍīmadhye tiṣṭhanti/ \E
%aparā nadyo nadānir jārāsrotāṃ   sītaṭānī vāpīkūpādvisaptatī sahasranāḍīnāṃ madhye tiṣṭaṃti  \P %7660.jpg
%aparā nadyo nadānir jñārāstyetāṃ sītaṭānī vāpīkūpādvisaptatī sahasranāḍīnāmadhye tiṣṭaṃti/ \B
%aparā nadyo nadānir jñārāstyetāṃ sitaṭāni vāpīkūpādvisaptatisahasranāḍīnāṃ madhye tiṣṭaṃti/ \L
%\om                                                                 \N1
%aparā nadyopa nadīńair bhu?rasrotataṭākavāpikupāḥ dvisaptatisahasranāḍīnāṃ madhye tiṣṭaṃti/  \D1
%\om                                                                 \N2
%gaṃḍakīnadhūpanadīnair bhurasrotataḍaga   vāpīkūpadvisaptatisahastranāḍīmadhye  tiṣṭhaṃṭī \U1
%aparā nadyo nadānir jñārāsrotā   sītaṭhānī vāpīkūpādvisaptati sahasranāḍīnāṃ madhye tiṣṭaṃti//  \U2
%-----------------------------
%
%-----------------------------
%saptaviṃśatinakṣatrāṇi dvisaptatikoṣṭhakābhyantare vasaṃti/    \E
%saptaviṃśatinakṣatrāṇi dvisaptatikoṣṭhakāṃtrābhyaṃtare vasaṃti     \P
%saptaviṃśatinakṣatrāṇi/ dvisaptatīkoṣṭhākāṃtrābhyāṃtare vasaṃti//    \B
%saptaviṃśatinakṣatrāṇi dvisaptatīkoṣṭākāṃtrābhyāṃtare vasaṃti    \L
%\om                                                                 \N1
%saptaviṃśatinakṣatrāṇi dvisaptatikoṣṭhakāścāṃ? abhyaṃtare vasṃati// \D1
%\om                                                                 \N2
%saptaviṃśatinakṣatrāṇi dvisaptatikoṣṭākāścāṃtrābhyaṃtare vasati    \U1 %%%289.jpg
%saptaviṃśatinakṣatrāṇi dvisaptatikoṣṭhakāṃtarābhyaṃtare vasaṃti//    \U2
%-----------------------------
%
%-----------------------------
%dvādaśa rāśayaḥ/ meṣaḥ vṛṣaḥ mithunaḥ karkaḥ siṃhaḥ kanyā tulā vṛściko dhanurmakarakumbhamīnāḥ/ \E
%dvādaśa rāśayaḥ meṣavṛṣamithūnaḥ karkasiṃhakanyātūlavṛścikadhanamakarakuṃbhamīna \P
%dvādaśa rā?śayāḥ/ meṣavṛṣabhamithūnakarkasiṃhakanyātūlavṛścikadhanamakarakuṃbhamīnaḥ// \B
%dvādaśa rā?śayaḥ meṣavṛṣamithunakarkasiṃhakanyātūlavṛścikadhanamakarakuṃbhamīnaḥ// \L
%\om                                                                 \N1
%dvādaśa rāśayaḥ// meṣavṛṣamithunakarkasiṃhakanyātūlavṛścikadhanamakarakuṃbhamīna \D1
%\om                                                                 \N2
%dvādaśa rāśayaḥ meṣavṛṣamithunakarkasiṃhakanyātūlavṛścikadhanamakarakuṃbhamīna \U1
%dvādaśa rāśayaḥ// meṣa// vṛṣabha// mithuna// karka// siṃha// kanyā// tula// vṛścika// dhana// makara// kuṃbha// mīna// \U2
%-----------------------------
%The twelf zodiacal signs (rāśi) are: 
%-----------------------------
%navagrahāḥ/ ādityasomamaṃgalabudhaguruśukraśanirāhuketavaḥ/ paṃcadaśatithayotra madhye vasaṃti// \E [P.53]
%navagrahaḥ ādityasomamaṃgalabudhabṛhaspatiḥ śukraśaniḥ rāhuḥ ketuḥ paṃcadaśatithayoṃ tra madhye vasaṃti \P
%navagrahāḥ// ādityasomamaṃgalabudhabṛhaspatiśukraśanirāhuketu// paṃcadaśatithiḥ// atra madhye vasaṃti// \B %%%DSCN7170.jpg Z.1
%navagrahāḥ// ādityasomamaṃgalabudhabṛhaspatiśukraśanirāhuketu   paṃcadaśatithayaḥ// atra madhye vasaṃti// \L  %%%0026.jpg
%\om                                                                 \N1
%navagrahāḥ// ādityasoma/ maṃgala/ budha/ bṛhaspati/ śukraśani rāhuketu/ paṃcadaśatippya?yyoṃ? tramadhye vasaṃti \D1 %%%p.14 verso drittletzte Zeile
%\om                                                                 \N2
%navagrahāḥ   ādityasomamaṃgalabudhabṛhaspatiśukraśaniirāhuketu.h paṃcadaśatithayo ātramadhye vasaṃti \U1
%navagrahāḥ/ ravi//caṃdra//maṃgala//budha// vṛhasyati// śukra// śanī// rāhu// ketuḥ// padaśatithayoṃtra madhye tiṣṭhaṃti// \U2
%-----------------------------
%
%-----------------------------
%                     yathā samudramadhye laharī varttate/ tathā śarīramadhye kūrmmīnāma laharī bhavati/ \E
%                     yathā .................................... sarīramadhye urmīnāma laharī bhavati \P
%                     yathā samudramadhye laharā vartate/  tathā śarīramadhye urmmīnāma laharī bhavati/ \B
%                     yathā samudramadhye laharī vartate//  tathā śarīramadhye urmmī nāma laharī bhavatī/ \L
%                     \om                                                                 \N1
%                     yathā samudramadhye laharī varttate/ tathā śarīramadhye ūrmīnāma laharī bhavati/ \D1
%\om                                                                 \N2
% pīṭhasya romamadhye yathā samudramadhye laharī vartate     tathā śarīramadhye urmi nāma laharī bhavati  \U1
%                     yathā samudramadhye lahari varttate//  tathā śarīramadhye urmmī nāma laharī bhavaṃti// \U2
%-----------------------------
%
%-----------------------------
%ūrmyaścalāstataḥ          calanaṃ bhavati/              tanmadhye samagraṃ tārāmaṇḍalaṃ varttate/ trayastriṃśatkoṭidevatāḥ/ bāhuromamadhye vasaṃti/ \E
%ūrmyaścalāścataḥ śarīre calanaṃ bhavati    dhāvanaṃ ca  tanmadyhe samagraṃ tārāmaṇḍalaṃ varttate  trayastriṃśatkoṭyo devatāḥ bāhuromamadhye vasaṃti    \P
%ūrmmīścalāścataḥ// śarire calanaṃ bhavati/ dhāvanaṃ ca/ tanmadhye samagrāṃ tārāmaṇḍalaṃ vartate/  trayastriṃśatkoṭayo devatāḥ/ bāhuromamadhye vasaṃti// \B
% dhāvanaṃ ca/ tanmadhye samagraṃ tārāmaṇḍalaṃ vartate/  trayastriṃśatkoṭayo devatāḥ/ bāhuromamadhye vasaṃti// \L
% \om                                                                 \N1
%tasyāḥ urmyaḥ calācharīre calanaṃ bhavati/ dhāvanaṃ bhavati// tanmadhye samagraṃ tārāmaṇḍalaṃ varttate  trayastriśatkoṭyo devatā bāhuromamadhye vasaṃtī// \D1 %%%p.15 recto 
%\om                                                                 \N2
%tathā urmeś ca lanāśarīre calanaṃ bhavati  dhāvanaṃ bhavati tanmadhye samagra tārāmaṇḍalaṃ vartate  trayaḥ striśatakoṭī devatā bāhuromamadhye vasaṃtī \U1
%ūrmiyaś ca lāḥ// tataḥ śarīracalanaṃ bhavati//  dhāvanaṃ ca// tanmadhye samagratārāmaṇḍalaṃ vartate//  trayaḥ triṃśatkoṭyo devatāḥ// bāhuromamadhye vasaṃti// \U2
%-----------------------------
%
%-----------------------------
%hṛdayaromamadhye takṣakaḥ mahānāgaḥ/ śaṃkhaḥ takṣakaḥ/ vāsukiḥ/ anantaśeṣaḥ ete nāga vasaṃti/               udararomamadhye apare nāgā vasaṃti    guṇagandharvakinnarāpsaro vidyādharaguhyakāḥ/ \E
%ṣṭaṣṭaromamadhye ṣaḍaśī sahasradivyatapasvinaḥ pīṭho papīṭhe dvavoṣṭo pariyāniromāṇi tanmadhye vasaṃti hṛdayaromamadhye takṣakamahānāga karkoṭakaḥ śaṃkhaḥ pulakaḥ vāsukiḥ anaṃtaḥ śoṣa ete nāgā vasaṃti udararomamadhye apare nāgā vasaṃti guṇagaṃdharvakinarā ...\P
%ṣṭaṭīromamadhye ṣḍaśatī sahastra divyatapaścinaḥ mīṣṭho papīṭher dvaiṣṭho pariyānitomāṇi tanmadhye vasaṃti/ udararomamadhye apare nāgā vasaṃti//  guṇagaṃdharvakinnarābharo vidyādharaguhyakāḥ... \B
%pṛṣṭīromamadhye ṣaḍaśatī sahastra divyatapaścinaḥ pīṭho papīṭhe  dvaiṣṭhi pariyāromāṇi tanmadhye vasaṃti//  udararomamadhye apare nāgā vasaṃti    guṇagaṃdharvakinnarābharo vidyādharaguhyakāḥ... \L
%\om                                                                 \N1
%\om                                                                 \N2
%pṛṣṭīromamadhye ṣaḍaśīti sahastra divyatapascino pīṭhamahāpīṭhau urdhvapṛṣṭho pariyāni romāni tanmadhye saṃti hṛdayaromamadhye takṣamā nāgaḥ karkoṭaḥ śaṃkhaḥ pulikaḥ vāsukī ānaṃta śoṣa ete nāgā vasaṃti udararomamadhye pare nāgā vasaṃti gaṇagaṃdharvakinnarapuruṣāsparovidyādharaguhyaka \U1
%pīṭhasya romamadhye ṣaḍaśīti sahasra divyatapascino pīṭhamahāpīṭhau ūrddhva tuṣṭo pariyoni??? romāṇi tanmadhye vasaṃti// hṛdayaromamadhye takṣakamahānāgaḥ// karkoṭakaḥ/ śaṃkhaḥ/ pulika/ vāsukī/ ānaṃta/ śeṣā ete nāgā vasaṃti udararomamadhye/ apare nāgā vasaṃti// gaṇagaṃdharvakiṃnarakiṃpuruṣa// apsarovidyādhāra/ guhyaka \D1
%pṛṣṭaromamadhye ṣaḍaśīti sahasra divyatapas vinaḥ// pīṭho papīṭhordhvapariyāt?/n?i romāṇi tanmadhye vaṃsaṃti// hṛdayaromamadhye takṣakaḥ mahānāgaḥ// karkoṭakaḥ// śaṃkhaḥ// kulakaḥ// vāsukiḥ// ānaṃta// śeṣaḥ// ete nāgā vasaṃti// udararomamadhye apare nāgā vasaṃti// gaṃdhagaṃdharvakinnara??puro??vidyādharaguhyakaḥ// \U2 %%420.jpg 
%-----------------------------
%%%%%%!!!roma? einer der beiden Hauptkanäle, aber der tibetische Begriff?!?!?! prüfen! 
%-----------------------------
%śarīramadhye anekatīrthāni vasaṃti/ aśrupātamadhye meghamaṇḍalaṃ vasati/ anaṃtāḥ siddhayo buddhayaś ca prakāśamadhye varttante/ \E
%      madhye nekatīrthā valī vasaṃti/ aśrupātamadhye meghamaṇḍalaṃ vasati anaṃtāḥ siddhayo buddhayaś ca prakāśamadhye varttante \P
%śarīramadhye anekatīrthāvalī vasaṃtī// aśrupātamadhye meghamaṇḍala vasaṃtī anaṃtā siddhayo buddhayac ca/ prakāśamadhye vartate/ \B
%śarīramadhye anekatīrthāvalī vasaṃtī// aśrupātamadhye meghamaṇḍalaṃ vasatī anaṃtā siddhayo buddhayaś ca prakāśamadhye vartate// \L
%\om                                                                 \N1
%śarīmadhye karmasthāne nenekatīrthavallī vasaṃti// aśrupātamadhye meghamaṃḍalaṃ vasaṃti// anaṃtāḥ siddhayo buddhayaś ca prakāśamadhye varttate// \D1
%\om                                                                 \N2
%śarīramadhye marmasthāne naikatīrthavallī vasaṃtī aśrupātamadhye meghamaṃḍalaṃ vasaṃti anaṃtā siddhayo budhayaś ca prakāśamadhye vartate \U1 %%%290.jpg
%śarīramadhye 'nekatīrthāvalī vasatī// aśrupātamadhye meghamaṃḍalaṃ vasati// anaṃtā siddhayo buddhayaś ca prakāśamadhye vartante// \U2
%-----------------------------
%
%-----------------------------
%caṃdrasūryau dvayor netrayormadhye varttete/ anekavanaspatigulmalatātṛṇāni jaṃghāromamadhye vasaṃti/ \E %%%[p.54]
%caṃdrasūryau dvayor netreyor madhye vartate  anekavanaspatigulmalatātṛṇāni jaṃghāromamadhye vasaṃti \P
%caṃdrasūryo dvayānetrayomadhye vartate// anekavanaspatigulmalatātṛṇāni jaṃghāroramadhye vasaṃti// \B
%caṃdrasūryo dvayo netrayor madhye vartate// anekavanaspatigulmalatātṛṇāni jaṃghāroramadhye vasaṃti... \L
%\om                                                                 \N1
%caṃdrasūryo dvayor netrayormadhye varttate// anaikavanaspatigulmatṛṇāni jaṃghāromasthāne varttaṃte/ \D1
%\om                                                                 \N2
%caṃdrasūryau netradvaya vasaṃti anekavanaspatī gulmalatāni jaṃghāromamadhye vasaṃti \U1
%caṃdrasūryau dvayo netrayoḥ madhye pravartate// anekavana/spatigulmalatātṛṇāni jaṃghāromamadhye vasati// \U2
%-----------------------------
%
%-----------------------------
%puruṣasya nṛtyadarśanāt gītaśravaṇāt/vallabhavastuno darśanāt/ yaḥ ānanda utpadyate saḥ svargalokaḥ kathyate/ rogapīḍito durjanebhyaḥ puruṣasya yat duḥkham utpadyate tadbahutaraṃ narakaṃ kathyate// \E
%puruṣasya nṛtyadarśanāt gītaśravaṇāt vallabhavastuno darśanāt  ya  ānanda utpadyate svargalokaḥ kathyate      rogapīḍato durjjanebhya  puruṣasya yaduḥkham utpadyate tadbahutaraṃ narakaṃ kathyate \P
%puruṣasya nityadarśanāt gītaśravaṇāt/ vallabhavastuno darśanāt/ yaḥ ānanda utpadyate svargalokaḥ kathyate     rogapīḍato durjanebhya  puruṣasya yat duḥkha utpadyate// tadbahutaraṃ narakaṃ kathyate// \B
%puruṣasya nityadarśanāt gītaśravaṇāt  vallabhavastuno darśanāt  yaḥ ānanda utpadyate svargalokaḥ kathyate     rogapīḍano durjanebhya  puruṣasya yad duḥkhaṃ utpadyate// tadbahutaraṃ narakaṃ kathyate// \L
%\om                                                                 \N1
%puruṣasya nṛtyadarśanād gītaśravaṇāt  vallabhavastuno darśanāt  yaḥ ānanda utpadyate sa bahurānaṃdaḥ svargaphulaḥ?? kathyate/ rogapīḍā durjanebhyaḥ puruṣasya duḥkhaṃ utpadyate// tat bahutaraṃ nakaṃ kathyate/ \D1
%\om                                                                 \N2
%puruṣasyā vādya?  nṛtyogdītaśravaṇād  vallabhavasttuno darśanād yā ānanda utpadyate  sa bahurānaṃdaḥ svargaphalaḥ? kathyate   rogapīḍa durjanebhyaḥ  puruṣasya duḥkham utpadyate bahutaraṃ narakaṃ kathyate \U1
%puruṣasya darśanāt// gītaśravaṇāt// vallabhavastuno darśanāt// ānanda utpadyate sa svargaloka kathyate// rogapīḍāto durjanebhyaḥ puruṣasya duḥkha utpadyate tadbahutaraṃ narakaṃ kathyate// \U2
%-----------------------------
%
%-----------------------------
%                                                                                                     atha ca yatkarmakaraṇāt manomadhye śaṃkā na bhavati tatkarma muktikāraṇam/ \E
%                                                                                                     atha ca yatkarmakaraṇān manomadhye śaṃkā na bhavati tatkarmamuktikāraṇam   \P %%%7662.jpg 
%                                                                                                     atha ca yatkarmakaraṇāt manobudhye śaṃkā na bhavati tatkarma kamuktikāraṇam// \B
%                                                                                                     atha ca yatkarmakaraṇāt manobudhye śaṃkā na bhavati tatkarma kamuktikāraṇam// \L
%                                                                                                     \om                                                                 \N1
%                                                                                                     atha ca yatkarmakaraṇāt manomadhye śaṃkā na bhaviti tatkarma muktikāraṇaṃ// \D1
%                                                                                                     \om                                                                  \N2
%atha ca yatkarmakaraṇāt sarveṣāṃ lokānāṃ svamanasī ca śubhaṃ na bharate tatkarma baṃdhanam ityucyate atha ca yatkarmakaraṇāt manomadhye śaṃkā na bhavati tatkarmamuktikāraṇam \U1
%                                                                                                     atha ca yatkarmakaraṇān manomadhye śakā na bhavaṃti// tatkarmamuktikāraṇaṃ// \U2
%-----------------------------
%
%----------------------------
%idānīṃ rājayogāccharīre yādṛśāni cihnāni bhavanti tāni kathyante// \E
%idānī  rogayogācharīre etādṛśāni cihnāni bhavaṃti tāni kathyaṃte   \P
%idānī  rājayogāccharīre// etādṛśāni cihnāni bhavaṃti// tāni kathyaṃte// \B 7170.jpg end 7171.jpg beginning
%idānīṃ rājayogāccharīre  etādṛśāni cihnāni bhavaṃti tāni kathyaṃte// \L
%\om                                                                 \N1
%idānīṃ rājayogāccharīre  etādṛśāni cihnāni bhavaṃti// tāni kathyaṃte// \D1
%\om                                                                 \N2
%idānīṃ rājayogācharīre  etādṛśāni cihnāni bhavaṃti tāni kathyaṃte \U1
%idānī  rājayogāśarīre   etādṛśāni cihnāni bhavaṃti// tāni kathyaṃte// \U2
%-----------------------------
%
%-----------------------------
%sakalaroganāśaḥ sakalapṛthvīṃ paśyati/ tadanaṃtaraṃ jñānam utpadyate// \E [p.55]
%sakalaroganāśaḥ sakalāṃ pṛthvīṃ paśyati tadaṃtaraṃ jñānam utpadyate \P
%sakalaroganāśaḥ sakalapṛthvīṃ paśyatī/ tadanaṃtaraṃ jñānam utpadyate// \B
%sakalaroganāśaḥ sakalapṛthvīṃ paśyati/ tad anaṃtaraṃ jñānam utpadyate// \L
%\om                                                                 \N1
%sakalaroganāśaḥ sakalapṛthvīṃ paśyatī/ tadanaṃtaraṃ tatvaviṣayaṃ jñānam utpadyate/ \D1 %%%p. 15 verso 
%\om                                                                 \N2
%sakalarogaḥ nāśaḥ sakalapṛthvīṃ paśyati tad anaṃtaraṃ tatvaviṣayaṃ jñānam utpadyate \U1
%sakalaroganāśaḥ sakalapṛthvīṃ paśyati// tadanaṃtarajñānam utpadyate// \U2
%-----------------------------
%
%-----------------------------
%samagrā bhāṣā jānāti/ tataḥ puruṣasya deho vajramayo bhavati/ sarpadaṃśena maraṇaṃ na bhavati/   \E
%samagrāṃ bhāṣāṃ jānāti tataḥ puruṣasya deho vajramayo bhavati sarpadaṃśo maraṇaṃ na bhavati      \P
%samagrā bhāṣa jānāti tataḥ puruṣasya deho vajramayo bhavati// sarpadaṃśema maraṇaṃ na bhavatī/   \B
%samagrabhāṣā  jānāti tataḥ puruṣasya deho vajramayo bhavati// sarpadaṃśe maraṇaṃ bhavati//       \L
%\om                                                                 \N1
%samagrāṃ bhāṣāṃ jānāti/ tataḥ puruṣasya deho vajramayo bhavati/ sarpadaṃśe satī maraṇaṃ na bhavati/ \D1
%\om                                                                 \N2
%samagrāṃ bhāṣāṃ jānāti tataḥ puruṣasya deho vajramayo bhavati  sarpadaṃśe satī maraṇaṃ na bhavati \U1
%samagrā bhāṣā jānāti// tataḥ puruṣasya deho vajramayo bhavati//sarpadaṃśe maraṇaṃ na vati//   \U2
%-----------------------------
%
%-----------------------------
%tataḥ puruṣasya bubhukṣāpipāsānidrollatāśītoṣṇatā bādhāṃ na kurvanti/ \E
%tataḥ puruṣasya bunnukṣāpipāsāni drolmatāśītatābādhā na kurvaṃti \P
%tatpuruṣasya babhukṣāpipāsānidrollatāśītabādhā na kurvanti/ \B
%tatpuruṣasya babhukṣāpipāsānidroṣṇatāśītabādhā na kurvanti... \L
%\om                                                                 \N1
%tataḥ puruṣasya bubhukṣāpipāsānidrā/ uṣṇatā// śīta nā bādhāṃ na kuroti???/ \D1
%\om                                                                 \N2
% \om                                                         \U1
%tataḥ puruṣasya bubhukṣāpipāsānidroṣṭṇatāśīta bādhāṃ na kurvaṃti// \U2
%-----------------------------
%
%-----------------------------
%vāksiddhir bhavati/ vidyatpāte kācidbādhāpi na bhavati// \E
%vāksiddhir bhavati                                        \P
%vāksiddhir bhavatī/ vidyutpāte kācidglānir  na bhavati// \B
%vāksiddhir bhavati/ vidyutpāte           kācidglānir  na bhavati// \L
%\om                                                                 \N1
%vāksiddhir bhavati/ vidyutpāte śarīre na kiṃcid glānir bhavati/ \D1
%\om                                                                 \N2
% vidyutpāte śarīre    kvācidglānir? na bhavati  \U1
%vāksiddhir bhavati// vidyut pāte kāciddhānir? na bhavati// \U2
%-----------------------------
%
%-----------------------------
%tadanaṃtaraṃ  pavanarūṣī puruṣī bhavati/  samagrāṃ pṛthvīṃ dṛṣṭyā paśyati/   aṇimādyaṣṭasiddhir bhavati/ \E
%tadanaṃtaraṃ  pavanarūpī puruṣo bhavati   samagrāṃ pṛthvīṃ dṛṣṭyā paśyati    aṇimādyaṣṭasiddhir bhavati  \P
%tadanaṃtara   pavanarūpi puruṣo bhavati// samagrāṃ pṛthvī dṛṣṭā paśyati/     aṇimādyāṣṭasiddhir bhavati/ \B scribe switches so much between i and ī
%tadanaṃtaraṃ  pavanarūpi puruṣo bhavati// samagrāṃ pṛthvīṃ dṛṣṭā paśyati//   aṇimādyāṣṭasiddhir bhavati// \L
%\om                                                                 \N1
%tadanaṃtaraṃ  pavanayopī .. .. ... puruṣo bhavati samagrāṃ pṛthvīṃ dṛṣṭyā paśyati/ aṇimādyaṣṭasiddhir bhavati//  \D1
%\om                                                                 \N2
%tadanaṃtaraṃ  pavanayogī puruṣo bhavati samagrāṃ pṛthvīṃ dṛṣṭvā paśyati      aṇimādyāṣṭasiddhir bhavati  \U1 %%%291.jpg
%tadanaṃtaraṃ  pavanarūpī puruṣo bhavati// samagrāṃ pṛthvīṃ dṛṣṭvā paśyati//  aṇimāmahimāgarimāladhimā tathā //  \U2
%-----------------------------
%
%-----------------------------
%mahāpadmādyā nava                                                                                                    nidhyayaḥ  samīpa āgacchanti/ \E
%śrīpadmaś ca mahāpadmaḥ saṃkho makarakachapa  kuṃdonukuṃdanīlaścavijñeyāni dhayonavamamahāpadmādhānavanidhaya samīpe āgachaṃti \P     %%%7663.jpg 
%śrīpadmaś ca mahāpadmaṃ śaṃkho makarakacchapaḥ// kuṃdonukuṃdoścanīlaśrvavajrayonīcīdārmakā// śrīnamaḥ mahāpadmājñānavinidhyayaḥ// samipe āgacchatī//  nava nidhayaḥ samīpa āgacchanti/ \B
%śrīpadmaś ca mahāpadmaṃ śaṃkho makarakachapaḥ// kuṃdonukuṃdoś ca nīlaś ca va jñayoḥ nīcidātmakā// śrīnamaḥ mahāpadmājñānanidhyayaḥ// samipe āgacchaṃti// -------------\om-------- \L
%\om                                                                 \N1
%mahāpadmādyā .. ..                                                                                                    nidhyayaḥ  samīpe āgacchaṃti// \D1
%\om                                                                 \N2
%mahāpadmādyā nava                                                                                                    nidhapa  samīpe āgacchaṃti \U1
%prātikāmyamīśatvaṃ// viśītvaṃ// ityāṣṭasiddhayaḥ// padmaś ca mahāpadmaś ca śaṃkho makarakachapaḥ// mukuṃdo kuṃdaś ca nīlaś ca vajrayonavanidhi//etādṛśaṃ samīpe āgacchati// \U2 %%%421.jpg  
%-----------------------------
%
%-----------------------------
%ākāśamadhye daśasu dikṣu gamanāgamane bhavataḥ balaṃ bhavati/                                                                           parameśvaraṃ [P.56] samīpe paśyati/ karaṇe haraṇe sāmarthyaṃ bhavati// \E
%ākāśamadhye daśasu dikṣu gamanāgamanabalaṃ bhavati       yatra loke gamanechābhavati tatra loke gacchati ajñā sarvatra sphurati         parameśvaraṃ samīpe paśyati karaṇe haraṇe sāmarthyaṃ bhavati \P
%ākāśamadhye daśasu dikṣu gamanāgamanavallabhaṃ bhavati// yatra loke gamanechābhavati/ yatra loke gacchati/ ajñā sarvatra sphurati//     parameśvaraṃ samīpe paśyaṃtī/ karaṇe haraṇe sāmarthyaṃ bhavati// \B
%ākāśamadhye daśasu dikṣu gamanāgamanavallabhaṃ bhavati// yatra loke gamanechā bhavati// yatra loke gacchati// ajñā sarvatra sphurati    parameśvaraṃ samīpe paśyati karaṇe haraṇe sāmarthyaṃ bhavati// \L %%%0028.jpg
%\om                                                                 \N1
%ākāśamadhye daśa sudikṣumadhye gaṃmanāgamanabalaṃ bhavatī/ yatra lo.. gamanechā bhavati tatra loke gacchati/ ājñā sarvatra sphurati//   parameśvaraṃ samīpe paśyati/ karaṇaṃ haraṇe .. ..marthyaṃ bhavati// \D1
%\om                                                                 \N2
%ākāśadaśasu dikṣumadhye  gamanāgamanabalaṃ bhavati       yatra loke gamanechā bhavatī   yatra loke gacchati ājñā sarvatra sphurati      parameśvaraṃ samīpe paśyati karaṇe haraṇe ca sāmarthyaṃ bhavati \U1
%ākāśamadhye daśadikṣu    gamanāgamanabalam bhavati//     yatra loke gamanechā bhavati//  tatra loke gacchati// ajñā sarvatra sphurati// parameśvaraṃ samipe paśyaṃti// karaṇe taraṇe sāmarthyaṃ bhavati// \U2
%-----------------------------
%
%-----------------------------
%idaṃ guru bhakteḥ phalaṃ ātmamadhye manaso viśrāmakaraṇamicchatā puruṣeṇa sadguroḥ sevāṃ kṛtvā sāvadhānaṃ manaḥ karaṇīyam/ \E
%idaṃ gurubhaktaiḥ phalaṃ ātmamadhye manaso viśrāmakaraṇamichatā puruṣeṇa sadguroḥ sevāṃ kṛtvā sāvadhānaṃ manaḥ karaṇīyaṃ \P
%idaṃ gurubhakteḥ phalaṃ// ātmamadhye manaso viśrāmaṃ karaṃṇaṃmicchatāṃ// puruṣeṇa sadguroḥ sevāṃ kṛtvā sāvadhānaṃ manaḥ kṛtvā karaṇīyam// \B
%idaṃ gurubhakteḥ phalaṃ// ātmamadhye manaso viśrāmaṃ karaṇam icchatāṃ// puruṣeṇa sadguroḥ sevāṃ kṛtvā sāvadhānaṃ manaḥ kṛtvā karaṇīyaṃ... \L
%\om                                                                 \N1
%idaṃ gurubhakteḥ phalaṃ   ātmamadhye manaso viśrāma karaṇam icchatā puruṣeṇa sadguruḥ sevāṃ kṛ.. sāvadhānaṃ manaḥ karaṇīyaṃ/ \D1
%\om                                                                 \N2
%idaṃ gurubhakteḥ phalaṃ   ātmamadhye manaso viśrāma karaṇam icchatā puruṣeṇa sadguruḥ sevāṃ kṛtvā sāvadhānaṃ manaḥ karaṇīyaṃ \U1
%idaṃ gurubhakteḥ phalaṃ bhavati// ātmamadhye manaso viśrāme karaṇam ichatā puruṣeṇa sadguroḥ sevāṃ kṛtvā// māvadhānaṃ manaḥ karaṇīyaṃ// \U2
%-----------------------------
%
%-----------------------------
%abhyāsabalāt paramaprāptiḥ/  tena svaśiṣyamanasaḥ svāsthyaṃ karttavyam/ candrasūryyau yāvatpiṃḍe niścalau bhavataḥ//[p.57] samyaksvabhāvakiraṇodayacidvilāsagrastaṃ svaśāṃtisamatāṃ svayameva yāti/ \E
%abhyāsabalāt paramaprāptiḥ   tena svasyamanasaḥ samarasyaṃ karttavyam caṃdrasūryau yāvatpiṃḍo niścalo bhavati samyaksvabhāvakiraṇodayacidvilāsagrastaṃ svaśāṃtimanasā svayameva yāmi \P
%abhyāsabalāt paramaprāptiḥ// tena svasyamanasaḥ                                                                                                samaradvilāsa// grastaṃ svaśāṃti manasā svameva śāṃti// \B %stemma point?! omission?!
%abhyāsabalāt// paramaprāptiḥ// tena svasyamanasaḥ samarasaṃ  karttavyaṃ caṃdrasūryayāt piṃḍoniścalo bhavati// ślokasamyak svabhāvakiraṇodayacidvilāsa grastaṃ svaśāṃti manasā svayam eva śāṃti... \L
%\om                                                                 \N1
%abhyāsabalāt paramaprāptiḥ/ tena saha svasya manaḥ samarasyaṃ karttavyaṃ/ caṃdrasūryau yāvit piṃde niścalau      bhavatiḥ// ślokaḥ// samyak svabhāvakiraṇodayacidvilāsaṃ/ grastaṃ svaśāṃti mavatāṃ svayam eva yāti/ \D1
%\om                                                                 \N2
%abhyāsabalāt paramaprāptiḥ tena saha svaschamanaḥ karttavyaṃ caṃdrasūryau yāvaptiṃdau? niścalo      bhavati   ślokasamyagaḥ svabhāvakiraṇodayacidvilāsaṃ grastasamagraṃ saśāṃti mahatāṃ svayam eva yāti \U1
%abhyāsabalāt paramapadaprāptiḥ tena svasya manasaḥ samarasyaṃ karttavyaṃ// caṃdrasūryavat piṃḍo niścalo bhavati// ślokaḥ// samyak svabhāvakaraṇotdṛdicidvilāsagrastaṃ svaśāṃti bhavatāṃ svayam eva yāti// \U2 %%%421verso.jpg 
%-----------------------------
%
%----------------------------
%graste svaveganicaye padapiṃḍamaikyaṃ satyaṃ bhavet samarasaṃ guruvatsalāṃ ca//1// \E
%graste svaveganicaye padapiṃḍamaikyaṃ satyaṃ bhavet samarasaṃ guruvatsalānāṃ 1  \P %%%7664.jpg
%graste svaveganicaye padapiṃḍamaikyaṃ sataṃ bhavet samarasaṃ guruvatsalābhaṃ //1// \B
%graste svaveganicaye padapiṃḍamaikyaṃ satāṃ bhavet samarasaṃ guruvatsalābhaṃ //1// \L
%\om                                                                 \N1
%graste svavegaṃ nicaye padapiḍamaikyaṃ satyaṃ bhavet samarasaṃguruvatsalānāṃ//1//  \D1
%\om                                                                 \N2
%graste svaveganiścaye padapiṃḍamaikyaṃ satyaṃ bhavet samarasaṃguruvatchalānāṃ 1  \U1
%grāme  sveraṃgani caye yada piṃḍam aikyaṃ satyaṃ bhavet-samarasaṃ guruvatsalānāṃ// \U2
%-----------------------------
%
%----------------------------
%idānīm avadhūtapuruṣasya lakṣaṇaṃ kathyate/ \E
%idānīm avadhūtapuruṣasya lakṣaṇaṃ kathyate \P
%idānīm avadhūtapuruṣasya lakṣaṇam āha/ \B DSCN7171.jpg last line
%idānīm avadhūtapuruṣasya lakṣaṇam āha// \L
%\om                                                                 \N1
%idānīm mavadhūtapuruṣasya lakṣaṇam kathyate// \D1
%\om                                                                 \N2
%idānīm avadhūtapuruṣasya lakṣaṇam kathyate \U1
%idānīm avadhūtapuruṣasya lakṣaṇaṃ kathyate// \U2
%-----------------------------
%
%----------------------------
%yasya haste dhairyadaṇḍaḥ kharparaṃ śūnyamāsanam/ yogaiśvaryeṇa saṃpannaḥ sovadhūta udātdṛtaḥ//2// \E %%%SSP 6.10
%
%yasya haste dhairyadaṇḍaḥ kharparaṃ śūnyamāsanam yogaiśvaryeṇa saṃpanna sovadhūta udātdṛtaḥ 2  \E
%
%yasya haste dhairyadaṇḍaḥ kharparaṃ śunyabhāsanam// yogaiśvaryai saṃpannaḥ sovadhūtam udātdṛtaṃ// \B DSCN7172 Z.1
%
%yasya haste dhairyadaṇḍaḥ kharparaṃ śubhāsanam// yogaiśvarye saṃpannaḥ sovadhūtam udātdṛtaṃ// \L
%\om                                                                                            \N1
%yasya haste dhairyadaṇḍaḥ kharaparaṃ śūnyamānasaṃ/ yogaiśvaryeṇa saṃpannaḥ sovadhūta udāhṛtaḥ//2// \D1
%\om                                                                                            \N2
%
%yasya haste dhairyadaṇḍaḥ kharaparaṃ śūnyanāmakaṃ yogaiśvaryeṇa saṃpannaḥ sovadhūta udāthṛtaḥ 2 \U1 %%%292.jpg
%yasya hastai dhairyadaṇḍaḥ kharparaṃ śūnyam āsanaṃ// yogaiśvaryeṇa sapannaḥ sovadhūta udāthṛtaḥ//  \U2
%-----------------------------
%whose staff in the hand is [royal?]courage, whose begging bowl is the shine of emptiness. Furnished with the power of yoga, he is called an accomplished Avadhūta.  
%----------------------------
%bhedābhedau yasya bhikṣā bharaṇaṃ jāraṇaṃ tathā/ SSP 6.11 
%etādṛśopi puraṣaḥ sovadhūta udātdṛtaḥ//3//[p.58] \E
%
%bhedābhedau yasya bhikṣā bharaṇaṃ jāgaraṃ tathā 
%etādṛśo pi puraṣaḥ sovadhūta udādṛtaḥ 3 \P
%
%bhedābhedau yasya bhikṣā bharaṇaṃ jāraṇaṃ tathā// 
%tādṛśopi puraṣaḥ sovadhūtam udāhṛtaḥ//2// \B
%
%bhedābhedau yasya bhikṣā bharaṇaṃ jāraṇaṃ tathā// 
%tādṛśopi puraṣaḥ sovadhūtam udāhṛtaḥ//2// \L
%\om                                                                 \N1
%bhedābhedau yasya bhikṣā bhakṣaṇaṃ jāraṇaṃ tathā// 
%etādṛśopi puraṣaḥ sovadhūta udāhṛtaḥ 3 \D1
%\om                                                                 \N2
%
%bhedābhedau yasya bhikṣā bhakṣaṇaṃ jāraṇaṃ tathā 
%etādṛśopi puraṣaḥ sovadhūta udāhṛtaḥ 3 \U1
%
%bhedābhedo yasya bhīkṣā bharaṇaṃ jīraṇaṃ tathā// 
%etādṛśopi puruṣaḥ sovadhūta udāhṛtaḥ// \U2
%-----------------------------
%Whose alms are "difference and non-difference", whose dress is armor (jāgara!!!), such a person is called an Avadhūta. 
%----------------------------
%ātmā hyakāro vijñeyo vakāro bhavavāsanā/
%dhūtaṃ saṃtāpanaṃ proktaṃ sovadhūto nigadyate// 4// \E
%
%ātmā hyakāro vijñeyo vakāro bhavavāsanā
%dhūtas tat kaṃpanaṃ proktaṃ sovadhūta nigadyate 3 \P
%
%ātmāt dyukāro vijño yau vikāro bhavavāsanā//
%dhūtas tatkaṃpanaṃ proktaṃ sovadhūta nigadyate// 3// \B
%
%ātmār dyukāro vijñeyo vikāro bhavavāsanā// %%%0028.jpg last line
%dhūtas tatkaṃpanaṃ proktaṃ sovadhūta nigadyate// 3// \L
%\om                                                                 \N1
%ātmā hyakāro vijñeyo vakāro bhavavāsanā//
%dhūtasa tatkaṃpanaṃ proktaṃ sovadhūto nigadyate/ \D1
%\om                                                                 \N2
%
%ātmai hyakāro vijñeyo vakāro bhavavāsanā
%dhūtas tatkaṃpanaṃ proktaṃ sovadhūto nirucyate 4 \U1
%
%āhyakāro vijñeyo vakāro bhavavāsanā
%dhūtas tatkaṃpanaṃ proktaṃ sovadhūto nigadyate// \U2
%-----------------------------
%The self is to be known as hyakāra and bhavavāsanā as vakāra. Sittlichkeit is said to be his weapon, he is called an avadhūta. 
%-----------------------------
%akārārtho jīvabhūto vakārārtho tha vāsanā/
%etaddūyaṃ japaṃ kuryātsovadhūta udāhṛtaḥ//5// \E
%
%ākārārtho jīvabhūto vikārārtho thavāsanā 
%etad dvayaṃ yaṃ jayati yaḥ sovadhūta udādhṛtaḥ 4 \P
%
%ākārārtho jīvabhūto vikārādir sthorya vāsanā// 
%etadvayaṃ yaḥ jānati sovadhūta udādhṛttā//4// \B
%
%ākārārtho jīvabhūto vikārādir sthortha vāsanā// 
%etad vayaṃ yaḥ jānati sovadhūta udādhṛtaḥ//4// \L
%\om                                                                 \N1
%akārārtho jīvabhūto vakārārtho tha vāsanā//
%etadvayaṃ jīyate yaḥ sovadhūta udāhṛtaḥ//4// \D1
%\om                                                                 \N2
%
%akārārtho jīvabhūto vakārārtho yavāsanā 
%etad vayaṃ jīryate yaḥ sovadhūta udārataḥ 5 \U1
%
%akārārtho jīvabhūto vakārārtho yavāsanā// 
%etad vayaṃ jayati yaḥ sovadhūta udāhṛtaḥ// \U2
%-----------------------------
%
%-----------------------------
%yaḥ puruṣo dvitīyaṃ na paśyati kevalaṃ svasvarūpaṃ paśyati sovadhūtaḥ/ \E
%yaḥ puruṣo dvitīya  na paśyati kevalaṃ svasvarūpaṃ paśyatī sovadhūtaḥ/ \P
%yaḥ puruṣo dvitiyaṃ na paśyaṃtī kevalaṃ svasvarūpaṃ paśyati sovadhūtaḥ \B
%yaḥ puruṣo dvitiyaṃ na paśyati kevalaṃ svasvarūpaṃ paśyati sovadhūtaḥ// \L
%\om                                                                 \N1
%yaḥ puruṣo dvitiyaṃ na paśyati kevalaṃ svasvarūpaṃ tiṣṭhati sovadhūtaḥ// \D1
%\om                                                                 \N2
%yaḥ puruṣo dvitiyaṃ na paśyati kevalaṃ svasvarūpaṃ tiṣṭhati sovadhūtaḥ \U1
%yaḥ puruṣo dvitīyaṃ na paśyati// kevalaṃ svasvarūpaṃ paśyati sovadhūtaḥ// \U2
%-----------------------------
%
%-----------------------------
%athavo yasya manaścaṃcalabhāvaṃ na dadhāti sovadhūtaḥ kathyate/ \E
%athavā yasya manaścaṃcalabhāvaṃ na dadhāti sovadhūtaḥ kathyate  \P
%athavā yasya manaś caṃcalaṃ bhāva na dadhāti/ sovadhūtaḥ/ \B
%athavā yasya manaś caṃcalaṃ bhāvaṃ na dadhāti sovadhūtaḥ// \L
%\om                                                                 \N1
%athacā yasya manaḥ caṃcala bhāvaṃ na dadhāti/ sovadhūtaḥ kathyate/ \D1
%\om                                                                 \N2
%athacā yasya manaḥ caṃcala bhāve na dadhāti sovadhūtaḥ kathyate \U1
%athavā yasya manaścaṃcalī bhāvaṃ na dadhāti sovadhūtaḥ kathyate//  \U2
%-----------------------------
%
%-----------------------------
%yanna dṛśyate tadavyaktam ity ucyate/ \E
%yanna dṛśyate tadavyaktam ity ucyate \P
%atha vā kasyase panna iśyate davyaktam ity ucyate/ \B
%atha vā kasyase panna dṛśyate davyaktam ity ucyate// \L
%\om                                                                 \N1
%yanma dṛśyate tad avyaktam ity ucyate/ \D1
%\om                                                                 \N2
%yanna dṛśyate tad avyaktam ity ucyate \U1
%              tad avyaktam ity ucyate \U2
%-----------------------------
%
%-----------------------------
%tad avyaktaṃ pratyakṣeṇa paśyati/ \E
%tad avyaktaṃ pratyakṣeṇa paśyati  \P
%tad avyaktaṃ pratyakṣeṇa yasyati / \B
%tad avyaktaṃ pratyakṣeṇa yasyati ... \L
%\om                                                                 \N1
%tad avyaktapratyakṣeṇa paśyati \D1
%\om                                                                 \N2
%tad avyaktapratyakṣeṇa paśyatī \U1
%tad avyaktaṃ pratyakṣeṇa paśyati//  \U2
%-----------------------------
%
%-----------------------------
%yatkiṃcidṛśyate tatsarvaṃ grastāti muktam iti jñānaṃ paśyati/ \E
%yatkiṃcid dṛśyate tatatsarvaṃ grasati muktam iti jñāyate    \P
%yatkiṃcidṛśyate tatsarvaṃ gasati muktamiti jñāyate          \B
%yatkiṃcid dṛśyate tatsarvagasati muktam iti jñāyate...      \L
%\om                                                                 \N1
%yatkiṃcit paśyati tatsarvaṃ grasatī muktam iti jñāyate      \D1
%\om                                                                 \N2
%yatkiṃcit paśyati tatsarvaṃ grasatī muktam iti jñāyate      \U1
%yatkiṃcit dṛśyate tatsarvaṃ grasaṃti muktim iti jñāyate//    \U2
%-----------------------------
%
%-----------------------------
%sovadhūtaḥ kathyate/ \E [p.59]
%sovadhūtaḥ kathyate  \P  %%%7665.jpg
%sāvadhūtaḥ kathyate  \B
%sovadhūtaḥ kathyate  \L
%\om                                                                 \N1
%sovadhūtaḥ kathyate/  \D1
%\om                                                                 \N2
%sovadhūtaḥ kathyate  \U1
%sovadhūtaḥ kathyaṃte//  \U2
%-----------------------------
%
%-----------------------------
%avadhūtatanuḥ somo nirākārapade sthitaḥ/  [p.59]    %%%%%%%SSP 3.30 
%sarveṣāṃ darśanānāṃ ca svasvarūpaṃ prakāśyate// 1// \E
%
%avadhūtatanu somo nirākārapade sthiraḥ
%sarveṣāṃ darśanānāṃ ca svasvarūpaṃ prakāśate 1  \P
%
%avadhūtatanuḥ somo nirākārapade sthita/
%sarveṣāṃ darśanānāṃ ca svasvarūpaṃ prakāśate/ \B
%
%avadhūta tanu somā nirākārapade sthitaḥ/
%sarveṣāṃ darśanānāṃ ca svasvarūpaṃ prakāśate/ \L
%\om                                                                 \N1
%avadhūta tanu somo nirākārapade sthitaḥ//
%sarveṣāṃ darśanānāṃ ca svasvarūpaṃ prakāśyate//1// \D1
%\om                                                                 \N2
%
%āvadhūta tanuḥ somo nirākārapare sthita
%sarveṣāṃ darśanānāṃ ca svasvarūpaṃ prakāśyate \U1
%
%avadhūtarutu? somo nirākārapade sthitaḥ//
%sarveṣāṃ darpaṇānāṃ ca svasvarūpaṃ prakāśyate// \U2 %%%422.jpg
%-----------------------------
%
%-----------------------------
%satyam ekam ajaṃ nityam anaṃtam akṣayaṃ dhruvam/
%jñātvā hyevaṃ vadeddhīmān satyavādī sa kathyate// 2// \E
%
%satyam ekam ajaṃ nityamanaṃ tam akṣayaṃ dhruvaṃ       %%SSP 6.60
%jñātvā hyevaṃ vadeddhīmān satyavādī sa kathyate 2 \P
%
%satyam ekam ajaṃ nityam anaṃtam akṣayaṃ dhruvam/
%jñātvā hyevaṃ vadeddhīmān satyavādī sa kathyate/ \B
%
%satyam ekam ajaṃ nityam anaṃtam akṣayaṃ dhruvaṃ/
%jñātvāt yevaṃ vadeddhīmān sa kathyate/ \L
%
%\om \N1
%
%satyam ekām ja nityaṃ manaṃ tam akṣayaṃ dhruvaṃ/
%jñātvāt ????? vadeddhīmān satyavādī sa kathyate//2// \D1
%
%\om \N2
%
%satyam ekām ajaṃ nityaṃ manaṃ tam a???kṣayaṃ dhruvaṃ
%jñātvā hyastaṃ? vadeddhīmān satyavādī sa kathyate 2 \U1
%
%satyam ekam ajaṃ nityaṃ manaṃ tam akṣayaṃ dhruvaṃ//
%jñātvā hyevaṃ vadet dhīmān satyavādī kathyate \U2
%-----------------------------
%
%-----------------------------
%yatkiṃcinna paśyati, sa eko hyevaṃ manaso vijānāti nāśā na tādṛśaṃ padārthaṃ jñātvā kāle ceṣṭā bhavati/ sa satyavādī kathyate//   \E [p.60]
%yatkiṃcid yena paśyati sa ekaḥ tasya manaso na jānāti na nāśo na tādṛśaṃ padārtha jñātvā kāle ceṣṭā bhavati sa satyavādī kathyate \P
%yatkiṃ kena paśyaṃti sa ekaḥ/ tasya manaso jānātir na nāśo na tādṛśaṃ padārthaṃ jñā kāle ceṣṭā bhavati/ sa satyavādi kathyate//   \B
%yatkiṃ kena paśyaṃti sa ekaḥ/ tasya manaso jānāti na nāśo na tādṛśaṃ padārthaṃ jñākāle ceṣṭā bhavati/ sa satyavādi kathyate//     \L
%\om                                                                 \N1
%yatkiṃcidaikye na paśyati sa sa ekaḥ// tasya mano jāti tānāśona tādṛśāṃ padārthaṃ jñātvā kāla ceṣṭā bhavati/ sa satyavādī kathyate/ \D1
%\om                                                                 \N2
%yatkiṃcidaike na paśyatī sa ekaḥ tasya mano jnānaṃti tādṛśot tā???dṛśāṃ padārthaṃ jñātvā kāla ceṣṭā bhavati sa satyavādī kathyate \U1 %%%293.jpg
%\om                                                                                                                               \U2
%-----------------------------
%
%-----------------------------
%vāsvare bhāsvare śaktiḥ saṃkoco bhāsvarepi ca/ tayoḥ saṃyogakarttā yaḥ sa bhavetsatyayogabhāk//3// \E  %%SSP 6.64
%vāsare bhāsvare śaktiḥ saṃkoco bhāsvare pi ca  tayoḥ saṃyogakarttā yaḥ sa bhavet satyayogabhāk 3  \P
%vāsvre bhāsvare pi ca //                             sayogaḥ kartavyaḥ    bhavat satyayogabhāk// \B
%vāsare bhāskare pi ca //                             saṃyogaḥ karttāyaḥ    sa bhavet satyayogabhāk// \L
%\om                                                                 \N1
%vasare bhāsvare śaktiḥ/ saṃkoco bhāsvarepi ca/ tayoḥ saṃyogakarttā yaḥ// sa bhavet satyayogabhāk//3// \D1
%\om                                                                 \N2
%vasare bhāskare śaktiḥ saṃkoco bhāskarepi ca         saṃyogakartā yaḥ saṃvit svabhāvāt satyayogabhāk  \U1
%vāsare bhāsvare śaktiḥ// saṃkoco bhāsvare pi ca//  tayoḥ saṃyogakarttāyaḥ sa bhavet satyayogabhāk//  \U2
%-----------------------------
%
%-----------------------------
%viśvānītatayā viśvam ekam eva virājate/
%saṃyogo na sadā yasya siddhayogī sa gadyate//4// \E   %%SSP 6.65 
%
%viśvānītatayā viśvam ekam eva virājate/
%saṃyogo na sadā yasya siddhayogī sa kathyate 4 \P
%
%visvātitātayā viśvam ekam eva virājate/
%saṃyogo na sadā yasya siddhayogī sa gadyate/ \B
%
%visvātitātayā viśvam ekam eva virājate/
%saṃyogo na sadā yasya siddhayogī sa gadyate/ \L 0030.jpg
%\om                                                                 \N1
%viśvātī tat tayā viśvam ekam eva virājate/
%saṃyoge na sadā yasya siddhayogī sa gadyate//4//  \D1
%\om                                                                 \N2
%
%viśvāso viśvāti tatayā visvaṃ ekam eva virājate
%saṃyogo na sadā yasya siddhayogī sa kathyate  \U1
%
%viśvātitatayāviśvam ekam eva virājate//
%saṃyogo na sadā yasya siddhayogī sa gadyate//  \U2
%-----------------------------
%
%-----------------------------
%sarvāsāṃ nijavṛttīnāṃ vismṛtīrbhajate ttu yaḥ/
%sa bhavet siddhasiddhānto siddhayogī sa gadyate//5// \E [p.61] %This quote stems from the Siddhasiddhāntapaddhati 6.66
%
%sarvāsāṃ nijavṛtīnāṃ vismṛtī bhajate tu yaḥ
%sa bhavet siddhasiddhāṃte siddhayogī sa gadyate 5 \P
%
%sarvāsāṃ bījavṛtīnāṃ vismṛtī bhajate tu yaḥ
%sa bhavet siddhasiddhānte siddhayogī sa gadyate/ \B
%
%sarvāsāṃ bījavṛtīnāṃ vismṛtīṃ bhajate tu yaḥ//
%sa bhavet siddhasiddhānte siddhayogī sa gadyate// \L
%\om                                                                 \N1
%sarvāsāṃ \om                                                        \D1
%\om                                                                 \N2
%
%sarvāsāṃ nijavṛtīnāṃ vismṛtiṃ bhajate tu yaḥ
%sa bhavet siddhasiddhāṃte siddhayogī sa gadyate \U1
%
%sarvāsāṃ nijavṛttīnāṃ vismṛtiṃ bhajate tu yaḥ//
%sa bhavet siddhasiddhāṃte siddhayogī sa gadyate// \U2
%-----------------------------
%
%-----------------------------
%udāsīnaḥ sadā śānto brahmānandamayo pi ca/
%yo bhavet siddhayogena siddhayogī sa kathyate//6// \E
%
%udāsīnaḥ sadā śānto brahmānandamayo pi ca/
%yo bhavet siddhayogena siddhayogī sa kathyate 6 \P  %%%7666.jpg
%
%\om in \L ????check! 
%
%udāsīnaḥ sadā śānto mahānaṃdamayo pi ca/
%yo bhavet siddhayogena siddhayogī sa kathyate// \B DSCN7173.JPG Z.1
%\om                                                                 \N1
%\om                                                                 \D1
%\om                                                                 \N2
%
%udāsīna sadā śānto mahānaṃdamayo pi ca
%yo bhavet siddhayogena siddhayogī sa kathyate \U1
%
%udāsīnaḥ sadā śāṃto mahānaṃdamayā pi ca//
%yo bhavet siddhayogena siddhayogī sa kathyate// \U2
%-----------------------------
%
%-----------------------------
%adhunā kamalānāṃ tu śrṛṇu saṃketamadbhutam/ anekākārabhedotthaṃ kaṃ svarūpātmakaṃ malam/ kamalaṃ tena vikhyātaṃ trividhaṃ tatra dehagam// 7// \E
%adhunā kamalānāṃ tu nuṣṛ e?! saṃketam adbhutaṃ anekākārabhedocchaṃ kaṃ svarūpātmakaṃ malaṃ 7 kamalaṃ tena vikhyātaṃ vividhaṃ tatra dehagaṃ    \P
%adhunā kamalānāṃ tu śṛṇu saṃketam adbhutaṃ/ anekakārabhedochaṃ kiṃ svarūpātmakaṃ malaṃ//7// kamalaṃ tena vikhyātaṃ trividhaṃ tatra dehagam//  \B
%adhunā kamalānāṃ tu śṛṇu saṃketam adbhutaṃ/ anekakārabhedātthaṃ kiṃ svarūpātmakaṃ malaṃ//7// kamalaṃ tena vikhyātaṃ trividhaṃ tatra dehagaṃ// \L
%\om                                                                 \N1
%\om                                                                 \D1
%\om                                                                 \N2
%adhunā kamalānāṃ tu śṛṇu saṃketam adbhutaṃ  anekakārabhedotthaṃ svasvarūpātmakaṃ malaṃ kamalaṃ tena vikhyātaṃ trividhaṃ tena dehagaṃ          \U1
%adhunā kamalānāṃ tu śṛṇu saṃketam adbhutaṃ// anekākārabhedotthaṃkaḥ// svarūpātmakaṃ paraṃ// kamalaṃ tena vikhyātaṃ trividhaṃ tatra dahagaṃ// \U2
%-----------------------------
%
%-----------------------------
%ādhārakamalamasya kamalamiti kaṃ kasmāt/kamātmā tasmātkamalamiti saṃjñā asyādhāraḥ kamaladalasya catuṣṭayaṃ bhavati/ \E [p.62]
%athādhaḥ kamalaṃ kathyate  ādhārakamalaṃ asya kamalam iti saṃjñā kasmāt kamātmasvarūpaṃ sa ātmanaṃ anekarūpaṃ paśyati tadṛśanaṃ mala ityucyate tasmāt kamalam iti saṃjñā asyādhāraḥ kamalasya  \P
%athādhakamalaṃ kathyate/ārakamalaṃ asya kamalam iti saṃjñā kasmātmasvarūpaṃ sa ātmanaṃ anarūpaṃ paśyati// tadṛśa na ityucyate// tasmāt kamalam iti saṃjñā/ asyādhāraḥ// kamalasya dalaṃ catuṣṭayaṃ bhavatī/ \B
%athādhakamalaṃ kathyate// ādhārakamalaṃ asya kamalam iti saṃjñā kasmāt kāmātmasvarūpaṃ sa ātmanaṃ anarūpaṃ paśyati// tadṛśa na ityucyate// tasmāt kamalam iti saṃjñāṃ asyādhāraḥ// kamalasya dalaṃ catuṣṭayaṃ bhavatī/ \L
%\om                                                                 \N1
%\om                                                                 \D1
%\om                                                                 \N2
%athādhaḥ kamalaṃ kathyate ādhārakamalaṃ asya kamalam iti saṃjñā kasmāt kaḥ ātmā sa ātmanaṃ anekarūpaṃ svarūpaṃ paśyate tadṛśanaṃ kamala iti kathyate tasmāt kamala iti saṃjñā asyādhārakamalasya dalacatuṣṭayaṃ bhavati \U1
%athādhaḥ kamalaṃ kathyate// ādhārakamalaṃ// asya kamalam iti saṃjñā kasmāt// ekam ātmasvarūpaṃ// sa ātmanaṃ anekarūpaṃ paśyati// tad darśanaṃ malaṃ// ity ucyate// tasmāt kamalam iti saṃjñā// asyādhārakamalasya dala catuṣṭayaṃ bhavati// \U2
%-----------------------------
%prathamaṃ sattvaguṇasya dvitīyaṃ rājayogaya tṛtīyaṃ tamoguṇaḥ caturtho dale manas tiṣṭhati/ \E
%                        dvitīyaṃ rājayogasya tṛtīyaṃ tamoguṇasya caturthe dalamenas tiṣṭhati \P
%prathamaṃ sattvaguṇasya/ dvitīyaṃ rājoguṇaḥ/ tṛtīyaṃ tamoguṇ/ \B
%prathamaṃ satyaguṇasya// dvitīyaṃ rājoguṇasya tṛtīyaṃ tamoguṇaḥ caturthe dale manas tiṣṭhati// \L
%\om                                                                 \N1
%\om                                                                 \D1
%\om                                                                 \N2
%prathamadalaṃ satvaguṇasya dvitīyaṃ rajoguṇatṛtīyaṃ tamoguṇasya caturthe dalaṃ manaḥ stiṣṭhati \U1 %%%294.jpg
%prathamaṃ satvaguṇasya// dvitīyaṃ rājoguṇasya // tṛtīyaṃ tamoguṇasya// caturthe dale manas tiṣṭhati// \U2
%-----------------------------
%
%-----------------------------
%etaddalacatuṣṭayaṃ ca saṃgādātmā sādhu karoti/ \E
%etad dala catuṣṭaya saṃgādātmā sāvadhvasādhu karoti    \P
% \om                                           \B
% etaddalacatuṣṭayaṃ saṃjñā gidātmā sādhu karoti// \L
% \om                                                                 \N1
%\om                                                                 \D1
%\om                                                                 \N2
%etac catuṣṭayasaṃgādātmasādhvasādhū karoti     \U1
%etaddalacatuṣṭaya saṃyogād ātmāsādhvasādhu karoti// \U2
%-----------------------------
%
%-----------------------------
%tasmin kamale niścalī kṛte sati puruṣasya samīpe maraṇaṃ na gacchati/  \E
%tasmin kamale niścalī kṛte sati puruṣasya samipe maraṇaṃ na gacchati   \P %%%7667.jpg
%tasmin kamale niccalī kṛte sati puruṣasya samipe maraṇaṃ na gacchati/  \B
%tasmin kamale niccalī kṛte sati puruṣasya samīpe maraṇaṃ na gacchati/  \L %%%0031.jpg
%\om                                                                 \N1
%\om                                                                 \D1
%\om                                                                 \N2
%\om \U1
%tasmin kamale niścalī kṛte sati puruṣasya samīpe maraṇaṃ nāgacchati/  \U2
%-----------------------------
%
%----------------------------- 
%idānīṃ hṛyakamalabhedāḥ kathyaṃte/ \E
%idānīṃ hṛdayakamalasya bhedaḥ kathyate/ \P
%idānīṃ hṛdayakamalasya bhedaḥ kathyate/ \B
%idānīṃ hṛdayakamalasya bhedaḥ kathyate// \L
%\om                                                                 \N1
%\om                                                                 \D1
%\om                                                                 \N2
%idānīṃ hṛdayakamalasya dvitīyo bhedaḥ kathyate \U1
%idānīṃ hṛdayakamalasya bhedāḥ kathyate// \U2
%-----------------------------
%
%-----------------------------
%asya dvādaśadalāni siddhapuruṣāḥ kathayaṃti/ \E
%asya dvādaśadalāni siddhapuruṣāḥ kathayaṃti/ \P
%asya dvādaśadalāni siddhapuruṣāḥ kathyaṃte/ \B
%asya dvādaśadalāni siddhapuruṣāḥ kathyaṃte// \L
%\om                                                                 \N1
%\om                                                                 \D1
%\om                                                                 \N2
%asya dvādaśadalāni siddhapuruṣāḥ kathyaṃte \U1
%asya dvādaśadalāni siddhāḥ puruṣāḥ kathayaṃtī// \U2
%-----------------------------
%
%-----------------------------
%tathā dviṣaśaktistṛtīyalokāṃtaḥ samyak samudrā khecarī cidānandādvayaścandracaṃdrikā vetināmānvitaḥ// \E %[P.63]
% \om \P
% \om \B
% \om \L
% \om \N1
% \om \N2
% \om \U1
% \om \U2
%-----------------------------
%
%-----------------------------
%tathā dviṣaṇādalanāmadhya ekaṃ kaṭhiṇaṃ bhavatī// \E
%tathā dviṣaṇāṃ dalānāmaṣṭadalanāṃ madhye ekaṃ kaṭhiṇaṃ bhavati \P
%\om                                               \B
%tathā dviṣaṇādalanāmadhya ekaṃ kaṭhiṇaṃ bhavati// \L
%\om                                                                 \N1
%\om                                                                 \D1
%\om                                                                 \N2
%tathāpi varṇadalānām aṣṭadalā eva kaṭitaṃ bhavati \U1
%tathā dviṣaṇāṃ dalānām aṣṭadalānāṃ madhye ekaṃ kaṭhiṇaṃ bhavati// \U2
%-----------------------------
%
%----------------------------
%tadaṣṭadalaṃ kamalaṃ hṛdaye tiṣṭhati te ubha hṛdaye tiṣṭhataḥ/ \E
%tadaṣṭadalaṃ kamalaṃ hṛdaye tiṣṭhati te ubhe hṛdaye tiṣṭhataḥ \P
%\om \B
%tadaṣṭadalaṃ kamalaṃ hṛdaye tiṣṭhati te ubhe hṛdaye?! tiṣṭhataḥ/ \L 0031.jpg Z.3
%\om                                                                 \N1
%\om                                                                 \D1
%\om                                                                 \N2
%tata aṣṭadalaṃ kamalaṃ hṛdaye tiṣṭhati te ubhe pi kathyate \U1
%tad aṣṭadalaṃ kamalaṃ hṛdaye tiṣṭhati// te ubha hṛdaye tiṣṭhataḥ// \U2
%-----------------------------
%
%----------------------------
%prathamadale/ śabdas tiṣṭhati/ dvitīyadale sparśa tiṣṭhati/ tritiyadale rūpaṃ tiṣṭhati/  \E
%prathamadale  śabdas tiṣṭhati  dvitīye dale sparśas tiṣṭhati tritīyadale rūpaṃ tiṣṭhati  \P
% \om \B
% prathamadale// śabdas tiṣṭhati// dvitīyadale sparśas tiṣṭhati// tritiyadale rūpaṃ tiṣṭhati// \L
% \om                                                                 \N1
%\om                                                                 \D1
%\om                                                                 \N2
%prathame dale śabdaḥ stiṣṭhati dvitīye dale sparśaḥ tiṣṭhati tritīyadale rūpaḥ tiṣṭhati \U1
%prathamadalaśabdaṃ tiṣṭhati// dvitīyadale sparśas tiṣṭhati// tritīyadale rūpaṃ tiṣṭhati// \U2
%-----------------------------
%
%----------------------------
%caturthadale rasa tiṣṭhati/paṃcamadale gaṃdha tiṣṭhati/ saṣṭhadale ciṃta tiṣṭhati/ saptamadale budhis tiṣṭhati/ aṣṭamadale ahaṃkāras tiṣṭhati/ etadaṣṭadale madhye/ \E
%caturthe dale rasas tiṣṭhati paṃcamadale gaṃdha tiṣṭhati saṣṭhadale cittaṃ tiṣṭhati saptamadale buddhis tiṣṭhati aṣṭame dale haṃkāras tiṣṭhati etadaṣṭadale madhye \P
% \om \B
% caturthadale rasas tiṣṭhati// paṃcamadale gaṃdhas tiṣṭhati// saṣṭhadale ciṃtta tiṣṭhati// saptamadale budhis tiṣṭhati// aṣṭamadale ahaṃkāras tiṣṭhati// etad aṣṭadalamadhye// \L
% \om                                                                 \N1
%\om                                                                 \D1
%\om                                                                 \N2
%caturthadale rasaḥ tiṣṭhati   paṃcame dale gaṃdhaḥ stiṣṭhati saṣṭhe dale cittaḥ stiṣṭhati saptame dale budhiḥ tiṣṭhati aṣṭame dale ahaṃkāraḥ tiṣṭhati etattatadalamadhye \U1
%caturthadalarasas tiṣṭhati// paṃcame dale gaṃdhas tiṣṭhati// saṣṭhe dale cittaṃ tiṣṭhati// saptame dale buddhis tiṣṭhati// aṣṭame dale ahaṃkāraḥ tiṣṭhati// etad aṣṭadalamadhye \U2
%-----------------------------
%
%-----------------------------
%samagrapṛthivyākāro cartate/ atha ca// tatkamalamadhye mukhaṃ tiṣṭhati/ asya kamalasya dhyānākāśo bhavati/ \E
%samagrapṛthivyākāro vartate atha ca tatkamalamadhye mukhaṃ tiṣṭhati asya kamalasya dhyānākāśo bhavati \P
%\om \B?????? check once more
%samagrapṛthivyākāro cartate// atha ca tatkamalamadhye mukhaṃ tiṣṭhati// asya kamalasya dhyānākāśo bhavati// \L
%\om                                                                 \N1
%\om                                                                 \D1
%\om                                                                 \N2
%samagryā pṛthvā kāro varttate atha ca tatkamalaṃ adhomukhaṃ tiṣṭhati asya kamalasya dhyānādātmaprakāśo bhavati \U1
%samagrapṛthivyākāro vartate// atha ca tat kamalamadhye mukhaṃ tiṣṭhati// asya kamalasya dhyānādāt prakāśo bhavati// \U2
%-----------------------------
%
%-----------------------------
%prakāśādanaṃtara/ [DSCN7174.jpg Z.1] kamalaṃ mūrdhvaṃ mukhaṃ bhavati/ \E
%prakāśādanaṃtaraṃ kamalam ūrddhvamukhaṃ bhavati \P
%\om \B????? check once more
%prakāśāvan aṃtaraṃ kamalam ūrdhvamukhaṃ bhavati// \L
%\om                                                                 \N1
%\om                                                                 \D1
%\om                                                                 \N2
%prakāśādanaṃtaraṃ kamalamūrdhvamukhaṃ bhavati \U1
%prakāśād anaṃtaraṃ kamalam ūrdhvamukhaṃ bhavati// \U2
%-----------------------------
%
%-----------------------------
%tathā sūryaprakāśānaṃtaraṃ/ tadā kamalamadhye kamalaṃ vikasati// \E
%tathā sūryaprakāśānaṃtaraṃ  tadā kamalamadhye kamalaṃ visati \P %%%7668.jpg 
% \om \B????? check once more
% tathā sūryaprakāśānaṃtaraṃ tadā kamalamadhye kamalaṃ vikasati// \L
% \om                                                                 \N1
%\om                                                                 \D1
%\om                                                                 \N2
%yathā sūryaprakāśānaṃtaraṃ tadā  kamalaṃ              vikasati// \U1 %%%295.jpg
%tathā sūryaprakāśād anaṃtaraṃ tadā malamadhye kamalaṃ vikasati// \U2
%-----------------------------
%
%-----------------------------
%tam api ātmaprakāśanaṃ taraṃ mūrdhvaṃ mukhaṃ vikasati/ tanmadhye paramānaṃdasūpābhūmir bhavati// \E
%tathe damapi ātmaprakāśānaṃ taramūrdhvaṃ mukhaṃ vikasati tanmadhye paramānaṃdarūpābhūmir bhavati \P
%\om \B???? check
%tam api ātmaprakāśanaṃ taram ūrdhvamukhaṃ vikasati// tanmadhye paramānaṃdarūpo bhūmir bhavati// \L
%\om                                                                 \N1
%\om                                                                 \D1
%\om                                                                 \N2
%tathā idam apy ātmaprakāśānataram ūrdhvamukhaṃ vikasati tanmadhye paramānaṃdarūpā bhūmir bhavatī \U1
%tathedam api ātmaprakāśānaṃtaramūrdhvamukhaṃ vikasati// tanamadhye paramānaṃdasūpābhūmir bhavati// \U2 %%%424.jpg
%-----------------------------
%
%-----------------------------
%tasyāhaṃ sohaṃ sa iti saṃjñā// tasya madhye svātmano dhyāddī?nedine āyur vardhayati/ \E
%tasyāhaṃ sohaṃ sa iti saṃjñā   tasyā madhye svātmano dhyānād dine dine āyur varddhate \P
%\om \B???? check
%tasyāhaṃ sohaṃ sa iti saṃjñā// tasya madhye svātmano dhyānād dine dine āyur varddhayati... \L
%\om                                                                 \N1
%\om                                                                 \D1
%\om                                                                 \N2
%tasyāhaṃ soha ?? iti saṃjñā    tasyā madhye svātmanaḥ dhyānād dine dine āyur varddhati \U1
%tasyāhaṃ sohaṃ sa iti saṃjñā// tasyā madhye svātmano dhyād dine dine āyur varddhati// \U2
%-----------------------------
%
%-----------------------------
%what is \E .... what is \B???
%rogādūre bhavaṃti śakti stri taya lokāṃ tasamyak mudrā ca khecarī  \P
%rogādūro bhavati/ śakti sri? tayo lokāṃ tasamyak mudrā ca khecari/ \B %%%%%%%%%%%%%%%%%%found later on!! in \E%%%%%%%%%%%%%%%%%%%%%%%%%%%%%
%rogādūrā bhavaṃti// śakti stri?? tayo lokāṃ tasamyak mudrā ca khecarī/  \L
%\om                                                                     \N1
%\om                                                                 \D1
%\om                                                                     \N2
%rogādūre bhavaṃti śakti strī valī kṛtaṃ samyak mudrā bhavati khecarī    \U1
%rogādūre bhavaṃti// śakti stri taya lokāṃtaḥ samyak mudrā ca khecarī// \U2
%-----------------------------
%
%-----------------------------
%                                 paramātmanā saha raśmipuṃjaprakāśaḥ prakāśānandayor aikyaṃ prakarttavyaṃ nirantaraṃ svayaṃ manasi mahājyotir ābhāti paramaṃ padam// \E
%cidānaṃdādayaś caḍriś ca drikācetanānvitāḥ paramātmāsahāsūryaraśmipuṃjaprakāśakaḥ prakāśānaṃdayor aikyaṃ prakartavyaṃ niraṃtaraṃ svayam agnir mahājyotir ābhāti paramaṃ padaṃ \P
%cidānaṃdādayoś caḍrikācetanvitāḥ paramātmāmahāsūryaraśmiyuṃjaprakāśakaḥ prakāśānaṃdayor aikyaṃ prakartavyaṃ/ niraṃtaraṃ svayam agnir mahājyotir ābhāti paramapadam \B
%cidānaṃdādayoś caṃḍrīkācetanvitāḥ paramātmāmahāsūryaraśmipuṃjaprakāśakaḥ prakāśānaṃdayor aikyaṃ prakartavyaṃ// niraṃtaraṃ svayam agnir mahājyotir ābhāti paramaṃ padaṃ// \L
%\om                                                                 \N1
%\om                                                                 \D1
%\om                                                                 \N2
%cidānaṃdodayaṃ ścaṃdraḥ ścetanā  śceaṃdrakānvitā paramātmāmahāsūryaraśmipuṃjaḥ prakāśakaḥ prakāśānaṃdayor aikyaṃ prakartavyaṃ niraṃtaraṃ svayam agnir mahājyotiśa bhāti paramaṃ padaṃ \U1
%cidānaṃdādayaḥ caṃdrāś ca drikācetanānvitaḥ// paramātmāmahāsūryaraśmipuṃjaprakāśakaḥ// prakāśānaṃdayor aikyaṃ prakarttavyaṃ niraṃtaraṃ// svayam agnir mahājyotir ābhāti paraṃmapadaṃ// \U2
%-----------------------------
%
%-----------------------------
%sadoditamanaścandraḥ sūryodayamavekṣate/ tena grasto manaścandraḥ sopi lipyaḥ svayaṃ pade// \E
%sadoditaṃ manaścaṃdraḥ  sūryodaya ivekṣate  tena grasto manaścaṃdraḥ sopi līnaḥ svayaṃ pade \P
%sadoditamanaścadraḥ  sūryodaya ivekṣate  tena grasto manaḥ/ ścaṃdraḥ molinasya yaṃ pade \B
%sadoditamanaścaṃdraḥ  sūryodaya ivekṣate//  tena grasto manaś caṃdraḥ so pi linaṃ svayaṃ pade... \L
%\om                                                                 \N1
%\om                                                                 \D1
%\om                                                                 \N2
%sadoditamanaḥścaṃdraḥ  sūryodaye calakṣyate tena graste manaś caṃdraḥ so pi linaṃ svayaṃ pade \U1
%madohitaṃ manaś candraḥ sūryodaya ivekṣate// tena graste manaś candraḥ sopi lipyaḥ svayaṃ pade// \U2
%-----------------------------
%
%-----------------------------
%padam eva mahānagniryame grastaṃ kalāmayam/ evaṃ candrārkavahnīnāṃ saṃketaḥ paramārthataḥ// \E [p.64]
%    m eva mahānagnir yena grastaṃ kalāmayaṃ  evaṃ caṃdrārkavahnīnāṃ saṃketaḥ paramārthataḥ   \P
%padam eva mahānagnirsūrya grastaṃ kalāmayam/ evaṃ caṃdrārkvavahnīnāṃ saṃketanaṃ paramārthataḥ// \B
%padam eva mahānagniḥ sūryagrastaṃ kalāmayaṃ// evaṃ caṃdrār kavavahnīnāṃ saṃketanaṃ paramārthataḥ// \L
%\om                                                                 \N1
%\om                                                                 \D1
%\om                                                                 \N2
%padam eva mahānagnir yena grastaṃ kalāmayaḥ   evaṃ caṃdrār kavatāṃ saṃketaḥ paramārthataḥ vā \U1
%padam eva mahān agnir yena grastaṃ kalāmayaṃ// evaṃ caṃdrārkavahnīnāṃ saṃketaḥ paramārthataḥ// \U2
%-----------------------------
%
%----------------------------
%idānīṃ yogasiddher anaṃtaram etādṛśaṃ jñānam atpadyate/ \E
%idānīṃ yogasiddhe naranaṃtaraṃ etādṛśaṃ jñānam utpadyate \P %%%7669.jpg 
%idānīṃ yo yogasiddhar anaṃtaraṃ/ etādṛśaṃ jñānam utpadyate \B
%idānīṃ yogasiddhar anaṃtaraṃ etādṛśaṃ jñānaṃ utpadyate... \L
%\om                                                                 \N1
%\om                                                                 \D1
%\om                                                                 \N2
%idānīṃ yogasiddhar anaṃtaraṃ etādṛśajñānam utpadyate... \U1
%idānīṃ yogasiddher anaṃtaraṃ etādṛśaṃ jñānam utpadyate// \U2
%-----------------------------
%
%----------------------------
%yadā nāsti svayaṃ karttā kāraṇaṃ na kulākulam// avyaktaṃ na paraṃ tattvamanāmā vidyate tadā//1//  \E
%yadā nāsti svayaṃ kartā  kāraṇaṃ na kulākulam   avyaktaṃ na paraṃ tatva manām āvidyate tadā/  \P
%yadā nāsti svayaṃ kartā  kāraṇaṃ na kulākulam// avyaktaṃ na paramanā sā vidyate tadā/  \B
%yadā nāsti svayaṃ kartā  kāraṇaṃ na kulākulam// avyaktaṃ na paramanā sā vidyate tadā...  \L
%\om                                                                 \N1
%\om                                                                 \D1
%\om                                                                 \N2
%padā nāsti svayaṃ karttā  kāraṇaṃ na kulākulam  avyaktaṃ na paraṃ tatvaṃ manā bhā?? vidyate tadā...  \U1
%yadā nāsti svayaṃ karttā kāraṇaṃ// na kulākulaṃ// avyaktaṃ na paraṃ tatvamanāmā vidyate tadā// \U2
%-----------------------------
%
%----------------------------
%anāmā ekaḥ kaścitpuruṣo varttate/ \E
%anāmā ekaḥ kaścitpuruṣo varttate \P
%anāmā ekapuruṣo varttate/ \B
%anāmā ekapuruṣo vartate// \L
%\om                                                                 \N1
%\om                                                                 \D1
%\om                                                                 \N2
%anāmayekakāścit puruṣo varttate \U1 %%%296.jpg
%anāmā ekaḥ kaścitpuruṣo varttate// \U2
%-----------------------------
%
%----------------------------
%anāmnaś ca parāvaraḥ parātparaḥ paraṃ padaṃ paramapadātparaṃ śūnyaṃ śūnyānniraṃjanam anāmnaḥ paṃcaguṇāsteṣvanutattvam akhaṇḍatva manuparṇadalānām aṣṭadalānāṃ madhya ekaṃ kaṭhinaṃ bhavati/ \E
%anāmnaḥ    parābaraḥ parāvarātparamapadaṃ paramapadātparamaśūnyaṃ śūnyānniraṃjanaḥ anāmnaḥ paṃcaguṇāḥ anutyannatvaṃ  akhaṇḍatvaṃ anupamatvaṃ  \P
%anāmnaś ca parāvarāparāvarāt paraṃ pada/ param apadāt param aśūnyaṃ śūnyā niraṃjanaṃ/ amnaḥ paṃcaguṇāḥ/ anutyanatvaṃ/ akhaṃḍatvaṃ/ \B
%anāmnaś ca parāvarāparāvarāt paraṃ padaṃ param apadāt param aśūnyaśūnyā niraṃjanaṃ/ anāmnaḥ paṃcaguṇāḥ// anutyanatvaṃ// akhaṃḍatvaṃ// anupamatvaṃ \L
%\om                                                                 \N1
%\om                                                                 \D1
%\om                                                                 \N2
%anāthaḥ    parāvaraś ca parāvarāt paraṃ padaṃ paramapadāt paramaṃ śūnyaṃ śūnyāniraṃjanaḥ anāmnāḥ paṃcaguṇāḥ anutpannatvaṃ akhaṇḍatvaṃ ācalatvaṃ anupamatvaṃ \U1
%anāmnaś ca parāvaraḥ// parāvarāt  paraṃ padaṃ// paramapadātparmaṃ śūnyaṃ// śūnyānniraṃjanaṃ// anāmnaḥ paṃcaguṇāḥ// anutpannatvaṃ// akhaṃḍatvaṃ// anupamatvaṃ// \U2
%-----------------------------
%
%----------------------------
%tadaṣṭadalaṃ kamalaṃ hṛdaye tiṣṭhati/ \E
%\om \P
%\om \B
%\om \L
%\om                                                                 \N1
%\om                                                                 \D1
%\om                                                                 \N2
%\om \U1
%\om \U2
%-----------------------------
%
%----------------------------
%te ubhaye hṛdaye tiṣṭhataḥ/ \E
%\om \P
%\om \B
%\om \L
%\om                                                                 \N1
%\om                                                                 \D1
%\om                                                                 \N2
%\om \U1
%\om \U2
%-----------------------------
%
%----------------------------
%prathame dale śabdāstiṣṭhanti/ \E
%\om \P
%\om \B
%\om \L
%\om                                                                 \N1
%\om                                                                 \D1
%\om                                                                 \N2
%\om \U1
%\om \U2
%-----------------------------
%
%----------------------------
%dvitīyadale sparśaḥ/ \E
%\om \B
%\om \L
%\om                                                                 \N1
%\om                                                                 \D1
%\om                                                                 \N2
%\om \U1
%\om \U2
%-----------------------------
%
%----------------------------
%tṛtīye dale rūpaṃ tiṣṭhati/ \E
%\om \P
%\om \B
%\om \L
%\om                                                                 \N1
%\om                                                                 \D1
%\om                                                                 \N2
%\om \U1
%\om \U2
%-----------------------------
%
%----------------------------
%caturthe dale rasastiṣṭhati/ \E
%\om \B
%\om \L
%\om                                                                 \N1
%\om                                                                 \D1
%\om                                                                 \N2
%\om \U1
%\om \U2
%-----------------------------
%
%----------------------------
%paṃcame dale gandhaṃ tiṣṭhati/ \E
%\om \P
%\om \B
%\om \L
%\om                                                                 \N1
%\om                                                                 \D1
%\om                                                                 \N2
%\om \U1
%\om \U2
%-----------------------------
%
%----------------------------
%paṣṭhadale cittaṃ tiṣṭhati/ \E
%\om \P
%\om \B
%\om \L
%\om                                                                 \N1
%\om                                                                 \D1
%\om                                                                 \N2
%\om \U1
%\om \U2
%-----------------------------
%
%----------------------------
%saptame dale buddhistiṣṭhati/ \E
%\om \P
%\om \B
%\om \L
%\om                                                                 \N1
%\om                                                                 \D1
%\om                                                                 \N2
%\om \U1
%\om \U2
%-----------------------------
%
%----------------------------
%aṣṭame dalehaṃkārastiṣṭhati/ \E
%\om \P
%\om \B
%\om \L
%\om                                                                 \N1
%\om                                                                 \D1
%\om                                                                 \N2
%\om \U1
%\om \U2
%-----------------------------
%
%----------------------------
%etadaṣṭadalamadhye pṛthivyākāro varttate/ \E
%\om \P
%\om \B
%\om \L
%\om                                                                 \N1
%\om                                                                 \D1
%\om                                                                 \N2
%\om \U1
%\om \U2
%-----------------------------
%
%----------------------------
%atha ca tatkamalamadhye mukhaṃ tiṣṭhati/ \E
%\om \P
%\om \B
%\om \L
%\om                                                                 \N1
%\om                                                                 \D1
%\om                                                                 \N2
%\om \U1
%\om \U2
%-----------------------------
%
%----------------------------
%asya kamalasya nādātprakāśo bhavati// \E
%\om \P
%\om \B
%\om \L
%\om                                                                 \N1
%\om                                                                 \D1
%\om                                                                 \N2
%\om \U1
%\om \U2
%-----------------------------
%
%----------------------------
%prakāśānaṃtaraṃ kamalamūrdhvamukhaṃ bhavati/ \E
%\om \P
%\om \B
%\om \L
%\om                                                                 \N1
%\om                                                                 \D1
%\om                                                                 \N2
%\om \U1
%\om \U2
%-----------------------------
%
%----------------------------
%tathā sūryaprakāśānantaraṃ tadā saromadhye kamalaṃ[ p.66] vikasati/ \E
%\om \P
%\om \B
%\om \L
%\om                                                                 \N1
%\om                                                                 \D1
%\om                                                                 \N2
%\om \U1
%\om \U2
%-----------------------------
%
%----------------------------
%tathedamapyātmā prakāśānantaramūrdhvamukhaṃ vikasati/ \E
%\om \P
%\om \B
%\om \L
%\om                                                                 \N1
%\om                                                                 \D1
%\om                                                                 \N2
%\om \U1
%\om \U2
%-----------------------------
%
%----------------------------
%tanmadhye paramānandarūpā bhūmirbhavati/ \E
%\om \P
%\om \B
%\om \L
%\om                                                                 \N1
%\om                                                                 \D1
%\om                                                                 \N2
%\om \U1
%\om \U2
%-----------------------------
%
%----------------------------
%tasyāhaṃ sohamiti saṃjñā tasyā madhye svātmano dhyānāddinedine hyāyur varddhate/ \E
%\om \P
%\om \B
%\om \L
%\om \N1
%\om                                                                 \D1
%\om \N2
%\om \U1
%\om \U2
%-----------------------------
%
%----------------------------
%rogo dūre bhavati// \E
%\om \P
%\om \B
%\om \L
%\om                                                                 \N1
%\om                                                                 \D1
%\om                                                                 \N2
%\om \U1
%\om \U2
%-----------------------------
%
%----------------------------
%guṇāḥ kartṛtvaṃ jñātṛtvamabhyāsatvaṃ kalatvaṃ sarvajñatvaṃ prakāśasya guṇāḥ sakalaḥ niṣkalaḥ sarvaiḥ saha samatā viśrāṃtiḥ tata etādṛśamutpadyate/ \E
%\om \P
%\om \B
%\om \L
%\om                                                                 \N1
%\om                                                                 \D1
%\om                                                                 \N2
%\om \U1
%\om \U2
%-----------------------------
%
%----------------------------
%ādyaḥ ātmā ātmana ākāśaḥ ākāśādvāyuḥ vāyostejaḥ tejaso jalaṃ jalāt pṛthvī/ \E
%\om \P
%\om \B
%\om \L
%\om                                                                 \N1
%\om                                                                 \D1
%\om                                                                 \N2
%\om \U1
%\om \U2
%-----------------------------
%
%----------------------------
%atrātmanaḥ pañcaguṇāḥ agrāhyaḥ, anantaḥ, avācyaḥ, agocaraḥ,[p.67] aprameyaś ca ākāśasya pañcaguṇāḥ/ praveśaḥ niṣkramaṇaṃ, chiṃdraṃ, śabdādhāraḥ, bhrāṃtinilayatvam/ \E
%\om \P
%\om \B
%\om \L
%\om                                                                 \N1
%\om                                                                 \D1
%\om                                                                 \N2
%\om \U1
%\om \U2
%-----------------------------
%
%----------------------------
%mahāvāyoḥ pañcaguṇāḥ/ calanaṃ śeṣasaṃcāraḥ, sparśaḥ, dhūmravarṇatā, tejaḥ saṃcaraḥ tejasaḥ pañcaguṇāḥ/ dahanaṃ, jvālarūpaṃ, uṣṇatā, rakto varṇaḥ// \E
%\om \P
%\om \B
%\om \L
%\om                                                                 \N1
%\om                                                                 \D1
%\om                                                                 \N2
%\om \U1
%\om \U2
%-----------------------------
%
%----------------------------
%apāṃ paṃca guṇāḥ/ pravāhaḥ śithilatā dravaḥ madhuratā śvetavarṇaḥ/ pṛthivyāḥ paṃca guṇāḥ/ [p.68] sthūlatā sākāratā kaṭhinatā gandhavattā pītavarṇatā \E
%\om \P
%\om \B
%\om \L
%\om                                                                 \N1
%\om                                                                 \D1
%\om                                                                 \N2
%\om \U1
%\om \U2
%-----------------------------
%
%---------------------------- 
%%%%%%%%%%%%%%%%%%%%%%%%%%%%check \B a few lines up!!!%%%%%%%%%% section \om in B and L%%%%%%%%%%%%%%%%%%%
%-----------------------------
%
%----------------------------
%ananyatvaṃ ceti/ parāvarasya paṃca guṇāḥ – niścalatvaṃ niṣkarmatvaṃ paripūrṇaṃtvaṃ vyāpakatvamakalatvaṃ ceti/ paramapadasya paṃca guṇāḥ nityaṃ nirantaraṃ nirākāraṃ nirniketanaṃ niścalatvaṃ ceti/ śūnyasya pañcaguṇāḥ – līnatā ghūrṇatā mūrchā unmanībhāvaḥ alasatvaṃ ceti/ niraṃjanasya paṃca guṇāḥ satyā sahabhāvā sattā svarūpatā samatā ceti/ \E
%
%ananyatvaṃ ceti nirmalatvaṃ paripūṇatvaṃ vyāpakatvaṃ akalatvaṃ ceti paramapadasya üaṃcaguṇāḥ nityaṃ niraṃtaraṃ nirniketanaṃ ceti śunyasya paṃcaguṇāḥ līnatā pūrṇatām urchā unmanībhāvaḥ alasatvaṃ ceti niraṃjanasya paṃcaguṇāḥ satyaḥ sahaḥ svabhāvaḥ sattasvarūpatāḥ \P
%
%anatvaṃ cetiḥ/ parāvarasya paṃcaguṇāḥ/ niścalatvaṃ nirmalatvaṃ/ paripūrṇatvaṃ/ vyāpakatvaṃ/ akalaṃtvaṃ ceti/ paramapadasya paṃcaguṇāḥ/ niśca????? raṃgani rākāraṃ nirniketunaṃ ceti/ śūnyasyapaṃcaguṇāḥ/ līnatāḥ pūrṇatāmurchā unmabhāvaḥ/ ālasyatvaṃ ceti/ niraṃjanasya paṃcaguṇāḥ/ satyaḥ saha/ svabhāvaḥ satasvarūpatā/ \B
%
%anatvaṃ ceti parāvarasya paṃcaguṇāḥ// niścalatvaṃ nirmalatvaṃ paripūrṇatvaṃ vyāpakatvaṃ akalatvaṃ ceti paramapadasya paṃcaguṇāḥ// nityaniraṃstaganirākāraṃ nirviketunaṃ ceti// śūnyasya paṃcaguṇāḥ// līnatāḥ pūrṇatām urchā unmanībhāvaḥ ālasyatvaṃ ceti niraṃjanasya paṃcaguṇāḥ satyaḥ sahasvabhāvaḥ satasvarūpatā... \L  0033.jpg
%
%                                                                                                                                                                               ti anasya paṃcaguṇāḥ// līnatā// pūrṇatā// mūrchā// unmanībhāva/ alasatvaṃ ceti niraṃjanasya paṃcaguṇāḥ// satya/ sahaja/ svabhāvasattā/ svarūpatā/  \N1?????????????????????????CHECK!!!!
%
%\om                                                                 \D1
%%                                                                                                                                                                               ti anyasya paṃcaguṇāḥ// līnatā// pūrṇatā// mūrchā// unmanībhāva/ alasatvaṃ ceti niraṃjanasya/ paṃcaguṇāḥ// satya/ sahaja/ svabhāvasattā/ svarūpatā/  \N2
%U2
%ananyastvaṃ ceti parāvarasya paṃcaguṇāḥ niścalatvaṃ nirmalatvaṃ paripūrṇatvaṃ vyāpakatvaṃ prakāśatvaṃ paramapadasya paṃcaguṇāḥ nityyynijaniraṃtara nirākāranimilaketanā śūnyaḥsya paṃcaguṇā līna tā pūrṇatā mūrchā unmanī bhāva alasatvaṃ niraṃjanasya paṃcaguṇāḥ satyasahajasvabhāvasattāsvarūpatā  \U1
%
%ananyatvaṃ nirmalatvaṃ ceti// parāvarasya paṃcaguṇā// niścalatvaṃ// paripūrṇatvaṃ// vyāpakatvaṃ// akalatvaṃ// nirvikāratvaṃ ceti// paramapadasya paṃcaguṇāḥ// nityaṃ// nirantarā// nirākārā// nirniketanaṃ// ....... ceti/ śūnyasya pañcaguṇāḥ// līnatā// ghūrṇatā// mūrchā// unmanībhāvaḥ// alasatvaṃ ceti// niraṃjanasya paṃca guṇaḥ satya// saha// svabhāva// sattā// svarūpatā ceti/ \U2  %425.jpg

%-----------------------------
%
%----------------------------
%idānīṃ piṃḍotpattiḥ kathyate//[p.69] \E
%idānīṃ piṃḍotpattiḥ kathyate  \P
%idānīṃ piṃḍotpattiṃ kathyate/ \B
%idānīṃ piṃḍotpattiṃ kathyate/ \L
%idānīṃ piṃḍotpattiḥ kathyate/ \N1
%idānīṃ piṃḍotpatti kathyate// \N2 %%%%S.15 of 20 transcription stopped here 
%\om                                                                 \D1
%idānīṃ piṃḍotpatti kathyate   \U1
%idānīṃ piṃḍotpattiḥ kathyate  \U2
%-----------------------------
%Now the generation of the body is taught. 
%-----------------------------
%anāditaḥ   paramātmā   paramātmanaḥ   paramānaṃdaḥ paramānaṃdātprabodhaḥ prabodhāccidudayaḥ cidudayātprakāśaḥ/ \E
%anāditaḥ   paramātmā   paramātmanaḥ   paramānaṃdaḥ paramānaṃdāt prabodhaḥ prabodhācci udayaḥ vidudayāt prakāśaḥ \P %%%7670.jpg 
%anāditaḥ   paramātmā   paramātmanaḥ/  paramānaṃdaḥ/ paramānaṃdāt prabodhaḥ prabodhācci udayaviduyātprakāśaḥ/ \B
%anāditaḥ   paramātmā   paramātmanaḥ   paramānaṃdaḥ paramānaṃdāt prabodhaḥ prabodhāccid udayacid udayāt prakāśaḥ... \L
%anāditaḥ   paramātmā/  paramātmanaḥ   paramānaṃdaḥ/ paramānaṃdāt prabodhaḥ/ prabodhāccid dayaḥcid dayacidudayāt prakāśaḥ// \N1
%anāditaḥ   paramātmā/  paramātmanaḥ   paramānaṃdaḥ/ paramānaṃdāt prabodhaḥ/ prabodhāccid dayaḥcid udayacidudayāt prakāśaḥ \N2
%\om                                                                 \D1
%anāditaḥ   paramātmā   paramātmanaḥ   paramānaṃdaḥ paramānaṃdāt prabodhaḥ cittapaḥ cittaprakāśaḥ  \U1
%anāditaḥ// paramātmā// paramātmanaḥ// paramānaṃdaḥ// paramānaṃdāt prabodhaḥ// prabodhācid udayaḥ// cidudayāt prakāśaḥ// \U2 XX
%-----------------------------
%From without beginning the supreme self, from the supreme self supreme bliss, from supreme bliss awakening, from awakending manifestation, from manifestation light. 
%-----------------------------
%tatra paramātmanaḥ paṃca guṇāḥ – akṣayaḥ, abhedyaḥ, acchedyaḥ, adāhyaḥ, avināśī/ \E
%tatra paramātmanaḥ paṃca guṇāḥ  akṣayaḥ  abhedyaḥ  aṣṭadyaḥ avināśī   \P
%tatra paramātmanaḥ paṃca guṇāḥ/ akṣayaḥ/ abhedyaḥ/ avināśī/ adāhyaḥ/  \B
%tatra paramātmanaḥ paṃca guṇāḥ//akṣayaḥ/ abhedyaḥ/ avināśī/ adāhyaḥ/  \L
%tatra paramātmanaḥ paṃcaguṇāḥ// akṣayaḥ/ abhedyaḥ/ acchedyaḥ/ adāhyaḥ/ avināśī/ // \N1
%tatra paramātmanaḥ paṃcaguṇāḥ// akṣayaḥ  abhedyaḥ  acchedyaḥ  adāhyaḥ  avināśī // \N2
%\om                                                                 \D1
%tatra paramātmanaḥ paṃca guṇāḥ akṣayyaḥ  avadyha abhedyaḥ ādṛṣya avināsī /  \U1
%tatra paramātmanaḥ// paṃca guṇāḥ// akṣayaḥ// abhedyaḥ// achedyaḥ/ adāhyaḥ avināsaḥ// \U2
%-----------------------------
%
%-----------------------------
%paramānaṃdasya paṃcaguṇāḥ – sphu raṇaḥ, kiraṇaḥ, visphuraṇaḥ, ahaṃtā, harṣavattvam/ \E
%paramānaṃdasya paṃcaguṇā   sphuraṇaḥ kiraṇaḥ visphuraṇaḥ ahaṃtā harṣatatvaṃ \P
%paramānaṃdasya paṃcaguṇāḥ/ sphuraṇakiraṇa visphuriṇa ahaṃtā harṣavatvaṃ/  \B
%paramānaṃdasya paṃcaguṇāḥ/ sphuraṇa/ kiraṇā// visphura/ ahaṃtā// harṣavatvaṃ/  \L
%paramānaṃdasya paṃcaguṇāḥ// sphuraṇakiraṇa visphuriṇa ahaṃtā harṣavatvaṃ/ \N1
%paramānaṃdasya paṃcaguṇāḥ// sphuraṇa/ kiraṇa visphura ahaṃtā harṣavatvaṃ/ \N2
%\om                                                                 \D1
%paramānaṃdasya paṃcaguṇāḥ  sphuraṇaḥ kiraṇaḥ visphuraḥ ahaṃtā dhairyatva \U1 
%paramānaṃdasya paṃcaguṇāḥ// sphuraṇa// kiraṇa// visphuraṇa// ahaṃtā// harṣavārttvaṃ// \U2 
%-----------------------------
%
%-----------------------------
%prabodhasya paṃca guṇāḥ – layaḥ, ullāsaḥ, vibhāsaḥ, vicāraḥ, prabhā/ \E
%prabodhasya paṃcaguṇāḥ layaḥ ullāsā vibhāsā vicāraḥ prabhā \P
%prabodhasya paṃcaguṇāḥ/ layā ullāsā vibhāsā vicāraḥ/ \B
%prabodhasya paṃcaguṇāḥ/ laya/ ullāsa/ vibhāsa/ vicāra/ prabhā/ \L
%prabodhasya paṃcaguṇāḥ/ laya/ ullāsā vibhāsā vicāraḥ/ \N1
%prabodhasya paṃcaguṇāḥ  laya/  ullāsā/ vibhāsā/ vicāra \N2
%\om                                                                 \D1
%bodhasya paṃcaguṇāḥ ullāsā vibhāsā vicāra samādhi \U1 %%%296 Ende.... weiter mit 297!!
%prabodhasya paṃcaguṇāḥ// layaḥ// ullāsaḥ// vibhāsaḥ// vicāraḥ// prabhā// \U2 
%-----------------------------
%
%----------------------------
%cidudayasya paṃca śarīramadhye paṃca mahābhūtāni// [P.70] teṣāṃ guṇāḥ kathyante tatra pṛthivyā guṇāḥ – asthimāṃsanāḍīlomāni vāk/ tatrodakaguṇāḥ- lālā, mūtraṃ, śuklaṃ, raktaṃ, prasvedaḥ/ tejaso guṇāḥ- kṣudhā tṛṣā nidrā glāniḥ ālasyam/ \E
%
%cidudayasya paṃcaguṇā kartṛtve jñātṛtvaṃ abhyāsatvaṃ kalanaṃtvaṃ sarvajñatvaṃ prakāśasya paṃcaguṇāḥ sakala niṣkvalā saṃbodhanā samatā viśrāṃtiḥ tat etādṛśaṃ jñānam utpadyate// ādyaḥ// ādhyādātmā ātmana ākāśaḥ ākāśād vayuḥ vāyos tejaḥ tejasojalaṃ jalāt pṛthivī ātrātmanaḥ paṃcaguṇāḥ agrātyaḥ anaṃtaḥ avācyaḥ agocaraḥ aprameyaś ca ākāśasya paṃcaguṇāḥ praveśaḥ niśkrumāṇaṃ chidraṃ śabdadhāraḥ bhrāṃtinilayatvaṃ mahāvāyoḥ paṃcaguṇāḥ calanaṃ śoṣaḥ saṃcāraḥ sparśaḥ dhūmravarṇatā tejasaḥ paṃcaguṇāḥ dahanaṃ jvālārūpaṃ uṣṇatārakto varṇāḥ apāṃpaṃcaguṇāḥ pravāhaśithilatādravaḥ madhurasatā śvetavarṇāḥ pṛthivyāḥ paṃcaguṇāḥ kathyaṃte asthimāṃsaṃ nāḍī lomāni vākṛttatro dakaguṇāḥ lālāmuvaṃ śukraṃ raktaṃ prasvedaḥ tejaso guṇāḥ kṣudhā tṛṣā nidrā glāniḥ ālasyaṃ \P
%
%abhāvihṛdayasya paṃcaguṇāḥ// katutvaṃ jñātṛtvaṃ abhyāsatvaṃ/ kalanatvaṃ saṃvajñatvaṃ/ prakāśasya paṃcaguṇāḥ/ sakala/ niṣkvala/ saṃbodhanasamatāviśrāṃti/ tat etādraśaṃ jñānaṃ mutpadyate/ ādyaḥ/ ādhyātmanaṃ/ ākāśaḥ/ ākāśādvayoḥ/ vāyostejaḥ/ tejasojalaṃ/ jalātpṛthvī/ ādyātmanaḥ paṃcaguṇāḥ/ preveśaḥ niśkru?māṇaṃ/ chidraṃ/ śabdadhāraṃ/ bhrāṃtinilayatvaṃ/ mahāvoyoḥ paṃcaguṇāḥ/ dahanaṃ jvālārūpaṃ/ uṣṇatāraktovarṇaḥ/ apapaṃcaguṇāḥ/ pravāhaḥ śithatādravaḥ madhuradatāśvetavarṇāḥ/ pṛthvīyāpaṃcaguṇāḥ/ athyate/ asthimāṃsnāḍītvak lomāni/ tatrodakaguṇāḥ/ lālamutraśukraṃ raktaṃ prasvedaḥ/ tejasoguṇāḥ/ kṣudhāṃ tṛṣā nidrāglāni ālasyaṃ// \B
%
%abhācid udadayasya paṃcaguṇāḥ// akartutvaṃ jñātṛtvaṃ abhyāsatvaṃ kalanatvaṃ saṃvajñatvaṃ/ prakāśasya paṃcaguṇāḥ// sakala/ niṣkvala/ saṃbodhanasamatā viśrāṃti tat etādraśaṃ jñānam utpadyate// ādyaḥ// ādhyātmanaḥ/ ākāśaḥ/ ākāśād vayuḥ/ vāyos tejaḥ tejasojalaṃ jalāt pṛthvī ādyātmanaḥ paṃcaguṇāḥ agrātdya anaṃtaḥ avācyaḥ agocaraḥ aprameyaś ca ākāśapaṃcaguṇāḥ preveśaḥ niśrumāṇaṃ chidraṃ śabdadhāraṃ bhrāṃtinilayatvaṃ mahāvāyoḥ paṃcaguṇāḥ calanaṃ śoṣaḥ saṃcāraḥ sparśa dhūmravarṇatā tejasaḥ paṃcaguṇāḥ dahanaṃ jvālārūpaṃ/ uṣṇatāraktavarṇaḥ apapaṃcaguṇāḥ pravāhaḥ śithilatādravaḥ madhurasatā śvetavarṇāḥ/ pṛthvīyāpaṃcaguṇāḥ kathyaṃte/ asthimāṃsanāḍītvak lomāni tatrodakaguṇāḥ lālamutraśukraṃ raktaṃ prasvedaḥ// tejasoguṇāḥ/ kṣudhā tṛṣā nidrā glāni ālasyaṃ// \L
%
%cidudayasya paṃcaguṇāḥ/ kartṛtvaṃ/ jñātṛtvaṃ/ abhyāsatvaṃ/ kalanaṃtvaṃ/ sarvajñatvaṃ prakāśasya paṃcaguṇāḥ/ sakala/ nikala/ saṃbodhana/ samatā/ viśrāṃti/  tata etādṛśaṃ jñānam upadyate/ ādyaḥ/ ādhyādātmā/ ātmanaḥ/ ākāśaḥ/ ākāśatya vanḥ pavanāt tejaḥ/ tejasodakāt pṛthvī/ tatra ātmanaḥ/ paṃcaguṇāḥ/ agrādyaḥ?/ anaṃtaḥ/ avācyaḥ/ agocaraḥ/ aprameyaś ca/ ākāśasya paṃcaguṇāḥ/ praveśaḥ niśkrumāṇaṃ/ cchidraṃ/ śabdadhāraḥ bhrāṃti nilayatvaṃ// mahāvāyoḥ paṃcaguṇāḥ/ calanaṃ/ śoṣaḥ/ saṃcāraḥ/ sparśaḥ/ dhūmravarṇatā/ tejasaḥ paṃcaguṇāḥ/ dahanaṃ/ jvālārūpaṃ/ uṣṇatā/ rakto/ varṇaḥ/ āpo paṃcaguṇāḥ// pravāha/ śithilatā/ drava/ madhurarasatā/ śvetavarṇtā/ pṛthivyā guṇāpaṃca// sthulatā???// (S.12 verso Z.1) sakāratā/ kathinatā/ gaṃdhava??tta/ pītavarṇaḥ/ idānīṃ śarīramadhye paṃca mahāsūtāni? kathyate/ teṣāṃ guṇāḥ kathyate/ tatra pṛthivyā guṇāḥ// asthi/ māṃsa/ nāḍī/ lomāni/ tvak/ netrodake guṇāḥ// lālāmutraṃ/ śukraṃ/ raktaṃ/ prasvedaḥ/ tejaso guṇāḥ// kṣudhā/ tṛṣā/ nidrā/ glāni/ alasyaṃ// \N1
%
%cidudayasya paṃcaguṇāḥ/ kartṛtvaṃ/ jñātvaṃ/ ...satvaṃ/ kalanātvaṃ/ sarvajñatvaṃ// prakāśasya paṃcaguṇāḥ// sakala/ nikala/ saṃbodhana/ samaṃtā/ viśrāṃti//  tata etādṛśaṃ jñānam upadyate/ ādya/ ādhyādātmā/ ātmana/ ākāśa/ ākāśatya vak? pavanāt tejaḥ/ tejaso udakāt pṛthvī// tatrātmanaḥ paṃcaguṇāḥ// agrāṃhya?/ anaṃtaḥ/ avācya/ agocaraḥ/ aprameyaś ca/ ākāśasya paṃcaguṇāḥ/ praveśaḥ niśkrumāṇaṃ/ cchidraḥ śabdadhāraḥ bhrāṃti nilayatvaṃ// mahāvāyoḥ paṃcaguṇāḥ// calana/ śoṣaḥ saṃcāraḥ sparśaḥ dhūmravarṇatā/ tejasaḥ paṃcaguṇāḥ// dahanaṃ/ jvālārūpaṃ/ uṣṇatā/ raktavarṇaḥ// āpo paṃcaguṇāḥ// pravāha/ śithilatā/ dravamadhura/ rasatā/ śvetavarṇtā// pṛthivyā guṇāpaṃca// syūlatā/ sākāratā/ kathinatā/ pītavarṇaḥ/ idānīṃ śarīramadhye paṃca mahābhūtāni kathyate// teṣā guṇāḥ kathyate// tatra pṛthivyā guṇāḥ// asthi/ māṃsa/ nāḍī/ lomāni/ tvak/ // netrodaka guṇāḥ// lālā/ mūtraṃ/ śukraṃ/ prasvedaḥ// tejaso guṇāḥ// kṣudhā/ tṛṣā/ nidrā/ glāni/ ālasyaṃ// \N2
%
%\om                                                                 \D1
%
%udadayasya paṃcaguṇāḥ katṛtvaṃ jñānatvaṃ abhyāsatvaṃ kalyanasarvaśatvaṃ prakāśasya paṃcaguṇāḥ sakalā saṃbodhanasamatā viśrāṃti tataḥ ***viśvāsa etādṛśaṃ jñānam utpadyate ātmanaḥ ākāśaḥ***INSERTION*** ākāśāt pavanaḥ pavanāttejaḥ tejaḥ sa udakaṃ udakāt pṛthvī tatra ātmanaḥ paṃcaguṇāḥ agrātdyaḥ anaṃtaḥ avācyaḥ agocaraḥ aprameyaś ca ākāśasya paṃcaguṇāḥ preveśaḥ nikrumāṇaḥ chidraṃ śabdadhāraṃ bhrāṃte nijatvaṃ mahāvāyor guṇāḥ pracālanā śoṣaḥ saṃcāraḥ nirodhanaṃ prasaraṇaṃ vaḥ pṛthivyāguṇāḥ sthalatā sākāratā kaṭhiṇa tāgaṃdha vettā pītavarṇā idānīṃ śrīramadhye paṃca āpaguṇaḥ mahāsveravarṇa tāvāt lālāmutraṃ śukraṃ raktaṃ svedaḥ tejaso guṇāḥ kṣudhā tṛṣā nidrā glāni ālasya \U1
%
%cidudayasya paṃcaguṇā// kartṛtvaṃ// jñātṛtvaṃ// abhāsatvaṃ// kalanatvaṃ// sarvajñatvaṃ// prakāśasya paṃcaguṇaḥ// sakalā ti??sā?? saṃbodhanaṃ// samatā// viśrāṃtiḥ// tataḥ// etādṛśyaṃ jñānam utpadyate// ādyaḥ// ādhyā// dātmā// ātmanaḥ// ākāśaḥ// ākāśād vayuḥ// vāyos tejaḥ// tejasorjalaṃ// jalāt pṛthvī// ātmanaḥ paṃcaguṇaḥ// agrātyaḥ// anaṃtaḥ// avācyā// agocaraḥ// aprameyaś ca// ākāśasya paṃcaguṇaḥ// praveśaniśkraṇaṃ chidraṃḥ// śabdādhāraḥ// bhrāṃtinilayatvaṃ// mahāvāyoḥ// paṃcaguṇaḥ// calanaṃ// śoṣaṇaṃ// saṃcāraḥ// sparśaḥ// dhūmravarṇatā// tejasaḥ paṃcaguṇaḥ// dahanaṃ// jvālā// rūpaṃ// uṣṭṇatā// raktavarṇāḥ// apāṃpaṃcaguṇāḥ// pravāhaḥ// śithilatādravaḥ// madhuratā// svetavarṇaḥ// pṛthivyāḥ paṃcaguṇāḥ//sthūlatā// kaṭhiṇatā gaṃdhavatā// pītavarṇatā//atha śarīrasya madhye paṃcamahābūtāni// teṣāṃ guṇāḥ kathyaṃte// tatra pṛthvyāguṇāḥ// asti// māṃsa// nāḍī// lomāni// tvakḥ// tatrodakaguṇaḥ lallā// mūtraṃ// śukraṃ//raktaṃ// prasvedaḥ// tejaso guṇaḥ// kṣudhā// tṛṣā// nidrā// glāni// ālasyaṃ// \U2
%-----------------------------
%
%----------------------------
%vāyor guṇāḥ - dhāvanaṃ majjanaṃ nirodhanaṃ prasāraṇamākuṃcanaṃ ceti/ \E
%vāyor guṇāḥ   dhāvanaṃ majjanaṃ nirodhanaṃ prasāraṇaṃ ākuṃcanaṃ ceti  \P
%vāyo  guṇāḥ/ dhāvanaṃ majjanaṃ nirodhanaṃ/ prasāraṇaṃ/ ākuṃcanaṃ ceti/ \B
%vāyor guṇāḥ// dhāvanaṃ majjanaṃ nirodhanaṃ prasāraṇaṃ ākuṃcanaṃ ceti... \L
%vāyor guṇāḥ/ dhāvanaṃ/ majjanaṃ/ nirodhanaṃ/ prasāraṇaṃ/ ākuṃcanaṃ/ ceti// \N1
%vāyo  guṇāḥ/ dhāvanaṃ/ majana/ virodhana/ praśaraṇāṃ/ ākūrcana ceti// \N2
%%\om                                                                 \D1
%vāyu guṇā dhāvanaṃ manorodhanaṃ prasāraṇaṃ ākuṃcanaṃ ceti \U1
%vāyo guṇaḥ// dhāvanaṃ// majjanaṃ// nirodhanaṃ// prasāraṇaṃ// ākuṃcanaṃ// \U2
%-----------------------------
%
%----------------------------
%ākāśasya guṇāḥ – rāgadveṣau bhayaṃ lajjā mohaḥ/ \E
%ākāśasya guṇāḥ   rāgadveṣaḥ bhayaṃ lajjā mohaḥ \P
%ākāśasya guṇāḥ/  rāgadveṣbhayaṃ lajjā moha/ \B
%ākāśasya guṇāḥ// rāgadveṣbhayaṃ lajjā moha// \L
%ākāsasya guṇāḥ/  rāgadveṣo/ bhayaṃ/ lajjā/ mohaḥ/   \N1
%ākāsasya guṇāḥ// rāga/ dveṣau/ bhayaṃ/ lajjā/ moha/ \N2
%\om                                                                 \D1 CHECK!!!!
%ākāśasya guṇaḥ   rāgadveṣau bhayaṃ lajjā mohā \U1
%ākāśasya guṇāḥ// rāgaḥ// dveṣaḥ// bhayaṃ// lajjā// mohaḥ// \U2
%-----------------------------
%
%----------------------------
%tad anaṃtaram ekādaśīkā buddhir utpadyate/ \E
%tad anaṃtaram ekādṛśyekā buddher utpadyate  \P
%tad anaṃtaraṃ metādaśī  buddhir utpadyate/ \B
%tad anaṃtaraṃ etādaśī   buddhir utpadyate.. \L
%tad anaṃtaraṃ etādṛśā  ekā buddhir utpadyate/ \N1
%tad anaṃtaraṃ etādṛśī  ekā buddhir utpadyate/ \N2
%\om                                                                 \D1
%tad anaṃtaraṃ etādaśī  ekā buddhir utpadyate.. \U1
%tad anaṃtaram etādṛśyekā buddhir utpadyate// \U2
%-----------------------------
%
%----------------------------
%manobuddhyihaṃkārāścittaṃ caitanyaṃ ceti/ \E
%manobuddhiraṃhaṃkāraścittaṃ caitanyaṃ ceti \P %%%7672.jpg 
%manobuddhirahaṃkāra/ ścittaṃ caitanyaṃ ceti/ \B
%mano buddhir ahaṃkāraścittaṃ caitanyaṃ ceti... \L
%mano buddhir ahaṃkāra cittaṃ ceti/    \N1
%mano buddhir ahaṃkāra cittaṃ ceti//   \N2
%\om                                                                 \D1
%mano buddhir ahaṃkāraścittaṃ ceti... \U1
%mano buddhir ahaṃkāraḥ// cittaṃ cautanyaṃ ceti// \U2
%-----------------------------
%
%----------------------------
%ete paṃcaprakārā    aṃtaḥ karaṇasya/      manasaḥ ye ca guṇāḥ saṃkalpavikalpamūrkhatvālasatā mananaṃ ceti// \E [p.71]
%ete paṃcāpi prakārā aṃtaḥ karaṇasya       manasaḥ paṃcaguṇāḥ saṃkalpavikalpamūrkhatvajaḍatā mananaṃ ceti    \P
%ete paṃcāpiprakāra/ aṃtaḥ karṇsya         manasaḥ paṃcaguṇāḥ saṃkalpavikalpamūrkhatvajaḍatā mananaṃ ceti/ \B
%ete paṃcāpiprakārāḥ aṃtaḥ karṇsya manasaḥ paṃcaguṇāḥ sakalpavikalpamūrkhatvajaḍatā mananaṃ ceti... \L
%ete paṃcāpiprakārā  aṃtaḥ karaṇasya/ manasaḥ paṃcaguṇāḥ saṃkalpa/ vikalpaḥ/ mūrṣatvaṃ/ jaḍatā/ mananaṃ ceti/ ete paṃcāpiprakārā aṃtaḥ karaṇasya ma ****doubling****\N1
%ete paṃcāprakārā    aṃtakaraṇasya//  manasaḥ paṃcaguṇāḥ saṃkalpaḥ vikalpa mūrkhatvaṃ/ jaḍatā/ mananaṃ ceti// \N2
%\om                                                                 \D1
%ete paṃcāpiprakārā  aṃtaḥ karṇva manasaḥ paṃcaguṇāḥ saṃkalpavikalpamūrṣatvaṃ jaḍatā mananaṃ ceti vā... \U1 %%%298.jpg
%ete paṃcaprakāraḥ   aṃtaḥ karaṇasya manasaḥ paṃcaguṇaḥ// saṃkalpa// vikalpa// mūrkhatva// jaḍatā mananaṃ ceti// \U2
%-----------------------------
%
%----------------------------
%buddheḥ paṃca guṇāḥ/ \E
%buddheḥ paṃcaguṇāḥ   \P
%buddhe  paṃcaguṇāḥ/  \B
%buddheḥ  paṃcaguṇāḥ/  \L
%buddheḥ  paṃcaguṇāḥ/  \N1
%buddheḥ  paṃcaguṇāḥ   \N2
%\om                                                                 \D1
%buddheḥ  paṃcaguṇāḥ  \U1
%buddheḥ  paṃcaguṇaḥ//  \U2
%-----------------------------
%
%----------------------------
%viveko vairāgyaṃ śāntiḥ santoṣaḥ kṣamā ceti/ \E
%vivekaḥ vairāgya śāṃtiḥ saṃtoṣaḥ kṣamā ceti  \P
%vivekavairāgyaśāntiḥ santoṣaḥ kṣamā ceti/ \B
%vivekavairāgyaśāntiḥ santoṣaḥ kṣamā ceti \L
%vivekaḥ/ vairāgya/ śāntiḥ/ santoṣaḥ/ kṣamā ceti/ \N1
%vivekaḥ vairāgya śāntiḥ  santoṣa kṣamā ceti// \N2
%                                           ceti/ \D1
%vivekavairāgyaśāntiḥ saṃtoṣaḥ kṣamā vā \U1
%viveko// vairāgyaṃ// śāntiḥ// santoṣāḥ// kṣamā ceti// \U2
%-----------------------------
%
%-----------------------------
%ahaṃkārasya paṃca guṇāḥ/ \E
%ahaṃkārasya paṃcaguṇāḥ   \P
%ahaṃkārasya paṃcaguṇāḥ/ \B
%ahaṃkārasya paṃcaguṇāḥ// \L
%ahaṃkārasya paṃcaguṇāḥ/  \N1
%ahaṃkārasya paṃcaguṇāḥ// \N2
%%ahaṃkārasya paṃcaguṇāḥ/ \D1
%ahaṃkārasya paṃcaguṇāḥ  \U1
%ahaṃkārasya paṃcaguṇaḥ//  \U2
%-----------------------------
%
%-----------------------------
%ahaṃ mameti etasya duḥkhaṃ svataṃtratā/ \E
% \om \P
% \om \B
% \om \L
% ahaṃ/ mama / etasya duḥkhaṃ/ svatantratā/ \N1
% ahaṃ/ mama/ etasya duḥkhaṃ/ svataṃtratā/  \N2
% ahaṃ/ mama/ etasya duḥkhaṃ/ svataṃtratāḥ/ \D1
% samā etasya svasvataṃtratā \U1
%ahaṃ mama etasya duḥkhaṃ// svataṃ tratāḥ// \U2
%-----------------------------
%
%-----------------------------
%cittasya paṃcaguṇāḥ/ \E
%\om \P
%\om \B
%\om \L
%cittasya paṃcaguṇāḥ/ \N1
%cittasya paṃcaguṇāḥ/ \N2
%cittasya paṃcaguṇāḥ/ \D1
%cittasya naḥ \U1
%cittasya paṃcaguṇaḥ// \U2
%-----------------------------
%
%-----------------------------
%dhṛtiḥ smṛtiḥ/ rāgadveṣau matiḥ/ \E
%dhṛtiḥ smṛtiḥ rāgadveṣamatiḥ     \P
%dhṛtismṛtirāgadveṣam iti/        \B
%dhṛtismṛtirāgadveṣabhīti/        \L
%dhṛtiḥ/ smṛtiḥ/ tyāgaḥ matih     \N1
%dhṛtiḥ smṛtiḥ tyāgaḥ matiḥ//     \N2
%dhṛtiḥ/ smṛtiḥ/ tyāgaṃ matiḥ    \D1
%vṛddhiḥ tyāgaḥ matiḥ             \U1
%dhṛtiḥ// smṛtiḥ// rāgaḥ// dveṣaḥ// matiḥ// \U2
%-----------------------------
%
%-----------------------------
%caitanyasya paṃcaguṇāḥ/ \E
%caitanyasya guṇāḥ  \P
%caitanyasya guṇāḥ/ \B
%caitanyasya guṇāḥ/ \L
%caitanyasya guṇāḥ paṃca// \N1
%caitanyasya guṇāḥ paṃca// \N2
%caitanyasya guṇāḥ/ \D1
%caitanyasya guṇāḥ   \U1
%caitanya    paṃcaguṇaḥ// \U2
%-----------------------------
%
%-----------------------------
%ārṣaṃ vimarśaḥ dhairyaṃ ciṃtanaṃ nispṛhatvam/ \E
%harṣaḥ vimar.. dhairyaṃ ciṃtanaṃ nispṛhatvaṃ  \P
%harṣavimarśadhairyaṃ ciṃtanaṃ nispṛhatvaṃ/    \B
%harṣavimarśadhairyaṃ ciṃtanaṃ nispṛhatvaṃ...  \L
%harṣaḥ, vimarśaḥ, dhairyaṃ, ciṃtanaṃ/ nispṛhatvaṃ//  \N1 %%%note the usage of comma instead of daṇḍa at this place!!
%harṣa   vimarśa dhairyaṃ ciṃtanaṃ nispṛhatvaṃ  \N2
%harṣaḥ vimarśaḥ dhairyaṃ ciṃtanaṃ/ nispṛhatvaṃ/ \D1
%harṣaḥ vimarśaḥ dhairyaṃ cetanā nispṛhatvaṃ   \U1
%harṣaḥ// vimarṣaḥ// dhairyaṃ// cetanaṃ// nispṛhatvaṃ   \U2
%-----------------------------
%
%-----------------------------
%ataḥ paraṃ  kulapaṃcakasya bhedāḥ kathyante/ \E
%ataḥ paraṃ  kulapaṃcakasya bhedāḥ kathyaṃte  \P
%ataḥ paraṃ/ kulapaṃcakasya bhedā kathyaṃte/ \B
%ataḥ paraṃ  kulapaṃcakasya bhedāḥ kathyaṃte/ \L
%tadanaṃtaraṃ kulapaṃcakasya bhedāḥ kathyaṃte/ \N1
%tadanaṃtaraṃ kulapaṃcakasya bhedāḥ kathyate/ \N2
%tadanaṃtaraṃ kulapaṃcakasya bhedāḥ kathyaṃte// \D1
%tadanaṃtaraṃ kulapaṃcakasya bhedāḥ kathyaṃte \U1
%aṃtaḥ paraṃ  kulapaṃcakasya bhedā  kathyaṃte// \U2
%-----------------------------
%
%-----------------------------
%sattvaṃ rajaḥ tamaḥ kālaḥ jīvanam/ \E
%satvaṃ rajaḥ tamaḥ kālaḥ jīvanam \P
%satvaṃ rajastamaḥ kājīvanaṃ/ \B
%satvaṃ rajas tamaḥ kālajīvanaṃ \L
%satva/ raja/ tamaḥ/ kālaḥ/ jīvanaṃ// \N1
%satvarajatamakālajīvanaṃ// \N2
%satvaṃ// rajaḥ/ tamaḥ/ kālaḥ/ jīvanaṃ/ \D1
%satvarajatamakālajīvanaṃ vā \U1
%satvaṃ// rajaḥ// tamaḥ// kālaḥ// jīvanaṃ//  \U2
%-----------------------------
%
%-----------------------------
%tatra sattvaguṇāḥ/ dayā dharmaḥ kṛpā bhaktiḥ śraddhā ceti/ \E
%tatra satvasya guṇāḥ dayā dharmaḥ kṛpā bhaktiḥ śraddhā ceti \P
%tatrasya satvaguṇāḥ dayāḥ dharmakṛpābhaktiśraddhā ceti/ \B
%tatra    satvaguṇāḥ// dayāḥ dharmakṛpābhaktiśraddhā ceti/ \L
%tatra  satvasya guṇāḥ// dayā, dharma, kṛpā/ bhaktiḥ/ śraddhā ceti// \N1
%tatra  satvasya guṇāḥ// dayā dharma kṛpā bhakti śraddhā ceti// \N2
%tatra sattvasya guṇāḥ/ dayā dharmaḥ kṛpā// bhaktiḥ/ śraddhā ceti/ \D1
%tatra    satvasya guṇāḥ dayā dharmakṛpābhaktiḥ śraddhā \U1
%tatra    satvasya guṇāḥ// dayā// dharmaḥ// kṛpā// bhaktiḥ// śraddhā ceti// \U2
%-----------------------------
%
%-----------------------------
%rajasoguṇāḥ/ tyāgaḥ/ bhogaḥ śṛṃgāraḥ svārthaḥ/ vastusaṃgrahaś ceti// \E [p.72]
%rajasoguṇāḥ  tyāgaḥ  bhedaḥ śṛṃgāraḥ svārthaḥ vastusaṃgrahaḥ \P
%rajasoguṇāḥ/ tyāgaḥ/ bhogaḥ śṛṃgāraḥ svārthavastunāsaṃgrahaḥ// \B
%rajasoguṇāḥ  tyāgaḥ bhogaḥ  śṛṃgāraḥ  svārthavastunāṃ saṃgrahaḥ// \L
%rajasoguṇāḥ//tyāgaḥ/ bhogaḥ/śṛṃgāraḥ/  svārthaḥ/ vastusaṃgrahaḥ/ \N1
%rajaso guṇāḥ// tyāga bhoga  śṛṃgāraḥ  svārtha vastu saṃgrahaḥ// \N2
%rajasoguṇāḥ/ tyāgaḥ  bhogaḥ śṛṃgāraḥ  svārthaḥ/ vastusaṃgrahaḥ \D1
%rajasoguṇāḥ  tyāgaḥ         śṛṃgāraḥ  svārthavastusaṃgrahaḥ \U1
%rajoguṇaḥ//  tyāgaḥ// bhogaḥ// śṛṃgāraḥ// svārthaḥ// vastusaṃgrahaḥ// \U2
%-----------------------------
%
%-----------------------------
%tamaso guṇāḥ  vivādaḥ kalahaḥ śokaḥ baṃdhaḥ vañcanam/ \E
%tamaso guṇāḥ  vivādaḥ kalahaḥ śokaḥ baṃdhaḥ vaṃcanaṃ \P
%tamaso guṇāḥ/ vivādaḥ kalahaśoka/ baṃdhavaṃcanaṃ/ \B
%tamo  guṇāḥ// vivādaḥ kalahaśokaiḥ baṃdhavaṃcanaṃ \L
%tamaso guṇāḥ// vivādaḥ kalahaṃ/ śokaḥ/ baṃdhaḥ/ vaṃcanaṃ/ \N1
%tamo  guṇāḥ//  vivāda kalahaṃ śoka vidha vā vaṃcanaṃ smṛtaṃ// \N2
%tamaso guṇāḥ vivādaḥ/ kalahaṃ/ śokaḥ/ baṃdhaḥ/ vaṃcanaṃ/ \D1
%tamaso  guṇaḥ  vivādaḥ kalahaśoka baṃdhavaṃcanā \U1
%tamo guṇaḥ//  viṣādaḥ// kalahaḥ// śokaḥ// baṃdhaḥ// caṃcalaṃ ceti// \U2  %%%427.jpg 
%-----------------------------
%
%-----------------------------
%kālasya guṇāḥ kalanā kalmaṣaṃbhrāntiḥ prasādaḥ unmādaḥ/ \E
%kālasya guṇāḥ kalanā kalpaḥ bhrāṃtiḥ prasādaḥ unmādaḥ \P
%kālasyaguṇāḥ/ kalanākalpanābhrāṃtipramādaḥ unmādaḥ/ \B
%kālasyaguṇāḥ// kalanā kalpanā bhrāṃtiḥ pramādaḥ unmādaḥ... \L
%tadanaṃtaraṃ kālasya guṇāḥ// kalanā/ kalpanā/ bhrāṃtiḥ/ pramādaḥ/ unmādaḥ/ \N1
%tadanaṃtaraṃ kālasya guṇāḥ// .....ṛest is ommitted... \N2
%tadanaṃtaraṃ kāraṇasya guṇāḥ/ kalanā/ kalpanā/ bhrāṃtiḥ/ pramādaḥ/ unmādaḥ/ \D1
%tadanaṃtaraṃ kālaguṇāḥ kalanā kalpanā bhrāṃti pramādaḥ unmādaḥ \U1
%kālasya guṇāḥ// kalanā// kalpanā// bhrāṃtiḥ// pramādaḥ// unmādaś ceti// \U2
%-----------------------------
%
%-----------------------------
%jīvasya guṇāḥ  jāgradavasthā svapnāvasthā suṣuptāvasthā turīyāvasthā/ \E
%jīvasya guṇāḥ  jāgradavasthā svapnāvasthā suṣuptāvasthā turīyāvasthā \P %%%7673.jpg 
%jīvasya guṇāḥ/ jāgravadasthāḥ svapnāvasthā suṣuptaturyāvasthā/ \B
%jīvasya guṇāḥ// jāgradavasthā svapnāvasthā suṣupti turyāvasthā// \L
%jīvasya guṇāḥ/  jāgravadasthā/ svapnāvasthā/ suṣuptavasthā/ turīyāvasthā/ \N1
%                jāgradavadasthā/ svapnāvasthā/ suṣumṇāvasthā/ turīyāvasthā/ \N2
%jīvasya guṇā// jāgradavasthā/ svapnāvasthā/ suṣuptavasthā/ turīyāvayāvasthā/ \D1
%jīvasya guṇāḥ jāgṛdavasthā svapnāvasthā suṣuptāvasthā turyāvasthā \U1
%jīvasya guṇaḥ//  jāgradavasthā// svapnāvasthā// suṣuptāvasthā// turīyāvasthā// \U2
%-----------------------------
%
%-----------------------------
%turīyātītāvasthā tadanaṃtaram etādṛśam ekajñānam utpadyate/ \E
%turīyātītāvasthā tadanaṃtaram etādṛśam ekajñānam utpadyate  \P
%turiyā/ tītāvasthā/ tadanaṃtaram etādṛśam ekaṃ jñānam utpadyate/ \B
%turiyātītāvasthā// tad anaṃtaram etādṛśam ekaṃ jñānam utpadyate// \L 0036.jpg
%turiyātītāvasthā// tadanaṃtaram etādṛśam ekaṃ jñānam utpadyate/ \N1
%turiyātītāvasthā// tadanaṃtaram etādṛśam ekajñānam utpadyate// \N2
%turiyātītāvasthā// tadanaṃtaraṃ etādṛśam ekaṃ jñānam utpadyate/ \D1
%turiyātītavasthā kaivalyā tad anaṃtaram etādṛśaṃ ekaṃ jñānam utpadyate \U1
%turīyātītāvasthā// tad anaṃtaram etādṛśo mekaṃ jñānam utpadyate// \U2
%-----------------------------
%
%-----------------------------
%icchāyāḥ paṃca guṇāḥ/ unmanyavasthā/ vāṃchā cittaṃ veṣṭanam vibhramaḥ/ kriyāyāḥ paṃca guṇāḥ/ smaraṇaṃ udyamaḥ udvegaḥ/ kāryaniścayaḥ/ satkulācāratvam// \E
%icchākriyāmāyā prakṛtiḥ vācāḥ ichāyā paṃcaguṇāḥ unmayavāsanā vāṃcha vittaṃ ceṣṭa kriyāyāḥ paṃcaguṇāḥ/ smaraṇaṃ udyamaḥ udvegakāryaniścayaḥ satkulācāratvaṃ \P
%icchātriyāyā paṃcaguṇāḥ/ unmanyāvāsanā/ vāṃcha krirraṃccoṣṭhā/ kriyāyā paṃca guṇāḥ/ smaraṇaṃ udyamaḥ udvegakāryaniścayaḥ satkulācāratvaṃ/.... [DSCN7177.JPG, Z.1] \B
%icchākriyāmāyā prakṛtivācyaḥ// ichāyā paṃcaguṇāḥ// unmany avāsanā// vāṃcha cittaṃceṣṭa kriyāyā paṃca guṇāḥ/ smaraṇaṃ udyamaḥ udvegakāryaniścayaḥ satkulācāratvaṃ... \L
%icchāyāḥ māyā/ prakṛtiḥ/ vāca// icchayāḥ paṃcaguṇāḥ// unmany/ vāsanā/ vāṃcchā/ caittaṃ/ ceṣṭā// kriyāyāḥ paṃcaguṇāḥ/ smaraṇaṃ/ udyamaḥ/ udvegaḥ/ kārya/ niścayaḥ/ satkulācāratvam// \N1
%icchā/ kriyā/ māyā/ prakṛtiḥ/ vāca// icchayā paṃcaguṇāḥ// unmany/ vāsanā/ vāṃcchā/ caittaṃ/ ceṣṭā// kriyāyā paṃcaguṇāḥ/ smaraṇaṃ/ udyama/ udvega/ kāryaniścayaḥ satkulācāratvam// \N2
%icchā kriyā māyā/ prakṛtiḥ// vāca// icchayāḥ paṃcaguṇāḥ// unmany/ vāsana/ vāṃchā/ caita/ ceṣṭā// kriyāyāḥ paṃcaguṇāḥ/ smaraṇaṃ/ udyamaḥ udvegaḥ/ kāryaniścayaḥ/ satkulācāratvaṃ/ \D1
%ichākriyāmāyā prakṛtivāca ichāyāḥ paṃcaguṇāḥ unmany           \U1
%icchā// kriyā// māyā// prakṛtiḥ// bhāvaḥ// icchāyāḥ paṃcaguṇāḥ// unmanyaṃ vāsanāḥ// vāṃchā// cittaḥ// ceṣṭāḥ// kriyāyāḥ paṃca guṇaḥ// smaraṇaṃ// udyamaḥ// udvegaḥ// kāryaniścayaḥ// satkulācāratvaṃ// \U2
%-----------------------------
%
%-----------------------------
%māyāyāḥ paṃca guṇāḥ/ madamātsaryādayaḥ/ kīrtiḥ asatyabhāvāḥ/ \E [S.73]
%māyāyāṃ       guṇāḥ madaḥ mātsaryaṃ daṃbhaḥ kīrtiḥ asatyabhāvaḥ  \P
%māyāyāḥ paṃca guṇāḥ/ madaḥ mātsaryaraṃbhaḥ/ kīrtiḥ asatyabhāvaḥ/ \B
%māyāyā paṃca guṇāḥ madaḥ mātsaryaraṃbhaḥ kīrtiḥ asatyabhāvaḥ... \L
%māyāyā        guṇāḥ/ madaḥ/ mātsaryaḥ/ daṃbhaḥ/ kīrtiś ca// asatyabhāvaḥ// \N1
%māyāyā        guṇāḥ/ mada/ mātsarya/ daṃbha/ kīrtiś ca// asatyabhāvaḥ// \N2
%māyāyā        guṇā// madaḥ/ mātsaryaḥ/ daṃbhaḥ/ kīrtiś ca/ asatyabhāvaḥ/ \D1
%\om \U1
%māyāyāḥ guṇāḥ// madaḥ// mātsaryaṃ// daṃbhaḥ// kīrttiḥ// asatyabhāvaḥ// \U2
%-----------------------------
%
%-----------------------------
%prakṛteḥ paṃca guṇāḥ āśā tṛṣṇā spṛhā kāṃkṣā mithyātvam/ \E
%prakṛter guṇāḥ āśā tṛṣṇā spṛhā bhikṣā mithyātvaṃ \P
%prakṛte guṇāḥ/ āśā tṛṣṇā spṛhā kāṃkṣā mithyātvaṃ/ \B
%prakṛte guṇāḥ/ āśā tṛṣṇā spṛhā kāṃkṣā mithyātvaṃ// \L
%prakṛte guṇāḥ/ āśā/ tṛṣṇā/ spṛhā/ kāṃkṣā/ mithyātvaṃ// \N1
%prakṛte guṇāḥ// āśā/ tṛṣṇā/ spṛhā/ kāṃkṣā/ mithyātvaṃ// \N2
%prakṛte guṇāḥ/ āśā tṛṣṇā// spṛhā/ kākṣā mithyātvaṃ// \D1
%\om \U1
%prakṛter guṇāḥ// āśā// tṛṣṇā// spṛhā// kāṃkṣā mithyātvaṃ// \U2
%-----------------------------
%
%-----------------------------
%vācāyāpaṃcaguṇāḥ/ parā paśyantī madhyamā vaikharī/ \E
%vācāyā guṇāḥ parā paśyaṃtī madhyamā vaikharī \P
%vācāyāpaṃcaguṇāḥ/ parā paśyanti madhyamā vaikharī \B
%vācāyā paṃcaguṇāḥ/ parā paśyanti madhyamā vaikharī \L
%vācāyā guṇāḥ/ parā/ paśyanti, madhyamā, vaikharī/ \N1 %usage of comma!
%vācāyā guṇāḥ/ parā/ paśyanti/ madhyamā/ vaiṣarī/ \N2
%vācā guṇāḥ// parā/ paśyantī madhyamā/ vaiṣarī/ \D1
%\om vaikharī \U1
%vācaḥ  paṃcaguṇaḥ// parā// paśyaṃti// madhyamā// vaikharī// \U2
%-----------------------------
%
%-----------------------------
%mātṛkā tadanaṃtarametādṛśaṃ jñānam utpadyate/ karmakāraḥ/ candraḥ/ sūryaḥ/ agniḥ eta tpaṃcakaṃ pratyakṣaṃ karttavyam tatra karmaṇaḥ paṃcaguṇāḥ                                                 kāmasya guṇāḥ ratiḥ prītiḥ/ krīḍā kāmanā anustutā// \E
%mātṛkā tadanaṃtaraṃ etādṛśaṃ jñānam utpadyate  karmakāmacandrasūryaḥ āgniḥ etat paṃcakaṃ pratyakṣaṃ karttavyaṃ tatra karmaṇā paṃcaguṇāḥ śubhaṃ yaśaḥ apakīrttiḥ iṣṭaphalasādhānaṃ kāmasya guṇāḥ ratiḥ prītiḥ krīḍā kāmanāḥ  anurattutā \P
%mātṛkā tadanaṃtaraṃ metādṛśaṃ jñānam utpadyate/ karmakāmacandrasūryaḥ āgniḥ/ etat paṃcakaṃ pratyakṣaṃ karttavyaṃ/ tatra karmaṇā paṃcaguṇāḥ/ śubhaṃ aśubhaṃ yaśaḥ apakīrtiḥ/ iṣṭaphalasādhānaṃ/ kāmasya guṇāḥ ratiḥ prītiḥ krīḍā kāminā anustuttā// \B
%mātṛkā tad anaṃtaram etādṛśaṃ jñānam utpadyate// karmakāmacandrasūryaḥ āgniḥ// enat paṃcakaṃ pratyakṣaṃ karttavyaṃ/ tatra karmaṇāṃ paṃcaguṇāḥ śubhaṃ aśubhaṃ yaśaḥ apakīrtiḥ iṣṭaphalasādhānaṃ kāmasya guṇāḥ ratiḥ prītiḥ krīḍā kāminy anuraktatā...\L
%mātṛkā tad anaṃtaraṃ etādṛśaṃ jñānam tpādyate// karmma, kāmaḥ/ candraḥ/ sūryaḥ/ āgniḥ/ etat paṃcakaṃ pratyakṣaṃ karttavyaṃ/ tatra karmaṇaḥ paṃcaguṇāḥ// śubhaṃ/ aśubhaṃ/ yaśaḥ/ apakīrttiḥ/ iṣṭaphalasādhanaṃ// kāmasya guṇāḥ/ ratiḥ prīti, krīḍā/ kāmanā/ anuratā// \N1
%mātṛkā tad anaṃtaraṃ etādṛśaṃ jñānam utpādyate// karmakāmacandrasūrya agni etat paṃcakaṃ pratyakṣaṃ karttavyaṃ// tatra karmaṇa paṃcaguṇāḥ// śubhaṃ/ aśubhaṃ/ yasa/ apakīrtti/ iṣṭaphalasādhanaṃ// kāmasya guṇāḥ// rati prīti krīḍā/ kāmanā/ anurajā// \N2
%mātṛkā/ tad anaṃtaraṃ etādṛśaṃ jñānam utpādyate/ karmma kāmaḥ/ candraḥ/ sūryaḥ/ āgniḥ/ etat paṃcakaṃ pratyakṣaṃ karttavyaṃ/ tatra karmaṇaḥ paṃcaguṇāḥ// śubha/ aśubhaṃ yaśaḥ/ apakīrttiḥ/ iṣṭaphalasādhanaṃ/ kāmasya guṇāḥ/ ratiḥ prīti krīḍā/ kāmanā/ anuratā/ \D1 %%%p.17 recto 
%mātṛkā tad anaṃtaraṃ etādṛśaṃ jñānam utpadyate karmakāmacaṃdrasūrya agnī etat paṃcakaṃ pratyakṣaṃ karttavyaṃ          tatrakarnaṇaḥ paṃcaguṇāḥ śubha aśubha yaśaḥ āvakīrtiḥ iṣṭaphalasādhanaṃ kāmasya guṇāḥ rati prīti kriḍā kāmanā ānuratā \U1
%mātṛkāḥ// tad anaṃtaram etādṛśaṃ jñānam utpadyate// karmaḥ// kāmaḥ// candraḥ// sūryaḥ// agniḥ// etat paṃcakaṃ pratyakṣaṃ karttavyaṃ// tatra karmaṇaḥ paṃcaguṇāḥ// śubhaṃ// aśubhaṃ// yaśaḥ// apakīrtiḥ// iṣṭaphalasādhānaṃ// kāmaḥsya guṇāḥ// ratiḥ// prītiḥ// krīḍā// kāmanā// anuraktā// \U2
%-----------------------------
%
%----------------------------
%idānīṃ caṃdrasya ṣoḍaśakalāḥ kathyante/ \E [p.74]
%idānīṃ caṃdrasya ṣoḍaśakalāḥ kathyaṃte \P %%%7674.jpg 
%idānīṃ caṃdrasya ṣoḍaśakalāḥ kathyate/ \B
%idānīṃ caṃdrasya ṣoḍaśa      kathyate... \L
%idānīṃ caṃdrasya śodaśakalāḥ kathyaṃte/ \N1
%idānīṃ caṃdrasya ṣodaśakalāḥ kathyaṃte/ \N2
%idānīṃ caṃdrasya śodaśakalāḥ kathyaṃte/ \D1
%idānīṃ caṃdrasya śodaśakalāḥ kathyaṃte \U1
%idānīṃ caṃdrasya saptadaśakalā vartaṃte// tasyānāmāni// ṣoḍaśakalā kathyaṃte// \U2
%-----------------------------
%
%----------------------------
%dallolā kallolinī uścalinī unmādinī taraṃgiṇī poṣayaṃtī laṃpaṭā laharī lolā lelihānā prasarantī pravṛttiḥ plavantī pravāhā saumyā prasannā// \E
%hallolā kallolinī uścalinī unmādinī taraṃgiṇī poṣayaṃtī laṃpaṭā laharī lolā lelihānā prasaraṃtī pravṛttiḥ sravaṃtī pravāhā saumyā prasannā   \P
%dullālā kallolinī ucaṃlini unmādinī taraṃgiṇī poṣāyaṃtī lapaṃṭāḥ laharī lolā lelihā/ prasarantī/ pravṛttī sravaṃtī mavāhā   somyā prasannā// \B
%hullātvā kallolini uchaṃlini unmādanī taraṃgiṇī poṣāyaṃtī lapaṭāḥ lahari lolā lelihā prasaraṃtī// prakṛtī sravaṃtī mavāhā   somyā prasannā... \L
%hallolā/ kallolinī/ unmādinī/ taraṃgiṇī/ poṣayanti, lapaḍā laharī/ lolā/ lelihānā/ prasaraṃtī/ pravṛttiḥ/ sravaṃtī/ pravāhā/ saumyā, prasannā,   \N1
%hallolā/ kalloli/ unmādinī/ taraṃgiṇī/ poṣayaṃti/ lapaḍā laharī/ lolā/ lelihānā/ prasaraṃtī/ pravṛttiḥ sravaṃtī pravāhā// saumyā prasannā   \N2
%hallolā/ kallolinī/ unmādinī/ taraṃgiṇī/ poṣayanti lapaḍā laharī/ lolā/ lelihānā/ prasaraṃtī/ pravṛttiḥ sravaṃtī/ pravāhā/ saumyā prasannā   \D1
%haṣṇo lākaṣṇoleni uchalanī unmādani taraṃgiṇī poṣayanī laṃpaṭā laharī lolā leliha nā prasaraṃti pravṛtiḥ sravaṃtī pravaṃtī śvāḥ saumya prasannā \U1
%hallolāḥ// kallolinī// ucchṛlinī// unmādinī// taraṃgiṇī// poṣayati// laṃpaṭā// laharī// lolāḥ// lelihānāḥ// prasaraṃti// pravṛttiḥ// sravaṃti// pravāhāḥ// saumyāḥ// prasannāḥ//   \U2
%----------------------------
%
%----------------------------
%candrasya saptadaśamī kalā varttate tasyā nāma nivṛttisametā kalā kathyate/ \E
%candrasya saptadaśī kalā varttate   tasya nāma nivṛttisametā kalā kathyate \P
%candrasya saptadaśamī kalā varttate tasyā nāma nivṛttisametā kalā kathyate/ \B
%candrasya saptadaśī kalā varttate// tasyā nāma nivṛttisametakalā kathyate/ \L
%caṃdrasya saptadaśī kalā varttate   tasyā nāma naivṛttiḥ sāmṛtakalā kathyate// \N1
%caṃdrasya saptadaśī kalā varttate// tasyā nāma naivṛttiḥ sāmṛtakalā kathyate// \N2
%caṃdrasya saptadaśī kalā varttate   tasyā nāma naivṛttaiḥ sāmṛtakalā kathyate/ \D1
%candrasya saptadaśī kā varttate     tasyā nāma nivṛttiḥ   sāmṛta kathyate \U1
%caṃdrasya saptadṛśī kalā vartate//  tasyāḥ nāmāni// vṛttiḥ sametaḥ// kalāḥ// kathyaṃte// \U2 %%%428.jpg 
%-----------------------------
%
%-----------------------------
%idānīṃ sūryasya kalāḥ kathyante/        \E
%idānīṃ sūryasya dvādaśakalāḥ kathyaṃte  \P
%idānīṃ sūryasya dvādaśakalā kathyate/   \B
%idānīṃ sūryasya dvādaśakalā kathyate//  \L
%idānīṃ sūryasya dvādaśakalā kathyaṃte/  \N1
%idānīṃ sūryasya dvādaśakalā kathyate/   \N2
%idānīṃ sūryasya dvādaśakalā kathyaṃte/  \D1
%idānīṃ sūryasya dvādaśakalā kathyaṃte   \U1
%idānīṃ sūryasya dvādaśakalāḥ kathyaṃte// \U2
%-----------------------------
%
%-----------------------------
%tapanī grāsakā ugrā akocanī śoṣaṇī prabodhinī ghasmarā ākarṣiṇī tuṣṭivarddhinī kūrmī reṣā kiraṇavatī prabhavati sūryasya trayodaśī kalā vidyate/  \E %[p.75]
%tāpanī grāsaka ugra ākocanī śoṣiṇī prabodhinī ghasmarā ākarṣayaṃtī tuṣṭivarddhinī kurmmī rekhā kiraṇāvatī prabhūtavatī sūryasya trayodaśī kalā vidyate \P
%tāpani grāsaka ugrā ākocanī/ śoṣaṇī prabodhanī ghasmarā/ ākarṣayaṃtī/ tuṣṭivardhanī/ ūrmmi rekhā kīrṇāvatī prabhavati/ sūryasya trayodaśī kalā vidyate/ \B
%tāpani grāsaka ugrā ākocanī śoṣaṇī prabodhanī ghasmarā// ākarṣayaṃtī tuṣṭivardhanī ūrmmi rekhā kīrṇāvitī prabhutavati// sūryasya trayodaśi kalā vidyate// \L
%tapanī/ grāsakā/ ugrā/ ākuṃcanī/ śoṣaṇī/ prabodhinī/ ghasmarā/ ākarṣayaṃtī/ tuṣṭi, varddhanī/ ūrmmi reṣā// kiraṇāvatī/ prabhutavatī/ sūryasya trayodaśī kalā vidyate/ \N1 %%%S.13
%tapanī/ grāsakā/ ugrā/ ākuṃcanī/ śoṣaṇī/ prabodhinī/ ghasmara/ ākarṣayaṃtī/ tuṣṭi varddhanī/ ūrmmi reṣā// kiraṇāvatī/ prabhutavatī// sūryasya trayodaśī kalā vidyate// \N2
%tapanī/ grāsakā/ ugrā/ ākuṃcanī/ śoṣaṇī/ prabodhinī/ ghasmarā/ ākarṣayaṃtī/ tuṣṭi varddhanī ūrmmi reṣā/ kiraṇāvatī/ prabhutavatī/ sūryasya trayodaśī kalā vidyate/ \D1
%tapani grāsakā ugrā ākuṃcanī śoṣaṇī prabodhinī ghasmarā ākarṣayaṃti tuṣṭivardhanī ūrmi rekhā kīrṇavatī prabhutavatī sūryasya trayodaśī kalā vidyate \U1
%tapanī// grāsakā// ugrā// akocanī// śoṣaṇī// prabodhinī// ghasmarā// ākarṣayatī// tuṣṭiḥ varddhanī// ūrmī// rekhā kiraṇavatī// prabhūtavatī// sūryasya trayodaśī kalā vidyate//  \U2
%-----------------------------
%
%-----------------------------
%tasya nāma nijakalāsvaprakāśā ca// \E
%tasya nāma nijakalāsvaprakāśā ca  \P
%tasya nāmaḥ nijakalāsvaprakāśā ca/ \B
%tasya nāma nijakalāsvaprakāśā ca... \L
%tasya saṃjñā nijakalāṃ svaprakāśā ca// \N1
%tasya saṃjñā nijakalāṃ svaprakāśā ca// \N2
%tasyāḥ saṃjñā nijakalāṃ svaprakāśā ca/ \D1
%tasyāḥ saṃjñā nijakalāsvaprakāśā ca \U1
%tasyā nāmāni nijakalā// svaprakāśā ca// \U2
%-----------------------------
%
%-----------------------------
%idānīm agnisaṃbaṃdhinyo daśa kalāḥ kathyante/ \E
%idānīm agnisaṃbaṃdhinyo daśakalāḥ kathyaṃte/ \P DSCN7675.jpg
%idānīm agnīsaṃbaṃdhini daśakalā kathyaṃte/ \B
%idānīm agnīsaṃbaṃdhini daśakalā kathyaṃte// \L
%idānīm agnīsaṃbaṃdhinī  daśakalāḥ kathyaṃte/ \N1
%idānīm agnīsaṃbaṃdhinī  daśakalā kathyaṃte/ \N2
%idānīm agnīsaṃbaṃdhinī  daśakalāḥ kathyaṃte/ \D1
%idānīm agnīsaṃbaṃdhinīṃ dvādaśakalā kathyaṃte \U1
%idānīṃ agnisaṃbaṃdhinī  daśakalāḥ// kathyaṃte// \U2
%-----------------------------
%
%-----------------------------
%dīpikā jvālā visphuliṃginī pracaṃḍā  pācikā raudrī dāhikā rāvaṇī/ śikhāvatī/ agner ekādaśī nijakalā jyotiḥ saṃjñā varttate// \E
%dīpikā jvālā visphuliṃginī pracaṃḍā  pāvakā  raudrī dāhakā rāvaṇī śikhāvatī agner ekādaśī nijakalā jyotiḥ saṃjñā varttate   \P
%dīpikā jvālā visphuliṃginī pracaṃḍā/ pāvakārau raudrī dāhaka rāvaṇī śikhāvatī/ agne ekādaśi nijakalā jyotiḥ/ saṃjñā vartate// \B
%dīpikā jvālā visphuliṃginī pracaṃḍā// pāvakārau drīdāhaka rāvaṇī śikhāvatī agne ekādaśi nijakalā jyotiḥ// saṃjñā vartate// \L
%dīpikā/ jārakā/ jvālā/ visphuliṃginī/ pracaṃḍā/ pācakā/ raudrīdāhakā/ rāvaṇi/ śikhāvatī/ a(jña?!)gner ekādaśi nijakalā jyoti saṃjñakā// \N1
%dīpikā/ jārakā/ jvālā/ visphuliṃginī/ prcaṇḍā/ pācakā/ raudrī/ dāhakā/ rāvaṇi/ śikhāvatī(/ agner ekādaśi nijakalā jyoti saṃjñakā// \N2
%dīpikā/ jārakā/ jvālā/ visphuliṃginī/ pracaṃḍā/ pācakā raudrīdāhakā/ rāvaṇi/ śikhāvatī/ ajñerekādaśī nijakalā jyoti saṃjñakā/ \D1
%dīpikār jakā jvālāviḥ ...................  visphuliṃginī pracaṃḍā// pāvakārau drīdāhaka rāvaṇī śikhāvatī agne ekādaśi nijakalā jyotiḥ// saṃjñā vartate// \U1 300.jpg
%dīpikā// jvālā// visphuliṃginī// pracaṃḍā//  pāvakā// raudrī// dāhakā// rāvaṇī// śikhāvatī// agner ekādaśī nijakalā// jyotiḥ// saṃjñā varttate// \U2
%-----------------------------
%
%----------------------------
%idānīṃ yogasya māhātmyaṃ kathyate/ \E
%idānīṃ yogasya māhātmyaṃ kathyate \P
%idānī  yogasya māhātmaṃ kathyate/ \B
%idānīṃ yogasya māhātmaṃ kathyate// \L
%idānīṃ yogasya māhātmyaṃ kathyate// \N1
%idānīṃ yogasya māhātmyaṃ kathyate/ \D1
%idānīṃ yasya a māhātmyaṃ kathyate \U1
%idānīṃ yogasya māhātmyaṃ kathyaṃte// \U2
%-----------------------------
%Now the magnanimity of yoga is taught. 
%-----------------------------
%\app{\lem[wit={ceteri}]{idānīṃ}
%  \rdg[wit={B}]{idānī}}
%\app{\lem[wit={ceteri}]{yogasya}
%  \rdg[wit={U1}]{yasya a}}
%\app{\lem[wit={ceteri}]{māhātmyaṃ}
%  \rdg[wit={L,B}]{māhātmaṃ}}
%\app{\lem[wit={ceteri}]{kathyate}
%  \rdg[wit={U2}]{kathyaṃte}}\dd{}
%-----------------------------
%guror anugrahāt   śāstrasya paṭhanāt   ācārakaraṇāt  vedāṃtarahasya śravaṇāt   dhyānakaraṇāt                 upavāsakaraṇāt   caturaśītyāsane sādhanāt   vairāgyasyotpatteḥ nairāśye karaṇāt ...      \E [P.76]
%guror anugrahāt   śāstrasya paṭhanāt   ācārakaraṇāt  vedāṃtarahasya śravaṇāt                                                  caturaśītyāsanasādhanāt    vairāgyasyotpattaḥ nairāśya karaṇāt          \P
%guru  anugrahāt/  śāstrasya paṭhanāt/  ācārakaraṇāt/ vedāṃtarahasya śravaṇāt/  dhyānakaraṇāt/                upavāsakaraṇāt/  caturāśītyāsana sādhanāt/  vairāgyasyotpatte/ nairāśa karaṇāt/ ... \B
%guru  agrahāt     śāstrasya paṭhanāt   ācārakaraṇāt  vedāṃtarahasya śravaṇāt   dhyānakaraṇāt                 upavāsakaraṇāt   caturāśītyāsanasādhanāt    vairāgyasyotpatteḥ nairāśya karaṇāt ... \L %%%0038.jpg
%guror anugrahāt/  śāstrasya paṭhanāt/  ācārakaraṇāt/ vedāntarahasya śravaṇāt/  dhyānakaraṇāt/ layasādhanāt/  upavāsakaraṇāt/  caturaśīti āsanasādhanāt/  vairāgyotpatteḥ/     vairāgyakaraṇāt//  \N1
%guror anugrahāt/  śāstrasya paṭhanāt/  ācārakaraṇāt/ vedāṃtarahasya śravaṇāt/  dhyānakaraṇāt  layasādhanāt/  upavāsakaraṇāt/  caturaśīti āsanasādhanāt/  vairāgyotpatteḥ/     vairāgyakaraṇāt/  \D1
%guror anugrahāt   śāstrasya paṭhanāt   ācārakaraṇāt  vedāṃtarahasya śravaṇāt   dhyānakaraṇāt  layasādhanāt   upavāsakaraṇāt   caturaśīti āsanasādhanāt   vairāgyotpatte       vairāgyakaraṇāt  \U1
%guror anugrahāt// śāstrasya paṭhanāt// ācārakathanāt vedāṃtarahasya śravaṇāt// dhyānakaraṇāt//               upavāsakaraṇāt// caturāśītyāsana sādhanāt// vairāgyasyotpatteḥ// vairāgyakaraṇāt//      \U2
%-----------------------------
%Because of favour of the teacher, because of studying the teaching, because of execution of good conduct, because of hearing the secret of Vedānta, because of execution of meditation, because of practicing dissolution, because of the execution of fasting, because of practising 84 āsanas, because of the generation of equanimity, because of executing equanimity, 
%-----------------------------
%haṭhayogasya karaṇāt   iḍāpiṃgalayoḥ  pavanadhāraṇāt          mahāmudrādidaśamudrāsādhanāt                 maunakaraṇāt   vanavāsāt  bahutarakleśakaraṇāt     bahukālayaṃtramaṃtrādisādhanāt       tapaḥ karaṇāt... \E
%haṭhayogasya karaṇāt   iḍāpiṃgalayoḥ  pavanadhāraṇāt          mahāmudrādidaśamudrāsādhanāt                 maunakaraṇāt   vanavāsāt  bahutarakleśakaraṇāt     bahutarakālaṃyaṃtramaṃtrādisādhanāt  tapaḥ karaṇāt... \P
%haṭayogasya  karaṇāt/                                         mahāmudrādi daśamudrāsādhanāt/               maunakaraṇāt/  vanavāsāt/ bahutarakleśakaraṇāt/    bahukālayaṃtramaṃtrādisādhanāt/      tapakaraṇāt/... \B
%haṭayogasya  karaṇāt// iḍāpiṃgalayoḥ  pāvanāpāvadhyānakaraṇāt mahāmudrādi daśamudrāsādhanāt                maunakaraṇāt   vanavāsāt  bahutarakleśakaraṇāt     bahutarakālamaṃtrayaṃtrādisādhanāt   tapakaraṇāt... \L
%haṭhayogakaraṇāt/      iḍāpiṃgalayoḥ  pāvanādhāraṇāt/         mahāmudrādi daśamudrāsādhanāt//              maunakaraṇāt/  vane vāsāt,bahutararakleśakaraṇāt// bahutarakālaṃ yaṃtrayaṃtrādisādhanāt tapakaraṇāt/ \N1 %%note important for Krakow!list!
%haṭhayogakaraṇāt/      iḍāpiṃgalayoḥ   pāvanādhāraṇāt/        mahāmudrādi daśamudrādi daśamūdrasādhanāt//  maunakaraṇāt   vane vāsāt bahutararakleśakaraṇāt/  bahutarakālaṃ yaṃtrayaṃtrādisādhanāt tapakaraṇāt/ \D1 %%%p.17 verso
%haṭayogasya  karaṇāt   iḍāpiṃgalayāḥ   pavanadharaṇāt         mahāmudrāsādhanāt                            maunakaraṇāt   vanevāsāt  bahutarakleśakaraṇāt     bahutarakālamaṃtrayaṃtrādisādhanāt   tapakaraṇāt \U1
%haṭhayogasya karaṇāt// iḍāpiṃgalayoḥ// pavanādhānākaraṇāt//   mahāmudrādidaśamudrāsādhanāt//               maunakaraṇāt// vanavāsāt//bahutarakleśakaraṇāt//   bahutarakālayaṃtramaṃtrādisādhanāt// tapaḥ karaṇāt// \U2
%-----------------------------
%because of doing haṭhayoga, because of holding the breath of the Iḍā- and Piṅgalā-channels, because of practicing the ten seals [like] the great-seal etc., because of [the observation of] silence, because of dwelling in the forest, because of the execution of many defilements?!, because of practicing Mantra and Yantra for a long time, because of austerities,  
%-----------------------------
%bahutarārpaṇadānāt                                     āśramācārapālanāt   saṃnyāsagrahaṇāt  ṣaḍdarśanagrahaṇāt   śiromuṃḍanāt   anyopāyakaraṇāt   yogatattvaṃ na  prāpyate// \E
%bahutarakleśakaraṇāt bahutarakaraṇāt bahutatārthadānāt āśramācārapālanāt   saṃnyāsagrahaṇāt  ṣaṭdarśanagrahaṇāt                                    yogatatvaṃ  na  prāpyate \P %%%7676.jpg
%bahutarārthādānāt/                                     āśramācārapālanāt/  sanyāsagrahaṇāt/  ṣaḍdarśanagrahaṇāt/  śiromuṃḍanāt/  anyopāyakaraṇāt&/ yogatattvaṃ nna prāpyate/ \B
%bahutarārthādānāt                                      āśramācārapālanāt   sanyāsagrahaṇāt   ṣaḍdarśanagrahaṇāt   śiromuṃḍanāt   anyopāyakaraṇāt   yogatattvaṃ nna prāpyate \L
%bahutarārthadānāt/ tīrthasevo?????!!karaṇāt/           āśramācārapālanāt/  saṃnyāsagrahaṇāt/ ṣaṭdarśanagrahaṇāt/  siromuṃḍanāt// anyopāyakaraṇāt/  yogatatvaṃ  na  prāpyate/ \N1
%bahutarārthadānāt// tīrthasevākaraṇāt/                 āśramācārapālanāt/  saṃnyāsagrahaṇāt/ ṣaṭdarśanagrahaṇāt   siromuṃḍanāt/  anyopāyakaraṇāt/  yogatatvaṃ  na  prāpyate/ \D1
%bahuttarārthadānāt niyamakaraṇāt                       āśramācyārapālanāt  sanyāsagrahaṇāt   ṣaḍdarśanagrahaṇāt   siromuṃḍanāt   anyopāyakaraṇāt   yogatatvaṃ  na  prāpyate \U1
%bahutarārthadānāt//                                    āśramācārapālanāt// sanyāsagrahaṇāt// ṣaṭdarśanagrahaṇāt// siromuṃḍanāt// anyopāyakaraṇāt// yogatatvaṃ  na  prāpyate// \U2 %%%429.jpg 
%-----------------------------
%because of giving up a lot of possession, because of frequenting places of pilgrimage, because of protection of the habit of the stages of life, because of undertaking renunciation, because of grasping the six philosoiphies,  becasue of shaving the head, because of the execution of other means, the reality of yoga is not attained. 
%-----------------------------
%[p.76]
%sa tu yogaḥ gurusevayā prāpyate/ \E
%\om                              \P
%sa tu yogo  gurusevayā prāpyate/ \B
%sa tu yogo  gurusevayā prāpyate/ \L
%sa tu yogo  gurusevayā prāpyate śrī// \N1
%sa tu yogo  gurusevayā prāpyate/ \D1
%sa tu yogo  gurusevayā prāpyate \U1
%sa tu yogo  gurusevayā prāpyate// \U2
%-----------------------------
%The [reality of] yoga is truly attained by frequenting the teacher. 
%-----------------------------
%gurukṛpātaḥ pātrāṇāṃ    dṛḍhānāṃ  satyavādinām/  kathanād dṛṣṭipātād vā    sāṃnidhyād avalokanāt/ \E
%gurudṛk pātapātrāṇāṃ    dṛḍhānāṃ  satyavādinām   kathanāt dṛṣṭipātād vā    sāṃnidhyād avalokanāt \P
%gurudrak/ pāt/ patrāṇāṃ dṛḍhānāṃ  satyavādinām/  kathanād viṣapātād vā     sānidhyāt dya? vatrokanāt//1//  \B
%gurudrak pāt patrāṇāṃ             satyavādinām// kathanād viṣapānād vā     sānnitdhyavalokanāt//1//  \L
%gurudṛkpātapātrāṇāṃ     dṛḍhānāṃ  satyavādinām/  kathanād dviṣapātād vā,   sānidhyāddhyavalokanāt//1// (recheck S.13 verso unten letzte Zeile) //1//  \N1
%gurudṛkpātapātrāṇo      dṛḍhānāṃ  satyavādināṃ/  kathanād dṛṣṭipātād vā    sānidhyād dyavalokanāt//1// \D1
%gurudak pātrāṇāṃ        dṛḍhānāṃ  satyavāridinām    ?upayādṛṣtipātād vā    sānidhyāty avalokanāt   \U1
%gurudrak pātāpā bhāṇāṃ  dṛḍhānāṃ? satyavādinām   kathanā dṛṣṭipātād vā     sāṃniddhyāvalokanāt//  \U2
%-----------------------------
%Of the firm and truthful vessels of the range of site (em. zu dṛkpātha) of the teacher, because of narration or because of casting of a glance, because of presence (sāṃnidhyāt), because of looking at 
%-----------------------------
%sadguruprasādāt samyak paramaṃ padaṃ pāpyate/ ata evaṃ vacaḥ proktaṃ na guroradhikaṃ param//1//      \E
%prasādāsya guruoḥ samyak prāpyate paramaṃ padaṃ ata eva vacaḥ proktaṃ na guror adhikaṃ paraṃ         \P
%prasātsadguroḥ saṃyak prāpyate paramaṃ padaṃ/ ara eva vacaḥ proktaṃ na guroradhikaṃ paraṃ 2          \B
%prasādāt sadguroḥ saṃyak prāpyate paramaṃ padaṃ// ata eva vacaḥ proktaṃ na guror adhikaṃ paraṃ//2//  \L
%prasādāt sadguroḥ saṃyak prāpyate paramaṃ padaṃ/ ata eva vacaḥ proktaṃ na guror adhikaṃ paraṃ//2//   \N1 S.13 verso
%prasādāt sadguroḥ saṃyak prāpyate paramaṃ padaṃ// ata eva vacaḥ proktaṃ na guror adhikaṃ paraṃ//2    \D1
%prasādāt sadguroḥ saṃyak prāpyate paramaṃ padaṃ// ata eva vacaḥ proktaṃ na guror adhikaṃ paraṃ//     \U1
%                                                                                                     \U2 CHECK!!!!!!!!!!!!!!!!!
%-----------------------------
%
%-----------------------------
%vāṅmātrādbodhadṛk pātādyaḥ karoti śamaṃ kṣaṇāt/  prasphuṭad bhrāṃtihṛttoṣaṃ svacchaṃ vaṃde guruṃ param// 2// \E
%vāṅmātrādbāthadṛk pātādyaḥ karoti śamaṃ kṣaṇāt  prasphuṭad bhrāṃtihṛttoṣaṃ svacchaṃ vaṃde guruṃ param      \P
%vāṅmātrādbathadṛk pītādyaḥ karoti śamaṃ kṣaṇāt/  prasphaṭad bhātihatoṣaṃ svachaṃ vaṃde guruṃ param// 3// \B
%vāṅmātrādbathadṛk pātādyaḥ karoti śamaṃ kṣaṇāt// prasphaṭad bhāti hatoṣaṃ svachaṃ vaṃde guruṃ param// 3// \L
%vāṅmātrādbathadṛka pātādyaḥ karoti śamaṃ kṣaṇāt// prasphaṭat bhrāṃti hatddo???ṣaṃ svacchaṃ vade karaṃ parī [<-oder->] parāṃ// 3// \N1
%vāṅmātrād bathadṛk  pātādyaḥ karoti śamaṃ kṣaṇāt/ prasphuṭat bhrāṃti hṛddoṣaṃ??? svachaṃ vedakakaraṃ paraṃ// 3/ \D1
%\om \U1
%vāṅmātrādbathadṛk pātādyaḥ karoti samaṃ kṣaṇāt//  prasphuṭad bhrāṃti ittoṣaṃ? svachaṃ vaṃde guruṃ paraṃ//\U2
%-----------------------------
%
%-----------------------------
%samyag ānandajananaḥ  sadguruḥ sobhidhīyate/  nimeṣārddhaṃ vā tatpādaṃ yadvākyād avalokanāt// 3// \E[p.78]
%samyag ānandajananaḥ  sadguruḥ sobhidhīyate   nimiṣārddhaṃ vā tatpādaṃ yadvākyād avalokanāt       \P
%samyag ānandajananaḥ/ sadguruḥ sobhidhīyate/  nimeṣārddhā vā tatpāda yadvākyād avalokanāt// 3//  \B
%samyag ānandajananaḥ  sadguruḥ sobhidhīyate// nimeṣārddhā vā tatpāda yadvākyād avalokanāt// 3// \L %%%%0039.jpg
%samyag ānandajananaṃ  sadguruḥ sobhidhīyate/  nimeṣārddhaṃ ca pādaṃ vāyad vākyād avalokanāt/ \N1
%samyag ānandajananaṃ  sadguruḥ sobhidhīyate/  nimeṣārddhaṃ ca pādaṃ vāyad vākyād avalokanāt/ \D1
%\om \U1
%samyag ānandajananaḥ sadguruḥ sobhidhīyate//  nimeṣārddhaṃ vā tatpādaṃ yad vākyād avalokanāt// \U2
%-----------------------------
%
%-----------------------------
%svātmā  sthiratvam āyāti   tasmai śrīgurave  namaḥ/  nānāviplavaviśrāntiḥ   kathanāt  kuru te tataḥ// 4// \E
%svātmā/ sthiratvam āyāti   tasmai śrīgurave  namaḥ   nānāvikalpaḥ viśrāṃtiḥ kathanāt  kuru te tuyaḥ      \P einziges daṇḍa im Text! 
%svātmā  sthiraṃ tvam āyāti tasmai śrīgurave  namaḥ/  nānāvikalpa  viśrāṃti  kathanāt/ kuru te tuyaḥ/   \B
%svātmā  sthiratvam āyāti   tasmai śrīgurubho namaḥ// nānāvikalpa  viśrāṃti  kathanāt  kuru te tuyaḥ//   \L
%svātmā  sthiratvam āyāti   tasmai śrīgurave  namaḥ/  nānāvikalpa  viśrāṃtiṃ kathanāt  kuru te tu saḥ/   \N1
%svātmā  sthiratvam āyāti   tasmai śrīgurave  namaḥ// nānāvikalpaṃ viśrāṃtiṃ kathanāt  kuru te tu saḥ/   \D1
%\om \U1
%svātmā  sthiratvam āyāti tasmai śrīguru namo namaḥ// nānāvikalpaviśrāntiṃ   kathanāt  kuru te tuyaḥ// \U2
%-----------------------------
%
%-----------------------------
%sadguruḥ sa tu vijñeyo na tu vai priyajalpakaḥ// 5// \E
%sadguruḥ sa tu vijñeyo na tu vipriyajalpakaḥ      5 \P
%sadguruḥ sa tu vijño nnu     viprāyajalākaḥ// 5//        \B
%sadguruḥ sa tu vijño nnu     viprāyajalākaḥ// 5//        \L
%sadguruḥ sa tu vijñeyo na tu vipriyajalpakaḥ//       \N1
%sadguruḥ sa tu vijñeyo na tu vipriyajalpakaḥ/       \D1
%\om \U1
%sadguruḥ sa tu vijñeyo na tu vipriyajalpakaḥ// \U2
%-----------------------------
%
%-----------------------------
%ata eva paramapadasya prāptyarthaṃ sadguruḥ sevyaḥ sarvadā yaḥ puruṣaḥ satyavādī bhavati/ \E
%ata eva paramapadasya prāptyarthaṃ sadguruḥ sevyaḥ sarvadā yaḥ puruṣaḥ satyavādī bhavati  \P %%%7677.jpg
%ata eva paramapada    prāptyarthaṃ sadguruḥ sevyasarvadā/  yaḥ puruṣaḥ satyavādī bhavati/ \B
%ata eva paramapadaprāptyarthaṃ sadguruḥ sevyasarvadā// yaḥ puruṣaḥ satyavādī bhavatī// \L
%ata eva paramapadaprāptyarthaṃ sadguruḥ sarvadā vaṃdyaḥ?/ yaḥ puruṣaḥ satyavādī bhavati/ \N1
%ata eva paramapadaprāptyarthaṃ sadguruḥ sarvadā vaṃdyaḥ/ yaḥ puruṣaḥ satyavādī bhavati \D1
%\om \U1
%ata eva paramapadaprāptyarthaṃ sadguruḥ sevyaḥ sarvadā yaḥ puruṣaḥ satyavādī bhavati// \U2
%-----------------------------
%
%-----------------------------
%niraṃtaraṃ gurusevātatparo bhavati/ \E [p.79]
%niraṃtaragurusevārato bhavati   \P
%niraṃtaraṃ gurusevā taro bhavati/   \B
%niraṃtaraṃ gurusevā rato bhavati...   \L
%niraṃtaraṃ gurusevā rato bhavati   \N1
%niraṃtaraṃ gurusevā rato bhava   \D1
%\om \U1
%niraṃtaraṃ gusevā   rato bhavati// \U2
%-----------------------------
%
%-----------------------------
%yasya manasi pāpaṃ na bhavati/ \E
%yasya manasi pāpaṃ na bhavati  \P
%yasya manasi pāpa nna bhavati/ \B
%yasya manasi pāpaṃ nna bhavati/ \L
%yasya manasi pāpaṃ nna bhavati/ \N1
%yasya manasi pāpaṃ nna bhavati/ \D1
%\om \U1
%yasya manasi pāpa na bhavati// \U2
%-----------------------------
%
%-----------------------------
%svācārarataḥ snānādiśīlo bhavati/ \E
%svācārarataḥ snānodiśīlo bhavati \P
%svācāraratāḥ snānādiśīlo bhavatī/ \B
%svācāraratāḥ snānādiśīlo bhavatī/ \L
%svasyācārarato snānādiśīlo bhavati/ \U1
%svasyācārarato snānādiśīlo bhavati/ \D1
%\om \U1
%svācārataḥ// snānādiśīlo bhavati// \U2
%-----------------------------
%
%-----------------------------
%kāpaṭyaṃ na bhavati yasya vaṃśaparaṃ parā jñāyate/ \E
%kāpaṭyaṃ na bhavati yasya vaṃśaparaṃ parā jñāyate  \P
%kāpaṭyaṃ bhavati/   yasya vaṃśaparaṃ parā jñāyate/ \B
%kāpaṭyaṃ na bhavati/ yasya vaṃśaparaṃ parā jñāyate... \L
%kāpaṭyaṃ nāsti/ yasya vaṃśaparaṃ parā jñāyate/ \N1
%kāpaṭyaṃ nāsti/ yasya parāparaṃ parā jñāyate/ \D1
%\om \U1
%kāpaṭyaṃ na bhavati// yasya vaṃśaparaṃ parā jñāyate// \U2
%-----------------------------
%
%-----------------------------
%etādṛśasya sadguroḥ saṃgatiḥ karttavyā  tena puruṣasya manaḥ śāṃtiṃ prāpnoti/ \E
%etādṛśasya sadguroḥ saṃgati  karttavyā       puruṣasya manaḥ śāṃtiṃ prāpnoti  \P
%etādṛśasya sadguroḥ saṃgatī  karttavyā/      puruṣasya manaḥ śāṃti  prāpnoti/ \B
%etādṛśasya sadguroḥ saṃgatī  karttavyā       puruṣasya manaḥ śāṃti  prāpnoti... \L
%etādṛśasya sadguroḥ saṃgatiḥ  kattavyāḥ      puruṣasya manaḥ śāṃtiṃ  prāpnoti/ \N1
%etādṛśasya sadguroḥ saṃgatiḥ  kattavyāḥ/     puruṣasya manaḥ śāṃtiṃ  prāpnoti/ \D1
%\om \U1
%etādṛśasya guroḥ saṃgatiḥ  karttavyā//       puruṣasya mano śāṃtiṃ  prāpnoti// \U2
%-----------------------------
%
%-----------------------------
%atha ca yasya manomadhye sthira ānanda utpadyate  sopi sadguruḥ kathyate/ \E
%atha ca yasya manomadhye sira ānanda utpadyate  sopi sadguruḥ kathyate \P
%atha ca yasya manomadhye sīraḥ  ānaṃda utpadyate sopi sadguruḥ kathyate... \B
%\om \L? recheck!!
%\om \N1
%atha ca yasya manomadhye sthira ānaṃda utpadyate/ so pi sadguruḥ kathyate/ \D1
%\om \U1
%atha ca manomadhye sthira ānanda utpadyate//  sopi sadguruḥ kathyate// \U2
%-----------------------------
% \om                                                                                                           \E
%atha ca ghaṭikārdhaṃ ghaṃṭikā caturthāṃ śovā yasya pārśvam upaviṣṭe satyatā dṛṣo bhāvo manomadhya utpadyate    \P
%atha ca ghaṭikārdhaṃ ghaṃṭikā caturthāṃ śovā yasya pārśvam upaviṣṭe satyatā dṛṣo bhāvo manomadhya uppapadyate/ \B
%atha caturthāḍaṃ śovā yasya pārśvam upaviṣṭe satyatā dṛṣo bhāvo manomadhye utpadyate/   \L
% \N1 CHECK!!!!!!!!!!!
%atha ca ghaṭīṃ mātraṃ ghaṭikārddhaṃ ghaṭikāyāḥ caturtho .. .. śovā yasya pārśvam upaviṣṭe satyetā dṛṣo bhāvo manomadhye utpadyate/ \D1 %%%%p.18 verso
%\om \U1
%atha ghaṭikāghaṭikā caturthāṃ śovā yasya pārśvam upaviṣṭe satyatādṛṣo bhāvo manomadhye utpadyate//    \U2
%-----------------------------
%
%-----------------------------
%\om                           \E
%gatvā vanamadhye  sthiyate    \P
%gatvā vanamadhye/ sthiyate/ \B
%gatvā vanamadhye sthīyate... \L
%gatvā vanamadhye sthīyate/ \N1
%gatvā vanamadhye sthīyate/ \D1
%\om \U1
%gatvā vanamadhye  sthīyate// \U2
%-----------------------------
%
%-----------------------------
%\om                                     \E
%gṛhaṃ tyajyate so pi sadguruḥ kathyate \P
%gṛhaṃ tyajyate so pi sadguruḥ kathyate/ \B
%gṛhaṃ tyajyate so pi sadguruḥ kathyate... \L
%gṛhaṃ tyajyate \om \N1
%gṛhaṃ tyajyate \om \D1
%\om \U1
%gṛhaṃ tyajyate// so pi sadguruḥ kathyate// \U2
%-----------------------------
%
%-----------------------------
%kasyāpi duḥkhaṃ na dīyate/ \E
%kasyāpi duḥkhaṃ na dīyate  \P
%kasyāpi duḥkhaṃ na dīyate/ \B
%kasyāpi duḥkhaṃ na dīyate... \L
%kasyāpi duḥkhaṃ na dīyate  \N1
%kasyāpi duḥkhaṃ na dīyate//  \D1
%\om \U1
%kasyāpi duḥkhaṃ na dīyate/ \U2
%-----------------------------
%
%-----------------------------
%prāṇimātreṇa saha maitrī kriyate  kasyāpi doṣaṃ na kathayati  so pi sadguruḥ kathyate// \E
%prāṇimātreṇa saha maitrī kriyate  kasyāpi doṣo  na prakāśyate  so pi sadguruḥ kathyate  \P %%%7678.jpg
%prāṇimātreṇa saha maitrī krīyate/ kasyāpi doṣau na prākāśate/ so pi sadguruḥ kathyate/  \B
%prāṇimātreṇa saha maitrī krīyate  kasyāpi doṣo  na prākāśate  so pi sadguruḥ kathyate \L
%prāṇimātreṇa saha maitrī krīyate/  kasyāpi doṣo  na prākāśyate/ yena so pi sadguruḥ kathyate// \N1
%prāṇimātreṇa saha maitrī krīyate/  kasyāpi doṣo  na prākāśyate/ yena so pi sadguruḥ kathyate/ \D1
%\om \U1
%prāṇimātre   saha maitrī kriyate// kasyāpi doṣo na prakāśyate//  so pi sadguruḥ kathyate// \U2 %%%430.jpg
%-----------------------------
%
%-----------------------------
%ajñātakulaśīlānāṃ yatīnāṃ brahmacāriṇām/   upadeśaṃ na gṛhṇīyādanyathā narakaṃ dhruvam// \E %[p.80]
%ajñātakulaśīlānāṃ yatīnāṃ brahmacāriṇām    upadeśo na gṛhṇīyādanyathā narakaṃ dhruvam 1 \P
%ajñānakulaśilānāṃ yatīnāṃ brahmacāriṇām    upadeśa na  gṛhītyāsthā/ yadānyathā na narakaṃ dhruvaṃ// 1// \B
%ajñānakulaśilānāṃ yatīnāṃ brahmacāriṇām//  upadeśaṃ na  gṛṇhīyādanyathā narakaṃ dhruvaṃ// 1// \L  0040.jpg
%ajñānakulaśīlānāṃ, yatīnāṃ brahmacāriṇām/  upadeśaṃ na  gṛhnīyāt anyathā narakaṃ dhruvaṃ// \N1 %%%S.14  %%%NOTEINMARGIN: ca yasya mano madhye sthira ānaṃda utpadyate mo pi sadguruḥ kathyate/ atha ca ghaṭi?mo ghaṭikārddha ghaṭīkāyāḥ 3
%ajñānakulaśīlānāṃ yatīnāṃ brahmacāriṇām/   upadeśaṃ na  gṛhnīyāt anyathā narakaṃ dhruvaṃ/ \D1
%\om \U1
%ajñātakulaśīlānāṃ yatīnāṃ brahmacāriṇām    upadeśo na gṛhṇīyāt// anyathā narakaṃ dhruvaṃ// \U2
%-----------------------------
%
%-----------------------------
%yasya vacasi manasi dhṛte sati svātmanaḥ parameśvarasyaikyaṃ bhavati/ \E
%yasya vacasi manasi dhṛte sati svātmanaḥ parameśvarasyaikyaṃ bhavati  \P
%yasya vacasi manasi dhṛte sati svātmanaḥ parameśvarasakyaṃ bhavati/ \B
%yasya vacasi manasi dhṛte sati svātmanaḥ parameśvarasakyaṃ bhavati// \L
%yasya vacasi manasi dhṛte sati/ svātmanaḥ parameśvarasyaikyaṃ bhavati/ \N1
%yasya vacasi manasi dhṛte sati/ svātmanaḥ parameśvarasyaikyaṃ bhavati/ \D1
%\om \U1
%yasya cavi          dhṛte sati svātmanaḥ parameśvarasyaikyaṃ bhavati// \U2
%-----------------------------
%
%-----------------------------
%etādṛśo manomadhye niścayo bhavati/ \E
%etādṛśo manomadhye niścayo bhavati \P
%etādṛśo manomadhye niścayo bhavati/ \B
%etādṛśo manomadhye niścayo bhavati// \L
%etādṛśo manomadhye niścayo bhavati \N1
%etādṛśo manomadhye niścayo bhavati/ \D1
%\om \U1
%etādṛśo manomadhye niścayo bhavati// \U2
%-----------------------------
%
%-----------------------------
%taṃ sadguruṃ vijānīyāt vikalpa etādaśo yathā samudramadhye mahattaraṃ kallolāḍambaram/ prapaṃce vāsanā tādṛśī yathodakamadhye mahattaraṃgāḥ/ \E
%taṃ sadguruṃ jānīyāt   vikalpa etādṛśo yathā samudramadhye mahattarakallolāḍambaraṃ prapaṃcavāsanā etādṛśī yathodakamadhye mahattarati        \P
%taṃ sadguruṃ jānīyāt   vikalpa etādṛśo yathā samudramadhye mahattarakallolāḍambara prapaṃca vāsanā etādṛśī yathodakamadhye mahattarati ....   \B
%taṃ sadguruṃ jānīyāt   vikalpa etādṛśo yathā samudramadhye mahattarakallolāḍambara prapaṃca vāsanā etādṛśī yathodakamadhye mahattarati ....   \L
%taṃ sadguruṃ jānīyāt//   vikalpa etādṛśo yathā samudramadhye mahattarakallolāḍambaraḥ/ prapaṃca vāsanā etādṛśī yathodakamadhye mahattarati  \N1
%taṃ sadguruṃ jānīyāt/   vikalpa etādṛśo yathā samudramadhye mihattarakallolāḍambaraḥ/ prapaṃca vāsanā etādṛśī yathodakamadhye mahattarati  \D1
%\om \U1
%taṃ sadguruṃ jānīyāt//   vikalpa etādṛśo yathā samudramadhye mahattarakallolāḍambaraṃ prapaca vāsanā// etādṛśī yathodakamadhye mahattarī  \U2
%-----------------------------
%
%-----------------------------
%tādṛśasya saṃsārasāgarasya yaḥ svavākyanāvā paraṃ pāraṃ prāpayati/ sa sadguruḥ kathyate// \E
%tādṛśya saṃsārārgāvādyo nāvā paraṃ prāpayati sa sadguruḥ kathyate  \P
%tādṛśāt saṃsārārṇavavādyau nāvā paraṃ prāpayati/ sa sadguruḥ kathyate// \B
%tādṛśāt saṃsārārṇavādyau nāvā paraṃ prāpayati sa sadguruḥ kathyate// \L
%tādṛśāt saṃsārātaṇavavādyo ?????? nāvā paraṃ prāpayati/ sa sadguruḥ kathyate// \N1 %%%S.14 Z. 3
%tādṛśāt saṃsārā taṛṇa vādyo nāvā paraṃ prāpayati/ sadguruḥ kathyate/ \D1
%\om \U1
%tādṛśāt saṃsārārṇavādyo nāvā pāraṃ pāraṃ prāpayati// sa sadguruḥ kathyate// \U2
%-----------------------------
%
%-----------------------------
%yasya puruṣasya mano'khaṇḍe paramamade līnaṃ bhavati/ [p.81]     \E
%yasya puruṣasya mano 'khaṃḍe pade līnaṃ bhavati/                 \P
%yasya puruṣasya manaḥ akhaṃḍe pade līnaṃ bhavatī/ DSCN7180.JPG   \B
%yasya puruṣasya manaḥ akhaṃḍe pade līnaṃ bhavati/                \L
%yasya puruṣasya mano khaṃḍe parapadalīna bhavati/                \N1
%yasya puruṣasya mano khaṃḍe parapadalīnaṃ bhavati/                \D1
%\om \U1
%yasya puruṣasya mano'khaṇḍe parapade līnaṃ bhavati     \U2
%----------------------------
%
%-----------------------------
%yaḥ puruṣaḥ svakulaṃ trividhāttāpānnivartya parame muktipade rakṣati/ \E
%yaḥ puruṣaḥ svīyaṃ kulaṃ trividhātra pānnivarttaparamamuktipade rakṣati \P %%%7579.jpg
%yaḥ puruṣaḥ svikulaṃ trividhaṃ/ tāpānnivartya paramuktipade rakṣati/ \B
%yaḥ puruṣaḥ sviyaṃ kulaṃ trividhatāpānnivartya paramuktipade rakṣati... \L
%yaḥ puruṣaḥ svīyaṃ kūlaṃ trividhāttāpānnivarttya paramamuktipade rakṣati/ \N1
%yaḥ puruṣaḥ svīyaṃ kūlaṃ trividhāttāpānmivarttya paramamuktipade rakṣati/ \D1
%\om \U1
%yaḥ puruṣa svīyaṃ kulaṃ  trividhatāpānnivartya paramamuktipakṣe rakṣati// \U2
%-----------------------------
%
%-----------------------------
%etādṛśasya puruṣasya śravaṇād darśanāt  samagravighnā naśyanti/    \E
%etādṛśā    puruṣasya śravaṇā  darśanāt  samagravighnā naśyaṃti     \P
%etādṛśā    puruṣasya śravaṇāt darśanāt/ samagravighnā na naśyaṃtī/ \B
%etādṛśā    puruṣasya śravaṇāt darśanāt  samagravighnā na naśyaṃti... \L
%etādṛśa/   puruṣaṃ   śravaṇād darśanāt  samagravighnā naśyaṃti/ \N1
%etādṛśapuruṣaṃ   śravaṇād darśanāt  samagravighnā naśyaṃti/ \D1
%etādṛśaṃ   puruṣaṃ   śravaṇād darśanāt  samagraviśvaśca vaśāṃ? bhavati \U1
%etādṛśa    puruṣasya śravaṇāt// darśanāt samagravighna naśyaṃti//    \U2
%-----------------------------
%
%-----------------------------
%dinedine kalyāṇaṃ bhavati/ niṣkalaṃkā buddhir utpadyate/    \E
%dine dine kalyāṇaṃ bhavati niṣkalaṃkā buddhir utpadyate     \P
%dinedine kalyāṇaṃ bhavati  niṣkalaṃkābuddhir  utpadyate/    \B
%dinedine kalyāṇaṃ bhavati  niṣkalaṃkābuddhir  utpadyate...  \L
%dinedine kalyāṇaṃ bhavati/ niṣkalaṃ buddhir utpadyate//     \N1
%dinedine kalyāṇaṃ bhavati/ niṣkalaṃkābuddhir utpadyate/    \D1
%dine     kalyāṇāṃ bhavatīr niṣkalaṃkābuddhir  utpadyate     \U1
%dine dine kalyāṇaṃ bhavati/ niṣkalaṃ ko buddhir utpadyate// \U2
%-----------------------------
%
%-----------------------------
%idaṃ yogaśāstrasya rahasyaṃ samastaśāstraprameyasya manaḥ           yathāṃdhakārasya    madhye dīpatejaḥ praviśati/        \E
%idaṃ yogaśāstrasya rahasyaṃ samagraśāstramadhye           yasya manaḥ yathāṃdhakārasya    madhye dīpasya tejaḥ praviśati   \P
%idaṃ yogaśāstrarahasyaṃ/    samastaśāstramadhye     manaḥ yasya manayathāṃdhakārasya    madhye dīpasya tejaḥ praviśyati... \B
%idaṃ yogaśāstrarahasyaṃ//   samastaśāstramadhye     mano  yasya manayathāṃdhakārasya    madhye dīpasya tejaḥ praviśyati...  \L
%idaṃ yogaśāstrarahasyaṃ/    samagraśāstramadhye           yasya mana/ yathāṃdhakāras    madhye dīpasya tejaḥ praviśati/     \N1
%idaṃ yogaśāstrarahasya/     samagraśāstramadhye           yasya mana/ yathāṃdhakāramadhye dīpasya tejaḥ praviśati/     \D1
%idaṃ yogaśāstreṣu rahasyaṃ  samagraśāstramadhye           yasya manaḥ yathāṃdhakārasya    madhye dīpasya tejaḥ praviśyati...  \U1
%idaṃ yogaśāstrarahasyaṃ     samagraśāstramadhye yasya manaḥ//             yathāṃdhakārasya    madhye dīpasya tejaḥ vipraśati//        \U2
%-----------------------------
%This is the secret of the Yogaśāstra. The mind of whom understands? the entire teaching, just as the light of the fire enters the centre of darkness  
%-----------------------------
%tathā śāstramadhye mano praviśati/     \E
%tathā śāstramadhye manaḥ praviśati     \P
%tathā.... \om                          \B
%tathā.... \om                          \L
%tathā śāstramadhye tasya manaḥ praviśati/     \N1
%tathā śāstramadhye tasya manaḥ praviśati/     \D1
%tathā.... \om                          \U1
%yathā śāstramadhye mano praviśati//     \U2
%-----------------------------
%in the same way the mind enters the central teaching. 
%-----------------------------
%yasya rājño madhye kalaho nāsti/ \E
%yasya rājño manomadhye kapaṭaṃ nāsti \P
%yasya rājño madhye manasi kapaṭaṃ nāsti/ \B
%yasya rājño madhye manasi kapaṭaṃ nāsti... \L
%yasya rājño manomadhye kapaṭaṃ nāsti/ \N1
%yasya rājño manomadhye kapaṭaṃ nāsti/ \D1
% rājño manomadhye    kapaṭaṃ nāsti \U1
%yasya rājño manomadhye kapaṭaṃ nāsti// \U2
%-----------------------------
%For such a king there is no deceit. 
%-----------------------------
%yasmin dṛṣṭe deśikatrāso na bhavati/ \E
%yasmin dṛṣṭe deśikasya trāso na bhavati  \P
%yasmiṃ dṛṣṭe deśikasya trāso na bhavati/ \B
%yasmiṃ dṛṣṭe deśikasya trāso na bhavati \L
%yasmiṃ dṛṣṭe deśakasya trāso na bhavati/  \N1
%yasmiṃ dṛṣṭe deśakasya trāso na bhavati/  \D1
%yasmiṃ dṛṣṭe darśakasya trāso na bhavati \U1
%yasmin dṛṣṭe deśikasya trāso na bhavati//  \U2
%-----------------------------
%Fear does not aries while the gaze is fixed on the teacher.  
%-----------------------------
%tasya manaḥ śuddhaṃ bhavati/  \E
%tasya manaḥ śuddhaṃ bhavati   \P
%tasya manaḥ śuddhaṃ bhavati/  \B
%tasya manaḥ śuddhaṃ bhavati   \L
%tasya manaḥ śuddhaṃ bhavati/  \N1
%tasya manaḥ śuddhaṃ bhavati/  \D1
%yasya manaḥ śuddhaṃ bhavati   \U1
%tasya manaḥ śuddhaṃ bhavati//  \U2
%-----------------------------
%His mind becomes pure. 
%-----------------------------
%yasya pṛthvyāṃ vītir  bhavati/ \E
%yasya pṛthivyāṃ kīrttir bhavati \P
%yasya pṛthvyāṃ kīrtir bhavati/ \B
%yasya pṛthivyāṃ kīrtir bhavati \L
%yasya pṛthivīkīrttir bhavati/ \N1
%yasya pṛthivīkīrttir bhavati/ \D1
% pṛithīvī kīrti bhavati \U1
%yasya pṛthvyāṃ kītīr  bhavati// \U2
%-----------------------------
%Fame on earth arises for him. 
%-----------------------------
%yasya manomadhye taspuruṣasya vaco viśvāso bhavati/        yo rājā sadānaṃdarūpo bhavati// \E
%yasya manomadhye satpuruṣavacanaviśvāso bhavati            yo rājā sadānaṃdapūrṇo bhavati   \P
%yasya manomadhye satpuruṣavacanaviśvāso bhavati            yo rājā sadānaṃdapūrṇo bhavati/  \B
%yasya manomadhye satpuruṣavacanaviśvāso bhavati            yo rājā sānaṃdapūrṇo bhavati   \L
%yasya manomadhye satpuruṣavacanaviśvāso bhavati/           yo rājā sadānaṃdapūrṇo bhavati   \N1
%yasya manomadhye satpuruṣavacanaviśvāso bhavati/           yo rājā sadānaṃdapūrṇo bhavati/   \D1
%yasya manomadhye satpuruṣasya vacanavihyābhyāso??? bhavati yo rājā sadānaṃdapūrṇo bhavati   \U1
%yasya manomadhye satpuruṣavacanaviśvāso bhavati//          yo rājā sadānaṃdapūrṇo bhavati// \U2
%-----------------------------
%In the middle of his mind the confidence of speech of a wise man arises. He who is highly noble is one who is full of permanent bliss. 
%-----------------------------
%yasya pārśve pratyakṣam anekamanohārivastūni tiṣṭhaṃti/ \E
%yasya pārśve pratyakṣam anekaṃ manohārivastu bhavati  \P
%yasya pārśve pratyakṣam anekamanohārivastu bhavati/ \B
%yasya pārśve pratyakṣam anekamanohārivastu bhavati \L
%yasya pārśve pratyakṣam anekaṃ manohārivastu bhavati/ \N1
%yasya pārśve pratyakṣam anekaṃ manohārivastu bhavati/ \D1
%yasya pārśve pratyakṣam anekaṃ manohārivastu bhavati \U1
%yasya pārśve pratyakṣam anekaṃ manohārivastu bhavati// \U2 %%%431
%-----------------------------
%E: Next to him in front of his eyes manifold beautiful things arise. 
%-----------------------------
%etādṛśasya rājña idaṃ yogarahasyaṃ kathanīyam/ \E
%etādṛśasya rājño ye yogarahasyaṃ kathanīyaṃ \P
%etādṛśasya rājño yethogarahasyaṃ kathyaniyaṃ/ \B
%etādṛśasya rājño yadyogarahasyaṃ kathyanīyaṃ// \L
%etādṛśasya rājño 'gre yogarahasyaṃ karttavyaṃ// \N1
%etādṛśasya rājño gre yogarahasyaṃ karttavya/ \D1
%etādṛśasya rājño gre yogarahasyaṃ karttavyaṃ \U1
%etādṛśasya rājño ye yogarahasyaṃ kathyate// \U2
%-----------------------------
%This secret of Yoga of such a highly noble person is the foremost secret of yoga that has to be told. 
%-----------------------------
%na snehān na bhayān na lobhān na mohān na dhanādbalān na maitrībhāvān naudāryān na sauṃdaryān na sevanāt/ \E
%na snehānno bhayāllno? bhāvānno dānānnasau daryānna sevanāt  \P %%%7680.jpg
%ni śnehānnā bhayāllobhānna mohān na dhanābdalāt/ na  maitrībhāvānno dānāt na sauṃdaryānna sevanāt/ \B
%ni śnehānnā bhayāllobhānna mohānna nadhanābdalāta// na maitrībhāvānno dānāt na sauṃdayanni sevanāt// \L
%na śnehāna bhayālobhān na mohān na dhanād balāt/ na maitrībhāvān na dāsān na sauṃdaryān na sevanāt/ \N1
%na śnehāna bhayāl lobhān na mohān na dhanād balāt/ na maitrī  \D1
%na śnehānna bhayān lobhān na mohān na dhanāt balāt na maitrībhāvān na dāsān na sauṃdaryān na sevatā \U1
%na snehān na bhayā llonna mohānna dhanād balāt//na maitrī bhāvān nodānān na sauṃdaryān na sevanāt// \U2
%-----------------------------
%
%-----------------------------
%sāmānyāgre yogo na kathanīyaḥ/ \E
%sāmānyād agre yogo na kathanīyaḥ \P
%sāmānyāgre yogo na kathaniyaṃ/ \B
%sāmānyāgre yogo na kathanīyaṃ// \L
%sāmānyād gre yogo na kathanīyaḥ/ \N1
%\om \D1
%sāmānyāgre yogo na kathanīyaḥ \U1
%sāmānyād gre yogo na kathanīyaḥ \U2
%-----------------------------
%
%-----------------------------
%yaḥ paraniṃdārato bhavati/ \E
%yaḥ paraniṃdārato bhavati \P
%yaḥ paraniṃdāṃ karoti/ \B
%yaḥ paraniṃdāṃ karoti \L
%yaḥ paraniṃdārato bhavati \N1
%\om \D1
% paraniṃdāṃ  rato bhavati \U1
%yaḥ paraniṃdārato bhavati// \U2
%-----------------------------
%Who has the pleasure of insulting others, 
%-----------------------------
%durācāro bhavati/ \E
%durācāro bhavati  \P
% \om              \B
% \om              \L
% dūrācāro bhavati/ \N1
%\om \D1
%dūrācāro bhavati  \U1
%dūrācāro bhavati//  \U2
%-----------------------------
%who is behaving badly, 
%-----------------------------
%durmaitryānyasya vastu na dadāti/ \E
%bhrātur mitrasya yogyaṃ vastu na dadāti \P
%durmaitryānyasya vastu na dadāti/ \B
%\om                               \L
%bhrātumitrasya ca yogyaṃ ca vastu na dadāti/ \N1
%\om \D1
%bhrātṛ rmitraṃ ca yogyaṃ vastu na dadāti \U1
%bhrātur mitrasya yogyaṃ  vastu na dadāti// \U2
%-----------------------------
%who does not gives [single] thing, which benefits friend and brother,  
%-----------------------------
%[p.83]
%ya asatyaṃ vadati/ yo yoganindāṃ karoti/ \E
%yo 'satyaṃ vadati yo yogināṃ manomadhye niṃdāṃ karoti \P
% \om                                     \B
% \om                                     \L
% so 'satyaṃ vadati/ yoginā manomadhye niṃdāṃ karoti/ \N1
%\om \D1
%so satyaṃ vadati yogināṃ manomadhye nikaroti \U1
%yo 'satyaṃ vadati// yogināṃ manomadhye niṃdāṃ karoti// \U2
%-----------------------------
%who does not speak the truth and complains about yoga, 
%-----------------------------
%yasya manomadhye dayā na bhavati/ \E
%yasya manomadhye dayā na bhavati  \P
%yasya manomadhye dayā na bhavati/ \B
%yasya manomadhye dayā na bhavati/ \L
%yasya manomadhye dayā na bhavati  \N1
%\om \D1
%yasya manomadhye dayā na bhavati  \U1
%yasya manomadhye dayā na bhavati// \U2
%-----------------------------
%in this mind compassion does not arise.  
%-----------------------------
%yaḥ    kalahapriyo bhavati/ \E
%yasya  kalaha priyo bhavati/ \P
%yasya  kalahaṃ priyo na bhavati/ \B
%yasya  kalahaṃ priyo na bhavati// \L
%yaḥ    kalahapriyo     bhavati/ \N1 [em. to  yasya kalahapriyo bhavati ||
%\om \D1
%yaḥ    kalahapriyo     bhavati \U1
%yasya  kalahaḥ priyo bhavati// \U2
%-----------------------------
% For him devotion to quarrel arises.  
%-----------------------------
%svakāryakaraṇe  sāvadhāno bhavati/ \E
%svakāryakaraṇe  sāvadhāno bhavati  \P
%svakāryākaraṇeṃ sāvadhāno bhavati/ \B
%svakāryākaraṇe  sāvadhāno bhavati// \L
%svakāryyākaraṇe  sāvadhāno bhavati/ \N1
%\om \D1
%svakāryakaraṇe  sāvadhāno bhavati \U1
%svakāryakaraṇe  sāvadhāno bhavati// \U2
%-----------------------------
%, attention arises for him only with regard to his selfish intentions,  [em. zu kāryakāraṇa]
%-----------------------------
%guroḥ  kāryakaraṇe na dattacitto bhavati/ \E
%guroḥ  kāryakaraṇe 'nādṛto??? bhavati        \P
%guro   kārye karaṇe anādarano    bhavati/ \B [em. to anādaro = disrespect] 
%guroḥ  kārykaraṇe  anādare no    bhavati/ \Ĺ
%guroḥ  kārykaraṇe  ādaro na     bhavati/ \N1
%\om \D1
%guroḥ  kāryakaraṇe  ādaro na    bhavati  \U1
%guro   kāryakaraṇe  nādṛto      bhavati//  \U2
%-----------------------------------
%disrespect arises towards the intentions of the teacher.
%-----------------------------
%etādṛśasyāgre na yogaḥ  kriyate  na paṭhyate// \E
%etādṛśasyāgre ta yogaḥ  kriyate  na paṭhyate    \P
%etādṛśasyāgre na yogaḥ  kriyate/ na paṭhayate/ \B
%etādṛśasyāgre na yogaḥ  kriyate  na paṭhayate... \L
%etādṛśasyāgre na kriyate/  na padyaṃte//  \N1
%\om \D1
%etādṛśasya agre na      kriyate  na paṭhyate \U1
%etādṛśasyāgre na yogaḥ  kriyate//  na paṭhyate// \U2
%-------------------------
%As a result, he can neither do yoga nor read.
%-----------------------------
%śrṛṇvan prītādikān śabdān paśyan rūpaṃ manoharam/ \E
%śrṛṇvan gītādikān śabdān paśyan rūpaṃ manoharaṃ   \P
%śrṛṇvan gītādikān  śabdān paśyan rūpaṃ manoharaṃ/ \B
%śṛṇvan gītādikān  śabdān/ paśyan rūpaṃ manoharaṃ/ \N1
%\om \D1
%śṛṇvan gītādikān  śabdān paśyan rūpaṃ manoharaṃ// \L
%śṛṇvan gītādikān  śabdān paśyat rūpaṃ manoharaṃ \U1
%śrṛṇvan gītādikān// śabdān paśyan rūpaṃ manoharam// \U2
%-----------------------------
%
%-----------------------------
%jāgrat sphuran spṛśansparśamṛdupriyam svādān manoramān bhrāmyan deśān/ \E
%jighran gaṃdhāṃśca surabhin spṛśan sparśaṃ mṛḍupriyaṃ  svādān manoramān khādan bhrāmyan deśān ... \P
%jighranagachan sphurabhi spṛśaṃ mṛdupriyaṃ svādān manorathān khādavan bhrāman deśān ... \B
%jighra na gachan surabhin spṛśaṃ mṛdupriyaṃ svādān manorathān khādavan bhrāman deśān ... \L
%jighran gaṃdhāṃ śusurabhīn spṛśyanasya ???mṛdupriyaṃ/ svādān manomān svādan bhrāmye na deśān\N1
%\om \D1
%jighraṃ nāṃdhaśca surabhīn sparśaṃ mṛdupriyaṃ svādāna manoramān khādaṃtabhrāmyan tveṣāṃn  \U1 %%%302.jpg
%jighran spṛśan  gaṃdhan surabhīn spṛśan sparśaṃ mṛḍu// priyaṃ  svādān manoramān bhrāmyan deśān  \U2
%-----------------------------
%
%-----------------------------
%manoramān bhāṣamāṇaḥ ramamāṇaḥ svalīlayā/ bhāvābhāvavinirmukto sarvagrahavivarjitaḥ// 1// %[p.84] \E
%manoramān bhāṣamāṇaḥ sumadhuraṃ ramamāṇaḥ svalīlayā bhāvābhāvavinirmuktaḥ sarvagrahavivarjjitaḥ   \P  %%%7681.jpg 
%manoramān bhakṣamāṇa samaghuraramāṇa svalilayā bhāvāvinirmuktaḥ/ sarvagrāhavivarjitaḥ/            \B
%manoramān bhakṣamāṇaḥ samadhuraramamāṇaṃ svalīlayā bhāvāvinir muktaḥ sarvagrāhavivarjitaḥ...      \L
%\om \D1
%manoramān/ bhāṣamāṇasya madhuraṃ ramamāṇaḥ svalīlayā// bhāvābhāvavinirmuktaḥ sarvagrāhavivarjitaḥ  \U1
%manoramān// bhāṣamāṇaḥ sumadhuraṃ ramamāṇaḥ svalīlayā// bhāvābhāvavinirmuktaḥ sarvāgrahavivarjjitaḥ// \U2
%-----------------------------
%
%-----------------------------
%sadānaṃdamayo yogī sadābhyāsī sadā bhavet/ viruddhaduḥkhade deśe virūpeti bhayānake//1// \E
%sadānaṃdamayo yogī sadābhyāsī sadā bhavet  viruddhaduḥkhade deśe virūpeti bhayānake      \P
%sadāmayo yogī sadābhyāsī sadā bhavet/ viruddhe duḥkhe deśe śovirūpe bhayānake/ \B
%sadāmayo yoyogī sadābhyāsī sadā bhavet// viruddhe duḥkhadeśe śovirūpe bhayānake... \L
%sadānaṃdamayo yogī sadābhyāsī sadā bhavet/ viruddhe duḥkhade deśe śovirūpeti bhayānake/ \N1
%\om \D1
%sadānaṃdamayo yogī sadābhyāso sadā bhavet  dviruddhe duḥkhade deśe vivarūpe bhayānake \U1
%\om \U2
%-----------------------------
%
%-----------------------------
%iṣṭādyaniṣṭasaṃsparśe rase ca lavaṇādike/ pratyādāvapi gaṃdhe ca kaṃkoṣṇādi vivarjayet//2// \E
%iṣṭādhaniṣṭaṃ saṃsparśe rase ca lavaṇādike pratyādāvapi gaṃdhe ca kaṃṭakoṣyādi vivarjjite   \P
%iṣṭādyaniṣṭasaṃsparśe rase ca lavaṇādike  pratyādāvapi gaṃdhe ca kaṭakoṣmādi varji/         \B
%iṣṭādyaniṣṭasaṃsparśe rase ca lavaṇādike  pūtyādāvapi gaṃdhe ca kaṃṭakoṣmādi varji//        \L
%iṣṭādyaniṣṭasaṃsparśe/ rase ca lavaṇādike  pūtyādāvapi gaṃdhe ca kaṃṭakoṣmādi varjjite/     \N1
%\om \D1
%iṣṭādyaniṣṭasaṃsparśe rase ca lavaṇādike  pūjādāvapi gaṃdhe ca kuṃṭakoṣmādi varjite         \U1
% \om \U2
%-----------------------------
%
%-----------------------------
%sarvadaiva sadābhyāsaḥ samaḥ syāt sukhaduḥkhayoḥ/ evaṃ yogasya karmmāṇi saṃkalparahitāni ca//3// %[p.85] \E
%sarvadaiva sadābhyāsaḥ samaḥ syāt sukhaduḥkhayoḥ  evaṃ yogasya karmāṇi  saṃkalparahitāni ca              \P
%sarvadeva  sadābhyāsāḥ samaḥ syāt sukhaduḥkhayoḥ/ evaṃ yogasya karmmāṇi saṃkalparahitāni ca/ \B
%sarvadeva  sadābhyāsāḥ samaḥ sya/ tsukhaduḥkhayoḥ// evaṃ yogasya karmmāṇi saṃkalparahitāni ca// \L
%sarvadeva  sadābhyāsāḥ samasyāt sukhaduḥkhayoḥ/ evaṃ yogasya karmmāṇi saṃkalparahitāni ca/ \N1
%\om \D1
%sarvadeva  sadābhyāsāḥ samasyā sukhaduḥkhayoḥ evaṃ bhūtakarmāṇī saṃkalparahitāni ca        \U1
%sarvadaivaṃ sadābhyāsaḥ samaḥ syāt sukhaduḥkhayoḥ// evaṃ yogasya karmāṇi saṃkalparahitāni ca//
%-----------------------------
%
%-----------------------------
%gacchan nṝṇāṃ ca saṃsparśāttapaḥ kurvanna lipyate/  utpannatattvabodhasya hyudāsīnasya sarvadā//4// \E %%%% Amanaska 2.36 
%gacchan nṝṇāṃ ca saṃsparśātpāpaḥ kurvanna lipyate   utpannatattvabodhasya udāsīnasya sarvadā        \P
%gacchan nṝṇāṃ ca saṃsparśotpāpaṃ kurvaṃ na lipyate/ utpannatattvabodhasya udāsīnasya sarvadā    \B
%gachan nṝṇāṃ ca saṃsparśāt pāpaṃ kurvan na lipyate/ utpannatattvabodhasya udāsīnasya sarvadā    \L
%gacchan nṝṇāṃ ca saṃsparśotpāpaṃ kurvan na lipyate/ utpannatattvabodhasya udāsīnasya sarvadā//    \N1
%\om \D1
%gacha  nṛṇāṃ ca saṃsparśāt pāpaṃ kurvaṃ nna lipyate utpannatatvabodhasyād udāsīnasya sarvadā    \U1
%gacchan nṝṇāṃ ca saṃsparśāt pāpaṃ kurvanna lipyate// utpannatattvabodhasya udāsīnasya sarvadā// \U2
%-----------------------------
%
%-----------------------------
%tadā dṛṣṭiviśeṣaḥ syādvividhānyāsanāni ca/ aṃtaḥ karaṇajā bhāvā yogino nopayoginaḥ//5// \E %%%Amanaska 2.37
%tadā dṛṣṭiviśeṣasyād vidhānyāsanāni ca   aṃtaḥ karaṇajā bhāvā yogino nopayoginaḥ \P
%tadā dṛṣṭiviśeṣasyād vidhānyāsanāni ca/ aṃtaḥ karaṇajā bhāvā yogino nopayoginaḥ/ \B
%tadā dṛṣṭiviśeṣasyād vidhānyāsanāni ca// aṃtaḥ karaṇajā bhāvā yogino nopi yoginaḥ// \L
%tadā dṛṣṭiviśeṣaḥ syād vidhānyāsanāni ca/ aṃtaḥ karaṇajā bhavā yogino nopa yoginaḥ/ \N1
%\om \D1
%tadā dṛṣṭir viśeṣasyād vidhānyāsanāni ca  aṃtaḥ karaṇayo bhāvā yogino nopa yoginaḥ \U1
%tadā dṛṣṭiviśeṣaḥ syād vividhāny āsanāni ca// aṃtaḥ karaṇajā bhāvā yogino nopi yoginaḥ// \U2
%-----------------------------
%
%-----------------------------
%sarvarājapadasthasya niṣkalādhyātmavedinaḥ/ yadyatprayatnaniḥ pāyaṃ tattatsarvamakāraṇam// 6// \E
%sarvadā sahajasthasya niṣkalādhyātmavedinaḥ yadyatprayatnaniḥ pārdhaṃ tattatsarvam akāraṇam    \P
%sarvadyasahajasyaniṣkalādhyātmavedinā/      yadyatprayatnaniḥ pādya tat sarvamakāraṇāt/ \B
%sarvadyasahajasthasya niṣkalādhyātmavedinā//      yadyatprayatnaniḥ pādya tat sarvemikāraṇāt//2// \L
%sarvadā sahajasthasya niṣkalādhyātmavedina/      yadyatprayatnaniḥ pādyaṃ tattatsarvamakāraṇaṃ/ \N1
%\om \D1
%sarvadā mahajarasya   niṣkalādhyātmavedanā  yadyataprayatra niṣyayiṃ tat sarvam akāraṇāṃ  \U1
%sarvadā sahajasthasya niṣkalādhyātmavedinaḥ//  yadyatprayatna niḥpādyaṃ tat sarvaṃ kāraṇa//  \U2
%-----------------------------
%
%-----------------------------
%vilāsinīnāṃ manohārigānaśravaṇāt/ \E
%vilāsinīnāṃ manohārigānaśravaṇāt  \P
%vilāsinīnāṃ manohārigānaśravaṇāt/ \B
%vilāsinīnāṃ manohārigānaśravaṇāt// \L
%vilāsinīnāṃ manohārigītaśravaṇāt/ \N1
%\om \D1
%vilāsinīnāṃ manohārigītaśravaṇāt \U1
%vilāsinīnāṃ manohārigānaśravaṇāt// \U2
%-----------------------------
%
%-----------------------------
%atisauṃdaryakāminīnāṃ rūpadarśanāt/ \E
%atisuṃdaraṃ kāmināṃ rūpadarśanāt    \P
%atisauṃdarakāminināṃ  rūpadarśanāt/ \B
%atisuṃdarakāminināṃ  rūpadarśanāt// \L
%atisuṃdarakāminīnāṃ  rūpadarśanāt// \N1
%\om \D1
%atisuṃdarakāminīnāṃ  rūpadarśanāt  \U1
%atisuṃdarakāminīnāṃ  rūpadarśanāt//  \U2
%-----------------------------
%
%-----------------------------
%kastūrīkar pūrayor gaṃdhagrahaṇāt/ \E
%kastūrīkar pūrayor gaṃdhagrahaṇāt \P
%kastūrīkar pūrayor gaṃdhagrahaṇāt/ \B
%kastūrīkar pūragaṃdhayor grahaṇāt/ \L  %%%S.15 Anfang!!
%\om \D1
% ????????????????????????? \N1???
%kastūrikar puro    gaṃdhagrahaṇāt \U1
%kastūrīkar pūrayo gaṃdhagrahaṇāt// \U2
%-----------------------------
%
%-----------------------------
%manaḥ śaityakāri komalavastunaḥ sparśakāraṇāt/ \E
%manaḥ śaityakāri komalavastunaḥ sparśakāraṇāt \P
%manaḥ śaityakāri komalavastunaḥ saṃsparśakāṃ... \B
%manaḥ śaityakāri komalavastunaḥ/ saṃsparśakaṃ... \L
%manaḥ śītalakārī atikomalaparavastunaḥ sparśakaraṇāt// \N1
%\om \D1
%manaḥ sīlakārī  atikomalaparavastutaḥ sparśakaraṇāt \U1
%manaḥ śaityakāri komalavastunaḥ sparśakāraṇāt// \U2
%-----------------------------
%
%-----------------------------
%atimādhuryaṃ citte karoti/ \E
%atimādhuryaṃ citte karoti \P
%atimādhuryaṃ citte laroti/ \B
%atimādhuryaṃ citte karoti// \L
%atimādhuryaṃ citte karoti/ \N1
%\om \D1
%atimādhuryaṃ citte karoti \U1
%atimādhuryaṃ cikrī karoti/ \U2
%-----------------------------
%
%-----------------------------
%tādṛśaḥ svādanāt/ \E
%tādṛśaḥ svādanāt \P
%tādṛśaḥ svādanāt/ \B
%tādṛśaḥ svādanāt// \L
%tādṛśā  svādanāt/ \N1
%\om \D1
%tādṛśā  svādanāt \U1
%tādṛśā  svādanāt// \U2
%-----------------------------
%
%-----------------------------
%anekadeśānāṃ sādhvasādhusthā na darśanāt/ \E
%anekadeśānāṃ sādhvasādhusthā na darśanāt  \P
%anekadeśānāṃ sādhvasādhusthā na darśanāt/ \B
%anekadeśānāṃ sādhvasādhusthā na darśanāt// \L
%anekadeśānāṃ sādhvasādhusthā na darśanāt/ \N1
%\om \D1
%anekadeśānāṃ sādhūsthā na darśanāt \U1
%anekadeśānāṃ sādhvasādhusthā na darśanāt// \U2
%-----------------------------
%
%-----------------------------
%mitreṇa saha komalavacanāt/ \E
%maitreṇa saha komalavacanāt \P
%maitreṇa saha komalavacanāt/ \B
%maitreṇa saha komalavacanāt// \L
%maitreṇa saha komalavacanāt/ \N1
%\om \D1
%maitreṇa saha komalavacanāt \U1
%maitreṇa saha komalavacanāt// \U2
%-----------------------------
%
%-----------------------------
%śatruṇā saha kaṭhinavacanāt/ \E
%śatruṇā sahakaṃ vinya vacanāt  \P
%śatruṇā saha kaṭhinya vacanāt/ \B
%śatruṇā saha kāṭhinya vacanāt// \L
%śatruṇāṃ saha kaṭhinya vacanān \N1
%\om \D1
%śatṛṇā saha kāṭhinya vacanāt    \U1
%śatruṇāṃ saha kāṭhinya vacanāt// \U2
%-----------------------------
%
%-----------------------------
%yasya manasi harṣo vā dveṣo na bhavati sa puruṣa īśvaropadeśiko jñeyaḥ/ \E
%yasya manasi harṣo vā dveṣo na bhavati sa puruṣa īśvaropadeśako jñeyaḥ \P
% missing last folio \B
% yasya mana harṣo vā dveṣo bhavati sa puruṣa īśvaropade ko jñeyaḥ/ \L
% yasya manasi harṣo vā dveṣo na bhavati/ sa puruṣa īśvaropadeśako jñeyaḥ// \N1
%                                   vati// sa puruṣa īśvaropadeśako jñeyaḥ// \D1
%yasya manasī harṣo vā dveṣo vā na bhavati sa puruṣa īśvaropadeśako jñeyaḥ \U1
%yasya manasī harṣo vā dveṣo na bhavati// pururṣo īśvaropadeśako jñeyaḥ// \U2
%-----------------------------
%
%-----------------------------
%[p.87] svalīlayā vadati calati bhāvābhāvayościttamudāsīnaṃ bhavati kasyāṃcidvārtāyāṃ harṣaviṣādaṃ na karoti yasya manaḥ sahajānaṃde magnaṃ bhavati/ \E
%       svalīlayā vadati calati va  bhāvābhāvayościttamudāsīnaṃ bhavati kasyāṃcidvārttāyāṃ haṭhaṃ na karoti yasya manaḥ sahajānaṃde magnaṃ bhavati \P
% missing last folio \B
% svalīlayā yā vadati calati ca bhāvābhāvayoś cittamudāsīna  bhavati// kasyāṃcid vārttāyāṃ haṭaṃ na karoti// yasya manaḥ// sahajānaṃdam añjaṃ bhavati/ \L
% svalīya yā vadati calati ca bhāvābhāvayoś cittamudāsīnaṃ  bhavati/ kasyāṃcid vārttāyāṃ haṭhaṃ na karoti yasya manaḥ sahajānaṃde magnaṃ bhavati/ \N1
%       svalīla yā vadati calati ca bhāvābhāvayoś cittamudāsīnaṃ  bhavati// kasyāṃcid vārttāyāṃ haṭhaṃ na karoti yasya manaḥ sahajānaṃde magnaṃ bhavati/ \D1
%       svalīlayā vadati calati ca bhāvābhāvayoś cittamudāsīnaṃ  bhavati kasyāṃcid vārttāpāṃ haṭaṃ na karoti yasya manaḥ   sahajānaṃda  saṃjñaṃ bhavati \U1 %%%304.jpg
% svalīlayā yā vadati calatī ca// bhāvābhāvayoś cittamudāsīnaṃ  bhavati// kasyāṃcid vārttāyāṃ haṭhaṃ na karoti// yasya manaḥ  sahajānaṃdaṃde magnaṃ bhavati// \U2
%-----------------------------
%
%-----------------------------
%tena puruṣeṇa dṛṣṭiḥ sthirā karttavyā/ \E
%tena bhya puruṣeṇa dṛṣṭiḥ sthirā karttavyā \P
% missing last folio \B
% tena puruṣeṇa dṛṣṭiḥ sthirā karttavyā/ \L
% tena puruṣeṇa dṛṣṭiḥ sthirā karttavyaṃ// \N1
% tena svapuruṣeṇa dṛṣṭiḥ sthirā karttavyaṃ// \D1
% tena puruṣeṇa dṛṣṭisthirā karttavyaḥ \U1
%tena puruṣeṇa dṛṣṭiḥ sthirā karttavyā// \U2
%-----------------------------
%
%-----------------------------
%āsanaṃ dṛḍhaṃ karttavyam/ \E
%āsanaṃ dṛḍhaṃ karttavyaṃ  \P
% missing last folio \B
% āsanaṃ dṛḍhaṃ karttavyaṃ/  \L
% āsanaṃ dṛḍhaṃ karttavyaṃ/  \N1
% āsanaṃ dṛḍhaṃ karttavyaṃ//  \D1
% āsanadṛḍhaṃ karttavyaṃ  \U1
%āsanaṃ dṛḍhaṃ karttavyam// \U2
%-----------------------------
%
%-----------------------------
%pavanaḥ sthiraḥ karttavyaḥ/ \E
%pavanaḥ sthiraḥ karttavyaḥ \P
% missing last folio \B
% \om \L
% pavanaḥ sthiraḥ karttavyaḥ \N1
%pavanaḥ sthiraḥ karttavyaḥ// \D1
%pavanaḥ sthiraḥ karttavyaḥ \U1
%pavanaḥ sthiraḥ karttavyaḥ// \U2
%-----------------------------
%
%-----------------------------
%etādṛśaḥ kaścinniyamaḥ/ \E
%etādṛśaḥ kaścinniyamaḥ   \P
% missing last folio \B
%etādṛśaḥ// kaścinniyamaḥ// \L
%etādṛśaḥ kaścinniyamaḥ/ \N1
%etādṛśaḥ kaścin niyamaḥ// \D1
%etādṛśaḥ kaścinīyamaḥ \U1
%etādṛśaḥ kaścinnīyamaḥ// \U2
%-----------------------------
%
%-----------------------------
%siddhasya noktaḥ manaḥpavanābhyāṃ yadā sahajānaṃdasvasvarūpeṇa prakāśyate sa sahajayogaḥ kathyate/ \E
%siddhasya noktaḥ manaḥ pavanābhyāṃ yadā sahajānaṃdaḥ svasvarūpeṇa prakāśyate sa sahajayogaḥ kathyate \P
% missing last folio \B
%siddhasya noktaḥ// manapavanābhyāṃ yadā sahajānaṃdasvasvarūpeṇa prakāśyate sa sahajayogaḥ// kathyate... \L
%siddhasya noktaḥ/ manaḥ pavanābhyāṃ yadā sahajānaṃdaḥ/ svasvarūpeṇa prakāśyate sa sahajayoga kathyate/ \N1
%siddhasya noktaḥ// manaḥ pavanābhyāṃ yadā sahajānaṃdaḥ// svasvarūpeṇa prakāśyate sa sahajayoga kathyate// \D1
%siddhasya noktaḥ  manaḥ pavanābhyāṃ padā sahajānaṃdaḥ svasvarūpeṇa prakāśate sa sahayogaḥ kathyate \U1
%siddhasya noktaḥ manaḥ pavanābhyāṃ yadā sahajānaṃdaḥ svasvarūpeṇa prakāśyate// sa sahajayogaḥ kathyate// \U2
%-----------------------------
%
%-----------------------------
%te madhye iti cakravarttikathanam// \E
%te madhye iti cakravarttikathanam// \P
% missing last folio \B
%rājayogamadhye... \L
%rājayogamadhye iti cakravarti kathyate
%rājayogamadhye iti cakravartti  nāma kathanaṃ// iti śrī paramarahasyāṃ śrīrāmacaṃdraviracitāyāṃ tatvayogabiṃdu samāptaḥ// //śrī svasti// //saṃvat 837   \N1  %%%% 1716 n. Chr.
%rājayogamadhye iti cakracarttī nāma kathanam // iti śrī paramarahasye śrīrāmacaṃdraviracitāyāṃ tatvayogabindu samāptam// //śubham// yadakṣarapadabhraṣṭaṃ mātrāhīnaṃcaya bhavet// tat sarvaṃ kṣamya ?? meca prasīdaparameśvara //1 // sūrye turaṅge navacandraghasre jyeṣṭhākhyakṛṣṇe bhṛguvārayuktam || tattvaprayogaḥ ṣaḍaharṣasaṇjñaṃ likhitaṃ suhetoḥ bhavatīha dehi || bhūyāt \N2
%rājayogamadhye iti cakravarttī  nāma kathanaṃ// iti paramarahasyāṃ śrīrāmacaṃdraviracitāyāṃ tatvayogabiṃdu samāptaḥ// śubhamastu/ saṃvat 1841// bhādau śudha 15tnīo vesarva śake rārāma rāma cha    \D1
%rājayogamadhye iti cakravaktya?? nāma madhye iti cakravaktye? nāma madhye kathanaṃ  \U1
%rājayogasya madhye iti cakracarti kathyate// \U2
%-----------------------------
%
%-----------------------------
%iti śrīsarvaguṇasampannapaṃḍita-sukhānandamiśrasūrisūnupaṇḍita-jvālāprasādamiśrakṛtabhāṣāṭīkāsahito rājayoge binduyogaḥ samāptaḥ// śubhamastu//śrīrastu// \E
%iti śrīrāmacaṃdraparamahaṃsa viracitas tatvabinduyogasamāptaḥ saṃvat 1867 pauṣakṛṣṇaḥ 12 ravau śubham bhuyāt //??//\P
% missing last folio \B
% iti rājamacaṃdraparahaṃsa viracites tatvabiṃduyogasamāptaṃ// śrī kṛṣṇārpaṇamastu// cha// \L
% iti śrī paramarahasyāṃ śrīrāmacaṃdraviracitāyāṃ tatvayogabiṃdu samāptaḥ// //śrī svasti// //saṃvat 837   \N1  %%%% 1716 n. Chr.
% iti śrī pāramahaṃsyāṃ śrī rāmacaṃdraviracitāyāṃ tatvayogaviduḥ samāptaḥ śubhaṃ bhūyāt // // atarlakṣyaṃ bahi dṛḍhir nirmeṣomeṣa varjitaḥ saiṣāśāṃbhavīmudrā sarvata,n treṣugopitā 1 aṃtark ..... %see last 2 folios verses beyond text quote from fourth chapter of HP \U1
% iti śrī rāmacaṃdraparamahaṃsaviracitas tatvabiṃduyogasamāptaḥ// śrī śubhaṃ bhavatu// śrīsītārāmārpaṇamastuḥ// idaṃ pustakaṃ// śake 1805// vikramārka saṃmat// 1140// jayanām asaṃvatsare// udagayaṇe// griṣmaṛtau?// vaiśālhemāse// kṛṣṇapakṣe// tithau 23// bhānuvāsare// prathamayāmye// śrī kṣetra avaṃtikāyāṃ// śrī mahārudramahākālasaṃnidhāne na saṃpūrṇaṃ// lekhanaṃ ānaṃt? suta bābājoo rājadherakareṇa likhyate// yādṛśaṃ pustakaṃ dṛṣtvā tādṛsaṃ likhitaṃ mayā// yadi śuddhaṃ aśuddho cā mama doṣo na dīyate//1// śrīrāma// cha//    \U2
%-----------------------------
%
%-----------------------------
%%%%deciphering last folio margin note of %N1!!! 

      
%\begin{alignment}[
%    texts=edition[class="edition"];
%    translation[class="translation"],
%  ]
%\begin{edition}
% \ekddiv{type=ed}
%\begin{prose}homa\end{prose}
%\end{edition}
%\begin{translation}
%  \ekddiv{type=trans}
%  \begin{tlate}\end{tlate}
%   \end{translation}
% \end{alignment}
\begin{otherlanguage}{english}
\chapter{Translation of the Yogatattvabindu}    
\ekddiv{type=trans}
          \centerline{\textrm{\small{[Introduction]}}}
           \bigskip
           \begin{tlate}
             Homage to Śrī Gaṇeśa. Now the methods of Rājayoga are laid down.
           \end{tlate}
           %%%%%%%%%%%%%%%%%%
\ekddiv{type=trans}
           \bigskip
           \begin{tlate}
             This is the result of Rājayoga\footnote{This statement seems unconnected to the definition of rājayoga that follows.}: Rājayoga is that by which longterm durability of the body arises even amongst manifold royal pleasures even amongst the manifold royal entertainments and spectacle. This truly is Rājayoga. These are the varieties of this Rājayoga:
      \end{tlate}
%%%%%%%%%%%%%%%%
\ekddiv{type=trans}
   \begin{tlate}
     \noindent 1. Kriyāyoga, the Yoga of [mental] action; 2. Jñānayoga, the Yoga of knowledge; 3. Caryāyoga, the Yoga of wandering;\footnote{The first three Yogas allude to the four \textit{pāda}s of the Śaiva \textit{āgama}s; namely \textit{kriyā[pāda], caryā[pāda], yoga[padā]} and \textit{jñāna[pāda]}.\parencite[77]{nishvasa2015}.} 4. Haṭhayoga, the Yoga of force; 5. Karmayoga, the Yoga of deeds; 6. Layayoga, the Yoga of absorption; 7. Dhyānayoga, the Yoga of meditation, 8.Mantrayoga, the Yoga of Mantras; 9. Lakṣyayoga, the Yoga of fixation objects, 10. Vāsanāyoga, Yoga of mental residues; 11. Śivayoga, the Yoga of Śiva, 12. Brahmayoga, the Yoga of Brahman; 13. Advaitayoga, the Yoga of non-duality; 14. Siddhayoga, the Yoga of the Siddhas; 15. Rājayoga, the King of Yogas. These are the fifteen \textit{yoga}s.\footnote{At the current stage of research it is not clear if this list is a later addition by another scribe or, if indeed it originally stems from Rāmacandra. The list suggests a text following the order of yogas according to this list. However, the order of the yogas given in the list is not followed closely in the text.}
   \end{tlate}
  %%%%%%%%%%%%%
     \ekddiv{type=trans}
      \centerline{\textrm{\small{[Description of \textit{kriyāyoga}]}}}
      \bigskip
      \begin{tlate}
        Now the characteristic of Kriyāyoga, the Yoga of [mental] action\footnote{In comparison to the Pātañjalean variant of Kriyāyoga, this variat consists of specific mental actions.} are described. \bigskip
      \end{tlate}
%%%%%%%%%%%%%%%%%%
\ekddiv{type=trans}
\begin{tlate}\textbf{1.}
  This Yoga is liberation through [mental] action, it bestows success(\textit{siddhi}) in ones own body. Each wave the mind creates at the beginning of an action, of all those one shall withdraw oneself. Then Kriyāyoga arises. \bigskip \bigskip
\end{tlate}
%%%%%%%%%%%%%%%%%%%%
      \ekddiv{type=trans}
      \begin{tlate}
        \textbf{2.} Patience, discrimination, equanimity, peace, modesty, desireless: The Yogī who is endowed with these means is said to be a Kriyāyogī. \bigskip \bigskip
      \end{tlate}
%%%%%%%%%%%%%%%%%%
     \ekddiv{type=trans}
     \begin{tlate}
       \textbf{3.} Envy, selfishness, cheating, violence, desire and intoxication, pride, lust, anger, fear, laziness, greed, error and impurity. \bigskip \bigskip
     \end{tlate}
%%%%%%%%%%%%%%%%%%%
      \ekddiv{type=trans}
      \begin{tlate}
        \textbf{4.} Attachment and aversion, indignation and idleness, impatience and dizzyness: Whoever doesn't experience these is called a Kriyāyogī.\footnote{The source of the four verses on Kriyāyoga is unknown.} \bigskip \bigskip
      \end{tlate}
%%%%%%%%%%%%%%%%%%%%
 \ekddiv{type=trans}
 \begin{tlate}
   Patience, discrimination, equanimity, peace, contentment etc. are generated in his mind. He alone is called a Yogī of many actions (\textit{bahukriyāyogī})\footnote{The term \textit{bahukriyāyogī} seems to be unique in yoga literature.}. \bigskip
 \end{tlate}
  %%%%%%%%%%%%%%%%%5
   \ekddiv{type=trans}
   \begin{tlate}
     Fraud, illusion, property,violence, craving, envy, ego, anger, anxiety, shame, greed, error, impurity, attachment, aversion, idleness, heterodoxy, false view, affection of the senses, sexual desire:\\
     He who diminishes these from day to day in is mind, he alone is called a Yogī of many actions (\textit{bahukriyāyogī}).
   \end{tlate}
%%%%%%%%%%%%%%%%%%%%
  \ekddiv{type=trans}
        \bigskip
    \centerline{\textrm{\small{[Varieties of \textit{rājayoga}: Siddhakuṇḍalinīyoga and Mantrayoga]}}}
    \bigskip
    \begin{tlate}
      Now varieties of Rājayoga will be described. Which are these? One is Siddhakuṇḍalinīyoga\footnote{On the one hand it suprises that we find the term Siddhakuṇḍalinīyoga instead of Siddhayoga as given in the initial list, on the other hand it is suprising that this type of Yoga, given as the second last item in the Yoga taxnomy is introduced as the second type right after Kriyāyoga, which was the first item in the initial list as well as in the following material.What makes this term even more strange is the fact that \textit{kuṇḍaliṇī} is not mentioned at all in the following description of this type of Yoga.} [and one\footnote{It is not entirely clear if those are two different Yogas or one and the same type of Yoga. Just the pretty late witness U2 gives us a sort of description of Mantrayoga. Judging on the basis of U2 only one could translate ``One is Siddhakuṇḍalinīyoga being Mantrayoga.'' Judging by the contents given by the rest of the witnesses this passage leaves a big queastion mark.}] is Mantrayoga\footnote{It seems odd that Mantrayoga is mentioned in the same breath as Sidhdakuṇḍalinīyoga, even though it is not directly expressed in the following. Just the additional descriptions of witness U2, highlighted in a different colour than the main text, indirectly refers to a certain practice of Mantra which is \textit{japājapa} of the \textit{so 'haṃ} for a certain duration of the practioce of meditation that is presrcibed to be performed on every \textit{cakra}.}. These two Rājayogas are described [in the following]. At the location of the root-bulb exists one major vessel in the form of energy. This single vessel reaches to these openings which are \textit{iḍā}, \textit{piṅgalā} and \textit{suṣumnā}.
    \end{tlate}
%%%%%%%%%%%%%%%%%%%%%%%%%%%
      \ekddiv{type=trans}
      \begin{tlate}
        On the left side is the \textit{iḍā}-channel, being a resemblence of the moon. On the right side exists the \textit{piṅgalā}-channel, being a resemblence of the sun. Within the middle path is a lotuspond being very subtle. [It is] made from a web of light [and it] shines like a thousand lightnings.
      \end{tlate}
%%%%%%%%%%%%%%%%%%%%%%
    \ekddiv{type=trans}
    \begin{tlate}
      She \extra{emerges as the central channel, assuming the form of benevolence (\textit{śiva}),} is the bestower of enjoyment and liberation. While abiding in (\textit{satyāṃ}) her (\textit{asyāṃ}) knowledge arises [to the point of which] the person becomes all-knowing. The means for the genesis of knowledge in the central channel will now be described.
    \end{tlate}
%%%%%%%%%%%%%%%%%%%5    
     \ekddiv{type=trans}
      \bigskip
    \centerline{\textrm{\small{[Description of the first Cakra]}}}
    \bigskip
    \begin{tlate}
      At the beginning\footnote{Supposedly at the beginning of the central channel.} exists the root \textit{cakra} having four petals.
    \end{tlate}
%%%%%%%%%%%%%%%%%%%%%%%%%
    \ekddiv{type=trans}
    \begin{tlate}
      \extra{The first \textit{cakra} of support (\textit{ādhāra}) is at the anus [and] is red-colored. Gaṇeśa is the deity. He is success, intelligence and power. A rat is the mount. The Ṛṣi is Kūrma. The seal is contraction. The vitalwind is \textit{apāna}. The \textit{kalā} is the ``wave of consciousness'' (\textit{urmī}). The concentration is ``she who is powerful'' (\textit{ojasvinī}). In the four petals [of it resides] \textit{rajas}, \textit{sattva}, \textit{tamas} and the mind-faculties (\textit{manāṃsi}), [symbolized by the syllables or \textit{bīja}s] vaṃ śaṃ ṣaṃ and saṃ. A trident is situated in the middle of the triangle\footnote{This passage is odd since a triagle wasn't mentioned before.}.}
    \end{tlate}
%%%%%%%%%%%%%%%%%%%%%
 \ekddiv{type=trans}
 \begin{tlate}
   In the middle is a trident, and \textit{kāmapīṭha}\footnote{Discuss the term \textit{kāmapīṭha}.} in the shape of a triangle. In the middle of this seat (\textit{pīṭha}) exists a single form in the shape of a flame. By meditating on this form the whole literature, all \textit{śāstra}s, all poems, dramas etc., everything [related to] elocution, appears in the mind of the person without learning.
 \end{tlate}
%%%%%%%%%%%%%%%%%%
  \ekddiv{type=trans}
  \begin{tlate}
    \extra{[Assigned to it] is external bliss\footnote{Discuss the four blisses.}, yogic bliss, heroic bliss [and] the bliss of coming to rest.}\footnote{It is noteworthy that only the first \textit{cakra} adds a detailled description of mounts, Ṛṣis, gods, seals and so forth among the current majority of witnesses at hand: E, P, L and U2. All other descriptions of the remaining eight \textit{cakra}s leave this out. The only exception is U2, a relatively late witness that adds similar descriptions for the other \textit{cakra}s as well. Since they are interesting for the history of the text I have added them to the edition's text. To indicate the extra status of those passages I have highlighted them in blue color.} An [over] hundredfold recitation of the non-recited [śataḥ = \ldots hundreds of?];  600 [repetitions for]; 9 \textit{ghaṭi}s [and] 40 \textit{palā}s.\footnote{Instructions for the duration of practice are found in all additions of U2 for each \textit{cakra}. It's not entirely clear if either the duration of meditation on the respective cakra, or the duration for the items in the list being visualised by the practitioner are meant here. However, to it seems to be done for the duration of 600 \textit{ajapājapa}, the ritualized repetition of the \textit{ajapā}, which is the voiceless uttering of the ``natural'' \textit{mantra} of the breath: \textit{so 'haṃ} - \textit{haṃ sa}. I suppose this means the practice is to be done for 600 in- and exhalations. The following part of the entry, namely ``\textit{ghaṭi} 9 \textit{palāni} 40'', probably refers to the exact time in which those 600 \textit{ajapājapa}s shall be performed. One \textit{ghaṭi} equals 1/60 of a day, which is 24 minutes. One \textit{pala} equals 1/60 of a \textit{ghaṭi} which is 24 seconds. This would equal 232 minutes or 3 hours and 52 minutes. Dividing the 600 \textit{ajapājapa}s by 232 minutes, this would result in a very slow frequence of breath of 2,586206897 in- and exhalations per minute.}
  \end{tlate}
%%%%%%%%%%%%%%%%%%%%
  \ekddiv{type=trans}
      \bigskip
    \centerline{\textrm{\small{[Description of the second Cakra]}}}
    \bigskip
    \begin{tlate}
      Now the second, the six-petalled \textit{Svādhiṣṭānacakra} known as the seat of \textit{Uḍḍīyāṇa}\footnote{Discuss the term \textit{uḍḍīyāna}.}.
    \end{tlate}
%%%%%%%%%%%%%%%%%%%%%%%
    \ekddiv{type=trans}
    \begin{tlate}
      \extra{The gender is the location. The color is yellow. The shine is yellow. \textit{Rajas} is the quality. The deity is Brahmā. The speech is \textit{vaikharī}\footnote{vaikharī f. in Kaśm. Śiv. °the 4. form of appearacne of \textit{parā}, the empirical speech sound, Utpala's Ṭīkā to Śivadṛṣṭi 2, 7. [B.]― Schmidt p. 337. Welches Buch???} (\textit{vaikharī vāca}). The power is Sāvitrī. The mount is the goose. The \textit{Rṣi} is Vahaṇa. The appearance (\textit{prabhā} is the fire of love (\textit{kāmāgni}). The body is gross, The state is that of being awake. The Veda is Ṛg. The spiritual guide is the characteristic (\textit{liṅga}). The liberation is residing in the world of Brahma. The principle is pure level (\textit{śuddhabhūmikā}). The sphere is smell. The vitalwind is \textit{apāna}. The internal matrix [is]: vaṃ bhaṃ maṃ yaṃ raṃ laṃ. The external matrix: Kāmā ``she who is desire'', Kāmākhyā ``she who is the \textit{tīrtha} of \textit{Kāmākhyā}''\footnote{The Kāmākhyā is situated in Kāmarūpa on the Nīlakūṭa mountain in present day Assam. It's strange that it appears here, since Kāmarūpa appears already as the \textit{tīrtha} associated with the first \textit{cakra}.}, Tejasvinī ``she who is shining'', Ceṣṭikā ``she who is active'', Alasā ``she who is lazy'' [and] Mithunā ``she who is ``\textit{mithunā}''. A [more than] thousandfold recitation of the non-recited; 6000 [repetitions for]; 16 \textit{ghaṭi}s [and] 40 \textit{palā}s.\footnote{The practice is supposed to be done for the duration of 6000 \textit{ajapājapa}s divided into \textit{ghaṭi}s and 40 \textit{pala}s, resulting in 2320 minutes or 38,67 hours. Again this would result in a frequence of breath of 2,586206897 in- and exhalations per minute.}}
    \end{tlate}
%%%%%%%%%%%%%%%%%%%
 \ekddiv{type=trans}
    \begin{tlate}
      In its middle exists extremely red glow. The adept becomes very handsome through meditation on it. \extra{He becomes one who is desired by young women.} The vital force increases from day to day.
    \end{tlate}
%%%%%%%%%%%%%%%%%%%
  \ekddiv{type=trans}
    \bigskip
    \centerline{\textrm{\small{[Description of the third Cakra]}}}
    \bigskip
    \begin{tlate}
      The third, a lotus with ten petals exists at the location of the navel.\
      \end{tlate}   
%%%%%%%%%%%%%%%%%%
 \ekddiv{type=trans}
    \begin{tlate}
\extra{The colour is red (\textit{kapila}). Viṣṇu is the deity. Lakṣmī is the power. Vāyu is the Rṣi. Samāna is the vitalwind. The mount is Garuḍa. The deity is the suble body\footnote{Why another deity is given here?}. The state is sleep. The speech is the inaudible speech (\textit{madhyamāvāg})\footnote{<Śā, Ling>name of the speech which is inaudible and which is of the type of a thought without any definite presence of words making up the expression. Vkp I.143.<Abhyankar 1986: 300>}. The Veda is the Yajurveda. The [fire is the] southern fire. The liberation is ``proximity'' (\textit{samīpatā}).\footnote{What is this exactly?}. Viṣṇu is the characteristic of the teacher (\textit{guruliṅga}). The principle is water. The sphere is athmosphere (\textit{rajo viṣaya}). There are ten petals [and] ten matrices. [The] inner matrix: \textit{ḍaṃ ṭaṃ ṇaṃ taṃ thaṃ daṃ dhaṃ naṃ paṃ phaṃ}. The external matrix: Śānti ``she who peaceful'', Kṣamā ``she who is patient'', Medhā ``she who is insightful'', Tanayā ``the daughter'', Medhavinī ``she who is a learned teacher'', Puṣkarā ``she who is a lotus'', Haṃsagamanā ``she who moves like a swan'', Lakṣyā ``she who is the object aimed at'', Tanmayā ``she who is absorption'' and Amṛtā ``she who is immortality''. A [more than] thousandfold recitation of the non-recited; 6000 [repetitions for]; 16 \textit{ghaṭi}s [and] 40 \textit{palā}s.\footnote{Here we find the same instruction as in the previous description of the second \textit{cakra}. The practice is supposed to be done for the duration of 6000 \textit{ajapājapa}s divided into \textit{ghaṭi}s and 40 \textit{pala}s, resulting in 2320 minutes or 38,67 hours. Again this would result in a frequence of breath of 2,586206897 in- and exhalations per minute.}}
    \end{tlate}  
%%%%%%%%%%%%%%%%%%
  \ekddiv{type=trans}
 \begin{tlate}
In its middle exists a \textit{cakra} with five angles. In its middle is a single [divine] form. It's not possible to describe her shine with speech. Through the execution of meditation on this [divine] form the body of the person is going to be strong.
 \end{tlate}
%%%%%%%%%%%%%%%%%%%%
         \ekddiv{type=trans}
       \bigskip
    \centerline{\textrm{\small{[Description of the fourth Cakra]}}}%%%fastmark 
       \bigskip
         \begin{tlate}
The fourth lotus having twelve-petals exists in the middle of the heart. 
         \end{tlate}
%%%%%%%%%%%%%%
 \ekddiv{type=trans}
    \begin{tlate}
\extra{[The] place of the Anāhatacakra is within the heart\footnote{This is redundant.}. The color is white. The quality is Tamas. The deity is Rudra. The power is Umā. The Ṛṣi is Hiraṇyagarbha. The mount is Nandi. The vitalwind is Prāṇa. The body is the cause of digits of light. The state is deep sleep. The speech is Paśyantī\footnote{Add footnote of entry in \textit{Tāntrikābhidhānakośa}.}. [The Veda is] Sāmaveda. The fire is the fire of the householder\footnote{Add explanation.}. The characteristic is Śiva. The level is the ability to attain everything on earth\footnote{Quote \textit{Tantrikābhidhānakośa}.}. The liberation is uniform [with the deity]. [There are] twelve petals, [and] twelve matrices: kaṃ khaṃ gaṃ ghaṃ ṇaṃ caṃ chaṃ jaṃ jhaṃ yaṃ taṃ [and] thaṃ. The external matrix: Rudrāṇī ``she who is Rudra's wife'', Tejasā ``she who is brilliant''\footnote{To be understood as \textit{tejasvinī}.}, Tāpinī ``she who is glow'', Sukhadā ``she who bestows happiness'', Caitanyā ``she who is consciousness'', Śivadā ``she who bestows grace'', Śānti ``she who is peaceful'', Umā ``she who is glorious'', Gaurī ``she who is beautiful'', Mātarā ``she who is bestowing the mother'', Jvalā ``she who is the flame'' [and] Prajvālinī ``she who is blazing''. A [more than] thousandfold recitation of the non-recited; 6000 [repetitions for]; 16 \textit{ghaṭi}s [and] 40 \textit{palā}s.\footnote{The \textit{ajapājapa} for this \textit{cakra} is to be performed 6000 times for a duration of 96 \textit{ghaṭi}s and 40 \textit{pala}s, resulting in 2320 minutes or 38.67 hours. Again this would result in a frequence of breath of 2,586206897 in- and exhalations per minute.}}
    \end{tlate}
%%%%%%%%%%%%%%%%%%
  \ekddiv{type=trans}
  \begin{tlate}
Due to being made of [such an] intense light [the fourth lotus] is not in the range of sight. In its middle exists a lotus facing downward having eight petals.
  \end{tlate}
%%%%%%%%%%%%%%%%%%%%
\ekddiv{type=trans}
    \begin{tlate}
      \extra{The mind resides in the \textit{cakra}. The mind is the deity. The power is external\footnote{n Muktabodha-Texte sehe ich 3 Belege für bahiśśakti Muktabodha/krīyakramādyotikā.html 2938 suṣirānte bahiśśaktiṃ vinyasedvyomarūpiṇīm | tasyā madhye tu Muktabodha/sakalāgamasārasaṅgraha.html 2186 suṣirāntabahiśśaktiṃ vyāpinīṃ cintayet tataḥ || Muktabodha/kriyakramadyotikavyākhyā.html 1846 tanmadhye ca bahiśśaktiṃ sudhābindu parisrutim}, [its] Ṛṣi is the self. In the middle of the navel exists a lotus. Its stalk measures ten \textit{aṅgula}s. The stalk of it is soft (\textit{komala}), pure [and] facing downwards. In its middle is [something] shining like a banana-flower. The mind has no determination of will, giving a firmer direction to man's thoughts for the moment by means of [conscious] submission. [It is] truly changeable in nature.}
    \end{tlate}
%%%%%%%%%%%%%%%%%%%%%%
 \ekddiv{type=trans}
    \begin{tlate}
      \extra{While the mind rests on the eastern petal [which is] white in colour clear intellekt arises, which is [endowed with] \textit{dharma}, fame and knowledge etc. While [the mind rests on] the south-east, [which is] reddish in color a mind that is weak due to sleep, laziness and illusion arises. While [the mind is situated] in the right south, [which is] black in color the generation of anger arises. While [the mind is situated] in the southwest, [which is] blue in color a mind of pride arises. While [the mind is situated] in the west, [which is] brown in color a mind that is longing for play, laughing, and celebration arises. While [the mind is situated] in the northwest, [which is] dark in color a mind which is restless by sorrow arises. While [the mind is situated] in the north, [which is] yellow in color a very happy mind with erotic and enjoyment arises. While [the mind is situated] in north-east [which is] whitish in color a mind of unity through knowledge arises.}  
    \end{tlate}
%%%%%%%%%%%%%%%%%%%
  \ekddiv{type=trans}
  \begin{tlate}
It's said that in its middle is the place of the \textit{prāṇa}-vitalwind [and] in the middle [of] the eight-petalled lotus is a pericarp (\textit{karṇikā}) in the form of a \textit{liṅga}. The technical designation of her is \textit{kalikā}. In the middle of this \textit{kalikā} exists a single thumbsized [divine] figurine (\textit{puttalikā}) being similiar to a ruby-gem in color. Her technical designation is embodied soul (\textit{jīva}). Not even with a thousand tongues it is possible to talk about her nature and her power. Here it is said [that]: ``Because of the exercise of meditation on this form the inhabitants of the universe [which are] Humans, Gandharvas, Kinnaras, Guhyakas, Vidyādharas and [their] females, in the heavenly world, underworld and open space are obedient to the will of the practicing person.''.
  \end{tlate}
%%%%%%%%%%%%%%%%
  \ekddiv{type=trans}
      \bigskip
    \centerline{\textrm{\small{[Description of the fifth Cakra]}}}
    \bigskip
    \begin{tlate}
Now the fifth lotus having sixteen petals existing at the location of the throat. 
    \end{tlate}
%%%%%%%%%%%%%%%%
 \ekddiv{type=trans}
    \begin{tlate}
\extra{The colour is grey. The deity is the embodied soul (\textit{jīva}). The power is ignorance (\textit{avidyā}). The Ṛṣi is Virāṭ\footnote{Who is this?}. The mount is the wind (\textit{vāyu}). The vitalwind is \textit{udāna}. The digit (\textit{kalā}) is the flame. The binding (\textit{bandha}) is Jālandhara. The body is the primordial cause (\textit{mahākāraṇa}). The state is the fourth state (\textit{tūrya}). The speech is Parā\footnote{Im Kaśm. Śiv. °das ewige Wort, in welchem potentiell alle Begriffe und Worte ruhen; vgl. das śabdabrahma des Vyākaraṇa. [B.]― Schmidt S. 246}. [The Veda is the] Atharvaṇa Veda. The movable characteristic (\textit{jaṅgamaṃ liṅgaṃ}). The earth is Jīvaprāptā\footnote{What is this?}. The liberation is union with the deity (\textit{sāyujyatā}). [There are] sixteen petals [and] sixteen matrices. The internal matrix: aṃ āṃ iṃ īṃ u ūṃ ṛṃ ṝṃ ḷṃ ḹṃ eṃ aiṃ oṃ auṃ aṃ aṃḥ. The external matrix: Vidyā ``she who is knowledge'', Avidyā ``she who is ignorance'', Icchā ``she who is desire'', Śakti ``she who is power'', Jñānaśakti ``she who is the power of knowledge'', Śatalā ``she who is manifold'', Mahāvidyā ``she who is great knowledge'', Mahāmayā ``she who is great illusion'', Buddhi ``she who is intellect'', Tāmasī ``she who is darkness'', Maitrā ``she who is love'', Kumārī ``she who is a young girl'', Maitrāyaṇī ``she who is onb the path of benevolence'', Rudrā ``she who is howling'', Puṣṭā ``she who is abundance'', Siṃhanī ``she who is a lioness''. A thousandfold recitation of the non-recited; 1000 [repetitions for]; 2 \textit{ghaṭi}s, 46 \textit{palā}s. and 40 \textit{akṣara}s.\footnote{It is not entirely clear what kind of measure an \textit{akṣara} is. Maybe see Amanaska 1. Chapter second half in thesis of Jason to clear things up.}}
    \end{tlate}
%%%%%%%%%%%%%%%%%%
  \ekddiv{type=trans}
  \begin{tlate}
    In its middle exists a single person which shines like a thousand moons. Because of the exercise of meditation on this person all diseases which are (otherwise) not possible to be controlled vanish. The person lives up to 1001 years.
    \end{tlate}
%%%%%%%%%%%%%%%%%
\ekddiv{type=trans}
    \bigskip
    \centerline{\textrm{\small{[Description of the sixth Cakra]}}}
    \bigskip
    \begin{tlate}
Now it exists a sixth \textit{cakra} named Ājñā.
    \end{tlate}
%%%%%%%%%%%%%%%%%%%%%
  \ekddiv{type=trans}
    \begin{tlate}
\extra{The deity is fire (\textit{agni}). The power is the godess of the centre (\textit{suṣumṇā}). The Ṛṣi is ``the violent'' (\textit{hiṃsa}). The mount is consciousness (\textit{caitanya}). The body is knowledge. The state is understanding. The speech is the ``incomparable'' (\textit{anupama}). The [Veda] is Sāmaveda. The \textit{liṅgaṃ} is intoxication (\textit{pramāda}). The half-matrix: the principle of ether. The gander is the living soul. The origin is the play of conciousness. Twofold matrix: haṃ kṣam is the inner matrix. The external matrix: Sthiti ``she who maintains'' [and] Prabhā ``she who is splendour''. A thousandfold recitation of the non-recited; 1000 [repetitions for]; 2 \textit{ghaṭi}s, 46 \textit{palā}s, and 40 \textit{akṣara}s.\footnote{It's not entirely clear what kind of measure is an \textit{akṣara}.}}
    \end{tlate}
%%%%%%%%%%%%%%%%%%%%%
\ekddiv{type=trans}
\begin{tlate}
  This \textit{cakra} is located in the middle of the eyebrows and is two-petalled. In its middle exists a certain object being a form of blazing fire without parts, not being female not being male. Because of the exercise of meditation on it the body of the person becomes non-aging and immortal.
\end{tlate}
%%%%%%%%%%%%%%%%%%%
 \ekddiv{type=trans}
    \bigskip
    \centerline{\textrm{\small{[Description of the seventh Cakra]}}}
    \bigskip
    \begin{tlate}
Now the seventh cakra having 64 petals and being full of nectar exists in the middle of the palate.
    \end{tlate}
%%%%%%%%%%%%%%%%%%%%
 \ekddiv{type=trans}
          \begin{tlate}
            \extra{The forehead is the Maṇḍala. The moon is the deity. The power is the nectar of immortality. The Rṣi is the supreme self. The seventeenth digit is the resident with the nectar of immortality. The wavy stream of nectar is great space. The uvula is the mother. The ornament/rhythm? (\textit{tālikā}) is a small bell. The own form of the body is the unspeakable Gāyatrī, [which has] the face of a crow, the eye of a human, the horn of a cow, a forehead that is Brahmapaṭhā?, a neck like a horse, the face of a peacock [and] limbs like a goose. [This is] the specific nature of the unspeakable Gayatrī.}    
            \end{tlate}
%%%%%%%%%%%%%%%%%%%
\ekddiv{type=trans}
\begin{tlate}
  It is endowed with superabundant beauty. [It is] very bright. In its middle, red in color [is that which is] known as "uvula" (\textit{ghāṃṭikā}). [It] exists as a single pericarp. In its middle is a [certain] site. In the middle of it exists a hidden digit of the moon, being a stream of nectar like a river (\textit{amṛtādhārāsravantī}.
\end{tlate}
%%%%%%%%%%%%%%%%%%%
\ekddiv{type=trans}
  \begin{tlate} 
Because of the exercise of meditation on this digit death does not come near him. Due to uninterrupted meditation, the stream (\textit{dhārā}) of nectar flows. Then the appearances of emaciation (\textit{kṣayaroga}), fever due to disordered bile (\textit{pittajvara}), heartburn (\textit{hṛdayadāha}), head-disease (\textit{śiroroga}) and tongue insensibility (\textit{jihvājaḍa}) vanish. Also eaten venom doesn't trouble him. If the mind is here, [it] becomes stable.     
  \end{tlate}
%%%%%%%%%%%%%%%%%%%%
 \ekddiv{type=trans}
    \bigskip
    \centerline{\textrm{\small{[Description of the eighth Cakra]}}}
    \bigskip
          \begin{tlate}
Now exists the eigth \textit{cakra} having one hundred petals located at the aperture of Brahman.
          \end{tlate}
%%%%%%%%%%%%%%%%%%
 \ekddiv{type=trans}
 \begin{tlate}
\extra{The teacher is the deity. Consciousness is the power. Virāṭ is the Ṛṣi, the witness above everything. Made of consciousness is that which is associated with (\textit{bhūta°}) the state beyond the fourth state. It has all colours. It has all matrices. It has all petals. The body is Virāṭ. The state is the standing still. The speech is wisdom.  The "I am that"-[expression] (\textit{sohaṃ}) is the Veda. The place is unsurpassed. A thousandfold recitation of the non-recited; 1000 [repetitions for]; 2 \textit{ghaṭi}s, 46 \textit{palā}s. and 40 \textit{akṣara}s.\footnote{It's not entirely clear what kind of measure is an \textit{akṣara}.} The count is all silent mutterings, [being] 21600. In this way it carries on day and night. He who knows the breath is a learned person. With the sound "sa" he exhales, with the sound "ha" he inhales again: "I'm he, he's I". Because of that the embodied soul constantly utters the Mantra.\footnote{Add intertextual evidence.}}
\end{tlate}
%%%%%%%%%%%%%%%%%
   \ekddiv{type=trans}
    \begin{tlate}
``The (divine) seat of  Jālaṃdhara'' is the designation of its lotus.\footnote{Find parallels where Jālandhara is situated on top of the head.} [It is] the place of the accomplished person. In its middle looking like a streak [and] having the form of smoke and fire, exists such a single [divine] form of the person (\textit{puruṣa}). Of her exists no end, nor a beginning.
    \end{tlate}
%%%%%%%%%%%%%%%%
  \ekddiv{type=trans}
  \begin{tlate}
 Due to the exercise of meditation on this [divine] form both coming and going of the person in space occurs. Affliction from the earth-element doesn't arise [anymore] even if one is situated in the middle of the earth. He constantly sees everything in front of his eyes and he becomes separated [from the material world]. The force of life increases eminently.    
     \end{tlate}
%%%%%%%%%%%%%%%%%%%%
\ekddiv{type=trans}
       \bigskip
    \centerline{\textrm{\small{[Description of the ninth Cakra]}}}
    \bigskip
    \begin{tlate}
Now the divisions/differentiations of the ninth cakra are explained. The designation of it is ``the \textit{cakra} of the great void''. Above that there is no other. Therefore it is declared to be the \textit{cakra} of the great perfection. [Another] such name of it is ``(divine) seat of Pūrṇagiri''. 
    \end{tlate}
%%%%%%%%%%%%%%%%%%%%
 \ekddiv{type=trans}
  \begin{tlate}
 In the middle of the \textit{mahāśūnyacakra} exists one lotus facing upward, very red in colour, with a thousand petals - an abode of brilliance and wholeness, whose fragrance is not in range of mind and speech. In the middle of this lotus exists one pericarp having the shape of a triangle. In the middle of the pericarp exists one seventeenth digit in the shape of a immaculé form. A light of the part exists shining like a thousand suns. [But] excessive heat is not arising. Shining like a thousand moons, excess of cold is not arising.
  \end{tlate}
%%%%%%%%%%%%%%%%%%%
  \ekddiv{type=trans}
     \begin{tlate}
\extra{Here at this location the ``I''(\textit{aham}) is the deity. The ``he is I'' (\textit{so 'ham}) is the power. This self is the Ṛṣi. The path is liberation. Brahma is the I above. ``I'm a circle''. In fire-area is the letter "sa". [There?] life arises, the living soul ascends and decends. The place is the hidden place of being. The colour is yellow. The light is the shine of ten million suns. The shine is always and visible. Śiva is the deity. The power is primordial illusion. The state is the dissolution of the self into Hara\footnote{Epiphet of Śiva.}. The transcendental sound has the nature of a sound with stable resonance. The seal is the ``fearless''. The illusion is the root. The body is the original matter. The range is speech and mind. Without delusion, without doubt, the unaffected and undefiled goal is dissolution, meditation [and] final absorption.}
     \end{tlate}
%%%%%%%%%%%%%%%%%%
 \ekddiv{type=trans}
  \begin{tlate}
Above that is the place of infinite supreme bliss. There above is power (\textit{śakti}). Being designated as such she is one single digit. Due to the exercise of meditation on this part the person manifests whatever he wishes for. He is furnished with royal pleasure and enjoyment. [Even] amusing oneself amongst women, and watching musical pleasures, the \textit{kāla} of the person grows daily like the \textit{kalā} of the moon in the bright half of the month. His body is not affected by merit and sin. Due to uninterrupted meditation the power of the light of the innate nature arises. He sees remotely located objects as if they'd be near.
\end{tlate}
%%%%%%%%%%%%%%%%
  \ekddiv{type=trans}
     \bigskip
    \centerline{\textrm{\small{[Lakṣyayoga, the yoga of fixation]}}}
    \bigskip
 \begin{tlate}
   Now the yoga of fixation (\textit{lakṣyayoga}), which is easily accomplished is explained. Of this yoga of fixation there are five subdivisions:
   1. The upward directed fixation (\textit{ūrdhvalakṣya}),
   2. the downward directed fixation (\textit{adholakṣya}),
   3. the outer fixation (\textit{baḥyalakṣya}),
   4. the central fixation (\textit{madhyalakṣya}),
   5. the inner fixation (\textit{antaralakṣya}).
 \end{tlate}
%%%%%%%%%%%%%%%%%%%%%%
  \ekddiv{type=trans}
     \bigskip
    \centerline{\textrm{\small{[1. Ūrdhvalakṣya - The upward directed fixation]}}}
    \bigskip    
  \begin{tlate}
At first the upward directed fixation (\textit{ūrdhvalakṣya}) is explained. The gaze (\textit{dṛṣṭi}) [should be] in the middle of the sky. And then having caused the mind to be directed upwards, it is caused to be fixed there. 
    \end{tlate}
%%%%%%%%%%%%%%%%%%%%%5
 \ekddiv{type=trans}
  \begin{tlate}
Due to the exercise of stabilizing of this fixation (\textit{lakṣya}) arises unity of the gazing point (\textit{dṛṣṭi}) with the light of the highest lord (\textit{parameśvara}). And then an indefinable invisible object arises in the middle of the sky. It arises in the range of sight of the practitioner. This is truly the upward directed fixation (\textit{ūrdhvalakṣya}).
  \end{tlate}
%%%%%%%%%%%%%%%%%%%%
\ekddiv{type=trans}
   \bigskip
    \centerline{\textrm{\small{[2. Adholakṣya - The downward directed fixation]}}}
    \bigskip
  \begin{tlate}
    Now the downward directed fixation object (\textit{adholakṣya}). One should stabilize the gaze within the circumference (\textit{paryanta}) of twelve \textit{aṅgula}s beyond the nose. Or one should stabilize the gaze onto the tip of the nose. The fixation becomes stable due to firm exercise [on one] of the twofold aims [of fixation].
    \end{tlate}
%%%%%%%%%%%%%%%%%%
\ekddiv{type=trans}
  \begin{tlate}
The breath becomes stable. Vitality increases. Just as this [aim] is twofold, also the external fixation is said to be [like this]. Internally or externally the aim of fixation is to be done onto the heavenly emptiness. The fear of dying doesn't arise due to the exercise of meditation on the void at all places during ones life - while eating, moving and waking.\footnote{Note that the description of the five types of Lakṣyayoga stops here and the new topic about the body of the Rājayogin is introduced. However, the subject is resumed later on in the text. Even though all witnesses follow this specific and suprising order. Maybe a copist in the early stages of transmission of the text copied the text without noticing the folios of his template to be in the wrong order.}
  \end{tlate}
%%%%%%%%%%%%%%%%%%%
  \ekddiv{type=trans}
    \bigskip
    \centerline{\textrm{\small{[Description of the Rājayogin's Body]}}}
    \bigskip
      \begin{tlate}
Now it is said that this is the characteristic of the embodied person who is endowed with the royal yoga:
      \end{tlate}
%%%%%%%%%%%%%%%
 \ekddiv{type=trans}
    \begin{tlate}
Abundance arises at all times. No distance exists on earth. He dwells on earth having pervaded [it]. Birth and death both don't exist. Happiness does'nt exist. Suffering does'nt exist. Impediment does'nt exist. Habit doesn't exist. Place does'nt exist.
    \end{tlate}
%%%%%%%%%%%%%%%%5
  \ekddiv{type=trans}
    \begin{tlate}
The manifestation of permanent perception of the connection with god arises in the middle of the mind of this accomplished one. And he is shining - not cold, and not hot, not white [and] not yellow. Neither is there birth of him, nor does he have any attributes. And he is without parts, immacule and uncharacterized. His desire etc. doesn't arise in [situations of] lust [and] is not located within the duality of the result.
    \end{tlate}
%%%%%%%%%%%%%%%%%%
\ekddiv{type=trans}
\begin{tlate}
  He attains expanded enjoyment. However, his mind does not suffer attachment in this very state.     
    \end{tlate}
%%%%%%%%%%%%%%%%%%%%%
 \ekddiv{type=trans}
    \bigskip
    \centerline{\textrm{\small{[Other Attributes]}}}
    \bigskip
  \begin{tlate}
    Another attribute of Rājayoga is described.
    \end{tlate}
%%%%%%%%%%%%%%%%%%%%%%
 \ekddiv{type=trans}
    \begin{tlate}
     Even ``of one who is in gain of a kingdom etc.'' [it is said that] perception of success does'nt arise. Even due to loss suffering does'nt arise in the mind. And then desire doesn't arise. And then with regards to an object that has been obtained for whatever reason towards ones object aversion does'nt arise. With regard to this object affection of the mind does'nt arise. Just this is said to be Rājayoga. \\
    \end{tlate}
%%%%%%%%%%%%%%%%%%%5
  \ekddiv{type=trans}
    \begin{tlate}
    And then his mind which knows the sacred speech is equal towards a person, friend and enemy. And a neutral view arises. In the mind of one who is entirely situated in the middle of the earth, the pride of authorship does't arise, because of death and rebirth, and because of happiness and enjoyment. Wile wandering the world he doesn't whish to know authorship. This is also said to be Rājayoga.
    \end{tlate}
%%%%%%%%%%%%%%%%%%%%
 \ekddiv{type=trans}
  \begin{tlate}
    New durable clothes made of silk, or however, old, worn [clothes] with holes smeared with sandalwood and musk, or smeared with mud. In whose mind joy and sorrow are not situated, just he is [in the state of] Rājayoga. Just he is in the state of Rājayoga for whom the mind is neither in abundance nor in lack, being located in a city, a forest, an uninhabited village or a village full of people.    
  \end{tlate}
%%%%%%%%%%%%%%%%%%%%%%
\ekddiv{type=trans}
      \bigskip
    \centerline{\textrm{\small{[Description of Caryāyoga]}}}
      \bigskip
     \begin{tlate}
        Now \textit{caryāyogaḥ}, the Yoga of wandering is explained.
    \end{tlate}
%%%%%%%%%%%%%%%%%%%%%
  \ekddiv{type=trans}
      \begin{tlate}
        Shapeless, unchangeable, permanent [and] unsplitable. Such is the self. It is seen as such by the one whose mind abides in the self without moving. His self is not touched by sin and merit. Just as the leave of the lotus situated in the amidst water doesn't touch the water; likewise the self [is not touched by sin and merit]. Just as the wind wanders according to its own will in space, likewise the mind of one who is absorbed into the universal spirit [wanders according to its own will in space]. This is \textit{caryāyoga}.
      \end{tlate}
%%%%%%%%%%%%%%%%%%%%%
  \ekddiv{type=trans}
       \bigskip
    \centerline{\textrm{\small{[Description of Haṭhayoga]}}}
      \bigskip
      \begin{tlate}
        Now \textit{haṭhayoga}, the forceful Yoga is explained. \\
    \end{tlate}
%%%%%%%%%%%%%%%%%%%%%%
 \ekddiv{type=trans}
    \begin{tlate}
 The practice of breath shall be done in this manner: "Exhalation, Inhalation [and] Retention etc. And then due to the six practices (\textit{ṣaṭkarma}), like \textit{dhauti} etc. the purification of the body arises. When the full breath abides in the middle of the sun-channel. Then the mind is unmovable. The form of bliss immediately shines through the motionless mind. Due to the execution of Haṭhayoga the mind becomes absorbed into emptiness. The time of death does not approach.
    \end{tlate}
%%%%%%%%%%%%%%%%%%
 \ekddiv{type=trans}
    \begin{tlate}
  Now, the second division of Haṭhayoga is explained.
    \end{tlate}
%%%%%%%%%%%%%%%%%
 \ekddiv{type=trans}
      \begin{tlate}      
        The shine of ten million suns in one's own body beginning from the feet to the top of head is contemplated in any color equal to white, yellow [or] red. Due to the execution of meditation in the entire body disease does'nt arise, fever doesn't arise and vitality grows.
      \end{tlate}
%%%%%%%%%%%%%%%%%%%%%%%
  \ekddiv{type=trans}
    \bigskip
        \centerline{\textrm{\small{[Description of \textit{Jñānayoga}]}}}
          \bigskip
    \begin{tlate}
Now the characteristic of \textit{jñānayoga} is explained. \\
    \end{tlate}
%%%%%%%%%%%%%%%%%
 \ekddiv{type=trans}
    \begin{tlate}
  \textbf{1.} He shall see the world truly as being one, shining in all selves.
            By applying indistinctness he shall accomplish \textit{Jñānayoga}. \\
    \end{tlate}
%%%%%%%%%%%%%%%%%
\ekddiv{type=trans}
    \begin{tlate}
      \textbf{2.} Wherever the world is established or made of omniscience,
            who knows thus by means of insight, he is a like an expert of knowledge. \\
    \end{tlate}
%%%%%%%%%%%%%%%%%
 \ekddiv{type=trans}
    \begin{tlate}
       \textbf{3.} He always attains the reality of \textit{śāmbhavī} - the goal of eternal non-duality.
            Just as the seed of the Nyagrodha\footnote{In rituals, the nyagrodha (Ficus indica or India fig or banyan tree) danda, or staff, is assigned to the kshatriya class, along with a mantra, intended to impart physical vitality or 'ojas'.[27. Brian K. Smith. Reflections on Resemblance, Ritual, and Religion, Motilal Banarsidass Publishe, 1998} scattered onto the soil [always] becomes a tree. \\
    \end{tlate}
%%%%%%%%%%%%%%%%%%%%
  \ekddiv{type=trans}
    \begin{tlate}
        \textbf{4.} The absolute unity (\textit{ekāntaṃ}), is seen as multibel (namely) made up of ten parts by oneself.
            The rolled up shoots of the branches are the sprouting stalks of the root shoot. \\
          \end{tlate}
%%%%%%%%%%%%%%%%%%
\ekddiv{type=trans}
    \begin{tlate}
            \textbf{5.} By virtue of its inherent nature, this branch with its branches, which is the fruit of the flower of love, is in the seed.
            Certainly, that is pure, eternal, unchanging and immaculate. \\
          \end{tlate}
%%%%%%%%%%%%%%%%%%%
  \ekddiv{type=trans}
   \begin{tlate}
            \textbf{6.} One, not one and self-existing, existing in manifold ways through its own rule and work,
            [as] five principles (\textit{tattva}) which are: thinking mind (\textit{manas}), intellect (\textit{buddhi}), illusion (\textit{māya}), individuation (\textit{ahaṃkāra}) and modifications (\textit{vikriyā}). \\       
   \end{tlate}
%%%%%%%%%%%%%%%
  \ekddiv{type=trans}
  \begin{tlate}
    \textbf{7.}In this way, the ten variations fully permeate the world and the non-world.
     Only one thing is and not something else: Whoever knows this is a connoisseur of reality. \\
  \end{tlate}
%%%%%%%%%%%%%%%%%%%%
 \ekddiv{type=trans}
 \begin{tlate}
   Transmigration is the appearance of the plant world, mountains, trees, earth etc. Transmigration is the appearance of living beings beginning with birds, horses, elephants and humans.
 \end{tlate}
 %%%%%%%%%%%%%%%%
  \ekddiv{type=trans}
  \begin{tlate}
    And then whoever is one who is a [sense] object of sight is said to be visible. He who is not seen by sight is said to be invisible.
  \end{tlate}
%%%%%%%%%%%%%%%%%%%%%
\ekddiv{type=trans}
\begin{tlate}
In this way the view of separation of one's own self which is subjected to transmigration is to be removed by means of [applying the view of] unity. Only this is Jñānayoga.  
\end{tlate}
%%%%%%%%%%%%%%%%5
\ekddiv{type=trans}
\begin{tlate}
\noindent Because of the execution of it, time does'nt destroy the body. \\
\end{tlate}
%%%%%%%%%%%%%%%
\ekddiv{type=trans}
\begin{tlate}    
Now the division of the inherent nature is described.\footnote{This refers to the mention of \textit{svabhāva} in verse 5 of the description of Jñānayoga.}
\end{tlate}
%%%%%%%%%%%%%5
\ekddiv{type=trans}
\begin{tlate}
\noindent Just as the seed of the banyan tree ripens into the shape of the banyan tree, and by its own inherent nature attains such a tenfold division. [Namely]:
\end{tlate}
%%%%%%%%%%%%%%%%
\ekddiv{type=trans}
\begin{tlate}
\noindent "Root, shoot, bark, branch, twig, bud, the unfolding flower, flower, fruit and nectar." The division reaches [those] ten parts.
\end{tlate}
%%%%%%%%%%%%%%%%%%%
\ekddiv{type=trans}
\begin{tlate}
\noindent In this way, the pure, unchanging, unblemished, attains such [division] precisely because of the inherent nature of the self. [Namely] the division "Earth, Water, Fire, Wind, Space, Mind, Intellect, Illusion, Transformations and Form".
\end{tlate}
%%%%%%%%%%%%%%%%%
\ekddiv{type=trans}
\begin{tlate}
Because of the power of Jñānayoga, there arises the certainty that "The Self is verily one."
\end{tlate}
%%%%%%%%%%%%5
\ekddiv{type=trans}
\begin{tlate}
As some particular soil (\textit{ekaika}) sometimes appears soft, sometimes beautiful, sometimes fragrant, sometimes unscented, sometimes golden, sometimes silver, is sometimes made of precious stone, sometimes appearing white, sometimes black, sometimes copper, sometimes yellow, sometimes mottled, sometimes like various fruit, sometimes like flowers, sometimes like the nectar of immortality, [and that only] because of its inherent nature.  
\end{tlate}
%%%%%%%%%%
\ekddiv{type=trans}
\begin{tlate}
In this way, the self also takes the form of a human, a bird, a gazelle, an elephant, a vidyādhara, a gandharva, a centaur, great scholar or a great fool, a sick or healthy, an angry or or peaceful person, by virtue of its inherent nature. 
\end{tlate}
%%%%%%%%%%%%5
\ekddiv{type=trans}
\begin{tlate}
Because of Jñānayoga, transformation is recognized as formless, Just as the place of origin of the fruit is only one. But the transformation of the fruit is seen as manifold. 
\end{tlate}
%%%%%%%%%%%%%
\ekddiv{type=trans}
\begin{tlate}
One fruit falls onto the ground. It is getting bright. 
\end{tlate}
%%%%%%%%%%%%
\ekddiv{type=trans}
\begin{tlate}
A bee drinks the flower juice of a fruit. The lover [bee] places itself on the flower wreath above the protuberant circular pistil.
\end{tlate}
%%%%%%%%%%%
\begin{tlate}
  \ekddiv{type=trans}
  What are the eight enjoyments?  \hfill \break
  A beautiful dwelling, good clothing, a good bed, a well-trained horse?, a nice place, food and drink.\footnote{The verse only gives 7 enjoyments!} Those are the eight enjoyments of the wise.
\end{tlate}
%%%%%%%%%%%%
\begin{tlate}
  \ekddiv{type=trans}
  1. Clothes made from silk; \hfill \break
  2. A site of the palace in which there are mainsions endowned with five or seven rooms.\hfill \break
  3. A huge, very soft and lovely bed; \hfill \break
  4. [on which] there is seated a lotus-like youthful, charming and virtuous wife;\hfill \break 
  5. An excellent throne;\hfill \break
  6. An exceptional valuable horse; \hfill \break
  7. Food that pleases the senses; \hfill \break
  8. Various drinks. \hfill \break 
\end{tlate}
%%%%%%%%%%%%
\begin{tlate}
  \ekddiv{type=trans}
Like the rays of the sun, the butter of milk, the burning of fire, the stupor of poison, the sesame oil from the sesame seed, the shade from the tree, the sweet odor from a fruit, the fire from a scabbard, the sweet sap of Śārkara\footnote{A liquor prepared from Dhātakī with sugar.} and so on, the cold of piles of snow, and so on is the inherent essence of things. In the same way, the course of the world is also in the center of the highest God's own form. And the Most High God is indivisible and all-filling.
\end{tlate}


\end{otherlanguage}
  
 \section{Bibliography}
 \label{sec:bibli}
 

\printshorthands[keyword=critEd]

\printbibliography[title=Consulted Manuskcipts, keyword=codex]

\printbibliography[title=Printed Editions, keyword=printsource]

\printbibliography[title=Secondary Literature, keyword=seclit]
\end{document}