%Ultimatives Tool zur Datierung:
%https://www.cc.kyoto-su.ac.jp/~yanom/pancanga/
%skp = ignored in edition
%skm = ignored in xml
%%%---2-DO---%%%:
% - ENTER RDGs of N2 and make apparatus negative?!
% - ENTER RDGs of D2!!
% - add xml ids for cladistics
% - produce diplomatic transcripts for saktumiva
% - make extra layer in Apparatus for parallels in SVARODAYA, Siddhasiddhantapaddhati and Amanaska
% - check all daṇḍas!!! now I think that it's more likely that many of them were lost in copies. Lectio difficilior! Very unconventional style of the autor! 
% - read Sarvangayogapradipika, Maya Burger! 
% - maybe add second ciritical edition of yogasvarodaya?!
% - Korrekturlesen von \E!! 
% - Verspattern einbauen!
% - add all Testtimonia of SSP & Ysv
% - Sigla alphabetisch ordnen und! daṇḍas mit einkollationieren
% - präambel auslagern wie Jürgen
% - grep-search alle Verse!!!!
% - Mss spreadsheet
% - sort N1,D1,B2 zu N1,N2,D1
% - sort all sigla alphabetically 
% - additions to U2: make footnotes for the bahir mātrā-s: explaining the inventions of female deities and tell that this is "schwer interpretierbar"
% - Belege für source und testimonia einfügen!!!
% - GIVE UNIQUE LABELS for TESTIMONIO AND SOurces
% - Edition mit Sätzen übereinander nennt sich: Synoptische Edition
% - Consider changing Lakṣya to Lakṣa
% - vEREINHEITLICHUNG von souce und testium! 
%%%%%%%%%%%%%%%%%%%%%%%%%%%%%%%%%%%%%%%%%
% Don't forget
% Siddhasiddhantapaddhati Yogic Body descriptions are followed by Rāmacandra
% Quotes of the Yogasvarodaya in the Yoga Karṇikā
% Rāmacandra more a compiler than an author!!!
% Identify quotes of YTB in Haṭhasanketacandrikā 
%%%%%%%%%%%%%%%%%%%%%%%%%%%%%%%%%%%%%%%%%%%
%MSS notes
%
%--B: i and ī are not differenciated
%--P: no punctuation no daṇdas nothing
%--U1: dot . serves as daṇḍa 
%--\L and \U2 very similar
%--figure out for U2: // ajapājapaḥ sahasra // 6000 //gha 0 16 pa 0 40// \U2?!?!?!?!?!?
%%%%%%%%%%%%%%%%%%%%%%%%%%%%%%%%%%%%%%%%%%
%
% Einleitung Ideen 
% - sprachliche Simplizität
% - Potenzial als Anfängertext
% - Großartige Einführung in die Textkritik -> Synoptische Edition 
% - Gelegenheit Yogasvarodaya und Yogatattvabindu zu edieren 
% - Historische Evidenz entweder für das königliche Leben in einer Maṭha in der Nähe von Benares während der Muslimischen Herrschaft, oder sogar Lehrtext für die Bildung junger Prinzen  
% - eines der raren Beispiele der engen Verknüfung mehrerer Texte 
% - eines der raren Beispiele der Prosaisierung eines metrischen Textes 
% - Anwendung rezenter Technologie! 
% - How the text was construed -> intermingling of Ysv and SSP
% - Martin Straube: "jeder kleine Dorfhäuptling kann Rāja genannt werden". 
%%%%%%%%%%%%%%%%%%%%%%%%%%%%%%%%%%%%%%%%%%%
\documentclass[10pt]{memoir}
\setstocksize{220mm}{155mm} 	        
\settrimmedsize{220mm}{155mm}{*}	
\settypeblocksize{170mm}{116mm}{*}	
\setlrmargins{18mm}{*}{*}
\setulmargins{*}{*}{1.2}
%\setlength{\headheight}{5pt}
\checkandfixthelayout[lines]
\linespread{1.16}

%%% more functions
\usepackage[dvipsnames]{xcolor}
%\usepackage[flushmargin]{footmisc}

%%% Hyphenation settings
\usepackage{hyphenat}
\hyphenation{he-lio-trope opos-sum}
\tracingparagraphs=1
%Hyphenation in Devanāgarī of the edition still missing? Probably this needs to be modified in babel-iast package? 

%%% babel
\usepackage[english]{babel}
\usepackage{babel-iast/babel-iast}
\babelfont[iast]{rm}[Renderer=Harfbuzz, Scale=1.3]{AdishilaSan}%AdishilaSan}
\babelfont[english]{rm}{Adobe Text Pro}


%%% more functionality
\PassOptionsToPackage{hyphens}{url}
\usepackage{hyperref}
\usepackage{cleveref}
\usepackage{url}
\usepackage{cleveref}
\usepackage{microtype}
\usepackage{lineno}
%\linenumbers
%\usepackage[type=lowerleftT]{fgruler}


%%%% test better pagebreaks
\def\fussy{%
  \emergencystretch\z@
  \tolerance 200%
  \hfuzz .1\p@
  \vfuzz\hfuzz}

\interfootnotelinepenalty=10000\relax

\usepackage[maxfloats=256]{morefloats}

\maxdeadcycles=500

\raggedbottomsectiontrue
\checkandfixthelayout

%%%
%\setlength{\parskip}{2cm plus0.5cm minus0.5cm}


%Index
\usepackage[backend=biber, bibstyle=verbose, citestyle=authoryear]{biblatex}

\DefineBibliographyStrings{german}{
  references = {Literaturverzeichnis},
  bibliography = {Bibliographie},
  shorthands = {Abkürzungen der Zeugen des kritischen Apparatus},
}

\makeatletter
\renewcommand*{\mkbibnamefamily}[1]{%
  \ifdefstring{\blx@delimcontext}{parencite}
    {\textsc{#1}}
    {#1}}
\makeatother

\DeclareFieldFormat{postnote}{#1}
\renewcommand{\postnotedelim}{:}
\addbibresource{bindu.bib}

%%% ekdosis
\usepackage[teiexport=tidy,parnotes=true]{ekdosis}% =tidy cleans up HTML and XML documents by fixing markup errors and upgrading legacy code to modern standards. parnotes=footnotes below or above critical apparatus

\SetLineation{lineation=page} %lineation=pagesets thenumbering to start afresh at the top of each page. modulo makes every fifth line numbered. 

\renewcommand{\linenumberfont}{\selectlanguage{english}\footnotesize} %sets language of lines to English

\SetTEIxmlExport{autopar=false} %autopar=falseinstructs ekdosis to ignore blank lines in the.tex sourcefile as markers for paragraph boundaries. As a result, each paragraph of the edition must be found within an environment associated with the xml <p> element

\SetHooks{
  lemmastyle=\bfseries,
  %refnumstyle=\selectlanguage{english}\bfseries,
  refnumstyle=\selectlanguage{english}\color{blue}\bfseries,
  appheight=0.8\textheight,
}

\newif\ifinapparatus
\DeclareApparatus{testium}[
%bhook=\inapparatustrue,
lang=english,
notelang=english,
% bhook=\selectlanguage{english},
bhook=\selectlanguage{english}\textbf{Testimonia:},
%maxentries=4, 
%ehook=.]
%sep={: },
]

\newif\ifinapparatus
\DeclareApparatus{source}[
%bhook=\inapparatustrue,
lang=english,
notelang=english,
% bhook=\selectlanguage{english},
bhook=\selectlanguage{english}\textbf{Sources:},%
%maxentries=4, 
%ehook=.]
%sep={: },
]

% Declare \ifinapparatus and set \inapparatustrue at the beginning of
% the apparatus criticus block. Also set the language.  
\newif\ifinapparatus
  \DeclareApparatus{default}[
  %bhook=\inapparatustrue, 
  lang=english,
  %maxentries=33,
  %bhook=\selectlanguage{english},
  sep = {] },
  delim=\hskip 0.75em,
  rule=\rule{0.7in}{0.4pt},
]

\newif\ifinapparatus
\DeclareApparatus{philcomm}[
%bhook=\inapparatustrue,
lang=english,
notelang=english,
bhook=\selectlanguage{english}\textbf{Philological Commentary:},
%bhook=\selectlanguage{english},
sep={: },
]

\ekdsetup{
showpagebreaks,
spbmk = \textcolor{blue}{spb},
hpbmk = \textcolor{red}{hpb}
}

% Macros and Definitions for the Print of Sigla
\def\acpc#1#2#3{{#1}\rlap{\textrm{\textsuperscript{#3}}}\textsubscript{\textrm{#2}}\space}
\def\sigl#1#2{{{#1}}\textsubscript{\textrm{#2}}}
\def\None{{\sigl{N}{1}}} \def\Noneac{\acpc{N}{1}{ac}\,} \def\Nonepc{\acpc{N}{1}{pc}\,}
\def\Ntwo{{\sigl{N}{2}}} \def\Noneac{\acpc{N}{2}{ac}\,} \def\Nonepc{\acpc{N}{2}{pc}\,}
\def\Done{{\sigl{D}{1}}} \def\Doneac{\acpc{D}{1}{ac}\,} \def\Donepc{\acpc{D}{1}{pc}\,}
\def\Dtwo{{\sigl{D}{2}}} \def\Dtwoac{\acpc{D}{2}{ac}\,} \def\Dtwopc{\acpc{D}{2}{pc}\,}
\def\Uone{{\sigl{U}{1}}} \def\Uoneac{\acpc{U}{1}{ac}\,} \def\Uonepc{\acpc{U}{1}{pc}\,}                 
\def\Utwo{{\sigl{U}{2}}} \def\Utwoac{\acpc{U}{2}{ac}\,} \def\Utwopc{\acpc{U}{2}{pc}\,}

%%%%%%%%%%%%%% Tattvabinduyoga - List of Witnesses   %%%%%%%%%%%%%%%%%%%
\DeclareWitness{ceteri}{\selectlanguage{english}cett.}{ceteri}[]   
\DeclareWitness{E}{\selectlanguage{english}E}{Printed Edition}[]    
\DeclareWitness{P}{\selectlanguage{english}P}{Pune BORI 664}[]  
\DeclareWitness{B}{\selectlanguage{english}B}{Bodleian 485}[]       
\DeclareWitness{N1}{\selectlanguage{english}N\textsubscript{1}}{NGMPP 38/31}[]
\DeclareWitness{N2}{\selectlanguage{english}N\textsubscript{2}}{NGMPP B 38/35}[]
\DeclareWitness{L}{\selectlanguage{english}L}{LALCHAND 5876}[]  
\DeclareWitness{D}{\selectlanguage{english}D}{IGNCA 30019}[] 
%\DeclareWitness{D2}{\selectlanguage{english}D\textsubscript{2}}{IGNCA 30020}[]  
\DeclareWitness{U1}{\selectlanguage{english}U\textsubscript{1}}{SORI 1574}[] 
\DeclareWitness{U2}{\selectlanguage{english}U\textsubscript{2}}{SORI 6082}[]
%%%%%%%%%%%%% Testimonia
\DeclareWitness{Ysv}{\selectlanguage{english}Ysv}{Yogasvarodaya}[] %%%add infos!  

%%%%%%%%%%%%%%%%%%%%%%%%%%%%%%%%%%%%%%%%%%%
% Macro for Editing Abbrevs.
\def\om{\textrm{\footnotesize \textit{om.}\ }} %prints om. for omitted in apparatus
\def\korr{\textrm{\footnotesize \textit{em.}\ }} %prints em. for emended in apparatus
\def\conj{\textrm{\footnotesize \textit{conj.}\ }} %prints conj. for conjectured in apparatus

% \supplied{text} EDITORIAL ADDITION -> Within \lem oder \rdg
% \surplus{text} EDITORIAL DELETION -> Within \lem oder \rdg
% \sic{text} CRUX
% \gap{text} LACUNAE -> [reason=??, unit=??, quantity=??, extent=??]


%%%%%%%%%%%%%%%%%%%%%%%%%%%%%%%%%%%%%%%%%%% All macros of this list can be used in 
% Macro for Editing Abbrevs.
\def\eyeskip{\textrm{{ab.\,oc. }}}
\def\aberratio{\textrm{{ab.\,oc. }}}
\def\ad{\textrm{{ad}}}
\def\add{\textrm{{add.\ }}}
\def\ann{\textrm{{ann.\ }}}
\def\ante{\textrm{{ante }}} 
\def\post{\textrm{{post }}}
%\def\ceteri{cett.\,}                   
\def\codd{\textrm{{codd.\ }}}

\def\coni{\textrm{{coni.\ }}}
\def\contin{\textrm{{contin.\ }}}
\def\corr{\textrm{{corr.\ }}}
\def\del{\textrm{{del.\ }}}
\def\dub{\textrm{{ dub.\ }}}

\def\expl{\textrm{{explic.\ }}} 
\def\explica t{\textrm{{explic.\ }}}
\def\fol{\textrm{{fol.\ }}}
\def\foll{\textrm{{foll.\ }}}
\def\gloss{\textrm{{glossa ad }}}
\def\ins{\textrm{{ins.\ }}}      
\def\inseruit{\textrm{{ins.\ }}} 
\def\im{{\kern-.7pt\lower-1ex\hbox{\textrm{\tiny{\emph{i.m.}}}\kern0pt}}} %\textrm{\scriptsize{i.m.\ }}}      
\def\inmargine{{\kern-.7pt\lower-.7ex\hbox{\textrm{\tiny{\emph{i.m.}}}\kern0pt}}}%\textrm{\scriptsize{i.m.\ }}}      
\def\intextu{{\kern-.7pt\lower-.95ex\hbox{\textrm{\tiny{\emph{i.t.}}}\kern0pt}}}%\textrm{\scriptsize{i.t.\ }}}           
\def\indist{\textrm{{indis.\ }}}  
\def\indis{\textrm{{indis.\ }}}
\def\iteravit{\textrm{{iter.\ }}} 
\def\iter{\textrm{{iter.\ }}}
\def\lectio{\textrm{{lect.\ }}}   
\def\lec{\textrm{{lect.\ }}}
\def\leginequit{\textrm{{l.n. }}} 
\def\legn{\textrm{{l.n. }}}
\def\illeg{\textrm{{l.n. }}}

\def\primman{\textrm{{pr.m.}}}
\def\prob{\textrm{{prob.}}}
\def\rep{\textrm{{repetitio }}}
\def\secundamanu{\textrm{\scriptsize{s.m.}}}            \def\secm{{\kern-.6pt\lower-.91ex\hbox{\textrm{\tiny{\emph{s.m.}}}\kern0pt}}}%   \textrm{\scriptsize{s.m.}}}
\def\sequentia{\textrm{{seq.\,inv.\ }}}  
\def\seqinv{\textrm{{seq.\,inv.\ }}}
\def\order{\textrm{{seq.\,inv.\ }}}
\def\supralineam{{\kern-.7pt\lower-.91ex\hbox{\textrm{\tiny{\emph{s.l.}}}\kern0pt}}} %\textrm{\scriptsize{s.l.}}}
\def\interlineam{{\kern-.7pt\lower-.91ex\hbox{\textrm{\tiny{\emph{s.l.}}}\kern0pt}}}   %\textrm{\scriptsize{s.l.}}}
\def\vl{\textrm{v.l.}}   \def\varlec{\textrm{v.l.}} \def\varialectio{\textrm{v.l.}}
\def\vide{\textrm{{cf.\ }}}
\def\cf{\textrm{{cf.\ }}} 
\def\videtur{\textrm{{vid.\,ut}}}
\def\crux{\textup{[\ldots]} }
\def\cruxx{\textup{[\ldots]}}
\def\unm{\textit{unm.}}
%%%%%%%%%%%%%%%%%%%%%%%%%%%%%%%%%%%%

% List of Scholars
\DeclareScholar{ego}{ego}[
forename=Nils Jacob,
surname=Liersch]

% Persons:14\DeclareScholar{ego}{ego}[15forename=Robert,16surname=Alessi]17% Useful shorthands:18\DeclareShorthand{codd}{codd.}{V,I,R,H}19\DeclareShorthand{edd}{edd.}{Lit,Erm,Sm}20\DeclareShorthand{egoscr}{\emph{scripsi}}{ego}

%Useful shorthands:
%\DeclareShorthand{codd}{codd.}{V,I,R,H}
%\DeclareShorthand{edd}{edd.}{Lit,Erm,Sm}
\DeclareShorthand{egoscr}{\emph{scripsi}}{ego}
\DeclareShorthand{egomute}{\unskip}{ego}

\usepackage{xparse}

%%% define environments and commands
\NewDocumentEnvironment{tlg}{O{}O{}}{\vspace{-1ex}\begin{verse}}{\hfill #1\\ \vspace{-1ex}\end{verse}} %verse environment
%\NewDocumentEnvironment{tlg}{O{}O{}}{\begin{verse}}{॥#1\hskip-4pt ॥\\ \end{verse}}
\NewDocumentCommand{\tl}{m}{{\selectlanguage{iast} #1}}

\NewDocumentCommand{\extra}{m}{{\textcolor{MidnightBlue}{#1}}} %command for additions to U2
\NewDocumentCommand{\crazy}{m}{{\textcolor{red}{#1}}} %totally corrupted passage 

\NewDocumentEnvironment{prose}{O{}}{\begin{otherlanguage}{iast}}{\end{otherlanguage}}
% \NewDocumentEnvironment{padd}{O{}}{\begin{otherlanguage}{iast}}{\end{otherlanguage}}
\NewDocumentEnvironment{tlate}{O{}}
%\NewDocumentEnvironment{tadd}{O{}}

%Define two commands: \skp ("sanskrit plus"), to be ignored by TeX in
%the edition text, but processed in the TEI output. Conversely, \skm
%("sanskrit minus") is to be processed in the edition text, but
%ignored if found in the apparatus criticus and in the TEI output:

\NewDocumentCommand{\skp}{m}{}
\TeXtoTEIPat{\skp {#1}}{#1}

%\NewDocumentCommand{\skpp}{m}{}
%\TeXtoTEIPat{\skpp {#1}}{#1}

\NewDocumentCommand{\skm}{m}{\unless\ifinapparatus#1-\fi}
\TeXtoTEIPat{\skm {#1}}{}

\NewDocumentCommand{\dd}{}{/\hskip-4pt/}
\TeXtoTEIPat{\dd {}}{//}


%%% modify environments and commands
%%% TEI mapping
\TeXtoTEIPat{\begin {tlg}}{<lg>} %lg=(Group of verse (s)) contains one or more verses or lines of verse that together form a formal unit (e.g. stanza, chorus).
\TeXtoTEIPat{\end {tlg}}{</lg>}

\TeXtoTEIPat{\begin {prose}}{<p>}
\TeXtoTEIPat{\end {prose}}{</p>}

\TeXtoTEIPat{\begin {tlate}}{<p>}
\TeXtoTEIPat{\end {tlate}}{</p>}

\TeXtoTEIPat{\\}{}
\TeXtoTEIPat{\linebreak}{<br/>}
\TeXtoTEIPat{\noindent}{}
%\TeXtoTEI{tl}{l}
\TeXtoTEI{emph}{hi}
\TeXtoTEI{bigskip}{}
\TeXtoTEI{None}{N1}
\TeXtoTEI{Ntwo}{N2}
\TeXtoTEI{Done}{D1}
\TeXtoTEI{Dtwo}{D2}
\TeXtoTEI{Uone}{U1}
\TeXtoTEI{Utwo}{U2}
%\TeXtoTEIPat{/}{ |}
%\TeXtoTEI{//}{ ||}
\TeXtoTEIPat{\korr}{em. }
\TeXtoTEIPat{\conj}{conj.}
\TeXtoTEIPat{\om}{om.}
\TeXtoTEIPat{english}{}
\TeXtoTEIPat{\hskip}{}
\TeXtoTEIPat{\hskip-4pt}{}
\TeXtoTEIPat{\hskip-2pt}{}
\TeXtoTEIPat{-}{ }
\TeXtoTEIPat{4pt}{}
\TeXtoTEIPat{2pt}{}
\TeXtoTEIPat{\textcolor {#1}{#2}}{<hi rend="#1">#2</hi>} 

% Nullify \selectlanguage in TEI as it has been used in
% \DeclareWitness but should be ignored in TEI.
\TeXtoTEI{selectlanguage}{}

\author{Nils Jacob Liersch}
\title{Yogatattvabindu of Rāmacandra\\ A Critical Edition and Annotated Translation}
\date{\today}

\parindent=15pt
\begin{document}
\maketitle
\clearpage
\tableofcontents
\addtocounter{page}{-1}
\thispagestyle{empty}
\clearpage


\chapter{The List of the 15 Yogas}
\begin{itemize}
\item It's not entirely clear if the list given at the beginning of the text codifying the fifteen \textit{yoga}s belongs to the original text or was a later addition by a another hand. One primary reason for this possibility is the structure of the \textit{yoga}s in the actual course of the text does not equal the list. The text begins with a description of \textit{kriyāyoga} and continues to describe \textit{siddhakuṇḍaliniyoga} and somewhat suprisingly mentions \textit{mantrayoga} in the same breath. One starts wondering why the structure of the text does not follow the codification. However the mention of \textit{jñānotpattav upāyaḥ} might be a clue why the second \textit{yoga} in the list might be \textit{jñānayoga}. So far it seems to me that there are three options or a combination of these to explain these apparent inconsistencies: 1. The text is highly corrupted. 2. The codification was a later addition of another hand. 3. The term \textit{jñānayoga} is listed due to the results of \textit{siddhakuṇḍalinīyoga}, which is the generation of knowledge due to the practice of a certain \textit{yoga} involving the central channel, as mentioned in this section of the text.
\end{itemize}

\chapter{Conventions in the Critical Apparatus}
\section{Sigla in the Critical Apparatus}

\begin{itemize}
\item E : Printed Edition
\item P : Pune BORI 664
\item L : Lalchand Research Library LRL5876
\item B : Bodleian Oxford D 4587
\item \None : NGMPP B 38-31
\item \Ntwo : NGMPP B 38-35 / A 1327-14
\item \Done : IGNCA 30019
\item \Uone : SORI 1574
\item \Utwo: SORI 6082
\end{itemize}

The order of the readings in the critical apparatus is arranged according to the quality of readings in decending order. The critical apparatus is positive. Gemitation is not recorded. 

\section{Punctuation}

The very inconsistent use of punctuation marks in the witnesses at hand makes standardization necessary. A close examination of the overall usage of punctuation suggest that in the course of the texts transmission punctuations have been dropped frequently or even have been added. Particularly in the lists given in the text the copists negliance or not properly dealing with punctuation resulted in various forms of those lists with and without punctuations. Due to missing punctuation in many instances copists either made up case endings, changed the text and combined the lists' items into compounds that weren't present in the assumed original text. Even though punctuation plays a role that should'nt be underestimated, the deviation of punctuation at the end of sentences, lists and verse-numbering will only be documented in the critical apparatus of the printed edition to meaningful extend. That means, for example that emendations of obvious mistakes in punctuation will not be recorded in the critical apparatus. However, the digital edition of this work provides a way more detailled documentation of deviations in punctuation in the form of diplomatic transcripts of each witness and even a function to display sentences cummulativly on top of each other.

In the printed edition of the \textit{Tattvayogabindu} the standard conventions of punctuation are followed:

In verse poetry, a \textit{daṇḍa} marks the end of a half verse, half of the \textit{śloka}, and the double \textit{daṇḍa} marks the end of a verse. A half verse is a \textit{pāda}, at least in some literary works, this is concluded by a \textit{daṇḍa} and the end of a \textit{śloka} by a double \textit{daṇḍa}. In prose the single \textit{daṇḍa} indicates the end of a sentence and the double \textit{daṇḍa} marks the end of a paragraph.

Variations in the usage of \textit{Avagraha} will be recorded. Items of lists will be separated by a single \textit{daṇḍa}. 

\section{Sandhi}

Among the witnesses we see deviating and inconsistent application of \textit{sandhi}. There is no clear evidence that originally \textit{sandhi} was intentionally not applied. This edition will therefore apply \textit{sandhi} consistently throughout the constituted text to provide a readable text sticking to contemporary conventions in Sanskrit. The variant readings concerning \textit{sandhi} are recorded consistently in the apparatus criticus. This is due to various textcritical problems arising from the inconsistent usage of punctuation which results in application or non-application of \textit{sandhi} wheter the respective witness applied a \textit{daṇḍa} or not. This is particularly the case within lists, which frequently occur in our compilation. Items were most likely originally separated by \textit{daṇḍa}. 

\section{Class Nasals}

Again, due to inconsistent use of class nasals among the witnesses \textit{anusvāra}s have been substituted with the respective class nasals throughout the edition.

\section{Lists}

Lists are very frequent in the \textit{Yogatattvabindu}. In fact, the text initially gives a list of 15 Yogas in the beginning and many more lists are have been utilized throughout the text. Many witness lost punctuation in the process of copying and as a consequence applied \textit{sandhi}, to arrive at a consistent and conveniently readable edition of the text, all list have been identified as such and normalized to the Nominativ Singular or Nominativ Plural form of the respective item. Items are separated by a double \textit{daṇḍa}. The differences in punctuation, as well as simple emendations regarding punctuation  won't be documented in the apparatus criticus. 
\clearpage

\chapter{Critical Edition}
\clearpage
\begin{ekdosis}
\ekddiv{type=ed}
    \centerline{\textrm{\small{[Introduction]}}}
    \bigskip
    \begin{prose}
%--------------------------
% śrī gaṇeśāya namaḥ /                                                     rājayogāntargataḥ //  binduyogaḥ   \E 
% śrī gaṇeśāya namaḥ /                                                     atha tattvabiṃduyogaprāraṃbhaḥ     \L
% śrī ṇe ya maḥ /                                                          atha rājayoga         liṣyate      \P
% \om                                                                                                         \B      
% śrī gaṇeśāya namaḥ // śrī gurave namaḥ //                                atha rājayogaprakāro  likhyate //  \N1
% śrī gaṇeśāya namaḥ //                                                //  atha rājayogaprakāro  likhyate //  \N2
% śrī gaṇeśāya namaḥ // śrī sarasvatyai namaḥ // śrī nirañjanāya namaḥ //  atha rājayogaprakāro  likhyate //  \D
% \om                                                                                                         \D2
% śrī gaṇeśāya namaḥ /  oṃ śrī niraṃjanāya //                              atha rājayogaprakāra  likhyate //  \U1
% śrī gaṇeśāya namaḥ /                                                     atha rājayoga         likhyate //  \U2
%--------------------------
%Homage to Śrī Gaṇeśa. Now the methods of rājayoga are laid down.
%--------------------------          
\noindent \app{\lem[wit={ceteri}]{śrī gaṇeśāya namaḥ}
        \rdg[wit={P}]{śrī ṇe ya maḥ}
        \rdg[wit={N1}]{śrī gaṇeśāya namaḥ || śrī gurave namaḥ ||}
        \rdg[wit={D}]{śrī gaṇeśāya namaḥ || śrī sarasvatyai namaḥ || śrī nirañjanāya namaḥ ||}
        \rdg[wit={U1}]{śrī gaṇeśāya namaḥ || oṃ śrī niraṃjanāya ||}}\dd{}
\app{\lem[wit={N1,N2,D}]{atha rājayogaprakāro likhyate}
        \rdg[wit={U1}]{atha rājayogaprakāra likhyate}
        \rdg[wit={E}]{rājayogāntargataḥ || binduyogaḥ}
        \rdg[wit={L}]{atha tattvabiṃduyogaprāraṃbhaḥ}
        \rdg[wit={P}]{atha rājayoga liṣyate}
        \rdg[wit={U2}]{atha rājayoga likhyate}}\dd{}
%-------------------------- 
% \om                        \E
% \om                        \L
% \om                        \B
% rājayogasyedaṃ phalaṃ      \P
% rājayogasya idaṃ phalaṃ    \N1
% rājayogasya idaṃ phalaṃ    \N2
% rājayogasya idaṃ phalaṃ // \D
% \om                        \D2
% rājayogasya idaṃ phalaṃ    \U1
% rājayogasyedaṃ phalaṃ /    \U2
%--------------------------    
\app{\lem[wit={P,U2}]{rājayogasyedaṃ phalaṃ}
  \rdg[wit={N1,N2,D}]{rājayogasya idaṃ phalaṃ}
  \rdg[wit={E,L}]{\om}}/
%--------------------------
%This is the result of \textit{rājayoga}:
%--------------------------
% \om                                                                                                                                                                         \E
% \om                                                                                                                                                                         \L
% \om                                                                                                                                                                         \B
% yena rājayogenāneka---rājyabhogasamaya   eva   anekapārthivavinodaprekṣaṇasamaya  eva   bahutarakālaṃ  śarīrasthitir  bhavati    sa eva  rājayogaḥ tasyaite     bhedāḥ      \P
% yena rājayogenāneka---rājyabhogasamaya   eva/  anekapārthivavinodaprekṣaṇasamaya  eva/  bahutarakālaṃ  śarīrasthitir  bhavati    sa eva  rājayogaḥ /  tasya ete bhedāḥ /    \N1
% yena rājayogena  anekarājyabhogasamaya   eva// anekapārthivavinodaprekṣaṇasamaya  eva   bahuttarakālaṃ śarīrasthitir  bhavati    sa eva  rājayogaḥ /  tasya ete bhedāḥ /    \N2
% yena rājayogena  anekarājyabhogasamaya   eva// anekapārthivavinodaprekṣaṇasamaya  eva// bahutarakālaṃ  śarīrasthitir  bhavati//  sa eva  rājayogaḥ // tasya ete bhedāḥ /    \D
% \om                                                                                                                                                                         \D2
% yena rājayogena  anekarājyabhogasamaya   eva// anekapārthivavinodaprekṣaṇasamaya  eva// bahutarakālaṃ  śarīrasthitir  bhavati    sa evaṃ rājayogaḥ    tasya ete bhedāḥ //   \U1 
% yena rājayogena  anekarājyabhogasamaya   eva// anekapārthivavinodaprekṣyaṇasamaya eva// bahutarakālaṃ  śarīrasthitir  bhavati//  sa eva  rājayogastaisyaite     bhedāḥ //   \U2
% --------------------------
%\textit{Rājayoga} is that by which longterm durability of the body arises even amongst manifold royal pleasures even amongst the manifold royal entertainments and spectacle. This truly is \textit{rājayoga}. Of this [\textit{rājayoga}] these are the varieties: \end{tlate}
%--------------------------
yena rāja\app{\lem[wit={P,N1}, alt={°yogenāneka°}]{yogenāneka}
  \rdg[wit={N2,D,U1,U2}]{°yogena aneka°}
}rājyabhogasamaya eva/ anekapārthivavinoda
      \app{\lem[wit={ceteri}]{prekṣaṇasamaya}
        \rdg[wit={U2}]{prekṣyaṇasamaya}}
      eva/ bahutarakālaṃ śarīrasthitir-bhavati/ sa
      \app{\lem[wit={ceteri}]{eva}
        \rdg[wit={U2}]{evaṃ}}
      \app{\lem[wit={ceteri}]{rājayogaḥ}
        \rdg[wit={U2}]{rājayogas}}/ 
      \app{\lem[wit={P,U2}]{tasyaite}
        \rdg[wit={ceteri}]{tasya ete}} bhedāḥ/
%-------------------------
%
% \om                                                                                                                                                                \E
% \om                                                                                                                                                                \L
% \om                                                                                                                                                                \B
% kriyāyogaḥ 1 jñānayogaḥ 2 caryāyogaḥ 3 haṭhayogaḥ 4 karmayogaḥ 5 layayogaḥ 6 dhyānayogaḥ 7 maṃtrayogaḥ 8 lakṣyayogaḥ 9 vāsanāyogaḥ 10 śivayogaḥ 11 brahmayogaḥ 12 advaitayogaḥ 13 siddhayogaḥ 14 rājayogaḥ 15 ete paṃcadaśayogāḥ \P
%
% kriyāyogaḥ / jñānayogaḥ / caryāyogaḥ / haṭhayogaḥ / karmayogaḥ / layayogaḥ / dhyānayogaḥ / maṃtrayogaḥ / lakṣyayogaḥ / vāsanāyogaḥ / śivayogaḥ / brahmayogaḥ / advaitayogaḥ / rājayogaḥ / siddhayogaḥ / ete paṃcadaśayogāḥ // \N1
%
% kriyāyogaḥ jñānayogaḥ caryāyogaḥ haṭhayogaḥ karmayogaḥ layayogaḥ dhyānayogaḥ maṃtrayogaḥ lakṣayogaḥ vāsanāyogaḥ śivayogaḥ brahmayogaḥ advaitayogaḥ rājayogaḥ siddhayogaḥ // ete paṃcadaśayogāḥ // \N2      
%      
% kriyāyogaḥ // jñānayogaḥ // caryāyogaḥ // haṭhayogaḥ // karmayogaḥ // layayogaḥ // dhyānayogaḥ // maṃtrayogaḥ // lakṣyayogaḥ // vāsanāyogaḥ // śivayogaḥ // brahmayogaḥ // advaitayogaḥ // rājayogaḥ // siddhayogaḥ // ete paṃcadaśayogāḥ // \D
% \om                                                                                                                                                                         \D2      
%
% kriyāyogaḥ // jñānayogaḥ // tvaryāyogaḥ // haṭhayogaḥ // karmayogaḥ // layayogaḥ // dhyānayogaḥ maṃtrayogaḥ  lakṣayogaḥ  vāsanāyogaḥ  śivayogaḥ  brahmayogaḥ  advaitayogaḥ  rājayogaḥ  siddhayogaḥ ete paṃcadaśayogāḥ  \U1
%
% kriyāyogaḥ // jñānayogaḥ // caryāyogaḥ // haṭhayogaḥ // karmayogaḥ // nayayogaḥ // dhyānayogaḥ // maṃtrayogaḥ // lakṣyayogaḥ // vāsanāyogaḥ // śivayogaḥ // brahmayogaḥ // advaitayogaḥ // siddhayogaḥ // rājayogaḥ // evaṃ paṃcadaśāyogā bhavaṃti // \U2
%-------------------------
         kriyāyogaḥ 1\dd{}
         jñānayogaḥ 2\dd{}
         \app{\lem[wit={ceteri}]{caryāyogaḥ}
          \rdg[wit={U1}]{tvaryāyogaḥ}} 3\dd{}
        haṭhayogaḥ 4\dd{}
        karmayogaḥ 5\dd{}
        \app{\lem[wit={ceteri}]{layayogaḥ}
          \rdg[wit={U2}]{nayayogaḥ}} 6\dd{}
        dhyānayogaḥ 7\dd{}
        mantrayogaḥ 8\dd{}
        \app{\lem[wit={ceteri}]{lakṣyayogaḥ}
          \rdg[wit={U1}]{lakṣayogaḥ}} 9\dd{}
        vāsanāyogaḥ 10\dd{}
        śivayogaḥ 11\dd{} 
        brahmayogaḥ 12\dd{}
        advaitayogaḥ 13\dd{} 
        \app{\lem[wit={P,U2}]{siddhayogaḥ}
          \rdg[wit={N1,N2,D,U1}]{rājayogaḥ}} 14\dd{}
        \app{\lem[wit={P,U2}]{rājayogaḥ}
          \rdg[wit={ceteri}]{siddhayogaḥ}} 15\dd{}     
        \note[type=philcomm, labelb=1, lem={rājayoga}]{The initial codification of 15 \textit{yoga}s appears in N\textsubscript{1}, N\textsubscript{2}, P, D, U\textsubscript{1} and U\textsubscript{2}. It is ommitted in E and L. B can't be determined due to missing folios. It is also missing in the Ysg.}
        \note[type=source, labelb=2, lem={pañcadaśāyogā}]{Ysv (PT): pañcadaśaprakāro 'yaṃ rājayogaḥ || kriyāyogo jñānayogaḥ karmayogo haṭhas tathā | dhyānayogo mantrayoga urayogaś ca vāsanā |  rājaty etad brahmavaśīva ebhiś ca pañcadaśadhā | idānīṃ lakṣaṇañ caiṣāṃ kathayāmi śṛṇu priye |}
        \note[type=testium, labelb=3, lem={pañcadaśāyogā}]{YSC: ity ādinā 'mnātāni | tatra nididhyāsanaṃ pradhānam | tat sahakṛtād eva manaso 'laukikā 'bādhitātmagocara pramāsambhavāt sarvavijñānādirūpaphalasaṃvādāc ca | nididhyāsanañcaika tānatādirūpo rājayogāparaparyāyaḥ samādhiḥ | tatsādhanaṃ tu kriyāyogaḥ, caryāyogaḥ, karmayogo, haṭhayogo, mantrayogo, jñānayogaḥ, advaitayogo, lakṣyayogo, brahmayogaḥ, śivayogaḥ, siddhiyogo, vāsanāyogo, layayogo, dhyānayogaḥ, premabhaktiyogaś ca |}
        \app{\lem[wit={P,N1,D,U1}]{ete pañcadaśayogāḥ}
          \rdg[wit={U2}]{evaṃ paṃcadaśāyogā bhavaṃti}}\dd{}
      \end{prose}
    \end{ekdosis}
    %%%%%%%%%%%%
    %%%%%%%%%%%%
    %%%%%%%%%%%
    %%%%%%%%%%%%%
    %%%%%%%%%%%%
\begin{ekdosis}
  \ekddiv{type=ed}
        \bigskip
        \centerline{\textrm{\small{[Kriyāyoga]}}}
        \bigskip
%--------------------------        
% \om                                      \E
% \om                                      \L
% \om                                      \B
% idānīṃ kriyāyogasya lakṣaṇaṃ kathyate/   \P
% idānīṃ kriyāyogasya lakṣaṇaṃ kathyate/   \N1
% idānī  kriyāyogasya lakṣaṇaṃ kathyate//  \N2
% idānīṃ kriyāyogasya lakṣaṇaṃ kathayate/  \D
% \om                                      \D2
% idānīṃ kriyāyogasya lakṣaṇaṃ kathyate/   \U1
% atha   kriyāyogas   lakṣaṇaṃ          // \U2
%--------------------------
%Now the characteristic of the Yoga of [mental] action (\textit{kriyāyoga}) described.
%--------------------------
 \begin{prose}
        \app{\lem[wit={ceteri}]{idānīṃ}
            \rdg[wit={N2}]{idānī}
            \rdg[wit={U2}]{atha}}
          \app{\lem[wit={ceteri}]{kriyāyogasya}
            \rdg[wit={U2}]{kriyāyogas}} lakṣaṇaṃ
          \app{\lem[wit={ceteri}]{kathyate}
            \rdg[wit={D}]{kathayate}
            \rdg[wit={U2}]{\om}}/
\end{prose}
 \ekddiv{type=ed}
 \begin{tlg}
%--------------------------   
% \om                                                    \E
% \om                                                    \L
% \om                                                    \B
% kriyāmuktir    ayaṃ yogaḥ    svapiṇḍe siddhidāyakaḥ    \P
% kriyāmuktir    ayaṃ yogaḥ /  svapiṇḍe siddhidāyakaḥ /  \N1
% kriyāmukti    layaṃ yogaḥ    svapiṇḍe siddhidāyakaḥ /  \N2
% kriyāmuktir    ayaṃ yogaḥ    svapiṇḍe siddhidāyakaḥ /  \D
% \om                                                    \D2
% kriyāyuktir    ayaṃ yogaḥ /  svapiṇḍe siddhidāyakaḥ /  \U1
% kriyāmuktiḥ // ayaṃ yogaḥ    svapiṇḍe siddhidāyakaṃ // \U2 
%--------------------------
%This Yoga is liberation through [mental] action, it bestows success(\textit{siddhi}) in ones own body.
%-------------------------- 
   \tl{\note[type=source, labelb=4, lem=kriyāmuktir]{Ysv (PT): kriyāmuktimayo [kriyāmuktir ayaṃ (YK)] yogaḥ sapiṇḍisiddhidāyakaḥ [sapiṇḍe (YK)] | yatkāromīti saṅkalpaṃ kāryārambhe manaḥ sadā ||}
     \app{\lem[wit={ceteri}, alt={kriyāmuktir}]{kriyāmukti\skp{r-a}}
    \rdg[wit={N2}]{kriyāmukti}
    \rdg[wit={U2}]{kriyāmuktiḥ ||}
}\app{\lem[wit={ceteri}, alt={ayaṃ}]{\skm{r-a}yaṃ}
  \rdg[wit={N2}]{layaṃ}}
\app{\lem[wit={ceteri}]{yogaḥ}
  \rdg[wit={N1,U1}]{yogaḥ |}} svapiṇḍe
\app{\lem[wit={ceteri}]{siddhidāyakaḥ}
  \rdg[wit={U2}]{siddhidāyakaṃ}}/}\\
%-------------------------
% \om                                                    \E
% \om                                                    \L
% \om                                                    \B
% yaṃ yaṃ karoti kallolaṃ kāryāraṃbhe manaḥ sadā         \P
% yaṃ yaṃ karoti kallolaṃ kāryāraṃbhe manaḥ sadā/        \N1
% yaṃ yaṃ karoti kallolaṃ kāryāraṃbhe manaḥ sadā//1//    \N2
% yaṃ yaṃ karoti kallolaṃ kāryāraṃbhe manaḥ sadā/        \D
% \om                                                    \D2
% yaṃ yaṃ karoti kallolaṃ kāryāraṃbhe manaḥ sadā/ 1      \U1
% yaṃ yaṃ karoti kallolaṃ kāryāraṃbhe manaḥ sadā/        \U2
%--------------------------
%Each wave the mind creates at the beginning of an action,
%-------------------------- 
\tl{yaṃ yaṃ karoti kallolaṃ kāryāraṃbhe manaḥ sadā/}\\
%--------------------------
% \om                                                        \E
% \om                                                        \L
% \om                                                        \B
% tattataḥ   kuñcanaṃ kurvan kriyāyogas tato bhavet            \P
% tattataḥ   kuñcanaṃ kurvan kriyāyogas ato bhava     //       \N1
% tattataḥ   kūrcanaṃ kurvan kriyāyogas ato bhava     //       \N2
% tattataḥ   kuñcanaṃ kurvan kriyāyogas ato bhava     //       \D
% \om                                                          \D2
% taṃkṛ taṃ  kuñcanaṃ kurvan kriyāyogas ato ?va      //1//    \U1
% tatastataḥ kuṃcanaṃ kurvan kriyāyogas tato bhavet //1//     \U2
%--------------------------
%of all those one shall withdraw oneself. Then \textit{kriyāyoga} arises.
%--------------------------
\tl{\note[type=source, labelb=5, lem=tattataḥ]{Ysv (PT=YK): tatsāṅgācaraṇaṃ kurvan kriyāyogarato bhavet |}
  \app{\lem[wit={ceteri}]{tattataḥ}
    \rdg[wit={U2}]{tatas tataḥ}
    \rdg[wit={U1}]{taṃkṛ taṃ}}
  \app{\lem[wit={ceteri}]{kuñcanaṃ}
    \rdg[wit={N2}]{kūrcanaṃ}}
  kurvan-kriyāyoga\skp{s-ta}\app{\lem[wit={P,U2}, alt={tato bhavet}]{\skm{s-t}ato bhavet}
    \rdg[wit={N1,N2,D}]{ato bhava}
    \rdg[wit={U1}]{ato va}}\dd{}1\hskip-2pt\dd{}}
\end{tlg}
\end{ekdosis}
\ekdpb*{}
%%%%%%%%%%%%%%%%%%%%%%%%%%%%%%%%%%%%%%%%%%
%%%%%%%%%%%%%%%%%%%%%%%%%%%%%%%%%%%%%%%%%%
%%%%%%%%PAGEBREAK%%%%%%%PAGEBREAK%%%%%%%%%
%%%%%%%%%%%%%%%%%%%%%%%%%%%%%%%%%%%%%%%%%%
%%%%%%%%%%%%%%%%PAGEBREAK%%%%%%%%%%%%%%%%%
%%%%%%%%%%%%%%%%%%%%%%%%%%%%%%%%%%%%%%%%%%
%%%%%%%%PAGEBREAK%%%%%%%PAGEBREAK%%%%%%%%%
%%%%%%%%%%%%%%%%%%%%%%%%%%%%%%%%%%%%%%%%%%
%%%%%%%%%%%%%%%%%%%%%%%%%%%%%%%%%%%%%%%%%%
%%%%%%%%%%%%%%%%%%%%%%%%%%%%%%%%%%%%%%%%%%
%%%%%%%%%%%%%%%%%%%%%%%%%%%%%%%%%%%%%%%%%%
%%%%%%%%PAGEBREAK%%%%%%%PAGEBREAK%%%%%%%%%
%%%%%%%%%%%%%%%%%%%%%%%%%%%%%%%%%%%%%%%%%%
%%%%%%%%%%%%%%%%PAGEBREAK%%%%%%%%%%%%%%%%%
%%%%%%%%%%%%%%%%%%%%%%%%%%%%%%%%%%%%%%%%%%
%%%%%%%%PAGEBREAK%%%%%%%PAGEBREAK%%%%%%%%%
%%%%%%%%%%%%%%%%%%%%%%%%%%%%%%%%%%%%%%%%%%
%%%%%%%%%%%%%%%%%%%%%%%%%%%%%%%%%%%%%%%%%%
%%%%%%%%%%%%%%%%%%%%%%%%%%%%%%%%%%%%%%%%%%
%%%%%%%%%%%%%%%%%%%%%%%%%%%%%%%%%%%%%%%%%%
%%%%%%%%PAGEBREAK%%%%%%%PAGEBREAK%%%%%%%%%
%%%%%%%%%%%%%%%%%%%%%%%%%%%%%%%%%%%%%%%%%%
%%%%%%%%%%%%%%%%PAGEBREAK%%%%%%%%%%%%%%%%%
%%%%%%%%%%%%%%%%%%%%%%%%%%%%%%%%%%%%%%%%%%
%%%%%%%%PAGEBREAK%%%%%%%PAGEBREAK%%%%%%%%%
%%%%%%%%%%%%%%%%%%%%%%%%%%%%%%%%%%%%%%%%%%
%%%%%%%%%%%%%%%%%%%%%%%%%%%%%%%%%%%%%%%%%%
\begin{ekdosis}
  \ekddiv{type=ed}
    \begin{tlg}
%--------------------------      
% \om                                                                                                 \B
% \om                                                                                                 \L
% kṣamā vivekaṃ vairāgyaṃ śāntiḥ santoṣaniṣpṛhā       etadyuktiyuto  yogī   kriyāyogī nigadyate       \E
% kṣamāvivekavairāgyaṃ    śāntiḥ santoṣanispṛhāḥ      etadyuktiyuto  yogī   kriyāyogī nigadyate       \P
% kṣamāvivekavairāgyaṃ    śāntiḥ santoṣanispṛhā       etat yuktiyuto yogī   kriyāyogī nigadyate       \N1
% kṣamāvivekavairāgyaṃ    śāntiḥ santoṣanispṛhā //2// etat yuktiyuto yo sau kriyāyogī nigadyate//     \N2
% kṣamāvivekavairāgyaṃ    śāntiḥ santoṣanispṛhaḥ      etat yuktiyuto yogī   kriyāyogī nigadyate       \D
% \om                                                                                            \D2
% kṣamāvivekavairāgya---- śāntisantoṣaniḥspṛhī        etad yuktiyuto  yo sau kriyāyogī nigadyate       \U1 
% kṣamā vivekaṃ vairāgyaṃ śāntisaṃtoṣaniṣpṛhāḥ //     etat muktiyuto yogī   kriyāyogī nigadyate //2// \U2
%--------------------------
%Patience, discrimination, equanimity, peace, modesty, desireless: The \textit{yogī} who is endowed with these means is said to be a \textit{kriyāyogī}.
%--------------------------
% The text of the Printed Edition starts here ---> 
%--------------------------
      \tl{\note[type=source, labelb=6, lem=kṣamā°]{Ysv (PT): kṣamāvivekavairāgyaśāntisantoṣanispṛhāḥ | etan muktiyuto yo'sau kriyāyogo nigadyate |}
        \note[type=source, labelb=7, lem={kṣamā°}]{Ysv (YK): kṣamāvivekavairāgyaśāntisantoṣanispṛhāḥ | etan muktiyutaś cāsau kriyāyogī nigadyate || 211 ||}
kṣamā\app{\lem[wit={ceteri}, alt={°viveka°}]{viveka}\rdg[wit={E,U2}]{vivekaṃ}}vairāgyaṃ\note[type=philcomm, labelb=8, lem={°kṣamā°}]{The printed edition E starts here.}śāntisantoṣa\app{\lem[wit={P},alt={°nispṛhāḥ}]{nispṛhāḥ}
          \rdg[wit={U2}]{°niṣpṛhāḥ ||}
          \rdg[wit={E,N1}]{°nispṛhā}
          \rdg[wit={N2}]{°niṣpṛhā ||2||}
          \rdg[wit={D}]{°nispṛhaḥ}
          \rdg[wit={U1}]{°niṣpṛhī}}/}\\
      \tl{\app{\lem[wit={E,P,U1},alt={etad}]{eta\skp{d-yu}}
          \rdg[wit={N1,N2,D,U2}]{etat}
}\app{\lem[wit={ceteri}, alt={yuktiyuto}]{\skm{d-yu}ktiyuto}  %%%SANDHI
    \rdg[wit={U2}]{muktiyuto}}
  \app{\lem[wit={E,P,N1,D,U2}]{yogī}    
    \rdg[wit={N2,U1}]{yo sau}}
kriyāyogī nigadyate\dd{}2\hskip-2pt\dd{}}
\end{tlg}
       \ekddiv{type=ed}
     \begin{tlg}
%-----------------------
% \om                                                \B
% \om                                                \L
% mātsaryaṃ mamatā māyā hiṃsā ca   madagarvitā /     \E
% mātsarya  mamatā māyā hiṃsāśā    madagarvitāḥ      \P
% mātsarya  mamatā māyā hiṃsāḥ //  madagarvatā /     \N1    -> the hiṃsā---''ḥ//'' in \nepal looks like a śā -> indicator that the others copied from \nepal? 
% mātsarya  mamatā māyā hiṃsāśā    madagārvatā //3// \N2
% mātsarya  mamatā māyā hiṃsāśā    madagarvatā /     \D
% \om                                                \D2
% mātsaryaṃ mamatā māyā hiṃsāśā    madagarvatā /     \U1
% mātsaryaṃ mamatā māyā hiṃsāśā    madagarvatā /     \U2
%-----------------------
%Envy, selfishness, cheating, violence, desire and intoxication, pride,
%-----------------------
       \tl{\note[type=source, labelb=9, lem=mātsaryaṃ]{Ysv (PT): mātsaryaṃ mamatā māyā hiṃsā ca madagarvitā | kāmaḥ krodho bhayaṃ lajjā lobho mohas tathā 'śuciḥ [śuciḥ (YK)] ||}
         \app{\lem[wit={E,U1,U2}]{mātsaryaṃ}
           \rdg[wit={P,N1,D}]{mātsarya}}
         mamatā māyā
         \app{\lem[wit={E}]{hiṃsā ca}
           \rdg[wit={ceteri}]{hiṃsāśā}
           \rdg[wit={E}]{hiṃsā ca}
           \rdg[wit={N1}]{hiṃsāḥ ||}}
         madagarvatā/}\\
%-----------------------
% \om                                                   \B
% \om                                                   \L
% kāmakrodhabhayaṃ   lajjā lobhamohau tathā śuciḥ //    \E
% kāmakrodhabhayaṃ   lajjā lobhamohau tathā 'śuciḥ      \P
% kāmakrodhabhayaṃ   lajjā lobhamohau tathā 'śuciḥ /    \N1    -> the hiṃsā---''ḥ//'' in \nepal looks like a śā -> indicator that the others copied from \nepal? 
% kāmakrodho bhayaṃ  lajjā lobhamohau tathā śuciḥ //    \N2
% kāmakrodho bhayaṃ  lajjā lobhamohau tathā 'śuciḥ //   \D
% \om                                                   \D2
% kāmakrodhau bhayaṃ lajjā lobhamohau tathā 'śuciḥ      \U1
% kāmakrodhau bhayaṃ lajjā lobhamohau tathā śuciḥ //3// \U2
% -----------------------
% lust, anger, fear, laziness, greed, error and impurity.
%-----------------------
       \tl{kāma\app{\lem[wit={U1,U2}, alt={°krodhau}]{krodhau}
           \rdg[wit={E,P,N1}]{krodha°}
           \rdg[wit={D}]{°krodho}}
         bhayaṃ lajjā lobhamohau tathā
         \app{\lem[wit={ceteri}]{'śuciḥ}
           \rdg[wit={E,N2,U2}]{śuciḥ}}\dd{}3\hskip-2pt\dd{}}    %%%AVAGRAHA
\end{tlg}
        \ekddiv{type=ed}
      \begin{tlg}
%-----------------------
%  \om                                                           \B
%  atha dveṣo ghṛṇālasyaṃ bhrāṃtir   daṃbho kṣamā bhramaḥ //     \L
%  rāgadveṣau ghṛṇālasyaṃ bhrāntitvaṃ     mokṣamā bhramaḥ /      \E
%  rāgadveṣau ghṛṇālasyaṃ bhrāṃtir   ddaṃbhokaṣmā bhramaḥ        \P
%  rāgadveṣau ghṛṇālasyaṃ bhrāṃtir   daṃbho kṣamā bhramaḥ //4//  \N1
%  rāgadveṣau ghṛnālasyaṃ bhrāṃtir   daṃbho kṣamā bhramaḥ //4    \N2
%  rāgadveṣau ghṛṇālasyaṃ bhrāṃtir   debho  kṣamā bhramaḥ //     \D
% \om                                                            \D2
%  rāgadoṣau  ghṛṇālasyaṃ bhrāṃti    daṃbha kṣamī bhramaḥ 4      \U1
%  rāgadveṣau ghṛṇālasyaṃ bhrāṃtir   daṃbho kṣamā bhramaḥ //     \U2
%-----------------------
%Attachment and aversion, indignation and idleness, impatience and dizzyness
%-----------------------
        \tl{\note[type=source, labelb=10, lem=rāgadveṣau]{Ysv (PT): rāgadveṣau ghṛṇālasyaśrāntidambhakṣamābhramāḥ [ghṛṇālasyaṃ bhrāntir dambho 'kṣamā bhramaḥ (YK)] | yasyaitāni na vidyante kriyāyogī sa ucyate ||}
          \app{\lem[wit={ceteri}]{rāgadveṣau}
            \rdg[wit={U1}]{rāgadoṣau}
            \rdg[wit={L}]{athadveṣo}}\note[type=philcomm, labelb=11, lem={rāga°}]{L starts here.}
          \app{\lem[wit={ceteri},alt={ghṛṇā°}]{ghṛṇā}
            \rdg[wit={N2}]{ghṛnā°}}lasyaṃ 
          \app{\lem[wit={ceteri}, alt={bhraṃtir daṃbho}]{bhrantir-daṃbho}
            \rdg[wit={D}]{bhrāṃtir debho}
            \rdg[wit={E}]{bhrāntitvaṃ}
            \rdg[wit={U1}]{bhrāṃti daṃbha°}}
          \app{\lem[wit={ceteri}]{kṣamā bhramaḥ}
            \rdg[wit={E}]{mokṣam ābhramaḥ}
            \rdg[wit={U1}]{kṣamī bhramaḥ}}/}\\
%-----------------------
%  \om                                               \B
%  yasyai tāni na vidyaṃte kriyāyogī sa ucyate //    \L
%  yasyai tāni ca vidyante kriyāyogī sa ucyate 3     \E
%  yasyai tāni na vidyaṃte kriyāyogī sa ucyate       \P
%  yasyai tāni na vidyaṃte kriyāyogī sa ucyate //    \N1
%  yasyai tāni na vidyaṃte kriyāyogī sa ucyate //    \N2
%  yasyai tāni na vidyaṃte kriyāyogī sa ucyate //    \D
%  yasyai tāni na vidyaṃte kriyāyogī sa ucyate       \U1
%  yasyai tāni na vidyaṃte kriyāyogī sa ucyate //4// \U2 
%  -----------------------
% Whoever doesn't experience these is called a \textit{kriyāyogī}. 
%  -----------------------        
        \tl{
yasyai tāni \app{\lem[wit={ceteri}]{na}\rdg[wit={E}]{ca}} vidyante kriyāyogī sa ucyate\dd{}4\hskip-2pt\dd{}}\\
      \end{tlg}
     \ekddiv{type=ed}
      \begin{prose}
%-----------------------
%  \om                                                                                          \B
%  yasyāntaḥkaraṇe kṣamāvivekavairāgyaśāntisantoṣādīny                         utpadyante //     \E
%  yasyāṃtaḥkaraṇe kṣamāvivekavairāgyaśāṃtisaṃtoṣa         ityādīny            utpādyaṃte        \P
%  tasyāṃtaḥkaraṇe kṣamāvivekavairāgyaśāṃtisaṃtoṣa         ityādīnotpādyaṃte                    \L
%  yasyāṃtaḥkaraṇe kṣamāḥ vivekavairāgya /    śāṃtisaṃtoṣa ityādīni            utpādyaṃte        \N1
%  yasyāṃtaḥkaraṇe kṣamā' vivekavairāgyā      śāṃtisaṃtoṣa ityādīni            utpādyaṃte /      \N2 %see Mss p3 recto vierte Zeile von unten  
%  yasyāṃtaḥkaraṇe kṣamā // vivekavairāgya // śāṃtisaṃtoṣa ityādīni            utpādyaṃte //     \D
%  yasyāṃtaḥkaraṇe kṣamāvivekavairāgyaśāṃtisaṃtoṣa         ityādīna niraṃtaram utyaṃte        \U1
%  yasyāṃtaḥkaraṇe kṣamāvivekavairāgyaśāṃtisaṃtoṣa         ityādayo niraṃtaraṃ utpādyaṃte       \U2
%  -----------------------
%  Patience, discrimination, equanimity, peace, contentment etc. are generated in his mind.
%  -----------------------        
        yasyāntaḥkaraṇe
        \app{\lem[wit={ceteri},alt={kṣamā°}]{kṣamā}
          \rdg[wit={N1}]{kṣamāḥ}
          \rdg[wit={N2}]{kṣamā'}
        }\app{\lem[wit={ceteri}]{vivekavairāgyaśānti}
          \rdg[wit={N1}]{kṣamāḥ vivekavairāgya | śāṃti°}
          \rdg[wit={N2}]{°vairāgyāśānti°}
          \rdg[wit={D}]{kṣamā || vivekavairāgya || śāṃti°}
        }\app{\lem[wit={ceteri}, alt={°santoṣa ityādīny}]{santoṣa ityādī\skp{ny-u}} %the°-problem
          \rdg[wit={E}]{°santoṣādīny}
          \rdg[wit={L}]{°santoṣa ity ādīno°}
          \rdg[wit={U1}]{°santoṣa ity ādīna niraṃtaram}
          \rdg[wit={U2}]{°santoṣa ity ādayo niraṃtaraṃ}
        }\app{\lem[wit={ceteri}]{\skm{ny-u}tpādyante}
          \rdg[wit={E}]{utpadyante}
          \rdg[wit={L}]{°tpādyaṃte}
          \rdg[wit={U1}]{utyaṃte}}/
%-----------------------
% \om \oxford
%  sa eva bahukriyāyogī kathyate /      \E
%  sa eva bahukriyāyogī kathyate        \P
%  sa eva bahukriyāyogī kathyate //     \L
%  sa eva bahukriyāyogī kathyate /      \N1
%  sa eva bahukriyāyogī sa kathyate /   \N2
%  sa eva bahukriyāyogā sa kathyate //  \D
%  sa eva bahukriyāyogī kathyate /      \U1
%  sa eva bahukriyāyogī tkacyate /      \U2
%-----------------------
% He alone is called a \textit{yogī} of many actions (\textit{bahukriyāyogī}).
%-----------------------
        sa eva
        \app{\lem[wit={ceteri}]{bahukriyāyogī}
          \rdg[wit={D}]{bahukriyāyogā}}
        \app{\lem[wit={ceteri}]{kathyate}
          \rdg[wit={D,N2}]{sa kathyate}
          \rdg[wit={U2}]{tkacyate}}/\\
%-----------------------
% \om \B
%               kāpaṭyaṃ      vittaṃ   hiṃsā    tṛṣṇā    mātsaryam    ahaṃkāraḥ    roṣaḥ kṣayaṃ    lajjā lobhamohā      aśucitvaṃ                       pākhaṃḍatvaṃ       bhrāntiḥ indriyavikāraḥ kāmaḥ          ete yasya manasi pratidinaṃ vyunā bhavanti /    \E
%               kāpaṭyaṃ      vittaṃ   hiṃsā    tṛṣṇā    mātsaryaṃ    ahaṃkāraḥ    roṣo bhayaṃ     lajjā lobhaḥ mohaḥ   aśucitvaṃ rāgaḥ dveṣaḥ   ālasyaṃ pākhaṃḍitvaṃ       bhrāṃtiḥ indriyaṃ vikāraḥ kāmaḥ        ete yasya manasi pratidinaṃ nyunā bhavanti     \P
%               kāpayaṃ     //vitaṃ // hiṃsā // tṛṣṇā // mātsaryaṃ // ahaṃkāraḥ // roṣo bhayaṃ //  lajjā lobhaḥ // moha aśucitvaṃ // rājadveṣa  alasyaṃ // pākhaṃḍitvaṃ // bhrāṃtiḥ // itivikāraḥ // kāmaḥ        eta yasya manasi pratidinaṃ nyunā bhavaṃti//    \L
%yasyāṃtakaraṇe kapatyaṃ māyā vitvaṃ   hiṃsā    tṛṣṇā    mātsaryaṃ    ahaṃkāraḥ    roṣo bhayaṃ     lajjā // lobhamohā   asucitvaṃ rāgadveṣaḥ // alasyaṃ pāṣaṃḍitvaṃ      bhraṃtiḥ / iṃdriyaivikāraḥ / kāmaḥ       ete yasya manasi pratidinaṃ nyunā bhavaīti/     \N1
%               kāpaṭyaṃ māyā vitvaṃ   hiṃsā    tṛṣṇā    mātsaryaṃ    ahaṃkāraḥ    e?ṣo bhayaṃ     lajjā/ lobhamoha     asūcitvaṃ rāgadveṣaḥ    ālasyaṃ pārṣaḍitvaṃ        bhrāṃtiḥ iṃdriyavikāraḥ // kāma         ete yasya manasi pratidinaṃ nyunā bhavaṃti //  \N2      
%               kāpaṭyaṃ māya vitvaṃ   hiṃsā    tṛṣṇā    mātsarya     ahaṃkāraḥ    roṣo bhayaṃ     lajjā // lobhamohā   asucitvaṃ rāgadveṣaḥ // ālasyaṃ pāṣaṃḍitvaṃ        bhraṃtiḥ // iṃdriyavikāraḥ // kāmaḥ // ete yasya manasi pratidinaṃ nyunā bhavaṃti //   \D
%               kāpachaṃ yāya vitvaṃ   hiṃsā    tṛṣṇā    mātsarya     ahaṃkāraḥ    roṣaḥ bhayaṃ    lajā     lobhamohā   aśucitvaṃ rāgadveṣaḥ    ālasyaṃ pākhaṃḍitvaṃ       bhraṃtiḥ iṃdriyavīkāraḥ    kāmaḥ       rāte yasya manasi pratidinaṃ nyunā bhavaṃti //  \U1
%               kāpaṭyaṃ pāpā titaṃ    hiṃsā    tṛṣṇā    mātsaryaṃ // ahaṃkāraḥ    roṣo bhayaṃ     lajjā ----mohā       aśucitvaṃ rāgadveṣaḥ    ālasyaṃ pākhaṃḍitvaṃ //    bhraṃtiḥ iṃdriyavikāraḥ //-----        etate yasya manasi pratidinaṃ nyunā bhavaṃti // \U2
%-----------------------
%Fraud, illusion, property, violence, craving, envy, ego, anger, anxiety, shame, greed, error, impurity, attachment, aversion, idleness, heterodoxy, false view, affection of the senses, sexual desire: He who diminishes these from day to day in is mind,
%-----------------------              
\note[type=testium, labelb=12, lem={lobhaḥ}]{Ysg: lobhamohau aśucitvaṃ rāgadveṣau ālasyaṃ pāṣaṃḍitvaṃ bhrāṃtiḥ iṃdryiavikāraḥ kāmaḥ ete yasya pratidinaṃ nyunā bhavaṃti}
        \app{\lem[wit={ceteri}]{kāpaṭyaṃ}
        \rdg[wit={N1}]{yasyāntaḥkaraṇe kapatyaṃ}
        \rdg[wit={L}]{kāpayaṃ}
        \rdg[wit={U1}]{kāpachaṃ}}\dd{}
      \app{\lem[wit={N1,N2}]{māyā}
        \rdg[wit={D}]{māya}
        \rdg[wit={U1}]{yāya}
        \rdg[wit={U2}]{pāpa}
        \rdg[wit={E,P,L}]{\om}}\dd{}
        %\rdg[wit={E,P,L}]{\textbf{omitted in}}}
      \app{\lem[wit={E,P}]{vittaṃ}
        \rdg[wit={L}]{vitaṃ}
        \rdg[wit={N1,N2,D,U1}]{vitvaṃ}
        \rdg[wit={U2}]{titaṃ}}\dd{}
      hiṃsā\dd{}
      tṛṣṇā\dd{}
      \app{\lem[wit={ceteri}]{mātsaryaṃ}
        \rdg[wit={E}]{mātsaryam}
        \rdg[wit={D,U1}]{mātsarya}}\dd{}
      ahaṃkāraḥ\dd{}
      \app{\lem[wit={E,U1}]{roṣaḥ}
        \rdg[wit={ceteri}]{roṣo}
        \rdg[wit={N2}]{eṣo}}\dd{}
      \app{\lem[wit={ceteri}]{bhayaṃ}
        \rdg[wit={E}]{kṣayaṃ}}\dd{}
      \app{\lem[wit={ceteri}]{lajjā}
        \rdg[wit={U1}]{lajā}}\dd{}
      \app{\lem[wit={P,L}]{lobhaḥ}
        \rdg[wit={ceteri}]{lobha°}
        \rdg[wit={U2}]{\om}}\dd{}
      \app{\lem[wit={P}]{mohaḥ}
        \rdg[wit={L,N2}]{moha}
        \rdg[wit={ceteri}]{mohā}}\dd{}        
      \app{\lem[wit={ceteri}]{aśucitvaṃ}  %%%Frage: vor daṇḍa wird m zu ṃ??? 
        \rdg[wit={N1,D}]{aśucitvaṃ}
        \rdg[wit={N2}]{aśūcitvaṃ}}\dd{}
      \app{\lem[wit={P}]{rāgaḥ}
        \rdg[wit={ceteri}]{rāga°}
        \rdg[wit={L}]{rāja°}
        \rdg[wit={E}]{\om}}\dd{}
      \app{\lem[wit={ceteri}]{dveṣaḥ}
        \rdg[wit={L}]{dveṣa}
        \rdg[wit={E}]{\om}}\dd{}
      \app{\lem[wit={ceteri}]{ālasyaṃ}
        \rdg[wit={E}]{\om}}\dd{}
      \app{\lem[wit={ceteri}]{pākhaṃḍitvaṃ}
        \rdg[wit={D,N1}]{pāṣaṃḍitvaṃ}
        \rdg[wit={E}]{pākhaṃḍatvaṃ}
        \rdg[wit={N2}]{pārṣaḍitvaṃ}}\dd{}
     bhrāntiḥ\dd{}
     \app{\lem[wit={ceteri}]{indriyavikāraḥ}
        \rdg[wit={U1}]{iṃdriyavīkāraḥ}
        \rdg[wit={P}]{iṃdriyaṃ vīkāraḥ}
        \rdg[wit={L}]{itivikāraḥ}}\dd{}
      \app{\lem[wit={ceteri}]{kāmaḥ}
        \rdg[wit={N2}]{kāma}
        \rdg[wit={U2}]{\om}}\dd{}
      \app{\lem[wit={ceteri}]{ete}
        \rdg[wit={L}]{eta}
        \rdg[wit={U1}]{rāte}
        \rdg[wit={U2}]{etate}}
      yasya manasi pradidinaṃ nyūna
      \app{\lem[wit={ceteri}]{bhavanti}
        \rdg[wit={N1}]{bhavaīti}}/ 
%-----------------------       
%sa eva bahukriyāyogī kathyate// \E
%sa eva bahukriyāyogī kathyate// \P
%sa eva bahukriyāyogī kathyate// \L
%sa eva bahukriyāyogī kathyate// \N1
%sa eva bahukriyāyogī kathyate// \N2
%sa eva bahukiyāyogī  kathyate//  \D
%sa eva bahukiyāyogī  kathyaṃte// \U1
%sa eva bahukiyāyogī  kathyaṃte// \U2
%-----------------------
%he alone is called a yogī of many actions (\textit{bahukriyāyogī})
%-----------------------
      \note[type=testium, labelb=13, lem={bahukriyāyogī}]{Ysg: sa eva kriyāyogī kathyate ||}
                    \note[type=philcomm, labelb=14, lem={bahukriyāyogī}]{The term \textit{bahukriyāyogī} currently seems to be unique in Sanskrit literature. The elaborations of Rāmacandra on Kriyāyoga after the quotes of the Ysv are either taken from an unknown source or his own creation.}
sa eva \app{\lem[wit={ceteri}]{bahukriyāyogī}
  \rdg[wit={D,U1,U2}]{bahukiyāyogī}}
      \app{\lem[wit={ceteri}]{kathyate}
        \rdg[wit={U1,U2}]{kathyaṃte}}\dd{}
    \end{prose}
  \end{ekdosis}
\ekdpb*{}
%%%%%%%%%%%%%%%%%%%%%%%%%%%%%%%%%%%%%%%%%%
%%%%%%%%%%%%%%%%%%%%%%%%%%%%%%%%%%%%%%%%%%
%%%%%%%%PAGEBREAK%%%%%%%PAGEBREAK%%%%%%%%%
%%%%%%%%%%%%%%%%%%%%%%%%%%%%%%%%%%%%%%%%%%
%%%%%%%%%%%%%%%%PAGEBREAK%%%%%%%%%%%%%%%%%
%%%%%%%%%%%%%%%%%%%%%%%%%%%%%%%%%%%%%%%%%%
%%%%%%%%PAGEBREAK%%%%%%%PAGEBREAK%%%%%%%%%
%%%%%%%%%%%%%%%%%%%%%%%%%%%%%%%%%%%%%%%%%%
%%%%%%%%%%%%%%%%%%%%%%%%%%%%%%%%%%%%%%%%%%
%%%%%%%%%%%%%%%%%%%%%%%%%%%%%%%%%%%%%%%%%%
%%%%%%%%%%%%%%%%%%%%%%%%%%%%%%%%%%%%%%%%%%
%%%%%%%%PAGEBREAK%%%%%%%PAGEBREAK%%%%%%%%%
%%%%%%%%%%%%%%%%%%%%%%%%%%%%%%%%%%%%%%%%%%
%%%%%%%%%%%%%%%%PAGEBREAK%%%%%%%%%%%%%%%%%
%%%%%%%%%%%%%%%%%%%%%%%%%%%%%%%%%%%%%%%%%%
%%%%%%%%PAGEBREAK%%%%%%%PAGEBREAK%%%%%%%%%
%%%%%%%%%%%%%%%%%%%%%%%%%%%%%%%%%%%%%%%%%%
%%%%%%%%%%%%%%%%%%%%%%%%%%%%%%%%%%%%%%%%%%
%%%%%%%%%%%%%%%%%%%%%%%%%%%%%%%%%%%%%%%%%%
%%%%%%%%%%%%%%%%%%%%%%%%%%%%%%%%%%%%%%%%%%
%%%%%%%%PAGEBREAK%%%%%%%PAGEBREAK%%%%%%%%%
%%%%%%%%%%%%%%%%%%%%%%%%%%%%%%%%%%%%%%%%%%
%%%%%%%%%%%%%%%%PAGEBREAK%%%%%%%%%%%%%%%%%
%%%%%%%%%%%%%%%%%%%%%%%%%%%%%%%%%%%%%%%%%%
%%%%%%%%PAGEBREAK%%%%%%%PAGEBREAK%%%%%%%%%
%%%%%%%%%%%%%%%%%%%%%%%%%%%%%%%%%%%%%%%%%%
%%%%%%%%%%%%%%%%%%%%%%%%%%%%%%%%%%%%%%%%%%
\begin{ekdosis}
    \ekddiv{type=ed}
    \bigskip
    \centerline{\textrm{\small{[Siddhakuṇḍalinīyoga and Mantrayoga]}}}
    \bigskip
    \begin{prose}
%-----------------------   
% \om                                   \B
%idānīṃ rājayogasya bhedāḥ kathyante // \E
%idānīṃ rājayogasya bhedāḥ kathyaṃte    \P
%idānīṃ rājayogasya bhedāḥ              \L
%idānīṃ rājayogasya bhedāḥ kathyaṃte    \N1
%idānīṃ rājayogasya bhedā  kathyate//    \N2
%idānīṃ rājayogasya bhedāḥ kathyaṃte // \D
% \om                                   \U1
%idānīṃ rājayogasya bhedāḥ kathyaṃte // \U2
%-----------------------
%Now varieties of \textit{rājayoga} will be described.
%-----------------------
      \noindent idānīṃ rājayogasya
      \note[type=testium, labelb=15, lem={rājayogasya}]{Ysg: atha rājayogasya bhedau kathyete ||}
       \app{\lem[wit={ceteri}]{bhedāḥ}
         \rdg[wit={N2}]{bhedā}}
       \app{\lem[wit={ceteri}]{kathyante}
         \rdg[wit={N2}]{kathyate}
         \rdg[wit={L}]{\om}}/\note[type=philcomm, labelb=16, lem={kathyante}]{The whole sentence is \om in U\textsubscript{1}.}     
%-----------------------
%te ke     \E
%te ke     \P
%te ke     \L
%ke te //  \D
%ke te /   \N1
%kriyate// \N2       
%ke te     \U1
%te ke     \U2
%-----------------------
%Which are these?
%-----------------------       
\app{\lem[wit={D,N1,U1}]{ke te}
         \rdg[wit={ceteri}]{te ke}
         \rdg[wit={N2}]{kriyate}}/ 
%-----------------------
%\om                                       \B
%ekaḥ siddhakuṇḍalinīyogaḥ / mantrayogaḥ / \E
%ekaḥ siddhakuṃḍaṃliṃ yogaḥ maṃtrayogaḥ    \P
%ekaḥ siddhakuṇḍalanīyoga /                \L 
%ekaḥ siddhakuṇḍalinīyogaḥ maṃtrayogaḥ /   \N1
%ekaḥ siddhakuṇḍalanīyogaḥ maṃtrayogaḥ //  \N2
%ekaḥ siddhakuṃḍalanīyogaḥ mantrayogaḥ //  \D
%ekaḥ siddhakuṇḍaliniyogaḥ mantrayogaḥ     \U1
%ekaḥ siddhakuṇḍalinīyoga // mantrayogaḥ   \U2
%-----------------------
%One is \textit{siddhakuṇḍalinīyoga} [and one] is \textit{mantrayoga}.       
%-----------------------
\note[type=testium, labelb=17, lem={siddhakuṇḍalinīyogaḥ}]{Ysg: siddhakuṃḍaliyogaḥ mantrayogaś ceti}
ekaḥ
\app{\lem[wit={E,N1}]{siddhakuṇḍalinīyogaḥ |}
   \rdg[wit={U1}]{siddhakuṇḍalinīyogaḥ}
   \rdg[wit={U2}]{siddhakuṇḍalinīyoga ||}
   \rdg[wit={L}]{siddhakuṇḍalanīyoga |}
   \rdg[wit={N2,D}]{siddhakuṃḍalanīyogaḥ}
   \rdg[wit={P}]{siddhakuṃḍaṃliṃ yogaḥ}}
\app{\lem[wit={ceteri}]{mantrayogaḥ}
   \rdg[wit={L}]{\om}}/ \note[type=philcomm, labelb=18, lem={siddhakuṇḍalinīyogaḥ mantrayogaḥ}]{The sudden appearance of the term \textit{mantrayoga} here seems odd: This section that follows doesn't mention the practice of \textit{mantra} at all. It might simply be an early scribal mistake that has been copied by most of the manuscripts. However, all witnesses preserve this reading except L. The sentence that follows confirms the reading of Mantrayoga by the usage of dual forms. Although the YTB follows the Ysv very closely in structure and content, the yoga introduced in the Ysv at this point is \textit{jñānayoga}. The subject of \textit{jñāna} in this context, however, is picked up by the YTB. It is also well in the range of realistic possibilities that already in the text's early transmission folios got lost and confused. This szenario is supported by the diffuse arrangement of the the five types of Lakṣyayoga and the Yogas missing from the list. Currently it seems not possible to fix this issue conclusively.}
       \note[type=source, labelb=19, lem={siddhakuṇḍalinīyogaḥ mantrayogaḥ}]{Ysv (PT): jñānayogaṃ pravakṣyāmi tajjñānī śivatāṃ vrajet | paṭhanāt smaraṇād vyānānmaṇḍanāt brahmasādhakaḥ | tad bhedasyaikasandhānamaṣṭaiśvaryamayo bhavet | tritīrthaṃ yatra nāḍī ca tripuṇyaṃ parameśvari | \ldots eṣo 'sya viśvarūpasya rājayogo mato budhaiḥ | viśeṣaṃ kathayiṣyāmi śṛṇu caikamanāḥ sati |}
%-----------------------
% \om                         \B
%astu rājayogaḥ kathyate /    \E
%amū rājayogau kathyete       \P
%amū rājayogau kathyate //    \L
%amū rājayogau kathyate       \N1
%amū rājayogau kathyate//     \N2  %%%p3verso
%amū rājayogau kathyate //    \D
%amū rājayogau kathyate       \U1
%amū rājayogau kathyaṃte //   \U2
%-----------------------
%These two rājayogas are described [in the following].
%-----------------------
       \app{\lem[wit={ceteri}]{amū}
         \rdg[wit={E}]{astu}}
       \app{\lem[wit={ceteri}]{rājayogau}
         \rdg[wit={E}]{rājayogaḥ}}
       \app{\lem[wit={P}]{kathyete}
         \rdg[wit={ceteri}]{kathyate}
         \rdg[wit={U2}]{kathyaṃte}}/
%-----------------------
% \om                                                              \B
%mūlakandasthāne    ekā tejorūpā    mahānāḍī varttate /            \E
%mūlaṃ kaṃdasthāne  ekā tejorūpā    mahānāḍī varttate              \P
%mūlakaṃdasthāne    ekā tejorūpā    mahānāḍī vartate               \L
%mūlakaṃdasthāne    eka tejorūpā    mahānāḍī varttate /            \N1
%mūlakaṃdasthāne    eka tejorūpā    mahānāḍī varttate /            \N2
%mūlakaṃdasthāne    ekā tejorūpā    mahānāḍī varttate //           \D
%mūlakaṃdasthāne    ekā tejorūpā    mahānāḍī vartate /             \U1
%mūlakaṃdasthāne // ekā tejorūpā // mahānāḍī pravarttate /         \U2
%-----------------------
%At the location of the root-bulb exists one major vessel in the form of energy.
%-----------------------       
\note[type=testium, labelb=20, lem={mūlakanda°}]{Ysg: mūlakandasthāne ekā tejomayā mahānāḍī vartate |}
       \app{\lem[wit={ceteri}]{mūlakandasthāne}
         \rdg[wit={U2}]{mūlakaṃdasthāne ||}
         \rdg[wit={P}]{mūlaṃ kaṃdasthāne}}
       \app{\lem[wit={ceteri}]{ekā}
         \rdg[wit={N1,N2}]{eka}}
       \app{\lem[wit={ceteri}]{tejorūpā}
         \rdg[wit={U2}]{tejorūpā ||}}
       mahānāḍī
       \app{\lem[wit={ceteri}]{vartate}
         \rdg[wit={U2}]{pravartate}}/
       \note[type=source, labelb=21, lem={mūlakanda°}]{Ysv (PT): mūlakande sthale caikā nāḍī tejasvatī parā | gudorddhe sā tribhāgābhūdiḍā nāma śaśiprabhā | śaktirūpāmahānāḍī dhyānāt sarvārthadāyinī | dakṣiṇe 'pi kulākhyeti puṃrūpā sūryavigrahā | madhyabhāge suṣumnākhyā brahmaviṣṇuśivātmikā | śuddhacittena sā vijñā vidyutkoṭisamaprabhā |}
\note[type=source, labelb=22, lem={mūlakanda°}]{Ysv (YK): mūlakandasthale caikā nāḍī tejasvitāparā || 246 || gudordhve sā tridhā bhūyādiḍāvāme śaśiprabhā | śaktirūpā mahānāḍī dhyānātsarvārthadāyinī || 247 || dakṣiṇe piṅgalākhyeti puṃrūpā sūryavigrahā | madhyabhāge suṣumnākhyā brahmaviṣṇuśivātmikā || 248 || śuddhacittena sā vijñā vidyutkoṭisamaprabhā |}
%-----------------------
% \om                                                            \B
%iyam ekanāḍī /  iḍāpiṃgalāsuṣumṇā      etān bhedān prāpnoti /    \E
%iyaṃ ekanāḍī    iḍāpiṃgalāsuṣumṇā      etān bhedān prāpnoti      \P
%trayaṃ kā nāḍī  iḍāpiṃgalāsuṣumnā //   etān bhedān prāpnoti      \L
%iyaṃ ekā nāḍī   iḍāpiṃgalāsuṣumnān /   ete  bhedān prāpnoti      \N1
%iyaṃ ekā nāḍī   iḍāpiṃgalāsuṣumnān//   ete  bhedān prāpnoti/     \N2
%iyaṃ ekā nāḍī   iḍāpiṃgalasuṣumnān //  ete  bhedān prāpnoti      \D    
%iyaṃ ekā nāḍī   iḍāpiṃgalāsuṣumnā      etān bhedān prāpnoti      \U1
%iyaṃ eka nāḍī   iḍāpiṃgalāsuṣumṇā      etān bhegān prāpnoti      \U2
%-----------------------
%This single vessel reaches to these openings which are \textit{iḍā}, \textit{piṅgalā} and \textit{suṣumnā}.
%-----------------------       
\note[type=testium, labelb=23, lem={ekā nāḍī}]{Ysg: iyaṃ iḍāpiṃgalasuṣumnā bhedā tridhā |}
\app{\lem[wit={E},alt={iyam}]{iya\skm{m-e}}
         \rdg[wit={ceteri}]{iyaṃ}
         \rdg[wit={L}]{trayaṃ}}
\app{\lem[wit={ceteri}, alt={ekā}]{\skp{m-e}kā}
         \rdg[wit={E}]{eka |}
         \rdg[wit={P}]{eka}
         \rdg[wit={L}]{kā}}
nāḍī iḍāpiṅgalā\app{\lem[wit={N1,N2,D},alt={°suṣumṇān ||}]{suṣumṇān ||}
         \rdg[wit={L}]{suṣumnā |}
         \rdg[wit={ceteri}]{°suṣumṇā}}
       \app{\lem[wit={ceteri}]{etān}
         \rdg[wit={N1,N2,D}]{ete}}
bhedān prāpnoti/
%-----------------------
%\om                                           \B
%vāmabhāge candrarūpā iḍā nāḍī varttate /      \E
%vāmabhāge caṃdrarūpā iḍā nāḍī varttate        \P
%vāmabhāge caṃdrarūpā iḍā nāḍī varttate //     \L
%vāmabhāge caṃdrarūpā iḍā nāḍī varttate /      \N1
%vāmabhāge caṃdrarūpā iḍā nāḍī varttate //     \N2
%vāmabhāge caṃdrarūpā iḍā nāḍī varttate /      \D
%vāmabhāge caṃdrarūpā iḍā nāḍī vartate         \U1
%vāmabhāge caṃdrarūpā     nāḍī pravarttate //  \U2
%-----------------------
%On the left side is the \textit{iḍā}-channel, being a resemblence of the moon.
%-----------------------        
\note[type=testium, labelb=24, lem={vāma°}]{Ysg: vāmabhāge caṃdrarūpā iḍā}
vāmabhāge candrarūpā
        \app{\lem[wit={ceteri}]{iḍā}
          \rdg[wit={U2}]{\om}}
        \app{\lem[wit={ceteri}]{nāḍī}}
        \app{\lem[wit={ceteri}]{vartate}
          \rdg[wit={U2}]{pravarttate}}/
%-----------------------
% \om                                                \B
%dakṣiṇabhāge  sūryarūpā piṅgalā  nāḍī    varttate /  \E
%dakṣiṇabhāge  sūryarūpā piṃgalā  nāḍī    varttate    \P
%dakṣiṇabhāge  sūryarūpā piṃgalā  nāḍī    varttate // \L
%dakṣiṇabhāge  sūryarūpā piṃgalā  nāḍī    varttate // \N1
%dakṣiṇabhāge  sūryarūpā piṃgalā  nāḍī    varttate/   \N2
%dakṣiṇabhāge  sūryarūpā piṃgalā  nāḍī    varttate // \D       
%dakṣiṇe bhāge sūryarūpā piṃgalā  nāḍī    vartate     \U1
%dakṣiṇabhāge  sūryarūpā piṃgalā  nāḍī pravartate //  \U2
%-----------------------
%On the right side exists the \textit{piṅgalā}-channel, being a resemblence of the sun.        
%-----------------------
\note[type=testium, labelb=25, lem={dakṣiṇa°}]{Ysg: dakṣiṇabhāge sūryarūpā piṃgalā |}
        \app{\lem[wit={ceteri}]{dakṣiṇabhāge}
          \rdg[wit={U1}]{dakṣiṇe bhāge}}
        sūryarūpā piṅgalā nāḍī
        \app{\lem[wit={ceteri}]{vartate}
          \rdg[wit={U2}]{pravarttate}}/
%-----------------------
% \om                                                                   \B
%madhyamārge `tisūkṣmā padminī taṃtusamākārā  koṭividyutsamaprabhā      \E
%madhyamārge `tisūkṣmā padmanī taṃtusamākāra  koṭividyutsamaprabhā      \P
%madhyamārge `tisūkṣmā padmanī taṃtusamākārā  koṭividyutsamaprabhā      \L
%madhyamārge atisūkṣmā padmanī taṃtusamākārā  koṭividyutsamaprabhā //   \N1
%madhyamārge atisūkṣmā padmanī taṃtusamākārā  koṭividyutsamaprabhā //   \N2
%madhyarge   atisūkṣmā padminī taṃtusamākārā  koṭividyutsamaprabhā //   \D
%madhyamārge atisūkṣmā padminī taṃtusamākārā  koṭividyutsamaprabaḥ      \U1
%madhyamārge  tisūkṣmā padminī taṃtusamākārā  koṭividyutsamaprabhā //   \U2
%-----------------------
%Within the middle path is a lotuspond being very subtle. [It is] made from a web of light [and it] shines like a thousand lightnings.
%----------------------- 
\note[type=testium, labelb=26, lem={madhyamārge}]{Ysg: madhyamārge atisūkṣmā visa? taṃtusamākārā koṭividyutprabhā}
        \app{\lem[wit={ceteri}]{madhyamārge}
          \rdg[wit={D}]{madhyarge}}
        \app{\lem[wit={E,P,L,U2}]{'tisūkṣmā}
          \rdg[wit={D,N1,N2,U1}]{atisūkṣmā}}
        \app{\lem[wit={ceteri}]{padminī}
          \rdg[wit={P,L,N1,N2}]{padmanī}}/
        \app{\lem[wit={ceteri}]{tantusamākārā}
          \rdg[wit={P}]{taṃtusamākāra°}}
      koṭividyutsama\app{\lem[wit={ceteri},alt={°prabhā}]{prabhā}
        \rdg[wit={U1}]{°prabhaḥ}}/
      \note[type=testium, labelb=27, lem={madhyamārge}]{SSP 2.26: mūlakandād daṇḍalagnāṃ brahmanāḍīṃ śvetavarṇāṃ brahmarandhraparyantaṃ gatāṃ saṃsmaret | tanmadhye kamalatantunibhāṃ vidyutkoṭiprabhām ūrdhvagāminīṃ tāṃ mūrtiṃ manasā lakṣayet | sarvasiddhipradā bhavati |}
    \end{prose}
  \end{ekdosis}
\ekdpb*{}
%%%%%%%%%%%%%%%%%%%%%%%%%%%%%%%%%%%%%%%%%%
%%%%%%%%%%%%%%%%%%%%%%%%%%%%%%%%%%%%%%%%%%
%%%%%%%%PAGEBREAK%%%%%%%PAGEBREAK%%%%%%%%%
%%%%%%%%%%%%%%%%%%%%%%%%%%%%%%%%%%%%%%%%%%
%%%%%%%%%%%%%%%%PAGEBREAK%%%%%%%%%%%%%%%%%
%%%%%%%%%%%%%%%%%%%%%%%%%%%%%%%%%%%%%%%%%%
%%%%%%%%PAGEBREAK%%%%%%%PAGEBREAK%%%%%%%%%
%%%%%%%%%%%%%%%%%%%%%%%%%%%%%%%%%%%%%%%%%%
%%%%%%%%%%%%%%%%%%%%%%%%%%%%%%%%%%%%%%%%%%
%%%%%%%%%%%%%%%%%%%%%%%%%%%%%%%%%%%%%%%%%%
%%%%%%%%%%%%%%%%%%%%%%%%%%%%%%%%%%%%%%%%%%
%%%%%%%%PAGEBREAK%%%%%%%PAGEBREAK%%%%%%%%%
%%%%%%%%%%%%%%%%%%%%%%%%%%%%%%%%%%%%%%%%%%
%%%%%%%%%%%%%%%%PAGEBREAK%%%%%%%%%%%%%%%%%
%%%%%%%%%%%%%%%%%%%%%%%%%%%%%%%%%%%%%%%%%%
%%%%%%%%PAGEBREAK%%%%%%%PAGEBREAK%%%%%%%%%
%%%%%%%%%%%%%%%%%%%%%%%%%%%%%%%%%%%%%%%%%%
%%%%%%%%%%%%%%%%%%%%%%%%%%%%%%%%%%%%%%%%%%
%%%%%%%%%%%%%%%%%%%%%%%%%%%%%%%%%%%%%%%%%%
%%%%%%%%%%%%%%%%%%%%%%%%%%%%%%%%%%%%%%%%%%
%%%%%%%%PAGEBREAK%%%%%%%PAGEBREAK%%%%%%%%%
%%%%%%%%%%%%%%%%%%%%%%%%%%%%%%%%%%%%%%%%%%
%%%%%%%%%%%%%%%%PAGEBREAK%%%%%%%%%%%%%%%%%
%%%%%%%%%%%%%%%%%%%%%%%%%%%%%%%%%%%%%%%%%%
%%%%%%%%PAGEBREAK%%%%%%%PAGEBREAK%%%%%%%%%
%%%%%%%%%%%%%%%%%%%%%%%%%%%%%%%%%%%%%%%%%%
%%%%%%%%%%%%%%%%%%%%%%%%%%%%%%%%%%%%%%%%%%
\begin{ekdosis}
     \ekddiv{type=ed}
    \centerline{\textrm{\small{[First Cakra]}}}
    \bigskip
    \begin{prose}
%-----------------------
%\om                                                    \B
%idānīṃ suṣumṇāyāṃ jñānotpattāv---upāyāḥ  kathyante      \E
%idānīṃ suṣumṇāyā  jñānotpattau   upāyāḥ  kathyaṃte      \P
%idānīṃ suṣumnā    jñānotpattau   upāyaḥ  kathyate //    \L
%idānīṃ suṣumnāyāḥ jñanotpanno    'pāyāḥ  kathyaṃte //   \N1
%idānīṃ suṣumnāyāḥ jñanotpanno    upāyāḥ  kathyaṃte //   \N2
%idānīṃ suṣumnāyāḥ jñanotpattau   upāyāḥ  kathyaṃte //   \D
%idānīṃ suṣumnāya--jñanotpattau    upāyāḥ kathyaṃte //   \U1
%idānīṃ suṣumṇāyā  jñānotpattau   upāyā   kathyaṃte //   \U2
%-----------------------
\noindent 
      \note[type=testium, labelb=28, lem={upāyāḥ}]{Ysg: atas taj jñotpattāv upāyā ucyaṃte ||}
\note[type=source, labelb=29, lem={mūlacakraṃ}]{Ysv (PT): mūlādhāraṃ catuṣpatraṃ gudorddhe [gudordhve (YK)] varttate mahat | tanmadhye svarṇapīṭhe tu trikoṇaṃ maṇḍalaṃ [trikoṇamaṇḍalaṃ (YK)] param | tatra vahniśikhākārā mūrttiḥ sarvatra siddhidā | asyā dhyānaṃ manomadhye vinā pīṭhena [pāṭhena (YK)] vāṅmayam | sarvaśāstrāṇi saṅkarṣaṃ [saṃkarṣa (PT)] sadā sphurati yogavit |}
idānīṃ  
    \app{\lem[wit={E}]{suṣumṇāyāṃ}
      \rdg[wit={P,U2}]{suṣumṇāyā}
      \rdg[wit={U1}]{suṣumnāya°}
      \rdg[wit={N1,N2,D}]{suṣumṇāyāḥ}
      \rdg[wit={L}]{suṣumnā°}}
    \app{\lem[wit={E}, alt={jñānotpattāv upāyāḥ}]{jñānotpattāv-upāyāḥ}
      \rdg[wit={ceteri}]{jñānotpattau upāyāḥ}
      \rdg[wit={U2}]{jñānotpattau upāyā}
      \rdg[wit={N1,N2}]{jñānotpanno 'pāyāḥ}}
    \app{\lem[wit={E,P,N1,N2,D,U1,U2}]{kathyante}
      \rdg[wit={L}]{kathyate}}\dd{}
%-----------------------
%\om                                            \B
%ādau caturdalaṃ mūlaṃ cakraṃ varttate /        \E
%ādau caturddalaṃ mūlaṃ cakraṃ varttate /       \P
%ādau caturdalamūlacakraṃ varttate //           \L
%ādau caturdalaṃ mūlacakraṃ varttate            \N1
%ādau prathamacaturdalamūlacakraṃ pravarttate// \N2      
%ādau caturdalaṃ mūlacakraṃ varttate            \D
%ādau caturdalaṃ mūlaṃ cakraṃ vartate           \U1
%ādau caturdalaṃ mūlacakraṃ pravarttate //      \U2
%-----------------------
%At the beginning\footnote{Supposedly at the beginning of the central channel.} exists the root-cakra having four petals.     
%-----------------------      
\note[type=testium, labelb=30, lem={mūlacakraṃ}]{Ysg: gudamūlacakraṃ caturdalaṃ |}
ādau
   \app{\lem[wit={N1,D,U2}]{caturdalaṃ mūlacakraṃ}
        \rdg[wit={E,P,U1}]{caturdalaṃ mūlaṃ cakraṃ}
        \rdg[wit={L}]{caturdalamūlacakraṃ}
        \rdg[wit={N2}]{prathamacaturdalamūlacakraṃ}}
      \app{\lem[wit={ceteri}]{vartate}
        \rdg[wit={U2}]{pravartate}}/
%-----------------------
%
%\om                                       \B
%prathamādhāracakraṃ varttate / gudāsthānaṃ    raktavarṇaṃ    gaṇeśadaivataṃ    siddhibuddhiśaktimuṣakavāhanam       kurmaṛṣiḥ /  ākuṃcamudrā /    apānavāyuḥ                                   caturdaleṣu     rajaḥsattvatamomanāṃsi /  vaṃ śaṃ ṣaṃ saṃ    madhyatrikoṇe triśikhāt    tanmadhye trikoṇākāraṃ kāmapīthaṃ varttate//    \E
%prathamaṃ ādhāracakraṃ         gudāsthānaṃ    raktavarṇaṃ    gaṇeśāṃ daivataṃ  siddhibuddhiśaktir mukhako vāhanam   kurmaṛṣiḥ    ākuṃcanamudrā    apānavāyuś-----------------------------------caturddaleṣu    rajaḥsattvatamomanāṃsi    vaṃ śaṃ ṣaṃ saṃ    madhyatrikoṇe triśikhā     tanmadhye trikoṇākāraṃ kāmapīthaṃ varttate //   \P
%prathamaṃ ādhāracakraṃ         gudāsthānaṃ    raktavarṇaṃ    gaṇeśadaivataṃ    siddhibuddhiśaktimuṣako vāhanaṃ //   kurmaṛṣiḥ    ākuṃcanamudrā    apānavāyuḥ                                   caturddaleṣu    rajaḥsattvatamomanāṃsi // vaṃ śaṃ ṣaṃ saṃ    madhyatrikoṇe triśikhā     tanmadhyatrikoṇākāraṃ kāmapīthaṃ vartate        \L
%---------------------------------------------------------------------------------------------------------------------------------------------------------------------------------------------------------------------------------------------------------------------------------------tanmadhyatrikoṇākāraṃ kāmapiṭhaṃ varttate /     \N1
%---------------------------------------------------------------------------------------------------------------------------------------------------------------------------------------------------------------------------------------------------------------------------------------tanmadhye trikoṇākāraṃ kāmapiṭhaṃ varttate /    \N2
%---------------------------------------------------------------------------------------------------------------------------------------------------------------------------------------------------------------------------------------------------------------------------------------tanmadhye trikoṇākāraṃ kāmapiṭhaṃ varttate /    \D
%---------------------------------------------------------------------------------------------------------------------------------------------------------------------------------------------------------------------------------------------------------------------------------------tanmadhye trikoṇākāraṃ kāmapiṭhaṃ varttate /    \U1
%prathamaṃ ādhāracakraṃ         gudāsthānaṃ // raktavarṇaṃ // gaṇeśadaivataṃ // siddhibuddhiśaktiḥ muṣako vāhanaṃ // kurmaṛṣiḥ // ākuṃcanamudrā // apānavāyu // urmīkalā // ojasvinīdhāraṇā // caturddaleṣu // rajaḥsattvatamomanāṃsi //  vaṃ śaṃ ṣaṃ saṃ // madhyatrikoṇe trirekhā //  tanmadhye trikoṇākāraṃ kāmapīthaṃ varttate //   \U2
%-----------------------
%The first cakra of support (\textit{ādhāra}) is at the anus [and] is red-colored. Gaṇeśa is the deity. He is success, intelligence and power. A rat is the mount. The Ṛṣi is Kūrma. The seal is contraction. The vitalwind is \textit{apāna}. The \textit{kalā} is the wave of consciousness (\textit{urmī}). The concentration is ``she who is powerful'' (\textit{ojasvinī})}. In the four petals [of it resides] \textit{rajas}, \textit{sattva}, \textit{tamas} and the mind-faculties (\textit{manāṃsi}), [symbolized by the syllables or \textit{bīja}s] vaṃ śaṃ ṣaṃ and saṃ. A trident is situated in the middle of the triangle\footnote{This passage is odd since a triagle wasn't mentioned before.}
%-----------------------
\note[type=testium, labelb=31, lem={kāmapiṭhaṃ}]{Ysg: tanmadhye trikoṇākāraṃ kāmapiṭhaṃ |}
      \extra{
          \app{\lem[wit={P,L,U2}]{prathamaṃ ādhāracakraṃ}
            \rdg[wit={E}]{prathamādhāracakraṃ vartate |}}/
                 gudā sthānaṃ\dd{}
                 \app{\lem[type=emendation, resp=egoscr]{raktaṃ}
                   \rdg[wit={E,P,L,U2}]{\korr rakta°}}varṇaṃ\dd{}
            \app{\lem[type=emendation, resp=egoscr]{gaṇeśaṃ daivataṃ}
                 \rdg[wit={E,L,U2}]{\korr gaṇeśadaivataṃ}
                 \rdg[wit={P}]{gaṇeśāṃ daivataṃ}}\dd{}
            \app{\lem[type=emendation, resp=egoscr]{siddhibuddhiśaktiṃ muṣako vāhanaṃ} %Emendation!!!
                 \rdg[wit={E}]{\korr °śaktimuṣakavāhanam}
                 \rdg[wit={P}]{°śaktir mukhako vāhanam}
                 \rdg[wit={L}]{°śaktimuṣako vāhanaṃ}
                 \rdg[wit={U2}]{°śaktiḥ muṣako vāhanaṃ}}\dd{}
            \app{\lem[type=emendation, resp=egoscr]{kūrma} %%sandhi aḥ vor ṛ wird zu a + ṛ 
                 \rdg[wit={U2}]{\korr kurma}}ṛṣiḥ\dd{}
            \app{\lem[type=emendation, resp=egoscr]{ākuñcanaṃ mudrā}
                 \rdg[wit={P,L,U2}]{\korr ākuñcana°}
                 \rdg[wit={E}]{ākuṃca°}}mudrā\dd{}
            \app{\lem[type=emendation, resp=egoscr]{apānaḥ vāyuḥ}
                 \rdg[wit={E,L}]{\korr apānavāyuḥ}
                 \rdg[wit={P}]{°vāyuś}
                 \rdg[wit={U2}]{°vāyu}}\dd{}
               \extra{
                 \app{\lem[type=emendation, resp=egoscr]{ūrmī}
                   \rdg[wit={U2}]{\korr urmī}} kalā\dd{}
                 ojasvinī dhāraṇā\dd{}}
                 caturdaleṣu rajaḥsattvatamomanāṃsi\dd{}
                 vaṃ śaṃ ṣaṃ saṃ\dd{} madhyatrikoṇe
            \app{\lem[wit={P,L}]{triśikhā}
                 \rdg[wit={E}]{triśikhāt}
                 \rdg[wit={U2}]{trirekhā}}\dd{}}
        %%%%%%%%%%%%%%%%%
        %%%%%%%%%%%%%%%%%
        %%%%%%%%%%%%%%%%%
        %%%%%%%%%%%%%%%%%
        %%%%%%%%%%%%%%%%%          
            \app{\lem[wit={ceteri}]{tanmadhye}
                 \rdg[wit={L,N1}]{tanmadhya}}
               trikoṇākāraṃ kāmapiṭhaṃ vartate/
\note[type=philcomm, labelb=32, lem={prathamaṃ \ldots triśikhā}]{The whole section from \textit{prathamaṃ} to \textit{triśikhā} is missing in D, N\textsubscript{1}, N\textsubscript{2} and U\textsubscript{1}. Equally detailled passages for the other \textit{cakra}s which include assigments to various categories like \textit{daivata}, \textit{bīja}s etc. occur only in witness \textit{U2}. It is currently not possible to decide if a) these passages were lost in transmission in all other witnesses and were preserved in U\textsubscript{2} only or b), if the extensive descriptions for the first \textit{cakra} occurred randomly and the additions found in U\textsubscript{2} are not authorial. At least we might assume that it was not the the scribe of U\textsubscript{2} himself who wrote the additions. He explicitly states that he copied his template faithfully in this colophon: \begin{quote} yādṛśaṃ pustakaṃ dṛṣtvā tādṛsaṃ likhitaṃ mayā || \\yadi śuddhaṃ aśuddho cā mama doṣo na dīyate ||1||\end{quote}}
%-----------------------
%\om                                                      \B
%tatpīṭhamadhye 'gniśikhākāraikā    mūrtir varttate /        \E
%tatpīṭhamadhye magniśikhākārā ekā  mūrtir varttate /      \P
%tatpīṭhamadhye   jniśikhāka!rāṇakā mūrti varttate //     \L
%tatpīṭhamadhye  agniśikhākārā ekā  mūrttir varttate //    \N1
%tatpīṭhamadhye  agniśikhākārā ekā  mūrttir varttate /     \N2
%tatpīṭhamadhye  agniśikhākārā ekā  mūrttir varttate //    \D
%tatpīṭhamadhye  agniśikhākārā ekā  mūrttir varttate //    \U1
%tatpīṭhamadhye  agniśikhākārā ekā  mūrttir asmi      //    \U2
%-----------------------
%In the middle of this seat (\textit{pīṭha}) exists a single form having the shape of a flame.             
%-----------------------
\note[type=testium, labelb=33, lem={'gniśikhākāraikā}]{Ysg: tatpīṭhamadhye agniśikhākārā gaṇeśamūrttir varttate ||}
tatpīṭhamadhye
\app{\lem[wit={E}]{'gniśikhākāraikā}
  \rdg[wit={ceteri}]{agniśikhākārā ekā}
  \rdg[wit={P}]{magniśikhākārā ekā}
  \rdg[wit={L}]{jñiśikhākarāṇakā}}
murti\skp{r-va}\app{\lem[wit={E,P,L,N1,N2,D,U1}, alt={vartate}]{\skm{r-va}rtate}
  \rdg[wit={U2}]{asmi}}/
%-----------------------%
%\om                                       \B
%tasyāḥ mūrtirdhyānakāraṇāt   sakalaśāstrakāvya-nāṭakādi-sakalavāṅmayaṃ vinābhyāsena puruṣasya manomadhye sphurati,     \E
%tasyā mūrter dhyānakaraṇāt   sakalaśāstrakāvya-nāṭakādi-sakalavāṅmayaṃ vinābhyāsena puruṣasya manomadhye sphurati      \P
%tasyā mūrtir dhyānakāraṇāt   sakalaśāstrakāvya-nāṭakādi //----vāṅmayaṃ vinābhyāsena puruṣasya manomadhye sphuraṃti!    \L
%tasyāḥ mūrter dhyānakaraṇāt  sakalaśāstrakāvya-nāṭakādi-sakalavāgmayaṃ vinābhyāsena puruṣasya manomadhye sphurati      \N1
%tasyā mūrtter dhyānakaraṇāt  sakalaśāstrakāvya-nāṭakādi-sakavāgmayaṃ   vinābhyāsena puruṣasya manomadhye sphurati//    \N2
%tasyāḥ mūrter dhyānakaraṇāt  sakalaśāstrakāvya-nāṭakādi-sakalavāgmayaṃ vinābhyāsena puruṣasya manomadhye sphurati      \D
%tasyā  mūrtair dhyānakaraṇāt sakalaśāstrakāvya-nāṭakādi-sakalavāgmayaṃ vinābhyāsena puruṣasya manomadhye sphurati      \U1
%tasyā          dhyānakaraṇāt sakalaśāstrakāvya-nāṭakādi-sakalavāṅmayaṃ vinābhyāsena puruṣasya manomadhye sphurati // asya bahir mānaṃdā // yogānaṃdā virānaṃdā // uparamānaṃdā // ajapājapa śat // 600 // ghaṭi 9 palāni 40 // \U2 %
%-----------------------
%Trough the practice of meditation on this form the whole literature, all \textit{śāstra}s, all poems, dramas etc., everything [related to] elocution, appears in the mind of the person without [prior] learning. \extra{[Assigned to it] is external bliss, yogic bliss, heroic bliss [and] the bliss of coming to rest.}
%-----------------------
\note[type=testium, labelb=34, lem={sakalaśāstra°}]{Ysg: tasyā mūrter dhyānakaraṇāt sakalakāvyanāṭakādisakalavāṅmayaṃ vinābhyāsena puruṣasya manomadhye sphurati}
\app{\lem[wit={ceteri}]{tasyā}
    \rdg[wit={E,N1,D}]{tasyāḥ}}
\app{\lem[wit={ceteri}, alt={mūrter}]{mūrte\skp{r-dhyā}}
    \rdg[wit={E,L}]{mūrtir}
    \rdg[wit={U1}]{mūrtair}
    \rdg[wit={U2}]{\om}
}\skm{r-dhyā}nakaraṇāt-śāstrakāvya\app{\lem[wit={ceteri}, alt={°nāṭakādi°}]{nāṭakādi}
    \rdg[wit={L}]{°nāṭakādi ||}}\app{\lem[wit={ceteri}, alt={°sakala}]{sakala}
    \rdg[wit={L}]{\om}
    \rdg[wit={N2}]{saka°}}\app{\lem[wit={E,P,L,U2}]{vāṅmayaṃ}
    \rdg[wit={N1,N2,D,U1}]{vāgmayaṃ}} vinābhyāsena puruṣasya manomadhye
\app{\lem[wit={ceteri}]{sphurati}
  \rdg[wit={L}]{sphuraṃti}}/
      \extra{asya
        \app{\lem[type=emendation, resp=egoscr, alt={bahir ānandā}]{bahir\skp{-}ānandā}
          \rdg[wit={U2}]{\korr bahir mānandā}}\dd{}
        yogānandā\dd{}
        \app{\lem[type=emendation, resp=egoscr]{vīrānandā}
          \rdg[wit={U2}]{\korr virānandā}}\dd{}
        uparamānandā\dd{}
        \app{\lem[type=emendation, resp=egoscr]{ajapājapaḥ śataḥ}
          \rdg[wit={U2}]{\korr ajapājapaśat}}\dd{} 600\dd{} ghaṭi 9 palāni 40\dd{}} 
     \end{prose}
  \end{ekdosis}
\ekdpb*{}
%%%%%%%%%%%%%%%%%%%%%%%%%%%%%%%%%%%%%%%%%%
%%%%%%%%%%%%%%%%%%%%%%%%%%%%%%%%%%%%%%%%%%
%%%%%%%%PAGEBREAK%%%%%%%PAGEBREAK%%%%%%%%%
%%%%%%%%%%%%%%%%%%%%%%%%%%%%%%%%%%%%%%%%%%
%%%%%%%%%%%%%%%%PAGEBREAK%%%%%%%%%%%%%%%%%
%%%%%%%%%%%%%%%%%%%%%%%%%%%%%%%%%%%%%%%%%%
%%%%%%%%PAGEBREAK%%%%%%%PAGEBREAK%%%%%%%%%
%%%%%%%%%%%%%%%%%%%%%%%%%%%%%%%%%%%%%%%%%%
%%%%%%%%%%%%%%%%%%%%%%%%%%%%%%%%%%%%%%%%%%
%%%%%%%%%%%%%%%%%%%%%%%%%%%%%%%%%%%%%%%%%%
%%%%%%%%%%%%%%%%%%%%%%%%%%%%%%%%%%%%%%%%%%
%%%%%%%%PAGEBREAK%%%%%%%PAGEBREAK%%%%%%%%%
%%%%%%%%%%%%%%%%%%%%%%%%%%%%%%%%%%%%%%%%%%
%%%%%%%%%%%%%%%%PAGEBREAK%%%%%%%%%%%%%%%%%
%%%%%%%%%%%%%%%%%%%%%%%%%%%%%%%%%%%%%%%%%%
%%%%%%%%PAGEBREAK%%%%%%%PAGEBREAK%%%%%%%%%
%%%%%%%%%%%%%%%%%%%%%%%%%%%%%%%%%%%%%%%%%%
%%%%%%%%%%%%%%%%%%%%%%%%%%%%%%%%%%%%%%%%%%
%%%%%%%%%%%%%%%%%%%%%%%%%%%%%%%%%%%%%%%%%%
%%%%%%%%%%%%%%%%%%%%%%%%%%%%%%%%%%%%%%%%%%
%%%%%%%%PAGEBREAK%%%%%%%PAGEBREAK%%%%%%%%%
%%%%%%%%%%%%%%%%%%%%%%%%%%%%%%%%%%%%%%%%%%
%%%%%%%%%%%%%%%%PAGEBREAK%%%%%%%%%%%%%%%%%
%%%%%%%%%%%%%%%%%%%%%%%%%%%%%%%%%%%%%%%%%%
%%%%%%%%PAGEBREAK%%%%%%%PAGEBREAK%%%%%%%%%
%%%%%%%%%%%%%%%%%%%%%%%%%%%%%%%%%%%%%%%%%%
%%%%%%%%%%%%%%%%%%%%%%%%%%%%%%%%%%%%%%%%%%
   \begin{ekdosis}    
     \ekddiv{type=ed}
    \centerline{\textrm{\small{[Second Cakra]}}}
    \bigskip
    \begin{prose}
%-----------------------
% \om                                       \oxford
%idānīṃ dvitīyaṃ svādhiṣṭānacakraṃ   ṣaḍdalaṃ upāyanapīṭhasaṃjñakaṃ bhavati //  \E
%idānīṃ dvitīyaṃ svādhiṣṭānacakraṃ   ṣaṭdalaṃ uḍḍīyānapīṭhaṃ saṃjñakaṃ bhavati  \P
%idānīṃ dvitīyaṃ svādhiṣṭānacakraṃ   ṣaṭdalaṃ uḍḍīyān pīṭhaṃ saṃjñakaṃ bhavati  \L
%idānīṃ dvitīyaṃ svādhiṣṭānacakraṃ   ṣaṭdalaṃ uḍyānapīṭhasaṃjñakaṃ bhavati /    \N1
%idānī  dvitīyaṃ svādhinacakraṃ      ṣaḍḍalaṃ uḍyānapīṭhasaṃjñakaṃ bhavati      \N2
%idānīṃ dvitīyaṃ svādhiṣṭānacakraṃ   ṣaṭdalaṃ uḍyāṇāpīṭhasaṃjñikaṃ bhavati //   \D
%idānīṃ dvitīyaṃ svādhiṣṭhānacakraṃ  ṣaṭdalaṃ uḍāganapīṭasaṃjñakaṃ bhavati      \U1
%idānīṃ dvitīye svādhiṣṭānacakraṃ // ṣaṭdalaṃ // uḍḍīyāṇapīṭhasaṃjñakaṃ bhavati // liṃgasthānaṃ // pītavarṇaṃ // pītaprabhā // rajoguṇa // brahmādevatā // vaikharīvāca // sāvitrīśaktiḥ // haṃsavāhanaṃ // vahaṇaṛṣiḥ // kāmāgniprabhā //sthūladehā // jāgradavasthā // ṛgveda // ācāryaliṃgaṃ // braṃhmasalokatāmokṣaḥ // śuddhabhumikātatvaṃ // gaṃdho viṣayaḥ // apānavāyuḥ // aṃtarmātṛkā // vaṃ bhaṃ maṃ yaṃ raṃ laṃ // bahir mātrā // kāmā // kāmākhyā // tejasī // ceṣṭṛikā // alasā // mithunā // ajapājapaḥ sahasra // 6000 //gha 0 96 pa 0 40// \U2
%-----------------------
%Now the second, the six-petalled \textit{Svādhiṣṭānacakra} known as the seat of \textit{uḍḍīyāṇa}\footnote{Discuss the term \textit{uḍḍīyāna}.}. \extra{The gender is the location. The color is yellow. The shine is yellow. \textit{Rajas} is the quality. The deity is Brahmā. The speech is \textit{vaikharī}\footnote{vaikharī f. in Kaśm. Śiv. °the 4. form of appearacne of \textit{parā}, the empirical speech sound, Utpala's Ṭīkā to Śivadṛṣṭi 2, 7. [B.]― Schmidt p. 337. Welches Buch???} (\textit{vaikharīvāca}). The power is Sāvitrī. The mount is the goose. The \textit{Rṣi} is Vahaṇa. The appearance (\textit{prabhā} is the fire of love (\textit{kāmāgni}). The body is gross, The state is that of being awake. [The Veda associated with it is] the Ṛgveda. The spiritual guide is the \textit{liṅga}. The liberation is residing in the world of Brahma. The level is the pure earth (\textit{śuddhabhumikā}). The sphere is smell. The vitalwind is \textit{apāna}. The internal alphabet [is]: vaṃ bhaṃ maṃ yaṃ raṃ laṃ. The outer alphabet?: desire, the Tīrtha of \textit{Kāmākhyā}\footnote{The Kāmākhyā is situated in Kāmarūpa on the Nīlakūṭa mountain in present day Assam. It's strange that it appears here, since Kāmarūpa appears already as the Tīrtha associated with the first \textit{cakra}.}, beauty of both\footnote{Why dual here?}, \textit{ceṣṭṛikā} (what is that?), lazy [and] copulation.}
%-----------------------      
\noindent
\note[type=testium, labelb=35, lem={svādhiṣṭānacakraṃ}]{Ysg: liṃgo dvitīyaṃ ṣaṭdalaṃ svādhiṣṭānasaṃjñakaṃ kamalaṃ udyānapīṭhasaṃjñakaṃ vartate ||}
\note[type=source, labelb=36, lem={svādhiṣṭhāna°}]{Ysv (PT): liṅgamūle tu pīṭhābhaṃ [raktābhaṃ (YK)] svādhiṣṭhānantu ṣaḍdalam | tanmadhye bālasūryābhaṃ mahajjyotiḥ susiddhidam | dhyānāc ca varddhate āyuḥ kandarpasamatāṃ vrajet |}
\app{\lem[wit={ceteri}]{idānīṃ}
          \rdg[wit={N2}]{idānī}}
        \app{\lem[wit={ceteri}]{dvitīyaṃ}
            \rdg[wit={U2}]{dvitīye}}
        \app{\lem[wit={U1}]{svādhiṣṭhānacakraṃ}
            \rdg[wit={E,P,L,N1,D,U2}]{svādhiṣṭānacakraṃ}
            \rdg[wit={N2}]{svādhinacakraṃ}}
        \app{\lem[wit={ceteri}]{ṣaṭdalaṃ}
            \rdg[wit={E}]{ṣaḍdalaṃ}
            \rdg[wit={N2}]{ṣaḍḍalaṃ}}
        \app{\lem[wit={U2},alt={uḍḍīyāṇapīṭha°}]{uḍḍīyāṇapīṭha}
            \rdg[wit={E}]{upāyanapīṭha°}
            \rdg[wit={L}]{uḍḍīyān pīṭhaṃ}
            \rdg[wit={N1,N2}]{uḍyānapīṭha°}
            \rdg[wit={D}]{uḍyāṇāpīṭha°}
            \rdg[wit={U1}]{uḍāganapīṭa°}}saṃjñakaṃ
bhavati/         
      %%%%%%%%%%%%%%%%
      %%%%%%%%%%%%%%%
      %%%%%%%%%%%%%%%%
      %%%%%%%%%%%%%%%
      %%%%%%%%%%%%%%%    
      \extra{\app{\lem[type=emendation, resp=egoscr]{liṅgaṃ}
          \rdg[wit={U2}]{\korr liṅga°}} sthānaṃ\dd{}
        \app{\lem[type=emendation, resp=egoscr]{pītaṃ}
          \rdg[wit={U2}]{\korr pīta°}} varṇaṃ\dd{}
        \app{\lem[type=emendation, resp=egoscr]{pītā}
          \rdg[wit={U2}]{\korr pīta°}} prabhā\dd{}
        rajo \app{\lem[type=emendation, resp=egoscr]{guṇaḥ}
          \rdg[wit={U2}]{\korr guṇa}}\dd{}
        brahmā devatā\dd{}
        vaikharī \app{\lem[type=emendation, resp=egoscr]{vāk}
          \rdg[wit={U2}]{\korr vāca}}\dd{}
        sāvitrī śaktiḥ\dd{}
        \app{\lem[type=emendation, resp=egoscr]{haṃso}
          \rdg[wit={U2}]{\korr haṃsa°}} vāhanaṃ\dd{}
        \app{\lem[type=emendation, resp=egoscr]{vahaṇo}
          \rdg[wit={U2}]{\korr vahaṇa}} ṛṣiḥ\dd{}
        \app{\lem[type=emendation, resp=egoscr, alt={kāmāgnir}]{kāmāgni\skp{r-pra}}
          \rdg[wit={U2}]{\korr kāmāgni°}}\skm{r-pra}bhā\dd{}
        \app{\lem[type=emendation, resp=egoscr]{sthūlo dehaḥ}
          \rdg[wit={U2}]{\korr sthūladehā}}\dd{}
        jāgrad-avasthā\dd{}
        \app{\lem[type=emendation, resp=egoscr]{ṛg vedaḥ}
          \rdg[wit={U2}]{\korr ṛg veda}}\dd{}
        \app{\lem[type=emendation, resp=egoscr]{ācāryaḥ}
          \rdg[wit={U2}]{\korr ācārya°}} liṅgaṃ\dd{}
        brahmasalokatā mokṣaḥ\dd{}
        \app{\lem[type=emendation, resp=egoscr]{śuddhabhumikā}
          \rdg[wit={U2}]{\korr śuddhabhumikā}} tattvaṃ\dd{}
        gaṃdho viṣayaḥ\dd{}
        \app{\lem[type=emendation, resp=egoscr]{apānaḥ}
          \rdg[wit={U2}]{apāna°}} vāyuḥ\dd{}
        aṃtar\skp{-}mātṛkā\dd{}
        vaṃ bhaṃ maṃ yaṃ raṃ laṃ\dd{}
        bahir-mātrā\dd{}
        kāmā\dd{}
        kāmākhyā\dd{}
        \app{\lem[type=emendation, resp=egoscr]{tejasvinī}
          \rdg[wit={U2}]{\korr tejasī}}\dd{}
        ceṣṭikā\dd{}
        alasā\dd{}
        mithunā\dd{}
        ajapājapaḥ \app{\lem[type=emendation, resp=egoscr]{sahasraḥ}
          \rdg[wit={U2}]{\korr sahasra}}\dd{} 6000 \dd{} gha. 16 pa. 40\dd{}}
%-----------------------
%
% \om                                        \B
%tanmadhye atiraktavarṇaṃ tejo varttate /    \E
%tanmadhye 'tiraktavarṇaṃ tejo varttate      \P
%tanmadhye  tiraktavarṇaṃ tejo varttate //   \L
%tanmadhye  atiraktavarṇaṃ tejo varttate     \N1
%tanmadhye  atiraktavarṇatejo varttate      \N2
%tanmadhye  atiraktavarṇaṃ tejo varttate     \D
%tanmadhye  atiraktavarṇatejo varttate       \U1
%tanmadhye 'tiraktavarṇaṃ tejo vartate //    \U2
%-----------------------
%In its middle exists extremely red glow. The adept becomes very handsome by meditation on it.       
%-----------------------          
\note[type=testium, labelb=37, lem={atirakta°}]{Ysg: tatra atiraktaṃ \sic{yahbhā} saṃjñakaṃ tejaḥ |}
tanmadhye         
        \app{\lem[wit={P,U2}]{'tiraktavarṇaṃ}
            \rdg[wit={ceteri}]{atiraktavarṇaṃ}
            \rdg[wit={U1,N2}]{atiraktavarṇa°}}
tejo vartate/
%-----------------------
% \om                                          \B
%tasya dhyānāt sādhako 'tisundaro bhavati /    \E
%tasya dhyānāt sādhako   tisuṃdaro bhavati      \P
%tasya dhyānāt sādhako   tisuṃdaro bhavati //   \L
%tasya dhyānāt sādhakaḥ  atisuṃdaro bhavati // \N1
%tasya dhyānāt sādhakaḥ  atisuṃdaro bhavati/   \N2
%tasya dhyānāt sādhakaḥ  atisuṃdaro bhavati // \D
%tasyā     nāt sādhakaḥ  atisuṃdarāṃgasan  // \D2
%tasya dhyānāt sādhakaḥ  atisuṃdaro bhavati    \U1
%tasya dhyānāt sādhako  'tisundaro bhavati //   \U2
%-----------------------
%The adept becomes very handsome through meditation on it.
%-----------------------       
\note[type=testium, labelb=38, lem={tasya dhyānāt}]{Ysg: tasyā nāt sādhakaḥ atisuṃdarāṃgasan ||}
tasya dhyānāt
\app{\lem[wit={E,P,L,U2}]{sādhako}
  \rdg[wit={ceteri}]{sādhakaḥ}}
\app{\lem[wit={E,P,L,U2}]{'tisundaro}
  \rdg[wit={D,N1,N2,U1}]{atisuṃdaro}}
bhavati/ 
%-----------------------
% \om                                  \B
%                                pratidinam-āyur vardhate /             \E
%                                pratidinam-āyur vardhate               \P
%                                pratidinam-āyur vardhate //2//         \L
%                                dinaṃ dinaṃ prati āyurvarddhate // //  \N1
%yuvatīnāṃ ativallabho? bhavati dinadinaṃ prati āyur varddhate//        \N2  %%%3verso
%                                dinaṃ prati āyurvarddhate //2//        \D
%                                dinaṃ dinaṃ prati āyurvarddhate        \U1
%                                pratidinaṃ āyur varddhate //          \U2
%-----------------------
%\extra{He becomes one who is very desired by virgins.} The vital force increases from day to day. \end{tlate}
%-----------------------
\note[type=testium, labelb=39, lem={yuvatīnām}]{Ysg: yuvatīnām ativallabhaḥ san pratidinam āyuṣyābhivṛddhimān bhavati | cha |} % \D2 %%%S.2 Z. 11}
\extra{yuvatīnāṃ ativallabho bhavati/\note[type=philcomm, labelb=40, lem={yuvatīnāṃ\ldots bhavati}]{This additional sentence occurs in N\textsubscript{2} and the Ysg only.}}
\app{\lem[wit={ceteri}, alt={pratidinam}]{pratidina\skp{m-ā}}
  \rdg[wit={N1,U1}]{dinaṃ dinaṃ prati}
  \rdg[wit={N2}]{dinadinaṃ prati}
  \rdg[wit={D}]{dinaṃ prati}}
\skp{m-ā}yur-vardhate\dd{}
    \end{prose}
  \end{ekdosis}
    %%%%%%%%%%%%%%
    %%%%%%%%%%%%%%%
    %%%%%%%%%%%%%%
    %%%%%%%%%%%%%%
  %%%%%%%%%%%%%%
\begin{ekdosis}
 \ekddiv{type=ed}
  \bigskip
    \centerline{\textrm{\small{[Third Cakra]}}}
    \bigskip
    \begin{prose}
      \note[type=source, labelb=50, lem={tṛtīyaṃ}]{Ysv (PT): tṛtīyaṃ nābhideśe tu digdalaṃ paramādbhutam | mahāmeghaprabhaṃ tattu koṭividyutsamanvitam | kalpāntāgnisamaṃ [kalpānto 'gni° (YK)] jyotis tanmadhye saṃsthitaṃ svayam | tasya [asya (YK)] dhyānāc cirāyuḥ syād arogo [arogī (YK)] jagatāṃ varaḥ [jagatāmvaraḥ (YK)] | sarvapāpavinirmukto jagatkṣobhakaro [jaganmokṣakaro (YK)] mahān |}
%-----------------------
% \om                                                 \B
%tṛtīye                      nābhisthāne    daśadalaṃ padmaṃ vartate      \E
%tṛtīyaṃ                     nābhisthāne    daśadalaṃ padmaṃ vartate      \P
%tṛtīyaṃ                     nābhisthāne // daśadalapadme vartate         \L
%tṛtīyaṃ                     nābhisthāne    daśadalaṃ padma varttate //   \N1
%tṛtīyacakraṃ                nābhisthāne    daśadalaṃ padma varttate /    \N2
%tṛtīyaṃ                     nābhisthāne    daśadalaṃ padma varttate //   \D
%tṛtīyaṃ                     nābhisthāne    daśadalakaṃ padmaṃ varttate   \U1
%atha tṛtīyaṃ maṇipūracakraṃ nābhisthāne // kapilavarṇaṃ // viṣṇudevatā // lakṣmīśaktiḥ // vāyuṛṣiḥ // samānavāyuḥ // garuḍavāhanaṃ // sūkṣmaliṃgadevatāha // svapnāvasthā // madhyamāvāk // yajurvedaḥ // dakṣināgniḥ // samipatāmokṣaḥ // guruliṃgaviṣṇuḥ // āpastatvaṃ // rajoviṣayaḥ daśadalāni // daśamātrāḥ // aṃtarmātrā // ḍaṃ ṭaṃ ṇaṃ taṃ thaṃ daṃ dhaṃ naṃ paṃ phaṃ // bahirmātrāḥ // śāṃtiḥ // kṣamā // medhā // tanyā // medhāvinī // puṣkarā // ahaṃsagamanā // lakṣyā //tanmayā // amṛtā // ajapājapa // 6000 gha 016 pa 040 //    \U2
%-----------------------
%\extra{The colour is red (\textit{kapila}). Viṣṇu is the deity. Lakṣmī is the power. Vāyu is the Rṣi. Samāna is the vitalwind. The mount is Garuḍa. The deity is the suble body\footnote{Why another deity is given here?}. The state is sleep. The speech is the inaudible speech (\textit{madhyamāvāg})\footnote{<Śā, Ling>name of the speech which is inaudible and which is of the type of a thought without any definite presence of words making up the expression. Vkp I.143.<Abhyankar 1986: 300>}. The Veda is the Yajurveda. The [fire is the] southern fire. The liberation is ``proximity'' (\textit{samīpatā}).\footnote{What is this exactly?}. Viṣṇu is the characteristic of the teacher (\textit{guruliṅga}). The principle is water. The sphere is athmosphere (\textit{rajo viṣaya}). There are ten petals [and] ten matrices. [The] inner matrix: \textit{ḍaṃ ṭaṃ ṇaṃ taṃ thaṃ daṃ dhaṃ naṃ paṃ phaṃ}. The external matrix : peace, patience, insight, the ``daughter''\textit{tanayā}, the ``learned teacher'', the ``lotus'', \textit{haṃsagamanā}, the ``fixation object'', absorption and immortality.} 
%-----------------------
\note[type=testium, labelb=60, lem={daśadalaṃ}]{Ysg: nābhistnāne daśadalaṃ cakraṃ |}
      \app{\lem[wit={ceteri}]{tṛtīyaṃ}
      \rdg[wit={E}]{tṛtīye}
      \rdg[wit={U2}]{atha tṛtīyaṃ maṇipūracakraṃ}
      \rdg[wit={N2}]{tṛtīyacakraṃ}}
    nābhisthāne
    \app{\lem[wit={ceteri}]{daśadalaṃ}
      \rdg[wit={L}]{daśadala°}
      \rdg[wit={U1}]{daśadalakaṃ}
      \rdg[wit={U2}]{\om}}
    \app{\lem[wit={E,P,U1}]{padmaṃ}
      \rdg[wit={L}]{°padme}
      \rdg[wit={N1,N2,D}]{padma}
      \rdg[wit={U2}]{\om}}
    \app{\lem[wit={ceteri}]{vartate}
      \rdg[wit={U2}]{\om}}/
      %%%%%%%%%%%%
      %%%%%%%%%%%%%%%
      %%%%%%%%%%%%%%
      %%%%%%%%%%%%%
      %%%%%%%%%%%% 
      \extra{
        \app{\lem[type=emendation, resp=egoscr]{kapilaṃ}
          \rdg[wit={U2}]{\korr kapila°}} varṇaṃ\dd{}
        \app{\lem[type=emendation, resp=egoscr]{viṣṇur}
          \rdg[wit={U2}]{\korr viṣṇu}} devatā\dd{}
        lakṣmī śaktiḥ\dd{}
        \app{\lem[type=emendation, resp=egoscr, alt={vāyur}]{vāyu\skp{r-ṛ}}
          \rdg[wit={U2}]{\korr vayu°}}\skm{r-ṛ}ṣiḥ\dd{}
        \app{\lem[type=emendation, resp=egoscr]{samāno}
          \rdg[wit={U2}]{\korr samāna°}} vāyuḥ\dd{}
        \app{\lem[type=emendation, resp=egoscr]{garuḍo}
          \rdg[wit={U2}]{\korr garuḍa°}} vāhanaṃ\dd{}
      \app{\lem[type=emendation, resp=egoscr]{sūkṣmaliṅgaṃ devatā}
        \rdg[wit={U2}]{\korr sūkṣmaliṅgadevatāha}}\dd{}
      \app{\lem[type=emendation, resp=egoscr,alt={svapnā avasthā}]{svapnā-avasthā}
        \rdg[wit={U2}]{\korr svapnāvasthā}}\dd{}
      madhyamā vāk\dd{}
      yajur-vedaḥ\dd{}}
     \end{prose}
  \end{ekdosis}
\ekdpb*{}
%%%%%%%%%%%%%%%%%%%%%%%%%%%%%%%%%%%%%%%%%%
%%%%%%%%%%%%%%%%%%%%%%%%%%%%%%%%%%%%%%%%%%
%%%%%%%%PAGEBREAK%%%%%%%PAGEBREAK%%%%%%%%%
%%%%%%%%%%%%%%%%%%%%%%%%%%%%%%%%%%%%%%%%%%
%%%%%%%%%%%%%%%%PAGEBREAK%%%%%%%%%%%%%%%%%
%%%%%%%%%%%%%%%%%%%%%%%%%%%%%%%%%%%%%%%%%%
%%%%%%%%PAGEBREAK%%%%%%%PAGEBREAK%%%%%%%%%
%%%%%%%%%%%%%%%%%%%%%%%%%%%%%%%%%%%%%%%%%%
%%%%%%%%%%%%%%%%%%%%%%%%%%%%%%%%%%%%%%%%%%
%%%%%%%%%%%%%%%%%%%%%%%%%%%%%%%%%%%%%%%%%%
%%%%%%%%%%%%%%%%%%%%%%%%%%%%%%%%%%%%%%%%%%
%%%%%%%%PAGEBREAK%%%%%%%PAGEBREAK%%%%%%%%%
%%%%%%%%%%%%%%%%%%%%%%%%%%%%%%%%%%%%%%%%%%
%%%%%%%%%%%%%%%%PAGEBREAK%%%%%%%%%%%%%%%%%
%%%%%%%%%%%%%%%%%%%%%%%%%%%%%%%%%%%%%%%%%%
%%%%%%%%PAGEBREAK%%%%%%%PAGEBREAK%%%%%%%%%
%%%%%%%%%%%%%%%%%%%%%%%%%%%%%%%%%%%%%%%%%%
%%%%%%%%%%%%%%%%%%%%%%%%%%%%%%%%%%%%%%%%%%
%%%%%%%%%%%%%%%%%%%%%%%%%%%%%%%%%%%%%%%%%%
%%%%%%%%%%%%%%%%%%%%%%%%%%%%%%%%%%%%%%%%%%
%%%%%%%%PAGEBREAK%%%%%%%PAGEBREAK%%%%%%%%%
%%%%%%%%%%%%%%%%%%%%%%%%%%%%%%%%%%%%%%%%%%
%%%%%%%%%%%%%%%%PAGEBREAK%%%%%%%%%%%%%%%%%
%%%%%%%%%%%%%%%%%%%%%%%%%%%%%%%%%%%%%%%%%%
%%%%%%%%PAGEBREAK%%%%%%%PAGEBREAK%%%%%%%%%
%%%%%%%%%%%%%%%%%%%%%%%%%%%%%%%%%%%%%%%%%%
%%%%%%%%%%%%%%%%%%%%%%%%%%%%%%%%%%%%%%%%%%
\begin{ekdosis}
    \ekddiv{type=ed}
  \begin{prose}
\noindent
\extra{\app{\lem[type=emendation, resp=egoscr]{dakṣiṇo 'gniḥ}
        \rdg[wit={U2}]{\korr dakṣināgniḥ}}\dd{}
      \app{\lem[type=emendation, resp=egoscr]{samīpatā}
        \rdg[wit={U2}]{samipatā}} mokṣaḥ\dd{}
      \app{\lem[type=emendation, resp=egoscr]{guruliṅgo}
        \rdg[wit={U2}]{\korr guruliṅga°}} viṣṇuḥ\dd{}
      āpas-tattvaṃ\dd{}
      rajo viṣayaḥ\dd{}
      daśadalāni\dd{}
      daśamātrāḥ\dd{}
      antar-mātrā\dd{}
      ḍaṃ ṭaṃ ṇaṃ taṃ thaṃ daṃ dhaṃ naṃ paṃ phaṃ\dd{}
      bahir-mātrāḥ\dd{}
      śāṃtiḥ\dd{}
      kṣamā\dd{}
      medhā\dd{}
      tanayā\dd{}
      medhāvinī\dd{}
      puṣkarā\dd{}
      \app{\lem[type=emendation, resp=egoscr]{haṃsagamanā}
        \rdg[wit={U2}]{\korr ahaṃsagamanā}}\dd{}
      lakṣyā\dd{}
      tanmayā\dd{}
      amṛtā\dd{}
      ajapājapaḥ \app{\lem[type=emendation, resp=egoscr]{sahasraḥ}
        \rdg[wit={U2}]{\korr sahasra}}\dd{} 6000\dd{} gha. 16 pa. 40\dd{}}   
%-----------------------
% \om                                       \B
%tanmadhye paṃcakoṇaṃ cakraṃ varttate//    \E
%tanmadhye paṃcakoṇaṃ cakraṃ varttate       \P
% \om  \L
%tanmadhye paṃcakoṇaṃ cakraṃ varttate//    \N1
%tanmadhye paṃcakoṇaṃ cakraṃ varttate/    \N2
%tanmadhye paṃcakoṇaṃ cakraṃ varttate//    \D
%tanmadhye paṃcakoṇaṃ cakraṃ varttate       \U1
%tanmadhye paṃcakoṇaṃ cakraṃ vartate//     \U2
%-----------------------
% In its middle exists a \textit{cakra} with five angles.
%-----------------------
\note[type=testium, labelb=61, lem={paṃcakoṇaṃ}]{Ysg: tanmadhye paṃcakoṇaṃ pīṭhe lakṣmīnāparvatī saṃjñakaṃ sahitā śiva saṃjñakaṃ rāmaṇaṃ rūpā}
tanmadhye pancakoṇaṃ cakraṃ vartate/ \note[type=philcomm, labelb=62, lem={tanmadhye ... cakraṃ vartate}]{This sentence is entirely \om L.}
%-----------------------
% \om                                  \B
%tanmadhye ekā mūrtir vartate/         \E
%tanmadhye ekā mūrtir vartate          \P
%\om                                   \L
%tanmadhye ekā mūrttir varttate //     \N1
%tanmadhye ekā mūrttir varttate/       \N2
%tanmadhye ekā mūrttir varttate//      \D
%tanmadhye ekā mūrtir vartate          \U1
%tanmadhye ekā mūrtir asmi//           \U2
%-----------------------
%In its middle is a single (divine) form. 
%-----------------------
\app{\lem[wit={ceteri}]{tanmadhye}
  \rdg[wit={L}]{\om}}
\app{\lem[wit={ceteri}]{ekā}
  \rdg[wit={L}]{\om}}
\app{\lem[wit={ceteri}]{mūrti\skp{r-va}}
  \rdg[wit={L}]{\om}}\app{\lem[wit={ceteri}, alt={vartate}]{\skm{r-va}rtate}
  \rdg[wit={U2}]{asmi}}/
%-----------------------
% \om                                           \B
%tasyās tejo jihvayā kathayituṃ na śakyate /    \E
%tasyās tejo jihvayā kathayituṃ na śakyate      \P
%tasyās tejo jihvayā kathyituṃ  na śakyate       \L
%tasyā  tejo jihvayā kathayituṃ  na śakyate //    \N1
%tasyā  tejo jihvayā kathayituṃ  na śakyate/      \N2
%tasyā  tejo jihvayā kathayituṃ  na śakyate //    \D
%tasyās tejo jihvayā kathatuṃ   na śakyate        \U1
%tasyās tejo jihvayā vaktuṃ     na śakyate //       \U2
%-----------------------
%It's not possible to describe her shine with speech (lit. with the tongue).
%-----------------------
\note[type=testium, labelb=63, lem={tasyās tejo}]{Ysg: yasyās tejo jihvayā kathituṃ na śakyate}
\app{\lem[wit={ceteri}, alt={tasyās}]{tasyā\skp{s-te}}
   \rdg[wit={N1,N2,D}]{tasyā}}\skm{s-te}jo jihvayā
 \app{\lem[wit={ceteri}]{kathayituṃ}
    \rdg[wit={L}]{kathyituṃ}
    \rdg[wit={U1}]{kathatuṃ}
    \rdg[wit={U2}]{vaktuṃ}}
  na śakyate/
%-----------------------
% \om                                                                    \B
%tasyāḥ mūrter dhyānakāraṇāt    puruṣasya śarīraṃ sthiraṃ bhavati //     \E
%tasyā  mūrter dhyānakaraṇāt    -------------------------------------    \P
%tasyā  mūrtir dhyānakaraṇāt // puruṣasya śarīraṃ sthiram bhavati //     \L
%tasyāḥ mūrter dhyānakaraṇāt    puruṣasya śarīraṃ sthiraṃ bhavati /      \N1
%tasyāḥ mūrter dhyānakaraṇāt    puruṣasya śarīraṃ sthiraṃ bhavati//      \N2
%tasyāḥ mūrter dhyānakaraṇāt    puruṣasya śarīraṃ sthiraṃ bhavati /      \D
%tasā          dhyānakaraṇāt    sādhakasya śarīraṃ sthiraṃ bhavati /cha/ \D2
%tasyāḥ mūrter dhyānakaraṇāt    puruṣasya śarīraṃ sthiraṃ bhavati vā     \U1
%tasyāḥ        dhyānakaraṇāt    puruṣasya śarīraṃ sthiraṃ bhavati //     \U2
%-----------------------
%Through the execution of meditation on this (divine) form the body of the person is going to be strong.   
%-----------------------
\note[type=testium, labelb=64, lem={tasyāḥ mūrter}]{Ysg: tasā dhyānakaraṇāt sādhakasya śarīraṃ sthiraṃ bhavati |cha|}
  \app{\lem[wit={ceteri}]{tasyāḥ}
  \rdg[wit={P,L}]{tasyā}}
  \app{\lem[wit={ceteri}, alt={mūrter}]{mūrte\skp{r-dhyā}}
      \rdg[wit={L}]{mūrtir}
      \rdg[wit={U2}]{\om}}\skm{r-dhyā}na\app{\lem[wit={ceteri}, alt={°karaṇāt}]{karaṇāt}
      \rdg[wit={L}]{karaṇāt ||}
      \rdg[wit={E}]{°kāraṇāt}}
\app{\lem[wit={ceteri}]{puruṣasya}
  \rdg[wit={P}]{\om}}
\app{\lem[wit={ceteri}]{śarīraṃ}
  \rdg[wit={P}]{\om}}
\app{\lem[wit={ceteri}]{sthiraṃ}
  \rdg[wit={P}]{\om}}    
  \app{\lem[wit={ceteri}]{bhavati}
    \rdg[wit={U1}]{bhavati vā}
    \rdg[wit={P}]{\om}}\dd{}
 \end{prose}
\end{ekdosis}
%%%%%%%%%%%%%%%%
%%%%%%%%%%%%%%%
%%%%%%%%%%%%%%%
%%%%%%%%%%%%%%
%%%%%%%%%%%%%%%
\begin{ekdosis}
    \ekddiv{type=ed}
  \ekddiv{type=ed}
   \bigskip
    \centerline{\textrm{\small{[Fourth Cakra]}}}
    \bigskip
    \begin{prose}
\note[type=source, labelb=65, lem={caturthaṃ}]{Ysv (PT): anāhatam aṣṭapīṭhaṃ [mahāpīṭhaṃ (YK)] caturthakamalaṃ hṛdi | sūryapatraṃ mahājyotir mahāsūkṣman tu cākṣuṣam | sūryapatraṃ dvādaśadalam [sentence \om in YK] | tanmadhye 'ṣṭadalaṃ padmamūrddhavaktraṃ mahāprabham |}
%-----------------------
% \om                                                   \B
%caturthaṃ hṛdayamadhye dvādaśadalaṃ kamalaṃ vartate/   \E
%caturthaṃ hṛdayamadhye dvadaśadalaṃ kamalaṃ varttate/  \P
%caturthaṃ hṛdayamadhye dvadaśadalaṃ kamalaṃ varttate/  \L
%caturthaṃ hṛdayamadhye dvadaśadalaṃ kamalaṃ varttate/  \N1
%caturthacakrakamalaṃ hṛdayamadhye dvadaśadalaṃ bhavati \N2    
%caturthaṃ hṛdayamadhye dvadaśadalaṃ kamalaṃ varttate   \D
%caturthaṃ hṛdayamadhye dvadaśadalaṃ kamalaṃ varttate/  \U1   
%caturthaṃ hṛdayamadhye dvadaśadalaṃ kamalam asti/      \U2
%
% anāhatacakraṃ hṛdayasthānaṃ // śvetavarṇaṃ tamoguṇaḥ // rudrodevatā // umāśaktiḥ // hiraṇyagarbhaṛṣiḥ // naṃdivāhanaṃ // prāṇavāyuḥ // jyotiḥ kalākāraṇaṃ dehe // suṣuptir avasthā // paśyaṃtivācā // sāmavedaḥ // gārhasyatyogniḥ? // śivaliṇgaṃ // prāptibhūmikā // sarūpatāmuktiḥ // dvādaśādalāni //dvādaśamātrā // kaṃ khaṃ gaṃ ghaṃ ṇaṃ caṃ chaṃ jaṃ jhaṃ yaṃ taṃ thaṃ // bahirmātrā // rudrāṇī // tejasā // tāpinī // sukhadā // caitanyā // śivadā // śānti // umā // gaurī // mātara // jvālā // prajvālinī // ajapājapasahasra // 6000 gha. 96 pa. 40 // U2
%-----------------------
%The fourth lotus having twelve-petals exists in the middle at the heart. \extra{[The] Anāhatacakras place is within the heart\footnote{This seems to be redundant.}. The color is white. The quality is \textit{tamas}. The deity is Rudra. The power is Umā. The Ṛṣi is Hiraṇyagarbha. The mount is Nandi. The vitalwind is Prāṇa. In the body it is the light that causes parts (\textit{kalākaraṇa})\footnote{What is this?!}. The state is deep sleep. The speech is \textit{Paśyantī}\footnote{Add footnote of entry in \textit{Tāntrikābhidhānakośa}.}.The [Veda] is Sāmaveda. The fire is Gārhapatya\footnote{Add explanation.}. The Liṅgam is Śivaliṅga. The ability to attain everything on the earth [and] the uniform liberation [are attributed to this \textit{cakra}]. [There are] twelve petals, [and] twelve measures: kaṃ khaṃ gaṃ ghaṃ ṇaṃ caṃ chaṃ jaṃ jhaṃ yaṃ taṃ [and] thaṃ. The external measure: Rudra's wife, light (\textit{tejasā?}), glow, \textit{sphakadā}?, consciousness (\textit{caitanyā}), bestower of grace, peace, Umā, Gaurī, Mātara, the flame [and] Prajvālinī.}
%-----------------------
\note[type=testium, labelb=66, lem={caturthaṃ}]{Ysg: hṛdayamadhye dvadaśadalaṃ}
\app{\lem[wit={ceteri}]{caturthaṃ}
      \rdg[wit={N2}]{caturthacakrakamalaṃ}} hṛdayamadhye dvādaśadalaṃ
    \app{\lem[wit={ceteri}]{kamalaṃ}
       \rdg[wit={N2}]{\om}} 
    \app{\lem[wit={ceteri}]{vartate}
       \rdg[wit={U2}]{asti}
       \rdg[wit={N2}]{bhavati}}/
       %%%%%%%%%%%%%%%%%
       %%%%%%%%%%%%%%%%
       %%%%%%%%%%%%%%%%%%
       %%%%%%%%%%%%%%%%%
       %%%%%%%%%%%%%%%%
      \extra{anāhatacakraṃ hṛdayasthānaṃ\dd{}
        \app{\lem[type=emendation, resp=egoscr]{śvetaṃ}
          \rdg[wit={U2}]{\korr śveta°}} varṇaṃ\dd{}
        tamo guṇaḥ\dd{}
        rudro devatā\dd{}
        umā śaktiḥ\dd{}
        hiraṇyagarbha ṛṣiḥ\dd{}
        nandi vāhanaṃ\dd{}
        \app{\lem[type=emendation, resp=egoscr]{prāṇo}
          \rdg[wit={U2}]{\korr prāṇa°}} vāyuḥ\dd{}
        \app{\lem[type=emendation, resp=egoscr]{jyotiskalākāraṇaṃ deham}
          \rdg[wit={U2}]{\korr jyotiḥ kalākāraṇaṃ dehe}}\dd{}
        suṣuptir-avasthā\dd{}
        \app{\lem[type=emendation, resp=egoscr]{paśyantī}
          \rdg[wit={U2}]{\korr paśyaṃti}} vācā\dd{}
        sāmavedaḥ\dd{}
        \app{\lem[type=emendation, resp=egoscr]{gārhapatyo 'gniḥ}
          \rdg[wit={U2}]{\korr gārhasyatyo gniḥ}}\dd{}
        \app{\lem[type=emendation, resp=egoscr]{śivo}
          \rdg[wit={U2}]{\korr śiva°}} liṅgaṃ\dd{}
        \app{\lem[type=emendation, resp=egoscr]{prāptiḥ}
          \rdg[wit={U2}]{\korr prāpti°}} bhūmikā\dd{}
        sarūpatā muktiḥ\dd{}
        dvādaśādalāni\dd{}
        dvādaśamātrā\dd{}
        kaṃ khaṃ gaṃ ghaṃ ṇaṃ caṃ chaṃ jaṃ jhaṃ yaṃ taṃ thaṃ\dd{}
        bahir-mātrā\dd{}
        rudrāṇī\dd{}
        tejasā\dd{}
        tāpinī\dd{}
        sukhadā\dd{}
        caitanyā\dd{}
        śivadā\dd{}
        \app{\lem[type=emendation, resp=egoscr]{śāntiḥ}
          \rdg[wit={U2}]{\korr śānti}}\dd{}
        umā\dd{}
        gaurī\dd{}
        \app{\lem[type=emendation, resp=egoscr]{mātarā} %%%?????
          \rdg[wit={U2}]{\korr mātara}}\dd{}
        jvālā\dd{}
        prajvālinī\dd{}
        \app{\lem[type=emendation, resp=egoscr]{ajapājapaḥ}
          \rdg[wit={U2}]{\korr ajapājapaḥ}} \app{\lem[type=emendation, resp=egoscr]{sahasraḥ}
          \rdg[wit={U2}]{\korr sahasra}}\dd{} 6000\dd{} gha. 96 pa. 40\dd{}}
  %%%%%%%%%%%%%
  %%%%%%%%%%%%%%
  %%%%%%%%%%%%%
  %%%%%%%%%%%%%%
  %%%%%%%%%%%%%%%
%-----------------------
% \om                                        \B
%atitejomayatvād   dṛṣṭigocaraṃ na bhavati   \E  
%atitejomayatvāt   dṛṣṭigocaraṃ na bhavati   \P
%atitejomayatvād   dṛṣṭigocaraṃ na bhavati// \L
%atitejomayatvāt / dṛṣṭigocaraṃ na bhavati/ \N1
%atitejomayatvāt   dṛṣṭigocaraṃ na bhavati/ \N2
%atitejomayatvāt / dṛṣṭigocaraṃ na bhavati/ \D
%atitejomayatvāt / dṛṣṭigocaraṃ na bhavati/ \U1
%atitejomayatvād   dṛṣṭigocaratāṃ na yāti// \U2 
%-----------------------
%Due to being made of [such an] intense light [the fourth lotus] is not in the range of sight.
%-----------------------
\note[type=testium, labelb=67, lem={dṛṣṭigocaraṃ}]{Ysg: tejomayatvāt | dṛṣṭigocaraṃ na bhavaty etādṛśaṃ vartate}
    atitejomayatvād-dṛṣṭi\app{\lem[wit={ceteri}, alt={°gocaraṃ}]{gocaraṃ}  %SANDHI einbauen?! 
       \rdg[wit={U2}]{gocaratāṃ}}
na
\app{\lem[wit={ceteri}]{bhavati}
  \rdg[wit={U2}]{yāti}}/
%-----------------------
% \om                                               \B
%tanmadhye 'ṣṭadalam adhomukhaṃ kamalaṃ varttate // \E  
%tanmadhye 'ṣṭadale  mukhaṃ kamalaṃ varttate //     \P
%tanmadhye ṣṭadalaṃ  adhomukha--kamalaṃ vartate //  \L
%tanmadhye aṣṭadalaṃ adhomukhaṃ kamalaṃ vartate //  \N1
%tanmadhye aṣṭadalaṃ adhomukhaṃ kamalaṃ varttate//  \N2
%tanmadhye aṣṭadalaṃ adhomukhaṃ kamalaṃ vartate //  \D
%tanmadhye aṣṭadalaṃ adhomukhaṃ kamalaṃ vartate /   \U1
%tanmadhye 'ṣṭadalaṃ adhomukhaṃ kamalaṃ asti / manaś-cakre// manodevatā// bahiśaktiḥ// ātmaṛṣih// nābhimadhye sthitaṃ padmaṃ nālaṃ tasya daśāgulaṃ/ komalaṃ tasya tan nālaṃ nirmalaṃ cāpy adhomukhaṃ/ kadalīpuṣpasaṃkāśaṃ tanmadhye ca pratiṣṭhitaṃ/ mana unnaty-asaṃkalpa/ vikalpātmakam-eva ca/ pūrvadale svetavarṇe yadā viśrāmate manaḥ// dharmakīrtividyādi sadbuddhir-bhavati/ agnikoṇe āraktavarṇe nidrā ālasyamāyāmandamatir-bhavati/ dakṣiṇe kṛṣṇavarṇeti tadā krodhotpattir bhavati/ naiṛtye nīlavarṇe mamatāmatir bhavati/ paścime kapilavarṇe/ krīḍāhāsotsavotsāhamatir bhavati/ vāyav ye śāmavarṇe cintodvegamatir bhavati/ uttare pītavarṇe bhogaśṛṇgāramahodayamatir bhavati/ īśāne gauravarṇe jñānasaṃdhāne matir bhavati/} \U2
%-----------------------
\note[type=testium, labelb=68, lem={'ṣṭadalaṃ}]{Ysg: tanmadhye 'ṣṭadalaṃ adhomukhaṃ kamalaṃ ||}
    tanmadhye \app{\lem[wit={E,U2},alt={'ṣṭadalam}]{'ṣṭadala\skp{m-a}}
      \rdg[wit={P}]{'ṣṭadale}
      \rdg[wit={L}]{ṣṭadalaṃ}
      \rdg[wit={N1,N2,D,U1}]{aṣṭadalaṃ}}\app{\lem[wit={ceteri},alt={adhomukhaṃ kamalaṃ}]{\skp{m-a}dhomukhaṃ kamalaṃ}
        \rdg[wit={L}]{adhomukhakamalaṃ}
        \rdg[wit={P}]{mukhaṃ kamalaṃ}}
      \app{\lem[wit={ceteri}]{vartate}
        \rdg[wit={U2}]{asti}}/    
%%%%%%%%%%%%%%%%
%%%%%%%%%%%%%%%
%%%%%%%%%%%%%%%
%%%%%%%%%%%%%%
%%%%%%%%%%%%%%%
  \extra{manaś-cakre\dd{}
    mano devatā\dd{}
        \app{\lem[type=conjecture, resp=egoscr]{bahiśśaktiḥ}
          \rdg[wit={U2}]{\conj bahiśaktiḥ}}\dd{}   
        \app{\lem[type=emendation, resp=egoscr]{ātmā}
          \rdg[wit={U2}]{\korr ātma°}} ṛṣiḥ\dd{}
        nābhimadhye}
\end{prose}
\end{ekdosis}
\ekdpb*{}
%%%%%%%%%%%%%%%%%%%%%%%%%%%%%%%%%%%%%%%%%%
%%%%%%%%%%%%%%%%%%%%%%%%%%%%%%%%%%%%%%%%%%
%%%%%%%%PAGEBREAK%%%%%%%PAGEBREAK%%%%%%%%%
%%%%%%%%%%%%%%%%%%%%%%%%%%%%%%%%%%%%%%%%%%
%%%%%%%%%%%%%%%%PAGEBREAK%%%%%%%%%%%%%%%%%
%%%%%%%%%%%%%%%%%%%%%%%%%%%%%%%%%%%%%%%%%%
%%%%%%%%PAGEBREAK%%%%%%%PAGEBREAK%%%%%%%%%
%%%%%%%%%%%%%%%%%%%%%%%%%%%%%%%%%%%%%%%%%%
%%%%%%%%%%%%%%%%%%%%%%%%%%%%%%%%%%%%%%%%%%
%%%%%%%%%%%%%%%%%%%%%%%%%%%%%%%%%%%%%%%%%%
%%%%%%%%%%%%%%%%%%%%%%%%%%%%%%%%%%%%%%%%%%
%%%%%%%%PAGEBREAK%%%%%%%PAGEBREAK%%%%%%%%%
%%%%%%%%%%%%%%%%%%%%%%%%%%%%%%%%%%%%%%%%%%
%%%%%%%%%%%%%%%%PAGEBREAK%%%%%%%%%%%%%%%%%
%%%%%%%%%%%%%%%%%%%%%%%%%%%%%%%%%%%%%%%%%%
%%%%%%%%PAGEBREAK%%%%%%%PAGEBREAK%%%%%%%%%
%%%%%%%%%%%%%%%%%%%%%%%%%%%%%%%%%%%%%%%%%%
%%%%%%%%%%%%%%%%%%%%%%%%%%%%%%%%%%%%%%%%%%
%%%%%%%%%%%%%%%%%%%%%%%%%%%%%%%%%%%%%%%%%%
%%%%%%%%%%%%%%%%%%%%%%%%%%%%%%%%%%%%%%%%%%
%%%%%%%%PAGEBREAK%%%%%%%PAGEBREAK%%%%%%%%%
%%%%%%%%%%%%%%%%%%%%%%%%%%%%%%%%%%%%%%%%%%
%%%%%%%%%%%%%%%%PAGEBREAK%%%%%%%%%%%%%%%%%
%%%%%%%%%%%%%%%%%%%%%%%%%%%%%%%%%%%%%%%%%%
%%%%%%%%PAGEBREAK%%%%%%%PAGEBREAK%%%%%%%%%
%%%%%%%%%%%%%%%%%%%%%%%%%%%%%%%%%%%%%%%%%%
%%%%%%%%%%%%%%%%%%%%%%%%%%%%%%%%%%%%%%%%%%
\begin{ekdosis}
    \ekddiv{type=ed}
  \begin{prose}
    \noindent
\extra{sthitaṃ padmaṃ nālaṃ tasya
        \app{\lem[type=emendation, resp=egoscr]{daśāṅgulaṃ}
          \rdg[wit={U2}]{\korr daśāgulaṃ}}/ %In the middle of the navel [exists] a place, being a lotus, its tube measures ten \textit{aṅgula}s,
        komalaṃ tasya tan-nālaṃ nirmalaṃ cāpy-adhomukhaṃ/ %The fluid (\textit{komala}) of the tube is pure facing upwards.
        kadalīpuṣpasaṃkāśaṃ tanmadhye ca pratiṣṭhitaṃ/ % In its middle is a place shining like a banana-flower.
        mana \app{\lem[type=conjecture, resp=egoscr, alt={ānati}]{āna\skp{ty-a}}
          \rdg[wit={U2}]{\conj unnaty}}
      \app{\lem[type=emendation, resp=egoscr,alt={asaṃkalpam}]{\skm{ty-a}saṃkalpam}
          \rdg[wit={U2}]{\korr asaṃkalpa}}/  
        vikalpātmakam-eva ca/} %The mind isn't willing to rise up and is of changing nature.     
%%%%%%%%%%
%%%%%%%%%%
%%%%%%%%%%%%
%%%%%%%%%%%%%%
%%%%%%%%%%%%%%%
        \extra{
          pūrvadale \app{\lem[type=emendation, resp=egoscr, alt={°śveta}]{śveta}
            \rdg[wit={U2}]{\korr sveta°}}varṇe yadā \app{\lem[type=emendation, resp=egoscr]{viśramate}
            \rdg[wit={U2}]{\korr viśrāmate}} manaḥ\dd{}
        dharmakīrtividyādisadbuddhir-bhavati/ %While the mind rests on the eastern petal [which is] white in colour clear intellekt arises, which is [endowed with]  \textit{dharma}, fame and knowledge etc. 
        %%%%%
        agnikoṇe āraktavarṇe \app{\lem[type=emendation, resp=egoscr, alt={nidrālasya}]{nidrālasya}
          \rdg[wit={U2}]{\korr nidrā ālasya°}}māyāmandamatir-bhavati/  %While [the mind rests on] the south-east, [which is] reddish in color a mind that is weak due to sleep, laziness and illusion arises.
        %%%%
        dakṣiṇe kṛṣṇavarṇeti tadā krodhotpattir-bhavati/ %While [the mind is situated] in the right south, [which is] black in color the generation of anger arises.
        %%%
        \app{\lem[type=emendation, resp=egoscr]{nairṛtye}
          \rdg[wit={U2}]{\korr naiṛtye}} nīlavarṇe mamatāmatir-bhavati/ %While [the mind is situated] in the southwest, [which is] blue in color a mind of pride arises.
        %%%
        paścime kapilavarṇe krīḍāhāsotsavotsāhamatir-bhavati/ %While [the mind is situated] in the west, [which is] brown in color a mind that is longing for play, laughing, and celebration arises.
        %%%
        vāyavye \app{\lem[type=emendation, resp=egoscr, alt={°śyāma}]{śyāma}
          \rdg[wit={U2}]{\korr śāma}}varṇe cintodvegamatir-bhavati/ %While [the mind is situated] in the northwest, [which is] dark in color a mind which is restless by sorrow arises.
        %%%
        uttare pītavarṇe bhogaśṛṅgāramahodayamatir-bhavati/ %While [the mind is situated] in the north, [which is] yellow in color a very happy mind with erotic and enjoyment arises.
        īśāne gauravarṇe
        \app{\lem[type=emendation, resp=egoscr, alt={jñānasaṃdhāna°}]{jñānasaṃdhāna}
          \rdg[wit={U2}]{\korr jñānasaṃdhāne}}
        matir-bhavati/} \\%While [the mind is situated] in north-east [which is] whitish in color a mind of unity arises through knowledge arises.
  %%%%%%%%%%%%
  %%%%%%%%%%%%
  %%%%%%%%%%%%
  %%%%%%%%%%%%
  %%%%%%%%%%%%
%-----------------------
% \om                                                     \B      
%tanmadhye prāṇavāyoḥ sthānam    aṣṭadalakamalamadhye liṃgākārā karṇikā  kathyate/  \E 
%tanmadhye prāṇavāyoḥ sthānam    aṣṭadalakamalamadhye liṃgākārā karṇikā  kathyate/  \P
%tanmadhye prāṇavāyoḥ sthānam    aṣṭadalakamalamadhye liṃgākārā karṇikā  kathyate// \L
%tanmadhye prāṇavāyoḥ sthānam    aṣṭadalakamalamadhye liṃgākārā karṇikā  kathyate// \N1
%tanmadhye prāṇavāyoḥ sthānam/   aṣṭadalakamalamadhye liṃgākārā karṇikā  kathyate// \N2
%tanmadhye prāṇavāyoḥ sthānam // aṣṭadalakamalamadhye liṃgākārā karṇi    kathyate// \D
%ta ca     prāṇavāyoḥ sthānam /  aṣṭadalakamalamadhye liṃgākārā karṇikā              \D2      
%tanmadhye prāṇavāyo  sthānam    aṣṭadalakamalamadhye liṃgākārā karṇikā  kathyate    \U1
%tanmadhye prāṇavāyo  sthānam // aṣṭadalakamalamadhye liṃgākārā karṇikā  kathyate    \U2
%-----------------------
%It's said that in its middle is the place of the \textit{prāṇa}-vitalwind [and] in the middle [of] the eight-petalled lotus is a pericarp (\textit{karṇikā}) in the form of a \textit{liṅga}.
%-----------------------
      \note[type=testium, labelb=69, lem={prāṇavāyoḥ }]{Ysg: ta ca prāṇavāyoḥ sthānam | aṣṭadalakamalamadhye liṃgākārā karṇikā}
      \note[type=source, labelb=70, lem={prāṇavāyoḥ}]{Ysv (PT): prāṇavāyoḥ sthalañcāsya liṅgākāran tu karṇikā | kālikākhyā karṇikeyaṃ asyā madhye tu kuṇḍalī |}
tanmadhye prāṇa\app{\lem[wit={ceteri},alt={°vāyoḥ}]{vāyoḥ}
       \rdg[wit={U1,U2}]{°vāyo}} sthānam-aṣṭadalakamalamadhye liṃgākārā
        \app{\lem[wit={ceteri}]{karṇikā}
          \rdg[wit={U2}]{karṇi}}
kathyate/   
%-----------------------
% \om                                                     \B
%tasyāḥ karṇiketi saṃjñā tatkarṇikāmadhye    padmarāgasamānavarṇāṃ-----------guṣṭhapramāṇaikā     puttalikā varttate //  \E  
%tasyāḥ kaliketi saṃjñā tatkalikāmadhye      padmarāgaratnasamānavarṇāṃ    aṃguṣṭhapramāṇā    ekā puttalikā varttate     \P
%tasyāḥ kalikeli                 madhye      padma    ratnasamānavarṇā //  aṃguṣṭhapramāṇā // ekā puttalikā varttate //  \L
%tasyāḥ kaliketi saṃjñā tatkalikāmadhye      padmarāgaratnasamānavarṇāṃ    aṃguṣṭhapramāṇā    ekā puttalikā varttate     \N1
%tasyāḥ kaliketi saṃjñā/tataḥ kalikāmadhye   padmarāgaratnasamānavarṇa     aṃguṣṭhapramāṇā    ekā putalikā  varttate/    \N2 %%%p4recto
%tasyāḥ kaliketi saṃjñā tatkalikāmadhye      padmarāgaratnasamānavarṇā     aṃguṣṭhapramāṇāt   ekā puttalikā varttate /   \D
%tasyāḥ kaliketi saṃjñā tatkalikāmadhye      padmarāgaratnasamānavarṇā     aṃguṣṭhapramāṇāt   ekā puttalikā varttate /   \U1
%tasyāḥ kaliketi saṃjñā tatkalikāmadhye      padmarāgaratnasamānavarṇā  // aṃguṣṭhapramāṇā    ekā puttalikā varttate /   \U2
%-----------------------
%The technical designation of her is kalikā. In the middle of this kalikā exists a single thumbsized (divine) figurine (puttalikā) being similiar to a ruby-gem in color.
%-----------------------        
\note[type=testium, labelb=71, lem={kaliketi}]{Ysg: kaliketi saṃjñikāsti tanmadhye padmarāgaratnasamānavarṇā aṃguṣṭhapramāṇā ekā puttalikā}
\note[type=source, labelb=72, lem={padma°}]{Ysv (PT): padmavatyāḥ [padmāvatyāḥ (YK)] prabhāṅguṣṭhapramāṇā [°prāmāṇa° (YK)] ratnasannibhā | tasyāsaṅgī [tasya saṅgī (YK)] jīva iti ananto balarūpataḥ | asya dhyānaṃ [dhyānād (YK)] jagadvaśyaṃ khecarīsarvago bhavet | bhavanti vaśyā devādyāś cintākartturna [citta° (YK)] cānyathā | iṣṭāniṣṭo [iṣṭāniṣṭa (YK)] bhaved vaśyaḥ [vaśyaṃ (YK)] satyaṃ satyaṃ na saṃśayaḥ | iṣṭasiddhir bhavet tasya sarvajñādiguṇodayaḥ |}
tasyāḥ
\app{\lem[wit={ceteri}]{kaliketi}
  \rdg[wit={L}]{kalikeli}
  \rdg[wit={E}]{karṇiketi}}
\app{\lem[wit={ceteri}]{saṃjñā}
  \rdg[wit={L}]{\om}}
\app{\lem[wit={ceteri}]{tatkalikāmadhye}
  \rdg[wit={N2}]{tataḥ}
  \rdg[wit={L}]{\om}}
\app{\lem[type=emendation, resp=egoscr]{padmarāgaratnasamānavarṇāṅguṣṭhapramāṇaikā}
  \rdg[wit={E}]{\korr padmarāgasamānavarṇāṃguṣṭhapramāṇaikā}
  \rdg[wit={P,N1}]{padmarāgaratnasamānavarṇāṃ || aṃguṣṭhapramāṇā || ekā}
  \rdg[wit={N2}]{padmarāgaratnasamānavarṇa aṃguṣṭhapramāṇā ekā}
  \rdg[wit={L}]{padmaratnasamānavarṇā aṃguṣṭhapramāṇā ekā}
  \rdg[wit={D,U1}]{padmarāgaratnasamānavarṇā aṃguṣṭhapramāṇāt ekā}} puttalikā
vartate/
%-----------------------
%
%tasyā  jīvasaṃjñā          tasyā  balamadhyasvarūpaṃ        koṭijihvābhir  vaktuṃ naiva śakyate // \E
%tasyā  jīvasaṃjñā          tasyā  balam atha svarūpaṃ       koṭijihvābhir  vaktuṃ naiva śakyate // \P 
%tasya                             bala sappa svarūpaṃ       koṭijihvāyābhi vaktuṃ na    śakyate // \L 
%tasyāḥ jīveti saṃjñāḥ      tasyāḥ balaṃ atha ca svarūpaṃ    koṭijihvābhir  vaktuṃ na    śakyate // \N1
%tasyāḥ jīveti saṃjñaḥ//    tasyā  balaṃ atha ca svarūpaṃ    koṭijihvābhir  vaktuṃ na    śakyate // \N2
%tasyāḥ jīveti saṃjña/      tasyāḥ balaṃ atha ca svarūpaṃ    koṭijihvābhir  vaktuṃ na    śakyate // \D
%       jīveti saṃjñikāsti/ tasyāḥ balaṃ         svarūpaṃ ca koṭijihvābhir  vaktuṃ na    śakyaṃ  //  D2
%tasyāḥ jīveti saṃjñā       tasyāḥ balaṃ atha ca svarūpaṃ    koṭijihvābhir  vaktuṃ na    śakyate // \U1
%tasyā  jīvasaṃjñā//        tasya  balaṃ tasya atha svarūpaṃ koṭijihvābhir  vaktuṃ na    śakyate // \U2
%-----------------------  
%Her technical designation is embodied soul. Not even with a thousand tongues it is possible to talk about her nature and her power.
%-----------------------
\note[type=testium, labelb=73 lem={jīveti}]{Ysg: jīveti saṃjñikāsti | tasyāḥ balaṃ svarūpaṃ ca koṭijihvābhir vaktuṃ na śakyaṃ ||}
\app{\lem[wit={E,P}]{tasyā}
     \rdg[wit={N1,N2,D,U1}]{tasyāḥ}
     \rdg[wit={L}]{tasya}}
\app{\lem[wit={U2}]{jīveti saṃjñā}
     \rdg[wit={N1}]{jīveti saṃjñāḥ}
     \rdg[wit={N2}]{jīveti saṃjñaḥ ||}
     \rdg[wit={D}]{jīveti saṃjña |}
     \rdg[wit={E,P,U2}]{jīvasaṃjñā ||}
     \rdg[wit={L}]{\om}}
\app{\lem[wit={E,N2,P}]{tasyā}
     \rdg[wit={N1,D,U1}]{tasyāḥ}
     \rdg[wit={U2}]{tasya}}
\app{\lem[wit={ceteri}]{balaṃ atha ca svarūpaṃ}
     \rdg[wit={P}]{balam atha svarūpaṃ}
     \rdg[wit={U2}]{balaṃ tasya atha svarūpaṃ}
     \rdg[wit={L}]{bala sappa svarūpaṃ}
     \rdg[wit={E}]{balamadhyasvarūpaṃ}}
\app{\lem[wit={ceteri}, alt={koṭijihvābhir}]{koṭijihvābhi\skp{r-va}}
    \rdg[wit={L}]{koṭijihvāyābhi}}\skp{r-va}ktuṃ
\app{\lem[wit={ceteri}]{na}
    \rdg[wit={E,P}]{naiva}}
śakyate/
%-----------------------
% \om \B
%asyā  mūrter   dhyānakāraṇāt           svarga-pātāl--ākaśamanuṣyagandharvakinnaraguhyakavidyādharalokasambandhinyaḥ strīyo 'pi--------------------       vaśyā bhavanti / \E
%asyā  mūrter   dhyānakaraṇāt           svarga-pātāl--ākāśamanuṣyagandharvakiṃnaraguhyakavidyādharalokasaṃbaṃdhinyaḥ strīyo 'pi--------------------       vaśyā bhavanti / \P
%asyā  mūrtir   dhyānāt                 svarga-pātāl--ākāśamanuṣyagaṃdharvakinnaraguhyakavidyādharalokasambandhinyaḥ strīyo 'pi--------------------       vaśyā bhavanti /L
%asyāḥ mūrter  dhyānakaraṇāt            svarga-pātāla ākāśamanuṣyagaṃdharvakinnaraguhyakavidyādharalokasaṃbaṃdhinyaḥ strīyaḥ sādhakasya puruṣasya         vaśyā bhavanti // \N1
%asyā  mūrttir dhyānakaraṇāt/           svarga-pātāla ākāśamanuṣya/ gaṃdharvakinnara/ guhyaka/vidyādhara/lokasaṃbaṃdhinyaḥ strīyaḥ sādhakasya puruṣasya   vaśyo bhavati/ \N2
%asyāḥ mūrter  dhyānakaraṇāt            svarga-pātāla ākāśamanuṣyagaṃdharvakiṃnaraguhyakavidyādharalokasaṃbaṃdhinyaḥ strīyaḥ sādhakasya puruṣasya         vaśyā bhavanti // \D
%asyāḥ mūrter  dhyānakaraṇāt            svarga-pātāla ākāśamanuṣyagaṃdharvakiṃnaraguhyakavidyādharalokasaṃbaṃdhinyaḥ strīyaḥ sādhakasya puruṣasya         vaśyā bhavanti // \U1
%tasyāḥ mūrter dhyānaṃ karaṇāt //       svarga-pātāl--ākāśamanuṣyagandharvakinnaraguhyakavidyādharalokasaṃbadhinya---striyo  pi---------------------------vaśyā bhavaṃti // \U2
%-----------------------
%“Because of the exercise of meditation on this form the inhabitants of the universe (which are) Humans, Gandharvas, Kinnaras, Guhyakas, Vidyādharas and (their) females, in the heavenly world, underworld and open space are obedient to the will of the practicing person.”, is what is said here.  
%-----------------------
\note[type=testium, labelb=74, lem={svarga°}]{Ysg: :asyā mūrtter dhyānakaraṇāt sādhakasya svargapātāla ākāśagaṃdharvakiṃnaraguhyakavidyādharastrīyo vaśā bhavati |}
\app{\lem[wit={ceteri}]{asyā}
    \rdg[wit={N1,D,U1}]{asyāḥ}
    \rdg[wit={U2}]{tasyāḥ}}
 \app{\lem[wit={ceteri}, alt={mūrter}]{mūrte\skp{r-dhyā}}
    \rdg[wit={L,N2}]{mūrtir}}\app{\lem[wit={ceteri}, alt={dhyānakāraṇāt}]{\skm{r-dhyā}nakāraṇā\skp{t-sva}}
    \rdg[wit={U2}]{dhyānaṃ karaṇāt ||}
    \rdg[wit={L}]{dhyānāt}
  }\skm{t-sva}rga\app{\lem[wit={E,P,L,U2},alt={°pātālākaśa°}]{pātālākaśa}
    \rdg[wit={D,N1,N2,U1}]{°pātāla ākāśa°}}\app{\lem[wit={ceteri},alt={°manuṣyagandharvakinnaraguhyakavidyādharaloka°}]{manuṣyagandharvakinnaraguhyakavidyādharaloka}
    \rdg[wit={N2}]{°manuṣya| gaṃdharvakinnara| guhyaka| vidyādhara| loka°}
  }\app{\lem[wit={ceteri}]{saṃbandhinyaḥ}
    \rdg[wit={U2}]{saṃdadhinya}}
 \app{\lem[wit={ceteri}]{strīyaḥ sādhakasya puruṣasya}
    \rdg[wit={E,P,L}]{strīyo 'pi}
    \rdg[wit={U2}]{striyo pi}}
 \app{\lem[wit={ceteri}]{vaśyā bhavanti}
   \rdg[wit={N2}]{vaśyo bhavati}}/
\note[type=philcomm, labelb=75, lem={vaśyā bhavanti}]{D\textsubscript{2} adds: pṛthvī loke manuṣyādi striṇāṃ kākathā cha |}
%tanmadhye koṭicaṃdrasamaprabhaḥ ekaḥ puruṣo varttate  \N1bhavanti/\note[type=philcomm, labelb=s16, lem={bhavanti}]{\getsiglum{U1} adds a flawed phrase hereafter: \textit{pṛtvī lokasaṃbaṃdhanyo pi striyaḥ vaśyā bhavaṃti/}. I refrained to include it in the apparatus due to its redundance.}
%-----------------------
% \om \B 
%ityatra kathyate// /E
%ityatra kathyate// \P
%ityatra kathyate// \L
%ityatra kiṃ kathyate // \N1
%ityatra kiṃ kathyate// \N2
%ityaṃtra kiṃ kathyate // \D
%ityatra kiṃ kathyate vā \U1
%ityatra kathyate // \U2
%-----------------------
%is what is said here.  
%-----------------------  
ity-atra  
\app{\lem[wit={ceteri}]{kiṃ}
  \rdg[wit={E,P,L,U2}]{\om}}
\app{\lem[wit={ceteri}]{kathyate}
  \rdg[wit={U1}]{kathyate vā}}\dd{}
\end{prose}
\end{ekdosis}
\ekdpb*{}
%%%%%%%%%%%%%%%%%%%%%%%%%%%%%%%%%%%%%%%%%%
%%%%%%%%%%%%%%%%%%%%%%%%%%%%%%%%%%%%%%%%%%
%%%%%%%%PAGEBREAK%%%%%%%PAGEBREAK%%%%%%%%%
%%%%%%%%%%%%%%%%%%%%%%%%%%%%%%%%%%%%%%%%%%
%%%%%%%%%%%%%%%%PAGEBREAK%%%%%%%%%%%%%%%%%
%%%%%%%%%%%%%%%%%%%%%%%%%%%%%%%%%%%%%%%%%%
%%%%%%%%PAGEBREAK%%%%%%%PAGEBREAK%%%%%%%%%
%%%%%%%%%%%%%%%%%%%%%%%%%%%%%%%%%%%%%%%%%%
%%%%%%%%%%%%%%%%%%%%%%%%%%%%%%%%%%%%%%%%%%
%%%%%%%%%%%%%%%%%%%%%%%%%%%%%%%%%%%%%%%%%%
%%%%%%%%%%%%%%%%%%%%%%%%%%%%%%%%%%%%%%%%%%
%%%%%%%%PAGEBREAK%%%%%%%PAGEBREAK%%%%%%%%%
%%%%%%%%%%%%%%%%%%%%%%%%%%%%%%%%%%%%%%%%%%
%%%%%%%%%%%%%%%%PAGEBREAK%%%%%%%%%%%%%%%%%
%%%%%%%%%%%%%%%%%%%%%%%%%%%%%%%%%%%%%%%%%%
%%%%%%%%PAGEBREAK%%%%%%%PAGEBREAK%%%%%%%%%
%%%%%%%%%%%%%%%%%%%%%%%%%%%%%%%%%%%%%%%%%%
%%%%%%%%%%%%%%%%%%%%%%%%%%%%%%%%%%%%%%%%%%
%%%%%%%%%%%%%%%%%%%%%%%%%%%%%%%%%%%%%%%%%%
%%%%%%%%%%%%%%%%%%%%%%%%%%%%%%%%%%%%%%%%%%
%%%%%%%%PAGEBREAK%%%%%%%PAGEBREAK%%%%%%%%%
%%%%%%%%%%%%%%%%%%%%%%%%%%%%%%%%%%%%%%%%%%
%%%%%%%%%%%%%%%%PAGEBREAK%%%%%%%%%%%%%%%%%
%%%%%%%%%%%%%%%%%%%%%%%%%%%%%%%%%%%%%%%%%%
%%%%%%%%PAGEBREAK%%%%%%%PAGEBREAK%%%%%%%%%
%%%%%%%%%%%%%%%%%%%%%%%%%%%%%%%%%%%%%%%%%%
%%%%%%%%%%%%%%%%%%%%%%%%%%%%%%%%%%%%%%%%%%
\begin{ekdosis}
  \ekddiv{type=ed}
    \centerline{\textrm{\small{[Fifth Cakra]}}}
    \bigskip  
    \begin{prose}
\noindent
%-----------------------      
%-------------pañcamaṃ kaṇṭhasthāne ṣoḍaśadalaṃ kamalaṃ                         vartate //  \E
%-------------paṃcamaṃ kaṃṭhasthāne ṣoḍaśadalaṃ kamalaṃ                         vartate     \P
%-------------paṃcamaṃ kaṃṭhasthāne ṣoḍaśadalaṃ kamalaṃ                         vartate     \L
%idānīṃ       paṃcamaṃ kamalaṃ      ṣodaśadalaṃ                   kaṃṭhasthāne varttate // \N1
%idānīṃ       paṃcamaṃ kamalaṣodaśadalaṃ                          kaṃṭhasthāne varttate // \N2
%idānīṃ       paṃcamaṃ kamalaṃ      ṣodaśadalaṃ                   kaṃṭhasthāne  varttate // \D --------> Was in diesem Falle machen?
%idānīṃ       paṃcamaṃ kamalaṃ      ṣodaśadalaṃ                   kaṃṭhasthāne        varttate // \U1
%-------------paṃcamaṃ                          viśuddhacakraṃ    kaṃṭhastāne              \U2     
%-----------------------
%Now (follows the description of) the fifth lotus having sixteen petals (which) exists at the location of the throat.
%-----------------------
%U2 continues: dhūmra?varṇe jīvodevatā// avidyāśaktiḥ// virāṭrṣiḥ// vāyurvāhanaṃ// udānavāyuḥ// jvālākalā jālaṃdharobaṃdhaḥ mahākāraṇadeha// tūryāvasthā// parāvācā// atharvaṇavedaḥ// jaṃgamaliṅgaṃ jīvaprāptābhūmikā// sāyujyatāmokṣaḥ// ṣoḍaśadalāni// ṣoḍaśamātrāḥ// atarmātrār-carāḥ// aṃ āṃ iṃ īṃ u ūṃ ṛṃ ṝṃ ḷṃ ḹṃ eṃ aiṃ oṃ auṃ aṃ aṃḥ// bahirmātrā vidyā// avidyā// ichā// śakti// jñānaśaktiḥ// śatalā// mahāvidyā// mahāmāyā// buddhiḥ// tamasī// maitrā?// kumārī// maitrāyaṇī// rudrā// puṣṭa// siṃhanī// ajapājapasahasra/ 1000 gha. 2 pa. 46 akṣara 40//
%-----------------------     
%The colour is smoke-colour. The deity is the embodied soul (\textit{jīva}). The power is ignorance (\textit{avidyā}). The Ṛṣi is Virāṭ\footnote{Who is this?}. The mount is the vitalwind (\textit{vāyu}). The vitalwind is \textit{udāna}. Its Kalā is the flame. The \textit{bandha} is Jālandhara. The body supra-causal (\textit{mahākāraṇa}). The state is the fourth state (\textit{tūrya}). The speech is Parā\footnote{Im Kaśm. Śiv. °das ewige Wort, in welchem potentiell alle Begriffe und Worte ruhen; vgl. das śabdabrahma des Vyākaraṇa. [B.]― Schmidt S. 246}. The [Veda is] Atharvaṇa Veda. The \textit{liṅga} is the living. The level is Jīvaprāptā\footnote{What is this?}. The liberation is absorption into the divine essence (\textit{sāyujyatā}). [There are] sixteen petals [and] sixteen matrices. The internal matrix: aṃ āṃ iṃ īṃ u ūṃ ṛṃ ṝṃ ḷṃ ḹṃ eṃ aiṃ oṃ auṃ aṃ aṃḥ. The external matrix: Vidyā ``she who is knowledge'', Avidyā ``she who is ignorance'', Icchā ``she who is desire'', Śakti ``she who is power'', Jñānaśakti ``she who is the power of knowledge'', Śatalā ``she who is manifold'', Mahāvidyā ``she who is great knowledge'', Mahāmayā ``she who is great illusion'', Buddhi ``she who is intellect'', Tamasī ``she who is darkness'', Maitrā ``she who is love'', Kumārī ``she who is a young girl'', Maitrāyaṇī ``she who is???'', Rudrā ``she who is howling'', Puṣṭā ``she who is abundance'', Siṃhanī ``she who is a lioness''. A thousandfold recitation of the non-recited; 1000 [repetitions for]; 2 \textit{ghaṭi}s, 46 \textit{palā}s. and 40 \textit{akṣara}s.
%-----------------------  
\note[type=testium, labelb=76, lem={paṃcamaṃ}]{Ysg: kaṃṭhasthāne paṃcamaṃ ṣodaśadalaṃ viśudhhasaṃjñakaṃ cakraṃ varttate ||}
\note[type=source, labelb=77, lem={paṃcamaṃ}]{Ysv (PT=YK): iṣṭasiddhir bhavet tasya sarvajñādiguṇodayaḥ | kalāpatraṃ pañcaman tu viśuddhaṃ kaṇṭhadeśataḥ | asya madhye pumān ekaḥ koṭicandrasamaprabhaḥ | naśyantya sādhyarogā hi sahasrāyuś ca cintanāt |}
\app{\lem[wit={N1,N2,D,U1}]{idānīṃ}
        \rdg[wit={ceteri}]{\om}}
pañcamaṃ
\app{\lem[wit={N1,D,U1}]{kamalaṃ ṣodaśadalaṃ kaṇṭhasthāne}
  \rdg[wit={N2}]{kamalaṣodaśadalaṃ kaṇṭhasthāne}
  \rdg[wit={E,P,L}]{kaṇṭhasthāne ṣoḍaśadalaṃ kamalaṃ}
  \rdg[wit={U2}]{viśuddhacakraṃ kaṃṭhastāne}}
vartate/
  %%%%%%%%%%%
  %%%%%%%%%%%
  %%%%%%%%%%%
  %%%%%%%%%%%
  %%%%%%%%%%%
      \extra{\app{\lem[type=emendation, resp=egoscr]{dhūmraṃ varṇaṃ}
          \rdg[wit={U2}]{\korr dhūmravarṇe}}\dd{}
        jīvo devatā\dd{}
        avidyā śaktiḥ\dd{}
        \app{\lem[type=emendation, resp=egoscr]{virāṭ}
          \rdg[wit={U2}]{\korr virāṭha}} ṛṣiḥ\dd{}
        vāyur-vāhanaṃ\dd{}
        \app{\lem[type=emendation, resp=egoscr]{udāno}
          \rdg[wit={U2}]{\korr udāna°}} vāyuḥ\dd{}
        jvālā kalā\dd{}
        jālaṃdharo bandhaḥ\dd{}
        \app{\lem[type=emendation, resp=egoscr]{mahākāraṇaḥ dehaḥ}
          \rdg[wit={U2}]{\korr mahākāraṇadeha}}\dd{}
        \app{\lem[type=emendation, resp=egoscr]{tūrya āvasthā}
          \rdg[wit={U2}]{\korr tūryāvasthā}}\dd{}
        parā vācā\dd{}
        \app{\lem[type=emendation, resp=egoscr]{atharvaṇo}
          \rdg[wit={U2}]{\korr atharvaṇa}} vedaḥ\dd{}
        \app{\lem[type=emendation, resp=egoscr]{jaṅgamaṃ}
          \rdg[wit={U2}]{\korr jaṃgama°}} liṅgaṃ\dd{}
        jīvaprāptā bhūmikā\dd{}
        sāyujyatā mokṣaḥ\dd{}
        ṣoḍaśadalāni\dd{}
        ṣoḍaśamātrāḥ\dd{}
        \app{\lem[type=emendation, resp=egoscr]{antarmātrā}
          \rdg[wit={U2}]{\korr antarmātrār carāḥ}}\dd{}  %%%what does carā here mean? I emend to the formulation found for the U2 additions in the previous cakra 
        aṃ āṃ iṃ īṃ u ūṃ ṛṃ ṝṃ ḷṃ ḹṃ eṃ aiṃ oṃ auṃ aṃ aṃḥ\dd{}
        bahir-mātrā\dd{}
        vidyā\dd{}
        avidyā\dd{}
        \app{\lem[type=emendation, resp=egoscr]{icchā}
          \rdg[wit={U2}]{\korr ichā}}\dd{}
        \app{\lem[type=emendation, resp=egoscr]{śaktiḥ}
          \rdg[wit={U2}]{\korr śakti}}\dd{}
        jñānaśaktiḥ\dd{}
        śatalā\dd{}
        mahāvidyā\dd{}
        mahāmāyā\dd{}
        buddhiḥ\dd{}
        \app{\lem[type=emendation, resp=egoscr]{tāmasī}
          \rdg[wit={U2}]{\korr tamasī}}\dd{} %%%She who is darkness????
        maitrā\dd{}
        kumārī\dd{}
        maitrāyaṇī\dd{} %%%what's this??? 
        rudrā\dd{}
        \app{\lem[type=emendation, resp=egoscr]{puṣṭā}
          \rdg[wit={U2}]{\korr puṣṭa°}}\dd{}
        siṃhanī\dd{}
        \app{\lem[type=emendation, resp=egoscr]{ajapājapaḥ sahasraḥ}
          \rdg[wit={U2}]{\korr ajapājapasahasra}}\dd{} 1000\dd{} gha. 2 pa. 46 akṣara 40\dd{}}%%%%%Kolloquium besprechen! Was is akṣara? 
 %%%%%%%%%%%
 %%%%%%%%%%%
 %%%%%%%%%%%
 %%%%%%%%%%%
 %%%%%%%%%%%
%----------------------- 
%tanmadhye koṭisūryasamāna       ekaḥ puruṣo vartate / \E
%tanmadhye koṭicaṃdrasamaprabhaḥ ekaḥ puruṣo vartate   \P
%tanmadhye koṭicaṃdrasamaprabhā  ekaḥ puruṣo vartate   \L
%tanmadhye koṭicaṃdrasamaprabhaḥ ekaḥ puruṣo varttate  \N1
%tanmadhye koṭicaṃdrasamaprabhaḥ ekaḥ puruṣo varttate  \N2
%tanmadhye koṭicaṃdrasamaprabhā  eka--puruṣo varttate  \D
%tatra     koṭicaṃdraprabha      ekaḥ puruṣo sti       \D2
%tanmadhye koṭicaṃdrasamaprabhaḥ ekaḥ puruṣo varttate  \U1
%tanmadhye koṭicaṃdrasamaprabhaḥ // eka pumān varttate // \U2
%----------------------- 
%In its  middle exists a single person which shines like a thousand moons.
%----------------------- 
\note[type=testium, labelb=78, lem={koṭicaṃdra°}]{Ysg: tatra koṭicaṃdraprabha ekaḥ puruṣo sti}
 tanmadhye
koṭi\app{\lem[wit={ceteri}, alt={°candrasamaprabhaḥ}]{candrasamaprabhaḥ}
  \rdg[wit={U2}]{°caṃdrasamaprabhaḥ ||}
  \rdg[wit={L,D}]{°caṃdrasamaprabhā}
  \rdg[wit={E}]{°caṃdrasūryasamāna}}
\app{\lem[wit={ceteri}]{ekaḥ puruṣo}
  \rdg[wit=D]{ekapuruṣo}
  \rdg[wit={U2}]{eka pumān}}
vartate/
%----------------------- 
%tasya puruṣasya dhyānakāraṇād--- asādhyarogā naśyanti // \E
%tasya puruṣasya dhyānakāraṇād--- asādhyarogā naśyanti // \L
%tasya puruṣasya dhyānakāraṇād--- asādhyarogā naśyaṃti // \P
%tasya puruṣasya dhyānakaraṇāt--  asādhyarogā naśyaṃti // \N1
%tasya puruṣasya dhyānakaraṇāt    asādhyarogā naśyaṃti    \N2
%tasya puruṣasya dhyānakaraṇāt /  asādhyarogā naśyaṃti // \D
%tasya puruṣasya dhyānakaraṇāt /  asādhyarogā naśyaṃti    \U1
%tasya puṃsaḥ    dhyānakaraṇāt // asādhyarogā naśyaṃti // \U2
%----------------------- 
%Because of the exercise of meditation on this person all diseases which are (otherwise) not possible to be controlled vanish.
%----------------------- 
\note[type=testium, labelb=79, lem={asādhyarogā}]{Ysg: tasya puruṣasya dhyānakaraṇād asādhyarogā naśyaṃti ||}
tasya
\app{\lem[wit={ceteri}]{puruṣasya}
  \rdg[wit={U2}]{puṃsaḥ}}
\app{\lem[wit={ceteri}, alt={dhyānakāraṇād}]{dhyānakaraṇā\skp{d-a}}
  \rdg[wit={N1,N2}]{dhyānakaraṇāt}
  \rdg[wit={D,U1,U2}]{dhyānakaraṇāt |}}\skm{d-a}sādhyarogā naśyanti/
%----------------------- 
%ekasahasravarṣaparyaṃtaṃ sa puruṣo jīvatīdānīṃ     \E
%ekasahasravarṣaparyaṃtaṃ sa puruṣo jīvati          \P
%ekasahasravarṣa             puruṣo jīvati //       \L
%ekasahasravarṣaparyaṃtaṃ    puruṣo jīvati /        \N1
%ekasahasravarṣaparyaṃta     puruṣo jīvati /        \N2
%ekasahasravarṣaparyaṃtaṃ    puruṣo jīvati /        \D
%ekasahasravarṣaparyaṃtaṃ    puruṣo jīvati / cha    \U1
%ekasahasravarṣaparyaṃtaṃ    puruṣo jīvati //       \U2
%----------------------- 
%The person lives up to 1001 years.
%----------------------- 
\note[type=testium, labelb=80, lem={°varṣa°}]{Ysg: sahasravarṣaṃ jīvati |}
ekasahasravarṣa\app{\lem[wit={ceteri},alt={°paryantaṃ}]{paryantaṃ}
  \rdg[wit={N2}]{°paryaṃta}
  \rdg[wit={L}]{\om}}
\app{\lem[wit={ceteri}]{puruṣo}
  \rdg[wit={E,P}]{sa puruṣo}}
  \app{\lem[wit={ceteri}]{jīvati}
    \rdg[wit={U1}]{jīvati |cha|}
    \rdg[wit={E}]{jīvatīdānīṃ}}\dd{}
    \end{prose}
  \end{ekdosis}
%%%%%%%%%%%%%%%%
%%%%%%%%%%%%%%%%
%%%%%%%%%%%%%%%%
%%%%%%%%%%%%%%%
%%%%%%%%%%%%%%%%
\begin{ekdosis}
  \ekddiv{type=ed}
   \bigskip
    \centerline{\textrm{\small{[Sixth Cakra]}}}
    \bigskip
    \begin{prose}
%----------------------- 
%īdānīṃ ṣaṣṭhaṃ bhrūmadhye ājñācakraṃ                vartate//   \E
%īdānīṃ ṣaṣṭhaṃ bhrūmadhye ājñācakraṃ                vartate//   \P
%īdānīṃ ṣaṣṭhaḥ bhrūmadhye ājñācakraṃ                vartate//   \L
%idānīṃ ṣaṣṭhacakraṃ       ajñānāmakaṃ               varttate // \N1
%idānīṃ ṣaṣṭhacakraṃ       ajñānāmaka                varttate    \N2
%idānīṃ ṣaṣṭhacakraṃ       ajñānāmakaṃ               varttate // \D
%idānīṃ ṣaṣṭhacakraṃ       ājñānāmakaṃ               vartate     \U1
%idānīṃ ṣaṣṭa   bhrūmadhye ājñācakraṃ raktavarṇaṃ //             \U2
%-----------------------
%āgnirdevatā suṣumṇāśaktiḥ// hiṃsaṛṣiḥ// caitanyavāhanaṃ// jñānadehī// vijñānāvathā// anupamavācā// sāmadevaḥ// pramādaliṃgaṃ// ardhamātrā// ākāśātatvaṃ// jīvahiṃsa// caitanyalīlraṃbhaḥ// dvemātrā// hiṃkṣaṃ// aṃtarmātrā// bahirmātrā//sthiti//prabhā?// ajapājapasahasra// 1000 gha. 2 pa. 46 akṣara 40// \U2
%-----------------------
%The deity is fire. The power is the godess of the centre (\textit{suṣumṇā}). The Ṛṣi is ``the violent'' (\textit{hiṃsa}). The mount is consciousness (\textit{caitanya}. The body is knowledge. The state is understanding. The speech is the ``incomparable'' (\textit{anupama}). The [Veda] is Sāmaveda.The \textit{liṅgaṃ} is intoxication (\textit{pramāda}). The half-measure: the reality of ether, ``the violence of living'' (\textit{jīvahiṃsa}) [and] the origin of the play of Conciousness. Two measures: haṃ kṣam. The inner measure is external measure: maintenance of life (\textit{sthiti}) [and] splendour (\textit{prabhā}).
%-----------------------
\note[type=source, labelb=81, lem={ajñā°}]{Ysv (PT): ājñākhyaṃ ṣaṣṭhakaṃ [ṣaṭkaṃ (YK)] cakraṃ bhruvor madhye dvipatrakam | agnijvālānibhaṃ jyotiḥ puṃsaḥ strīto [pūṃsastrīto (YK)] vivarjitam | dhyānāc cāsya sarvasiddhirajarāmaratāṃ vrajet |}
\note[type=testium, labelb=82, lem={ajñā°}]{Ysg: bhrūvor madhye dvidalaṃ ājñācakraṃ ṣaṣṭhaṃ |}
idānīṃ
    \app{\lem[wit={ceteri}]{ṣaṣṭhacakraṃ}
       \rdg[wit={E,P}]{ṣaṣṭhaṃ bhrūmadhye}
       \rdg[wit={L}]{ṣaṣṭhaḥ bhrūmadhye}
       \rdg[wit={U2}]{ṣaṣṭa bhrūmadhye}}
     \app{\lem[wit={ceteri}]{ājñā}
      \rdg[wit={N1,N2,D}]{ajñā}
    }\app{\lem[wit={U1,D,N1}]{nāmakaṃ}
       \rdg[wit={E,P,L}]{cakraṃ}
       \rdg[wit={U2}]{cakraṃ raktavarṇaṃ}
       \rdg[wit={N2}]{nāmaka}}
vartate/
  %%%%%%%%%%%%%%%
  %%%%%%%%%%%%%%
  %%%%%%%%%%%%%%
  %%%%%%%%%%%%%%
  %%%%%%%%%%%%%%
     \extra{\app{\lem[type=emendation, resp=egoscr, alt={agnir}]{agni\skp{r-de}}
         \rdg[wit={U2}]{\korr āgnir}
       }\skm{r-de}vatā\dd{}
       suṣumṇā śaktiḥ\dd{}
       \app{\lem[type=emendation, resp=egoscr]{hiṃso}
         \rdg[wit={U2}]{\korr hiṃsa°}} ṛṣiḥ\dd{}
       \app{\lem[type=emendation, resp=egoscr]{caitanyaṃ}
         \rdg[wit={U2}]{\korr caitanya°}} vāhanaṃ\dd{}
       \app{\lem[type=emendation, resp=egoscr]{jñāno dehaḥ}
         \rdg[wit={U2}]{\korr jñānadehī}}\dd{}
       vijñānāvasthā\dd{}
       \app{\lem[type=emendation, resp=egoscr]{anupamā}
         \rdg[wit={U2}]{\korr anupama°}} vācā\dd{}
       sāmavedaḥ\dd{}
       \app{\lem[type=emendation, resp=egoscr]{pramādaḥ}
         \rdg[wit={U2}]{\korr pramāda°}} liṃgaṃ\dd{}
       \app{\lem[type=emendation, resp=egoscr]{ardhā mātrā}
         \rdg[wit={U2}]{\korr ardhamātrā}}\dd{}}
\end{prose}
\end{ekdosis}
\ekdpb*{}
%%%%%%%%%%%%%%%%%%
%%%%%%%%%%%%%%%%%%
%%%%PAGEBREAK%%%%
%%%%%%%%%%%%%%%%%%
%%%%%%%%%%%%%%%%%%
\begin{ekdosis}
    \ekddiv{type=ed}
  \begin{prose}
\noindent
    \extra{\app{\lem[type=emendation, resp=egoscr]{ākāśaṃ}
         \rdg[wit={U2}]{\korr ākāśā}}tattvaṃ\dd{}
       \app{\lem[type=emendation, resp=egoscr]{jīvo haṃsaḥ}
         \rdg[wit={U2}]{\korr jīvahiṃsa}}\dd{}
       caitanya\app{\lem[type=emendation, resp=egoscr, alt={°līlā}]{līlā āraṃbhaḥ}
         \rdg[wit={U2}]{\korr °līlāraṃbhaḥ}}\dd{}
       dve mātrā\dd{}
       haṃ kṣaṃ\dd{}
       aṃtar-mātrā\dd{}
       bahir-mātrā\dd{}
       \app{\lem[type=emendation, resp=egoscr]{sthitiḥ}
         \rdg[wit={U2}]{\korr sthiti}}\dd{}
       prabhā\dd{}
       \app{\lem[type=emendation, resp=egoscr]{ajapājapaḥ sahasraḥ}
         \rdg[wit={U2}]{\korr ajapājapasahasra}}\dd{} 1000\dd{} gha. 2 pa. 46 akṣara 40\dd{}}     
%----------------------- 
                                       %dvidalaṃ tanmadhye  'gnijvālākārakamalaṃ     kiṃcid vastu vartate/    \E
                                       %dvidalaṃ tanmadhye  agnijvālākārakamalaṃ     kiṃcid vastu vartate/    \P
                                       %dvidalaṃ tanmadhye  agnijvālākārakamalaṃ     kiṃcid vastu vartate/    \L
%                                                           agnijvālākārakamalaṃ     kiṃcid vastu vartate/    \B
%tac cakraṃ bhruvor madhye dvidalakaṃ sthitaṃ // tanmadhye  agnijvālākāraṃ akalaṃ    kiṃcid vastu varttate/   \N1
%tac-cakraṃ bhruvor-madhye dvidalakaṃ sthitaṃ /  tanmadhye  agnijvālākāraṃ akalaṃ    kiṃcid-vastu vartate/    \N2
%tac cakraṃ bhruvor madhye dvidalakaṃ sthitaṃ // tanmadhye  agnijvālākāraṃ akalaṃ    kiṃcid vastu varttate/   \D
%tac-cakraṃ bhruvor-madhye dvidalakaṃ sthitaṃ    tanmadhye  agnijvālākāraṃ akala     kiṃcit vastu vartate/    \U1  
%                                                tanmadhye  agnijvālākārakamalaṃ //  kiṃcid-vastu varttate/   \U2   
%-----------------------    
\note[type=testium, labelb=83, lem={agnijvālā°}]{Ysg: agnijvālākāraṃ paramātmasaṃjñakaṃ vastvāsti |}
     \app{\lem[wit={ceteri}, alt={tac cakraṃ bhruvor madhye dvidalakaṃ sthitaṃ}]{tac\skp{-}cakraṃ bhruvor-madhye dvidalakaṃ sthitaṃ}
     \rdg[wit={E,P,L}]{dvidalaṃ}
     \rdg[wit={U2}]{\om}}
tanmadhye
   \app{\lem[wit={N1,N2,D}]{'gnijvālākāraṃ akalaṃ}
     \rdg[wit={ceteri}]{agnijvālākāraṃ akalaṃ}
     \rdg[wit={U1}]{agnijvālākāraṃ akala}}
   \note[type=philcomm, labelb=84, lem={agnijvālākāra°}]{Witness B starts here.}
   kiṃcid-vastu vartate/
%-----------------------  
%na strī pumān     / tasya dhyānakāraṇāt  puruṣasya  śarīraṃ  ajarāmaraṃ bhavati /     \E
%na strī pumān    // tasyā dhyānakaraṇāt  puruṣasya  śarīraṃ  ajarāmaro  bhavati /     \B
%na strī pumān    // tasyā dhyānakaraṇāt  puruṣasya  śarīraṃ  ajarāmaro  bhavati /     \L
%na strī na pumān // tasyā dhyānakaraṇāt  puruṣasya  śarīraṃ  ajarāmaro  bhavati /     \P
%na strī na pumān /  tasya dhyānakaraṇāt  puruṣasya  śarīraṃ  ajarāmaraṃ bhavati      \N1
%na strī na pumān /  tasya dhyānakaraṇāt  puruṣasya  śarīraṃ  ajarāmaraṃ bhavati //   \N2
%na strī na pumān /  tasya dhyānakaraṇāt  puruṣasya  śarīraṃ  ajarāmaraṃ bhavati      \D
%na strī na pumān    tasya dhyānakaraṇāt  puruṣasya  śarīraṃ  ajarāmaraṃ bhavati vā   \U1
%na strī na pumān /  tasya dhyānakāraṇāt/ puruṣasya--śarīram--ajarāmaraṃ bhavati /    \U2   
%-----------------------
\note[type=testium, labelb=85, lem={na strī}]{Ysg: tac ca na strīpumān | tasya dhyānakaraṇād ajarāmaraḥ sādhako bhavati |cha|}
\app{\lem[wit={ceteri}]{na strī na pumān}
     \rdg[wit={E,B,L}]{na strī pumān}}/
  tasya dhyāna\app{\lem[wit={ceteri},alt={°karaṇāt}]{karaṇāt}
    \rdg[wit={U2}]{°karaṇāt |}}
puruṣasya
  \app{\lem[wit={U2}, alt={śarīram ajarāmaraṃ}]{śarīram\skp{-}ajarāmaraṃ}
    \rdg[wit={E,N1,N2,D,U1}]{śarīraṃ ajarāmaraṃ}
    \rdg[wit={B,L,P}]{śarīraṃ ajarāmaro}}
   \app{\lem[wit={ceteri}]{bhavati}
     \rdg[wit={U2}]{bhavati vā}}\dd{}   
 \end{prose}
\end{ekdosis}
%%%%%%%%%%%%%%%
%%%%%%%%%%%%%%%
%%%%%%%%%%%%%%%
%%%%%%%%%%%%%%%
%%%%%%%%%%%%%%%
\begin{ekdosis}
  \ekddiv{type=ed}
   \bigskip
    \centerline{\textrm{\small{[Seventh Cakra]}}}
    \bigskip
    \begin{prose}
%-----------------------
% idānīṃ saptamaṃ  tālumadhye catuḥṣaṣṭidalaṃ              amṛtapūrṇaṃ vartate / \E
% idānīṃ saptamaṃ  tālumadhye catuḥṣaṣṭhidalaṃ             amṛtapūrṇaṃ vartate / \P
% idānīṃ saptamaṃ  // tāludeśe madhye catuḥṣaṣṭhidala      amṛtapūrṇaṃ vartate / \L
% idānīṃ saptamaṃ  // tāludeśe madhye catuḥṣaṣṭhidala      amṛtapūrṇaṃ vartate / \B
% idānīṃ saptamaṃ  cakraṃ     catuḥṣaṣṭhidalaṃ tālumadhye  amṛtapūrṇaṃ varttate // \N1
% idānīṃ saptamaṃ  cakraṃ     catuṣaṣṭhidalaṃ tālumadhye   amṛtapūrṇa  varttate // \N2      
% idānīṃ saptamaṃ  cakraṃ     catuḥṣaṣṭhidalaṃ tālumadhye  amṛtapūrṇaṃ varttate // \D
% idānīṃ saptamaṃ  cakraṃ     catuḥṣaṣṭhidalaṃ tālumadhye  amṛtapūrṇaṃ varttate // \U1
% idānīṃ saptamaṃ  tālumadhye catuḥṣaṣṭidalaṃ //           amṛtapūrṇaṃ vartate / \U2      
%-----------------------
% Now the seventh cakra having 64 petals and being full of nectar exists in the middle of the palate.
%-----------------------
%U2: \extra{lalāṭa maṃḍalaṃ// caṃdro devatā// amṛtā śaktiḥ// paramātmā ṛṣiḥ// amṛtavāsinīkalāsaptadaśī amṛtakallolanadī// mahākāśa// aṃbikā// laṃbikā// ghaṃṭikā// tālikā// ajapāgāyatrīdehasvarūpaṃ// kākamukhī// naranetrāgośṛṃgālalāṭabrahmapaṭhāhayagrīvā// mayūramukhā// haṃsavadaṃgāni// ajapāgāyatrīsvarūpaṃ// 
%-----------------------
      \note[type=testium, labelb=86, lem={tālu°}]{Ysg: tālumadhye catuḥṣaṣṭhidalaṃ amṛtapūrṇaṃ}
      \note[type=source, labelb=87, lem={tālu°}]{Ysv (PT): catuḥṣaṣṭidalaṃ tālumadhye cakran tu madhyamam | pīyūṣapūrṇaṃ [pīyūṣapūrṇa° (YK)] koṭīndusannibhaṃ [°sannibha° (YK)] cāmṛtasthalī | tanmadhye ghaṭikāsaṃjñā karṇikā raktasannibhā | saha cendukalā tatrāmṛtadhārāṃ [tābdrā° (YK)] sravaty asau | etad dhyātvāmṛtaiḥ snātvā sadā yogāt pramucyate | unmādajvarapittādidāhaśūlādivedanāḥ [°śūnyā° (YK)] | naśyanti ca śiroduḥkhaṃ jāḍyabhāvo 'pi naśyati | sadyodhyānādbhuktaviśvaṃ jihvājāḍyañ ca naśyati [last sentence \om in YK] |}
idānīṃ saptamaṃ
      \app{\lem[wit={N1,D,U1}]{cakraṃ catuḥṣaṣṭhidalaṃ tālumadhye}
        \rdg[wit={N2}]{cakraṃ catuṣaṣṭhidalaṃ tālumadhye}
        \rdg[wit={E,P,U2}]{tālumadhye catuḥṣaṣṭidalaṃ}
        \rdg[wit={L,B}]{tāludeśe madhye catuḥṣaṣṭhidala}}
      \app{\lem[type=emendation, resp=egoscr]{'mṛtapūrṇaṃ}
        \rdg[wit={ceteri}]{\korr amṛtapūrṇaṃ}
        \rdg[wit={N2}]{amṛtapūrṇa}}
vartate/
  %%%%%%%%%%%%%%
  %%%%%%%%%%%%%%
  %%%%%%%%%%%%%%
  %%%%%%%%%%%%%%
      \extra{\app{\lem[type=emendation, resp=egoscr]{lalāṭaṃ}
          \rdg[wit={U2}]{\korr lalāṭa°}} maṇḍalaṃ\dd{}
        caṃdro devatā\dd{}
        amṛtā śaktiḥ\dd{}
        paramātmā ṛṣiḥ\dd{}
        amṛtavāsinī kalāsaptadaśī\dd{}
        amṛtakallolanadī \app{\lem[type=emendation, resp=egoscr]{mahākāśā}
          \rdg[wit={U2}]{\korr mahākāśa}}\dd{}
        aṃbikā laṃbikā\dd{}
        ghaṃṭikā tālikā\dd{}
        ajapāgāyatrī dehasvarūpaṃ\dd{}
        kākamukhī\dd{}
        naranetrā\dd{}
        gośṛṃgā\dd{}
        lalāṭabrahmapaṭhā\dd{}
        hayagrīvā\dd{}
        mayūramukhā\dd{}
        haṃsavad-aṃgāni\dd{}  
        ajapāgāyatrī svarūpaṃ\dd{}\note[type=philcomm, labelb=88, lem={lalāṭaṃ maṇḍalaṃ}]{This additional passage is found in U2 only. Suprisingly after the additions to this \textit{cakra}, the scribe/author of these additions does'nt add instructions for the duration of practice as before.}}
        %%%%%%%%%%%%%%%
        %%%%%%%%%%%%%%%
        %%%%%%%%%%%%%%%
        %%%%%%%%%%%%%%%
        %%%%%%%%%%%%%%%
%-----------------------
%adhikaśobhāyuktam-----atiśvetaṃ          tanmadhye       raktavarṇaṃ ghāṃṭikāsaṃjñaikā      karṇikā varttate / \E 
%adhikataraśobhayuktaṃ atiśvetaṃ          tanmadhye       raktavarṇaṃ ghaṭikāsaṃjñā ekā      karṇikā varttate / \P
%adhikataraśobhayuktaṃ // atiśvetaṃ //    tanmadhye       raktavarṇaṃ ghaṇikāsaṃjñā ekā ekā  karṇikā varttate / \L
%adhikataraśobhayuktaṃ // atiśvetaṃ //    tanmadhye       raktavarṇaṃ ghaṃṭikāsaṃjñā ekā ekā karṇikā varttate / \B
%adhikataraśobhayuktaṃ atiśvetaṃ          tanmadhye       raktavarṇaṃ ghaṃṭikāsaṃjñā ekā     karṇikā varttate / \N1
%adhikataraśobhāyuktaṃ  atiśvetaṃ         tanmadhye       raktavarṇa--ghaṇṭikāsaṃjñā ekā     karṇikā vartate /  \N2
%adhikataraśobhayuktaṃ atiśvetaṃ          tanmadhye       raktavarṇaṃ ghaṃṭikāsaṃjñā ekā     karṇikā varttate / \D
%adhikataraśobhayuktaṃ atiśvetaṃ          tanmadhye       raktavarṇaṃ ghaṃṭikāsaṃjñā ekā     karṇikā varttate / \U1      
%adhikataraprabhāmuktaṃ // atiśvetaṃ //   tanmadhye       raktavarṇaṃ ghaṃṭikāsaṃjñā// ekā   karṇikā varttate / \U2   
%-----------------------
%[It is] endowed with superabundant beauty. [It is] very bright. In its middle, red in color [is that] known as "uvula" (\textit{ghāṃṭikā}). [It] exists as a single pericarp.  
%-----------------------      
\note[type=testium, labelb=89, lem={adhikatara°}]{Ysg: adhikataraśobhayuktaṃ atiśvetaṃ cakraṃ | tanmadhye raktavarṇaghaṃṭikāsaṃjñā varttate |}
      adhi\app{\lem[wit={ceteri},alt={°kataraśobhayuktaṃ}]{kataraśobhayuktaṃ}
        \rdg[wit={N2}]{°kataraśobhāyuktaṃ}
        \rdg[wit={E}]{°kaśobhāyuktam}
        \rdg[wit={U2}]{°kataraprabhāmuktaṃ}}\dd{}
      \app{\lem[wit={ceteri}]{atiśvetaṃ}
        \rdg[wit={L,B,U2}]{||atiśvetaṃ||}}\dd{}
        tanmadhye
        \app{\lem[wit={ceteri}]{raktavarṇaṃ}
          \rdg[wit={N2}]{raktavarṇa°}}
        \app{\lem[wit={ceteri},alt={ghaṇṭikā°}]{ghaṇṭikā}
          \rdg[wit={E}]{ghāṃṭikā°}
          \rdg[wit={P}]{ghaṭikā°}
          \rdg[wit={L}]{ghaṇikā°}}saṃjñā/
        \app{\lem[wit={ceteri}]{ekā}
          \rdg[wit={L,B}]{ekā ekā}}
          karṇikā vartate/
%-----------------------          
%tanmadhye bhūmiḥ / \E
%tanmadhye bhūmiḥ / \P
%tanmadhye bhūmiḥ / \L
%tanmadhye bhūmiḥ / \B
%tanmadhye bhūmiḥ / \N1
%tanmadhye bhūmiḥ / \N2
%tanmadhye bhūmiḥ / \D
%tanmadhye bhūmis- / \U1
%tanmadhye bhūmi   / \U2         
%-----------------------
%In its middle is a place. 
%-----------------------        
tanmadhye
\app{\lem[wit={ceteri}]{bhūmiḥ}
     \rdg[wit={U1}]{bhūmis°}
     \rdg[wit={U2}]{bhūmi}}/
%-----------------------  
%tanmadhye prakaṭacandrakalā 'mṛtādhārā bhavati         / \E
%tanmadhye prakaṭacandrakalā 'mṛtādhārā sravati         / \P
%tanmadhye prakaṭacandrakalā 'mṛtādhārā sravaṃti        / \L
%tanmadhye prakaṭacandrakalā 'mṛtādhārā sravaṃti        / \B
%tanmadhye prakaṭacandrakalā amṛtādhārāsravaṃtī varttate/ \N1
%tanmadhye prakaṭacaṃdrakalā amṛtādhārāsravaṃtī varttate/ \N2
%tanmadhye prakaṭacandrakalā 'mṛtādhārāsravaṃtī varttate/ \D %sravantī f. Fluss Nom Sg
%tanmadhye pragaṭacaṃdrakalā amṛtadhārāsravaṃtī varttate  \U1
%tanmadhye-ṃdrakaṭaṃ caṃdrakalā amṛtadhārā sravati       /\U2       
%-----------------------
%In its middle exists a hidden digit of the moon, being a stream of nectar like a river (\textit{amṛtādhārāsravantī}. 
%-----------------------
\note[type=testium, labelb=90, lem={prakaṭa°}]{Ysg: tanmadhye prakaṭacandrakalā amṛtādhārāsravaṃtī varttate |}
tanmadhye
       \app{\lem[wit={ceteri},alt={prakaṭa°}]{'prakaṭa}
         \rdg[wit={U1}]{pragaṭa}
         \rdg[wit={U2}]{°ṃdrakaṭaṃ}}candrakalā
       \app{\lem[wit={ceteri}]{amṛtadhārāsravantī}
         \rdg[wit={L,B}]{'mṛtādhārā sravaṃti}
         \rdg[wit={P,U2}]{'mṛtādhārā sravati}
         \rdg[wit={E}]{'mṛtādhārā bhavati}}
       \app{\lem[wit={N1,N2,D,U1}]{vartate}
         \rdg[wit={ceteri}]{\om}}/
\end{prose}
\end{ekdosis}
\ekdpb*{}
%%%%%%%%%%%%%%%%%%
%%%%%%%%%%%%%%%%%%
%%%%PAGEBREAK%%%%
%%%%%%%%%%%%%%%%%%
%%%%%%%%%%%%%%%%%%
\begin{ekdosis}
 \ekddiv{type=ed}
 \begin{prose}
   \noindent
%-----------------------
%tasyāḥ kalāyā     dhyānakāraṇāt tasya samīpe maraṇaṃ nāyāti/     \E -> does not come near to death -> na-ā-yāti
%tasyāḥ kalāyā     dhyānakaraṇāt tasya samīpe maraṇaṃ nāyāti/     \P
%tasyāḥ karṇikāyā  dhyānakaraṇāt tasya samīpe maraṇaṃ na yāti     \L
%tasyāḥ karṇikāyā  dhyānakaraṇāt tasya samīpe maraṇaṃ na yāti     \B
%tasyāḥ kalāyāḥ    dhyānakaraṇāt tasya samīpe maraṇaṃ nāyāti      \N1
%tasyāḥ kalāyāḥ    dhyānakaraṇāt tasya samīpe maraṇaṃ nāyāti/     \N2       
%tasyāḥ kalāyāḥ    dhyānakaraṇāt tasya samīpe maraṇaṃ nāyāti      \D
%tasyāḥ kalāyā     dhyānakaraṇāt tasya samīpe maraṇaṃ nāyāti/     \U1
%tasyāḥ kalāyā     dhyānakāraṇāt// tasya samīpe maraṇaṃ na yāti/  \U2
%-----------------------
%Because of the exercise of meditation on this digit death does not come near him. 
%-----------------------
\note[type=testium, labelb=91, lem={maraṇaṃ}]{Ysg: tasyāḥ kalāyā nirantaraṃ dhyānakartum maraṇaṃ}
       tasyāḥ
       \app{\lem[wit={ceteri}]{kalāyā}
         \rdg[wit={N1,N2,U1}]{kalāyāḥ} %Sandhi-mistake in apparatus in this case?
         \rdg[wit={L,B}]{karṇikāyā}}
dhyānakaraṇāt tasya samīpe maraṇaṃ
       \app{\lem[wit={ceteri}]{nāyāti}
         \rdg[wit={L,B,U2}]{na yāti}}/
%-----------------------
%nirantaradhyānād        -amṛtadhārāyāḥ sajīvo bhavati /    \E
%niraṃtaradhyānāt---------amṛtadhārā plāvanaṃ   bhavati /   \P
%niraṃtaradhyānakaraṇād   amṛtadhārā           sravati /    \L
%niraṃtaradhyānakaraṇād   amṛtadhārā           sravati /    \B
%niraṃtaradhyānakaraṇāt / amṛtadhārā           sravaṃti /   \N1
%niraṃtaradhyānakaraṇāt   amṛtadhārā            sravaṃti    \N2
%niraṃtaradhyānakaraṇāt / amṛtadhārā           sravaṃti /   \D   
%niraṃtaradhyānakaraṇāt   amṛtadhārā             sravati /  \U1
%niraṃtaradhyānakaraṇāt / amṛtadhārā plavanaṃ  bhavati /    \U2
%-----------------------
%Due to uninterrupted meditation the stream (\textit{dhārā}) of nectar flows. 
%-----------------------
nirantara\app{\lem[wit={ceteri},alt={°dhyānakaraṇād}]{dhyānakaraṇā\skm{d-a}}
        \rdg[wit={E,P}]{°dhyānād}}
      \app{\lem[wit={ceteri}, alt={amṛtadhārā}]{\skp{d-a}mṛtadhārā}
         \rdg[wit={E}]{amṛtadhārāyāḥ sajīvo}
         \rdg[wit={P}]{amṛtadhārā plāvanaṃ}
         \rdg[wit={U2}]{amṛtadhārā plavanaṃ}}
       \app{\lem[wit={L,B,U1}]{sravati}
         \rdg[wit={N1,N2,D}]{sravaṃti}
         \rdg[wit={E,P,U2}]{bhavati}}/       
%-----------------------
%tadā  yakṣam-aroga----pittajvarahṛdayadāha-śiroroga-jihvā--jaḍa-bhāvā           naśyanti / \E
%tadā     kṣayaroga----pittajvarahṛdayadāha-śiroroga-jihvā--jaḍa-bhāvān          naśyanti / \P
%tadā     kṣayaroga----pittajvarahṛdayadāha-----roga-jihvāyājaḍa-bhāvān          naśyanti / \L
%tadā     kṣayaroga----pittajvarahṛdayadāha-----roga-jihvāyājaḍa-vān             naśyanti / \B
%         kṣayarogaṃ   pittajvarahṛdayadāha-śiroroga-jihvāyājaḍa-bhāvā           naśyanti / \N1 %besser kṣayarogaṃ emendieren zu vollem Kompositum?
%         kṣayarogaṃ   pittajvarahṛdayadāha-śiroroga-jihvāyājaḍa-bhāvātā         naśyanti / \N2
%         kṣayaṃ rogaṃ pittajvarahṛdayadāha-śiroroga-jihvāyājaḍa-bhāvā           naśyanti / \D
%         kṣayaroga----pittajvarahṛdayadāha-śiroroga-jihvāyājaḍa-bhāvā           naśyanti / \U1  
%tadā     kṣayarogo----ptatti// jvara hṛdayadāha// śiroroga// jihvājaḍatā// dayo naśyanti /cha/ \U2       
%-----------------------
%Then the appearances of emaciation (\textit{kṣayaroga}), fever due to disordered bile (\textit{pittajvara), heartburn (\textit{hṛdayadāha}), head-disease (\textit{śiroroga}) and tongue insensibility (\textit{jihvājaḍa}) vanish. %!!!Krankheiten in Ayurvedabuch checken! medizinische Identifikationen!
%-----------------------
\note[type=testium, labelb=92, lem={kṣaya°}]{Ysg: kṣayarogaḥ pettajvarahṛdayadāhaśiro..jihvāyājaḍyaṃ ca naśyati |}
       \app{\lem[wit={E,P,L,B,U2}]{tadā}
         \rdg[wit={ceteri}]{\om}}
       \app{\lem[type=emendation, resp=egoscr]{kṣayarogapittajvarahṛdayadāhaśirorogajihvājaḍabhāvā}
         \rdg[wit={E}]{\korr yakṣamarogapittajvarahṛdayadāhaśirorogajihvājaḍabhāvā}
         \rdg[wit={P}]{kṣayarogapittajvarahṛdayadāhaśirorogajihvājaḍabhāvān}
         \rdg[wit={L}]{kṣayarogapittajvarahṛdayadāharogajihvāyājaḍabhāvān}
         \rdg[wit={B}]{kṣayarogapittajvarahṛdayadāharogajihvāyājaḍavān}
         \rdg[wit={N1}]{kṣayarogaṃ pittajvarahṛdayadāhaśirorogajihvāyājaḍabhāvā}
         \rdg[wit={N2}]{kṣayarogaṃ pittajvarahṛdayadāhaśirorogajihvāyājaḍabhāvātā}
         \rdg[wit={D}]{kṣayaṃ rogaṃ pittajvarahṛdayadāhaśirorogajihvāyājaḍabhāvā}
         \rdg[wit={U1}]{kṣayarogapittajvarahṛdayadāhaśirorogajihvāyājaḍabhāvā}
         \rdg[wit={U2}]{kṣayarogoptatti || jvara hṛdayadāha || śiroroga || jihvājaḍatā || dayo}}
naśyanti/
%-----------------------       
%bhakṣitam--api   viṣan    na bādhate  / \E
%bhakṣitam--api   viṃṣa    na bādhate  / \P
%bhākṣitam--api   viṣaṃ    na bādhyate / \L
%bhākṣitamār pi   viṣaṃ    na bādhyate / \B
%bhakṣitam        viṣamapi na bādhyate / \N1
%bhakṣitaṃ        viṣamapi na bādhate  / \N2
%bhakṣitāṃ        viṣamapi na bādhyate / \D
%bhakṣitaṃ        viṣamapi na bādhyate   \U1
%bhakṣitam--api   viṣaṃ    na bādhyate / \U2       
%-----------------------       
%Also eaten venom doesn't trouble him. 
%-----------------------
         \app{\lem[wit={N2,U1}]{bhakṣitaṃ}
           \rdg[wit={N1}]{bhakṣitam}
           \rdg[wit={D}]{bhakṣitāṃ}
           \rdg[wit={E,P,L,U2}]{bhakṣitam api}
           \rdg[wit={B}]{bhākṣitamār pi}}
         \app{\lem[wit={N1,N2,D,U1}, alt={viṣam api}]{viṣam-api}
           \rdg[wit={L,B,U2}]{viṣaṃ}
           \rdg[wit={E}]{viṣan}
           \rdg[wit={P}]{viṃṣa}}
na \app{\lem[wit={E,P,N2}]{bādhate}
           \rdg[wit={ceteri}]{bādhyate}}/     
%-----------------------
%yady-atra manaḥ sthiraṃ   bhavati /  \E
%yady-atra manaḥ sthiraṃ   bhavati /  \P
%yady-atramapi manasthiraṃ bhavati /  \L              %VARIANTE UNSICHER!!!WAS MEINT JÜRGEn??
%yady-atramapi manasthiraṃ bhavati /  \B
%yady-atra     manasthiraṃ bhavati /  \N1
%yadyanna      manasthiraṃ bhavati // \N2
%yadyanna      manasthiraṃ bhavati /  \D
%yadyatra      manasthiraṃ bhavati    \U1
%yadyatra      manasthiraṃ bhavati//  \U2       
%-----------------------
%If here the mind becomes stable.       
%-----------------------
         \app{\lem[wit={ceteri}]{yadyatra}
           \rdg[wit={L,B}]{yadyatram api}
           \rdg[wit={N1,D}]{yadyanna}}
         \app{\lem[wit={E,P}]{manaḥ sthiraṃ}
           \rdg[wit={ceteri}]{manasthiraṃ}}
         \app{\lem[wit={ceteri}]{bhavati}}\dd{}         
    \end{prose}
  \end{ekdosis}
%%%%%%%%%%%%%%% 
%%%%%%%%%%%%%%%
%%%%%%%%%%%%%%%
%%%%%%%%%%%%%%%
%%%%%%%%%%%%%%%
\begin{ekdosis}
  \ekddiv{type=ed}
   \bigskip
    \centerline{\textrm{\small{[Eighth Cakra]}}}
    \bigskip
    \begin{prose}
%-----------------------
%idānīṃ brahmarandhrasthāne 'ṣṭamaṃ śatadalaṃ cakraṃ varttate / \E
%idānīṃ brahmaraṃdhrasthāne 'ṣṭamaṃ śatadalaṃ cakraṃ vartate / \P
%idānīṃ brahmaraṃdhrasthāne aṣṭamaṃ śatadalaṃ cakraṃ vartate / \L
%idānīṃ brahmaraṃdhrasthāne aṣṭamaṃ śatadalaṃ cakraṃ vartate / \B
%idānīṃ aṣṭamacakraṃ brahmaraṃdhrasthāne śatadalaṃ   vartate / \N1
%idānīṃ aṣṭamacakraṃ brahmaraṃdhrasthāne śatadalaṃ   vartate  \N2
%idānīṃ aṣṭamacakraṃ brahmaraṃdhrasthāne śatadalaṃ   vartate / \D
%idānīṃ aṣṭamaṃ cakraṃ brahmaraṃdhrasthāne śatadalaṃ   vartate . \U1
%idānīṃ brahmaraṃdhrasthāne 'ṣṭamaṃ śatadalaṃ cakraṃ varttate // \U2
%-----------------------
%guru devatā// caitanya śaktiḥ// virāṭ ṛṣiḥ// sarvotkṛṣṭasākṣiḥ// bhūtaturyātītacaitanyātmakaṃ// sarvavarṇāḥ// sarvamātrāḥ// sarvadalāni virāṭdeha sthitāvasthā prajñāvācā sohaṃ veda anupamasthānaṃ// ajapājapasahasra/ 1000 gha 02 pa 046 akṣara 40// sarvajapasaṃkhyā// 21600// ekaviṃśatisahasrāṇiṣaṭśatāni// tathaivaca niśāhevahate// prāṇaḥ yojānātisapaṃḍitaḥ// sakāreṇa bahiryātihakāreṇaviśotpunaḥ// haṃsaḥ sohaṃ// tato maṃtraṃ jīvojapati sarvadā//    
%-----------------------
%Now exists the eigth \textit{cakra} having one hundred petals located at the aperture of Brahman.
%-----------------------
      \note[type=testium, labelb=93, lem={śatadala}]{Ysg: brahmaraṃdhre śatadalaṃ}
      \note[type=source, labelb=94, lem={śatadala}]{Ysv (PT): brahmarandhre 'ṣṭamaṃ cakraṃ śatapatraṃ mahāprabham | jālandharaṃ nāma pīṭhaṃ etat tu parikīrttitam | siddhapuṃsaḥ [°puṃsa° (YK)] sthalaṃ jñātvā agnidhūmanibhā śikhā | ādimadhyāntahīnā strīpuṃmūrtti [°mūrtir (YK)] varttate parā | antajñānī [antaryāmī (YK)] bhaved dhyānād ākāśe 'pi samāgamaḥ | nirantaraṃ sarvavettā ity ūccāno mahān bhavet | jaganmadhye sthito jantur jagadbādhāvivarjitaḥ |}    
idānīṃ
\app{\lem[wit={N1,N2,D}]{aṣṭamacakraṃ brahmaraṃdhrasthāne śatadalaṃ}
    \rdg[wit={E,P,U2}]{brahmarandhrasthāne 'ṣṭamaṃ śatadalaṃ cakraṃ}
    \rdg[wit={L,B}]{brahmaraṃdhrasthāne aṣṭamaṃ śatadalaṃ cakraṃ}
    \rdg[wit={U1}]{cakraṃ brahmaraṃdhrasthāne śatadalaṃ}}
  vartate/
  %%%%%%%%%%
  %%%%%%%%%%%
  %%%%%%%%%%
\extra{\app{\lem[type=emendation, resp=egoscr, alt={gurur}]{guru\skp{r-de}}
          \rdg[wit={U2}]{\korr guru°}}\skm{r-de}vatā\dd{}
        \app{\lem[type=emendation, resp=egoscr]{caitanyaḥ}
          \rdg[wit={U2}]{\korr caitanya°}} śaktiḥ\dd{}
        virāṭ ṛṣiḥ sarvotkṛṣṭasākṣiḥ\dd{}
        \app{\lem[type=emendation, resp=egoscr]{bhūtaturyātītaṃ}
          \rdg[wit={U2}]{\korr bhūtaturyātīta°}} caitanyātmakaṃ\dd{}
        sarvavarṇāḥ\dd{}
        sarvamātrāḥ\dd{}
        sarvadalāni\dd{}
        virāṭ \app{\lem[type=emendation, resp=egoscr]{dehaḥ}
          \rdg[wit={U2}]{\korr deha°}}
        sthitāvasthā\dd{} 
        prajñā vācā\dd{}
        sohaṃ \app{\lem[type=emendation, resp=egoscr]{vedaḥ}
          \rdg[wit={U2}]{\korr veda}}\dd{}
        \app{\lem[type=emendation, resp=egoscr]{anupamaṃ}
          \rdg[wit={U2}]{\korr anupama°}} sthānaṃ\dd{}
         \app{\lem[type=emendation, resp=egoscr]{ajapājapaḥ sahasraḥ}
          \rdg[wit={U2}]{\korr ajapājapasahasra}}\dd{} 1000 ghaṭi 2 palā 46 akṣara 40\dd{}
        \app{\lem[type=emendation, resp=egoscr]{sarvajapaḥ}
          \rdg[wit={U2}]{\korr sarvajapa°}} saṃkhyā\dd{}
        21600\dd{}
        ekaviṃśatisahasrāṇiṣaṭśatāni\dd{}
        tathaiva ca niśāhe vahate\dd{}
        prāṇaḥ yo jānāti sa paṃḍitaḥ\dd{} %%prāṇaḥ = m nom pl
        sakāreṇa bahir-yāti hakāreṇa viśet punaḥ\dd{}  
        haṃsaḥ sohaṃ\dd{}
        tato mantraṃ jīvo japati sarvadā\dd{}}
%The teacher is the deity. Consciousness is the power. Virāṭ is the Ṛṣi, the witness above everything. Made of consciousness is that which is associated with (\textit{bhūta°) the state beyond the fourth state. It has all colours. It has all matrices. It has all petals. The body is Virāṭ. The state is the standing still. The speech is wisdom.  The "I am that"-[expression] (\textit{sohaṃ}) is the Veda. The place is unsurpassed. A thousandfold recitation of the non-recited; 1000 [repetitions for]; 2 \textit{ghaṭi}s, 46 \textit{palā}s. and 40 \textit{akṣara}s.\footnote{It's not entirely clear what kind of measure is an \textit{akṣara}.} The count is all silent mutterings, [being] 21600. Day and night in this way it carries on. He who knows the breath is a learned person. With the sound of "sa" he exhales, with the sound of "ha" he inhales again: "I'm he, he's I". Because of that the embodied soul constantly utters the Mantra.\footnote{Add intertextual evidence.}
  %%%%%%%%%%%%%%%
  %%%%%%%%%%%%%%%
  %%%%%%%%%%%%%%%
  %%%%%%%%%%%%%%
  %%%%%%%%%%%%%%%
%----------------------
%tasya kamala----jātyadharaṇīpīṭha iti saṃjñā / \E
%tasya kamalasya jālaṃdharapīṭha iti saṃjñā / \P
%tasya kamalasya jālaṃdharapīṭha iti saṃjñā ...  \L
%tasya kamalasya jālaṃdharapīṭhasaṃjñā ...  \B
%tasya kamalasya jālaṃdharapīṭha iti saṃjñā ...  \N1
%tasya kamalasya jālaṃdharapīṭha iti saṃjñā ...  \N2
%tasya kamalasya jālaṃdharapīṭha iti saṃjñā ...  \D
%tasya kamalasya jālaṃdharapīṭha iti saṃjñā ...  \U1      
%tasya kamalasya jālaṃdharapīṭha iti saṃjñā //   \U2
%----------------------
%``The (divine) seat of  Jālaṃdhara'' is the designation of the lotus of it. 
%----------------------      
\note[type=testium, labelb=95, lem={jālaṃdhara°}]{Ysg: jālaṃdharapīṭhasaṃjñakaṃ}
tasya
\app{\lem[wit={ceteri}]{kamalasya}
  \rdg[wit={E}]{kamala°}}
      \app{\lem[wit={ceteri}]{jālandharapīṭha}
        \rdg[wit={B}]{jālandharapīṭha°}
        \rdg[wit={E}]{jātyadharaṇīpīṭha}}
      \app{\lem[wit={ceteri}]{iti}
        \rdg[wit={B}]{\om}}
      \app{\lem[wit={ceteri}]{saṃjñā}
        \rdg[wit={B}]{°saṃjñā}}/
%---------------------- 
%siddhapuruṣasya sthānam / \E
%siddhapuruṣasya sthānam / \P
%siddhapuruṣasya sthānam mūrti vartate // \L                         %%% schwerer Satz -> wie soll ich hier entscheiden?! 
%siddhapuruṣasya sthānam mūrti vartate // \B %Zeilensprung
%siddhapuruṣasya sthānam // \N1
%siddhapuruṣasya sthānam // \N2
%siddhapuruṣasya sthānam // \D  
%siddhapuruṣasya sthānam    \U1
%siddhapuruṣasya sthānaṃ    \U2
%----------------------      
%[It is] the place of the accomplished person.
%----------------------
\note[type=testium, labelb=96, lem={siddha°}]{Ysg: siddhapuruṣasyānacakraṃ}
      siddha\app{\lem[wit={ceteri},alt={°puruṣasya sthānam}]{puruṣasya\skp{-}sthānaṃ}
        \rdg[wit={L,B}]{sthānam mūrti vartate}}/
     \end{prose}
  \end{ekdosis}
\ekdpb*{}
%%%%%%%%%%%%%%%%%%%%%%%%%%%%%%%%%%%%%%%%%%
%%%%%%%%%%%%%%%%%%%%%%%%%%%%%%%%%%%%%%%%%%
%%%%%%%%PAGEBREAK%%%%%%%PAGEBREAK%%%%%%%%%
%%%%%%%%%%%%%%%%%%%%%%%%%%%%%%%%%%%%%%%%%%
%%%%%%%%%%%%%%%%PAGEBREAK%%%%%%%%%%%%%%%%%
%%%%%%%%%%%%%%%%%%%%%%%%%%%%%%%%%%%%%%%%%%
%%%%%%%%PAGEBREAK%%%%%%%PAGEBREAK%%%%%%%%%
%%%%%%%%%%%%%%%%%%%%%%%%%%%%%%%%%%%%%%%%%%
%%%%%%%%%%%%%%%%%%%%%%%%%%%%%%%%%%%%%%%%%%
%%%%%%%%%%%%%%%%%%%%%%%%%%%%%%%%%%%%%%%%%%
%%%%%%%%%%%%%%%%%%%%%%%%%%%%%%%%%%%%%%%%%%
%%%%%%%%PAGEBREAK%%%%%%%PAGEBREAK%%%%%%%%%
%%%%%%%%%%%%%%%%%%%%%%%%%%%%%%%%%%%%%%%%%%
%%%%%%%%%%%%%%%%PAGEBREAK%%%%%%%%%%%%%%%%%
%%%%%%%%%%%%%%%%%%%%%%%%%%%%%%%%%%%%%%%%%%
%%%%%%%%PAGEBREAK%%%%%%%PAGEBREAK%%%%%%%%%
%%%%%%%%%%%%%%%%%%%%%%%%%%%%%%%%%%%%%%%%%%
%%%%%%%%%%%%%%%%%%%%%%%%%%%%%%%%%%%%%%%%%%
%%%%%%%%%%%%%%%%%%%%%%%%%%%%%%%%%%%%%%%%%%
%%%%%%%%%%%%%%%%%%%%%%%%%%%%%%%%%%%%%%%%%%
%%%%%%%%PAGEBREAK%%%%%%%PAGEBREAK%%%%%%%%%
%%%%%%%%%%%%%%%%%%%%%%%%%%%%%%%%%%%%%%%%%%
%%%%%%%%%%%%%%%%PAGEBREAK%%%%%%%%%%%%%%%%%
%%%%%%%%%%%%%%%%%%%%%%%%%%%%%%%%%%%%%%%%%%
%%%%%%%%PAGEBREAK%%%%%%%PAGEBREAK%%%%%%%%%
%%%%%%%%%%%%%%%%%%%%%%%%%%%%%%%%%%%%%%%%%%
%%%%%%%%%%%%%%%%%%%%%%%%%%%%%%%%%%%%%%%%%%
\begin{ekdosis}
  \begin{prose}
%%%%%%%%%%%%%%% 
%%%%%%%%%%%%%%%
%%%%%%%%%%%%%%%
%%%%%%%%%%%%%%
%%%%%%%%%%%%%%%
%----------------------
%tanmadhye    'gnidhūmākārarekhā     yādṛśy    ādṛśy ekā  puruṣasya mūrttir varttate /  \E
%tanmadhye    'gnidhūmākārarekhā     yādṛśī   tādṛśy ekā  puruṣasya mūrttir varttate /  \P
%tanmadhye    'gnidhūmākārārekhā     yādṛśī   tādṛśy ekā  puruṣasya mūrttir varttate /  \L               
%tanmadhye    'gnidhūmākārārekhā     yādṛśī   tādṛśy ekā  puruṣasya mūrttir varttate /  \B     
%tanmadhye    'gnidhūmākārāreṣā      yādṛśī   tādṛśī ekā  puruṣasya mūrttir varttate /  \N1
%tanmadhye    agnidhūmrākārarekhā    yādṛśī / tādṛśī ekā  puruṣasya mūrttir varttate /  \N2
%tanmadhye    agnidhūmākārāreṣā      yādṛśī   tādṛśī ekā  puruṣasya mūrttir varttate /  \D
%tanmadhye    agnidhūmrākārārekhā    yādṛśī   tādṛśī ekā  puruṣasya mūrtir  vartate     \U1
%tanmadhye    'gnidhūmrākārārekhāyāḥ  etādṛśī         ekā  puruṣasya mūrtir  vartate // \U2
%----------------------      
%In its middle [is] something like a streak having the form of smoke and fire. Such a single [divine] form of the person (\textit{puruṣa}) exists [there].        
%---------------------      
\noindent
    \note[type=testium, labelb=97, lem={Ysg: 'gnidhūmrā°}]{tanmadhye gnidhūmrāreṣākārā ādimadhyaṃtarahitā puruṣasya mūrttir asti |}
tanmadhye \app{\lem[wit={E,P,L,B}]{'gnidhūmākārarekhā}
        \rdg[wit={N1,D}]{'gnidhūmākārāreṣā}
        \rdg[wit={N2,U1}]{agnidhūmrākārarekhā}
        \rdg[wit={U2}]{'gnidhūmrākārārekhāyāḥ}}
      \app{\lem[wit={ceteri}]{yādṛśī}
        \rdg[wit={E}]{yādṛśy°}
        \rdg[wit={U2}]{etādṛśī}}/
      \app{\lem[wit={P,L,B}]{yādṛśy}
        \rdg[wit={E}]{ādṛsy}
        \rdg[wit={N1,N2,D,U1}]{yādṛśī}
        \rdg[wit={U2}]{\om}}
      ekā puruṣasya mūrtir-vartate/
%---------------------
%tasyā  nādir nāṃto 'sti / \E
%tasyā  nādināṃ 'to sti / \P
%tasyā  nādir nāṃto sti / \L -> vor dem bei allen anderen vorigen Satz!?!?!?! 
%tasyā  nādir nāṃto sti / \B -> vor dem bei allen anderen vorigen Satz!?!?!?! 
%tasyāḥ nāsty aṃtaḥ ādir-api nāsti / \N1????
%tasyāḥ nāsty aṃtaḥ ādir-api nāsti / \N2
%tasyāḥ nāsty aṃtaḥ ādir api nāsti / \D
%tasyāḥ nāsty aṃtaḥ ādir-api nāsti    \U1
%tasyā  nādir naṃto sti              \U2
%---------------------
% Of her exists no end, nor a beginning.
%---------------------      
\app{\lem[wit={E,P,L,B,U2}]{tasyā}
  \rdg[wit={D,N1,N2,U1}]{tasyāḥ}}
\app{\lem[alt={nādir nānto 'sti}, wit={ceteri}]{nādir-nānto 'sti}
        \rdg[wit={N1,N2,D,U1}]{nāsty aṃtaḥ ādir api nāsti}
        \rdg[wit={P}]{nādināṃ 'to sti}}/
%---------------------    
%tasyā  mūrtter dhyānakāraṇāt pratyakṣaṃ niraṃtaraṃ  puruṣasyākāśe   gamāgamau   bhavataḥ / \E
%tasyā  mūrtter dhyānakaraṇāt pratyakṣaniraṃtaraṃ    puruṣasyākāśe   gamāgamau   bhavataḥ / \P
%tasyā  mūrtir  dhyānakaraṇāt pratyakṣaniraṃtaraṃ    puruṣasyākāśe   gamāgamau   bhavataḥ / \L         
%tasyā  mūrtir  dhyānakaraṇāt pratyakṣaṃ niraṃtaraṃ  puruṣasyākāśe   gamāgamau   bhavataḥ / \B
%tasyāḥ mūrttir dhyānakaraṇāt pratyakṣaniraṃtaraṃ    puruṣasya ākāśe gamāgamau   bhavataḥ / \N1
%tasyāḥ mūrttir dhyānakaraṇāt pratyakṣaniraṃtaraṃ    puruṣa ākāśe    gamāgame    bhavataḥ / \N2
%tasyāḥ mūrtir  dhyānakaraṇāt pratyakṣaniraṃtaraṃ    puruṣasya ākāśe gamāgamau   bhavataḥ / \D
%tasyāḥ mūrter  dhyānakaraṇāt/ pratyakṣaniraṃtaraṃ   puruṣasya ākāśi gamāmamo   bhavataḥ   \U1
%tasyāḥ mūrter  dhyānakaraṇāt pratyakṣaniraṃtaraṃ    puruṣasyākāśa---gamāgamau bhavata //      \U2
%---------------------    
%BEDEUTUNG DES SATZES BIS JETZT UNKLAR! Idee: Zeilensprung aus übernächstem Satz! Streiche pratyakṣaṃ niraṃtaraṃ und der Satz ergibt Sinn!  
%gamāgamau nom.  dual = coming and going ; bhavataḥ = 3p du ind pres von bhū
%Due to the exercise of meditation on this (divine) form both coming and going of the person in space occurs. 
%Kolloquium: Meinung zu Kompositum pratyakṣaniraṃtaraṃ = macht wenig Sinn oder?
%{\englishnote{\small Even though every single witness at hand transmits the latter reading right after \textit{°karaṇāt}, several considerations make it reasonable to conject that the original sentence is corrupted and was written without it. The main consideration to assume the corruption is that \textit{pratyakṣaṃ nirantaraṃ} is ungrammatical. The second is that the sentence is way more meaningful without it. The third that two sentences later we get the phrase in a meaningful context. Due to the last consideration my best guess is an interlace at an early stage of transmission.}}
%---------------------
\note[type=testium, labelb=98, lem={dhyānakaraṇāt}]{Ysg: tasyāḥ dhyānakartuḥ}
      \app{\lem[wit={B,E,L,P}]{tasyā}
  \rdg[wit={ceteri}]{tasyāḥ}}
\app{\lem[alt={mūrter},wit={E,P,U1,U2}]{mūrte\skp{r-dhyā}}
  \rdg[wit={B,D,L,N1,N2}]{mūrtir}}
\app{\lem[alt={dhyānakaraṇāt},type=conjecture, resp=egoscr]{\skm{r-dhyā}nakaraṇāt}
        \rdg[wit={E,B}]{\conj dhyānakāraṇāt pratyakṣaṃ niraṃtaraṃ}
        \rdg[wit={ceteri}]{dhyānakaraṇāt pratyakṣaniraṃtaraṃ}}
         \note[type=philcomm, labelb=99, lem={°kāraṇāt pratyakṣaṃ niraṃtaraṃ}]{Even though every single witness at hand transmits the latter reading right after °\textit{karaṇāt}, several considerations make it reasonable to conject that the original sentence is corrupted and was written without it. The main consideration to assume the corruption is that the syntactical unit \textit{pratyakṣaṃ nirantaraṃ} is ungrammatical in this construction. The second is that the sentence is way more meaningful without it. The third that two sentences later we get the phrase in a meaningful context. Due to the last consideration my best guess is an interlace at an early stage of transmission.}
      \app{\lem[wit={ceteri}]{puruṣasyākāśe}
        \rdg[wit={N2}]{puruṣa ākāśe}
        \rdg[wit={U2}]{puruṣasyākāśa°}
        \rdg[wit={U1}]{puruṣasya ākāśi}}
      gamā\app{\lem[wit={ceteri},alt={°gamau}]{gamau}
        \rdg[wit={U1}]{°gamo}
        \rdg[wit={N2}]{°game}}
        \app{\lem[wit={ceteri}]{bhavataḥ}
          \rdg[wit={U2}]{bhavata}}/
%---------------------     
%pṛthvīmadhye  sthitasyāpi    pṛthvī-------bādho   na bhavati / \E
%pṛthvīmadhye  sthitasyāpi    pṛthaka                 bhavati   \P %Zeilenspringer führt zu Verlust von Zeile in Pune
%pṛthvīmadhye  sthitasyāpi    pṛthvī-------bādho   na bhavati / \L
%pṛthivīmadhye sthitasyāpi // pṛtvī--------bādho   na bhavati // \B
%pṛthvīmadhye  sthitāv-api    pṛthvī kṣato bādho   na bhavati // \N1
%pṛthvīmadhye  sthitāv-api    pṛthvī kṣato bādho   na bhavati // \N2      
%pṛthvīmadhye  sthitāv-api    pṛthvī kṣato bādho   na bhavati // \D
%pṛthvīmadhye  sthitāv-api    pṛthvī kṣato bādho   na bhavati     \U1
%pṛthīvīmadhye sthitasyāpi    pṛthvī       bādhoko na bhati     \U2
%---------------------
%Affliction from the earth-element does not arise [anymore] even if one is situated in the middle of the earth.        
%---------------------
\note[type=testium, labelb=100 lem={pṛthvīmadhye}]{Ysg: pṛthivyāṃ sthitāv api pṛthvī kṛtabādho na bhavati ||}
        \app{\lem[wit={ceteri}]{pṛthvīmadhye}
          \rdg[wit={B,U2}]{pṛtivīmadhye}}
        \app{\lem[wit={ceteri}]{sthitasyāpi}     
          \rdg[wit={D,N1,N2,U1},alt={sthitāv api}]{sthitāv\skp{-}api}}
        \app{\lem[wit={E,L}]{pṛthvībādho}
          \rdg[wit={B}]{pṛtvībādho}
          \rdg[wit={N1,N2,D,U1}]{pṛthvī kṣato bādho}
          \rdg[wit={P}]{pṛthaka}
          \rdg[wit={U2}]{pṛthvī bādhoko}}
        \app{\lem[wit={ceteri}]{na bhavati}
          \rdg[wit={P}]{bhavati}}/
%---------------------
%sakalān pratyakṣaṃ niraṃtaraṃ paśyati ca pṛthagbhavati / \E
% \om                                                       \P      
%sakalāḥ pratyakṣaṃ niraṃtara paśyatī  ca pṛthak bhavati // \B
%sakalāḥ pratyakṣaṃ niraṃtara paśyatī  ca pṛthak bhavati / \L
%sakalāpratyakṣaniraṃtaraṃ    paśyati  ca pṛthak ca bhavati // \N1
%sakalapratyakṣaniraṃtaraṃ    paśyati  ca pṛthak ca bhavati    \N2      
%sakalāpratyakṣaniraṃtaraṃ    paśyati  ca pṛthak pṛthak bhavati \D      
%sakalāpratyakṣaniraṃtaraṃ    paśyati  ca/ pṛthak ca bhavati // \U1
%\om                                                     \U2
%---------------------
%He constantly sees everything in front of his eyes and he becomes separated (from the material world).
%---------------------
        \app{\lem[type=emendation, resp=egoscr]{sakalaṃ pratyakṣaṃ nirantaraṃ}
          \rdg[wit={N1,N2,D,U1}]{\korr sakalāpratyakṣaṃ nirantaraṃ}
          \rdg[wit={B,L}]{sakalāḥ pratyakṣaṃ niraṃtara}
          \rdg[wit={E}]{sakalān pratyakṣaṃ niraṃtaraṃ}
          \rdg[wit={P,U2}]{\om}}
        \app{\lem[wit={ceteri}]{paśyati}
          \rdg[wit={L,B}]{paśyatī}
          \rdg[wit={P,U2}]{\om}}
        \app{\lem[wit={E}]{pṛthagbhavati}
          \rdg[wit={B,L}]{ca pṛthak bhavati}
          \rdg[wit={N1,N2,U1}]{ca pṛthak ca bhavati}
          \rdg[wit={P,U2}]{\om}}/  
%---------------------
%atiśayenāyur vardhate /   \E
%atiśayenāyur vardhate     \P      
%atīśayanāyur vardhayate / \B
%atīśayanāyur vardhayate // \L
%atiśayena āyur varddhate // \N1
%atiśayena āyur varddhate // \N2     
%atiśayena āyur varddhate // \D
%atiśayena āyur varddhate // \U1
%\om                         \U2
%---------------------
% The force of life increases eminently. 
%---------------------
        \app{\lem[alt={atiśayenāyur},wit={E,P}]{atiśayenāyu\skp{r-va}}
          \rdg[wit={B,L}]{atīśayanāyur}
          \rdg[wit={N1,N2,D,U1}]{atiśayena āyur}
          \rdg[wit={U2}]{\om}}\app{\lem[alt={vardhate},wit={ceteri}]{\skm{r-va}rdhate}
          \rdg[wit={B,L}]{vardhayate}}\dd{}        
    \end{prose}
  \end{ekdosis}
%%%%%%%%%%%%%%%
%%%%%%%%%%%%%%
%%%%%%%%%%%%%
%%%%%%%%%%%%%
%%%%%%%%%%%%%
\begin{ekdosis}
 \ekddiv{type=ed}
   \bigskip
    \centerline{\textrm{\small{[Ninth Cakra]}}}
    \bigskip
    \begin{prose}
%---------------------
%idānīṃ navamacakrasya   bhedāḥ kathyante /  \E
%idānīṃ navamacakrasya   bhedāḥ kathyante /  \P
%idānīṃ navamacakrasya   bhedāḥ kathyate     \L
%idānīṃ navamaṃ cakrasya bhedāḥ kathyate //  \B
%idānīṃ navamacakrasya   bhedāḥ kathyaṃte // \N1
%idānīṃ navamacakrasya   bheda  kathyate  // \N2
%idānīṃ navamacakrasya   bhedāḥ kathyaṃte // \D
%idānīṃ navamaś cakrasya bhedāḥ kathyaṃte    \U1   
%idānīṃ navamacakrasya   bhedaḥ kathyate /   \U2
%---------------------
%Now the divisions/differentiations of the ninth cakra are explained.
%---------------------
\note[type=testium, labelb=101, lem={mahāśūnyacakram}]{Ysg: brahmaraṃdhre eva śatadalacakropari mahāśūnyacakraṃ mahāsiddhacakraṃ pūrṇagiricakraṃ iti saṃjñakaṃ sahasradalaṃ cakraṃ asti | tad upari kiṃcin nāsti | tac cakraṃ atiraktaṃ ūrdhvamukhaṃ sakalaśobhāspadaṃ anekakalyāṇapūrṇaṃ mano vācā ma gocara parimalo petaṃ | tat kamalamadhye trikoṇākarṇikā | tasyāṃ karṇikāyāṃ saptadaśī niraṃjanarūpā koṭisūryaprabhā satī uṣṇabhava hīnā koṭicandrasama sītalaikākal nāsti | tasyāṃ anaṃta paramānaṃta paramānaṃdānāṃ sthānaṃ tasyāḥ kalāyā dhyānakaraṇāt sādako yadyādi śati tatra bhavati |}
\note[type=source, labelb=102, lem={mahāśūnyacakram}]{Ysv (PT): navaman tu mahāśūnyaṃ cakran tu tatparātparam | tad upari paraṃ kiñcin nāsti kiñcin mahāparam | mahācakraṃ siddhacakraṃ pūrṇagauryādisaṃjñakam | tanmadhye varttate padmaṃ sahasradalamadbhutam | ūrddhvavaktraṃ mahāvaktre [mahāvaktraṃ (YK)] varṇaśobhāpadaṃ mahat | sarvakalyāṇasampūrṇamasya tulyaṃ na vidyate | parimāṇaṃ vaktam asya [vaktum (YK)] manasā vacasā na hi | trikoṇakarṇikā tatra [°tantraṃ (YK)] varttate jagad īśvari |}
idānīṃ
\app{\lem[wit={ceteri},alt={°navama}]{navama}
  \rdg[wit={B}]{navamaṃ}
  \rdg[wit={U1}]{navamaś°}}cakrasya
\app{\lem[wit={ceteri}]{bhedāḥ}
  \rdg[wit={N2}]{bheda}}
\app{\lem[wit={ceteri}]{kathyante}
  \rdg[wit={L,B,N2,U2}]{kathyate}}/
%------------------------------
%tasya mahāśūnyacakram    iti  saṃjñā /  \E
%tasya mahāśūnyacakram    iti  saṃjñā /  \P
%tasya mahāśūnye cakram   iti  saṃjñā    \L
%tasye mahāśūnye cakram   iti  saṃjñā    \B
%tasya mahāśūnye cakreti       saṃjñā // \N1
%tasya mahāśūnyacakreti        saṃjñā // \N2
%tasya mahāśūnyacakreti        saṃjñā // \D
%tasya mahāśūnyacakreti        saṃjñā /  \U1
%\om /                                   \U2
%---------------------
%The designation of it is ``the \textit{cakra} of the great void (\textit{mahāśūnyacakra})''.
%------------------------------
tasya \app{\lem[wit={ceteri}, alt={mahāśūnya°}]{mahāśūnya}
  \rdg[wit={L,B,N1}]{mahāśūnye}
  \rdg[wit={U2}]{\om}
}\app{\lem[wit={ceteri},alt={°cakreti}]{cakreti}
  \rdg[wit={E,P}]{°cakram iti}
  \rdg[wit={L,B}]{cakram iti}
  \rdg[wit={U2}]{\om}}
\app{\lem[wit={ceteri}]{saṃjñā}
  \rdg[wit={U2}]{\om}}/
%------------------------------
%tadupary aparaṃ kimapi nāsti / \E
%tadupary aparaṃ kimapi nāsti \P
%tadupary        kimapi nāsti \B ??-> auch mögliche Lesart
%tadupari        kimapi nāsti \L
%tadupari aparaṃ kiṃapi nāsti / \N1
%tadupari aparaṃ kiṃapi nāsti / \N2
%tadupari aparaṃ kiṃapi nāsti / \D
%tadupari aparaṃ kiṃapi nāsti   \U1
% \om                           \U2
%---------------------
%kim api: somewhat, to a considerable extent, rather, much more, still, further. Śa
%---------------------
%Above that there is no other. 
%---------------------
\app{\lem[wit={E,P,B},alt={tad upary}]{tad\skp{-}upar\skm{y-a}}
  \rdg[wit={ceteri}]{tad upari}
  \rdg[wit={U2}]{\om}}\app{\lem[wit={ceteri}, alt={aparaṃ}]{\skp{y-a}paraṃ}
  \rdg[wit={B,L,U2}]{\om}}
\app{\lem[wit={ceteri}]{kimapi}
  \rdg[wit={N1,N2,D,U1}]{kiṃ api}
  \rdg[wit={U2}]{\om}} nāsti/
     \end{prose}
  \end{ekdosis}
\ekdpb*{}
%%%%%%%%%%%%%%%%%%%%%%%%%%%%%%%%%%%%%%%%%%
%%%%%%%%%%%%%%%%%%%%%%%%%%%%%%%%%%%%%%%%%%
%%%%%%%%PAGEBREAK%%%%%%%PAGEBREAK%%%%%%%%%
%%%%%%%%%%%%%%%%%%%%%%%%%%%%%%%%%%%%%%%%%%
%%%%%%%%%%%%%%%%PAGEBREAK%%%%%%%%%%%%%%%%%
%%%%%%%%%%%%%%%%%%%%%%%%%%%%%%%%%%%%%%%%%%
%%%%%%%%PAGEBREAK%%%%%%%PAGEBREAK%%%%%%%%%
%%%%%%%%%%%%%%%%%%%%%%%%%%%%%%%%%%%%%%%%%%
%%%%%%%%%%%%%%%%%%%%%%%%%%%%%%%%%%%%%%%%%%
%%%%%%%%%%%%%%%%%%%%%%%%%%%%%%%%%%%%%%%%%%
%%%%%%%%%%%%%%%%%%%%%%%%%%%%%%%%%%%%%%%%%%
%%%%%%%%PAGEBREAK%%%%%%%PAGEBREAK%%%%%%%%%
%%%%%%%%%%%%%%%%%%%%%%%%%%%%%%%%%%%%%%%%%%
%%%%%%%%%%%%%%%%PAGEBREAK%%%%%%%%%%%%%%%%%
%%%%%%%%%%%%%%%%%%%%%%%%%%%%%%%%%%%%%%%%%%
%%%%%%%%PAGEBREAK%%%%%%%PAGEBREAK%%%%%%%%%
%%%%%%%%%%%%%%%%%%%%%%%%%%%%%%%%%%%%%%%%%%
%%%%%%%%%%%%%%%%%%%%%%%%%%%%%%%%%%%%%%%%%%
%%%%%%%%%%%%%%%%%%%%%%%%%%%%%%%%%%%%%%%%%%
%%%%%%%%%%%%%%%%%%%%%%%%%%%%%%%%%%%%%%%%%%
%%%%%%%%PAGEBREAK%%%%%%%PAGEBREAK%%%%%%%%%
%%%%%%%%%%%%%%%%%%%%%%%%%%%%%%%%%%%%%%%%%%
%%%%%%%%%%%%%%%%PAGEBREAK%%%%%%%%%%%%%%%%%
%%%%%%%%%%%%%%%%%%%%%%%%%%%%%%%%%%%%%%%%%%
%%%%%%%%PAGEBREAK%%%%%%%PAGEBREAK%%%%%%%%%
%%%%%%%%%%%%%%%%%%%%%%%%%%%%%%%%%%%%%%%%%%
%%%%%%%%%%%%%%%%%%%%%%%%%%%%%%%%%%%%%%%%%%
\begin{ekdosis}
  \begin{prose}
\noindent
%------------------------------
%tadeva-mahāsiddhacakraṃ kathyate // \E
%tadeva-mahāsiddhacakraṃ kathyate    \P 
%tadeva-mahāsiddhacakraṃ kathyate // \B
%tadeva-mahāsiddhacakraṃ kathyate // \L
%tadeva-mahāsiddhacakraṃ kathyate // \N1
%tadeva-mahāsiddhacakraṃ kathyate // \N2
%tadeva-mahāsiddhacakraṃ kathyate // \D
%tadeva-mahāsiddhacakraṃ kathyate /  \U1
% \om                                \U2
%---------------------
%Therefore it is declared to be the \textit{cakra} of the great perfection (\textit{mahāsiddhacakra}).
%---------------------
tad-eva mahāsiddhacakraṃ kathyate/
%------------------------------
%       tasya           pūrṇagiripīṭha               etadṛśaṃ nāma /  \E 
%       tasya           pūrṇagiripīṭham-iti          etādṛśaṃ nāma    \P
%       tasya           pūrṇagiripīṭham-iti saṃjñā   etādṛsaṃ nāma    \B ->!!! 
%       tasya           pūrṇagiripīṭham-iti saṃjñā   etādṛsaṃ nāma    \L
%       tasya cakrasya  pūrṇagiri                    etādṛśaṃ nāma /  \N1
%       tasya cakrasya  pūrṇagiri                    etādṛśaṃ nāma /  \N2
%       tasya cakrasya  pūrṇagiri                    etādṛśaṃ nāma /  \D
%       tasya cakrasya  pūrṇagire                    etādṛśaṃ nāmaḥ   \U1
%madhye tasya           pūrṇagiripīṭham-iti          ekādaśaṃ nāma // \U2   
%-----------------------------
%Such a name of it is ``(divine) seat of Pūrṇagiri''.   
%------------------------------
\app{\lem[wit={ceteri}]{tasya}
  \rdg[wit={N1,N2,D,U1}]{tasya cakrasya}
  \rdg[wit={U2}]{madhye tasya}}
pūrṇagiri\app{\lem[wit={P,B,L,U2}, alt={°pīṭham}]{pīṭha\skm{m-i}}
  \rdg[wit={E}]{pīṭha}
  \rdg[wit={ceteri}]{\om}
}\app{\lem[wit={P,U2},alt={iti}]{\skp{m-i}ti}
  \rdg[wit={B,L}]{iti saṃjñā}
  \rdg[wit={ceteri}]{\om}}
\app{\lem[wit={ceteri}]{etādṛśaṃ}
  \rdg[wit={E}]{etadṛśaṃ}
  \rdg[wit={U2}]{ekādaśaṃ}}
\app{\lem[wit={ceteri}]{nāma}
  \rdg[wit={U1}]{nāmaḥ}}/
%------------------------------
%tasya mahāśūnyacakrasya madhye ūrdhvamukham iti raktavarṇaṃ sakalaśobhāspadam    \E
%tasya mahāśūnyacakrasya madhye ūrdhvamukham iti raktavarṇa--sakalaśobhāspadaṃ     \P
%tasya mahāśūnyacakrasya madhye ūrdhvamukhem iti raktavarṇaṃ sakalaśobhāspadaṃ // \B    
%tasya mahāśūnyacakrasya madhye ūrdhvamukham iti raktavarṇaṃ sakalaśobhāspadaṃ // \L
%tasya mahāśūnyacakramadhye     ūrdhvamukhaṃ atiraktavarṇaṃ  sakalaśobhāspadaṃ /   \N1 ->!!!
%tasya mahāśūnyacakramadhye     ūrdhvamukhaṃ atiraktavarṇaṃ  sakalaśobhāspadaṃ     \N2
%tasya mahāśūnyacakramadhye     ūrdhvamukhaṃ atiraktavarṇaṃ  sakalaśobhāspadaṃ /   \D
%tasya mahāśūnyacakramadhye     ūrdhvamukhaṃ atiraktavarṇaṃ  sakalaśobhāspadaṃ     \U1
%tasya mahāśūnyacakrasya        urdhvamukham-ativarṇaṃ       sakalaśobhanāsyadaṃ / \U2                                             
%------------------------------
%anekakalyāṇapūrṇaṃ sahasradalan      ekaṃ kamalaṃ  varttate / \E
%anekakalyāṇapūrṇaṃ sahasradalaṃ      ekaṃ kamalaṃ  vartate    \P
%anekakalyāṇapūrṇa--sahasradalaṃ      ekaṃ kamalaṃ  vartato    \B
%anekakalyāṇapūrṇaṃ sahasradalaṃ      ekaṃ kamalaṃ  vartate    \L
%anekakalyāṇapūrṇaṃ sahasradalaṃ      eka--kamalaṃ  varttate   \D
%anekakalyāṇapūrṇaṃ sahasradalaṃ      ekaṃ kamalaṃ  vartate    \N1
%anekakalyāṇapūrṇa--sahasradalaṃ      ekaṃ kamalaṃ  varttate    \N2
%anekakalyāṇapūrṇaṃ sahasradalaṃ           kamalaṃ  vartate /   \U1
%anekakalyāṇapūrṇaṃ // sahasradalaṃ   ekaṃ kamalaṃ  vartate / \U2
%Fragezeichen in |nepal ... schreiber Einfügung? 
%------------------------------
%In the middle of the \textit{mahāśūnyacakra} exists one lotus facing upward, very red in color with a thousand petals - an abode of brilliance and wholeness.
%------------------------------
tasya mahāśūnya\app{\lem[wit={ceteri},alt={°cakramadhye}]{cakramadhye}
  \rdg[wit={E,P,B,L}]{°cakrasya madhye}
  \rdg[wit={U2}]{°cakrasya}}
\app{\lem[wit={ceteri},alt={ūrdhvamukham}]{ūrdhvamukha\skp{m-a}}
  \rdg[wit={E,P,L}]{ūrdhmukham}
  \rdg[wit={U2}]{urdhvamukham}
  \rdg[wit={B}]{ūrdhvamukhem}}
\app{\lem[wit={ceteri}]{\skm{m-a}tiraktavarṇaṃ}
  \rdg[wit={E,L,B}]{iti raktavarṇaṃ}
  \rdg[wit={P}]{iti raktavarṇa°}
  \rdg[wit={U2}]{ativarṇaṃ}}
sakala\app{\lem[wit={ceteri},alt={°śobhāspadaṃ}]{śobhāspadaṃ}
  \rdg[wit={E}]{°śobhāspadam}
  \rdg[wit={U2}]{°śobhanāsyadaṃ}}
\app{\lem[wit={ceteri}]{anekakalyāṇapūrṇaṃ}
  \rdg[wit={B,N2}]{°pūrṇa°}}
sahasradalaṃ
\app{\lem[wit={ceteri}]{ekaṃ}
  \rdg[wit={D}]{eka°}
  \rdg[wit={U1}]{\om}}
kamalaṃ
\app{\lem[wit={ceteri}]{vartate}
  \rdg[wit={B}]{vartato}}/
%---------------------
%yasya           parimalo manaso vacaso na gocaraḥ // \E
%yasya           parimalo manasā vacasā na gocaraḥ /  \P
%yasya           parimalo manasā vacasā    gocaraḥ /  \L
%yasya           parimalo manasā vacasā na gocaraḥ /  \B
%yasya           parimalo manasā vacasā na gocaraḥ /  \N1
%yasya           parimalo manasā vacasā na gocara /   \N2
%yasya           parimalo manasā vacasā na gocaraḥ /  \D
%yasya           parimalo vacasā manasā na gocaraḥ    \U1
%yasya kamalasya parimalo manasā vācā   na gocara ..  \U2
%---------------------
%Whose fragrance is not in range by mind and speech. 
%Dessen Duft ist nicht in Reichweite von Geist und Sprache. 
%---------------------
\app{\lem[wit={ceteri}]{yasya}
  \rdg[wit={U2}]{yasya kamalasya}}
parimalo
\app{\lem[wit={E}]{manaso vacaso}
  \rdg[wit={P,L,B,N1,N2,D}]{manasā vacasā}
  \rdg[wit={U1}]{vacasā manasā}
  \rdg[wit={U2}]{manasā vācā}
}
\note[type=philcomm, labelb=103, lem={°manaso vacaso}]{All manuscripts at hand share this usage of the instrumentals. Only the printed edition conjectures the forms into the exspected genitiv. I adopted the variant of the printed edition to arrive at a grammatically correct text.}
\app{\lem[wit={ceteri}]{na}
  \rdg[wit={L}]{\om}
}
\app{\lem[wit={ceteri}]{gocaraḥ}
  \rdg[wit={N2,U2}]{gocara}}/
%---------------------
%tasya kamalasya madhye trikoṇarūpa-ikā karṇikā varttate /    \E
%tasya kamala----madhye trikoṇārūpā ekā karṇikā varttate/ \P
%tasya kamalasya madhye trikoṇarūpā ekā karṇikā varttate/     \L
%tasya kamalasya madhye trikoṇarūpā ekā karṇikā varttate/     \B
%tasya kamalasya madhye trikoṇarūpā eka karṇikā varttate/     \N1
%tasya kamalasya madhye trikoṇarūpā eka karṇikā varttate/     \N2
%tasya kamalasya madhye trikoṇarūpā ekā karṇikā varttate/     \D
%tasya kamalasya madhye trikoṇarūpā ekā karṇikā vartate       \U1
%tasya kamalasya madhye trikoṇarūpā ekā karṇikā vartate //    \U2
%---------------------
%In the middle of this lotus exists one pericarp having the shape of a triangle. 
%------------------------------
tasya
\app{\lem[wit={ceteri}]{kamalasya}
  \rdg[wit={P}]{kamala°}}
madhye
\app{\lem[wit={E}]{trikoṇarūpaikā}
  \rdg[wit={ceteri}]{trikoṇārūpā ekā}
  \rdg[wit={N1,N2}]{trikoṇārūpā eka}}
karṇikā vartate\dd{}
%------------------------------
%tatkarṇikāmadhye saptadaśī         niraṃjanarūpā kalā varttate/ \E
%tatkarṇikāmadhye saptadaśireṇa ekā niraṃjanarūpā kalā vartate// \L
%tatkarṇikāmadhye saptadaśireṇa ekā niraṃjanarūpā kalā vartate// \B
%tatkarṇikāmadhye saptadaśī     ekā niraṃjanarūpā kalā vartate// \P
%tatkarṇikāmadhye saptadaśī     ekā niraṃjanarūpā kalā vartate// \N1
%tatkarṇikāmadhye saptadaśī     ekā niraṃjanarūpā kalā vartate/  \N2
%tatkarṇikāmadhye saptadaśī     ekā niraṃjanarūpā kalā vartate// \D
%tatkarṇikāmadhye saptadaśī     ekā niraṃjanarūpā kalā vartate  \U1
%tatkarṇikāmadhye saptadaśī     eka niraṃjanarūpā kalā varttate/ \U2
%---------------------
%In the middle of the pericarp exists one seventeenth digit in the shape of a immaculé form.
%---------------------
\note[type=source, labelb=104, lem={saptadaśī}]{Ysv (PT): kalā saptadaśī tatra varttate parameśvari | nirañjanakalā sā tu koṭisūryasamaprabhā | koṭicandraprabhā caiva śītoṣṇādivivarjitā | asya dhyānāt sādhakasya manoduḥkhaṃ bhaven na hi | anantaparamānandasthānaṃ jñeyaṃ tadūrddhvataḥ [tadarddhataḥ (YK)] | ūrddhvagatakalā tatra tasya dhyānād bhaved iti | iti siddhirājayogaṃ strīṇāṃ bhogaṃ mahāsukham | gītavādyavinodādi saśivaṃ varddhate kṣitau | dhyānaṃ nirantarañ cāsya puṇyapāpe sthire [sthirau (YK)] na hi | nijarūpasya dṛṣṭiḥ syād dūrasyārthañ ca paśyati ||}
tatkarṇikāmadhye
\app{\lem[wit={ceteri}]{saptadaśī}
  \rdg[wit={L,B}]{saptadaśireṇa}}\note[type=philcomm, labelb=105, lem={saptadaśī}]{A \textit{saptadaśī kalā} appears frequently in Śaiva literature. References need to be added here.}
\app{\lem[wit={ceteri}]{ekā}
  \rdg[wit={E}]{\om}}
nirañjanarūpā kalā varttate/
%---------------------
%koṭisūryasamaprabhaṃ kalāyās tejo vartate /    \E
%koṭisūryasamaprabhā kalāyās tejo vartate /     \L
%koṭisūryasamaprabhā kalāyās tejo vartate /     \B
%koṭisūryasamaprabha kalāyās tejo vartate /     \P
%koṭisūryasamaprabhaṃ kalāyās tejo vartate /    \N1
%koṭisūryasamaprabhaṃ kalāyā  tejo varttate //  \N2
%koṭisūryasamaprabhaṃ kalāyās tejo vartate /    \D
%koṭisūryasadṛṣaprabhaṃ kalāyās tejo vartate /  \U1
%koṭisūryasamaprabhā // kalāyās tejo varttate / \U2
%---------------------
%A light of the part exists shining like a thousand suns. 
%------------------------------
koṭisūrya\app{\lem[alt={°samaprabhaṃ}, wit={ceteri}]{samaprabhaṃ}
  \rdg[wit={L,B,U2}]{samaprabhā}
  \rdg[wit={P}]{samaprabha}
  \rdg[wit={U1}]{sadṛṣaprabhaṃ}}
kalāyās-tejo vartate/
%------------------------------
%param udbhavo nāsti /     \E
%parim uṣṇabhavo nāsti /   \P
%parim uṣṇabhavo nāsti /   \L
%parim uṣṇabhavo nāsti /   \B
%parim uṣṇabhāvo nāsti /   \N1
%para  uṣṇabhāvo nāsti     \N2
%parim auṣṇabhāvo nāsti /  \D
%paraṃ uṣṇabhāvo nāsti     \U1
%param uṣṇabhāvo nāsti /   \U2
%---------------------
%[But] excessive heat is not arising. 
%------------------------------
\app{\lem[alt={param},wit={E,U1,U2}]{para\skp{m-u}}
  \rdg[wit={U1}]{paraṃ}
  \rdg[wit={N2}]{para}
  \rdg[wit={ceteri}]{parim}
}\app{\lem[wit={ceteri}, alt={uṣṇabhāvo}]{\skm{m-u}ṣṇabhāvo}
  \rdg[wit={P,L,B}]{uṣṇabhavo}
  \rdg[wit={D}]{auṣṇabhāvo}
  \rdg[wit={E}]{udbhavo}
}
nāsti/
%------------------------------
%koṭicandrasamaprabhā    śītalaṃ paraṃ   śītabhāvo   nāsti / \E
%koṭicandrasamaprabhā    śītalaṃ paraṃ   śītabhavo   nāsti / \P
%\om /                                                      \L
%koṭicandrasamaprabhā    śītalaṃ paraṃ   śītabhavo   nāsti / \B
%koṭicandrasamaprabhaṃ   śītalaparaṃ         bhavo   nāsti / \N1
%koṭicandrasamaprabhaṃ   śītalapara----------bhavo   nāsti // \N2
%koṭicaṃdrasamaprabhaṃ   śītalaparaṃ         bhavo   nāsti / \D
%koṭicaṃdrasamaṃ prabhaṃ śītalaṃ paraṃ       bhavo   nāsti / \U1
%koṭicaṃdrasamaprabhā    śītalaṃ paraṃ śītalabhāvo   nāsti / \U2
%---------------------
%Shining like a thousand moons, excess of cold is not arising.
%---------------------
koṭicandra\app{\lem[alt={°samaprabhaṃ},wit={N1,N2,D}]{samaprabhaṃ}
  \rdg[wit={E,P,B,U2}]{°samaprabhā}
  \rdg[wit={U1}]{°samaṃ prabhaṃ}
  \rdg[wit={L}]{\om}}
\app{\lem[wit={N1,D}]{śītalaparaṃ}
  \rdg[wit={ceteri}]{śītalaṃ paraṃ}
  \rdg[wit={N2}]{śītalapara}
  \rdg[wit={L}]{\om}}
\app{\lem[wit={ceteri}]{bhāvo} 
  \rdg[wit={E,P,B}]{śītabhāvo}
  \rdg[wit={U2}]{śītalabhāvo}
  \rdg[wit={L}]{\om}}
nāsti/
%------------------------------
%asyāḥ kalāyā   dhyānayogāt    sādhakasya manasi duḥkhaṃ na bhavati / \E
%asyāḥ kalādhyānayogāt         sādhakasya manasi duḥkhaṃ na bhavati / \P
%asyāḥ kalāyāḥ  dhyānakaraṇāt  sādhakasya manasi duḥkhaṃ na bhavati / N1
%asyā kalāyā    dhyānakaraṇāt  sādhaka----manasi duḥkhaṃ na bhavati / N2
%asyāḥ kalāyāḥ  dhyānakaraṇāt  sādhakasya manasi duḥkhaṃ na bhavati / D
%
%asyāḥ kalāyā   dhyānayogāt    sādhakasya manasi duḥkhaṃ bhavati /B
%asyāḥ kalāyā   dhyānayogāt    sādhakasya manasi duḥkhaṃ bhavati /L
%asyāḥ kalāyā   dhyānakaraṇāt/ sādhakasya manasi duḥkhaṃ na bhavati / U1
%asyā  kalāyāḥ  dhyānayogāt//  sādhakasya manasi duḥkhaṃ na bhavati // \U2
%atrastāne 'haṃ devatā// sohaṃ śaktiḥ// ātmāṛṣiḥ// mokṣamārhaḥ// haṃbhrahmordhaṃ// haṃcakra iti// agnicakre sakaro bhavatī// prāṇīrūḍho bhave jjīva ārohaty avarohati bhavaguhāsthānaṃ pitavarṇaṃ// koṭisūryapratikāśaṃ tejaḥ sadoditaprabhā śīvodevatā// mūlamāyā śaktiḥ// hara ātmālayāvsthā dhvanisthirānādātmako khaṃḍa 'dhvani// adhorāmudrā// mūlamāyā// prakṛtidehaḥ// vāṅmanogocaraḥ// niḥprapaṃcaḥ// niḥsaṃśayaḥ// nistaraṃganirlepalakṣaṃ laya// dhyānasamādhi 
%---------------------
%asyāḥ kalāyā dhyānakaraṇāt\varc{\emend kalāyāḥ dhyānakaraṇāt \nepal \dehlia}{kalāyā dhyānayogāt \nepal \dehlia kalādhyānayogāt \pune} sādhakasya manasi duḥkhaṃ na\varc{na \edprint \pune \nepal \dehlia}{\om \oxford \lalchand} bhavati /
%Due to the exercise of meditation upon the digit suffering does not arise in the mind of the practitioner (anymore). 
%------------------------------
\app{\lem[wit={ceteri}]{asyāḥ}
  \rdg[wit={N2,U2}]{asyā}}
kalā\app{\lem[wit={E,B,L,N2,U1}, alt={°yā}]{yā}
  \rdg[wit={N1,D}]{°yāḥ}
  \rdg[wit={E,B,L}]{°yā}
  \rdg[wit={U2}]{°yāḥ}
  \rdg[wit={P}]{\om}}
dhyāna\app{\lem[wit={N1,N2,D,U1}, alt={°karaṇāt}]{karaṇāt}
  \rdg[wit={ceteri}]{°yogāt}}
\app{\lem[wit={ceteri}]{sādhakasya}
  \rdg[wit={N2}]{sādhaka°}}
duḥkhaṃ
\app{\lem[wit={ceteri}]{na}
  \rdg[wit={B,L}]{\om}}
bhavati/
%%%%%%%%%%%%%
%%%%%%%%%%%%
%%%%%%%%%%%%
%%%%%%%%%%%%
%%%%%%%%%%%%
\extra{atra
   \app{\lem[type=emendation, resp=egoscr]{sthāne}
    \rdg[wit={U2}]{\korr stāne}} 'haṃ devatā\dd{}
  sohaṃ śaktiḥ\dd{}
  ātmāṛṣiḥ\dd{}
  \app{\lem[type=emendation, resp=egoscr]{mokṣo}
    \rdg[wit={U2}]{\korr mokṣa°}} mārgaḥ\dd{}
   \app{\lem[type=emendation, resp=egoscr]{ahaṃ brahmordhvaṃ}
    \rdg[wit={U2}]{\korr haṃ brahmordhaṃ}}\dd{}
   \app{\lem[type=emendation, resp=egoscr]{ahaṃ cakra iti}
     \rdg[wit={U2}]{\korr haṃcakra iti}}\dd{}
   agnicakre
   \app{\lem[type=emendation, resp=egoscr]{sakāro}
     \rdg[wit={U2}]{\korr sakaro}}
   \app{\lem[type=emendation, resp=egoscr]{bhavati}
     \rdg[wit={U2}]{\korr bhavatī}}\dd{}
   prāṇī rūḍho bhavej-jīva ārohaty-avarohati\note[type=philcomm, labelb=106, lem={prāṇī}]{Find parallels of hemistich.}\dd{}
bhavaguhā sthānaṃ\dd{}
   \app{\lem[type=emendation, resp=egoscr]{pitaṃ}
     \rdg[wit={U2}]{\korr pita°}} varṇaṃ\dd{}
   koṭisūryapratikāśaṃ tejaḥ\dd{}
   \app{\lem[type=emendation, resp=egoscr]{sadoditā}
     \rdg[wit={U2}]{\korr sadodita°}} prabhā\dd{}
   \app{\lem[type=emendation, resp=egoscr]{śivo}
     \rdg[wit={U2}]{\korr śīvo}} 
   devatā\dd{}
   mūlamāyā śaktiḥ\dd{}
   \app{\lem[type=emendation, resp=egoscr]{harātmālayāvasthā}
     \rdg[wit={U2}]{\korr hara ātmālayāvasthā}}\dd{}
   dhvanisthirānādātmako \app{\lem[type=emendation, resp=egoscr]{'khaṇḍadvaniḥ}
     \rdg[wit={U2}]{\korr khaṃḍadhvani}}\dd{} 
   aghorā mudrā\dd{}
   \app{\lem[type=emendation, resp=egoscr]{mūlā} %macht diese emdendation wirklich Sinn? 
     \rdg[wit={U2}]{\korr mūla°}} māyā\dd{}
   \app{\lem[type=emendation, resp=egoscr,alt={prakṛtir}]{prakṛti\skp{r-de}}
     \rdg[wit={U2}]{\korr prakṛti°}}\skm{r-de}haḥ\dd{}
   vāṅmano 'gocaraḥ\dd{} %%
   niḥprapañcaḥ\dd{}
   niḥsaṃśayaḥ\dd{}
   nistaraṃganirlepalakṣaṃ %%%see pw Vol. 3, S. 229 for nistaranga
  \app{\lem[type=emendation, resp=egoscr]{layo}
     \rdg[wit={U2}]{\korr laya}}
   \app{\lem[type=emendation, resp=egoscr]{dhyānaḥ samādhiḥ}
     \rdg[wit={U2}]{\korr dhyānasamādhi}}\dd{}}
%\extra{Here at this location the ``I''(\textit{aham}) is the deity. The ``he is I'' (\textit{so 'ham}) is the power. This self is the Ṛṣi. The path is liberation. Brahma is the I above. ``I'm a circle''. In the circle of fire is the letter "sa". [There?] life arises, the living soul ascends and decends. The place is the hidden place of being. The colour is yellow. The light is the shine of ten million suns. The shine is always and visible. Śiva is the deity. The power is primordial illusion. The state is the dissolution of the self into Hara\footnote{Epiphet of Śiva.}. The transcendental sound has the nature of a sound with stable resonance. The seal is the ``fearless''. The illusion is the root. The body is the original matter. It is not within reach of speech and mind. It is without delusion. It is without doubt. The unaffected and undefiled goal is dissolution, meditation [and] final absorption.}
%---------------------
%tadupari anaṃtaparamānandasya sthānam / \E
%tadupari anaṃtaparamānandasya sthānaṃ   \P
%tadupari anantaparamānaṃdasya sthānam / \N1
%tadupari anantaparamānaṃdasya sthānam / \N2
%tadupari anantaparamānaṃdasya sthānaṃ / \D
%tadupari anantaparamānaṃdasya sthānam vartate/ \B
%tadupari anaṃtaparamānaṃdasya sthānam vartate/ \L
%tadupari alakṣaparamānaṃdasya sthānam   \U1
%tadupari anaṃtaparamānaṃdasya sthānaṃ// U2
%---------------------
%Above that is the place of infinite supreme bliss.
%---------------------
tadupari
\app{\lem[wit={ceteri}, alt={ananta°}]{ananta}
  \rdg[wit={U1}]{alakṣa°}}paramānaṃdasya
\app{\lem[wit={ceteri}]{sthānam}
  \rdg[wit={D,U2}]{stānaṃ}
  \rdg[wit={B,L}]{sthānam vartate}}/
 \end{prose}
\end{ekdosis}
\ekdpb*{}
%%%%%%%%%%%%%%%%%%%%%%%%%%%%%%%%%%%%%%%%%%
%%%%%%%%%%%%%%%%%%%%%%%%%%%%%%%%%%%%%%%%%%
%%%%%%%%PAGEBREAK%%%%%%%PAGEBREAK%%%%%%%%%
%%%%%%%%%%%%%%%%%%%%%%%%%%%%%%%%%%%%%%%%%%
%%%%%%%%%%%%%%%%PAGEBREAK%%%%%%%%%%%%%%%%%
%%%%%%%%%%%%%%%%%%%%%%%%%%%%%%%%%%%%%%%%%%
%%%%%%%%PAGEBREAK%%%%%%%PAGEBREAK%%%%%%%%%
%%%%%%%%%%%%%%%%%%%%%%%%%%%%%%%%%%%%%%%%%%
%%%%%%%%%%%%%%%%%%%%%%%%%%%%%%%%%%%%%%%%%%
%%%%%%%%%%%%%%%%%%%%%%%%%%%%%%%%%%%%%%%%%%
%%%%%%%%%%%%%%%%%%%%%%%%%%%%%%%%%%%%%%%%%%
%%%%%%%%PAGEBREAK%%%%%%%PAGEBREAK%%%%%%%%%
%%%%%%%%%%%%%%%%%%%%%%%%%%%%%%%%%%%%%%%%%%
%%%%%%%%%%%%%%%%PAGEBREAK%%%%%%%%%%%%%%%%%
%%%%%%%%%%%%%%%%%%%%%%%%%%%%%%%%%%%%%%%%%%
%%%%%%%%PAGEBREAK%%%%%%%PAGEBREAK%%%%%%%%%
%%%%%%%%%%%%%%%%%%%%%%%%%%%%%%%%%%%%%%%%%%
%%%%%%%%%%%%%%%%%%%%%%%%%%%%%%%%%%%%%%%%%%
%%%%%%%%%%%%%%%%%%%%%%%%%%%%%%%%%%%%%%%%%%
%%%%%%%%%%%%%%%%%%%%%%%%%%%%%%%%%%%%%%%%%%
%%%%%%%%PAGEBREAK%%%%%%%PAGEBREAK%%%%%%%%%
%%%%%%%%%%%%%%%%%%%%%%%%%%%%%%%%%%%%%%%%%%
%%%%%%%%%%%%%%%%PAGEBREAK%%%%%%%%%%%%%%%%%
%%%%%%%%%%%%%%%%%%%%%%%%%%%%%%%%%%%%%%%%%%
%%%%%%%%PAGEBREAK%%%%%%%PAGEBREAK%%%%%%%%%
%%%%%%%%%%%%%%%%%%%%%%%%%%%%%%%%%%%%%%%%%%
%%%%%%%%%%%%%%%%%%%%%%%%%%%%%%%%%%%%%%%%%%
\begin{ekdosis}
  \begin{prose}
    \noindent
%---------------------
%tatrordhvaśaktiḥ / \E
%tatordhvaśaktiḥ \P
%rdhaśakti ardhaśakti \B
%rdhaśakti ardhaśakti \L
%tatrordhvaśaktiḥ / \N1
%tatra ūrdhva śaktiḥ / \D
%tatra ūrdhva śakti / \N2
%urdhvaśaktir         \U1
%tatrordhvaśaktiḥ// \U2
%---------------------
%There above is \textit{śakti},
%------------------------------
\app{\lem[wit={E,N1,U2}]{tatrordhvaśaktiḥ}
  \rdg[wit={P}]{tatordhvaśaktiḥ}
  \rdg[wit={U1}]{urdhvaśaktir}
  \rdg[wit={D}]{tatra ūrdhva śaktiḥ}
  \rdg[wit={N2}]{tatra ūrdhva śakti}
  \rdg[wit={B,L}]{rdhaśakti ardhaśakti}}/
%------------------------------
%etādṛśī  saṃjñā   ekā kalā vartate / \E
%ekādaśā  saṃjñā   ekā kalā vartate   \P
%etādṛśī  saṃjñā   ekā kalā vartate /  \N1
%etādṛśī  saṃjñā   ekā kalā varttate / \N2
%etādṛsaṃ saṃjñā   ekā kalā vartate / \D
%ekādaśā  saṃjñā   ekā kalā vartate / \B
%ekādaśā  saṃjñā   ekā kalā vartate / \L
%etādṛśī  saṃjñakā ekā kalā vartate /  \U1
%etādṛśā  saṃjñā   ekā kalā vartate/ \U2 
%---------------------
%Being designated as such she is one single digit. 
%------------------------------
\app{\lem[wit={ceteri}]{etādṛśī}
  \rdg[wit={U2}]{etādṛśā}
  \rdg[wit={D}]{etādṛsaṃ}
  \rdg[wit={P,B,L}]{ekādaśā}}
\app{\lem[wit={ceteri}]{saṃjñā}
  \rdg[wit={U1}]{saṃjñakā}}
ekā kalā vartate/ 
%------------------------------
%asyāḥ  kalāyā   dhyānakāraṇāt     puruṣo yadicchati / \E
%asyāḥ  kalāyā   dhyānakāraṇāt     puruṣo yadicchati ?Zeichen? \P
%asyāḥ  kalāyā   dhyānakāraṇāt     puruṣo yadicchati  tad bhavati \N1
%tasyāḥ kalāyāḥ  dhyānakāraṇāt     puruṣo yadicchati  tad bhavati \N2
%asyāḥ  kalāyā   dhyānakāraṇā      puruṣo yadicchati  tad bhavati \D
%asyāḥ  kalāyā   dhyānakāraṇāt /   puruṣo yadicchati / \B
%asyāḥ  kalāyā   dhyānakāraṇāt /   puruṣo yadicchati / \L
%asyā   kalāyā   dhyānakāraṇāt     puruṣo yadicchati tad bhavati vā \U1
%asyāḥ  kalāyāḥ  dhyānakāraṇāt //  puruṣo yadicchati // \U2
%---------------------
%Due to the exercise of meditation on this part the person manifests whatever he wishes for.
%------------------------------
\app{\lem[wit={ceteri}]{asyāḥ}
  \rdg[wit={U1}]{asyā}
  \rdg[wit={N2}]{tasyāḥ}}
\app{\lem[wit={ceteri}]{kalāyā}
  \rdg[wit={N2,U2}]{kalāyāḥ}}
\app{\lem[wit={ceteri}]{dhyānakāraṇāt}
  \rdg[wit={D}]{dhyānakāraṇā}}
puruṣo yad-icchati
\app{\lem[wit={N1,N2,D}, alt={tad bhavati}]{tad-bhavati}
  \rdg[wit={U1}]{tad bhavati vā}
  \rdg[wit={ceteri}]{\om}}/ 
%------------------------------
%tasya sukhabhogavataḥ / \E
%tasya sukhabhogavataḥ \P
%rājya-sukhabhogavataḥ \N1
%rājya-sukhabhogavataḥ \N2
%rājya-sukhabhogavṛtaḥ \D !!!
%tasya-khaṃ bhogavataṃ / \B
%tasya-sukhaṃ bhogavaṃtaṃ / \L
%rājya-sukhabhogavataḥ \U1
%tasya-sukhabhogavataḥ / \U2
%---------------------
%He is furnished with royal pleasure and enjoyment. 
%------------------------------
\note[type=testium, labelb=107, lem={rājyasukhabhoga°}]{Ysg: rājyasukhabhogavatah̤ strī vilāsavataḥ saṃgītavinoda prekṣāvato pi sādhakasya śuklapakṣacaṃdravat pratidinaṃ tejaso vapuṣaś ca vṛddiḥ puṇyapāpasya śārbhāvaḥ nijasva rūpaprakāśasāmarthaṃ dūrasthapy arthasya samīpastham iva darśanaṃ ca bhavati | cha | tad uktaṃ tattvajñānapradīpikāyāṃ ||}
\note[type=philcomm, labelb=108, lem={rājyasukhabhoga°}]{Here ends the testimonia of the \textit{Yogasaṃgraha}.}
\app{\lem[wit={D}]{rājyasukhabhogavṛtaḥ}
  \rdg[wit={N1,N2,U1}]{rājyasukhabhogavataḥ}
  \rdg[wit={E,P,U2}]{tasya sukhabhogavataḥ}
  \rdg[wit={B}]{tasya khaṃ bhogavataṃ}
  \rdg[wit={L}]{tasya sukhaṃ bhogavaṃtaṃ}}/
%------------------------------
%strīmadhye     vilāsavataḥ    saṃgītavilāsavataḥ vinodaprekṣāvataḥ        puruṣasya pratidinaṃ śuklapakṣe candrakalāvat   kalā     vardhate/   \E
%strīmadhye     vilāsavataḥ    saṃgītavinodaprekṣāvataḥ              eva   puruṣasya pratidinaṃ śuklapakṣe candrakalāvat   kalā     vardhate /  \P
%strīmadhye     vilāsavaṃtaṃ   saṃgītaṃ prekṣāvatāḥ //               evaṃ  puruṣasya pratidinaṃ śuklapakṣe caṃdrakalāvat / kalā     vartate /   \L
%strīmadhye     vilāsavaṃtaṃ   saṃgītaṃ vinodavaṃtaṃ prekṣāvaṃtāḥ // eva   puruṣasya pratidinaṃ śuklapakṣe caṃdrakalāvat / kalā     vartate /   \B
%strīmadhye     vilāsavataḥ    saṃgītavinodaprekṣyāvataḥ             evaṃ  puruṣasya pratidinaṃ śuklapakṣe candrakalā vṛddhivato?   vardhate / \N1
%śrī strīmadhye vilāsavataḥ    saṃgītavinodaprekṣāvataḥ              evaṃ  puruṣasya pratidinaṃ śuklapakṣa candrakalā vṛddhi vaṃto  varttate /  \N2
%strīmadhye     vilāsavataḥ // saṃgītavinodaprekṣyāvataḥ //          evaṃ  puruṣasya pratidinaṃ śuklapakṣe candrakalā vṛddhivato    vardhate / \D
%strīmadhye     vilāśavataḥ    saṃgītavinodaprekṣyāvataḥ             eka   puruṣasya pratidinaṃ śuklapakṣe caṃdrakalā vṛddhir       varddhate / \U1
%strīmadhye     vilāsavata     saṃgītavinodaprekṣāvata//             evaṃ  puruṣasya pratidinaṃ śuklapakṣe candrakalāvat   kalā     varttate/   \U2
%---------------------
%(Selbst) bei einem Menschen, der sich inmitten von Frauen vergnügt, (und) ein Musikvergnügen
%ansieht, wächst täglich die Kraft (kalā = śakti?) wie die "kalā" (Phase) des Mondes in der hellen Monatshälfte.
%The \textit{kalā} of a person grows daily, like the \textit{kalā} of the moon in the bright half of the month, even amusing oneself amongst women and watching a musical pleasure.
%(Even) amusing oneself amongst women, and watching musical pleasures, the \textit{kāla} of the person grows daily like the \textit{kalā} of the moon in the bright half of the month. 
%------------------------------
\app{\lem[wit={ceteri}]{strīmadhye}
  \rdg[wit={N2}]{śrī strīmadhye}}
\app{\lem[wit={ceteri}]{vilāsavataḥ}
  \rdg[wit={U2}]{vilāsavata°}
  \rdg[wit={L,B}]{vilāsavaṃtaṃ}} 
saṃgīta\app{\lem[wit={N1,D,U1},alt={°vinodaprekṣyāvataḥ}]{vinodaprekṣyāvataḥ}
  \rdg[wit={P,N2}]{°vinodaprekṣāvataḥ}
  \rdg[wit={U2}]{°vinodaprekṣāvata}
  \rdg[wit={B}]{°ṃ vinodavaṃtaṃ prekṣāvaṃtāḥ}
  \rdg[wit={E}]{°vilāsavataḥ vinodaprekṣāvataḥ}
  \rdg[wit={L}]{°ṃ prekṣāvatāḥ}}
 \app{\lem[wit={P,B}]{eva}
  \rdg[wit={ceteri}]{evaṃ}
  \rdg[wit={U1}]{eka}}
puruṣasya pratidinaṃ śuklapakṣe
candrakalā\app{\lem[wit={E,P,L,B,U2},alt={°vat kalā}]{vat kalā}
  \rdg[wit={N1,D}]{vṛddhivato}
  \rdg[wit={N2}]{vṛddhi vaṃto}
  \rdg[wit={U1}]{vṛddhir}}
\app{\lem[wit={E,P,N1,D,U1}]{vardhate}
  \rdg[wit={ceteri}]{vartate}}/
%------------------------------
%puṇyapāpe  'sya śarīraṃ   na spṛśataḥ /    \E
%\om                                     \P
%puṇyapāpe  asya śarīrena     spṛśataḥ /      \N1
%puṇyapāpe  asya śarīrena     spṛśataḥ /      \N2
%puṇyapāpe  asya śarīrena     spṛśataḥ /      \D
%puṇyapāpe  asya śarīrasya na spṛśataḥ // \B
%puṇyapāpe  asya śarīrasya na spṛśataḥ // \L
%puṇyapāpau asya śarīrena     spṛśāt         \U1
%puṇyapāpe  asya śarīraṃ   na spṛśataḥ // \U2
%---------------------
%puṇyapāpe\varc{puṇyapāpe \edprint \lalchand \oxford \nepal \dehlia}{\om \pune} 'sya\varc{'sya \edprint}{asya \nepal \dehlia \oxford \lalchand \om \pune} śarīrasya\varc{śarīrasya \lalchand \oxford}{śarīraṃ \edprint śarīrena \nepal \dehlia \om \pune} na\varc{na \edprint \oxford \lalchand}{\om \nepal \dehlia \pune} spṛśataḥ\varc{spṛśataḥ \edprint \lalchand \oxford \nepal \dehlia}{\om \pune} /
%---------------------
%His body is not affected by merit and sin. 
%------------------------------
\app{\lem[wit={ceteri}]{puṇyapāpe}
  \rdg[wit={U1}]{puṇyapāpau}
\rdg[wit={P}]{\om}}
\app{\lem[wit={E}]{'sya}
  \rdg[wit={P}]{\om}
  \rdg[wit={ceteri}]{asya}}  
śarīr\app{\lem[wit={B,L}, alt={°asya}]{asya}
  \rdg[wit={N1,N2,D,U1}]{°ena}
  \rdg[wit={E,U2}]{°aṃ}
  \rdg[wit={P}]{\om}}
\app{\lem[wit={E,B,L,U2}]{na}
  \rdg[wit={N1,N2,D,U1,P}]{\om}}
spṛ\app{\lem[wit={ceteri},alt={°śataḥ}]{śataḥ}
  \rdg[wit={U1}]{°śāt}}/
%------------------------------
%                          nirantaradhyānakaraṇāt     nijasvarūpaṃ prakāśanasāmarthyaṃ bhavati / \E
%                          \om until .....            nijasvarūpaprakāśasāmarthyaṃ     bhavati / \P
%                          niraṃtaraṃ dhyānakaraṇāt   nijasvarūpaprakāśasāmarthyaṃ     bhavati / \B
%                          niraṃtaraṃ dhyānakaraṇāt// nijasvarūpaprakāśasāmarthyaṃ     bhavati / \L
%                          nirantaradhyānakaraṇāt /   nijasvarūpaprakāśasāmarthyaṃ     bhavati / \N1 <-----
%                          niraṃtaradhyānakaraṇāt /   nijasvarūpaprakāśasāmarthyaṃ     bhavati // \N2
%                          nirantaradhyānakaraṇāt /   nijasvarūpaprakāśasāmarthyaṃ     bhavati / \D
%                          nirantaradhyānakaraṇāt /   nijasvarūpaprakāśasāmarthyaṃ     bhavati    \U1
%evaṃ puruṣasya pratidinaṃ niraṃtaraṃ dhyānakaraṇāt   nijasvarūpaṃ prakāśanasāmarthyaṃ bhavati// \U2 
%---------------------
%Due to uninterrupted meditation the power of the light of the innate nature arises. 
%------------------------------
\app{\lem[wit={ceteri}]{nirantaradhyānakaraṇāt}
  \rdg[wit={B,L}]{niraṃtaraṃ dhyānakaraṇāt}
  \rdg[wit={U2}]{evaṃ puruṣasya pratidinaṃ niraṃtaraṃ dhyānakaraṇāt}
  \rdg[wit={P}]{\om}}
nijasvarūpa\app{\lem[wit={ceteri},alt={°prakāśa°}]{prakāśa}
  \rdg[wit={E,U2}]{°ṃ prakāśana°}
}sāmarthyaṃ bhavati/
%------------------------------
%dūrasthopi ca dūrasthavastu                   samīpa iva   paśyati // \E
%dūrasthamapi                                  samīpam iva  paśyati // \N1
%dūrasthamapi                                  samīpaṃ iva  paśyati // \N2
%dūrasthamapy-arthaṃ                           samīpa iva   paśyati // \D
%dūrasthamapi padārthaṃ                        samīpa iva   paśyati // \B
%dūrasthamapi parārthaṃ                        samīpa iva   paśyati // \L
%dūrasthamapi padārthaṃ                        samīpa iva   paśyati // \P
%dūrasthamapy-arthaṃ                           samīpam eva  paśyati // \U1
%dūrasthamapi bhavati //dūrasthamapi padārthaṃ samīpa iva   paśyati // \U2
%------------------------------
%dūrasthamapyarthaṃ\varc{dūrasthamapyarthaṃ \dehlia}{dūrasthamapi padārthaṃ \oxford \pune durasthamapi parārthaṃ \lalchand sūrastamapi \nepal ca dūrasthavastu \edprint} samīpa\varc{samīpa \dehlia \edprint \lalchand \oxford \pune}{samīpam \nepal} iva paśyati //
%------------------------------
%He sees remotely located objects as if they'd be near.
%------------------------------
dūra\app{\lem[wit={D,U1},alt={°stham apy arthaṃ}]{stham-apy-arthaṃ}
  \rdg[wit={B,P}]{°stham api padārthaṃ}
  \rdg[wit={L}]{°stham api parārthaṃ}
  \rdg[wit={E}]{°sthopi ca dūrasthavastu}
  \rdg[wit={N1,N2}]{°stham api}
  \rdg[wit={U2}]{°stham api bhavati || dūrastham api padārthaṃ}}
\app{\lem[wit={ceteri}]{samīpa}
  \rdg[wit={N1}]{samīpam}
  \rdg[wit={N2}]{samīpaṃ}
  \rdg[wit={U1}]{samīpam}}
\app{\lem[wit={ceteri}]{iva}
  \rdg[wit={U1}]{eva}} 
paśyati\dd{}
\end{prose}
\end{ekdosis}
%%%%%%%%%%%%%%%%
%%%%%%%%%%%%%%%%
%%%%%%%%%%%%%%%%
%%%%%%%%%%%%%%%%
%%%%%%%%%%%%%%%
\begin{ekdosis}
 \ekddiv{type=ed}
   \bigskip
    \centerline{\textrm{\small{[Lakṣyayoga]}}}
    \bigskip
    \begin{prose}
%------------------------------
%idānīṃ sukhasādhyo lakṣyayogaḥ kathyate / \E
%idānīṃ sukhasādho  lakṣyayogaḥ kathyate / \P
%idānīṃ sukhasādho  lakṣayogaḥ  kathyate / \B
%idānīṃ sukhasādhe  lakṣayogaḥ  kathyate // \L
%idānīṃ sukhasādhyo lakṣyayogaḥ kathyate / \N1
%idānīṃ sukhasādhya lakṣanayogaḥ kathyate / \N2
%idānīṃ sukhasādhyo lakṣyayogaḥ kathyate / \D
%idānīṃ sukhasādhyopalakṣayogaḥ kathyate / \U1
%idānīṃ sukhasādhyo lakṣyayogaḥ kathyate / \U2
%------------------------------
%Now the yoga of fixation{\textit{lakṣyayoga}}, which is easily accomplished is explained. 
%------------------------------
      \note[type=source, labelb=109, lem={lakṣyayogaḥ}]{Ysv (YK): sukhasādhyaṃ lakṣayogam idānīṃ śrṛṇu pārvati | pañcadhā lakṣayogaś ca ūrdhvalakṣādibhedataḥ [ūrddha (PT)] ||1||}  
      idānīṃ
      \app{\lem[wit={ceteri}]{sukhasādhyo}
        \rdg[wit={N2}]{°sādhya}
        \rdg[wit={P,B}]{°sādho}
        \rdg[wit={L}]{°sādhe}
        \rdg[wit={U1}]{°sādhyopa°}}
    \app{\lem[wit={ceteri}]{lakṣyayogaḥ}
        \rdg[wit={B,L}]{lakṣayogaḥ}
        \rdg[wit={U1}]{°lakṣayogaḥ}
        \rdg[wit={N2}]{lakṣanayogaḥ}}
      kathyate/
%------------------------------      
%asya lakṣyayogasya  paṃcabhedā     bhavanti   ūrdhvalakṣyam / adholakṣyam / lakṣyam /      bāhyalakṣyam /  aṃtaralakṣyam /  \E
%asya lakṣyayogasya  paṃcabhedā     bhavanti   ūrdhvalakṣyam   adholakṣyam / madhyalakṣyam  bāhyalakṣyam    aṃtaralakṣyam /  \P
%asya lakṣayogasya   paṃce bhedāḥ   bhavaṃtī   ūrdhvalakṣam//  adholakṣam// bāhyakṣam//                     aṃtaralakṣam //  \B
%asya lakṣayogasya   paṃcabhedāḥ    bhavaṃti   ūrdhvalakṣam    adholakṣam// madhyalakṣam//  bāhyakṣam//     aṃtaralakṣam //  \L
%     lakṣyayogasya  paṃcabhedā     bhavaṃti// urdhvalakṣya    adholakṣya   bāhyalakṣya     madhyalakṣya    antaralakṣya //  \N1
%     lakṣanayogasya paṃcabhedā     bhavati//  urdhvalakṣa     adholakṣa    bāhyalakṣa      madhyalakṣa     antaralakṣa //   \N2
%     lakṣyayogasya  paṃcabhedā     bhavaṃti// urdhvalakṣya    adholakṣya   bāhyalakṣya     madhyalakṣya    antaralakṣya //  \D
%a----lakṣayogasya   paṃcabhedā     bhavati    urdhvalakṣa                  bāhyalakya      madhyalakṣa     aṃtaralakṣya     \U1
%asya lakṣayogasya   paṃcabhedā     bhavaṃti// ūrdhvalakṣam//  adholakṣam/  bāhyalakṣyam /  madhyalakṣaṃ/   sarvalakṣyam /   \U2
%------------------------------
%Of this yoga of fixation (\textit{lakṣyayoga}) there are five subdivisions: 1. The upward directed fixation {\textit{ūrdhvalakṣya}), 2. the downward directed fixation (\textit{adholakṣya}),3. the central fixation (\textit{madhyalakṣya}) 4. the outer fixation (\textit{baḥyalakṣya}), 5. the inner fixation (\textit{antaralakṣya}).
%------------------------------
      \note[type=source, labelb=110, lem={ūrdhvalakṣyam}]{Ysv (YK): ūrdhvalakṣam [ūrddha° (PT)] adholakṣaṃ [°lakṣo (PT)] vāhyalakṣaṃ [bāhyalakṣas (PT)] tathaiva ca | madhyalakṣaṃ [°lakṣas (PT)] tathā jñeyam [jñeyo (PT)] antarlakṣaṃ [°lakṣas (PT)] tathaiva ca ||2||}
      \app{\lem[wit={E,P,B,L,U2}]{asya}
        \rdg[wit={ceteri}]{\om}}
      \app{\lem[wit={ceteri},alt={lakṣya°}]{lakṣya}
        \rdg[wit={B,L,U2}]{lakṣa°}
        \rdg[wit={U1}]{alakṣa°}
        \rdg[wit={N2}]{lakṣana°}}yogasya
      \note[type=philcomm, labelb=111, lem={lakṣyayogasya}]{The designation of this type of yoga is transmitted in various variants. Given the list of the 15 yogas at the beginning of the text it is very likely that the correct name of the yoga is \textit{lakṣyayoga} and not \textit{lakṣayoga} or \textit{lakṣanayoga}.}
      \app{\lem[wit={ceteri}]{pañcabhedā}
        \rdg[wit={L}]{paṃcabhedāḥ}
        \rdg[wit={B}]{paṃce bhedāḥ}}
     \app{\lem[wit={ceteri}]{bhavanti}
       \rdg[wit={B}]{bhavaṃtī}
       \rdg[wit={N2,U1}]{bhavati}}/
    1 \app{\lem[wit={E,P}]{ūrdhvalakṣyam}
       \rdg[wit={L,B,N2}]{ūrdhvalakṣam}
       \rdg[wit={N1,D}]{urdhvalakṣya}
       \rdg[wit={N2,U1}]{urdhvalakṣa}}/
    2 adho\app{\lem[wit={E,P}, alt={°lakṣyam}]{lakṣyam}
       \rdg[wit={B,L,U2}]{°lakṣam}
       \rdg[wit={N1,D}]{°lakṣya}
       \rdg[wit={N2}]{°lakṣa}
       \rdg[wit={U1}]{\om}}/
    3 \app{\lem[wit={U2}]{bāhyalakṣyam}
       \rdg[wit={N1,D}]{bāhyalakṣya}
       \rdg[wit={N2}]{bāhyalakṣa}
       \rdg[wit={U1}]{bāhyalakya}
       \rdg[wit={B}]{bāhyakṣam}
       \rdg[wit={E}]{lakṣyam}
       \rdg[wit={P}]{madhyalakṣyam}
       \rdg[wit={L}]{madhyalakṣam}}/
\end{prose}
   \end{ekdosis}
   \ekdpb*{}
%%%%%%%%%%%%%%%%%%%%%%%%%%%%%%%%%%%%%%%%%%
%%%%%%%%%%%%%%%%%%%%%%%%%%%%%%%%%%%%%%%%%%
%%%%%%%%PAGEBREAK%%%%%%%PAGEBREAK%%%%%%%%%
%%%%%%%%%%%%%%%%%%%%%%%%%%%%%%%%%%%%%%%%%%
%%%%%%%%%%%%%%%%PAGEBREAK%%%%%%%%%%%%%%%%%
%%%%%%%%%%%%%%%%%%%%%%%%%%%%%%%%%%%%%%%%%%
%%%%%%%%PAGEBREAK%%%%%%%PAGEBREAK%%%%%%%%%
%%%%%%%%%%%%%%%%%%%%%%%%%%%%%%%%%%%%%%%%%%
%%%%%%%%%%%%%%%%%%%%%%%%%%%%%%%%%%%%%%%%%%
%%%%%%%%%%%%%%%%%%%%%%%%%%%%%%%%%%%%%%%%%%
%%%%%%%%%%%%%%%%%%%%%%%%%%%%%%%%%%%%%%%%%%
%%%%%%%%PAGEBREAK%%%%%%%PAGEBREAK%%%%%%%%%
%%%%%%%%%%%%%%%%%%%%%%%%%%%%%%%%%%%%%%%%%%
%%%%%%%%%%%%%%%%PAGEBREAK%%%%%%%%%%%%%%%%%
%%%%%%%%%%%%%%%%%%%%%%%%%%%%%%%%%%%%%%%%%%
%%%%%%%%PAGEBREAK%%%%%%%PAGEBREAK%%%%%%%%%
%%%%%%%%%%%%%%%%%%%%%%%%%%%%%%%%%%%%%%%%%%
%%%%%%%%%%%%%%%%%%%%%%%%%%%%%%%%%%%%%%%%%%
%%%%%%%%%%%%%%%%%%%%%%%%%%%%%%%%%%%%%%%%%%
%%%%%%%%%%%%%%%%%%%%%%%%%%%%%%%%%%%%%%%%%%
%%%%%%%%PAGEBREAK%%%%%%%PAGEBREAK%%%%%%%%%
%%%%%%%%%%%%%%%%%%%%%%%%%%%%%%%%%%%%%%%%%%
%%%%%%%%%%%%%%%%PAGEBREAK%%%%%%%%%%%%%%%%%
%%%%%%%%%%%%%%%%%%%%%%%%%%%%%%%%%%%%%%%%%%
%%%%%%%%PAGEBREAK%%%%%%%PAGEBREAK%%%%%%%%%
%%%%%%%%%%%%%%%%%%%%%%%%%%%%%%%%%%%%%%%%%%
%%%%%%%%%%%%%%%%%%%%%%%%%%%%%%%%%%%%%%%%%%
   \noindent
\begin{ekdosis}
     \ekddiv{type=ed}
  \begin{prose}
    4 \app{\lem[type={emendation}, resp={egoscr}]{madhyalakṣyam}
       \rdg[wit={N1,D}]{\korr madhyalakṣya}
       \rdg[wit={N2,U1}]{madhyalakṣa}
       \rdg[wit={U2}]{madhyalakṣaṃ}
       \rdg[wit={E,P}]{bāhyalakṣyam}
       \rdg[wit={L}]{bāhyakṣam}
       \rdg[wit={B}]{\om}}/
    5 \app{\lem[wit={E,P}]{antaralakṣyam}
       \rdg[wit={N1,D,U1}]{antaralakṣya}
       \rdg[wit={B,L}]{aṃtaralakṣam}
       \rdg[wit={N2}]{antaralakṣa}
       \rdg[wit={U2}]{sarvalakṣyam}}/
   \end{prose}
   \end{ekdosis}
%%%%%%%%%%%%%
%%%%%%%%%%%%%
%%%%%%%%%%%%%
%%%%%%%%%%%%%
%%%%%%%%%%%%%
 \begin{ekdosis}
   \ekddiv{type=ed}
   \bigskip
     \centerline{\textrm{\small{[1. Ūrdhvalakṣya]}}}
     \bigskip
\begin{prose}    
%------------------------------      
%prathamam ūrdhvalakṣyaṃ kathyate/  \E
%prathamam ūrdhvalakṣyaḥ kathyate/  \P
%atha      ūrdhvalakṣaṃ          // \L
%athama    urdhalakṣaṃ           // \B
%prathamaṃ urdhvalakṣaḥ  kathyate/  \N1
%prathamaṃ urdhvalakṣaḥ  kathyate/  \N2
%prathamaṃ urdhvalakṣaḥ  kathyate/  \D
%prathamaṃ urdhvalakṣya/ kathyate/  \U1
%prathamaṃ urdhvalakṣaṃ  kathyate/  \U2
%------------------------------
%At first the upward directed fixation{\textit{adholakṣya} is explained. 
%------------------------------
  \note[type=source, labelb=112, lem={ūrdhvalakṣyaṃ}]{Ysv (YK): lakṣaṇaṃ śrṛṇu caiṣāṃ hi phalaṃ jñātvā maheśvari | ākāśe dṛṣṭim āsthāya mana ūrdhvan [ūrddhan (PT)] tu kārayet ||3||}
  \app{\lem[wit={E,P},alt={prathamam}]{prathama\skp{m-ū}}
       \rdg[wit={N1,N2,D,U1,U2}]{prathamaṃ}
       \rdg[wit={L}]{atha}
       \rdg[wit={B}]{athama}}\app{\lem[wit={E},alt={ūrdhvalakṣyaṃ}]{\skm{m-ū}rdhvalakṣyaṃ}
       \rdg[wit={P}]{ūrdhvalakṣyaḥ}
       \rdg[wit={U1}]{urdhvalakṣya}
       \rdg[wit={L}]{ūrdhvalakṣaṃ}
       \rdg[wit={U2}]{urdhvalakṣaṃ}
       \rdg[wit={N1,N2,D}]{urdhvalakṣaḥ}
       \rdg[wit={B}]{urdhalakṣaṃ}}
     \app{\lem[wit={ceteri}]{kathyate}
       \rdg[wit={L,B}]{\om}}/
%------------------------------     
%ākāśamadhye dṛṣṭiḥ / \E
% \om                 \P
%ākāśamadhye dṛṣṭiḥ / \L
%ākāśamadhye dṛṣṭi    \B
%ākāśamadhye dṛṣṭiḥ / \N1
%ākāśamadhye dṛṣṭiḥ / \N2
%ākāśamadhye dṛṣṭiḥ / \D
%ākāśamadhye dṛṣṭiḥ / \U1
%ākāśamadhye dṛṣṭiḥ / \U2
%------------------------------
%The gaze (\textit{dṛṣṭi)) [should be] in the middle of the sky. 
%------------------------------
  \app{\lem[wit={ceteri}]{ākāśamadhye}
    \rdg[wit={P}]{\om}}
  \app{\lem[wit={ceteri}]{dṛṣṭiḥ}
    \rdg[wit={B}]{dṛṣṭi}
    \rdg[wit={P}]{\om}}/
%------------------------------     
%kadā ca    mana    ūrdhvaṃ      kṛtvā sthāpayati /     \E x
%atha ca    mana    ūrdhvaṃ      kṛtvā sthāpyate /      \P x
%atha vā            ūrdhvaṃ mana kṛtvā sthāpyate        \L
%atha vā            ūrdhvamana   kṛtvā sthāpyate        \B
%atha ca // mana    urdhvaṃ      kṛtvā sthāpyate /      \N1 x
%atha ca mana       ūrdhvaṃ      kṛtvā sthāpyate /      \N2 x
%atha vā mana       ūrdhaṃ       kṛtvā sthāpyate        \D x
%atha ca maner------ddhvaṃ       kṛtvā sthāpyate        \U1
%atha    mana       urdhvaṃ      kṛtvā sthāpyate//      \U2 x
%------------------------------
%And then having caused the mind to be directed upwards, it is caused to be fixed there. 
%------------------------------
  \app{\lem[wit={P,N1,N2,U1}]{atha ca}
    \rdg[wit={L,B,D}]{atha vā}
    \rdg[wit={U2}]{atha}
    \rdg[wit={E}]{kadā ca}}
  \app{\lem[wit={E,P,N2}]{mana ūrdhvaṃ}
    \rdg[wit={N1,U2}]{mana urdhvaṃ}
    \rdg[wit={D}]{mana ūrdhaṃ}
    \rdg[wit={U1}]{manerddhvaṃ}
    \rdg[wit={L}]{ūrdhvaṃ mana}
    \rdg[wit={B}]{ūrdhvamana}}
  kṛtvā
  \app{\lem[wit={ceteri}]{sthāpyate}
    \rdg[wit={E}]{sthāpayati}}/
%------------------------------ 
%etasya lakṣyasya  dṛḍhakaraṇāt   parameśvarasya tejasā saha dṛṣṭer-aikyaṃ  bhavati /  \E
%etasya lakṣyasya  dṛḍhakaraṇāt   parameśvarasya tejasā saha dṛṣṭer-aikyaṃ  bhavati /  \P
%etasya lakṣasya   dṛḍhīkṛtvā//   parameśvarasya teja---saha dṛṣṭair-aikā   bhavati //  \L
%etasya lakṣasya   dṛḍhīkṛtvā//   parameśvarasya teja---saha dṛṣṭair-aikā   bhavati //  \B
%etasya lakṣyasya  dṛḍhīkaraṇāt / parameśvarasya tejasā saha dṛṣteḥ aikyaṃ  bhavati /  \N1
%etasya lakṣaṇasya dṛḍhīkaraṇāt   parameśvarasya tejasā saha dṛṣteḥ ekaṃ    bhavati //  \N2
%etasya lakṣasya   dṛḍhīkaraṇāt// parameśvarasya tejasā saha dṛṣṭeḥ aikyaṃ  bhavati // \D
%etasya lakṣasya   dṛḍhīkaraṇāt/  parameśvarasya tejasā saha dṛṣṭer-aikyaṃ  bhavati/ \U1
%etasya lakṣasya   dṛḍhīkaraṇāt   parameśvarasya tenasā saha dṛṣṭer-aikyaṃ  bhavati // \U2
%------------------------------
%Due to the exercise of stabilizing of this fixation (\textit{lakṣya}) arises unity of the gazing point (\textit{dṛṣṭi}) with the light of the highest lord (\textit{parameśvara}). 
%------------------------------
\note[type=testium, labelb=113, lem={parameśvarasya}]{Ysv (YK): ūrdhvalakṣaṃ [ūrdha° (PT)] bhaved eṣā parameśasya caikatā |}
  etasya
  \app{\lem[wit={E,P,N1}]{lakṣyasya}
    \rdg[wit={ceteri}]{lakṣasya}
    \rdg[wit={N2}]{lakṣaṇasya}}
  \app{\lem[wit={ceteri}]{dṛḍhīkaraṇāt}
    \rdg[wit={E,P}]{dṛḍhakaraṇāt}
    \rdg[wit={L,B}]{dṛḍhīkṛtvā}}
  parameśvarasya
\app{\lem[wit={ceteri}]{tejasā}
  \rdg[wit={U2}]{tenasā}
  \rdg[wit={L,B}]{teja°}}
saha
\app{\lem[wit={E,P,U1,U2}]{dṛṣṭer-aikyaṃ}
  \rdg[wit={N1,D}]{dṛṣṭeḥ aikyaṃ}
  \rdg[wit={N2}]{dṛṣteḥ ekaṃ}
  \rdg[wit={L,B}]{ dṛṣṭair aikā}}
bhavati/  
%------------------------------
%atha cākāśa----madhye    yaḥ kaścidadṛṣṭaḥ   padārtho bhavati /  \E x
%atha cākāśa----madhye    yaḥ kaścidadṛṣṭaḥ   padārtho bhavati /  \P  x
%atha vākāśa----madhye    yaḥ kacciddṛṣṭaḥ    padārtho bhavati    \L  x
%athā cākāśa----madhye    yaḥ kaccit dṛṣṭaḥ   padārtho bhavati    \B   x
%atha ca ākāśa--madhye    yaḥ kaścitadṛṣtaḥ   padārthe bhavati /  \N1   x
%atha// ākāśa---madhye    yaḥ kaścita adṛṣtaḥ padārtha bhavati /  \N2  x 
%atha ca ākāśa--madhye    yaḥ kaścitadṛsṭaḥ   padārtho bhavati /  \D    x
%atha ca/ ākāśa-madhye    yaḥ kaścidadṛsṭaḥ   padārtho bhavati    \U1    x
%atha cākāśa----madhye    yaḥ kaściddṛsṭa-----padārtho bhavati /  \U2
%------------------------------
%And then an indefinable invisible object arises in the middle of the sky.
%------------------------------
\app{\lem[wit={ceteri}]{atha}
  \rdg[wit={B}]{athā}}
\app{\lem[wit={E,P,B,U2},alt={cākāśa°}]{cākāśa}
  \rdg[wit={N1,D,U1}]{ca ākāśa°}
  \rdg[wit={L}]{vākāśa°}
  \rdg[wit={N2}]{ākāśa°}}madhye
yaḥ
\app{\lem[wit={ceteri},alt={kaścid adṛṣṭaḥ}]{kaścid\skp{-}adṛṣṭaḥ}
  \rdg[wit={L}]{kaccid dṛṣṭaḥ}
  \rdg[wit={B}]{kaccit dṛṣṭaḥ}
  \rdg[wit={N2}]{kaścita adṛṣtaḥ}
  \rdg[wit={U2}]{kaścid dṛsṭa°}}
\app{\lem[wit={ceteri}]{padārtho}
  \rdg[wit={N1}]{padārthe}
  \rdg[wit={N2}]{padārtha}}
bhavati/ 
%------------------------------
%sa sādhakasya dṛṣṭigocaro bhavati//  \E
%sa sādhakasya dṛṣṭigocaro bhavati//  \P
%   sādhakasya dṛṣṭigocaro bhavati//  \L
%   sādhakasya dṛṣṭigocaro bhavatī    \B
%sa sādhakasya dṛṣṭigocare bhavati // \D  saḥ-Sonderregel -> ḥ fällt aus vor allen Konsonanten
%sa sādhakasya dṛṣṭigocare bhavati // \N1
%   sādhakasya dṛṣṭigocarā bhavati // \N2
%sa sādhakasya dṛṣṭigocaro bhavati    \U1
%   sādhakasya dṛṣṭigocare bhavati // \U2
%------------------------------
%It arises in the range of sight of the practitioner.  
%------------------------------
\app{\lem[wit={ceteri}]{sa}
  \rdg[wit={L,B,N2,U2}]{\om}}
sādhakasya
\app{\lem[wit={D,N1,U2}]{dṛṣṭigocare}
  \rdg[wit={ceteri}]{dṛṣṭigocaro}
  \rdg[wit={N2}]{dṛṣṭigocarā}}
\app{\lem[wit={ceteri}]{bhavati}
  \rdg[wit={B}]{bhavatī}}/
%------------------------------
%ayam evordhvalakṣyaḥ      \E
%ayam evordhvalakṣyaḥ      \P
%ayam evordhvalakṣaḥ  //   \L
%ayam evordhalakṣaḥ  //    \B
%ayam evordhvalakṣya  //   \N1
%ayam eva vodhalakṣaṇam // \N2
%ayam evordhvalakṣyaḥ //   \D
%ayam evordhvalakṣyaḥ      \U1
%ayam evordhvalakṣya //    \U2
%------------------------------
%This is truly the upward directed fixation (\textit{ūrdhvalakṣya}).
%------------------------------
aya\skp{m-e}\app{\lem[wit={E,P,D,U1},alt={evordhvalakṣyaḥ}]{\skm{m-e}vordhvalakṣayaḥ}
  \rdg[wit={L}]{°lakṣaḥ}
  \rdg[wit={B}]{evordhalakṣaḥ}
  \rdg[wit={N1,U2}]{°lakṣya}
  \rdg[wit={N2}]{eva vodhalakṣaṇam}}/
\end{prose}
\end{ekdosis}
%%%%%%%%%%%%%%%%
%%%%%%%%%%%%%%%
%%%%%%%%%%%%%%%%
%%%%%%%%%%%%%%%%
%%%%%%%%%%%%%%%%%
\begin{ekdosis}
  \ekddiv{type=ed}
   \bigskip
    \centerline{\textrm{\small{[2. Adholakṣya]}}}
    \bigskip
 \begin{prose}
%------------------------------
%                            nāsikāyāḥ  upari     dvādaśāṃgulamūlaparyantaṃ dṛṣṭiḥ sthirā karttavyā /   \E
%       athādholakṣaḥ        nāsikāyā   upari     dvādaśāṃgulaparyantaṃ     dṛṣṭiḥ sthirā karttavyā /   \P
%       athādholakṣaḥ //     nāsikāyā   upari     dvādaśāṃgulaparyaṃtaṃ     dṛṣṭiḥ sthirā karttavyā     \L
%       athādholakṣa //      nāsikāyā   upari     dvādaśāṃgulaparyaṃtaṃ     dṛṣṭiḥ sthirā karttavyā     \B
%       atha adholakṣyaḥ //  nāsikāyā   upari     dvādaśaṃgulaparyaṃtaṃ     dṛṣṭiḥ sthirā karttavyā //  \N1
%       atha adholakṣanaḥ // nāsikāyā   upari     dvādaśāṃgulaparyaṃtaṃ     dṛṣṭiḥ sthirā karttavyā //  \N2
%       atha adholakṣaḥ //   nāsikāyā   upari     dvādaśaṃgulaparyaṃtaṃ     dṛṣṭiḥ sthirā karttavyā //  \D
%       atha adholakṣa       nāsikāyā   upari     dvādaśaṃgulaparyaṃtaṃ     dṛṣṭi--sthirā karttavyā     \U1
%                            nāsikāyāḥ  upariṣṭāt    daśāṃgulaparyaṃtaṃ     dṛṣṭiḥ sthirā karttavyā //  \U2
%------------------------------
%Now the downward directed fixation object (\textit{adholakṣya}). One should stabilize the gaze within the circumference (\textit{paryanta}) of twelve \textit{aṅgula}s beyond the nose.
%------------------------------
   \note[type=source, labelb=114, lem={athādholakṣyaḥ}]{Ysv (YK): nāsikopari deveśi dvādaśāṅgulamānataḥ ||4|| dṛṣṭisthiran [dṛṣṭiḥ sthirā (PT)] tu karttavyam [karttavyā (PT)] adholakṣam idaṃ bhajet [bhaja (PT)] | tathā ca [athavā (PT)] nāsikāgre tu sthirā dṛṣṭir iyaṃ śṛṇu [bhavet (PT)] ||5|| yasya bhavet sthirā dṛṣṭiś cirāyuḥ [sthirā dṛṣṭiś cirāyuḥ syāt tathāsau (PT)] sthiradṛṣṭimān|}   
\app{\lem[type=emendation, resp=egoscr]{athādholakṣyaḥ}
  \rdg[wit={N1}]{\korr atha adholakṣyaḥ}
  \rdg[wit={P,L}]{athādholakṣaḥ}
  \rdg[wit={B}]{athādholakṣa}
  \rdg[wit={N2}]{atha adholakṣanaḥ}
  \rdg[wit={D}]{atha adholakṣaḥ}
  \rdg[wit={U1}]{atha adholakṣa}
  \rdg[wit={E,U2}]{\om}}/
\app{\lem[wit={ceteri}]{nāsikāyā}
  \rdg[wit={E,U2}]{nāsikāyāḥ}}
\app{\lem[wit={ceteri}]{upari}
  \rdg[wit={U2}]{upariṣṭāt}}
\app{\lem[wit={ceteri}]{dvādaśāṃgulaparyantaṃ}
  \rdg[wit={E}]{dvādaśāṃgulamūlaparyantaṃ}
  \rdg[wit={U2}]{daśāṃgulaparyaṃtaṃ}}
\app{\lem[wit={ceteri}]{dṛṣṭiḥ}
  \rdg[wit={U1}]{dṛṣṭi°}}
sthirā karttavyā/
%------------------------------
%atha vā nāsikāyā agre dṛṣṭiḥ sthirā karttavyā / \E
%atha vā nāsikāyā agre dṛṣṭiḥ sthirā karttavyā / \P
%\om / \L
%\om / \B
%atha vā nāsikāyā  agre dṛṣṭiḥ sthirā karttavyā // \N1
%atha vā nāsikā    agre dṛṣṭi-sthirā karttavyā      \N2
%atha vā nāsikāyā  agre dṛṣṭiḥ sthirā karttavyā // \D
%atha vā nāśikāyāḥ/ agre dṛṣṭiḥ/ sthirā karttavyā / \U1
%atha vā nāsikāyā  agre dṛṣṭiḥ sthirā karttavyā // \U2
%------------------------------
%Or one should stabilize the gaze onto the tip of the nose.
%------------------------------
\app{\lem[wit={ceteri}]{atha vā}
  \rdg[wit={L,B}]{\om}}
\app{\lem[wit={ceteri}]{nāsikāyā}
  \rdg[wit={U1}]{nāsikāyāḥ}
  \rdg[wit={N2}]{nāsika}}
\app{\lem[wit={ceteri}]{agre}
  \rdg[wit={L,B}]{\om}}
\app{\lem[wit={ceteri}]{dṛṣṭiḥ}
  \rdg[wit={N2}]{dṛṣṭi°}}
\app{\lem[wit={ceteri}]{sthirā}
  \rdg[wit={L,B}]{\om}}
\app{\lem[wit={ceteri}]{karttavyā}
  \rdg[wit={L,B}]{\om}}/ 
%------------------------------
%lakṣadūyasya  dṛḍhīkaraṇāt / dṛṣṭiḥ sthirā bhavati / \E
%lakṣadvayasya dṛṣṭīkaraṇāt / dṛṣṭiḥ sthirā bhavati / \P
%lakṣadvayasya dṛḍhīkaraṇāt   dṛṣṭi--sthiro bhavati / \L
%lakṣadvayasya dṛḍhīkaraṇān---dṛṣṭiḥ sthiro bhavatī   \B
%lakṣadvayasya dṛdhīkaraṇāt   dṛṣṭiḥ sthirā bhavati / \N1
%lakṣadvayasya dṛḍhīkaraṇād---dṛṣṭi--sthirā bhavati / \N2
%lakṣadvayasya dṛḍhīkaraṇāt   dṛṣṭiḥ sthirā bhavati / \D
%lakṣadvayasya dṛḍhīkaraṇāt   dṛṣṭiḥ sthirā bhavati / \U1
%lakṣadvayasya dṛḍhīkaraṇāt   dṛṣṭi--sthirā bhavati // \U2
%------------------------------
%The fixation becomes stable due to firm exercise [on one] of the twofold aims [of fixation]. 
%------------------------------
\app{\lem[wit={ceteri}]{lakṣadvayasya}   %emend to lakṣyadvayasya??? 
  \rdg[wit={E}]{lakṣadūyasya}} 
\app{\lem[wit={N2}, alt={dṛḍhīkaraṇād}]{dṛḍhīkaraṇā\skm{d-ṛ}}
  \rdg[wit={E,L,N1,D,U1,U2}]{dṛḍhīkaraṇāt}
  \rdg[wit={P}]{dṛṣṭīkaraṇāt}
  \rdg[wit={B}]{dṛḍhīkaraṇān}
}\app{\lem[wit={ceteri}, alt={dṛṣṭiḥ}]{\skp{d-ṛ}ṣṭiḥ}
  \rdg[wit={L,N2,U2}]{dṛṣṭi°}}
\app{\lem[wit={ceteri}]{sthirā}  
  \rdg[wit={B}]{sthiro}
  \rdg[wit={L}]{°sthiro}}
\app{\lem[wit={ceteri}]{bhavati}
  \rdg[wit={B}]{bhavatī}}/
%------------------------------
%pavanaḥ sthiro bhavati / \E
%pavanaḥ sthiro bhavati / \P
%\om                    / \L
%\om                    / \B
%pavanaḥ sthiro bhavati / \N1
%pavana--sthiro bhavati /   \N2
%pavanaḥ sthiro bhavati / \D
%pavana--sthiro bhavati  / \U1
%pavana--sthiro bhavati  / \U2
%------------------------------
%The breath becomes stable. 
%------------------------------
\app{\lem[wit={E,P,N1,D}]{pavanaḥ}
  \rdg[wit={N2,U1,U2}]{pavana°}
  \rdg[wit={L,B}]{\om}}
\app{\lem[wit={ceteri}]{sthiro}
  \rdg[wit={L,B}]{\om}}
\app{\lem[wit={ceteri}]{bhavati}
  \rdg[wit={L,B}]{\om}}/
%------------------------------
%āyurvarddhate / \E
%āyurvarddhate / \P
%āyurvarddhate / \L
%āyurvardhate /  \B
%āyurvardhate /  \N1
%āyurvardhate /  \N2
%āyurvardhate /  \D
%āyurvarddhate   \U1
%āyurvarddhate //  \U2
%------------------------------
%Vitality increases. 
%------------------------------
āyur-varddhate/
%%%%%%%%%%%%
%%%%%%%%%%%%
%%%%%%%%%%%%
%%%%%%%%%%%%
%%%%%%%%%%%%
\end{prose}
\end{ekdosis}
\ekdpb*{}
%%%%%%%%%%%%%%%%%%%%%%%%%%%%%%%%%%%%%%%%%%
%%%%%%%%%%%%%%%%%%%%%%%%%%%%%%%%%%%%%%%%%%
%%%%%%%%PAGEBREAK%%%%%%%PAGEBREAK%%%%%%%%%
%%%%%%%%%%%%%%%%%%%%%%%%%%%%%%%%%%%%%%%%%%
%%%%%%%%%%%%%%%%PAGEBREAK%%%%%%%%%%%%%%%%%
%%%%%%%%%%%%%%%%%%%%%%%%%%%%%%%%%%%%%%%%%%
%%%%%%%%PAGEBREAK%%%%%%%PAGEBREAK%%%%%%%%%
%%%%%%%%%%%%%%%%%%%%%%%%%%%%%%%%%%%%%%%%%%
%%%%%%%%%%%%%%%%%%%%%%%%%%%%%%%%%%%%%%%%%%
%%%%%%%%%%%%%%%%%%%%%%%%%%%%%%%%%%%%%%%%%%
%%%%%%%%%%%%%%%%%%%%%%%%%%%%%%%%%%%%%%%%%%
%%%%%%%%PAGEBREAK%%%%%%%PAGEBREAK%%%%%%%%%
%%%%%%%%%%%%%%%%%%%%%%%%%%%%%%%%%%%%%%%%%%
%%%%%%%%%%%%%%%%PAGEBREAK%%%%%%%%%%%%%%%%%
%%%%%%%%%%%%%%%%%%%%%%%%%%%%%%%%%%%%%%%%%%
%%%%%%%%PAGEBREAK%%%%%%%PAGEBREAK%%%%%%%%%
%%%%%%%%%%%%%%%%%%%%%%%%%%%%%%%%%%%%%%%%%%
%%%%%%%%%%%%%%%%%%%%%%%%%%%%%%%%%%%%%%%%%%
%%%%%%%%%%%%%%%%%%%%%%%%%%%%%%%%%%%%%%%%%%
%%%%%%%%%%%%%%%%%%%%%%%%%%%%%%%%%%%%%%%%%%
%%%%%%%%PAGEBREAK%%%%%%%PAGEBREAK%%%%%%%%%
%%%%%%%%%%%%%%%%%%%%%%%%%%%%%%%%%%%%%%%%%%
%%%%%%%%%%%%%%%%PAGEBREAK%%%%%%%%%%%%%%%%%
%%%%%%%%%%%%%%%%%%%%%%%%%%%%%%%%%%%%%%%%%%
%%%%%%%%PAGEBREAK%%%%%%%PAGEBREAK%%%%%%%%%
%%%%%%%%%%%%%%%%%%%%%%%%%%%%%%%%%%%%%%%%%%
%%%%%%%%%%%%%%%%%%%%%%%%%%%%%%%%%%%%%%%%%%
\begin{ekdosis}
  \ekddiv{type=ed}
    \centerline{\textrm{\small{[3. Bāhyalakṣya]}}}
    \bigskip
 \begin{prose}
\noindent
%------------------------------
%etad dūyam       api bāhyalakṣyam eva  bhavati      bāhyāṃtara       ākāśe         śūnyalakṣyaṃ    karttavyaḥ / \E
%etad dvayam      api bāhyalakṣyam eva  bhavati      bāhyābhyaṃtare   ākāśe cet     śūnyalakṣyaṃ    karttavyaḥ / \P
%etad dvayam      api bāhyalakṣam  eva  bhavati//    bāhyābhyaṃtare   ākāśacen      śūnyaṃ lakṣaṃ   karttavyā // \L
%etad dvayadvayam api bāhyalakṣam  eva  bhavatī//    bāhyābhyaṃtare   ākāśacvat     śūnyaṃ lakṣaṃ   karttavyā // \B
%etat advayam     eva bāhyalakṣam  api  kathyate //  bāhyo bhyaṃtaraṃ ākāśavat------śūnyalakṣyaḥ    karttavyaḥ / \N1
%etad dvayam      eva bāhyalakṣam  api  kathyate //  bāhyābhyaṃtaram--ākāśavat------śūnyalakṣaḥ     karttavyaḥ   \N2
%etat advayam     eva bāhyalakṣam  api  kathyate //  bāhyo bhyaṃtaraṃ ākāśavat //   śūnyalakṣyaḥ    karttavyaḥ / \D
%etat dvayam      eva bāhyalakṣyam api  kathyate/    bāhyābhyaṃtare   ākāśavat------śūnyalakṣyaḥ    karttavyaḥ  \U1
%etat dvayam      api bāhyalakṣyam eva  bhavati//    bāhyābhyaṃtare   ākāśe cet     śūnyalakṣyaṃ    karttavyaḥ / \U2
%------------------------------
%Just as this [aim] is twofold, also the external fixation is said to be [like this]. Internally or externally the aim of fixation is to be done onto the heavenly void.  
%------------------------------
\note[type=source, labelb=115, lem={bāhyalakṣyam}]{Ysv (YK): bāhyalakṣaṃ [vāhya° (PT)] svayaṃ jñeyaṃ yāti tattvanirāsinām [nivāsinām (PT)] ||6|| kāmināṃ tu bahir dṛṣṭiś cintādiṣu susiddhidā | etad bāhyamadhyalakṣaṃ dṛṣṭicintānirākulaḥ [iṣṭacintā nirākulam (PT)] ||7||}
\app{\lem[wit={P,L,N2},alt={etad dvayam}]{etad-dvaya\skp{m-e}}
  \rdg[wit={E}]{etad dūyam}
  \rdg[wit={B}]{etad dvayadvaya}
  \rdg[wit={N2,D}]{etat advayam}
  \rdg[wit={U1,U2}]{etat dvayam}}\app{\lem[wit={N1,N2,D,U1}, alt={eva}]{\skm{m-e}va}
  \rdg[wit={ceteri}]{api}} 
\app{\lem[wit={E,P,U1,U2},alt={bāhyalakṣyam}]{bāhyalakṣya\skp{m-a}}
  \rdg[wit={ceteri}]{°lakṣam}}\app{\lem[wit={N1,N2,D,U1},alt={api}]{\skm{m-a}pi}
  \rdg[wit={E,P,L,B,U2}]{eva}}
\app{\lem[wit={N1,N2,D,U1}]{kathyate}
  \rdg[wit={E,P,L,U2}]{bhavati}
  \rdg[wit={B}]{bhavatī}}/
\app{\lem[wit={N2},alt={bāhyābhyantaram}]{bāhyābhyantara\skm{m-ā}}                %Übersetzung nochmal überdenken! 
  \rdg[wit={N1,D}]{bāhyo bhyaṃtaraṃ}
  \rdg[wit={P,L,B,U1,U2}]{bāhyābhyaṃtare}
  \rdg[wit={E}]{bāhyāṃtara}}\app{\lem[wit={N1,N2,D,U1},alt={ākāśavat}]{\skp{m-ā}kāśavat}
  \rdg[wit={B}]{ākāśacvat}
  \rdg[wit={L}]{ākāśacen}
  \rdg[wit={P,U2}]{ākāśe cet}
  \rdg[wit={E}]{ākāśe}}
\app{\lem[wit={N1,D,U1}]{śūnyalakṣyaḥ}
  \rdg[wit={E,P,U2}]{śūnyalakṣyaṃ}
  \rdg[wit={N2}]{śūnyalakṣaḥ}
  \rdg[wit={L,B}]{śūnyaṃ lakṣaṃ}}
\app{\lem[wit={ceteri}]{karttavyaḥ}
  \rdg[wit={L,B}]{karttavyā}}/
%------------------------------
%jāgraddaśāyāṃ    calanadaśāyāṃ   bhojanadaśāyāṃ   sthitikāle sarvasthāne   śūnyasya dhyānakāraṇāt //                              \E
%jāgraddaśāyāṃ    calanadaśāyāṃ   bhojanaṃ daśāyāṃ sthitikāle sarvasthāne   śūnyasya dhyānakāraṇāt //                              \P
%jāgradādidaśāyāṃ calanadaśāyāṃ// bhojanadaśāyāṃ   sthitikāle sarvasthāneṣu śūnyasya dhyānakāraṇāt //                              \L
%jāgradādidaśāyāṃ calanadaśāyāṃ// bhojanadaśāyāṃ   sthitikāle sarvasthāneṣu śūnyasya dhyānakaraṇāt //                              \B
%jāgraddaśāyāṃ    cakabadaśāyāṃ   bhojanadaśāyāṃ   sthitikāle sarvvasthāne  śūnyasya dhyānakaraṇāt  maraṇatrāso na bhavati//       \N1
%jāyadaśāyāṃ      calanadaśāyāṃ/  bhojanadaśāyāṃ   sthitikāle sarvasthāne   śūnyasya dhyānakaraṇāt  maraṇatrāśo na bhavati//       \N2
%jāgraddaśāyāṃ    calanadaśāyāṃ   bhojanadaśāyāṃ   sthitikāle sarvvasthāne  śūnyasya dhyānakaraṇāt  maraṇatrāso na bhavati// śūnya \D
%jāgraddaśāyāṃ    calanadaśāyāṃ                    sthitikāle sarvasthāne   śūnyasya dhyānakaraṇāt/ maraṇasautrāṃ na bhavati vā    \U1
%jāgṛaddaśāyāṃ    calanadaśāyāṃ   bhojanadaśāyāṃ   sthitikāle sarvasthāne   śūnyasya dhyānakaraṇāt//                               \U2
%------------------------------
%The fear of dying does not arise due to the exercise of meditation on the void at all places during ones life - while eating, moving and waking. 
%------------------------------
\app{\lem[wit={ceteri}]{jāgraddaśāyāṃ}
    \rdg[wit={N2}]{jāgṛaddaśāyāṃ}
    \rdg[wit={N2}]{jāyadaśāyāṃ}
    \rdg[wit={L,B}]{jāgradādidaśāyāṃ}}
\app{\lem[wit={ceteri}]{calanadaśāyāṃ}
    \rdg[wit={N1}]{cakabadaśāyāṃ}}
\app{\lem[wit={ceteri}]{bhojanadaśāyāṃ}
    \rdg[wit={P}]{bhojanaṃ daśāyāṃ}
    \rdg[wit={U1}]{\om}}
  sthitikāle
\app{\lem[wit={ceteri}]{sarvasthāne}
    \rdg[wit={L,B}]{sarvasthāneṣu}}
  śūnyasya dhyānakāraṇāt
\app{\lem[wit={N1,D}]{maraṇatrāso}
    \rdg[wit={N2}]{maraṇatrāśo}
    \rdg[wit={U1}]{maraṇasautrāṃ}
    \rdg[wit={E,P,L,B,U2}]{\om}}
\app{\lem[wit={ceteri}]{na}
    \rdg[wit={E,P,B,U2}]{\om}}
\app{\lem[wit={N1,N2}]{bhavati}
    \rdg[wit={D}]{bhavati || śūnya}
    \rdg[wit={U1}]{bhavati vā}
    \rdg[wit={ceteri}]{\om}}\dd{}
 \end{prose}
\end{ekdosis}
%%%%%%%%%%%%%%%%%
%%%%%%%%%%%%%%%
%%%%%%%%%%%%%%%%
%%%%%%%%%%%%%%%%
%%%%%%%%%%%%%%%%%
\begin{ekdosis}
  \ekddiv{type=ed}
       \bigskip
    \centerline{\textrm{\small{[The Rājayogin's Body]}}}
    \bigskip
    \begin{prose}
%------------------------------  
%idānīṃ rājayogayuktasya           śarīre yaccihnaṃ  tat    kathyate / \E
%idānīṃ rājayogayuktasya puruṣasya yaccharīracihnaṃ         kathyate / \P
%idānīṃ rājayogayuktasya puruṣasya          cinhnaṃ         kathyate / \L
%idānīṃ rājayogayuktasya puruṣasya          cinhnaṃ         kathyate // \B
%idānīṃ rājayogayuktasya puruṣasya yaccarīracihnaṃ   tat    kathyate / \N1
%idānīṃ rājayogayuktasya puruṣasya yaccharīracihūṃ   tat    kathyate// \N2
%idānīṃ rājayogayuktasya puruṣasya yaccarīracihnaṃ   tat    kathyate / \D
%idānīṃ rājayogayuktasya puruṣasya yaccharīre cinhaṃ tata   kathyate \U1
%idānīṃ rājayogayuktasya puruṣasya yat śarīracinhaṃ         kathyate / \U2
%------------------------------
%Now it is said that this is the characteristic of the embodied person who is endowed with the royal yoga:
%------------------------------
\note[type=source, labelb=116, lem={rājayoga°}]{Ysv (PT): idānīṃ kathayiṣyāmi rājayogasya lakṣaṇam | rājayoge kṛte puṃbhiḥ siddhicihnaṃ bhavediti |}
      idānīṃ rājayogayuktasya
  \app{\lem[wit={ceteri}]{puruṣasya}
    \rdg[wit={E}]{\om}}
  \app{\lem[wit={N1,D,P},alt={yac carīracihnaṃ}]{yac-carīracihnaṃ}
    \rdg[wit={U2}]{yat śarīracinhaṃ}
    \rdg[wit={E}]{śarīre yac cihnaṃ}
    \rdg[wit={U1}]{yac charīre cinhaṃ}
    \rdg[wit={N2}]{yac charīracihūṃ}
    \rdg[wit={L,B}]{cinhnaṃ}}
  \app{\lem[wit={E,N1,N2,D}]{tat}
    \rdg[wit={U1}]{tata}
    \rdg[wit={ceteri}]{\om}}
  kathyate/
%------------------------------  
%tatsarvatra pūrṇo bhavati / \E
%tatsarvatra pūrṇā bhavati / \P
%tatsarvatra pūrṇo bhavati / \L
%tatsarvatra pūrṇo bhavatī / \B
%  sarvvatra pūrṇo bhavati / \N1
%  sarvvatra pūrṇā bhavati  \N2
%  sarvvatra pūrṇo bhavati  \D
%  sarvvatra pūrṇo bhavati   \U1
%tatsarvatra pūrṇo bhavati// \U2
%------------------------------
%Abundance arises at all times. %Alternative=permanent Abundance arises because of that.   
%------------------------------
\note[type=source, labelb=117, lem={pūrṇo}]{Ysv (PT): paripūrṇaṃ bhavec cittaṃ jagatstho 'pi jagadbahiḥ |}
  \app{\lem[wit={N1,N2,D,U1},alt={sarvatra°}]{sarvatra}
  \rdg[wit={ceteri}]{tatsarvatra°}}
\app{\lem[wit={ceteri}, alt={°pūrṇo}]{pūrṇo}
  \rdg[wit={P,N2}]{pūrṇā}}
\app{\lem[wit={ceteri}]{bhavati}
  \rdg[wit={B}]{bhavatī}}/
%------------------------------  
%pṛthivyāḥ dūre tiṣṭhati / \E
%pṛthivyāḥ hare tiṣṭhati / \P
%\om                      \L
%\om                      \B
%pṛthivyāḥ dūre  tiṣṭhati / \N1
%pṛthivyāḥ dūra  tiṣṭhati / \N2
%pṛthivyāḥ dūre  tiṣṭhati / \D
%pṛthivyāḥ ddūre tiṣṭhati / \U1 %emend to na tiṣṭhati? 
%pṛthivyā dūraṃ  tiṣṭhati // \U2 !!dūraṃ
%------------------------------
%No distances exist on earth.
%------------------------------
\app{\lem[type=conjecture, resp=egoscr]{pṛthivyāṃ}
  \rdg[wit={ceteri}]{\conj pṛthivyāḥ}
  \rdg[wit={U2}]{pṛthivyā}
  \rdg[wit={L,B}]{\om}} 
\app{\lem[wit={U2}]{dūraṃ}
  \rdg[wit={E,N1,D}]{dūre}
  \rdg[wit={U1}]{ddūre}
  \rdg[wit={N2}]{dūra}
  \rdg[wit={L,B}]{\om}}
\app{\lem[type=conjecture, resp=egoscr]{na tiṣṭhati}
  \rdg[wit={ceteri}]{\conj tiṣṭhati}
  \rdg[wit={L,B}]{\om}}/
%------------------------------
%pṛthivyāṃ vyāpya tiṣṭhati / \E
%pṛthi-----vyāpya tiṣṭhati / \P
%\om                         \L
%\om                         \B
%pṛthvāṃ vyāpya   tiṣṭhati /   \N1
%pṛthvīṃ vyāpya   tiṣṭhati /   \N2
%pṛthvīṃ vyāpya   tiṣṭhati /   \D  %geht auch pṛthu für Erde? 
%\om   \U1
%pṛthivyā vyāti   tiṣṭhati     \U2
%------------------------------
%He dwells on earth having pervaded [it]. 
%------------------------------
\app{\lem[type=emendation, resp=egoscr]{pṛthivīṃ}
  \rdg[wit={E}]{pṛthivyāṃ}
  \rdg[wit={P}]{pṛthi°}
  \rdg[wit={N1}]{pṛthvāṃ}
  \rdg[wit={N2,D}]{pṛthvīṃ}
  \rdg[wit={U2}]{pṛthivyā}
  \rdg[wit={L,B,U2}]{\om}}
\app{\lem[wit={ceteri}]{vyāpya}
  \rdg[wit={U2}]{vyāti}
  \rdg[wit={L,B,U1}]{\om}} 
\app{\lem[wit={ceteri}]{tiṣṭhati}
  \rdg[wit={L,B,U2}]{\om}}/
%------------------------------
% yasya janmamaraṇe  na staḥ sukhaṃ na bhavati /  \E
% yasya janmamaraṇe  na staḥ sukhaṃ na bhavati /  \P
% \om                                            \L
% \om                                            \B
% yasya janmamaraṇe  na staḥ sukhaṃ na bhavati /  \N1
% yasya janmamaraṇe  na staḥ sukhaṃ na bhavati /  \N2
% yasya janmamaraṇe  na staḥ sukhaṃ na bhavati /  \D
% \om                                            \U1
% yasya jananamaraṇe na staḥ sukhaṃ na bhavati /  \U2 maraṇe nom/acc dual! staḥ von as 3. dual 
%------------------------------
% Birth and death both do not exist. Happiness does not exist. 
% ------------------------------
\note[type=source, labelb=118, lem={janma°}]{Ysv (PT): na kṣobho janma mṛtyuś ca na duḥkhaṃ na sukhaṃ tathā |}
\app{\lem[wit={ceteri}]{yasya}
  \rdg[wit={L,B,U1}]{\om}}
\app{\lem[wit={ceteri}]{janmamaraṇe}
  \rdg[wit={U2}]{jananamaraṇe}
  \rdg[wit={L,B,U1}]{\om}}
\app{\lem[wit={ceteri}]{na}
  \rdg[wit={L,B,U1}]{\om}}
\app{\lem[wit={ceteri}]{staḥ}
  \rdg[wit={L,B,U1}]{\om}}/
\app{\lem[wit={ceteri}]{sukhaṃ}
  \rdg[wit={L,B,U1}]{\om}}
\app{\lem[wit={ceteri}]{na}
  \rdg[wit={L,B,U1}]{\om}}
\app{\lem[wit={ceteri}]{bhavati}
  \rdg[wit={L,B,U1}]{\om}}/
% ------------------------------
% \om                 \E
% \om                 \P
% \om                 \L
% \om                  \B
% duḥkhaṃ na bhavati / \N1
% duḥkhaṃ na bhavati / \N2
% duḥkham na bhavati / \D
% \om                  \U1
% \om                  \U2
% ------------------------------
%Suffering does not exist. 
%------------------------------
\app{\lem[wit={N1,N2,D}]{duḥkhaṃ}
  \rdg[wit={ceteri}]{\om}} 
\app{\lem[wit={N1,N2,D}]{na}
  \rdg[wit={ceteri}]{\om}} 
\app{\lem[wit={N1,N2,D}]{bhavati}
  \rdg[wit={ceteri}]{\om}}/
%------------------------------
% \om               \E
% kalaṃ na bhavati  \L
% kulaṃ na bhavatī// \B
% kūlaṃ na bhavati / \P
% kūlaṃ na bhavati / \N1
% kūlaṃ na bhavati / \N2
% kūlaṃ na bhavati / \D
% \om               \U1
% kulaṃ na bhavatī// \U2
%------------------------------
%Impediment does not exist.
%------------------------------
\note[type=source, labelb=119, lem={kūlaṃ}]{bhedābhedau manaḥsthau na jñānaṃ śīlaṃ kulaṃ tathā |}
\app{\lem[wit={P,N1,N2,D}]{kūlaṃ}
  \rdg[wit={B,U2}]{kulaṃ}
  \rdg[wit={L}]{kalaṃ}
  \rdg[wit={E,U1}]{\om}}
\app{\lem[wit={ceteri}]{na}
  \rdg[wit={E,U1}]{\om}}
\app{\lem[wit={ceteri}]{bhavati}
  \rdg[wit={B,U2}]{bhavatī}
  \rdg[wit={E,U1}]{\om}}/
%------------------------------
% \om                  \E
% śītalaṃ na bhavati / \P
% \om                  \L
% \om                  \B
% śīlaṃ na bhavati /   \N1
% śīlaṃ na bhavati /   \N2
% śīlaṃ na bhavati /   \D
% śīlaṃ na bhavati /   \U1
% śīlaṃ na bhavati /   \U2
%------------------------------
% Habit doesn't exist. 
% ------------------------------
\app{\lem[wit={ceteri}]{śīlaṃ}
  \rdg[wit={P}]{śītalaṃ}
  \rdg[wit={E,L,B}]{\om}}
\app{\lem[wit={ceteri}]{na}
  \rdg[wit={E,L,B}]{\om}}
\app{\lem[wit={ceteri}]{bhavati}
  \rdg[wit={E,L,B}]{\om}}/
%------------------------------
% \om                 \E
% sthānaṃ na bhavati / \P
% \om                  \L
% \om                  \B
% sthānaṃ na bhavati / \N1
% sthānaṃ na bhavati / \N2
% sthānaṃ na bhavati / \D
% sthānaṃ na bhavati / \U1
% sthānaṃ na bhavati / \U2
%------------------------------
% Place does not exist. 
%------------------------------
\app{\lem[wit={ceteri}]{sthānaṃ}
  \rdg[wit={E,L,B}]{\om}}
\app{\lem[wit={ceteri}]{na}
  \rdg[wit={E,L,B}]{\om}}
\app{\lem[wit={ceteri}]{bhavati}
  \rdg[wit={E,L,B}]{\om}}/
%------------------------------
% \om                                                                             \E
%asya siddhasya manomadhye īśvarasaṃbaṃdhī prakāśo niraṃtaraṃ     pratyakṣo bhavati  \P
%asya siddhasya manomadhye īśvarasaṃbaṃdhi prakāśo  niraṃtaraṃ    pratyakṣo bhavati  \L
%asya siddhasya manomadhye īśvaraṃ saṃbaṃdhī prakāśo  niraṃtaraṃ  pratyakṣo bhavatī//  \B
%asya siddhasya manomadhye īśvarasaṃbaṃdhī prakāśaḥ niraṃtaraṃ    pratyakṣa bhavati  \N1
%asya siddhasya manomadhye īśvarasaṃbaṃdhī prakāśaḥ niraṃtaraṃ    pratyakṣa bhavati/  \N2
%asya siddhasya manomadhye īśvarasaṃbaṃdhi prakāśaḥ niraṃtaraṃ    pratyakṣo bhavati  \D
%asya siddhasyaṃ pṛthivī vyāpya tiṣṭhati yasya yanma maraṇai na saḥ sukhaṃ na bhati kulaṃ na bhavati śīlaṃ na bhavati sthānaṃ na bhavati ..... asya siddhasya manomadhye īśvarasaṃbaṃdhī prakāśaḥ niraṃtaraṃ pratyakṣo bhavati  \U1
%asya siddhasya manomadhye īśvarasaṃbaṃdhī prakāśo nirattaraṃ  pratyakṣo bhavati//  \U2
%------------------------------
%The manifestation of permanent perception of the connection with god arises in the middle of the mind of this accomplished one. 
%------------------------------
\note[type=source, labelb=120, lem={prakāśo}]{Ysv (PT): prakāśakuśasambandhiprasaṅgo 'yaṃ nirantaram | sarvaprakāśako'sau tu naṣṭabhedādir eva ca |}
\app{\lem[wit={ceteri}]{asya}
  \rdg[wit={E}]{\om}}
\app{\lem[wit={ceteri}]{siddhasya}
  \rdg[wit={U1}]{siddhasyaṃ pṛthivī vyāpya tiṣṭhati yasya yanma maraṇai na saḥ sukhaṃ na bhati kulaṃ na bhavati śīlaṃ na bhavati sthānaṃ na bhavati asya siddhasya}
  \rdg[wit={E}]{\om}}
%\note[type=philcomm, labelb=s34.z3, lem={asya siddhasyaṃ}]{U\textsubscript{1} repeats the whole section from \textit{pṛthivī} to \ldots \textit{sthānaṃ na bhavati} due to an eyeskip in the process of copying.}
\app{\lem[wit={ceteri}]{manomadhye}
  \rdg[wit={E}]{\om}}
\app{\lem[wit={ceteri}]{īśvarasaṃbandhī}
  \rdg[wit={B}]{īśvaraṃ saṃbaṃdhī}
  \rdg[wit={E}]{\om}}
\app{\lem[wit={ceteri}]{prakāśo}
  \rdg[wit={N1,N2,D,U1}]{prakāśaḥ}
  \rdg[wit={E}]{\om}}
\app{\lem[wit={ceteri}]{nirantaraṃ}
  \rdg[wit={U2}]{nirattaraṃ}
  \rdg[wit={E}]{\om}}
\app{\lem[wit={ceteri}]{pratyakṣo}
  \rdg[wit={N1}]{prakyakṣa}
  \rdg[wit={E}]{\om}}
\app{\lem[wit={ceteri}]{bhavati}
  \rdg[wit={B}]{bhavatī}
  \rdg[wit={E}]{\om}}/
\end{prose}
\end{ekdosis}
\ekdpb*{}
%%%%%%%%%%%%%%%%%%%%%%%%%%%%%%%%%%%%%%%%%%
%%%%%%%%%%%%%%%%%%%%%%%%%%%%%%%%%%%%%%%%%%
%%%%%%%%PAGEBREAK%%%%%%%PAGEBREAK%%%%%%%%%
%%%%%%%%%%%%%%%%%%%%%%%%%%%%%%%%%%%%%%%%%%
%%%%%%%%%%%%%%%%PAGEBREAK%%%%%%%%%%%%%%%%%
%%%%%%%%%%%%%%%%%%%%%%%%%%%%%%%%%%%%%%%%%%
%%%%%%%%PAGEBREAK%%%%%%%PAGEBREAK%%%%%%%%%
%%%%%%%%%%%%%%%%%%%%%%%%%%%%%%%%%%%%%%%%%%
%%%%%%%%%%%%%%%%%%%%%%%%%%%%%%%%%%%%%%%%%%
%%%%%%%%%%%%%%%%%%%%%%%%%%%%%%%%%%%%%%%%%%
%%%%%%%%%%%%%%%%%%%%%%%%%%%%%%%%%%%%%%%%%%
%%%%%%%%PAGEBREAK%%%%%%%PAGEBREAK%%%%%%%%%
%%%%%%%%%%%%%%%%%%%%%%%%%%%%%%%%%%%%%%%%%%
%%%%%%%%%%%%%%%%PAGEBREAK%%%%%%%%%%%%%%%%%
%%%%%%%%%%%%%%%%%%%%%%%%%%%%%%%%%%%%%%%%%%
%%%%%%%%PAGEBREAK%%%%%%%PAGEBREAK%%%%%%%%%
%%%%%%%%%%%%%%%%%%%%%%%%%%%%%%%%%%%%%%%%%%
%%%%%%%%%%%%%%%%%%%%%%%%%%%%%%%%%%%%%%%%%%
%%%%%%%%%%%%%%%%%%%%%%%%%%%%%%%%%%%%%%%%%%
%%%%%%%%%%%%%%%%%%%%%%%%%%%%%%%%%%%%%%%%%%
%%%%%%%%PAGEBREAK%%%%%%%PAGEBREAK%%%%%%%%%
%%%%%%%%%%%%%%%%%%%%%%%%%%%%%%%%%%%%%%%%%%
%%%%%%%%%%%%%%%%PAGEBREAK%%%%%%%%%%%%%%%%%
%%%%%%%%%%%%%%%%%%%%%%%%%%%%%%%%%%%%%%%%%%
%%%%%%%%PAGEBREAK%%%%%%%PAGEBREAK%%%%%%%%%
%%%%%%%%%%%%%%%%%%%%%%%%%%%%%%%%%%%%%%%%%%
%%%%%%%%%%%%%%%%%%%%%%%%%%%%%%%%%%%%%%%%%%
\begin{ekdosis}
  \begin{prose}
    \noindent
%------------------------------
%sa ca prakāśo na śīto na coṣṇo na śveto na pīto bhavati/ \E
%sa ca prakāśo na śīto na coṣṇo na śveto na pīto bhavati/ \P
%sa ca prakāśo na śīto na coṣṇo na śveto na pīto bhavatī// \L
%sa ca prakāśo na śīto na coṣṇo na śveto na pīto bhavatī// \B
%sa ca prakāśo na śīto na coṣṇo na śveto na pīto bhavati/ \N1
%sa ca prakāśo na śīto na coṣṇo na śveto na pīto bhavati    \D
%sa ca prakāśo na śīto na coṣṇo na kheto na pīto bhavati/ \N2
%sa ca prakāśo na śīto na ?hbho?na kheto na pīto bhavati // \U1
%sa ca prakāśo// na śīto na coṣṇo na śveto pīto na bhavati // \U2
%------------------------------
%And he is shining - not cold, and not hot, not white [and] not yellow. 
%------------------------------
sa ca prakāśo na śīto na
\app{\lem[wit={ceteri}]{coṣṇo}
  \rdg[wit={U1}]{...o}}
na
\app{\lem[wit={ceteri}]{śveto}
  \rdg[wit={N2,U1}]{kheto}}
\app{\lem[wit={ceteri}]{na pīto}
  \rdg[wit={U2}]{pīto na}}
\app{\lem[wit={ceteri}]{bhavati}
  \rdg[wit={L,B}]{bhavatī}}/
%------------------------------
%tasya na jātir na kiñciccihnam  \E
%tasya na jātir na kiñciccihnaṃ  \P
%tasya na jātir na kiṃciccinhaṃ  \L
%tasya na jātir na kiṃciccinhaṃ  \B
%tasya na jātir na kiṃciccihūṃ  \N1
%tasya na jāti na kiṃciccihūṃ//  \D
%tasya na jāti na  kiṃciccihūṃ  \N2
%tasya na jātir na kiṃcit khecha cinhaṃ  \U1
%tasya na jānāti na kiṃcit cinhaṃ //  \U2
%------------------------------
%Neither is there birth of him, nor does he have any attributes.
%------------------------------
\note[type=source, labelb=121, lem={jātir}]{asya jāterna cihnañ ca niṣkalo 'yaṃ nirañjanaḥ | ananto 'yaṃ mahājyotir vāñchāṃ bhogaṃ dadāti ca |}
tasya na
\app{\lem[wit={ceteri}, alt={jātir}]{jāti\skp{r-na}}
  \rdg[wit={D,N2}]{jāti}
  \rdg[wit={U2}]{jānāti}
}\skm{r-na}
\app{\lem[wit={ceteri}, alt={kiñcic cihnaṃ}]{kiñcic\skp{-}cihnaṃ}
  \rdg[wit={E}]{°cihnam}
  \rdg[wit={D,N1,N2}]{°cihūṃ}
  \rdg[wit={U1}]{kiṃcit khecha cinhaṃ}
  \rdg[wit={U2}]{na kiṃcit cinhaṃ}}/
%------------------------------
%ayaṃ   ca niṣkalo   niraṃjanaḥ   alakṣyaś ca bhavati \E
%ayaṃ   ca niṣkalo   niraṃjanaḥ   alakṣyaś ca bhavati \P
%vyayaṃ ca niṣkalo   niraṃjanaṃ// alakṣaś  ca bhavati// \L
%vyayaṃ ca nīṣkalo   niraṃjanaṃ// alakṣaś  ca bhavatī// \B
%ayaṃ   ca niṣkalo   niraṃjanaḥ// alakṣyaś ca bhavati// \D
%ayaṃ   ca nīṣkalo   niraṃjanaḥ   alakṣaś  ca bhavati// \N1
%ayaṃ   ca niṣkalo   niraṃjanaḥ   alakṣaś  ca bhavati// \N2
%ayaṃ   ca niḥkalo   niraṃjanaḥ   alakṣyaḥ    bhavati/ \U1
%ayaṃ   ca nīṣkalo   niraṃjanaḥ// alakṣyaḥ    bhavati// \U2
%------------------------------
%And he is without parts, immacule and uncharacterized.  
%------------------------------
\app{\lem[wit={ceteri}]{ayaṃ}
  \rdg[wit={L,B}]{vyayaṃ}}
ca
\app{\lem[wit={ceteri}]{niṣkalo}
  \rdg[wit={B,U2}]{nīṣkalo}
  \rdg[wit={U1}]{niḥkalo}}
nirañjanaḥ/
\app{\lem[wit={ceteri}, alt={alakṣyaś}]{alakṣya\skp{ś-ca}}
  \rdg[wit={U1,U2}]{alakṣyaḥ}
  \rdg[wit={L,B,N1,N2}]{alakṣaś}
}\app{\lem[wit={ceteri}, alt={ca}]{\skm{ś-ca}}
  \rdg[wit={U1,U2}]{\om}}
\app{\lem[wit={ceteri}]{bhavati}
  \rdg[wit={B}]{bhavati}}/
%------------------------------
%atha ca phaladvaṃde  na         kāminy āder   yasyecchā         na bhavati // \E
%atha ca phalacaṃda   na         kāminy āder   yasyochā          na bhavati    \P
%atha ca phalavaṃda   na         kāminy ādir   yasya             na bhavati    \L
%atha ca phalaṃ jaṃda na         kāminy ādar   yasye             na bhavatī    \B
%atha ca phalacaṃdra  na         kāminy āder   yasya  yasyeccha     bhavati/   \N1
%atha ca phalacaṃda   na         kāminy āde    yasya  yasyechā      bhavati//  \D
%atha ca phalaṃ/caṃdra           kāminy āder   yasya  yasyeccha     bhavati/   \N2
%atha ca phalaṃ caṃda na         kāminy āder   yasya  yaṃ           bhavati    \U1
%atha ca phalacaṃda   na         kāminy āder   yasyechā             bhavati//  \U2
%------------------------------
%And his desire etc. doesn't arise in [situations of] lust [and] is not located within the duality of the result.  
%------------------------------
\note[type=source, labelb=122, lem={yasyecchā}]{Ysv (PT): asya citte nānurāgo virāgo na bhaved iti | rājya prāpte 'pi no harṣo hānau duḥkhaṃ bhaven nahi | kvacid vastuni deśasya niḥsvane keṣu kutracit |}
aatha ca
\app{\lem[wit={E}]{phaladvande}
     \rdg[wit={P,D,U2}]{phalacaṃda}
     \rdg[wit={U1}]{phalaṃ caṃda}
     \rdg[wit={L}]{phalavaṃda}
     \rdg[wit={B}]{phalaṃ jaṃda}
     \rdg[wit={N1}]{phalacaṃdra}
     \rdg[wit={N2}]{phalaṃ/ caṃdra}}
\app{\lem[wit={ceteri}]{na}
     \rdg[wit={N2}]{\om}}
kāmi\skp{ny-ā}\app{\lem[wit={ceteri}, alt={āder}]{\skm{ny-ā}de\skp{r-ya}}
     \rdg[wit={D}]{āde}
     \rdg[wit={B}]{ādar}
     \rdg[wit={L}]{ādir}
}\app{\lem[wit={E},alt={yasyecchā}]{\skm{r-ya}syecchā}
     \rdg[wit={P}]{yasyochā}
     \rdg[wit={L}]{yasya}
     \rdg[wit={B}]{yasye}
     \rdg[wit={N1,N2}]{yasya yasyeccha}
     \rdg[wit={D}]{yasya yasyechā}
     \rdg[wit={U1}]{yasya yaṃ}
     \rdg[wit={U2}]{yasye chā}}
\app{\lem[wit={E,P,L,B}]{na}
     \rdg[wit={ceteri}]{\om}}
\app{\lem[wit={ceteri}]{bhavati}
  \rdg[wit={B}]{bhavatī}}/
%------------------------------
% \om                      \E
% \om                      \P
% \om                      \L
% \om                      \B
%taṃ taṃ bhogaṃ prāpnoti   \D
%taṃ taṃ bhogaṃ prāpnoti   \N1
%taṃ taṃ bhogaṃ prāpnoti// \N2
%tataṃ bhogaṃ prāpnoti     \U1
% \om                      \U2
%------------------------------
%He attains widespread enjoyment. 
%------------------------------
\app{\lem[wit={D,N1,N2}]{taṃ taṃ}
  \rdg[wit={U1}]{tataṃ}
\rdg[wit={ceteri}]{\om}}
\app{\lem[wit={D,N1,N2,U1}]{bhogaṃ prāpnoti}
  \rdg[wit={ceteri}]{\om}}/ 
%------------------------------
% \om                      \P
% \om                      \L
% \om                      \B
%atha vā yasya    mana    eva   sthāne 'nurāgaṃ        na prāpnoti// \D
%atha vāsya/vātya mana   eva   sthāne 'nurāgaṃ        na prāpnoti/ \N1
%atha vā syamana         eva   sthāne 'nurāgaṃ        na prāpnoti/ \N2
%atha vā svāmana         etata sthāne  nurāgaṃ         na prāpnoti/ \U1
% \om                      \U2
%------------------------------
%However, his mind does not suffer attachment in this very state.  
%------------------------------
\app{\lem[wit={D,N1,N2,U1}]{atha}
  \rdg[wit={ceteri}]{\om}} 
\app{\lem[wit={D}]{vā yasya}
  \rdg[wit={N1}]{vāsya}
  \rdg[wit={N2}]{vā syamana}
  \rdg[wit={U1}]{vā svāmana}
  \rdg[wit={ceteri}]{\om}}
\app{\lem[wit={D,N1,N2,U1}]{mana}
  \rdg[wit={ceteri}]{\om}}
\app{\lem[wit={D,N1,N2,U1}]{eva}
  \rdg[wit={U1}]{etata}
  \rdg[wit={ceteri}]{\om}}
\app{\lem[wit={D,N1,N2,U1}]{sthāne}
  \rdg[wit={ceteri}]{\om}}
\app{\lem[wit={D,N1,N2}]{'nurāgaṃ}
  \rdg[wit={U1}]{nurāgaṃ}
  \rdg[wit={ceteri}]{om}}
\app{\lem[wit={D,N1,N2,U1}]{na prāpnoti}
  \rdg[wit={ceteri}]{\om}}/ 
\end{prose}
\end{ekdosis}
%%%%%%%%%%%%%%%%
%%%%%%%%%%%%%%%
%%%%%%%%%%%%%%%
%%%%%%%%%%%%%%%%
%%%%%%%%%%%%%%%
  \begin{ekdosis}
 \ekddiv{type=ed}
   \bigskip
    \centerline{\textrm{\small{[Other Attributes]}}}
    \bigskip
    \begin{prose}
%------------------------------
%anyad  rājayogasya cihnaṃ kathyate   \E
% \om                                 \P
%anyata rājayogasya cinhaṃ kathyate// \L
%anyata rājayogasya cinhaṃ kathyate// \B
%anyat  rājayogasya cinhaṃ kathyate// \N1 yasyecchā bhavati??? taṃ taṃ bhogaṃ prāpnoti/ atha vāsya mana eva sthāne 'nu rāgaṃ na prāpnoti/ anyat rājayogasya cinhaṃ kathate//
%anyat  rājayogasya cihuṃ  kathyate// \D
%anyad  rājayogasya ciṃhuṃ kathyate// \N2
%anyat  rājayogacinhaṃ     kathyate/  \U1
%anyat  rājayogasya cinhaṃ kathyate// \U2
%------------------------------
%[Now] another attribute of Rājayoga is described. 
%------------------------------
\app{\lem[wit={E,N2},alt={anyad}]{anya\skm{d-rā}}
  \rdg[wit={N1,D,U1,U2}]{anyat}
  \rdg[wit={L,B}]{anyate}
  \rdg[wit={P}]{\om}
}\app{\lem[wit={ceteri},alt={rājayogasya}]{\skp{d-rā}jayogasya}
  \rdg[wit={U1}]{rājayoga°}
  \rdg[wit={P}]{\om}}
\app{\lem[wit={E}]{cihnaṃ}
  \rdg[wit={L,B,N1,U2}]{cinhaṃ} %????
  \rdg[wit={N2}]{ciṃhuṃ}
  \rdg[wit={D}]{cihuṃ}
  \rdg[wit={P}]{\om}}
\app{\lem[wit={ceteri}]{kathyate}
  \rdg[wit={P}]{\om}}/
%------------------------------
%yasya rājyādilābhe 'pi    phalalābho na bhavati/ \E
% \om                                            \P
%yasya rājādilābhe   ty     aphalalābho       na bhavatī \L
%yasya rājādilābhe   ty     aphalalābho       na bhavatī \B
%yasya rājyādilābhe  pi     phalalābho       ba bhavati/ \N1
%yasya rājyādilābhe  pi     phalalābho       na bhavati// \D
%yasya rājyādilobhe  pi ca  phalalābho       na bhavati// \N2
%yasya rājyādilābe  'pi ca  palalābho        na bhavati/ \U1
%yasya rājyādilābho                          na bhavati/ \U2
%------------------------------
%Even ``of one who is in gain of a kingdom etc.'' [it is said that] perception of success does'nt arise.
%------------------------------
\app{\lem[wit={ceteri}]{yasya}
  \rdg[wit={P}]{\om}} 
\app{\lem[wit={E,N1,D}]{rājyādilābhe}
  \rdg[wit={L,B}]{rājā°}
  \rdg[wit={N2}]{°lobhe}
  \rdg[wit={U1}]{°lābe}
  \rdg[wit={U2}]{°lābho}
  \rdg[wit={P}]{\om}}
\app{\lem[wit={E,N1,D}]{'pi}
  \rdg[wit={N2,U1}]{'pi ca}
  \rdg[wit={L,B}]{ty}
  \rdg[wit={P,U2}]{\om}}
\app{\lem[wit={E,N1,D,N2}]{phalalābho}
  \rdg[wit={U1}]{pala°}
  \rdg[wit={L,B}]{aphala°}
  \rdg[wit={P,U2}]{\om}}
\app{\lem[wit={E,D,N2,U1,U2}]{na bhavati}
  \rdg[wit={L,B}]{na bhavatī}
  \rdg[wit={N1}]{ba bhavati}
  \rdg[wit={P}]{\om}}/
%------------------------------
%hānāv api manomadhye duḥkhaṃ na bhavati/ \E
% \om                                      \P
%hananād pi mānomadhye duḥkahṃ na bhavatī/ \L
%hananād pi mānomadhye duḥkahṃ na bhavatī/ \B
%hānāv api manomadhye duḥkhaṃ na bhavati/ \N1 %emend to hānau loc. sg. of hāni -> abandonment
%hānāv api manomadhye duḥkhaṃ na bhavati// \D
%hānāv  pi manomadhye duḥkhaṃ na bhavati// \N2
%hānāv api manomadhye duḥkhaṃ na bhavati/  \U1
%hānād api manomadhye duḥkhaṃ na bhavati// \U2
%------------------------------
%Even due to loss suffering does'nt arise in the mind.  
%------------------------------
\app{\lem[wit={ceteri},alt={hānāv}]{hānā\skp{v-a}}
  \rdg[wit={U2}]{hānād}
  \rdg[wit={P,L}]{nahanād}
  \rdg[wit={P}]{\om}
}\app{\lem[wit={ceteri},alt={api}]{\skm{v-a}pi}
  \rdg[wit={L,B,N2}]{pi}
  \rdg[wit={P}]{\om}}
manomadhye duḥkhaṃ na
\app{\lem[wit={ceteri}]{bhavati}
  \rdg[wit={L,B}]{bhavatī}}/
%------------------------------
%atha ca tṛṣṇā na bhavati/ \E
% \om                      \P
%atha ca tṛṣṇā na bhavati/ \L
%atha ca tṛṣṇā na bhavatī/ \B
%atha ca tṛṣṇā na bhavati/ \N1
%atha ca tṛṣṇā na bhavati  \D
%atha ca tṛṣṇā na bhavati/ \N2
%atha ca tṛṣṇā na bhavati/ \U1
%atha ca tṛṣṇā na bhavati/ \U2
%------------------------------
%And then desire doesn't arise. 
%------------------------------
\app{\lem[wit={ceteri}]{atha ca}
  \rdg[wit={P}]{\om}}
\app{\lem[wit={ceteri}]{tṛṣṇā}
  \rdg[wit={P}]{\om}}
\app{\lem[wit={ceteri}]{na}
  \rdg[wit={P}]{\om}}
\app{\lem[wit={ceteri}]{bhavati}
  \rdg[wit={B}]{bhavatī}
  \rdg[wit={P}]{\om}}/
%------------------------------
%atha ca kasmin                                  padārthasyopary   anicchā na bhavati/ \E
% \om                                                                                       \P
%atha ca kasmin na    padārtho   prāpte kasyāpi  padārthasyopari   ānīcha  na  bhavati//    \L
%atha ca kasmin na    padārthau  prāpte kasyāpi  padārthāsyopari   ānīchā  ni  bhavati//    \B
%atha ca kasminn pi   padārthe   prāpta kasyāpi  padārthasya upari anusthā na  bhavaṃti//   \N1 
%atha ca kasminn api  padārthe   prāpte kasyāpi  padārthasya upari anichā      bhavaṃti     \D
%atha ca kasminn pi   padārthe   prāpte kasyāpi  padārthasya upari anisthā na  bhavati//    \N2
%atha ca kasminn api  padārthe   prātpe kasyāpi  padārthasya upari aniṣṭā  na  bhavati      \U1
%atha ca kasmin   adhipadārtha   prāpte kābhyādi padārthopari      anicha  na  bhavati//    \U2 %%%407.jpg
%------------------------------
%And then with regards to some object that has been obtained for whatever reason towards ones object aversion does'nt arise.   
%------------------------------
\app{\lem[wit={ceteri}]{atha ca}
  \rdg[wit={P}]{\om}}
kasmi\skp{n-na}\skm{-a}\app{\lem[wit={D,U1},alt={api}]{\skm{n-na}pi}
  \rdg[wit={L,B}]{na}
  \rdg[wit={N1,N2}]{pi}
  \rdg[wit={U2}]{adhi}
  \rdg[wit={E,P}]{\om}}
\app{\lem[wit={ceteri}]{padārthe}
  \rdg[wit={L}]{padārtho}
  \rdg[wit={B}]{padārthau}
  \rdg[wit={U2}]{padārtha°}
  \rdg[wit={E,P}]{\om}}
\app{\lem[wit={ceteri}]{prāpte}
  \rdg[wit={N1}]{prāpta}
  \rdg[wit={E,P}]{\om}}
\app{\lem[wit={ceteri}]{kasyāpi}
  \rdg[wit={U2}]{kābhyādi}
  \rdg[wit={E,P}]{\om}}
\app{\lem[wit={E},alt={padārthasyopary}]{padārthasyopa\skp{ry-a}}
  \rdg[wit={L,B}]{padārthasyopari}
  \rdg[wit={U2}]{padārthopari}
  \rdg[wit={ceteri}]{padārthasya upari}
  \rdg[wit={P}]{\om}}\app{\lem[wit={E},alt={anicchā}]{\skm{r-ya}nicchā}
  \rdg[wit={L}]{ānīcha}
  \rdg[wit={B}]{ānīchā}
  \rdg[wit={N1}]{anusthā}
  \rdg[wit={D}]{anichā}
  \rdg[wit={N2}]{anisthā}
  \rdg[wit={U1}]{aniṣṭā}
  \rdg[wit={U2}]{anicha}}
\app{\lem[wit={ceteri}]{na}
  \rdg[wit={B}]{ni}
  \rdg[wit={P,D}]{\om}}
\app{\lem[wit={ceteri}]{bhavati}
  \rdg[wit={N1,D}]{bhavaṃti}
  \rdg[wit={P}]{\om}}/
%------------------------------
%kasmin    padārthe manaso   nurāgo na bhavati/    \E
%asminnapi padārthe manaso   nurāgo na bhavati... ayam api padārthe manasonurāgo na bhavati... \P
%asminn    padārthe manaso   nurāgo na bhavatī/    \L
%asminn    padārthe manaso   nurāgo na bhavatī/    \B
%asminnapi padārthe manasaḥ anurāgo    bhavati/    \N1
%asminnapi padārthe manasaḥ anurāgo    bhavati//   \D
%asminnapi padārthe manasaḥ anurāgo    bhavati/    \N2
%asminnapi padārthe manasa  anurāgo    bhavati     \U1 
%kasminnpi padārthe         anurāgo na bhavati// ayam api padārthe anurāgo na bhavati//  \U2
%------------------------------
%With regard to this object also affection of the mind does'nt arise. 
%------------------------------
\app{\lem[wit={ceteri},alt={asminn}]{asmi\skp{n-a}}
  \rdg[wit={E,U2}]{kasmin}
}\app{\lem[wit={ceteri},alt={api}]{\skm{n-a}pi}
  \rdg[wit={E,L,B}]{\om}} 
padārthe
\app{\lem[wit={E,P,L,B}]{manaso}
  \rdg[wit={N1,D,N2,U1}]{manasaḥ}
  \rdg[wit={U1}]{manasa}
  \rdg[wit={U2}]{\om}}
\app{\lem[wit={E,P,L,B}]{'nurāgo}
  \rdg[wit={ceteri}]{anurāgo}}
\app{\lem[wit={E,P,U2}]{na bhavati}
  \rdg[wit={L,B}]{na bhavatī}
  \rdg[wit={ceteri}]{bhavati}}/\note[type=philcomm, labelb=123, lem={na bhavati}]{P and U2 add \textit{ayam api padārthe anurāgo na bhavati ||} after this sentence, which is clearly a corruption.}
%------------------------------
%ayam  api rājayogaḥ kathyate/  \E
%atham api rājayogaḥ kathyate   \P
%atha  samarājayogaḥ kathyate/  \L
%ayam  api rājayogaḥ kathyate/  \B
%ayam  api rājayogaḥ kathyate/  \N1
%ayam  api rājayogaḥ kathyate// \D
%ayam  api rājayoga  kathyate// \N2
%ayam  api rājayogaḥ kathyate/  \U1
%ayam  api rājayoga  kathyate// \U2
%------------------------------
%Just this is said to be Rājayoga. 
%------------------------------
\app{\lem[wit={ceteri},alt={ayam}]{aya\skp{m-a}}
  \rdg[wit={P}]{atham}
  \rdg[wit={L}]{atha}
}\app{\lem[wit={ceteri},alt={api}]{\skm{m-a}pi}
  \rdg[wit={L}]{sama}}
\app{\lem[wit={ceteri}]{rājayogaḥ}
  \rdg[wit={N2,U2}]{rājayoga}}
kathyate/
%------------------------------ %%%%split in stemma?! maitre mitre!!!
%atha caḥ yasya manaḥ   munividvat  puruṣeṣu maitre        ca samaṃ bhavati/ \E
%atha ca  yasya manaḥ   śunividvat  puruṣe   maitre śatrau ca samaṃ bhavati \P
%atha ca  yasya manaḥ   bhunividvat puruṣe   maitre śatrau ca samaṃ bhavati/ \L
%atha ca  yasya manaḥ   śrunividvat puruṣe   maitre śatro  ca samaṃ bhavatī/ \B
%atha ca  yasya manaḥ/  śrutividyut puruṣe   mitre  śatrau ca samaṃ bhavati/ \N1
%atha ca  yamanaḥ       śrutividyut puruṣe   mitre  śatrau ca samaṃ bhavati// \D
%atha ca  yasya manaḥ   śrutividyut puruṣe   mitre  śatrau ca samaṃ bhavati/ \N2
%atha ca  yasya mana    śrunividvat puruṣe   mitre  śatrau ca samaṃ bhavati \U1
%atha ca  yasya manaḥ   śuciviśuddhapuruṣe   mitre  śatrau ca samaṃ bhavati// \U2
%------------------------------
%And then his mind which knows the sacred speech is equal towards a person, friend and enemy.  
%------------------------------
\note[type=source, labelb=124, lem={mitre  śatrau}]{Ysv (PT): vidyāvidyāmitraśatrau samā dṛṣṭiś ca sarvaśaḥ | bhogāsaktādikarttṛtvena mano no bhavet khavat |}
atha
\app{\lem[wit={ceteri}]{ca}
  \rdg[wit={E}]{caḥ}}
\app{\lem[wit={ceteri}]{yasya}
  \rdg[wit={D}]{ya}}
manaḥ
\app{\lem[resp=egoscr, type=emendation]{śrutividvat}
  \rdg[wit={E}]{munividvat}
  \rdg[wit={P}]{śunividvat}
  \rdg[wit={L}]{bhunividvat}
  \rdg[wit={B,U1}]{śrunividvat}
  \rdg[wit={N1,N2,D}]{śrutividyut}
  \rdg[wit={U2}]{śuciviśuddha°}
}\app{\lem[wit={ceteri}]{puruṣe}
  \rdg[wit={E}]{puruṣeṣu}}
\app{\lem[wit={ceteri}]{mitre}
  \rdg[wit={E,P,L,B}]{maitre}}
 \app{\lem[wit={ceteri}]{śatrau}
   \rdg[wit={B}]{śatro}
   \rdg[wit={E}]{\om}}
 ca samaṃ bhavati/
   \end{prose}
\end{ekdosis}
\ekdpb*{}
%%%%%%%%%%%%%%%%%%%%%%%%%%%%%%%%%%%%%%%%%%
%%%%%%%%%%%%%%%%%%%%%%%%%%%%%%%%%%%%%%%%%%
%%%%%%%%PAGEBREAK%%%%%%%PAGEBREAK%%%%%%%%%
%%%%%%%%%%%%%%%%%%%%%%%%%%%%%%%%%%%%%%%%%%
%%%%%%%%%%%%%%%%PAGEBREAK%%%%%%%%%%%%%%%%%
%%%%%%%%%%%%%%%%%%%%%%%%%%%%%%%%%%%%%%%%%%
%%%%%%%%PAGEBREAK%%%%%%%PAGEBREAK%%%%%%%%%
%%%%%%%%%%%%%%%%%%%%%%%%%%%%%%%%%%%%%%%%%%
%%%%%%%%%%%%%%%%%%%%%%%%%%%%%%%%%%%%%%%%%%
%%%%%%%%%%%%%%%%%%%%%%%%%%%%%%%%%%%%%%%%%%
%%%%%%%%%%%%%%%%%%%%%%%%%%%%%%%%%%%%%%%%%%
%%%%%%%%PAGEBREAK%%%%%%%PAGEBREAK%%%%%%%%%
%%%%%%%%%%%%%%%%%%%%%%%%%%%%%%%%%%%%%%%%%%
%%%%%%%%%%%%%%%%PAGEBREAK%%%%%%%%%%%%%%%%%
%%%%%%%%%%%%%%%%%%%%%%%%%%%%%%%%%%%%%%%%%%
%%%%%%%%PAGEBREAK%%%%%%%PAGEBREAK%%%%%%%%%
%%%%%%%%%%%%%%%%%%%%%%%%%%%%%%%%%%%%%%%%%%
%%%%%%%%%%%%%%%%%%%%%%%%%%%%%%%%%%%%%%%%%%
%%%%%%%%%%%%%%%%%%%%%%%%%%%%%%%%%%%%%%%%%%
%%%%%%%%%%%%%%%%%%%%%%%%%%%%%%%%%%%%%%%%%%
%%%%%%%%PAGEBREAK%%%%%%%PAGEBREAK%%%%%%%%%
%%%%%%%%%%%%%%%%%%%%%%%%%%%%%%%%%%%%%%%%%%
%%%%%%%%%%%%%%%%PAGEBREAK%%%%%%%%%%%%%%%%%
%%%%%%%%%%%%%%%%%%%%%%%%%%%%%%%%%%%%%%%%%%
%%%%%%%%PAGEBREAK%%%%%%%PAGEBREAK%%%%%%%%%
%%%%%%%%%%%%%%%%%%%%%%%%%%%%%%%%%%%%%%%%%%
%%%%%%%%%%%%%%%%%%%%%%%%%%%%%%%%%%%%%%%%%%
\begin{ekdosis}
  \begin{prose}
    \noindent
%------------------------------
%dṛṣṭiś ca samā bhavati/   \E
%dṛṣṭiś ca namnā bhavati   \P
% \om                      \L
% \om                      \B
%dṛṣṭiś ca samā bhavati//  \N1
%dṛṣṭiś ca samā bhavati//  \D
%dṛṣṭiś ca samā bhavati//  \N2
%dṛṣṭiś ca samā bhavati/   \U1
%dṛṣṭiś ca samā bhavati/   \U2
%------------------------------
%And a neutral view arises. 
%------------------------------
\app{\lem[wit={ceteri},alt={dṛṣṭiś}]{dṛṣṭi\skm{ś-ca}}
  \rdg[wit={L,B}]{\om}
}\app{\lem[wit={ceteri}, alt={ca}]{\skp{ś-ca}}
  \rdg[wit={L,B}]{\om}}
\app{\lem[wit={ceteri}]{samā}
  \rdg[wit={P}]{namnā}
  \rdg[wit={L,B}]{\om}}
\app{\lem[wit={ceteri}]{bhavati}
  \rdg[wit={L,B}]{\om}}/
%------------------------------
%sakalapṛthvīmadhye gamanavataḥ       sukhabhogavataḥ      yasya manasi karttṛtvābhimāno   nāsti/ \E
%sakalapṛthvīmadhye gamanāgamanavataḥ sukhabhogavataḥ      yasya manasi kartṛtvābhimāno    nāsti/ \P
%sakalapṛtvīmadhye  gamanāgamanataḥ   sukhabogho bhavataḥ  yasya manasi kartu tvābhimano   nāsti/ \L
%sakalapṛthvīmadhye gamanāgamanataḥ   sukhabogho bhavataḥ  yasya manasi kartutvābhimano    nāsti// \B
%sakalapṛthvīmadhye gamanavataḥ//     sukhabhogavataḥ/     yasya manasi kartṛtvādyabhimāno nāsti/  \N1
%sakalapṛthvīmadhye gamanaṃvataḥ//    sukhabhogavataḥ      yasya manasi kartṛtvādyabhimāno nāsti// \D
%sakalapṛthvīmadhye gamavataḥ         sukhabhogavataḥ      yasya manasi kartṛtvādyabhimāno nāsti// \N2
%sakalapṛthvīmadhye gamanavataḥ       sukho bhogavataḥ     yasya manasi kartṛtvābhimāno    nāsti   \U1
%sakalapṛthvīmadhye gamanāgamanavat// sukhabhogavat        yasya manasi kartṛtvābhimāno    nāsti// \U2
%------------------------------
%In the mind of one who is situated in the centre of the entire earth, the pride of authorship does't arise, because of death and rebirth, and because of happiness and enjoyment.  %%%check translation think about the Sanskrit 
%------------------------------
\app{\lem[wit={ceteri}]{sakalapṛthvīmadhye}
  \rdg[wit={L}]{°pṛtvī°}}
\app{\lem[wit={P}]{gamanāgamanavataḥ}
  \rdg[wit={U2}]{gamanāgamanavat}
  \rdg[wit={L,B}]{gamanāgamanataḥ}
  \rdg[wit={E,N1,U1}]{gamanavataḥ}
  \rdg[wit={D}]{gamanaṃvataḥ}
  \rdg[wit={U1}]{gamavataḥ}}
\app{\lem[wit={ceteri}]{sukhabhogavataḥ}
  \rdg[wit={L,B}]{sukhabogho bhavataḥ}
  \rdg[wit={U1}]{sukho bhogavataḥ}
  \rdg[wit={U2}]{sukhabhogavat}}
yasya manasi
\app{\lem[wit={E,P,U1,U2}]{kartṛtvābhimāno}
  \rdg[wit={B}]{kartutvābhimano}
  \rdg[wit={L}]{kartu tvābhimano}
  \rdg[wit={N1,N2,D}]{kartṛtvādyabhimāno}}
nāsti/
%------------------------------
%atha ca lokamadhye gamanavataḥ sukhabhogavataḥ yasya manasi karttṛtvābhimāno nāsti/....
%atha ca lokamadhye kartṛtvaṃ na jñāpayati/ \E
%anucalokamadhye    kartṛtvaṃ na jñāpayati/ \P
%anucaralokamadhya  kartṛtvābhimano nāsti \L
%anucaralokamadhya--kartṛtvābhimano nāsti// \B
%anucalokamadhye    kartṛtvaṃ    jñāpayati// \N1
%anucalokamadhye    kartṛtvaṃ na jñātvā payati/ \D
%anucalokamadhye    kartṛtvaṃ na jñāpayati/ \N2
%anucalokamadhye    kartṛtvaṃ    jñātva payati \U1
%anucalokamadhye    kartṛtvaṃ na jñāpayati \U2
%------------------------------
%while wandering the world he doesn't whish to know authorship. 
%------------------------------
\note[type=source, labelb=125, lem={lokamadhye°}]{Ysv (PT): lokamadhye bhavet karttā manomadhye 'pi niṣkriyaḥ |}
\app{\lem[wit={L,B}]{anucara}
  \rdg[wit={N1,N2,D,U1,U2,P}]{anuca°}
  \rdg[wit={L,B}]{anucara°}
  \rdg[wit={E}]{atha ca}
}\app{\lem[wit={ceteri}]{lokamadhye}
  \rdg[wit={L,B}]{°madhya}}
\app{\lem[wit={E,P,D,N2,U2}]{kartṛtvaṃ na}
  \rdg[wit={L,B}]{kartṛtvābhimano}
  \rdg[wit={N1,U1}]{kartṛtvaṃ}}
\app{\lem[wit={E,P,N1,N2,U2}]{jñāpayati}
  \rdg[wit={D,U1}]{jñātva payati}
  \rdg[wit={L,B}]{nāsti}}/
%------------------------------
%so  pi  rājayogaḥ kathyate// \E
%so  pi  rājayogaḥ kathyate   \P
%so  pi  rājayoga  kathyate/   \L
%so  pi  rājayoga  kathyate/   \B
%so  pi  rājayogaḥ kathyate//  \N1
%so  pi  rājayoga  kathyate//  \D
%so 'pi  rājayoga  kathyate// \N2
%so  pi  rājayoga  kathyate/   \U1
%so  pi  rājayoga  kathyate    \U2
%------------------------------
%This is also said to be Rājayoga. 
%------------------------------
\note[type=source, labelb=126, lem={so 'pi \ldots}]{eṣo 'pi rājayogīti sukhe duḥkhe samas tathā |}
so 'pi
\app{\lem[wit={E,P,N1}]{rājayogaḥ}
  \rdg[wit={ceteri}]{rājayoga}}
kathyate/
%------------------------------
%navīnāni         paṭṭasūtramaya     dhṛtāni vastrāṇi   \E
%navīnāni         paṭasūtramayāni    dhṛtāni vastrāṇi   \P
%navinīnīśpī      paṭṭasūtramayāni   dhṛtāni vastrāṇi// \L
%navinīnīr api    paṭṭasūtramayāni   dhṛtāni vastrāṇi// \B
%navīnāni         paṭasūtramayāni    dhṛtāni vastrāṇi/  \N1
%navīnāni         paṭasūtramayāni    dhṛtāni vastrāṇi// \D
%navīnāni         paṭasūtramayāni    dhṛtāni vastrāṇi/  \N2
%navīnāni         padasūtramayāni       tāni vastrāṇi   \U1
%navīnāni      paṭ(h)asūtramayāni    dhṛtāni            \U2
%------------------------------
%New durable clothes made of silk,  
%------------------------------
\app{\lem[wit={ceteri}]{navīnāni}
  \rdg[wit={L}]{navīnīnīś pī}
  \rdg[wit={B}]{navinīnīr api}}
\app{\lem[wit={E,L,B}, alt={paṭṭa°}]{paṭṭa}
  \rdg[wit={P,N1,D,N2,U2}]{paṭa°}
  \rdg[wit={U1}]{pada}
}sūtra\app{\lem[wit={ceteri},alt={°mayāni}]{mayāni}
  \rdg[wit={E}]{maya}}
\app{\lem[wit={ceteri}]{dhṛtāni}
  \rdg[wit={U1}]{tāni}}
\app{\lem[wit={ceteri}]{vastrāṇi}
  \rdg[wit={U2}]{\om}}
%------------------------------ %%%%KOLLOQUIUM: was hier tun? kastūrī/kasturikā = gleichwertig 
%atha vā jīrṇāni chidrāṇi    dhṛtāni    kastūrīcandanalepair   vā  kardamalepena   yasya manasi harṣaśokau  na staḥ/ \E
%atha vā jīrṇāni sachadrāṇi  dhūtāni    kastūrīcaṃdanalepo     vā  karddamalepo vā yasya manasi harṣaśokau na staḥ/ \P
%atha vā jīrṇāni svachidrāṇi dhṛtāni    kasturīcaṃdanalepo     cā  kardamalepo  vā yasya manasi harṣaśokau na sthaḥ// \L
%atha vā jīrṇāni svachidrāṇi dhṛtāni    kastūrīcaṃdanalepo     vā  kardamalepo  vā yasya manasi harṣaśokau na sthaḥ// \B
%atha vā jīrṇāni sacchidrāṇi dhṛtāni/   kasturikā caṃdanalepo vā/ kardamalepo  vā yasya manasi harṣaśoko  na sthaḥ  \N1
%atha vā jīrṇāni sacchidrāṇi dhṛtāni//  kasturikā caṃdanalepo vā/ kardamalepo  vā yasya manasi harṣaśoko  na sthaḥ  \D
%atha vā jīrṇāni sacchidrāṇi dhṛtāni // kasturikā caṃdanalepo vā/ kardamalepo  vā yasya manasi harṣaśoka  na sthāḥ \N2
%atha vā jīrṇāni sachidrāṇi  dhvatāni   kasturikā caṃdanalepo vā  kardamalepo  vā yasya manasi harṣaśokau na sthāḥ \U1 %%272.jg
%                                       kastūrīcaṃdanalepo     vā                  yasya manasi harṣaśoko  na sta// \U2
%------------------------------
%or however, old, worn (clothes) with holes smeared with sandalwood and musk, or smeared with mud. In whose mind joy and sorrow are not situated,
%------------------------------
atha vā jīrṇāni
\app{\lem[wit={N1,N2,D}]{sacchidrāṇi}
  \rdg[wit={U2}]{sachidrāṇi}
  \rdg[wit={P}]{sachadrāṇi}
  \rdg[wit={L,B}]{svachidrāṇi}
  \rdg[wit={E}]{chidrāṇi}}
\app{\lem[wit={ceteri}]{dhṛtāni}
  \rdg[wit={U2}]{dhvātāni}
  \rdg[wit={P}]{dhūtāni}}
\app{\lem[wit={E,P,B,U2}]{kastūrī}
  \rdg[wit={L}]{kasturī}
  \rdg[wit={N1,N2,D,U1}]{kasturikā}
}\app{\lem[wit={E},alt={candana°}]{candana}
  \rdg[wit={ceteri}]{caṃdana°}
}\app{\lem[wit={E},alt={lepair}]{lepai\skp{r-vā}}
  \rdg[wit={ceteri}]{lepo}} 
\app{\lem[wit={ceteri},alt={vā}]{\skm{r-vā}}
  \rdg[wit={L}]{cā}}
\app{\lem[wit={E}]{kardamalepena}
  \rdg[wit={ceteri}]{kardamalepo}}
\app{\lem[wit={ceteri}]{vā}
  \rdg[wit={E}]{\om}}
yasya manasi
harṣa\app{\lem[wit={ceteri},alt={°śokau}]{śokau}
  \rdg[wit={N1,D,U2}]{°śoko}
  \rdg[wit={N2}]{°śoka}}
na
\app{\lem[type=emendation, resp=egoscr]{sthau}
  \rdg[wit={ceteri}]{\korr sthaḥ}
  \rdg[wit={N2,U1}]{sthā}
  \rdg[wit={U2}]{sta}}
%------------------------------
%sa evātra tiṣṭhati/         \E
%sa eva rājayogaḥ            \P
%sa eva rājayogaḥ// idānīṃ// \L
%sa eva rājayogaḥ// idānīṃ// \B
%sa eva rājayogaḥ//          \N1
%sa eva rājayogaḥ//          \D
%sa eva rājayogaḥ//          \N2
%sa eva rājayogaḥ            \U1
%sa eva rājayoga             \U2
%------------------------------
%just he is in the state of Rājayoga. 
%------------------------------
%yasya janmamaraṇe na staḥ sukhaṃ na bhavati/ kulaṃ na bhavati śīlaṃ na bhavati/ sthānaṃ na bhavati/ \E
%\om \P
%\om \L
%\om \B
%\om \N1
%\om \D
%\om \N2
%\om \U1
%\om \U2
%------------------------------
%One who is not situated in birth and death has no happiness, has no family, and cold does not arise, place does not arise.?!?!!?
%----------------------------
\app{\lem[wit={ceteri}]{sa eva}
  \rdg[wit={E}]{sa evātra}}
\app{\lem[wit={ceteri}]{rājayogaḥ}
  \rdg[wit={U2}]{rājayoga}
  \rdg[wit={L,B}]{rājayogaḥ || idānīṃ ||}
  \rdg[wit={E}]{tiṣṭhati}}/\note[type=philcomm, labelb=127, lem={°tiṣṭhati}]{E adds \textit{yasya janmamaraṇe na staḥ sukhaṃ na bhavati | kulaṃ na bhavati śīlaṃ na bhavati | sthānaṃ na bhavati |} here, which seems to be a dittography of previous sentences.}
%----------------------------
%rājayogaḥ naramadhye      atha ca vanamadhye             yuddhe saṃgrāmamadhye                        vā yasya manaḥ        bhayapūrṇaṃ vā  na bhavati/  so pi rājayogaḥ kathyate// \E
%          nagaramadhye    'tha ca vanamadhye                  utasaṃgrāmamadhye                       vā yasya mana      ūnaṃ    pūrṇaṃ vāṃ na bhavati   so pi rājayogaḥ            \P
%          nagaramadhye     tha ca vanamadhye                 udvastagrāmamadhye                       vā yasya manaḥ     unaṃ    pūrṇaṃ vā  na bhavati   so pi rājayogaḥ//          \L
%          nagaramadhye  (')tha ca vanamadhye                udvastagrāmaṃmadhye                       vā yasya manaḥ     unaṃ    pūrṇaṃ vā  na bhavatī   so pi rājayogaḥ//          \B
%          nagaramadhye    atha ca vanamadhye/                 udvesūgrāmamadhye .. ..pūrṇagrāmamadhye vā yasya manaḥ     ūnaṃ na pūrṇaṃ vā  na bhavati// so pi rājayogaḥ//          \N1
%          ṣagaramadhye    atha ca vanamadhye//                udvesūgrāmamadhye svetapūrṇagrāmamadhye vā yasya manaḥ     ūnan na pūrṇaṃ vā  na bhavati/  so pi rājayogaḥ//          \D
%          nagaramadhye    atha ca vanamadhye//                udvesūgrāmamadhye svetapūrṇagrāmamadhye vā yasya manaḥ     ūnan na pūrṇaṃ vā  na bhavati/  so pi rājayogaḥ//          \N2
%       vā nagaramadhye    atha ca vanamadhye                 udassaṃgrāmamadhye  lokapūrṇagrāmamadhye vā yasya manaḥ     unaṃ    pūrṇaṃ     na bhavati   so pi rājayogaḥ            \U1
%          nagaramadhye    'tha vā vanamadhye                  udvasagrāmamadhye                       vā yasya mana      ūnaṃ    pūrṇaṃ vāṃ na bhavati   so pi rājayogaḥ            \U2
%------------------------------
%Just he is in the state of Rājayoga for whom the mind is neither in abundance nor in lack, being located in a city, a forest, an uninhabited village or a village full of people. 
%----------------------------
\app{\lem[wit={ceteri}]{nagaramadhye}
  \rdg[wit={E}]{rājayogaḥ nagaramadhye}
  \rdg[wit={D}]{ṣagaramadhye}
  \rdg[wit={U1}]{vā nagaramadhye}}
\app{\lem[wit={P,L,B,U2}]{'tha ca}
  \rdg[wit={E,N1,N2,D,U1}]{atha ca}}
vanamadhye
\app{\lem[wit={U2},alt={udvasa°}]{udvasa}
  \rdg[wit={E}]{yuddhe saṃ°}
  \rdg[wit={P}]{utasaṃ°}
  \rdg[wit={L,B}]{udvasta°}
  \rdg[wit={N1,N2,D}]{udvesū°}
  \rdg[wit={U1}]{udassaṃ°}
}\app{\lem[wit={ceteri}]{grāmamadhye}
  \rdg[wit={B}]{grāmaṃ madhye}}
\app{\lem[wit={U1}]{lokapūrṇagrāmamadhye}
  \rdg[wit={N1}]{....pūrṇagrāmamadhye}
  \rdg[wit={D,N2}]{svetapūrṇagrāmamadhye}}
vā yasya
\app{\lem[wit={P,U2}]{mana}
  \rdg[wit={ceteri}]{manaḥ}}
\app{\lem[wit={P,N1,N2,U2}]{ūnaṃ}
  \rdg[wit={D,N2}]{ūnan}
  \rdg[wit={L,B,U1}]{unaṃ}
  \rdg[wit={E}]{bhaya°}}
\app{\lem[wit={N1,N2,D}]{na}
  \rdg[wit={ceteri}]{\om}}
pūrṇaṃ
\app{\lem[wit={ceteri}]{vā}
  \rdg[wit={P,U2}]{vāṃ}
  \rdg[wit={U1}]{\om}}
na bhavati/ so
\app{\lem[type=emendation, resp=egoscr]{'pi}
  \rdg[wit={ceteri}]{\korr pi}}
\app{\lem[wit={ceteri}]{rājayogaḥ}
  \rdg[wit={E}]{rājayogaḥ kathyate}}\dd{}
\end{prose}
\end{ekdosis}
%%%%%%%%%%%%%%
%%%%%%%%%%%%%%
%%%%%%%%%%%%%%
%%%%%%%%%%%%%
%%%%%%%%%%%%%%%
\begin{ekdosis}
  \ekddiv{type=ed}
  \bigskip
   \centerline{\textrm{\small{[Caryāyoga]}}}
      \bigskip
      \begin{prose}
%----------------------------
%idānīṃ      yogaḥ  kathyate/ \E
%idānīṃ caryāyogaḥ  kathyate   \P
%idānīṃ caryāyogaḥ  kathyate// \L
%idānīṃ caryāyogaḥ  kathyate// \B
%idānīṃ caryāyoga   kathyate// \N1
%idānīṃ caryāyogaḥ  kathyate// \D [S.7, Z.7]
%idānīṃ caryāyoga   kathyate// \N2
%idānīṃ tvaryāyogaḥ kathyate \U1
%idānīṃ caryāyoga   kathyate// \U2
%------------------------------
%Now \textit{caryāyogaḥ}, the Yoga of wandering is explained.
%----------------------------
idānīṃ
\app{\lem[wit={ceteri}]{caryāyogaḥ}
     \rdg[wit={U1}]{tvaryāyogaḥ}
     \rdg[wit={E}]{yogaḥ}} kathyate/
%----------------------------
%nirākāro         nityo 'bhedyaḥ    sa etādṛśaḥ ātmani                  mano   yasya  niścalaṃ tiṣṭhati/  \E
%nirākāro  'calo  nityo  bhedhyaḥ   sa etādṛa   ātmā    etādṛśo  ātmani mano   yasya  niścala  tiṣṭhati   \P %%%7639.jpg
%nirākāro  calo   nityo  bhedhyaḥ   sa etādṛa   ātmā sa etādṛśe  ātmani               niścala  tiṣṭhati/  \L     %daṇḍa nach ātmā besser -> emend? oder in weiteren Hss?
%nirākāro  calo   nityo  bhedhyaḥ   sa etādṛa   ātmā sa etādṛśye ātmani               niścalaṃ tiṣṭhati/  \B
%nirākālo  nityo   calo 'bhedhyaḥ/  sa etādṛśaḥ ātmā    etādṛśe  ātmani manaḥ  yasya  niścalaṃ tiṣṭhati   \N1
%nirākālo  nityo   calo 'bhedhyaḥ// sa etādṛśaḥ ātmā    etādṛśe  ātmani manaḥ  yasya  niścalaṃ tiṣṭhati   \D
%nirākālo  nityo   calo 'bhedhyaḥ   sa etādṛśaḥ ātmā    etādṛśa  ātmani manaḥ  yasya  niścala  tiṣṭhati/  \N2
%nirākāro  nityo   calo abhedhyaḥ   sa etādṛśaḥ ātmā    etādṛśo  ātmani mano   yasya  niścalaṃ bhavati    \U1
%nirvikāro  'calo nityo 'bhedhya    sa etādṛśā  ātmani                  mano   yasya  niścalaṃ tiṣṭhati// \U2
%------------------------------
%Shapeless, unchangeable, permanent [and] unsplitable. Such is the self. It is seen as such by the one whose mind abides in the self without moving. 
%------------------------------
   \note[type=source, labelb=128, lem={caryāyogaḥ}]{harṣaśokau na jātveṣāṃ nodvego lokasaṅgame | nityollāse nirākāre nirāsane nirātmani | manasā niścalo bhūtvā sadā tiṣṭhet samo 'pi ca |}
   \note[type=philcomm, labelb=129, lem={caryāyogaḥ}]{Notwithstanding that \textit{cāryayoga} ist not mentioned in Ysv, Rāmacandra decides to utilizes this passage to construe another type of yoga from his list.}
\app{\lem[wit={E,P,L,B,U1}]{nirākāro}
  \rdg[wit={N1,N2,D}]{nirākālo}
  \rdg[wit={U2}]{nirvikāro}}
\app{\lem[wit={P,U2}]{'calo}
  \rdg[wit={L,B}]{calo}
  \rdg[wit={N1,N2,D,U1}]{nityo}
  \rdg[wit={E}]{\om}}
\app{\lem[wit={E,P,L,B,U2}]{nityo}
  \rdg[wit={ceteri}]{calo}}
\app{\lem[wit={E,N1,N2,D}]{'bhedyaḥ}
  \rdg[wit={P,L,B}]{bhedhyaḥ}
  \rdg[wit={U1}]{abhedhyaḥ}
  \rdg[wit={U2}]{'bhedyha}}
   sa
\app{\lem[wit={P,L,B}]{etādṛśa}
  \rdg[wit={E,N1,N2,D,U1}]{etādṛśaḥ}
  \rdg[wit={U2}]{etādṛśā}}
\app{\lem[wit={ceteri}]{ātmā}
  \rdg[wit={E,U2}]{ātmani}}
\app{\lem[wit={L,B}]{sa}
  \rdg[wit={ceteri}]{\om}}
\app{\lem[wit={N2}]{etādṛśa}
  \rdg[wit={P,U1}]{etādṛśo}
  \rdg[wit={L,N1,D}]{etādṛśe}
  \rdg[wit={B}]{etādṛśye}
  \rdg[wit={E,U2}]{\om}}
\app{\lem[wit={ceteri}]{ātmani}
  \rdg[wit={E,U2}]{\om}}
\app{\lem[wit={E,P,U1,U2}]{mano}
  \rdg[wit={N1,N2,D}]{manaḥ}
  \rdg[wit={L,B}]{\om}}
\app{\lem[wit={ceteri}]{yasya}
  \rdg[wit={L,B}]{\om}}
\app{\lem[wit={ceteri}]{niścalaṃ}
  \rdg[wit={P,L,N2}]{niścala}}
\app{\lem[wit={ceteri}]{tiṣṭhati}
  \rdg[wit={U1}]{bhavati}}/
%------------------------------
%tasyātmanaḥ puṇyapāpasparśo na bhavati/ \E
%tasyātmanaḥ puṇyapāpasparśo na bhavati  \P
%tasyātmanaḥ puṇyapāpasparśo na bhavati/ \L
%tasyātmanaḥ puṇyapāpasparśo na bhavatī/ \B
%tasyātmanaḥ punyapāpasparśo na bhavati/  \N1
%tasyātmanaḥ punyapāpasparśo na bhavati// \D
%tasyātmanaḥ puṇyapāpasparśo na bhavati/ \N2
%tasya ātmanaḥ puṇyapāsya sparśo na bhavati  \U1
%tasya ātmanaḥ puṇyapāsya sparśo na bhavati//  \U2
%------------------------------
%His self is not touched by sin and merit. 
%------------------------------
\app{\lem[wit={ceteri}]{tasyātmanaḥ}
  \rdg[wit={U1,U2}]{tasya ātmanaḥ}}
\app{\lem[wit={ceteri}]{puṇyapāpasparśo}
  \rdg[wit={U1,U2}]{puṇyapāsya sparśo}}
na bhavati/
%------------------------------
%udakamadhye sthitasya padmapatre       yathodakasya sparśo    bhavati/  tathaivātmani   \E
%udakamadhye sthitasya padmanī patrasya yathodakasya sparśo na bhavati   tathaivātmani   \P
%udakamadhye sthitasya padmanī patrasya yathodakasya sparśo na bhavati/  tathaivātmani   \L
%udakamadhye sthitasya padmanī patrasya yathodakasya sparśā na bhavatī/  tathaivātmani   \B
%udakamadhye sthitasya padminī patrasya yathā/ udakasparśo  na bhavati/  tathaivātmani   \N1
%udakamadhye sthitasya padminī patrasya yathā  udakasparśo  na bhavati// tathaivātmani   \D
%udakamadhye sthitasya padminī patrasya yathā  udakasparśo  na bhavati/  tathaivātmani   \N2
%udakamadhye sthitasya padminī patrasya yathā  udakasparśo  na bhavati   tathaivātmani   \U1
%udakamadhye sthitasya padminī patrasya yathodakasparśo     na bhavati// tathaivātmani   \U2
%------------------------------
%Just as the leave of the lotus situated in the amidst water doesn't touch the water; likewise the self [is not touched by sin and merit].
%------------------------------
udakamadhye sthitasya
\app{\lem[wit={ceteri}]{padminī patrasya}
  \rdg[wit={P,L,B}]{padmanī patrasya}
  \rdg[wit={E}]{padmapatre}}
\end{prose}
\end{ekdosis}
\ekdpb*{}
%%%%%%%%%%%%%%%%%%%%%%%%%%%%%%%%%%%%%%%%%%
%%%%%%%%%%%%%%%%%%%%%%%%%%%%%%%%%%%%%%%%%%
%%%%%%%%PAGEBREAK%%%%%%%PAGEBREAK%%%%%%%%%
%%%%%%%%%%%%%%%%%%%%%%%%%%%%%%%%%%%%%%%%%%
%%%%%%%%%%%%%%%%PAGEBREAK%%%%%%%%%%%%%%%%%
%%%%%%%%%%%%%%%%%%%%%%%%%%%%%%%%%%%%%%%%%%
%%%%%%%%PAGEBREAK%%%%%%%PAGEBREAK%%%%%%%%%
%%%%%%%%%%%%%%%%%%%%%%%%%%%%%%%%%%%%%%%%%%
%%%%%%%%%%%%%%%%%%%%%%%%%%%%%%%%%%%%%%%%%%
%%%%%%%%%%%%%%%%%%%%%%%%%%%%%%%%%%%%%%%%%%
%%%%%%%%%%%%%%%%%%%%%%%%%%%%%%%%%%%%%%%%%%
%%%%%%%%PAGEBREAK%%%%%%%PAGEBREAK%%%%%%%%%
%%%%%%%%%%%%%%%%%%%%%%%%%%%%%%%%%%%%%%%%%%
%%%%%%%%%%%%%%%%PAGEBREAK%%%%%%%%%%%%%%%%%
%%%%%%%%%%%%%%%%%%%%%%%%%%%%%%%%%%%%%%%%%%
%%%%%%%%PAGEBREAK%%%%%%%PAGEBREAK%%%%%%%%%
%%%%%%%%%%%%%%%%%%%%%%%%%%%%%%%%%%%%%%%%%%
%%%%%%%%%%%%%%%%%%%%%%%%%%%%%%%%%%%%%%%%%%
%%%%%%%%%%%%%%%%%%%%%%%%%%%%%%%%%%%%%%%%%%
%%%%%%%%%%%%%%%%%%%%%%%%%%%%%%%%%%%%%%%%%%
%%%%%%%%PAGEBREAK%%%%%%%PAGEBREAK%%%%%%%%%
%%%%%%%%%%%%%%%%%%%%%%%%%%%%%%%%%%%%%%%%%%
%%%%%%%%%%%%%%%%PAGEBREAK%%%%%%%%%%%%%%%%%
%%%%%%%%%%%%%%%%%%%%%%%%%%%%%%%%%%%%%%%%%%
%%%%%%%%PAGEBREAK%%%%%%%PAGEBREAK%%%%%%%%%
%%%%%%%%%%%%%%%%%%%%%%%%%%%%%%%%%%%%%%%%%%
%%%%%%%%%%%%%%%%%%%%%%%%%%%%%%%%%%%%%%%%%%
\begin{ekdosis}
  \begin{prose}
    \noindent
\app{\lem[wit={E,P,L}]{yathodakasya sparśo}
  \rdg[wit={B}]{yathodakasya sparśā}
  \rdg[wit={N1,N2,D,U1}]{yathā udakasparśo}
  \rdg[wit={U2}]{yathodakasparśo}}
na
\app{\lem[wit={ceteri}]{bhavati}
  \rdg[wit={B}]{bhavatī}}
tathaivātmani/
%------------------------------
%yathākāśamadhye   pavanaḥ svecchayā bhramati/ \E
%yathākāśamadhye   pavanaḥ svechayā  bhramati \P
%yathā ākāśamadhye pavanaḥ svechayā  bhramati/ \L
%yathā ākāśamadhye pavanaḥ svechayā  bhramatī/ \B
%yathā ākāśamadhye pavanasvachayā    bhramati/ \N1
%yathā ākāśamadhye pavanasvachayā    bhramati \D
%yathā ākāśamadhye pavanasvachayā    bhramati/ \N2
%yathā ākāśamadhye pavanaḥ svechayā  bhramayati \U1
%yathā 'kāśamadhye pavanaḥ svechayā  bhramati// \U2
%------------------------------
%Just as the wind wanders according to its own will in space,...  
%------------------------------
\note[type=source, labelb=130, lem={pavanaḥ}]{Ysv (PT): yathākāśe bhraman vāyur ākāśaṃ vrajate svayam | tathākāśe mano līnaṃ rājayogakriyā matā | jagatsaṃsarganirlepaṃ padmapatrajalaṃ yathā ||}
yathā\app{\lem[wit={E,P}]{kāśamadhye}
  \rdg[wit={U2}]{'kāśamadhye}
  \rdg[wit={ceteri}]{ākāśamadhye}}
\app{\lem[wit={ceteri}]{pavanaḥ svechayā}
  \rdg[wit={N1,N2,D}]{pavanasvachayā}}
\app{\lem[wit={ceteri}]{bhramati}
  \rdg[wit={U1}]{brahmayati}}
%------------------------------
%tathā yasya manaḥ nirākāramadhye līnaṃ bhavati/  sa eva caryāyogaḥ// \E
%tathā yasya manaḥ nirākāramadhye līnaṃ bhavati   sa eva caryāyogaḥ   \P
%tathā yasya manaḥ nirākāramadhye līnaṃ bhavati   sa eva caryāyogaḥ// \L
%tathā yasya manaḥ nirākāramadhye līnaṃ bhavatī   sa eva caryāyogaḥ// \B
%tathā yamanaḥ     nirākāramadhye līnaṃ bhavati/  sa eva kriyāyogaḥ// \N1
%tathā yasya manaḥ nirākāramadhye līnaṃ bhavati/  sa eva kriyāyogaḥ// \D !!!!!Stemma point!!!!!!
%tathā       pavananirākāramadhye līnaṃ bhavati/  sa eva kriyāyogaḥ// \N2
%tathā yasya manaḥ nirākāramadhye līnaṃ bhavati   sa eva kriyāyogaḥ   \U1 
%tathā yasya manaḥ nirākāramadhye līnaṃ bhavati// sa eva caryāyogaḥ// \U2
%------------------------------
%Likewise is the mind of whom is absorbed into the universal spirit [wanders according to its own will in space]. This is \textit{\caryāyoga}.  
%------------------------------
tathā
\app{\lem[wit={ceteri}]{yasya manaḥ}
  \rdg[wit={D}]{yamanaḥ}
  \rdg[wit={N2}]{pavana°}}
nirākāramadhye līnaṃ
\app{\lem[wit={ceteri}]{bhavati}
  \rdg[wit={B}]{bhavatī}}/
sa eva
\app{\lem[wit={ceteri}]{caryāyogaḥ}
  \rdg[wit={N1,N2,D,U1}]{kriyāyogaḥ}}\dd{}
\end{prose}
\end{ekdosis}
 %%%%%%%%%%%%%%%%%%%%%%%%%%%%%
 %%%%%%%%%%%%%%%%%%%%%%%%%%%%%
 %%%%%%%%%%%%%%%%%%%%%%%%%%%%%
 %%%%%%%%%%%%%%%%%%%%%%%%%%%%%
 %%%%%%%%%%%%%%%%%%%%%%%%%%%%%
\begin{ekdosis}
  \ekddiv{type=ed}
  \bigskip
  \centerline{\textrm{\small{[Haṭhayoga]}}}
    \bigskip
      \begin{prose}       
%------------------------------
%idānīṃ grahayogaḥ kathyate/  \E %[p.23]
%idānīṃ haṭhayogaḥ kathyate   \P
%idānīṃ haṭhayogaḥ kathyate/  \L
%idānīṃ haṭayoga   kathyate/  \B
%idānīṃ haṭhayogaḥ kathyate//  \N1
%idānīṃ haṭhayogaḥ kathyate/  \D
%idānīṃ haṭhayoga  kathyate// \N2
%idānīṃ haṭhayogaḥ kathyate   \U1
%idānīṃ haṭhayoga  kathyate   \U2
%------------------------------
%Now \textit{haṭhayoga} is explained. 
%------------------------------
\app{\lem[wit={P,L,N1,D,U1}]{haṭhayogaḥ}
          \rdg[wit={U2}]{haṭhayoga}
          \rdg[wit={B}]{haṭayoga}
          \rdg[wit={E}]{grahayogaḥ}} kathyate/\note[type=source, labelb=131, lem={haṭhayogaḥ}]{Ysv (PT): idānīṃ haṭhayogas tu kathyate haṭhasiddhidaḥ | kṛtvāsanaṃ pavanāśaṃ śarīre rogahārakam | pūrakaṃ kumbhakañcaiva recakaṃ vāyunā bhajet | itthaṃ kramotkramaṃ jñātvā pavanaṃ sādhayet sadā | dhauty ādikarmaṣaṭkañ ca prakuryādd haṭhasādhakaḥ | etan nāḍyān tu deveśi vāyupūrṇaṃ pratiṣṭhitam | tato mano niścalaṃ syāt tata ānanda eva hi | haṭhayogān na kālaḥ syān manonāśo bhaved yadi |}
idānīṃ
%------------------------------
%recakapūrakakumbhaka  ityādiprakāreṇa   pavanasādhanaṃ     kartavyam/ \E
%recakapūrakakuṃbhaka  ityādiprakāreṇa   pavanasādhanaṃ     karttavyaṃ \P
%recakapūrakakumbhaka  ityādiprakāreṇa   pavanasya sādhanaṃ kartavyam// \L
%recakapūrakakuṃbhaka  ityādiprakāreṇa// pavanasya sādhanaṃ kartavyam \B
%recakapūrakakuṃbhaka/ ityādiprakāreṇa   pavanasya sādhanaṃ kartavyaṃ/ \N1
%recakapūrakakuṃbhaka  ityādiprakāreṇa   pavanasya sādhanaṃ kartavyaṃ// \D
%recakapūrakakuṃbhaka  ityādhiprakāreṇa  pavanasya sādhanaṃ kartavyaṃ// \N2
%recakapūrakakuṃbhaka  ityādiprakāreṇa   pavanasya sādhanaṃ kartavyaṃ \U1
%recakapūrakakuṃbhaka  ityādiprakāreṇa   pavanasya sādhanaṃ kartavyaṃ// \U2
%------------------------------
%The practice of breath shall be done in this manner: "Exhalation, Inhalation [and] Retention etc.
%------------------------------        
        recakapūrakakuṃbhaka
        \app{\lem[wit={ceteri}, alt={ityādi}]{ityādi}
          \rdg[wit={N2}]{ityādhi°}
        }prakāreṇa
        \app{\lem[wit={ceteri}]{pavanasya sādhanaṃ}
          \rdg[wit={E,P}]{pavanasādhanaṃ}}
 \app{\lem[wit={E,L,B}]{kartavyam}
   \rdg[wit={ceteri}]{kartavyaṃ}}/
%------------------------------
%atha ca dhautyādiṣaṭkarmakāraṇāt   śarīrasya śuddhir bhavati/ \E
%atha ca dhautyādiṣaṭkarmakāraṇāt   śarīrasya śuddhir bhavati \P
%atha ca dhautyādiṣaṭkarmakāraṇāt// śarīrasya śuddhir bhavati \L
%atha ca  dhotyādiṣaṭkarmakaraṇāt// śarīrasya śuddhir bhavatī \B
%atha ca dhautyādiṣaṭkarmakaraṇāt/  śarīrasya śuddhir bhavati/ \N1
%atha ca dhautyādiṣaṭkarmakaraṇāt   śarīrasya śuddhir bhavati// \D
%atha ca dhautyādiṣaṭkarmakaraṇāt// śarīrasya śuddhir bhavati// \N2
%atha   vidhotyādiṣaṭkarmakaraṇāt   śarīrasya śuddhir bhavati/ \U1
%atha ca dhautyādiṣaṭkarmakaraṇāt// śarīrasya śuddhir bhavati// \U2 %%%408.jpg 
%------------------------------
%And then due to the six practices(\textit{ṣaṭkarma}), like \textit{dhauti} etc. the purification of the body arises. 
%------------------------------        
 atha
 \app{\lem[wit={ceteri}]{ca}
   \rdg[wit={U1}]{\om}}
 \app{\lem[wit={ceteri}]{dhautyādi}
   \rdg[wit={B}]{dhotyādi}
   \rdg[wit={U1}]{vidhotyādi}
 }ṣaṭkarmakāraṇāt śarīrasya śuddhir\skp{-}bhavati/
 %------------------------------
%sūryanāḍīmadhye       pavanaḥ pūrṇo yadā tiṣṭati/   \E %!
%sūryanāḍīmadhye       pavanaḥ pūrṇo yadā tiṣṭati    \P
%sūryanāḍīmadhye       pavanapūrṇo   yadāti/         \L
%sarvasūryanāḍīmadhye  pavanapūrṇo   yadāti/         \B
%sūryanāḍīmadhye       pavanaḥ pūrṇo yadā tiṣṭhati/  \N1
%sūryanāḍīmadhye       pavanaḥ pūrṇo yadā tiṣṭhati   \D
%sūryanāḍīmadhye       pvanaḥ  pūrṇo yadā tiṣṭhati/  \N2
%sūryanāḍīmadhye       pavanaḥ pūrṇo yadā tiṣṭhati/  \U1
%sūryanāḍīmadhye       pavanaḥ sūryo yadā tiṣṭhati// \U2
%------------------------------
%When the full breath abides in the middle of the sun-channel, ... 
%------------------------------
 \app{\lem[wit={ceteri}]{sūryanāḍīmadhye}
   \rdg[wit={B}]{sarvasūryanāḍīmadhye}}
 \app{\lem[wit={ceteri}]{pavanaḥ pūrṇo}
   \rdg[wit={L,B}]{pavanapūrṇo}
   \rdg[wit={N2}]{pvanaḥ pūrṇo}}
 \app{\lem[wit={ceteri}]{yadā tiṣṭhati}
   \rdg[wit={L,B}]{yadāti}}/
%------------------------------
%tadā mano  niścalaṃ bhavati/  \E
%tadā mano  niścalo  bhavati   \P
%tadā mano  niścalo  bhavati/  \L
%tadā mano  niścalo  bhavatī// \B
%tadā manaḥ niścalaṃ bhavati/  \N1
%tadā manaḥ niścalaṃ bhavati   \D
%tadā manaḥ niścalaṃ bhavati   \N2
%tadā manaḥ niścalaṃ bhavati   \U1
%tadā mano  niścalaṃ bhavati// \U2
%------------------------------
%Then the mind is unmovable. 
%------------------------------
 tadā
 \app{\lem[wit={ceteri}]{mano}
   \rdg[wit={N1,N2,D,U1}]{manaḥ}}
\app{\lem[wit={ceteri}]{niścalaṃ}
  \rdg[wit={P,L,B}]{niścalo}}
bhavati/
%------------------------------
%manaso  niścalatvena ānandarūpaṃ      pratyakṣaṃ bhāsate/  \E
%manaso  niścalatve   ānandaṃ svarūpa--pratyakṣaṃ bhāsate   \P %%%%7640.jpg
%manaso  niścalatve   ānandaṃ svarūpaṃ pratyakṣaṃ bhāsate/  \L
%manaso  niścalatve   ānaṃdaṃ svarūpaṃ pratyakṣaṃ bhāsate// \B
%manasaḥ niścalatve   ānaṃdasvarūpaṃ   pratyakṣaṃ bhāsate/  \N1
%manasaḥ niścalatve   ānaṃdasvarūpaṃ   pratyakṣaṃ bhāsate/  \D
%manasaḥ niścalatve   ānaṃdasvarūpaṃ   pratyakṣaṃ bhāṣate/  \N2
%manasaḥ niścalatve   ānaṃdasvarūpaṃ   pratyakṣaṃ bhāṣate/  \U1 %%%273.jpg
%manaso  niścalatve   ānaṃdasvarūpaṃ   pratyakṣaṃ bhāsate// \U2
%------------------------------
%The form of bliss immediately shines through the motionless mind.  
%------------------------------
\app{\lem[wit={ceteri}]{manaso}
  \rdg[wit={N1,N1,D,U1}]{manasaḥ}}
\app{\lem[wit={ceteri}]{niścalatve}
  \rdg[wit={E}]{niścalatvena}}
\app{\lem[wit={ceteri}]{ānandasvarūpaṃ}
  \rdg[wit={L,B}]{ānaṃdaṃ svarūpaṃ}
  \rdg[wit={P}]{ānandaṃ svarūpa°}
  \rdg[wit={E}]{ānandarūpaṃ}}
pratyakṣaṃ
\app{\lem[wit={ceteri}]{bhāsate}
  \rdg[wit={N2,U1}]{bhāṣate}}/
%------------------------------
%haṭhayogakāraṇāt  manaḥ   śūnyamadhye līnaṃ   bhavati/  kālaḥ samīpe   nāgacchati/  \E
%haṭhayogakāraṇāt  manaḥ   śūnyamadhye līnaṃ   bhavati   kālaḥ samīpe   nāgacchati   \P %%%%7640.jpg
%haṭhayogakāraṇāt  manaḥ   śūnyamadhye līnaṃ   bhavati/  kālaḥ samīpe   nāgacchati// \L
%haṭayogākāraṇāt   manaḥ// śūnyamadhye līnaṃ   bhavatī/  kālāsamīpe nāma gacchati//  \B
%haṭhayogakaraṇāt  manaḥ   śūnyamadhye līnaṃ   bhavati/  kālaḥ samīpe   nāgachati//  \N1
%haṭhayogakaraṇāt  manaḥ   śūnyamadhye līnaṃ   bhavati// kālaḥ samīpe   nāgachaṃti// \D
%haṭhayogakaraṇāt  mana----śūnyamadhye līnaṃ   bhavati/  kālasamīpe     nāgachati//  \N2
%haṭhayogakaraṇāt/ manaḥ   śūnyamadhye līnaṃ   bhavati/  kālasamīpe ti  nāgachati    \U1 %%%273.jpg
%haṭhayogakaraṇāt  manaḥ   śūnyamadhye sthānaṃ bhavati// kāsaḥ samīpe   nāgachati//  \U2
%------------------------------
%Due to the execution of haṭhayoga the mind becomes absorbed into emptiness. The time of death does not approach.
%------------------------------
\app{\lem[wit={ceteri}, alt={haṭha°}]{haṭha}
  \rdg[wit={B}]{haṭa}
}\app{\lem[wit={ceteri},alt={yoga°}]{yoga}
  \rdg[wit={B}]{yogā°}
}\app{\lem[wit={ceteri}]{karaṇāt}
  \rdg[wit={E,P,L,B}]{kāraṇāt}}
\app{\lem[wit={ceteri}]{manaḥ}
  \rdg[wit={N2}]{mana}}
śūnyamadhye
\app{\lem[wit={ceteri}]{līnaṃ}
  \rdg[wit={U2}]{sthānaṃ}}
bhavati/
\app{\lem[wit={ceteri}]{kālaḥ}
  \rdg[wit={B}]{kālā°}
  \rdg[wit={N2,U1}]{kāla°}
  \rdg[wit={U2}]{kāsaḥ}}
samīpe
\app{\lem[wit={ceteri}]{nāgacchati}
  \rdg[wit={B}]{nāma gacchati}
  \rdg[wit={D}]{nāgachaṃti}
  \rdg[wit={U1}]{ti nāgachati}}\dd{}
%------------------------------
%idānīṃ haṭhayogasya dvitīyo  bhedaḥ kathyate/   \E
%idānīṃ haṭhayoga----dvitīya--bhedaḥ kathyate    \P
%idānīṃ haṭhayogasya dvitīya--bhedāḥ kathyante/  \L
%idānīṃ haṭayogasya  dvitīyaṃ bhedāḥ kathyaṃte// \B
%idānīṃ haṭhayogasya dvitīyo  bhedaḥ kathyate//  \N1
%idānīṃ haṭhayogasya dvitīya--bhedaḥ kathyate    \D
%idānīṃ haṭayogasya  dvitīyo  bhedaḥ kathyate    \U1
%idānīṃ haṭhayogasya dvitīyo  bhedaḥ kathyate//  \U2 
%------------------------------
%Now, the second division of haṭhayoga is explained.
%------------------------------
\app{\lem[wit={ceteri}]{haṭhayogasya}
  \rdg[wit={B,U1}]{haṭayogasya}
  \rdg[wit={P}]{haṭhayoga°}}
\app{\lem[wit={ceteri}]{dvitīyo}
  \rdg[wit={P,L,D}]{dvitīya°}
  \rdg[wit={B}]{dvitīyaṃ}}
\app{\lem[wit={ceteri}]{bhedaḥ}
  \rdg[wit={L,B}]{bhedāḥ}}
\app{\lem[wit={ceteri}]{kathyate}
  \rdg[wit={L,B}]{kathyante}}/ \note[type=source, labelb=132, lem={dvitīyo bhedaḥ}]{Ysv (PT): idānīṃ haṭhayogasya dvitīyaṃ bhedam acchṛṇu | ākāśe nāsikāgre tu sūryakoṭisamaṃ smaret | śvetaṃ raktaṃ tathā pītaṃ kṛṣṇamityādirūpataḥ | evaṃ dhyātvā cirāyuḥ syād aṅgājananavarjitam | śivatulyo mahātmāsau haṭhayogaprasādataḥ | haṭhāj jyotir mayo bhūtvā hyantareṇa śivo bhavet | ato 'yaṃ haṭhayogaḥ syāt siddhidaḥ siddhasevitaḥ |}
idānīṃ
%------------------------------
%pādādārabhya śiraḥ paryaṃtaṃ    svaśarīre  koṭisūryatejaḥ   samānaṃ śvetaṃ pītaṃ       raktaṃ kiṃcidvarṇaṃ ciṃtyate/  \E
%pādādārabhya śiraḥ paryaṃtaṃ    svaśarīre  koṭisūryatejaḥ   samānaṃ śvetaṃ pītaṃ nīlaṃ raktaṃ kiṃdrupaṃ    cityate    \P
%pādādārabhya śira--paryaṃtaṃ    svaśarīre  koṭisūryatejaḥ   samānaśvetaṃ nīlaṃ         raktaṃ tiṃdrupaṃ    ciṃtate/   \L
%pādādārabhya śira--paryaṃtaṃ    svaśarīre  koṭisūryatejaḥ// samānaśvetanīlaṃ           raktaṃ kiṃdrupaṃ    ciṃtate//  \B
%pādādārabhyā śiraḥ paryentaṃ    svaśarīre  koṭisūryatejaḥ   samānaṃ śvetaṃ pītaṃ nīlaṃ laktaṃ kiṃcidrūpaṃ  ciṃtyate   \N1 
%pādādārabhyā śiraḥ paryaṃtaṃ    svaśarīre  koṭisūryatejaḥ   samānaṃ śvetaṃ pītaṃ nīlaṃ raktaṃ kiṃcidrūpaṃ  ciṃtyate   \D
%pādādārabhya śiraḥ pariyataṃ    svaśarīraṃ koṭisūryatejaḥ   samānaṃ śvetaṃ pītaṃ nīlaṃ raktaṃ ciṃrūpaṃ     ciṃtyate   \U1
%pādādārabhya śiro  paryaṃtaṃ    svaśarīre  koṭisūryye tejaḥ samānaṃ śvetaṃ pītaṃ nīlaṃ raktaṃ kiṃcidrūpaṃ  ciṃtyate// \U2
%------------------------------
%The shine of ten million suns in one's own body beginning from the feet to the top of head is contemplated in any color equal to white, yellow [or] red.
%------------------------------
\app{\lem[wit={ceteri}]{pādādārabhya}
  \rdg[wit={N1,D}]{pādādārabhyā}}
\app{\lem[wit={ceteri}]{śiraḥ}
  \rdg[wit={L,B}]{śira°}
  \rdg[wit={U2}]{śiro}}
\app{\lem[wit={ceteri}]{paryantaṃ}
  \rdg[wit={N1}]{paryentaṃ}
  \rdg[wit={U1}]{pariyataṃ}}
\app{\lem[wit={ceteri}]{svaśarīre}
  \rdg[wit={U1}]{svaśarīraṃ}}
\app{\lem[wit={ceteri}]{koṭisūryatejaḥ}
  \rdg[wit={U2}]{koṭisūryye tejaḥ}}
\app{\lem[wit={ceteri}]{samānaṃ}
  \rdg[wit={L,B}]{samāna°}
  \rdg[wit={ceteri}]{śvetaṃ}
  \rdg[wit={B}]{śveta°}}
\app{\lem[wit={ceteri}]{pītaṃ}
  \rdg[wit={L,B}]{\om}}
nīlaṃ
\app{\lem[wit={ceteri}]{raktaṃ}
  \rdg[wit={N1}]{laktaṃ}}
\app{\lem[wit={N1,D,U2}]{kiṃcidrūpaṃ}
  \rdg[wit={P,B}]{kiṃdrupaṃ}
  \rdg[wit={L}]{tiṃdrupaṃ}
  \rdg[wit={U1}]{ciṃrūpaṃ}
  \rdg[wit={E}]{kiṃcidvarṇaṃ}}
\app{\lem[wit={ceteri}]{cintyate}
  \rdg[wit={P}]{cityate}
  \rdg[wit={L,B}]{ciṃtate}}/
%------------------------------
%ttad  dhyānakāraṇāt     sakalaṃ   rogajvalanaṃ     bhavati/                      āyur          vardhate/          \E
%tad   dhyānakāraṇāt     sakalāṃge rogajvalanaṃ  na bhavati                       āyur vṛddhir  bhavati   \P
%tad   dhyānakāraṇāt     sakalaṃge rogajvalanaṃ  na bhavati/                      āyur          vardhate/          \L
%tat   dhyānakāraṇāt     sakalaṃge rogajvalanaṃ  na bhavati/                      āyur vṛddhir  bhavatī/  \B
%na    dhyānaṃ kāraṇāt/  sakalāṃge roga          na bhavati/  jvalanaṃ na bhavati āyur vṛddhir  bhavati/  \N1
%ta    dhyānaṃ karaṇāt// sakalāṃge rogajvalanaṃ  na bhavati//                                             \D
%tad---dhyānaṃ karaṇāt / sakalāṃge roga          na bhavati   jvaranaṃ na bhavati āyu--vṛddhir  bhavati// \N2
%ta    dhyānaṃ karaṇāt   sakalāṃge roga kṣataṃ?  na bhavati                       āyur vṛddhir  bhavati   \U1
%tat   dhyānakāraṇāt     sakalāṃge rogajvalanaṃ     bhavati//                     āyur vṛddhir  bhavati// \U2
%------------------------------
%aDue to the execution of meditation in the entire body disease does'nt arise, fever doesn't arise and vitality grows.  
%------------------------------
\app{\lem[wit={E,P,L,N2},alt={tad}]{ta\skp{d-dhyā}}
  \rdg[wit={B,U2}]{tat}
  \rdg[wit={D,U1}]{ta}
  \rdg[wit={N1}]{na}
}\app{\lem[wit={ceteri},alt={dhyānakāraṇāt}]{\skm{d-dhyā}nakāraṇāt}
  \rdg[wit={N1,N2,D,U1}]{dhyānaṃ karaṇāt}}
\app{\lem[wit={P,N1,D,N2,U1,U2}]{sakalāṅge}
  \rdg[wit={L,B}]{sakalaṃge}
  \rdg[wit={E}]{sakalaṃ}}
\app{\lem[type=emendation, resp=egoscr]{rogaḥ}
\rdg[wit={N1,N2}]{\korr roga}
\rdg[wit={E,P,L,B,D,U2}]{rogajvalanaṃ}
\rdg[wit={U1}]{roga kṣataṃ}}
\app{\lem[wit={ceteri}]{na}
  \rdg[wit={E,U2}]{\om}}
bhavati/
\end{prose}
\end{ekdosis}
\ekdpb*{}
%%%%%%%%%%%%%%%%%%%%%%%%%%%%%%%%%%%%%%%%%%
%%%%%%%%%%%%%%%%%%%%%%%%%%%%%%%%%%%%%%%%%%
%%%%%%%%PAGEBREAK%%%%%%%PAGEBREAK%%%%%%%%%
%%%%%%%%%%%%%%%%%%%%%%%%%%%%%%%%%%%%%%%%%%
%%%%%%%%%%%%%%%%PAGEBREAK%%%%%%%%%%%%%%%%%
%%%%%%%%%%%%%%%%%%%%%%%%%%%%%%%%%%%%%%%%%%
%%%%%%%%PAGEBREAK%%%%%%%PAGEBREAK%%%%%%%%%
%%%%%%%%%%%%%%%%%%%%%%%%%%%%%%%%%%%%%%%%%%
%%%%%%%%%%%%%%%%%%%%%%%%%%%%%%%%%%%%%%%%%%
%%%%%%%%%%%%%%%%%%%%%%%%%%%%%%%%%%%%%%%%%%
%%%%%%%%%%%%%%%%%%%%%%%%%%%%%%%%%%%%%%%%%%
%%%%%%%%PAGEBREAK%%%%%%%PAGEBREAK%%%%%%%%%
%%%%%%%%%%%%%%%%%%%%%%%%%%%%%%%%%%%%%%%%%%
%%%%%%%%%%%%%%%%PAGEBREAK%%%%%%%%%%%%%%%%%
%%%%%%%%%%%%%%%%%%%%%%%%%%%%%%%%%%%%%%%%%%
%%%%%%%%PAGEBREAK%%%%%%%PAGEBREAK%%%%%%%%%
%%%%%%%%%%%%%%%%%%%%%%%%%%%%%%%%%%%%%%%%%%
%%%%%%%%%%%%%%%%%%%%%%%%%%%%%%%%%%%%%%%%%%
%%%%%%%%%%%%%%%%%%%%%%%%%%%%%%%%%%%%%%%%%%
%%%%%%%%%%%%%%%%%%%%%%%%%%%%%%%%%%%%%%%%%%
%%%%%%%%PAGEBREAK%%%%%%%PAGEBREAK%%%%%%%%%
%%%%%%%%%%%%%%%%%%%%%%%%%%%%%%%%%%%%%%%%%%
%%%%%%%%%%%%%%%%PAGEBREAK%%%%%%%%%%%%%%%%%
%%%%%%%%%%%%%%%%%%%%%%%%%%%%%%%%%%%%%%%%%%
%%%%%%%%PAGEBREAK%%%%%%%PAGEBREAK%%%%%%%%%
%%%%%%%%%%%%%%%%%%%%%%%%%%%%%%%%%%%%%%%%%%
%%%%%%%%%%%%%%%%%%%%%%%%%%%%%%%%%%%%%%%%%%
\begin{ekdosis}
  \ekddiv{type=ed}
  \begin{prose}
\app{\lem[wit={N2}]{jvaranaṃ na bhavati}
  \rdg[wit={N1}]{jvalanaṃ na bhavati}
  \rdg[wit={ceteri}]{\om}}/
\app{\lem[wit={ceteri}, alt={āyur}]{āyu\skp{r-vṛ}}
  \rdg[wit={N2}]{āyu°}
  \rdg[wit={D}]{\om}
}\app{\lem[wit={ceteri},alt={vṛddhir}]{\skm{r-vṛ}ddhi\skp{r-bha}}
  \rdg[wit={E,L,D}]{\om}
}\app{\lem[wit={ceteri},alt={bhavati}]{\skm{r-bha}vati}
  \rdg[wit={B}]{bhavatī}
  \rdg[wit={E,L}]{vardhate}
  \rdg[wit={D}]{\om}}\dd{}
\end{prose}
\end{ekdosis}
 %%%%%%%%%%%%%%%%%%%%%%%%%%%%%
 %%%%%%%%%%%%%%%%%%%%%%%%%%%%%
 %%%%%%%%%%%%%%%%%%%%%%%%%%%%%
 %%%%%%%%%%%%%%%%%%%%%%%%%%%%%
 %%%%%%%%%%%%%%%%%%%%%%%%%%%%%
\begin{ekdosis}
  \ekddiv{type=ed}
          \bigskip
    \centerline{\textrm{\small{[Jñānayoga]}}}
          \bigskip
          \begin{prose}
%------------------------------
%idānīṃ jñānayogasya lakṣaṇaṃ kathyate/ \E
%idānīṃ jñānayogasya lakṣaṇaṃ kathyate \P
%idānīṃ jñānayogasya lakṣaṇaṃ// \L 5976_0011.jpg 
%idānīṃ jñānayogasya lakṣaṇaṃ// \B
%idānīṃ jñānayogasya lakṣaṇaṃ// \N1 %%%%p.6 verso 
%idānīṃ jñānayogasya lakṣaṇaṃ// \D
%idānīṃ jñānayogasya lakṣaṇaṃ kathyate// \N2
%idānī  jñānayogasya lakṣaṇaṃ kathyate   \U1
%idānīṃ jñānayogasya lakṣaṇaṃ kathyate// \U2
%------------------------------
%Now the characteristic of jñānayoga is explained. 
%-----------------------------
\note[type=source, labelb=133, lem={jñānayogasya}]{Ysv (PT): idānīṃ jñānayogasya lakṣaṇaṃ kathyate śive | yaj jñātvā jñānasampūrṇaḥ śivaḥ syān na punarbhavaḥ |}
\app{\lem[wit={ceteri}]{idānīṃ}
  \rdg[wit={U1}]{idānī}}
jñānayogasya lakṣaṇaṃ
\app{\lem[wit={E,P,N2,U1,U2}]{kathyate}
  \rdg[wit={L,B,N1,D}]{\om}}/
\end{prose}
%--------------------------------------
%ekam eva jagat paśyed viśvāvasu vibhāsvaram/
%avikalpatayā yuktyā jñānayogaṃ samācaret//1// \E
%
%ekam eva cayat paśyed viśvātmāsu vibhāsvaram       
%avikalpatayā yuktyā jñānayogaṃ samācaret 1 \P
%
%ekam evā jagat paśyed viśvātmāsu vibhāsvaraṃ//
%avikalpatayā yuktā jñānayogaṃ samācaret// \L
%
%ekam evā jagat paśyad visvātmāsu vibhāsvaraṃ//
%avikalpatayā yuktā jñānayogaṃ samācaret// \B
%
%ekam eva jagat paśyed viśvātmā viśvabhāvanaḥ/
%iti kṛtvā tu vai yukto jñānayogaṃ samācaret// SVARODAYA
%
%ekam eva jagat paśyed dviśvātmāsu vibhāsvaraṃ/
%avikalpatayā yuktyā jñānayogaṃ samācaret//1// \N1
%
%ekam eva jagat paśyed dviśvātmāsu vibhāsvaraṃ//
%avikalpatayā yuktyā jñānayogaṃ samācaret//1// \D
%
%ekam eva jagat paśyed dviśvātmāsu vibhāsvaraṃ//
%avikalpatayā yuktyā jñānayogaṃ samācaret//1// \N2
%
%ekam eva jagataḥ paśyed dviśvātmāsu vibhāsvaraṃ
%āvikalpatayā yuktyā jñānayogaṃ samācaret//1// \U1
%
%ekam eva jagataḥ paśyed dviśvātmāsu vibhāsvaraṃ
%āvikalpatayā yuktyā jñānayogaṃ samācaret// \U2
%------------------------------
%He shall see the world truly as being one, shining in all selves. 
%By applying indistinctness he shall accomplish \textit{jñānayoga}.   
%------------------------------
\begin{tlg}
  \tl{\note[type=source, labelb=134, lem={ekam eva}]{Ysv (PT): ekam eva jagat paśyed viśvātmā viśvabhāvanaḥ | iti kṛtvā tu vai yukto jñānayogaṃ samācaret ||}
eka\skp{m-e}\app{\lem[wit={ceteri}, alt={eva}]{\skm{m-e}va}
  \rdg[wit={L,B}]{evā}}
\app{\lem[wit={ceteri},alt={jagat}]{jaga\skp{t-pa}}
  \rdg[wit={P}]{cayat}
}\app{\lem[wit={ceteri},alt={paśyed}]{\skm{t-pa}śye\skp{d-vi}}
  \rdg[wit={B}]{paśyad}
}\app{\lem[wit={P,L,B},alt={viśvātmāsu}]{\skm{d-vi}śvātmāsu}
  \rdg[wit={E}]{viśvāvasu}
  \rdg[wit={N1,D,N2,U1,U2}]{dviśvātmāsu}}
vibhāsvaraṃ/}\\
\tl{\app{\lem[wit={ceteri}]{avikalpatayā}
  \rdg[wit={U1,U2}]{āvikalpatayā}}
\app{\lem[wit={ceteri}]{yuktyā}
  \rdg[wit={L,B}]{yuktā}} 
jñānayogaṃ samācaret\dd{}1\hskip-2pt\dd{}}
\end{tlg}
%------------------------------
%yatra yatra sthito vāpi sarvajñānamayaṃ jagat/ 
%sa evaṃ vetti bodhena so pi jñānādhikāraṇāt//2// \E 
%
%yatra yatra sthito vāpi sarvajñānamayaṃ jagat  
%ya evaṃ vetti bodhena so pi jñānādhikāravān \P
%
%yatra yatra sthito vāpi sarvajñānamayaṃ jagat//  
%ya evaṃ vetti bodhena so pi jñānādhikāravān// \L
%
%yatra yatra sthito vāpi sarvajñānamayaṃ jagat//  
%ya evaṃ ve bodhena so pi jñānādhikāravān// \B
%
%yatra tatra sthito vāpi sarvajñānamayaṃ jagat/
%ya evam asti bodhena so'pi jñānādhikāravān/ \SVARODAYA
%
%yatra yatra sthito vāpi sarvajñānamayaṃ jagat/
%ya evaṃ vetti bodhena so pi jñānādhikāravān//2//\N1
%
%yatra yatra sthito vāpi sarvajñānamayaṃ jagat//
%ya evaṃ vetti bodhena so pi jñānādhikāravān//2//\D
%
%yatra yatra sthito vāpi sarvajñānamayaṃ jagat//
%ya evaṃ vetti bodhena so pi jñānādhikāravān//2//\N2
%
%yatra yatra sthito vāpi sarvajñānamayaṃ jagat  %%%273.jpg
%evaṃ vette na bodhena so pi jñānādhikāravān 2    \U1
%
%yatra yatra sthito hiṃsa sarvajñānamayaṃ jagat//  
%evaṃ vetti bodhena so pi jñānādhikāravān// 2    \U2
%------------------------------
%Wherever the world is established or made of omniscience,
%who knows thus by means of insight, he is a like an expert of knowledge.      
%------------------------------
\begin{tlg}
  \tl{\note[type=testium, labelb=135, lem={yatra yatra}]{Ysv (PT): yatra tatra sthito vāpi sarvajñānamayaṃ jagat | ya evam asti bodhena so'pi jñānādhikāravān ||}
    yatra tatra sthito \app{\lem[wit={ceteri}]{vāpi}
      \rdg[wit={U2}]{hiṃsa°}} sarvajñānamayaṃ jagat/}\\
  \tl{\app{\lem[wit={ceteri}]{ya evaṃ}
      \rdg[wit={U1,U2}]{evaṃ}}
    \app{\lem[wit={ceteri}]{vetti}
      \rdg[wit={U1}]{vette na}
      \rdg[wit={B}]{ve}} bodhena so pi
    \app{\lem[wit={ceteri}]{jñānādhikāravān}
      \rdg[wit={E}]{jñānādhikāraṇāt}}\dd{}2\hskip-2pt\dd{}}
\end{tlg}
%------------------------------
%
%\om!!!!!                                                                                                        \E
%
%prāpnoti śāmbhavīmantrān  sadā nityaparāyaṇaḥ/   yathā nyagrodhavījaṃ hi kṣitau   vaptur drumāyate/               \SVARODAYA  
%prāpnoti śāmbhavīṃ sattāṃ sadāṃdvaitaparāyaṇaḥ   yathā nyagrodhabījaṃ hi kṣitāv   uptaṃ drumāyate likāṃ pa..vāḥ 4 \P  7640.jpg last line check word!!!
%prāpnoti śāmbhavīṃ sattān sadādvaitaparāyaṇaḥ//  yathā nyagrodhavīja  hi kṣitāv   utpadyate yathā//               \L
%prāpnoti śāmbhaviṃ sattāṃ sadādvaitaparāyaṇaḥ//  yathā nyagrodhabījāṃ hi kṣitī    utpadyate//                      \B
%prāpnoti sāṃbhavīṃ satta  sadādvaitaparāyaṇaḥ//  yathā nyagrodhavījaṃ hi kṣitāv   uptaṃ drumāyate 3//              \N1
%prāpnoti sāṃbhavīsattāṃ   sadādvaitaparāyaṇaḥ//  yathā nyagrodhavījaṃ hi kṣitāv   uptaṃ drumāyate//                \D
%prāpnoti sāṃbhavīsattā    sadādvaitaparāyaṇaḥ//  yathā nyagrodhavījaṃ hi kṣitāv   uptaṃ drumāyate//                \N2 %drumaayate=denom. wie ein beim  sein 
%prāpnoti sāṃbhavīsattāṃ   sadādvaitaparāyaṇaḥ    yathā nyagrodhabījaṃ hi kṣitāptā ukta drumāyate 3              \U1
%prāpnoti sāṃbhavīsattāṃ   yadādvaitaparāyaṇaḥ//  yathā nyagrodhabījaṃ hi kṣitāv   uptaṃ drumāyate//               \U2
%------------------------------
%He always attains the reality of śāmbhavī - the goal of eternal non-duality.  
%Just as the seed of the Nyagrodha scattered onto the soil [always] becomes a tree.
%------------------------------
\begin{tlg}
  \tl{\note[type=source, labelb=136, lem={prāpnoti}]{Ysv (PT): prāpnoti śāmbhavīmantrān sadā nityaparāyaṇaḥ | yathā nyagrodhavījaṃ hi kṣitau vaptur drumāyate ||}
    \app{\lem[wit={ceteri}]{prāpnoti}
      \rdg[wit={E}]{\om}}
    \app{\lem[wit={D,U1,U2}]{sāṃbhavīsattāṃ}
      \rdg[wit={P,B}]{śāmbhavīṃ sattāṃ}
      \rdg[wit={L}]{śāmbhavīṃ sattān}
      \rdg[wit={N1}]{sāṃbhavīṃ satta}
      \rdg[wit={N2}]{sāṃbhavīsattā}
      \rdg[wit={E}]{\om}}
    \app{\lem[wit={ceteri}]{sadādvaitaparāyaṇaḥ}
      \rdg[wit={U1}]{sadāṃdvaita°}
      \rdg[wit={E}]{\om}}/}\\
  \tl{\app{\lem[wit={ceteri}]{yathā}
      \rdg[wit={E}]{\om}}
    \app{\lem[wit={ceteri}]{nyagrodhabījaṃ}
      \rdg[wit={N1,N2,D}]{°vījaṃ}
      \rdg[wit={L}]{°vīja}
      \rdg[wit={E}]{\om}}
    \app{\lem[wit={ceteri}]{hi}
      \rdg[wit={E}]{\om}}
    \app{\lem[wit={ceteri},alt={kṣitāv}]{kṣitā\skp{v-u}}
      \rdg[wit={B}]{kṣitī}
      \rdg[wit={U1}]{kṣitāptā}
      \rdg[wit={E}]{\om}
 }\app{\lem[wit={ceteri},alt={uptaṃ drumāyate}]{\skm{v-u}ptaṃ drumāyate}
      \rdg[wit={P}]{uptaṃ drumāyate likāṃ pa..vāḥ}
      \rdg[wit={L}]{utpadyate yathā}
      \rdg[wit={B}]{utpadyate}
      \rdg[wit={U1}]{ukta drumāyate}
      \rdg[wit={E}]{\om}}\dd{}3\hskip-2pt\dd{}}
\end{tlg}
%------------------------------
%ekāntaṃ  naikadā  svena   dṛśyate  daśadhā  kṛtaḥ/  mūlāṅkurasya  coddaṇḍāḥ śākhākuṇḍalapallavāḥ//3//   \E    cod?von cud? Wurzel in guṇa + daṇḍa? !!! em. zu śaśvadhā = immer wieder, jederzeit 
% \om                                                                                                    \P
%ekāṃte   nekadhā  svena   dṛśyaṃte daśadhāt kṛp?tā/ mūlāṃkurutva kudaṃḍaḥ  śākhākilekālapallavā        \B
%ekāṃte   nekadhā  svena   dṛśyaṃte daśadhāt kṛtaḥ/  mūlāṃkurutva kudaṃḍa   śākhākalikālapallavā        \L
%ekāṃtaṃ  naikadhā śveta   dṛśyate  daśadhā  kṛtā//  mūlāṃkurutva codaṃḍaḥ  śāvārakumbhalapallavaḥ//4// \N1   
%ekāṃtaṃ  naikadhā śvetana dṛśyate  daśadhā  kṛtā//  mūlāṃkurutva codarāṭaḥ śālavākumapadṛtravā//4//    \D
%ekāṃtaṃ  naikadhā śvetana dṛśyet   śadhā    kṛtā//  mūlāṃkurutva codarāṭaḥ śākhākumbhalapallavā//4//   \N2
%yekāṃtaṃ naikadhā svena   dṛśyate  śadhā    kṛtā    mūlāṃkurutva codaṃḍa   śākhākumbhalapallavaḥ       \U1
%ekāṃtaṃ  naikadhā svetana dṛśyate  daśadhā  kṛtiḥ// mūlāṃkurutva codaṃḍaḥ  śākhākusumapallavāḥ//       \U2
%------------------------------
%Nur eines, nicht zusammen mit dem Ich wird das zehnfach gemachte gesehen. 
%Die aufgerollten Sprossen der Äste, welche die austreibendem Stöcke sind vom Spross der Wurzel. 
%------------------------------
%Die absoluten Einheit (ekāntaṃ), wird als multibel (nämlich) aus zehn Teilen gemacht von einen selbst gesehen. !!!!!
%Die aufgerollen Sprösslinge der Zweige sind austreibende Stengel des Wurzeltriebes.  
%------------------------------
%The absolute unity (ekāntaṃ), is seen as a multiple (namely) made up of ten parts by oneself.
%The rolled up shoots of the branches are the sprouting stalks of the root shoot.  
%------------------------------
%The absolute unity (ekāntaṃ), is seen as manifoldly created again and again by oneself.
%The rolled up shoots of the branches are the sprouting stalks of the root shoot.  
%------------------------------
\note[type=source, labelb=137, lem={naikadhā}]{Ysv (PT): ādāv ekas tato 'nekaḥ svabhāvāc chādanādibhiḥ | varddhate'harniśaṃ vṛkṣaḥ patrapallavavistṛtaḥ|}
\begin{tlg}
  \tl{\app{\lem[wit={ceteri}]{ekāntaṃ}
  \rdg[wit={B,L}]{ekānte}
  \rdg[wit={U1}]{yekāṃtaṃ}
  \rdg[wit={P}]{\om}}
\app{\lem[wit={ceteri}]{naikadhā}
  \rdg[wit={E}]{naikadā}
  \rdg[wit={B,L}]{nekadhā}
  \rdg[wit={P}]{\om}}
\app{\lem[wit={ceteri}]{svena}
  \rdg[wit={N1}]{śveta}
  \rdg[wit={D,N1}]{śvetana}
  \rdg[wit={P}]{\om}}
\app{\lem[wit={ceteri}]{dṛśyate}
  \rdg[wit={B,L}]{dṛśyaṃte}
  \rdg[wit={N2}]{dṛśyet}
  \rdg[wit={P}]{\om}}
\app{\lem[wit={E,N1,N2}]{daśadhā}    %%[type=conjecture, resp=egoscr]{śaśvadhā}????
  \rdg[wit={B,L}]{daśadhāt}
  \rdg[wit={N2,U1}]{śadhā}
  \rdg[wit={P}]{\om}}
\app{\lem[type=emendation, resp=egoscr]{kṛtaṃ}
  \rdg[wit={E,L}]{\korr kṛtaḥ}
  \rdg[wit={N1,N2,D,U1}]{kṛtā}
  \rdg[wit={B}]{kṛptā}
  \rdg[wit={U2}]{kṛtiḥ}
  \rdg[wit={P}]{\om}}/}\\
 \tl{\app{\lem[wit={E}]{mūlāṅkurasya}
  \rdg[wit={ceteri}]{mūlāṃkurutva}
  \rdg[wit={P}]{\om}}
\app{\lem[wit={E,N1,U2}]{coddaṇḍāḥ}
  \rdg[wit={D,N2}]{codarāṭaḥ}
  \rdg[wit={B}]{kudaṃjaḥ}
  \rdg[wit={L}]{kudaṃḍa}
  \rdg[wit={P}]{\om}}
\app{\lem[wit={E}]{śākhākuṇḍalapallavāḥ}
  \rdg[wit={B,L}]{śākhākilekālapallavā}
  \rdg[wit={N1,U1}]{śāvārakumbhalapallavaḥ}
  \rdg[wit={N2}]{śākhākumbhalapallavā}
  \rdg[wit={D}]{śālavākumapadṛtravā}
  \rdg[wit={U2}]{śākhākusumapallavāḥ}
  \rdg[wit={P}]{\om}}\dd{}4\hskip-2pt\dd{}}
\end{tlg}
\end{ekdosis}
\ekdpb*{}
%%%%%%%%%%%%%%%%%%%%%%%%%%%%%%%%%%%%%%%%%%
%%%%%%%%%%%%%%%%%%%%%%%%%%%%%%%%%%%%%%%%%%
%%%%%%%%PAGEBREAK%%%%%%%PAGEBREAK%%%%%%%%%
%%%%%%%%%%%%%%%%%%%%%%%%%%%%%%%%%%%%%%%%%%
%%%%%%%%%%%%%%%%PAGEBREAK%%%%%%%%%%%%%%%%%
%%%%%%%%%%%%%%%%%%%%%%%%%%%%%%%%%%%%%%%%%%
%%%%%%%%PAGEBREAK%%%%%%%PAGEBREAK%%%%%%%%%
%%%%%%%%%%%%%%%%%%%%%%%%%%%%%%%%%%%%%%%%%%
%%%%%%%%%%%%%%%%%%%%%%%%%%%%%%%%%%%%%%%%%%
%%%%%%%%%%%%%%%%%%%%%%%%%%%%%%%%%%%%%%%%%%
%%%%%%%%%%%%%%%%%%%%%%%%%%%%%%%%%%%%%%%%%%
%%%%%%%%PAGEBREAK%%%%%%%PAGEBREAK%%%%%%%%%
%%%%%%%%%%%%%%%%%%%%%%%%%%%%%%%%%%%%%%%%%%
%%%%%%%%%%%%%%%%PAGEBREAK%%%%%%%%%%%%%%%%%
%%%%%%%%%%%%%%%%%%%%%%%%%%%%%%%%%%%%%%%%%%
%%%%%%%%PAGEBREAK%%%%%%%PAGEBREAK%%%%%%%%%
%%%%%%%%%%%%%%%%%%%%%%%%%%%%%%%%%%%%%%%%%%
%%%%%%%%%%%%%%%%%%%%%%%%%%%%%%%%%%%%%%%%%%
%%%%%%%%%%%%%%%%%%%%%%%%%%%%%%%%%%%%%%%%%%
%%%%%%%%%%%%%%%%%%%%%%%%%%%%%%%%%%%%%%%%%%
%%%%%%%%PAGEBREAK%%%%%%%PAGEBREAK%%%%%%%%%
%%%%%%%%%%%%%%%%%%%%%%%%%%%%%%%%%%%%%%%%%%
%%%%%%%%%%%%%%%%PAGEBREAK%%%%%%%%%%%%%%%%%
%%%%%%%%%%%%%%%%%%%%%%%%%%%%%%%%%%%%%%%%%%
%%%%%%%%PAGEBREAK%%%%%%%PAGEBREAK%%%%%%%%%
%%%%%%%%%%%%%%%%%%%%%%%%%%%%%%%%%%%%%%%%%%
%%%%%%%%%%%%%%%%%%%%%%%%%%%%%%%%%%%%%%%%%%
\begin{ekdosis}
  \ekddiv{type=ed}
  \noindent
%------------------------------
%srehapuṇyaphalaṃ   bīje vistaro yaṃ svabhāvataḥ/  tathāsau   nirmalo  nityo nirvikāro niraṃjanaḥ//4// \E
%snehapuṣpaphalaṃ   bīje vistāro yaṃ svabhāvataḥ   tāthāpasau nirmalau nityo nirvikāro niraṃjanaḥ     \P   %%7641.jpg Z.1
%snehe puṣpaphala---bīja-vistāro ya  svabhāvatāḥ   yāthāsau   nirmalo  nityo nirvikāro niraṃjanaḥ//    \B
%snehe puṣpaphala---bīja-vistāro ya  svabhāvatāḥ// tāthāsau   nirmalo  nityo nirvikāro niraṃjanaḥ//    \L
%snehapuṣpaphalaṃ   bīje vistārā yaṃ svabhāvataḥ/  tathāsau   nirmalo  nityo nirvikāro niraṃjanaḥ//5// \N1
%snehapuṣpaphalaṃ   bīje vistārā yasya  bhāvataḥ// tathāsau   nirmalo  nityo nirvikāro niraṃjanaḥ//5// \D
%snehapuṣpaphalaṃ   vīje vistāro yaṃ svabhāvataḥ// tathāsau   nirmalo  nityo nirvikāro niraṃjanaḥ//5// \N2
%snehapuṣpaṃ phalaṃ bīje vistāro yaḥ svabhāvataḥ   tathāsau   nirmalo  nityo nirvikāro niraṃjanaḥ 5    \U1  %%%%274.jpg
%snehapuṣpaphalaṃ   bīje vistāro yaṃ svabhāvataḥ// tathāsau   nirmalo  nityo nirvikāro niraṃjanaḥ// 5  \U2 %%%first Śloka in this series that is numbered in U2 
%------------------------------
%Aufgrund seines inhärenten Wesens ist dieser Ast mit seinen Zweigen, welcher die Frucht der Blüte der Liebe ist, im Samen.
%Gewiss, ist jenes rein, ewig, unveränderlich und makellos. 
%------------------------------
%By virtue of its inherent nature, this branch with its branches, which is the fruit of the flower of love, is in the seed.
%Certainly, that is pure, eternal, unchanging and immaculate.
%------------------------------
\note[type=source, labelb=138, lem={sneha°}]{Ysv (PT): snehapuṣpaphalair vījair vistāro 'yaṃ svabhāvataḥ | tathāsau nirmalo nityo nirvikāro nirañjanaḥ |}
\begin{tlg}
  \tl{
\app{\lem[wit={P,N1,N2,D,U2}]{snehapuṣpaphalaṃ}
  \rdg[wit={B,L}]{snehe puṣpaphala°}
  \rdg[wit={U1}]{snehapuṣpaṃ phala}
  \rdg[wit={E}]{srehapuṇyaphalaṃ}}
\app{\lem[wit={ceteri}]{bīje}
  \rdg[wit={B,L}]{bīja}
  \rdg[wit={N2}]{vīje}}
\app{\lem[wit={ceteri}]{vistāro}
  \rdg[wit={N1,D}]{vistārā}}
\app{\lem[wit={E,P,N1,N2,U2}]{'yaṃ}
  \rdg[wit={B,L}]{ya}
  \rdg[wit={U1}]{yaḥ}
  \rdg[wit={D}]{yasya}}
\app{\lem[wit={ceteri}]{svabhāvataḥ}
  \rdg[wit={B,L}]{svabhāvatāḥ}
  \rdg[wit={D}]{bhāvataḥ}}/}\\
\tl{\app{\lem[wit={ceteri}]{tathāsau}
    \rdg[wit={B}]{yathāsau}
    \rdg[wit={P}]{tathāpasau}}
  \app{\lem[wit={ceteri}]{nirmalo}
    \rdg[wit={P}]{nirmalau}}
nityo nirvikāro niraṃjanaḥ\dd{}5\hskip-2pt\dd{}}
\end{tlg}
%------------------------------
%eko  nekaḥ  svayaṃbhūś ca dhāmnā ca    bahudhā sthitaḥ/   paṃcatattvamanobuddhi-māyāhaṃkāravikriyāḥ //5//   \E
%eko  nekaḥ  svayaṃbhūś ca svadhāmnā    bahudhā sthitāḥ    paṃcatatvamanobuddhir māyāhaṃkāravikriyāḥ   6     \P
%eko  neka   svayaṃbhūś ca dhāmnāya     bahudhā sthitaḥ//  paṃcatatvamanobuddhi--māyāhaṃkāravikriyā  //      \B
%eko  nekaḥ  svayaṃbhūś ca svadhābhāva  bahudhā sthitāḥ//  paṃcatatvamanobuddhi--māyāhaṃkāravikriyā  //      \L
%eko  nekaḥ  svayaṃbhuś ca svayāṃmnā    bahudhā sthitaḥ/   paṃcatatvamanobuddhir māyāhaṃkāravikriyā  //6//   \N1
%eko  nekaḥ  svayaṃbhaś ca svadhā...ṣ   bahudhā sthitāḥ//  paṃcatatvamanobuddhir māyāhaṃkāravikriyā  //6//   \D
%eko  neka   svayaṃbhūś ca svadhāmnāva  bahudhā sthitaḥ//  paṃcatatvamanobuddhir māyāhaṃkāravikriyā  //6//   \N2
%yeko naika/ svayaṃbhūtyā  svabhāvā     bahudhā sthitaḥ    paṃcatatvamanobuddhir māyāhaṃkāravikriyāḥ   6     \U1
%eko  naiko  svayaṃbhūś ca svadhāmnā    bahudhā sthitaḥ//  paṃcatatvamanobuddhir māyāhaṃkāravikriyā  //6//   \U2
%------------------------------
%Eins, nicht eins und aus sich selbst heraus seiend durch das eigene Walten und Wirken mannigfach existierend,
%[als] fünf Prinzipien (\textit{tattva}), welche da sind: denkender Verstand (\textit{manas}), Intellekt (\textit{buddhi}), Illusion (\textit{māya}), Individuation (\textit{ahaṃkāra}) und Modifikationen (\textit{vikriyā}). 
%------------------------------
%One, not one and self-existing, existing in manifold ways through its own rule and work,
%[as] five principles (\textit{tattva}) which are: thinking mind (\textit{manas}), intellect (\textit{buddhi}), illusion (\textit{māya}), individuation (\textit{ahaṃkāra}) and modifications ( \textit{vikriya}).
%------------------------------
\note[type=source, labelb=139, lem={eko}]{Ysv (PT): eko 'nekaḥ khayaṃ bhūyān sādhanād bahudhā sthitaḥ | pañcatattvamayo buddhimāyāhaṅkāravikriyaḥ |}
\begin{tlg}
  \tl{
\app{\lem[wit={ceteri}]{eko}
  \rdg[wit={U1}]{yeko}}
\app{\lem[type=emendation, resp=egoscr]{naikaḥ}
  \rdg[wit={U1}]{\korr naika}
  \rdg[wit={U2}]{naiko}
  \rdg[wit={ceteri}]{nekaḥ}
  \rdg[wit={B,N2}]{neka}}
\app{\lem[wit={ceteri}]{svayaṃbhūś-ca}
  \rdg[wit={U1}]{svayaṃbhūtyā}}
\app{\lem[wit={P,U2}]{svadhāmnā}
  \rdg[wit={E}]{dhāmnā ca}
  \rdg[wit={B}]{dhāmnāya}
  \rdg[wit={L}]{svadhābhāva}
  \rdg[wit={N1}]{svayāṃmnā}
  \rdg[wit={D}]{svadhā..ṣa}
  \rdg[wit={N2}]{svadhāmnāva}
  \rdg[wit={U1}]{svabhāvā}}
bahudhā
\app{\lem[wit={P,L,D}]{sthitāḥ}
  \rdg[wit={ceteri}]{sthitaḥ}}/}\\
\tl{paṃcatattvamano\app{\lem[wit={E,P,L},alt={°buddhi°}]{buddhi}
    \rdg[wit={ceteri}]{°buddhir}
}māyāhaṃkāra\app{\lem[wit={ceteri},alt={°vikriyā}]{vikriyā}
  \rdg[wit={E,P,U1}]{°vikriyāḥ}}\dd{}6\hskip-2pt\dd{}}
\end{tlg}
%------------------------------ 
%evaṃ daśavidhaṃ viśvaṃ lokālokasavistaram/   eka  eva na cānyo sti yo jānāti sa tattvavit//6// \E
%evaṃ daśavidhaṃ viśvaṃ lokālokasavistaraṃ    eka  eva na cānyo sti yo jānāti sa tatvavit 6 \P
%evaṃ daśavidhā  viśvaṃ lokālokasavistaraṃ//  eka  eva na cānyā sti yo jānāti sa tatvavit// \B
%evaṃ daśavidhā  viśvaṃ lokālokasavistaraṃ//  eka  eva na cānyo sti yo jānāti sa tatvavit// \L
%evaṃ daśavidhaṃ viśvaṃ lokālokasavistarāṃ/   eka  eva na cānyo sti yo nānāti sa tatvavit//7// \N1
%evaṃ daśavidhaṃ viśvaṃ lokālokasavistaraṃ//  eka  eva na cānyo sti yo jānāti sa tatvavit//7// \D
%evaṃ daśavidhā  viśvaṃ lokālokasavistaraṃ//  eka  eva na cānyo sti yo jānāti sa tatvavit//7// \N2
%evaṃ daśavidha--viśvaṃ lokālokasavistaraṃ    eka yeva na cānyo sti yo jānāti sa tatvavit 7 \U1
%evaṃ daśavidhaṃ viśvaṃ lokāloke savistaraṃ// ekam eva na cānyo sti yo jānāti sa tatvavit//7// \U2 %%%409.jpg 
%------------------------------
%Auf diese Weise durchdringen die zehn Variationen die Welt und die Nicht-Welt im vollen Umfang.  
%Nur das Eine ist und nicht etwas anderes: Wer das weiß ist ein Kenner der Realität.  
%------------------------------
%In this way, the ten variations fully permeate the world and the non-world.
%Only one thing is and not something else: Whoever knows this is a connoisseur of reality.
%------------------------------
\note[type=source, labelb=140, lem={daśavidhā}]{Ysv (PT): evaṃ bahuvidhaṃ viśvaṃ lokālokasuvistaram | ekam eva na cānvo 'sti yo jānāti sa tattvavit |}
\begin{tlg}
   \tl{
     evaṃ
     \app{\lem[wit={B,L,N2}]{daśavidhā viśvaṃ}
       \rdg[wit={E,P,N1,D,U2}]{daśavidhaṃ viśvaṃ}
       \rdg[wit={U1}]{daśavidhaviśvaṃ}}
     \app{\lem[wit={ceteri}]{lokālokasavistaram}  
       \rdg[wit={N1}]{°savistarāṃ}
       \rdg[wit={U2}]{°loke savistaraṃ}}/}\\
   \tl{\app{\lem[wit={ceteri}]{eka}
       \rdg[wit={U2}]{ekam}}
       \app{\lem[wit={ceteri}]{eva}
         \rdg[wit={U1}]{yeva}}
       na cānyo 'sti yo jānāti sa tattvavit \dd{}7\hskip-2pt\dd{}}\\
   \end{tlg}
    \begin{prose}     
%------------------------------
%pṛthvīvanaspatiparvatādisthārarūpaḥ         saṃsāra---manuṣyahastyaśvapakṣītyādiko    jaṃgamarūpaḥ   saṃsāraḥ// \E
%pṛthvīvanaśpatiparvatādisthāvararūpaḥ       saṃsāraḥ  manuṣyahastyaś ca pakṣītyādiko  jaṃgamarūpaḥ   saṃsāraḥ \P
%pṛthvīvanaspatīparvatādisthāvararūpā        saṃsāraḥ/ manuṣyahasteśvapakṣītyādiko     jaṃgamarūpaḥ   saṃsāraḥ// \B
%pṛthvīvanaspatiparvatādisthāvararūpā        saṃsāraḥ  manuṣyahasteśvapakṣītyādiko     jaṃgamarūpā    saṃsāraḥ// \L
%pṛthvīvanaspatīparvvate tyādisthāvararūpaḥ  saṃsāraḥ  manuṣyahastīaśvapakṣītyādiko    jaṃgamarūpaḥ   saṃsāraḥ// \N1
%pṛthvīvanaspatīparvato tyādisthāṃvararūpaḥ  saṃsāraḥ  manuṣyahastīaśvapakṣītyādiko    jaṃgamaḥ rūpaḥ saṃsāraḥ// \D
%pṛthvīvanaspatiparvate 'thyādisthāvararūpa  saṃsāraḥ  manuṣyahastipakṣītyādiko        jaṃgamarūpaḥ   saṃsāraḥ// \N2
%pṛthivīvanaspatīparvate iyādisthāvararūpaḥ  saṃsāra---manuṣyahastiasvapakṣītyādiko    jagadrūpaḥ     saṃsāro \U1
%pṛthvīvanaspatiparvatādisthāvararūpaḥ       saṃsāraḥ//manuṣyahasttyaś ca pakṣītyādiko jaṃgamarūpaḥ   saṃsāraḥ//8// \U2
%------------------------------
%Der Geburtenkreislauf ist die Erscheinung der Pflanzenwelt, der Berge, der Bäume, der Erde etc. Der Geburtenkreislauf ist die Erscheinung der Lebewesen beginnend mit Vögeln, Pferden, Elefanten und Menschen. 
%------------------------------
%The cycle of birth is the appearance of the plant world, mountains, trees, earth etc. The cycle of birth is the appearance of living beings beginning with birds, horses, elephants and humans. 
%------------------------------
\note[type=source, labelb=141, lem={saṃsāraḥ}]{Ysv (PT): sthāvarāḥ parvatādyā hi jaṅgamāḥ khecarādayaḥ | jaṅgamasthāvarākāraḥ saṃsāraḥ syāt sa īśvaraḥ |}
\app{\lem[wit={ceteri},alt={pṛthvī°}]{pṛthvī}
        \rdg[wit={U1}]{pṛthivī°}
      }\app{\lem[wit={E,N2,U2},alt={°vanaspati°}]{vanaspati}
        \rdg[wit={P}]{vanaś°}
        \rdg[wit={B,L,N1,D,U1}]{°patī°}
      }\app{\lem[wit={P,B,L,U2}, alt={°parvatādisthāra°}]{parvatādisthāvara}
        \rdg[wit={E}]{°parvatādisthāra°}
        \rdg[wit={N1}]{°parvvate tyādisthāvara°}
        \rdg[wit={N2}]{°parvate 'thyādisthāvara°}
        \rdg[wit={D}]{°parvato tyādisthāṃvara°}
        \rdg[wit={N2}]{°parvate 'thyādisthāvara°}
        \rdg[wit={U1}]{°parvate iyādisthāvara°}
      }\app{\lem[wit={ceteri}]{rūpaḥ} \rdg[wit={L,B}]{rūpā}
        \rdg[wit={N2}]{rūpa}} \app{\lem[wit={ceteri}]{saṃsāraḥ}
        \rdg[wit={E,U1}]{saṃsāra°}}/
      manuṣya\app{\lem[wit={B,L},alt={°hasteśvapakṣīty ādiko}]{hasteśvapakṣīty\skp{-}ādiko}
          \rdg[wit={E}]{°hasty aśvapakṣīty ādiko}
          \rdg[wit={N1,D}]{°hastīaśvapakṣīty ādiko}
          \rdg[wit={N2}]{°hastipakṣīty ādiko}
          \rdg[wit={U1}]{°hastiasvapakṣīty ādiko}
          \rdg[wit={U2}]{°hasttyaś ca pakṣīty ādiko}}
        \app{\lem[wit={ceteri}]{jaṃgamarūpaḥ}
          \rdg[wit={L}]{°rūpā}
          \rdg[wit={D}]{jaṃgamaḥ rūpaḥ}
          \rdg[wit={U1}]{jagad°}}
        \app{\lem[wit={ceteri}]{saṃsāraḥ}
          \rdg[wit={U1}]{saṃsāro}}/
%------------
%atha ca   yo  dṛṣṭiviṣayaḥ  sa dṛśya  ucyate/  yo dṛṣṭyā na vīkṣyate sa adṛśya ity  ucyate/ \E
%atha ca   yo  dṛṣṭiviṣayaḥ  sa dṛśya  ucyate   yo dṛṣṭyā na vīkṣyate sa adṛśya ity  ucyate  %%%7641.jog
%atha ca// yo  daṣṭiviṣayaḥ  sa dṛśya  ucyate// yo dṛṣṭyā na vīkṣyate sa adṛśya ty   ucyate// \B
%atha ca   yo ddṛṣṭiviṣayaḥ  sa dṛśya  ucyate// yo dṛṣṭyā na vīkṣyate sa adṛśye ty   ucyate... \L
%atha ca   ya ddṛṣṭiviṣayaḥ  sa dṛśyad ucyate   yo dṛṣṭyā na vīkṣyate sa adṛśya ity  ucyate// \N1
%atha vā   ya dārṣṭiviṣayaḥ  sa dṛśya  ucyate/  yo dṛṣṭyā na vīkṣyate sa adṛśya ity  ucyate// \D
%atha ca   ya  drṣṭiviṣayaḥ  sa dṛśya  ucyate/  yo dyā    na vīkṣyate sa adṛśya śaty ucyate/ \N2
%atha ca   yaḥ drṣṭiviṣayaḥ  sa dṛśy---ucyate   yo dṛṣṭvā na vīkṣyate sa adṛśya ity  ucyate \U1
%atha ca   yo  dṛṣṭiviṣayaḥ  sa dṛśya  ucyate// yo dṛṣṭyā na vīkṣyate sa adṛśya ity  ucyate// \U2
%------------------------------
%Und dann, wer einer ist, der ein [Sinnes]objekt des Sehens ist, der wird gesagt, ist sichtbar. Wer nicht durch das Sehen gesehen wird, der wird gesagt ist unsichtbar. 
%------------------------------
\note[type=source, labelb=142, lem={drṣṭiviṣayaḥ}]{Ysv (PT): svabhāvalīlayā bhāti śūnye'sau śūnyabuddhitaḥ | yad dṛṣṭaṃ viṣayaṃ vastu tad dṛśyam iti kathyate | yo dṛṣṭātītaḥ so'dṛśyas tadā dṛṣṭaṃ hi manyate | svatanūbhedam evan tu saṃsāraṃ duḥkhasaṅkulam |}
atha
      \app{\lem[wit={ceteri}]{ca}
        \rdg[wit={D}]{vā}}
      \app{\lem[wit={ceteri}]{yo}
        \rdg[wit={U1}]{yaḥ}
        \rdg[wit={N1,N2,D}]{ya}}
      \app{\lem[wit={ceteri}]{dṛṣṭi}
        \rdg[wit={L,N1}]{ddṛṣṭi}
        \rdg[wit={B}]{daṣṭi}
        \rdg[wit={D}]{dārṣṭi}
}viṣayaḥ sa
\app{\lem[wit={ceteri}]{dṛśya}
  \rdg[wit={N1}]{dṛśyad}
  \rdg[wit={U1}]{dṛṣy°}}
ucyate/
yo
\app{\lem[wit={ceteri}]{dṛṣṭyā}
  \rdg[wit={N2}]{dyā}}
na vīkṣyate sa adṛṣya
\app{\lem[wit={ceteri},alt={ity}]{i\skp{ty-u}}
  \rdg[wit={L,B}]{ty}
  \rdg[wit={N2}]{śaty}
}\skm{ty-u}cyate/
%------------------------------
%evaṃ saṃsārasya svātmano  bhedaṃ dūrīkṛty---aikam eva darśanaṃ sa eva jñānayogaḥ/   \E
%evaṃ saṃsāra----svātmano  bhedaṃ dūrīkṛtya  aikyena   darśanaṃ        jñānayogaḥ    \P
%evaṃ saṃsārasya svātmano  bheda--dūrīkṛtya  aikyona   darśanaṃ        jñānayogaḥ/   \B
%evaṃ saṃsāra----svātmano  bhedaṃ dūrīkṛtya  aikyona   darśanaṃ        jñānayogaḥ... \L
%evaṃ saṃsārasya svātmanaḥ bhedāṃ dūrīkṛtya  ekyena    darśanaṃ        jñānayogaḥ//  \N1
%evaṃ saṃsārasya svātmanaḥ bhedāṃ dūrīkṛtya  ekyena    darśanaṃ        jñānayogaḥ/   \D
%evaṃ saṃsārasya svātmanaḥ bhedaṃ dūrīkṛtya  ekena     darśanaṃ        jñānayogaḥ/   \N2
%evaṃ saṃsārasya svātmanaḥ bhedaṃ dūrīkṛtya  ekānta    darśanaṃ        jñānayogaḥ    \U1
%evaṃ saṃsāra....svātmanoḥ bhedaṃ dūrīkṛtyaṃ ekye?     darśanaṃ        jñānayoga     \U2
%------------------------------
%In this way the view of separation of one's own self which is subjected to transmigration is to be removed by means of [applying the view of] unity. Only this is Jñānayoga.  
%------------------------------
evaṃ
       \app{\lem[wit={ceteri}]{saṃsārasya}
         \rdg[wit={P,L,U2}]{saṃsāra°}}
       \app{\lem[wit={E,P,B,L}]{svātmano}
         \rdg[wit={N1,D,N2,U1}]{svātmanaḥ}
         \rdg[wit={U2}]{svātmanoḥ}}
       \app{\lem[wit={ceteri}]{bhedaṃ}
         \rdg[wit={B}]{bheda}
         \rdg[wit={D,N1}]{bhedāṃ}}
\app{\lem[wit={U2}]{dūrīkṛtyaṃ}
  \rdg[wit={ceteri}]{°kṛtya}
  \rdg[wit={E}]{°kṛty}}
\app{\lem[wit={P}]{aikyena}
  \rdg[wit={E}]{aikam eva}
  \rdg[wit={P,B,L}]{aikyona}
  \rdg[wit={N1,D}]{ekyena}
  \rdg[wit={N2}]{ekena}
  \rdg[wit={U1}]{ekānta}
  \rdg[wit={U2}]{ekye}}
darśanaṃ
\app{\lem[wit={E}]{sa eva}
  \rdg[wit={ceteri}]{\om}}
\app{\lem[wit={ceteri}]{jñānayogaḥ}
  \rdg[wit={U2}]{jñānayoga}}/ 
%------------------------------
%tasya         kāraṇāt kālaḥ śarīranāśaṃ na karoti/ \E
%tasya         kāraṇāt kālaḥ śarīranāśaṃ na karoti/ \P
%tasya         karaṇāt kālaḥ śarīranāśaṃ na karoti// \B
%tasya         karaṇāt kālaḥ śarīranāśaṃ na karoti... \L
%tasya         karaṇāt kālaḥ śarīranāśaṃ na karoti// \N1
%tasya         karaṇāt kālaḥ śarīranāśaṃ na karoti// \D
%tasya         karaṇāt kālaḥ śarīranāśaṃ    karoti/ \N2
%gatasya dhyānakaraṇāt kālaḥ śarīranāśaṃ na karoti 8 \U1
%tasya         karaṇāt kālaśarīranāśanaṃ    karoti// \U2
%------------------------------
%Because of the execution of this time does not destroy the body. 
%------------------------------
\app{\lem[wit={ceteri}]{tasya}
  \rdg[wit={U1}]{gatasya}}
\app{\lem[wit={ceteri}]{kāraṇāt}
  \rdg[wit={U1}]{dhyānakaraṇāt}}
\app{\lem[wit={ceteri}]{kālaḥ}
  \rdg[wit={U1}]{kāla°}}
śarīranāśaṃ
\app{\lem[wit={ceteri}]{na}
  \rdg[wit={N2,U2}]{\om}}
karoti\dd{}
    \end{prose}
  \end{ekdosis}
  \ekdpb*{}
%%%%%%%%%%%%%%%%%%%%%%%%%%%%%%%%%%%%%%%%%%
%%%%%%%%%%%%%%%%%%%%%%%%%%%%%%%%%%%%%%%%%%
%%%%%%%%PAGEBREAK%%%%%%%PAGEBREAK%%%%%%%%%
%%%%%%%%%%%%%%%%%%%%%%%%%%%%%%%%%%%%%%%%%%
%%%%%%%%%%%%%%%%PAGEBREAK%%%%%%%%%%%%%%%%%
%%%%%%%%%%%%%%%%%%%%%%%%%%%%%%%%%%%%%%%%%%
%%%%%%%%PAGEBREAK%%%%%%%PAGEBREAK%%%%%%%%%
%%%%%%%%%%%%%%%%%%%%%%%%%%%%%%%%%%%%%%%%%%
%%%%%%%%%%%%%%%%%%%%%%%%%%%%%%%%%%%%%%%%%%
%%%%%%%%%%%%%%%%%%%%%%%%%%%%%%%%%%%%%%%%%%
%%%%%%%%%%%%%%%%%%%%%%%%%%%%%%%%%%%%%%%%%%
%%%%%%%%PAGEBREAK%%%%%%%PAGEBREAK%%%%%%%%%
%%%%%%%%%%%%%%%%%%%%%%%%%%%%%%%%%%%%%%%%%%
%%%%%%%%%%%%%%%%PAGEBREAK%%%%%%%%%%%%%%%%%
%%%%%%%%%%%%%%%%%%%%%%%%%%%%%%%%%%%%%%%%%%
%%%%%%%%PAGEBREAK%%%%%%%PAGEBREAK%%%%%%%%%
%%%%%%%%%%%%%%%%%%%%%%%%%%%%%%%%%%%%%%%%%%
%%%%%%%%%%%%%%%%%%%%%%%%%%%%%%%%%%%%%%%%%%
%%%%%%%%%%%%%%%%%%%%%%%%%%%%%%%%%%%%%%%%%%
%%%%%%%%%%%%%%%%%%%%%%%%%%%%%%%%%%%%%%%%%%
%%%%%%%%PAGEBREAK%%%%%%%PAGEBREAK%%%%%%%%%
%%%%%%%%%%%%%%%%%%%%%%%%%%%%%%%%%%%%%%%%%%
%%%%%%%%%%%%%%%%PAGEBREAK%%%%%%%%%%%%%%%%%
%%%%%%%%%%%%%%%%%%%%%%%%%%%%%%%%%%%%%%%%%%
%%%%%%%%PAGEBREAK%%%%%%%PAGEBREAK%%%%%%%%%
%%%%%%%%%%%%%%%%%%%%%%%%%%%%%%%%%%%%%%%%%%
%%%%%%%%%%%%%%%%%%%%%%%%%%%%%%%%%%%%%%%%%%
\begin{ekdosis}
    \ekddiv{type=ed}
          \bigskip
        \centerline{\textrm{\small{[Division of the Inherent Nature]}}}
          \bigskip
          \begin{prose}
            \noindent
%------------------------------
%idānīṃ tasya---bhedaḥ    kathyate/   \E
%idānīṃ svabhāvabhedaḥ kathyate    \P
%idānī  svābhāvabhedaḥ kathyate//  \B
%idānīṃ svābhāvabhedaḥ kathyate//  \L
%idānīṃ svabhāvabhedaṃ kathyate//  \N1
%idānīṃ svabhāvabhedaṃ kathyate//  \D
%idānīṃ svabhāvabheda  kathyate//  \N2
%idānīṃ svabhāvabhedāḥ kathyate    \U1
%idānīṃ svabhāvabhedaḥ kathyate//  \U2
%------------------------------
%Now the division of the inherent nature is described. 
%------------------------------  
\note[type=source, labelb=143, lem={}]{Ysv (PT): svabhāvabhedam etat śṛṇu devi prayatnataḥ |}
\app{\lem[wit={ceteri}]{idānīṃ}
  \rdg[wit={B}]{idānī}}
\app{\lem[wit={ceteri},alt={svabhāva°}]{svabhāva}
  \rdg[wit={B,L}]{svābhāva°}
  \rdg[wit={E}]{tasya}
}\app{\lem[wit={D,N1},alt={°bhedaṃ}]{bhedaṃ}
  \rdg[wit={N2}]{°bheda}
  \rdg[wit={ceteri}]{°bhedaḥ}}
kathyate/
%------------------------------  
%yathā vaṭabījam/ vaṭarūpeṇa pariṇataṃ    sat    daśadhā    bhedaṃ svabhāvata eva prāpnoti/  \E %%%[P.27]
%yathā vaṭabījaṃ  vaṭarūpeṇa pariṇāte     sat    dṛśadhā    bhedaṃ svabhāvata eva prāpnoti   \P
%yathā vaṭabījena rūpeṇa     pariṇamate/  śata   daśadhā    bhedaṃ svābhāva   eva prāpnotī// \B
%yathā vaṭabījena rūpeṇa     pariṇamate   śata   daśadhā    bhedaṃ svābhāva   eva prāpnotī// \L
%yathā vaṭabījaṃ  vaṭarūpeṇa pariṇataṃ//  satṛ   daśadhā    bhedaṃ svabhāvata eva prāpnoti/  \N1
%yathā vaṭabījaṃ  vaṭarūpeṇa pariṇataṃ/   sa     daśadhā    bhedaṃ svabhāvata eva prāpnoti// \D
%yathā vathabījaṃ vaṭarūpeṇa pariṇataṃ/   sa tu  daśadhā    bhedaṃ svabhāvata eva prāpnoti/  \N2
%yathā vaṭabījaṃ  vaṭarūpeṇa pariṇataṃ    sa tat daśadhā    bhedaṃ svabhāvata eva prāpnotī   \U1
%yathā vaṭabīja---vaṭarūpeṇa pariṇamate// sa     dasat                            prāpnoti// \U2
%------------------------------
%Wie der Samen des Banyan-Baumes zur Gestalt des Banyan-Baumes heranreift und er sich aufgrund seiner eigenen inhärenten Natur so eine zehnfachen Auftheilung erreicht. [Nämlich]: 
%------------------------------
%Just as the seed of the banyan tree ripens into the shape of the banyan tree, and by its own inherent nature attains such a tenfold division. [Namely]:
%------------------------------
\note[type=philcomm, labelb=144, lem={daśadhā}]{Remarkably, the tenfold division of \textit{svabhāva} is missing in the Ysv and SSP.}
yathā
\app{\lem[wit={ceteri},alt={vaṭa°}]{vaṭa}
  \rdg[wit={N2}]{vatha°}
}\app{\lem[wit={D,P,N1,N2,U1},alt={°bījaṃ}]{bījaṃ}
        \rdg[wit={E}]{°bījam}
        \rdg[wit={U2}]{°bīja°}
        \rdg[wit={B,L}]{°bījena}}
      \app{\lem[wit={ceteri}]{vaṭarūpeṇa}
        \rdg[wit={L,B}]{rūpeṇa}}
      \app{\lem[wit={B,L,U2}]{pariṇamate}
        \rdg[wit={P}]{pariṇāte}
        \rdg[wit={ceteri}]{pariṇataṃ}}
      \app{\lem[wit={U1}]{sa tat}
        \rdg[wit={N2}]{sa tu}
        \rdg[wit={N1}]{satṛ}
        \rdg[wit={E,P}]{sat}
        \rdg[wit={B,L}]{śata}
        \rdg[wit={D,U2}]{sa}}
      \app{\lem[wit={ceteri}]{daśadhā}
        \rdg[wit={P}]{dṛśadhā}
        \rdg[wit={U2}]{dasat}}
      \app{\lem[wit={ceteri}]{bhedaṃ}
        \rdg[wit={U2}]{\om}}
      \app{\lem[wit={ceteri}]{svabhāvata}
        \rdg[wit={B,L}]{svabhāva}
        \rdg[wit={U2}]{\om}}
      \app{\lem[wit={ceteri}]{eva}
        \rdg[wit={U2}]{\om}}
      \app{\lem[wit={ceteri}]{prāpnoti}
        \rdg[wit={B,L,U1}]{prāpnotī}}/
%------------------------------ %%%%STEMMA POINT!!!!
%mūlāṃkura---tvagdaṇḍaśākhā--kalikāpallavapuṣpaphalasnehā                  iti daśabhedān    prāpnoti// \E
%mūla aṃkura-tvakdaṃdaśākhā----kilpikāpallavā puṣpaphalasneha              iti daśabhedān    prāpnotīti \P  %%%7642.jpg
%mūlaṃ aṃkuratvakdaṃdaśākhā----kilakālapallavā// vistāroyaṃ svābhāvataḥ    iti daśabhedān    prāpnoti// \B DSCN7160 Z. 4
%mūlaṃ aṃkuratvakdaṃdaśākhā----kilāpallavā// vistāroyaṃ svābhāvataḥ//      iti daśabhedān    prāpnoti... \L
%mūlāṃ aṃkuratvakdaṃḍaśākhāṃ kalikāpallavapuṣpaphalasneha//                iti bhedo daśadhā prāpnoti// \N1
%mūlāṃkura---tvakdaṇdaśākhāṃ kalikāpallavapuṣpaphalasnehaṃ                 iti bhedo daśadhā prāpnoti// \D
%mūlāṃkura---tvakdaṇdaśākhāṃ kalikāpallavapuṣpaphalasneha/                 iti bhedo daśadhā prāpnoti// \N2
%mūlāṃaṃkura-tvakdaṇdaśākhā--kalikāpallavapuṣpaphalasneha                  iti bhedo daśadhā prāpnoti \U1
%\om                                                                                \U2
%------------------------------
%"Wurzel, Spross, Rinde, Ast, Zweig, Knospe, die sich entfaltende Blüte, Blüte, Frucht und Nektar." Die Auftheilung erreicht [diese] zehn Teile. 
%------------------------------
%"Root, shoot, bark, branch, twig, bud, the unfolding flower, flower, fruit and nectar." The division reaches [those] ten parts.
%------------------------------
\app{\lem[wit={E}]{mūlāṃkuratvagdaṇḍaśākhākalikāpallavapuṣpaphalasnehā}
          \rdg[wit={P}]{mūla aṃkuratvakdaṃdaśākhākilpikāpallavā puṣpaphalasneha}
          \rdg[wit={B}]{mūlaṃ aṃkuratvakdaṃdaśākhākilakālapallavā || vistāroyaṃ svābhāvataḥ}
          \rdg[wit={L}]{mūlaṃ aṃkuratvakdaṃdaśākhākilāpallavā || vistāroyaṃ svābhāvataḥ ||}
          \rdg[wit={N1}]{mūlāṃ aṃkuratvakdaṃḍaśākhāṃ kalikāpallavapuṣpaphalasneha ||}
          \rdg[wit={N2}]{mūlāṃkuratvakdaṇdaśākhāṃ kalikāpallavapuṣpaphalasneha|}
          \rdg[wit={D}]{mūlāṃkuratvakdaṇdaśākhāṃ kalikāpallavapuṣpaphalasnehaṃ}
          \rdg[wit={U1}]{mūlāṃaṃkuratvakdaṇdaśākhākalikāpallavapuṣpaphalasneha}
          \rdg[wit={U2}]{\om}}
        \app{\lem[wit={ceteri}]{iti}
          \rdg[wit={U2}]{\om}}
        \app{\lem[wit={N1,D,N2,U1}]{bhedo daśadhā}
          \rdg[wit={E,P,L,B}]{daśabhedān}
          \rdg[wit={U2}]{\om}}
        \app{\lem[wit={ceteri}]{prāpnoti}
          \rdg[wit={P}]{prāpnotīti}
          \rdg[wit={U2}]{\om}}/
%------------------------------
%yathā nirmalo  nirvikāraḥ niraṃjana   eka  etādṛśa  ātmā svabhāvād eva/ pṛthivyaptejovāyvākāśamanobuddhimāyāvikārarūpabhedān    prāpnoti/ \E
%tathā nirmalaḥ nirvikāraḥ niraṃjanaḥ  eka  etādṛśa  ātmasvabhāvād eva   pṛthvyetetejo vādvyākāśamanobuddhimāyāvikārarūpabhedāt  prāpnoti \P
%tathā nirmalo  nirvikāraḥ niraṃjanaḥ  eka  etādṛśa  ātmasvabhāvād eva   pṛthvyāpatejovādvyākāśamanobuddhimāyāvikārarūpabhedāna  prāpnoti// \B
%tathā nirmalo  nirvikāraḥ niraṃjanaḥ/ eka  etādṛśa  ātmasvabhāvād eva   pṛthvyāpatejovāybākāśamanobuddhimāyāvikārarūpābhedāna   prāpnoti  \L
%tathā nirmalaḥ nirvikāraḥ niraṃjanaḥ  ekaḥ etādṛśaḥ ātmasvabhāvād eva   pṛthvyāpatejovāybākāśamanobuddhimāyāvikārarūpābhedān    prāpnoti/ \N1
%tathā nirmalaḥ nirvikāraḥ niraṃjanaḥ  eka  etādṛśaḥ ātmasvabhāvād eva   pṛthvīpate/ jīvīkāśamanobuddhir māyāvikārarūpabhedāt    prāpnoti \D
%tathā nirmalaḥ nirvikāraḥ niraṃjanaḥ  ekaḥ etādṛśaḥ ātmasvabhāvād eva   pṛthvīpate/ jīvīkāśamanobuddhir māyāvikārarūpabhedāt    prāpnoti/ \N2
%tathā nirmalaḥ nirvikāraḥ niraṃjanaḥ  ekaḥ etādṛśaḥ ātmascabhāvād eva   pṛthakte jīvāyuvākāśamanobuddhir māyāyāvikārarūpabhedāt prāpnoti \U1 %%%275.jpg
%yathā nirmalaḥ nirvikāraḥ niraṃjanaḥ  eka  etādṛśa  ātmasvabhāvād eva// pṛthvyaptejovāyyākāśa// manobuddhimayāvikārarūpabhedān  prāpnoti/ \U2
%------------------------------
%In dieser Weise erreicht auch das reine, unveränderliche, makellose, eine solche [Auftheilung] eben aufgrund der inhärenten Natur des Selbst. [Nämlich] die Aufteilung "Erde, Wasser, Feuer, Wind, Raum, Geist, Intellektekt, Illusion, Umwandlungen und Gestalt".
%------------------------------
%In this way, the pure, unchanging, unblemished, attains such [division] precisely because of the inherent nature of the self. [Namely] the division "Earth, Water, Fire, Wind, Space, Mind, Intellect, Illusion, Transformations and Form".
%------------------------------
        \app{\lem[wit={ceteri}]{tathā}
            \rdg[wit={E,U2}]{yathā}}
          \app{\lem[wit={E,B,L}]{nirmalo}
            \rdg[wit={ceteri}]{nirmalaḥ}}
          nirvikāraḥ
          \app{\lem[wit={E}]{niraṃjana}
            \rdg[wit={ceteri}]{niraṃjanaḥ}}
          \app{\lem[wit={ceteri}]{eka}
            \rdg[wit={N1,N2,U1}]{ekaḥ}}
          \app{\lem[wit={E}]{etādṛśa}
            \rdg[wit={N1,N2,U1}]{etādṛśaḥ}}
          \app{\lem[wit={ceteri}]{ātmasvabhāvād}
            \rdg[wit={E}]{ātmā°}}
          eva
          \app{\lem[wit={N1}]{pṛthvyāpatejovāybākāśamanobuddhimāyāvikārarūpābhedān}
            \rdg[wit={E}]{pṛthivyap°}
            \rdg[wit={B,L}]{°bhedāna}
            \rdg[wit={P}]{pṛthvyetetejovādvyākāśa°}
            \rdg[wit={D,N2}]{pṛthvīpate | jīvīkāśamanobuddhir māyāvikārarūpabhedāt}
            \rdg[wit={U1}]{pṛthakte jīvāyuvākāśamanobuddhir māyāyāvikārarūpabhedāt}
            \rdg[wit={U2}]{pṛthvyaptejovāyyākāśa || manobuddhimayāvikārarūpabhedā}}
          prāpnoti/
%------------------------------
%jñānayogaprabhāvād     eka eva  ātmā iti niścayo bhavati// \E
%jñānayogaḥ prabhāvād   eka eka  ātmā iti niścayo bhavati \P
%jñānayogaḥ// prabhāvād eka eka  ātmā iti niścayā bhavatī// \B
%jñānayogaḥ// prabhāvād eka eka  ātmā iti niścayo bhavati// \L
%jñānayogaprabhāvāt     eka eva  ātmā iti niścayo bhavati// \N1
%jñānayogaprabhāvāt     eka eva  ātmā iti niścayo bhavati// \D
%jñānayogaprabhāvāt     eka eva  ātmā iti niścayo bhavati// \N2
%jñānayogaprabhāvāt tu  eka yeva ātmā iti niścayo bhavati \U1
%jñānayogaprabhāvād     eka eva  ātmā iti niścayo bhavati// \U2
%------------------------------
%Aufgrund der Macht von Jñānayoga entsteht so die Gewissheit "Das Selbst ist wahrlich eins".
%------------------------------
%Because of the power of jñānayoga, there arises the certainty that "The Self is verily one."    
%------------------------------
\app{\lem[wit={E,U2}, alt={jñānayogaprabhāvād}]{jñānayogaprabhāvā\skp{d-e}}
  \rdg[wit={N1,D,N2,U1}]{°bhavāt}
  \rdg[wit={L,B}]{jñānayogaḥ || prabhāvād°}
  \rdg[wit={P}]{jñānayogaḥ prabhāvād}
}\skm{d-e}ka
\app{\lem[wit={ceteri}]{eva}
  \rdg[wit={P,B,L}]{eka}
  \rdg[wit={U1}]{yeva}}
ātmā iti niścayo bhavati/
%------------------------------
%yathaikaiva   pṛthvī  kvacit komalarūpā                                                   kvacit parimalarūparahitā kvacit suvarṇarūpā   kvacid raupyarūpā    \E %%%p.28 
%yathā ekaika  pṛthvī  kvacit komalarūpā                                                                                                                       \P   
%yathā ekaika  pṛthvī  kvacit komalarūpā// kvacit manohararūpā//  kvacit parimalarūpayuktā// kvacit parimalarohitā// kvacit suvarṇarūpa                        \B
%yathā ekaika  pṛthvī  kvacit komalarūpā   kvacit manohararūpāḥ// kvacit parimalarūpayuktā// kvacit parimalarahitā// kvacit suvarṇarūpā                        \L
%yathā ekaiva  pṛthivī kvacit komalarūpa/  kvacit manoharā/       kvacit parimalarūpāyuktā// kvacit parimalarahitā/  kvacit suvarṇarūpā/  kvacit rūpyarūpā/    \N1
%yathā ekaiva  pṛthivī kvacit komalarūpa   kvacit manoharā//      kvacit parimalarūpāyuktā/  kvacit parimalarohitā   kvacit suvarṇarūpa// kvacit rūpyarūpa//   \D
%yathā ekaṃ ca pṛthivī kvacit komalarūpa   kvacit manoha?rā       kvacit parimalarūpāyuktaḥ/ kvacit parimalarohitā   kvacit suvarṇarūpā   kvacit rūpyarūpa     \N2
%yathā ekai ca pṛthivī kvacit                                                                                              khavarṇakupā   kvacit rūpyarūpā     \U1
%yathā ekaika  pṛthvī  kvacit komalarūpā// kvacit manohararūpa//  kvacit parimalarūpāyuktā/  kvacit parimalarohitā// kvacit suvarṇarūpā// kvacit rajatarūpā//  \U2
%------------------------------
%Wie irgendein bestimmter Erdboden (\textit{ekaika}) manchmal weich erscheint, manchmal schön erscheint, manchmal mit Wohlgeruch versehen ist, manchmal ohne Wohlgeruch ist, manchmal golden erscheint, manchmal silbern erscheint, ...
%
%As some particular soil (\textit{ekaika}) sometimes appears soft, sometimes appears beautiful, sometimes fragrant, sometimes unscented, sometimes golden, sometimes silver,... 
%------------------------------
\note[type=source, labelb=145, lem={pṛthivī}]{Ysv (PT): ātmano vā pṛthivyādyāḥ svabhāvaḥ kiñcid ucyate | ātmaiva pṛthivī dhātrī komalā ca kvacid dṛḍhā | kvacin manoharā sā ca vimalā ca malāmalā | durgandhā ca sugandhā ca nirgandhā gandhamohinī | svarṇarūpā dhāturūpā citrā ratnamayī parā | kvacit śvetā kvacid raktā kvacit pītā ca kṛṣṇalā | ūrvarā ūrvarā sā tu viṣāmṛtamayī sadā |}
\app{\lem[type=emendation, resp=egoscr]{yathaikaikaḥ}
  \rdg[wit={E}]{\korr yathaikaiva}
  \rdg[wit={P,B,L,U2}]{yathā ekaika}
  \rdg[wit={N1,D}]{yathā ekaiva}
  \rdg[wit={N2}]{yathā ekaṃ ca}
  \rdg[wit={U1}]{yathā ekai ca}}
\app{\lem[wit={E,P,B,L,U2}]{pṛthvī}
  \rdg[wit={ceteri}]{pṛthivī}}
kvacit
komala\app{\lem[wit={E,P,B,L,U2},alt={°rūpā}]{rūpā}
    \rdg[wit={ceteri}]{°rūpa}}\dd{}
\app{\lem[wit={ceteri}]{kvacit}
  \rdg[wit={E,P,U1}]{\om}}
\app{\lem[wit={B}]{manohararūpā}
  \rdg[wit={L}]{°rūpāḥ}
  \rdg[wit={U2}]{°rūpa}
  \rdg[wit={N1,N2,D}]{manoharā}
  \rdg[wit={E,P,U1}]{\om}}\dd{}
\app{\lem[wit={ceteri}]{kvacit}
  \rdg[wit={E,P,U1}]{\om}}
\app{\lem[wit={ceteri},alt={°parimala}]{parimala}
  \rdg[wit={E,P,U1}]{\om}
}\app{\lem[wit={B,L},alt={°rūpayuktā}]{rūpayuktā}
  \rdg[wit={N1,D}]{°rūpā°}
  \rdg[wit={N2}]{°rūpāyuktaḥ}
  \rdg[wit={E,U1}]{\om}}\dd{}
\app{\lem[wit={ceteri}]{kvacit}
  \rdg[wit={P,U1}]{\om}}
\app{\lem[wit={ceteri},alt={°parimala}]{parimala}
  \rdg[wit={E}]{°parimalarūpa°}
  \rdg[wit={P,U1}]{\om}
}\app{\lem[wit={E,L,N1},alt={°rahitā}]{rahitā}
  \rdg[wit={B,N2,U2}]{°rohitā}
  \rdg[wit={ceteri}]{\om}}\dd{}
\app{\lem[wit={ceteri}]{kvacit}
  \rdg[wit={P,U1}]{\om}}
\app{\lem[wit={E,L,N2,U2}]{suvarṇarūpā}
  \rdg[wit={B,D}]{°rūpa}
  \rdg[wit={U1}]{khavarṇakupā}
  \rdg[wit={P}]{\om}}\dd{}
\app{\lem[wit={ceteri}]{kvacit}
  \rdg[wit={P,B,L}]{\om}}
\app{\lem[wit={E}]{raupyarūpā}
  \rdg[wit={N1,U1}]{rūpyarūpā}
  \rdg[wit={D,N2}]{rūpyarūpa}
  \rdg[wit={U2}]{rajatarūpā}
  \rdg[wit={P,B,L}]{\om}}\dd{}
%------------------------------
%kvacid ratnamayī   kvacic ca śvetā                                kvacidraktā   kvacitpītā    \E %%%p.28 
%                                                                                             \P   
%kvacid ratnamaī//  kvacit śverūpā// kvacitkṛṣṇā//                 kvacidraktā/  kvacitpītā//  \B
%kvacid ratnamaī//  kvacit śvetarūpā kvacitkṛṣṇā//                 kvacidraktā// kvacitpītā//  \L
%kvacid ratnamayī/  kvacit śveta/    kvacitkṛṣṇa??/                kvacidrakta/  kvacitpītā/   \N1
%kvacid ratnamayī// kvacit śvetā//   kvacitkṛṣṇā [S8., Z.7]        kvacidrakta   kvacitpītā//  \D
%kvacid ratnamayī   kvacit śveta     kvacitkṛṣṇā// [S6. verso]     kvacidrakta   kvacitpītā    \N2
%kvacid ratnamayī   kvacit śveta     kvacitkṛṣṇā                   kvacidrakta   kvacitpītā    \U1
%kvacid ratnamayī// kvacit śvetā//   kvacitkṛṣṇā//                 kvacidraktā// kvacitpītā//  \U2
%------------------------------
% ... manchmal aus Edelstein gemacht ist, manchmal weiß erscheint, manchmal schwarz, manchmal kupfern, manchmal gelb,
%
%... is sometimes made of precious stone, sometimes appearing white, sometimes black, sometimes copper, sometimes yellow, 
%------------------------------
\app{\lem[wit={ceteri},alt={°kvacid}]{kvaci\skp{d-ra}}
  \rdg[wit={P}]{\om}
}\app{\lem[wit={ceteri},alt={ratnamayī}]{\skm{d-ra}tnamayī}
  \rdg[wit={B,L}]{°maī}
  \rdg[wit={P}]{\om}}\dd{}
\app{\lem[wit={ceteri}]{kvacit}
  \rdg[wit={E}]{kvacic ca}
  \rdg[wit={P}]{\om}}
\app{\lem[wit={E,D,U2}]{śvetā}
  \rdg[wit={N1,N2,U1}]{śveta}
  \rdg[wit={L}]{śvetarūpā}
  \rdg[wit={B}]{śverūpā}
  \rdg[wit={P}]{\om}}\dd{}
\app{\lem[wit={ceteri}]{kvacit kṛṣṇā}
  \rdg[wit={N1}]{kṛṣṇa}
  \rdg[wit={E,P}]{\om}}\dd{}
\app{\lem[wit={ceteri},alt={°kvacid}]{kvaci\skp{d-ra}}
  \rdg[wit={P}]{\om}
}\app{\lem[wit={E,B,L,U2},alt={raktā}]{\skm{d-ra}ktā}
  \rdg[wit={ceteri}]{°rakta}}\dd{}
kvacit pītā\dd{}
\end{prose}
\end{ekdosis}
\ekdpb*{}
%%%%%%%%%%%%%%%%%%%%%%%%%%%%%%%%%%%%%%%%%%
%%%%%%%%%%%%%%%%%%%%%%%%%%%%%%%%%%%%%%%%%%
%%%%%%%%PAGEBREAK%%%%%%%PAGEBREAK%%%%%%%%%
%%%%%%%%%%%%%%%%%%%%%%%%%%%%%%%%%%%%%%%%%%
%%%%%%%%%%%%%%%%PAGEBREAK%%%%%%%%%%%%%%%%%
%%%%%%%%%%%%%%%%%%%%%%%%%%%%%%%%%%%%%%%%%%
%%%%%%%%PAGEBREAK%%%%%%%PAGEBREAK%%%%%%%%%
%%%%%%%%%%%%%%%%%%%%%%%%%%%%%%%%%%%%%%%%%%
%%%%%%%%%%%%%%%%%%%%%%%%%%%%%%%%%%%%%%%%%%
%%%%%%%%%%%%%%%%%%%%%%%%%%%%%%%%%%%%%%%%%%
%%%%%%%%%%%%%%%%%%%%%%%%%%%%%%%%%%%%%%%%%%
%%%%%%%%PAGEBREAK%%%%%%%PAGEBREAK%%%%%%%%%
%%%%%%%%%%%%%%%%%%%%%%%%%%%%%%%%%%%%%%%%%%
%%%%%%%%%%%%%%%%PAGEBREAK%%%%%%%%%%%%%%%%%
%%%%%%%%%%%%%%%%%%%%%%%%%%%%%%%%%%%%%%%%%%
%%%%%%%%PAGEBREAK%%%%%%%PAGEBREAK%%%%%%%%%
%%%%%%%%%%%%%%%%%%%%%%%%%%%%%%%%%%%%%%%%%%
%%%%%%%%%%%%%%%%%%%%%%%%%%%%%%%%%%%%%%%%%%
%%%%%%%%%%%%%%%%%%%%%%%%%%%%%%%%%%%%%%%%%%
%%%%%%%%%%%%%%%%%%%%%%%%%%%%%%%%%%%%%%%%%%
%%%%%%%%PAGEBREAK%%%%%%%PAGEBREAK%%%%%%%%%
%%%%%%%%%%%%%%%%%%%%%%%%%%%%%%%%%%%%%%%%%%
%%%%%%%%%%%%%%%%PAGEBREAK%%%%%%%%%%%%%%%%%
%%%%%%%%%%%%%%%%%%%%%%%%%%%%%%%%%%%%%%%%%%
%%%%%%%%PAGEBREAK%%%%%%%PAGEBREAK%%%%%%%%%
%%%%%%%%%%%%%%%%%%%%%%%%%%%%%%%%%%%%%%%%%%
%%%%%%%%%%%%%%%%%%%%%%%%%%%%%%%%%%%%%%%%%%
\begin{ekdosis}
  \begin{prose}
    \noindent
%------------------------------
%kvacitkarburā   kvacin nānāvidharūpā        kvacid viṣarūpā    kvacid amṛtarūpamayī svabhāvata eva bhavati//  \E  %%%p.28
%                                                               kvacid amṛtamayī     svabhāvata eva bhavati    \P  %%%rest is \om
%kvacitkarburā// kvacin nānāvidhaphalarūpā   kvacit viṣarūpā//  kvacid amṛtamaī/     svabhāvata eva bhavataḥ// \B
%kvacitkarburā// kvacin nānāvidhāphalarūpā   kvacit viṣarūpā//  kvacid amṛtamaī//    svabhāvata eva bhavataḥ// \L
%kvacitkarburā,  kvacin nānāvidhaphalarūpā/  kvacid puṣparūpā,  kvacid amṛtamayī     svabhāvata eva bhavati/   \N1
%kvacitkarburā   kvacin nānāvidhaphalarūpā// kvacid puṣparūpā// kvacid amṛtamayī/    svabhāvata eva bhavati//  \D
%kvacitkarburā   kvacin nānāvidhaphalarūpā                      kvacid amṛtamayī/    svabhāvata eva bhavati//  \N2
%kvacitkarpurā   kvacin nānāvidhophalarūpā   kvacid ....[rest omitted]                                         \U1
%kvacitkarburā// kvacit nānāvidhaphalarūpā// kvacir viśarūpā//  kvacit amṛtamayī//   svabhāvata eva bhavati//  \U2
%------------------------------
%machmal gesprenkelt, machmal wie verschiedenartige Frucht erscheint, manchmal wie Blumen erscheint, machmal wie der Nektar der Unsterblichkeit erscheint, [und das nur] nur aufgrund seiner inhärenten Natur.
%
%sometimes mottled, sometimes appearing like various fruit, sometimes appearing like flowers, sometimes appearing like the nectar of immortality, [and that only] only because of its inherent nature. 
%------------------------------
kvavit
\app{\lem[wit={ceteri}]{karburā}
  \rdg[wit={U1}]{karpurā}}\dd{}
\app{\lem[wit={ceteri}]{kvaci\skp{n-nā}}
  \rdg[wit={U2}]{kvacit}
  \rdg[wit={P}]{\om}
}\app{\lem[wit={ceteri},alt={nānāvidhaphalarūpā}]{\skm{n-nā}nāvidhaphalarūpā}
  \rdg[wit={U1}]{nānāvidhophalarūpā}
  \rdg[wit={E}]{nānāvidharūpā}
  \rdg[wit={P}]{\om}}\dd{}
\app{\lem[wit={ceteri},alt={kvacid}]{kvaci\skp{d-pu}}
  \rdg[wit={B,L}]{kvacit}
  \rdg[wit={U2}]{kvacir}
  \rdg[wit={P,N2}]{\om}
}\app{\lem[wit={N1,D},alt={puṣparūpā}]{\skm{d-pu}ṣparūpā}
\rdg[wit={E,B,L}]{viṣarūpā}
\rdg[wit={U2}]{vśarūpā}
\rdg[wit={U1}]{\om}}\dd{}
\app{\lem[wit={ceteri}, alt={kvacid}]{kvaci\skp{d-a}}
  \rdg[wit={U2}]{kvacit}
  \rdg[wit={U1}]{\om}
}\app{\lem[wit={ceteri},alt={amṛtamayī}]{\skm{d-a}mṛtamayī}
  \rdg[wit={E}]{amṛtarūpamayī}
  \rdg[wit={B,L}]{amṛtamaī}
  \rdg[wit={U1}]{\om}}\dd{}
\app{\lem[wit={ceteri}]{svabhāvata}
  \rdg[wit={U1}]{\om}}
\app{\lem[wit={ceteri}]{eva}
  \rdg[wit={U1}]{\om}}
\app{\lem[wit={ceteri}]{bhavati}
  \rdg[wit={B,L}]{bhavataḥ}
  \rdg[wit={U1}]{\om}}\dd{}
%------------------------------
%tathaivātmā   manuṣyapakṣihariṇahastividyādharagandharvakinnaramahāpaṃḍitamahāmūrkharogyarogikrodhiśāṃtarūpaḥ           svabhāvād eva bhavati/ \E
%tathaivātmā   manuṣyapakṣihariṇāhastividyādharagaṃdharvakinnaramahāṃpiṃḍitamahārmūkharogī    krodhiśāṃtarūpāḥ           svabhāvād eva bhavati \P
%tathaivātmā// manuṣyapakṣihariṇahastividyādharagaṃdharvakinnaramahāpiṃḍatamahāmūrkharogī     krodhadhiśāṃtarūpaḥ        svabhāvād eva bhavatī/ \B
%tathaivātmā   manuṣyapakṣihariṇahastividyādharagaṃdharvakinnaramahāpaṃḍitamahāmūrkharogī     krodhadhīśāṃtarūpāḥ        svabhāvād eva bhavatī/ \L
%tathātmā//    manuṣya,pakṣi,hariṇa,hastī,vidyādhara,gandharvakiṃnara/mahāpaṃḍitamahāmūrva,rogī, arogī/krodhī,śāntarūpa,svabhāvād eva bhati/ \N1 %%%%%%%CRAZY SWITCH BETWEEN DAṆḌA AND COMMA
%tathātmā//    manuṣyapakṣi// hariṇahastīvidyādharagandharvakinnaramahāpaṃḍitamahāmūrvarogī arogīkrodhīśāṃtarūpa---------svabhāvād eva bhavati/ \D
%tathātmā//    manuṣyapakṣihariṇahastividyādharagandharvakinnaramahāpaṇḍitamahāmūrkharogīarogīkrodhīśāṃtarūpa------------svabhāvād eva bhavati/ \N2
%                                     vidyādharagaṃdharvakinnaramahāpaṇḍitamahāmūrṣarogīarogīkrodhīśāṃtarūpa        evaṃ svabhāvaṃ dharati  \U1
%tathaivātmā   manuṣyapakṣihariṇahastividyādharagaṃdharvakinnaramahāpaṃḍitamahāmūrkharogī arogī krodhiśāṃtarūpaḥ          svabhāvād eva bhavati// \U2 %%%410.jpg
%------------------------------
%Auf diese Weise nimmt auch das Selbst aufgrund seiner inhärenten Natur die Form eines Menschen, Vogels, einer Gazelle, eines Elefants, eines Vidyādharas, eines Gandharvas, Zentauren, eines großen Gelehrten oder großen Dummkopfes, eines Kranken oder Gesunden, eines Zornigen oder Friedlichen an. 
%------------------------------
%------------------------------
%tathaivātmā   manuṣyapakṣihariṇahastividyādharagandharvakinnaramahāpaṃḍitamahāmūrkha  rogyarogikrodhi---śāṃtarūpaḥ      svabhāvād eva bhavati/ \E
%tathaivātmā   manuṣyapakṣihariṇāhastividyādharagaṃdharvakinnaramahāpiṃḍitamahārmūkha  rogī-----krodhi---śāṃtarūpāḥ      svabhāvād eva bhavati \P
%tathaivātmā// manuṣyapakṣihariṇahastividyādharagaṃdharvakinnaramahāpiṃḍatamahāmūrkha  rogī-----krodhadhiśāṃtarūpaḥ      svabhāvād eva bhavatī/ \B
%tathaivātmā   manuṣyapakṣihariṇahastividyādharagaṃdharvakinnaramahāpaṃḍitamahāmūrkha  rogī-----krodhadhīśāṃtarūpāḥ      svabhāvād eva bhavatī/ \L
%tathātmā//    manuṣyapakṣihariṇahastīvidyādharagandharvakiṃnaramahāpaṃḍitamahāmūrva   rogīarogīkrodhī---śāntarūpa-------svabhāvād eva bhati/ \N1 %%%%%%%CRAZY SWITCH BETWEEN DAṆḌA AND COMMA
%tathātmā//    manuṣyapakṣihariṇahastīvidyādharagandharvakinnaramahāpaṃḍitamahāmūrva   rogīarogīkrodhī---śāṃtarūpa-------svabhāvād eva bhavati/ \D
%tathātmā//    manuṣyapakṣihariṇahastividyādharagandharvakinnaramahāpaṇḍitamahāmūrkha  rogīarogīkrodhī---śāṃtarūpa-------svabhāvād eva bhavati/ \N2
%                                     vidyādharagaṃdharvakinnaramahāpaṇḍitamahāmūrṣa   rogīarogīkrodhī---śāṃtarūpa       evaṃ svabhāvaṃ dharati  \U1
%tathaivātmā   manuṣyapakṣihariṇahastividyādharagaṃdharvakinnaramahāpaṃḍitamahāmūrkha  rogīarogīkrodhi---śāṃtarūpaḥ      svabhāvād eva bhavati// \U2 %%%410.jpg
%------------------------------
%Auf diese Weise nimmt auch das Selbst aufgrund seiner inhärenten Natur die Form eines Menschen, Vogels, einer Gazelle, eines Elefants, eines Vidyādharas, eines Gandharvas, Zentauren, eines großen Gelehrten oder großen Dummkopfes, eines Kranken oder Gesunden, eines Zornigen oder Friedlichen an.
%
%In this way, the self also takes the form of a human, a bird, a gazelle, an elephant, a vidyādhara, a gandharva, a centaur, great scholar or a great fool, a sick or healthy, an angry or or peaceful person, by virtue of its inherent nature.       
%------------------------------      
\note[type=source, labelb=146, lem={tathaivātmā}]{Ysv (PT): tathā ca devagandharvakinnarādyāḥ khagādayaḥ | sukhasampiṇḍito rogī tathaiva krodhaśāntadhīḥ |aśeṣarūpabalito nānābuddhirataḥ svayam | devatattvaṃ bhūtaśaktyā jīvasaṃjñā bhramātmikā | jñānayogī nirvikāro nistāpa eka īśvaraḥ | ātmaikamūrttimān bhūtvā nirvikalpo nirañjanaḥ | sukhī duḥkhī mohayukto 'nantacetāḥ svabhāvataḥ |}
\app{\lem[wit={E,P,B,L,U2}]{tathaivātmā}
  \rdg[wit={ceteri}]{tathātmā}}
\app{\lem[wit={ceteri},alt={manuṣya°}]{manuṣya}
  \rdg[wit={U1}]{\om}
}\app{\lem[wit={ceteri},alt={°pakṣi°}]{pakṣi}
  \rdg[wit={U1}]{\om}
}\app{\lem[wit={ceteri},alt={°hariṇa°}]{hariṇa}
  \rdg[wit={P}]{°hariṇā°}
  \rdg[wit={U1}]{\om}
}\app{\lem[wit={N1,D},alt={°hastī°}]{hastī}
  \rdg[wit={ceteri}]{hasti}
  \rdg[wit={U1}]{\om}
}vidyādharagaṃdharvakinnaramahā\app{\lem[wit={ceteri},alt={°paṇḍita°}]{paṇḍita}
  \rdg[wit={B}]{piṃḍata}
}mahā\app{\lem[wit={ceteri},alt={°mūrkha°}]{mūrkha}
  \rdg[wit={P}]{°rmūkha°}
  \rdg[wit={N1,D}]{°mūrva°}
  \rdg[wit={U1}]{°mūrṣa°}
}\app{\lem[type=emendation, resp=egoscr]{rogyarogī}
  \rdg[wit={E}]{\korr °rogyarogi}
  \rdg[wit={N1,N2,D,U1,U2}]{°rogī arogī}
  \rdg[wit={P,B,L}]{°rogī}
}\app{\lem[wit={ceteri},alt={°krodhī°}]{krodhī}
  \rdg[wit={E,P}]{°krodhi°}
  \rdg[wit={B,L}]{°krodha°}
}\app{\lem[wit={ceteri},alt={°śānta°}]{śānta}
  \rdg[wit={B,L}]{°dhiśānta°}
}\app{\lem[wit={ceteri},alt={°rūpaḥ}]{rūpaḥ}
  \rdg[wit={P,L}]{°rūpāḥ}
  \rdg[wit={N1,N2,D,U1}]{°rūpa}}
\app{\lem[wit={ceteri},alt={svabhāvād eva}]{svabhāvād-eva}
  \rdg[wit={U1}]{evaṃ svabhāvaṃ}}
\app{\lem[wit={ceteri}]{bhavati}
  \rdg[wit={B,L}]{bhavatī}
  \rdg[wit={N1}]{bhati}
  \rdg[wit={D}]{dharati}}\dd{}
%------------------------------      
%jñānayogādhikārarūparahito  jñāyate/  yathā plakṣasyotpattiḥ/ sthānam eva bhavati// \E
%jñānayogādhikārarūparahito  jñāyate   yathā phalasyotpattisthānam ekam eva bhavati \P  %%%7643.jpg                          
%jñānayogādhikārarūparahito  jñāyate// yathā phalasyotpattisthānam ekam eva bhavatī// \B
%jñānayogādhikārarūparahito  jñāyate// yathā phalasyotpattisthānam ekam eva bhavati// \L
%jñānayogād vikārarūparahito jñāyate/  yathā phalasyotpattisthānam ekam eva bhavati/ \N1
%jñānayogādhikārarūparahito  jñāyate// yathā phalasyotpattisthānam ekaseva  bhavati// \D
%jñānayogadhikārarūparahito  jñāyate// yathā phalasyotpattisthānam eva kameva bhavati// \N2
%jñānayogāt vikārarūparahito jñāyate   yathā phalasyotpattisthāna  ekam eva ti \U1
%jñānayogādhikārarūparahito  jāyate//  yathā phalasyotpattisthānam ekam eva bhavati// \U2
%------------------------------
%em. zu jñānayogādhikāriṇā? em. zu vikāraṃ rūparahitaṃ -> By the man of Jñānayoga the modifications are known as formless.?! Just as the place of origin of the fruit is only one.
%Aufgrund von Jñānayoga wird die Umwandlung als Formlos erkannt.
%Because of jñānayoga, transformation is recognized as formless. 
%------------------------------
\app{\lem[wit={N1,U1}, alt={jñānayogād vikāra}]{jñānayogād-vikāra}
  \rdg[wit={ceteri}]{jñānayogadhikāra}
}rūparahito
\app{\lem[wit={ceteri}]{jñāyate}
  \rdg[wit={U2}]{jāyate}}\dd{}
yathā
\app{\lem[wit={ceteri}]{phalasyotpatti}
  \rdg[wit={E}]{plakṣasyotpattiḥ}
}\app{\lem[wit={ceteri},alt={°sthānam}]{sthāna\skp{m-e}}
  \rdg[wit={E}]{sthānam}
  \rdg[wit={U1}]{°sthāna}
}\app{\lem[wit={ceteri},alt={ekam}]{\skm{m-e}ka\skp{m-e}}
  \rdg[wit={D}]{ekas}
  \rdg[wit={N2}]{eva}
  \rdg[wit={E}]{\om}
}\app{\lem[wit={ceteri},alt={eva}]{\skm{m-e}va}
  \rdg[wit={N2}]{kam eva}}
\app{\lem[wit={ceteri}]{bhavati}
  \rdg[wit={B}]{bhavatī}
  \rdg[wit={U1}]{ti}}/
%------------------------------
%atha ca phalasya gatir bahudhā dṛśyate/ \E
%atha ca phalasya gati  bahudhā dṛśyate    \P
%atha ca phalasya gatir bahudhā dṛśyate// \B
%atha ca phalasya gatir bahudhā dṛśyate// \L
%atha ca phalasya gatir bahudhā dṛśyate/ \N1
%atha ca phalasya gatir bahudhā dṛśyate// \D
%atha ca phalasya gati  bahudhā dṛśyate/ \N2
%atra ca phalasya gati  bahudhā dṛśyate \U1
%atha ca phalasya gatir bahudhā dṛśyate// \U2
%------------------------------
%But the path of the fruit is seen manifold. 
%------------------------------
atha ca phalasya \app{\lem[wit={ceteri},alt={gatir}]{gati\skp{r-ba}}
  \rdg[wit={P,N2,U1}]{gati}
}\skm{r-ba}hudhā dṛśyate\dd{}
%------------------------------ %%%STEMMAPOINT śuklaṃ//śuṣkaṃ
% ekaṃ phalaṃ pṛthvīmadhye  patati/  śuklaṃ bhavati/   \E
% ekaṃ phalaṃ pṛthvīmadhye  patati   śuklaṃ bhavati    \P
% ekaṃ phalaṃ pṛthvīmadhye  patiśuklaṃ      bhavatī//  \B
% ekaṃ phalaṃ pṛthvīmadhye  patati   śuṣkaṃ bhavatī    \L
% ekaṃ phala--pṛthvīmadhye  patati/  śuklaṃ bhavati/   \N1 %%%p.7 recto letzte Zeile 
% ekaṃ phala--pṛthvīmadhye  patati// śuklaṃ bhavati//  \D
% eva  phala--pṛthvīmadhye  patati   śuklaṃ bhavati//  \N2
% ekaṃ phalaṃ pṛthivīmadhye patati   śuṣkaṃ bhavati    \U1
% ekaphalaṃ   pṛthvīmadhye  patati// śuṣkaṃ bhavati//  \U2
%------------------------------
%One fruit falls onto the ground. It is getting bright.  Dürre entsteht// Licht entsteht// Es wird hell!!!. 
%------------------------------
\app{\lem[wit={ceteri}]{ekaṃ}
  \rdg[wit={U2}]{eka°}
  \rdg[wit={N2}]{eva}}
\app{\lem[wit={ceteri}]{phalaṃ}
  \rdg[wit={N1,N2,D}]{phala°}}
\app{\lem[wit={ceteri},alt={pṛthvī°}]{pṛthvī}
  \rdg[wit={U1}]{pṛthivī°}
}madhye patati/
\app{\lem[wit={ceteri}]{śuklaṃ}
  \rdg[wit={L,U1,U2}]{śuṣkaṃ}}
\app{\lem[wit={ceteri}]{bhavati}
  \rdg[wit={B}]{bhavatī}}/
%------------------------------
% ekasya phalasya makaraṃdaṃ bhramaraḥ  pibati/  \E
% ekasya phalasya makaraṃdaṃ bhramaraḥ  pibaṃti  \P
% ekasya            karaṃdaṃ bhramaraṃ  pibatī/  \B
% ekasya          makaraṃdaṃ bhramaraṃ  pibati   \L
% ekasya phalasya makaraṃdabhramaraḥ    pibati/  \N1 %%%p.7 recto letzte Zeile 
% ekasya phalasya makaraṃdabhramaraḥ    pibati/  \D
% ekasya phalasya makaraṃdaṃ bhramara   pibati/  \N2
% ekasya phalasya makaraṃdaṃ bhramanaḥ  pibati   \U1
% ekasya phalasya makaraṃdaṃ bhramaraḥ  pibati// \U2
% ------------------------------
% Eine Biene trinkt den Blumensaft der einen Frucht.
% A bee drinks the flower juice of a fruit.     
%------------------------------
ekasya
\app{\lem[wit={ceteri}]{phalasya}
  \rdg[wit={P,L}]{\om}} 
\app{\lem[wit={E,P,L,N2,U1,U2}]{makaraṃdaṃ}
  \rdg[wit={L,N1}]{makaraṃda°}
  \rdg[wit={B}]{karaṃdaṃ}}
\app{\lem[wit={ceteri}]{bhramaraḥ}
  \rdg[wit={B,L}]{bhramaraṃ}
  \rdg[wit={N2}]{bhramara}}
\app{\lem[wit={ceteri}]{pibati}
  \rdg[wit={P}]{pibaṃti}
  \rdg[wit={B}]{pibatī}}/
%------------------------------
% ekasya phalasya  mālāṃ kāminī tuṃgakucamaṃḍalopari dadhāti/ \E
% ekasya phalasya  mālāṃ kāminī tuṃgakucamaṃḍalopari dadhāti \P
% ekasya phalasya  mālāṃ kāminī tuṃgakucamaṃḍalopari dadhātī// \B
% ekasya phalasya  mālāṃ kāminī tuṃgakucamaṃḍalopari dadhāti// \L
% ekasya phalasya  mālāṃ kāminī tuṃgakucamaṃḍalopari dadhāvati/ \N1 %%%p.7 recto letzte Zeile 
% ekasya phalasya  mālāṃ kāmibī tuṃgakucamaṇḍalopari dadhāti// \D
% ekasya phalasyaṃ mālākāminī   tuṃgakucamaṇḍalopari dadhovati// \N2
% ekasya phalasya  mālāṃ kāmini tuṃ  kucamaṃḍalopari dadhāti \U1
% ekasya phalasya  mālāṃ kāminī tuṃgakucamaṃḍalopari dadhāti// \U2
%------------------------------
% of the one fruit Blütenkranz/Girlande die Verliebte (biene) führt ein unmittelbar über dem Kreis des Blütenstempels der wie eine Brust ist ein.  %tu.mga = hervorstehend 
%Die Verliebte (Biene) platziert sich auf dem Blütenkranz über dem emportstehenden Kreisförmigen Blütenstempel.    
%------------------------------
ekasya
\app{\lem[wit={ceteri}]{phalasya}
  \rdg[wit={N2}]{phalasyaṃ}}
\app{\lem[wit={ceteri}]{mālāṃ}
  \rdg[wit={N2}]{mālā°}}
\app{\lem[wit={ceteri}]{kāminī}
  \rdg[wit={D}]{kāmibī}}
\app{\lem[wit={ceteri},alt={tuṅga°}]{tuṅga}
  \rdg[wit={U1}]{tuṃ°}
}kucamaṃḍalopari
\app{\lem[wit={ceteri}]{dadhāti}
  \rdg[wit={N1}]{dadhāvati}
  \rdg[wit={N2}]{dadhovati}}/
%------------------------------ 
%ekaṃ phalaṃ mṛtamanuṣyopari   kṣipyate/  ayaṃ vastunaḥ svabhāvaḥ/  tathā eka evātmā   svīyabhāvād evāṣṭau    bhogān  bhunakti/ \E
%ekaṃ phalaṃ mṛtamanuṣyopari   kṣipyate   ayaṃ vastunaḥ svabhāvaḥ   tathā eka evātmā   svīyabhāvād evāṣṭau    bhogān  bhunakti \P
%ekaṃ phalaṃ mṛtamanuṣyopari   kṣapyate// ayaṃ vastunaḥ svabhāvaḥ/  tathā eka evātmā   svabhāvād   evāṣṭau    bhogān  bhunakte// \B
%ekaṃ phalaṃ mṛtamanuṣyopari  kṣipyate// ayaṃ vastunaḥ svabhāvaḥ   tathā eka evātmā   svabhāvād   evāṣṭau    bhogān  bhunakte// \L
%ekaphalaṃ   mṛtamanuṣyopari   kṣipyate// ayaṃ vastunaḥ svabhāvaḥ/  tathā eka evātmā   svīyabhāvād evāṣṭau  bhogānā bhunakti/ \N1
%ekaphalaṃ   mṛtamanuṣyopari  kṣipyate// ayaṃ vastunaḥ svabhāvaḥ// tathā eka evātmā   svīyabhāvād evāṣṭau  bhogān  bhunakti// \D
%ekaphalaṃ   mṛtamanuṣyopari   kṣipyate/  ayaṃ castunaḥ svabhāvaḥ/  tathā ekaevātmā    svīyabhāvād evāstau  bhogāt  bhunakti/ \N2
%ekaphalaṃ   mṛtamanuṣyopari   kṣipyate/  ayaṃ castunaḥ svabhāvaḥ/  tathā ekaevātmā    svīyabhāvād evāstau bhogāt  bhunakti/ \U1 %%%276.jpg
%ekaṃ phalaṃ mṛtamanuṣyopari   kṣipyate// ayaṃ vastunaḥ svabhāvaḥ// tathā ekameva ātmā svīyabhāvād evāṣṭabhogān    bhunakti// \U2
%------------------------------
%Die eine Frucht schleudert den Nektar über die Blüte. (em zu anu.s.a = Blüthe?) Dies ist die inhärente Natur der Sache. So genießt auch das eine Selbst genießt aufgrund des eigenen Seins die acht Genüsse. 
%------------------------------
\note[type=testium, labelb=147, lem={svīyabhāvād}]{strīpuṃrūpī mahān so hi parasparavimohitaḥ | amanaskaḥ svīyabhāvāt jñānayogī nirākulaḥ | srakcandanādivāmāsu svabhāvād bhogam icchukaḥ |}
\app{\lem[type=emendation, resp=egoscr,alt={ekaṃ phalam}]{ekaṃ phala\skp{m-a}}
  \rdg[wit={E,P,B,L}]{\korr ekaṃ phalaṃ}
  \rdg[wit={N1,N2,D,U1}]{eka°}}
\app{\lem[type=emendation, resp=egoscr, alt={amṛtam}]{\skm{m-a}mṛta\skp{m-a}}
  \rdg[wit={ceteri}]{\korr mṛta°}
}\app{\lem[type=emendation, resp=egoscr,alt={anuṣṇopari}]{\skp{m-a}nuṣṇopari}
  \rdg[wit={ceteri}]{\korr manuṣyopari}}
\app{\lem[wit={ceteri}]{kṣipyate}
  \rdg[wit={B}]{kṣapyate}}/
%------------------------------
%ke te ṣṭau  bhogāḥ – suvāsaś ca   suvastrañ ca  suśayyā    sunitaṃbinī/       susthānañ cānnapānāni    aṣṭau bhogāś ca dhīmatām/       padṛsūtramayāni vasrāṇi// \E
%ke te ṣṭau  bhogauḥ  suvāsaś ca   suvāsaś   ca  suyyā      sunitāṃbinīḥ//     susthānāś cānpanānp------aṣṭau bhogāś ca dhīmatāṃ 1      padasūtramayāni vastrāṇi?? \P %%%7643.jpg
%      aṣṭau bhogāḥ   suvāsac ca   suvasaś   ca  suśayyāḥ   sūnitaṃbinī/       susthānaś vānnapānāny----aṣṭau bhogāś cā sudhīmatām//1// paṭasūtrāmayāni vasrāṇi// \B
%      aṣṭau bhogāḥ   suvāsaś ca   suvāsaś   ca  suśayyāḥ   sūnitaṃbinī//      susthānāś cānnapānāny----aṣṭau bhogāś cā sudhīmatāṃ//1// paṭasūtrāmayāni vastrāṇi// \L
%ke te ṣṭau  bhogāḥ// suvāyaś ca/                suśayyā    sunitaṃbinī/       susthātāś cātmapanasyā----ṣṭau bhogāḥ    sudhipaṇa----------padṛsūtrayāni   vasrāṇi \N1
%ke te ṣṭau  bhogāḥ// suvāyaś ca//               suśayyā    sunittaṃbinī//     susthātāś cānmanasyā------ṣṭau bhogāḥ    sudhiṣaṇa----------padṛsūtrayāni   vasrāṇi   \D
%ke te ṣṭau  bhogāḥ   suvāyaś ca                 suśayya    sunitaṃbinī/       susthānāś cānmanasyā------ṣṭau bhogāḥ    sudhiyane          padṛsūtrayāni   vasrāṇi \N2
%ke te ṣṭe   bhogā –  suvāsaś ca                 suśayyā ca sunītavinīta       susthātāś cānnapānaḥ syādaṣṭau bhogāḥ    sudhiṣaṇāṃ         padṛsūtramayāni vasrāṇi \U1
%ke te aṣṭau bhogā // suvāsaś ca// suvaṃśaś ca// suśayyā//  sunitaṃbinī//      sudehaṃ// sukhasaṃtānaṃ// abhayādicāṣṭakaṃ//              paṭasūtramayāni vasrāṇi \U2
%------------------------------
%What are the eight enjoyments? A beautiful dwelling, good clothing, a good bed, a well-trained horse?, a nice place, food & drink. Tose are the eight enjoyments of the wise. Clothes made from silk.
%------------------------------
\app{\lem[wit={ceteri}]{ke te}
  \rdg[wit={L,B}]{\om}}
\app{\lem[wit={ceteri}]{'ṣṭau}
  \rdg[wit={L,B}]{aṣṭau}
  \rdg[wit={U1}]{ṣṭe}}
\app{\lem[wit={ceteri}]{bhogāḥ}
  \rdg[wit={P}]{bhobauḥ}
  \rdg[wit={U1,U2}]{bhogā}}
\end{prose}
\end{ekdosis}
%%%%%%%%%%%
%%%%%%%%%%
%%%%%%%%%%
%%%%%%%%%%%
\begin{ekdosis}
  \ekddiv{type=ed}
  \begin{tlg}
\tl{\app{\lem[wit={ceteri},alt={suvāsaś ca}]{suvāsaś-ca}
  \rdg[wit={B}]{suvāsac ca}}
\app{\lem[wit={E},alt={suvastrañ ca}]{suvastrañ-ca}
  \rdg[wit={U2}]{suvaṃśaś ca}}
\app{\lem[wit={ceteri}]{suśayyā}
  \rdg[wit={U1}]{suśayyā ca}
  \rdg[wit={L,B}]{suśayyāḥ}
  \rdg[wit={P}]{suyyā}}
\app{\lem[wit={ceteri}]{sunitaṃbinī}
  \rdg[wit={P}]{sunitāṃbinīḥ}
  \rdg[wit={U1}]{sunītavinīta}}/}\\
\tl{\app{\lem[wit={E},alt={susthānañ}]{susthāna\skp{ñ-cā}}
  \rdg[wit={P,L,N2}]{susthānāś}
  \rdg[wit={N1,D,U1}]{susthātāś}
  \rdg[wit={U2}]{sudehaṃ}
}\app{\lem[wit={L},alt={°ānnapānāny}]{\skm{ñ-cā}nnapānā\skp{ny-a}}
  \rdg[wit={B}]{°vānna°}
  \rdg[wit={E}]{°pānāni}
  \rdg[wit={P}]{cānpanānp°}
  \rdg[wit={N1}]{cātmapanasyā°}
  \rdg[wit={N2,D}]{cānmanasyā°}
  \rdg[wit={U1}]{cānnapānaḥ syād°}
  \rdg[wit={U2}]{sukhasaṃtānaṃ}
}\app{\lem[wit={E,P},alt={aṣṭau bhogāś ca dhīmatām}]{\skm{ny-a}ṣṭau bhogāś-ca dhīmatām}
  \rdg[wit={B,L}]{aṣṭau bhogāś cā sudhīmatām}
  \rdg[wit={N1}]{ṣṭau bhogāḥ sudhipaṇa°}
  \rdg[wit={D}]{ṣṭau bhogāḥ sudhiṣaṇa°}
  \rdg[wit={U1}]{aṣṭau bhogāḥ sudhiṣaṇāṃ}
  \rdg[wit={U2}]{abhayādicāṣṭakaṃ}
  \rdg[wit={N1,N2,D,U1}]{aṣṭau bhogāḥ}
  \rdg[wit={U2}]{abhayādicāṣṭakaṃ}}\dd{}1\hskip-2pt\dd{}}
\end{tlg}
\end{ekdosis}
\ekdpb*{}
%%%%%%%%%%%%%%%%%%%%%%%%%%%%%%%%%%%%%%%%%%
%%%%%%%%%%%%%%%%%%%%%%%%%%%%%%%%%%%%%%%%%%
%%%%%%%%PAGEBREAK%%%%%%%PAGEBREAK%%%%%%%%%
%%%%%%%%%%%%%%%%%%%%%%%%%%%%%%%%%%%%%%%%%%
%%%%%%%%%%%%%%%%PAGEBREAK%%%%%%%%%%%%%%%%%
%%%%%%%%%%%%%%%%%%%%%%%%%%%%%%%%%%%%%%%%%%
%%%%%%%%PAGEBREAK%%%%%%%PAGEBREAK%%%%%%%%%
%%%%%%%%%%%%%%%%%%%%%%%%%%%%%%%%%%%%%%%%%%
%%%%%%%%%%%%%%%%%%%%%%%%%%%%%%%%%%%%%%%%%%
%%%%%%%%%%%%%%%%%%%%%%%%%%%%%%%%%%%%%%%%%%
%%%%%%%%%%%%%%%%%%%%%%%%%%%%%%%%%%%%%%%%%%
%%%%%%%%PAGEBREAK%%%%%%%PAGEBREAK%%%%%%%%%
%%%%%%%%%%%%%%%%%%%%%%%%%%%%%%%%%%%%%%%%%%
%%%%%%%%%%%%%%%%PAGEBREAK%%%%%%%%%%%%%%%%%
%%%%%%%%%%%%%%%%%%%%%%%%%%%%%%%%%%%%%%%%%%
%%%%%%%%PAGEBREAK%%%%%%%PAGEBREAK%%%%%%%%%
%%%%%%%%%%%%%%%%%%%%%%%%%%%%%%%%%%%%%%%%%%
%%%%%%%%%%%%%%%%%%%%%%%%%%%%%%%%%%%%%%%%%%
%%%%%%%%%%%%%%%%%%%%%%%%%%%%%%%%%%%%%%%%%%
%%%%%%%%%%%%%%%%%%%%%%%%%%%%%%%%%%%%%%%%%%
%%%%%%%%PAGEBREAK%%%%%%%PAGEBREAK%%%%%%%%%
%%%%%%%%%%%%%%%%%%%%%%%%%%%%%%%%%%%%%%%%%%
%%%%%%%%%%%%%%%%PAGEBREAK%%%%%%%%%%%%%%%%%
%%%%%%%%%%%%%%%%%%%%%%%%%%%%%%%%%%%%%%%%%%
%%%%%%%%PAGEBREAK%%%%%%%PAGEBREAK%%%%%%%%%
%%%%%%%%%%%%%%%%%%%%%%%%%%%%%%%%%%%%%%%%%%
%%%%%%%%%%%%%%%%%%%%%%%%%%%%%%%%%%%%%%%%%%
%------------------------------
%padṛsūtramayāni vasrāṇi//  \E
%padasūtramayāni vastrāṇi?? \P %%%7643.jpg
%paṭasūtrāmayāni vasrāṇi//  \B
%paṭasūtrāmayāni vastrāṇi// \L
%padṛsūtrayāni   vasrāṇi    \N1
%padṛsūtrayāni   vasrāṇi    \D
%padṛsūtrayāni   vasrāṇi    \N2
%padṛsūtramayāni vasrāṇi    \U1
%paṭasūtramayāni vasrāṇi    \U2
%------------------------------
%Clothes made from silk,...
%------------------------------
%paṃcasaptā dṛālikā         yuktāni harmyāṇi teṣu vāsaḥ    ativipulā  mṛdutarasukhāsuśayyā/     \E
%paṃcasaptā dadhikā         yuktāni harmyāṇi teṣu cāsaḥ 2  ativipulā  mṛduttarachadavatīśayyā 2  \P
%paṃcasatyā dātikā          yuktāni harmyāṇi teṣu vāstu    ativipulā  mṛdutaralāśayyā//2//        \B
%paṃcasatyā dātikā          yuktāni harmyāṇi teṣu vāstu    ativipulā  mṛdutaralāśayyā//3//        \L
%paṃca vā sapta vā dṛālikā  yuktāni harmyāṇi/              ativapulā  mṛdu/uttaracchaṃdavatīśayyā/  \N1
%paṃca vā sapta vā dṛāṃlikā yuktāni harmyāṇi               ativapulā  mṛduuttarachaṃdavatīśayyā/     \D
%paṃca vā sapta vā tālikā---yuktāni harmyāni               ativipulā  mṛduuttarachaṃdavatīśayyā    \N2
%paṃca vā sapta vā dālikā---yuktāni harmyāṇi               ativipulāṃ mṛduuttarachadavatiśaiyyā     \U1
%--------------------------saudhāni harmyāṇi vāsāya kecit// aṣṭau bhogān āha// sugrahaṃ// suvastraṃ// suśayā sustrī//  \U2
%--------------------------------------------
%,a site of the palace in which there are mainsions endowned with five or seven rooms. A huge, very soft and lovely bed.  
%-------------------------------------------
\begin{ekdosis}
  \ekddiv{type=ed}
  \begin{prose}
\noindent
    \app{\lem[type=emendation, resp=egoscr,alt={paṭṭa°}]{paṭṭa}
  \rdg[wit={E,N1,D,N2,U1}]{\korr padṛ°}
  \rdg[wit={P}]{pada°}
  \rdg[wit={B,L,U2}]{paṭa°}
}\app{\lem[wit={ceteri},alt={sūtra°}]{sūtra}
  \rdg[wit={B,L}]{sūtrā}
}\app{\lem[wit={ceteri}]{mayāni}
  \rdg[wit={N1,N2,D}]{yāni}}
\app{\lem[wit={P,L}]{vastrāṇi}
  \rdg[wit={ceteri}]{vasrāṇi}} 1\dd{}
\app{\lem[wit={N1,N2,D,U1}]{paṃca vā sapta vā}
  \rdg[wit={E,P}]{paṃcasaptā}
  \rdg[wit={L,B}]{paṃcasatyā}}
\app{\lem[type=emendation, resp=egoscr]{śālikā}
  \rdg[wit={E,N1}]{\korr dṛālikā}
  \rdg[wit={D}]{dṛāṃlikā}
  \rdg[wit={P}]{dadhikā}
  \rdg[wit={B,L}]{dātikā}
  \rdg[wit={N2}]{tālikā}
  \rdg[wit={U1}]{dālikā}
}\app{\lem[wit={ceteri}]{yuktāni}
  \rdg[wit={U2}]{saudhāni}}
harmyāṇi
\app{\lem[wit={L,B}]{teṣu vāstu}
  \rdg[wit={E}]{teṣu vāsaḥ}
  \rdg[wit={P}]{teṣu cāsaḥ}
  \rdg[wit={U2}]{vāsāya kecit}
  \rdg[wit={ceteri}]{\om}} 2\dd{}
\app{\lem[wit={ceteri}]{ativipulā}
  \rdg[wit={N1,D}]{ativapulā}
  \rdg[wit={U1}]{ativipulāṃ}
  \rdg[wit={U2}]{aṣṭau bhogān āha ||}}
\app{\lem[type=emendation, resp=egoscr,alt={mṛdūttara}]{mṛdūttara}
  \rdg[wit={E,P,L,B}]{\korr mṛdutara°}
  \rdg[wit={N1,N2,D,U1}]{mṛdu | uttara°}
  \rdg[wit={U2}]{sugrahaṃ ||}
}\app{\lem[wit={N1,N2,D},alt={°chandavatī°}]{chandavatī}
  \rdg[wit={P}]{°chadavatī°}
  \rdg[wit={U1}]{°chadavati°}
  \rdg[wit={U2}]{suvastraṃ ||}
}\app{\lem[wit={ceteri}]{śayyā}
  \rdg[wit={U2}]{suśayā sustrī}} 3\dd{}
%------------------------------
%padminī tārūṇyavatī  manoharā guṇavatī  tatropaviṣṭā kāṃtā/      \E
%padminī tārūṇyavatī  manoharā guṇavatī  tatopaviṣṭā  kāṃtā 4     \P
%padminī tārūnyavatī  manoharā guṇavatī//tatrāpavistā kāṃtā 4     \B
%padminī tārūnyavatī  manoharā guṇavatī//tatropavistā kāṃtā// 4// \L
%padmanī tārūṇyavatī  manoharā guṇavatī  tatropavistā//           \N1
%padminī tārūrāyavatī manoharā guṇavatī  tatropavistā//           \D
%padminī tārūnyavatī  manoharā guṇavatī  tatropavistā             \N2
%padminī tārūnyavati  manoharā guṇavati  tatropavistā             \U1
%                                                                 \U2
%--------------------------------------------
%[On which] there is situated [tatropaviṣṭā] a lotus-like [em. zu tāruṇyavatī] youthful, charming and virtuous wife. 
%-------------------------------------------
\app{\lem[wit={ceteri}]{padminī}
  \rdg[wit={N1}]{padmanī}
  \rdg[wit={U2}]{\om}}
\app{\lem[type=emendation, resp=egoscr]{tāruṇyavatī}
  \rdg[wit={ceteri}]{\korr tārūṇyavatī}
  \rdg[wit={N2}]{tārūrāyavatī}
  \rdg[wit={U2}]{\om}}
\app{\lem[wit={ceteri}]{manoharā guṇavatī}
   \rdg[wit={ceteri}]{tatropavistā}
  \rdg[wit={P}]{tato°}
  \rdg[wit={B}]{tatrā°}
  \rdg[wit={U2}]{\om}}
\app{\lem[wit={E,P,B,L}]{kāntā}
  \rdg[wit={ceteri}]{\om}} 4\dd{}
%------------------------------
%sādhu āśanam/      atimūlyañ ca/         manoramam annaṃ।       tathā vidhaṃ pānam/   \E
%sādhu āsanaṃ 5     atimūlo 'śvaḥ 6       manoramam annaṃ    7   tathā vidhaṃ pānaṃ 8  \P
%sādhu āsanaṃ 5     atimūlyo asvaṃ//6     manoramyam attaṃ //7   tathā vidhapānaṃ//8   \B
%sādhu āsanaṃ// 5// atimūlyo aśvaṃ//6//   manoramyam annaṃ //7// tathā vidhapānaṃ//8// \L
%sādhyāsanaṃ//      amūlyo svaś ca//      manoramam attaṃ        tathā vidhaṃ pānaṃ/   \N1
%sādhyāsanaṃ//      amūlyo svaś ca//      manoramam attaṃ        tathā vidhaṃ pānaṃ//  \D
%sādhyāsanaṃ        amūlyo svaś ca        manotamam annaṃ        tathā vidhapānaṃ//    \N2
%sādhyāsanaṃ        amolyo svaś ca        manoramam annaṃ        tathā vidhaṃ pānaṃ    \U1
%sādhu āsanaṃ//           suśvaḥ//        suṣṭu annaṃ//          tathā vidhayānaṃ//    \U2
%--------------------------------------------
%good throne/seat; atimūlyo (überaus wertvolles) 'śvaṃ (Pferd), manorama ( die Sinne erfreuendes) Essen, verschiedenes Trinken. 
%-------------------------------------------
\app{\lem[type=emendation, resp=egoscr]{sādhvāsanaṃ}
  \rdg[wit={E}]{\korr sādhu āśanam}
  \rdg[wit={P,B,L,U2}]{sādhu āsanaṃ}
  \rdg[wit={N1,N2,D}]{sādhyāsanaṃ}} 5\dd{}
\app{\lem[type=emendation, resp=egoscr]{atimūlyo 'śvaḥ}
  \rdg[wit={E}]{\korr atimūlyañ ca}
  \rdg[wit={P}]{atimūlo 'śvaḥ}
  \rdg[wit={L,B}]{atimūlyo asvaṃ}
  \rdg[wit={N1,N2,D,U1}]{amūlyo svaś ca}
  \rdg[wit={U2}]{suśvaḥ}} 6\dd{}
\app{\lem[wit={ceteri},alt={manoramam annaṃ}]{manoramam-annaṃ}
  \rdg[wit={B}]{manoramyam attaṃ}
  \rdg[wit={L}]{manoramyam annaṃ}
  \rdg[wit={N1,D}]{manoramam attaṃ}
  \rdg[wit={U2}]{suṣṭu annaṃ}} 7\dd{}
tathā
\app{\lem[wit={ceteri}]{vidhaṃ pānaṃ}
  \rdg[wit={L,B,N2}]{vidhapānaṃ}
  \rdg[wit={U2}]{vidhayānaṃ}} 8\dd{}
%------------------------------
%ete   ṣṭau bhogāḥ   kathitāḥ/   eke  duḥkhaṃ   bhajante/  bhikṣāṃ  yācante// kiñca \E
%ete   ṣṭau bhogāḥ   kathitā 9   eke  duḥkha    bhajaṃte   bhikṣāṃ  yāṃcaṃte ca  \P
%ete   ṣṭau bhogāḥ//             eka  duḥkhā    bhajaṃte/  bhikṣā   yāṃcate ca//  \B
%ete   ṣṭau bhogāḥ//             eka  duḥkhā    bhajaṃte// bhikṣā   yāṃcate ca//  \L
%ete  aṣṭau bhogā    kathyate/   eke  duḥkhaṃ   bhajaṃte/  bhikṣyāṃ yācate ca/   \N1
%ete  aṣṭau bhogāḥ   kathyaṃte// ete  duḥkhaṃ   bhajaṃte/  bhikṣyāṃ yācaṃte ca// \D
%ete  aṣṭau ghogā    kathyate//  ete  duḥkhataṃ bhajate    bhikṣāṃ  yācate ca//  \N2
%rāte aṣṭau bhogāḥ   kathyate    ete  duḥkhaṃ   bhajate    bhikṣāṃ  pācate ca    \U1
%ete  ṣṭau  bhogāḥ// kathitāḥ//  ekaṃ duḥkhaṃ   bhajaṃte// bhikṣā   yācaṃte ca// \U2
%------------------------------
%die acht genüsse wurden erzählt. Sie bringen Leid und die Bet. 
%------------------------------
%\note[type=philcomm, labelb=148, lem={'ṣṭau bhogāḥ}]{The eight enjoyments are not attested in any of the sources.}
\app{\lem[wit={ceteri}]{ete}
  \rdg[wit={U1}]{rāte}}
\app{\lem[wit={ceteri}]{'ṣṭau}
  \rdg[wit={N1,N2,D,U1}]{aṣṭau}}
\app{\lem[wit={ceteri}]{bhogāḥ}
  \rdg[wit={N1,N2}]{bhogā}
  \rdg[wit={U1}]{ghogā}}
\app{\lem[wit={E,U2}]{kathitāḥ}
  \rdg[wit={P}]{kathitā}
  \rdg[wit={N1,N2,U1}]{kathyate}
  \rdg[wit={D}]{kathyaṃte}
  \rdg[wit={L,B}]{\om}}\dd{}
%------------------------------
%      yathā sūryasya tejaḥ   dugdhasya    ghṛtam   agner jvalanaṃ viṣān mūrchā   tilāttailam/    vṛkṣāc-chāyā/  phalāt parimalaḥ       kāṣṭhād agniḥ    arkarādibhyo   madhuro rasaḥ/ \E
%      yathā sūryasys tejaḥ   dugdhasya    ghṛtaḥ   agne dvāhaḥ    viṣān mūrchāti tilāttailaṃ     vṛkṣāt-chāyā   phalāsarimalaḥ         kāṣṭād  agniḥ    śarkvarādibhyo madhuro rasaḥ  \P
%      yathā sūryasye tejāḥ   dugdha-------ghṛtaḥ   agne dvāhaḥ//  viṣān mūrchā   tilāttailaṃ//   vṛkṣā--chāyā   phalāsarimalaḥ         kaṣṭād  agniḥ    śarkadībhyo    madhuro  \B
%      yathā sūryasya tejāḥ   dugdha-------ghṛtaḥ   agne dvāhaḥ//  viṣān mūrchā   tilātailaṃ//    vṛkṣā--chāyā   phalāt parimalaḥ       kaṣṭād  agniḥ    śarkadībhyo    madhuro  \L
%      yathā sūryasya tejaḥ/  dugdhasya    ghṛtaṃ/  agne dahiḥ??   viṣān mūrchā   tilāttailaṃ,    vṛkṣāc-chāyā/  phalāt parimalaḥ/      kāṣṭhād āgniḥ/   śarkkarādibhyo madhuro rasaḥ/ \N1
%      yathā sūryasya tejaḥ// dugdhasya    ghṛtaṃ// agne dadhiḥ    viṣān mūrchā   tilāttailaṃ//   vṛkṣā--chāyā// phalāt palātparimalaḥ//kāṣṭhād āgniḥ//  śarkarādibhyo  madhuro rasaḥ/ \D
%      yathā sūryasya tejaḥ   dusya        ghṛtaṃ   agne dadhi     viṣān mūrchā   tilatailaṃ      vṛkṣā--chāyā   phalāt parimalaḥ       kāṣṭhād āgniḥ    śarkarādibhyo  madhuro rasaḥ/ \N2
%      yathā sūryaśca tejaḥ   dugdhasy     ghṛttaṃ  agne dārhaṃ    viṣāt mūrchā   tilātailaṃ      vrakṣā-chāyā   phalāt parimalaḥ       kāṣṭhād āgniḥ    śarkarādibhyo  madhuro rasaḥ \U1
%      yathā sūryasya tejaḥ// dugdhasya    ghṛtaṃ// agne dāhiḥ//   viṣān mūrchā   tilātailaṃ//    vṛkṣā--chāyā// phalāt parimalaḥ//     kāṣṭād  agniḥ    śarkarādibhyo  madhuro rasaḥ// \U2
%------------------------------
%Gleichwie die Strahlen der Sonne, die Butter der Milch, das Brennen des Feuers, die Betäubung aufgrund von Gift, das Sesamöl aus dem Sesamkorn, der Schatten vom Baum, der Wohlgeruch von einer Frucht, das Feuer von einem Holzscheid, der Süße Saft [em. zu śārkara] a liquor prepared from Dhātakī with sugar] und so weiter,   
%------------------------------
%Like the rays of the sun, the butter of milk, the burning of fire, the stupor of poison, the sesame oil from the sesame seed, the shade from the tree, the sweet odor from a fruit, the fire from a scabbard, the sweet sap [em . to śārkara] a liquor prepared from Dhātakī with sugar] and so on,
%------------------------------
\note[type=source, labelb=149, lem={sūryasya}]{Ysv (PT): ravī tejo ghṛtaṃ dugdhe tile tailaṃ svabhāvataḥ | śaśam indau kule śākhaṃ kṣāre ca lavaṇaṃ yathā | tathā brahmaṇi saṃsāro hyakhaṇḍaparipūrvake ||}
yathā
\app{\lem[wit={ceteri}]{sūryasya}
  \rdg[wit={U1}]{sūryaś ca}}
\app{\lem[wit={ceteri}]{tejaḥ}
  \rdg[wit={L,B}]{tejāḥ}}\dd{}
\app{\lem[wit={E,P,N1,D,U2}]{dugdhasya}
  \rdg[wit={L,B}]{dugdha°}
  \rdg[wit={N2}]{dusya}
  \rdg[wit={U1}]{dugdhasy}}
\app{\lem[wit={ceteri}]{ghṛtaṃ}
  \rdg[wit={P,L,B}]{ghṛtaḥ}}\dd{}
\app{\lem[wit={E}, alt={agner}]{agne\skp{r-dā}}
  \rdg[wit={ceteri}]{agne}
}\app{\lem[type=emendation, resp=egoscr, alt={dāhaḥ}]{\skm{r-dā}haḥ}
  \rdg[wit={P,L,B}]{\korr dvāhaḥ}
  \rdg[wit={N1}]{dahiḥ}
  \rdg[wit={N2}]{dadhi}
  \rdg[wit={D}]{dadhiḥ}
  \rdg[wit={U1}]{dārhaṃ}
  \rdg[wit={U2}]{dāhiḥ}
  \rdg[wit={E}]{jvalanaṃ}}\dd{}
\app{\lem[wit={ceteri},alt={viṣān}]{viṣā\skp{n-mū}}
  \rdg[wit={U1}]{viṣāt}
}\skm{n-mū}rchā\dd{}
\app{\lem[wit={ceteri},alt={tilāt}]{tilā\skp{t-tai}}
  \rdg[wit={P}]{titilāt}
  \rdg[wit={N2}]{tila}
  \rdg[wit={U1}]{tilā}
}\skm{t-tai}laṃ\dd{}
\app{\lem[wit={E,N1}, alt={vṛkṣāt}]{vṛkṣā\skp{c-chā}}
  \rdg[wit={P}]{vṛkṣāt}
  \rdg[wit={L,B,N2,D,U2}]{vṛkṣā}
  \rdg[wit={U1}]{vrakṣā}
}\skm{c-chā}yā\dd{}
\app{\lem[wit={ceteri},alt={phalāt}]{phalā\skp{t-pa}}
  \rdg[wit={L,B}]{phalā}
}\app{\lem[wit={ceteri},alt={parimalaḥ}]{\skm{t-pa}rimalaḥ}
  \rdg[wit={L,B}]{sarimalaḥ}
  \rdg[wit={D}]{palāt parimalaḥ}}\dd{}i%\note[type=philcomm, labelb=150, lem={parimalaḥ}]{Clarification: Witness \getsiglum{D} reads \textit{phalāt palāt parimala}.}
\app{\lem[wit={ceteri}, alt={kāṣṭhād}]{kāṣṭhā\skp{d-a}}
  \rdg[wit={P,U2}]{kāṣṭād}
  \rdg[wit={B,L}]{kaṣṭād}
}\app{\lem[wit={ceteri}, alt={agniḥ}]{\skm{d-a}gniḥ}
  \rdg[wit={N1,N2,D,U1}]{āgniḥ}}\dd{}
\app{\lem[type=emendation, resp=egoscr]{śārkarādibhyo}
  \rdg[wit={E}]{\korr arkarādibhyo}
  \rdg[wit={P}]{śarkvarādibhyo}
  \rdg[wit={L,B}]{śarkadībhyo}}
madhuro
\app{\lem[wit={ceteri}]{rasaḥ}
  \rdg[wit={L,B}]{\om}}\dd{}
%------------------------------
%himānībhyaḥ   śītam      ityādipadārthānāṃ svabhāvaḥ         tathā    saṃsāro'pi parameśvarasvarūpamadhye      tiṣṭhati/ \E
%himānībhyaḥ   śītaṃ      ityādipadārthasvabhāva        eva   tathā    saṃsāro'pi parameśvarasvarūpamadhye      tiṣṭhati \P
%sahīmānībhyaḥ śītaḥ/     ityādipadārthāsvabhāvataḥ// eva     tathā    saṃsāro pi paremesvara svarūpasya madhye tiṣṭhatī/ \B
%sahimānibhyaḥ śītaḥ//    ityādiphadārthāḥ svabhāvataḥ// eva  tathā    saṃsāro pi paremesvara svarūpasya madhye tiṣṭhati// \L
%himānibhyaḥ   śaityāṃ    ityādipadārthasvabhāva evā/         tathā    saṃsāro pi parameśvarasvarūpamadhye      tiṣṭhati// \N1
%himānibhyaḥ   śaityaṃ // ityādipadārthasvabhāva eva//        tathā    saṃsāro pi parameśvarasvarūpamadhye      tiṣṭhati// \D 
%himānitpa     śaityāś    atyādipadārtharthasvabhāva eva//    tathā    saṃsāro pi parameśvarasvarūpamadhye      tiṣṭhati \N2
%himānībhyaḥ   śaityaṃ    ityādipadārthasvabhāvaḥ ravaḥ?      tathā vā saṃsāro pi parameśvararūpamadhye         tiṣṭhati/ \U1
%himānībhyaḥ   śītyaṃ//   ityādipadārthāsvabhāva eva//        tathā    saṃsāro pi parameśvarasvarūpamadhye      tiṣṭhaṃti// \U2
%------------------------------
%die Kälte von Schneehaufen, und so weiter ist das inhärente Wesen der Dinge. IN gleicher Weise befindet sich auch der Weltengang im Zentrum der eigenen Gestalt von höchsten Gott.
%the cold of piles of snow, and so on is the inherent essence of things. In the same way, the course of the world is also in the center of the highest God's own form. 
%------------------------------
\app{\lem[wit={ceteri}]{himānībhyaḥ}
  \rdg[wit={L,B}]{sahimānibhyaḥ}
  \rdg[wit={N2}]{himānitpa}}
\app{\lem[wit={D,U1}]{śaityaṃ}
  \rdg[wit={N1}]{śaityāṃ}
  \rdg[wit={U2}]{śītyaṃ}
  \rdg[wit={N2}]{śaityāś}
  \rdg[wit={E,P}]{śītaṃ}
  \rdg[wit={L,B}]{śītaḥ}}\dd{}
\app{\lem[wit={N1,D,P}]{ityādipadārthasvabhāva}
  \rdg[wit={U2}]{°padārthā°}
  \rdg[wit={B}]{ityādipadārthāsvabhāvataḥ}
  \rdg[wit={N2}]{atyādipadārtharthasvabhāva}
  \rdg[wit={U1}]{°svabhāvaḥ}
  \rdg[wit={L}]{ityādiphadārthāḥ svabhāvataḥ}
  \rdg[wit={E}]{ityādipadārthānāṃ svabhāvaḥ}}
\app{\lem[wit={ceteri}]{eva}
  \rdg[wit={N1}]{evā}
  \rdg[wit={U1}]{ravaḥ}
  \rdg[wit={E}]{\om}}\dd{}
\app{\lem[wit={ceteri}]{tathā}
  \rdg[wit={U1}]{tathā vā}}
saṃsāro 'pi
\app{\lem[wit={ceteri}]{parameśvarasvarūpamadhye}
  \rdg[wit={L,B}]{paremesvara svarūpasya madhye}
  \rdg[wit={U1}]{parameśvararūpamadhye}}
\app{\lem[wit={ceteri}]{tiṣṭhati}
  \rdg[wit={B}]{tiṣṭhatī}
  \rdg[wit={U2}]{tiṣṭhaṃti}}\dd{}
%------------------------------
%parameśvaro 'khaṇḍa--paripūrṇaḥ/  \E
%parameśvaro khaṃḍa---paripūrṇaś ca    \P
%parameśvaro khaṃḍa---paripūrṇaś ca// \B
%parameśvaro khaṃḍa---paripūrṇaś ca//  \L
%parameśvaro 'ṣaṃḍa---paripūrṇaś ca//  \N1
%parameśvaro  ṣaṃḍa---paripūrṇaś ca//  \D %%%S.9 verso
%parameśvaro yarāṇḍa--paripūrṇaś ca//  \N2
%parameśvaro khaṃḍaḥ  paripūrṇaś ca   \U1 %%%277.jpg
%parameśvaro 'khaṃḍa--paripūrṇaś ca//   \U2
%------------------------------
%Und der höchste Gott ist unteilbar und das All erfüllend.
%And the Most High God is indivisible and all-filling.
%------------------------------
parameśvaro
\app{\lem[wit={ceteri}, alt={'khaṇḍa°}]{'khaṇda}
  \rdg[wit={N1,D}]{'ṣaṃḍa°}
  \rdg[wit={N2}]{yarānda°}
  \rdg[wit={U1}]{khaṃḍaḥ}
}\app{\lem[wit={ceteri},alt={°paripūrṇaś ca}]{paripūrṇaś\skp{-}ca}
  \rdg[wit={E}]{paripūrṇaḥ}}\dd{}
\end{prose}
\end{ekdosis}
%------------------------------
%idānīṃ lakṣyaṃ       kathyate/ \E
%idānīṃ bāhyalakṣyaṃ  kathyate \P
%idānīṃ ṣāhyalakṣa    kathyate// \B
%idānīṃ bāhyalakṣa    kathyate// \L
%idānīṃ bāhyalakṣaṃ   kathyate// \N1
%idānīṃ bāhyalakṣaṇa  kathyate// \D %%%S.9 verso
%idānīṃ bāhyalakṣaṇa  kathyate/ \N2
%idānīṃ bāhyalakṣyaḥ  kathyate \U1 %%%277.jpg
%idānīṃ bāhyalakṣaṇaṃ kathyate// \U2
%------------------------------
%Now the external fixation is taught.
%------------------------------
\begin{ekdosis}
  \ekddiv{type=ed}
      \bigskip
        \centerline{\textrm{\small{[Bāhyalakṣya]}}}
          \bigskip
          \begin{prose}
                \noindent
\note[type=source, labelb=151, lem={bāhyalakṣyaṃ}]{Ysv (PT): idānīṃ vāhyalakṣāṇi siddhidāni śṛṇu priye | dhāraṇākhyā tu caitāni jñātavyāni viśeṣataḥ |}
idānīṃ
\app{\lem[wit={P}]{bāhyalakṣyaṃ}
  \rdg[wit={E}]{lakṣyaṃ}
  \rdg[wit={B}]{ṣāhyalakṣa}
  \rdg[wit={L}]{bāhyalakṣa}
  \rdg[wit={N1}]{°lakṣaṃ}
  \rdg[wit={D,N2}]{°lakṣaṇa}
  \rdg[wit={U1}]{°lakṣyaḥ}
  \rdg[wit={U2}]{lakṣaṇaṃ}}
kathyate/
%------------------------------
%nāsāgrād ārabhyāṃgulacatuṣṭaya--pramāṇaṃ nīlākāraṃ tejaḥ   pūrṇam ākāśaṃ  lakṣyaṃ  karttavyam/ \E
%nāsāgrād ārabhyāṃgulacatuṣṭaya--pramāṇaṃ nilākāraṃ tejaḥ   pūrṇam ākāśaṃ lakṣyaṃ  kartavyaṃ  \P
%nāsāgrād ārabhyāṃgulacatuṣṭayaṃ pramāṇaṃ nilākāraṃ   jaḥ   pūrṇam ākāśa--lakṣaṃ   kartavyaṃ//    \B
%nāsāgrād ārabhyāṃgulacatuṣṭayaṃ pramāṇaṃ nilākāraṃ tejaḥ// pūrṇam ākāśaṃ lakṣaṃ   kartavyaṃ// \L
%nāsāgrād ārabhyāṃgulacatuṣṭaya--pramāṇaṃ nīlākāraṃ teja----pūrṇam ākāśa--lakṣaṃ   karttavyaṃ \N1
%nāsāgrād ārabhyāṃgulacatuṣṭaya--pramāṇaṃ nīlākāraṃ teja----pūrṇam ākāśa---lakṣaṃ   karttavyaṃ \D
%nāsāgrād ārabhyāṃgulacatuṣṭaya--pramāṇaṃ nirākāraṃ teja----pūrṇam ākāśa---lakṣaṇaṃ karttavyaṃ// \N2
%nāsāgrād ārabhyāṃgulacatuṣṭaya--pramāṇaṃ nīlākāraṃ tejaḥ   pūrṇam ākāśaṃ lakṣyaṃ  karttavyam \U1
%nāsāgrād ārabhyāṃgulacatuṣṭaya--pramāṇaṃ nīlākāraṃ tejaḥ   pūrṇakām ākāśa-lakṣyaṃ  karttavyaṃ \U2 %%%411.jpg
%------------------------------
%Beginning with a four finger wide distance from the tip of the nose, the space[-element?] full of light whose appearance is blue shall be made the object of fixation.
%------------------------------
nāsāgrād-ārabhyāṃgula\app{\lem[wit={ceteri}, alt={catuṣṭaya°}]{catuṣṭaya\skp{-}}
  \rdg[wit={B,L}]{catuṣṭayaṃ}
}pramāṇaṃ
\app{\lem[wit={ceteri}]{nīlākāraṃ}
  \rdg[wit={P,B,L}]{nilākaraṃ}
  \rdg[wit={N2}]{nirākāraṃ}}
\app{\lem[wit={N1,N2,D},alt={teja°}]{teja}
  \rdg[wit={ceteri}]{tejaḥ}
  \rdg[wit={B}]{jaḥ}
}\app{\lem[wit={ceteri}, alt={pūrṇam}]{pūrṇa\skp{m-ā}}
  \rdg[wit={U2}]{pūrṇakām}
}\app{\lem[wit={ceteri},alt={ākāśa°}]{\skm{m-ā}kāśa}
    \rdg[wit={E,P,L,U1}]{ākāśaṃ}
}\app{\lem[wit={E,P,U1,U2}]{lakṣyaṃ}
  \rdg[wit={B,L,N1,D}]{lakṣaṃ}
  \rdg[wit={N2}]{lakṣaṇaṃ}}
kartavyaṃ/
\end{prose}
\end{ekdosis}
\ekdpb*{}
%%%%%%%%%%%%%%%%%%%%%%%%%%%%%%%%%%%%%%%%%%
%%%%%%%%%%%%%%%%%%%%%%%%%%%%%%%%%%%%%%%%%%
%%%%%%%%PAGEBREAK%%%%%%%PAGEBREAK%%%%%%%%%
%%%%%%%%%%%%%%%%%%%%%%%%%%%%%%%%%%%%%%%%%%
%%%%%%%%%%%%%%%%PAGEBREAK%%%%%%%%%%%%%%%%%
%%%%%%%%%%%%%%%%%%%%%%%%%%%%%%%%%%%%%%%%%%
%%%%%%%%PAGEBREAK%%%%%%%PAGEBREAK%%%%%%%%%
%%%%%%%%%%%%%%%%%%%%%%%%%%%%%%%%%%%%%%%%%%
%%%%%%%%%%%%%%%%%%%%%%%%%%%%%%%%%%%%%%%%%%
%%%%%%%%%%%%%%%%%%%%%%%%%%%%%%%%%%%%%%%%%%
%%%%%%%%%%%%%%%%%%%%%%%%%%%%%%%%%%%%%%%%%%
%%%%%%%%PAGEBREAK%%%%%%%PAGEBREAK%%%%%%%%%
%%%%%%%%%%%%%%%%%%%%%%%%%%%%%%%%%%%%%%%%%%
%%%%%%%%%%%%%%%%PAGEBREAK%%%%%%%%%%%%%%%%%
%%%%%%%%%%%%%%%%%%%%%%%%%%%%%%%%%%%%%%%%%%
%%%%%%%%PAGEBREAK%%%%%%%PAGEBREAK%%%%%%%%%
%%%%%%%%%%%%%%%%%%%%%%%%%%%%%%%%%%%%%%%%%%
%%%%%%%%%%%%%%%%%%%%%%%%%%%%%%%%%%%%%%%%%%
%%%%%%%%%%%%%%%%%%%%%%%%%%%%%%%%%%%%%%%%%%
%%%%%%%%%%%%%%%%%%%%%%%%%%%%%%%%%%%%%%%%%%
%%%%%%%%PAGEBREAK%%%%%%%PAGEBREAK%%%%%%%%%
%%%%%%%%%%%%%%%%%%%%%%%%%%%%%%%%%%%%%%%%%%
%%%%%%%%%%%%%%%%PAGEBREAK%%%%%%%%%%%%%%%%%
%%%%%%%%%%%%%%%%%%%%%%%%%%%%%%%%%%%%%%%%%%
%%%%%%%%PAGEBREAK%%%%%%%PAGEBREAK%%%%%%%%%
%%%%%%%%%%%%%%%%%%%%%%%%%%%%%%%%%%%%%%%%%%
%%%%%%%%%%%%%%%%%%%%%%%%%%%%%%%%%%%%%%%%%%
\begin{ekdosis}
  \begin{prose}
    \noindent
%------------------------------
%atha vā nāsāgrād ārabhya ṣaḍaṃgulapramāṇaṃ    pavanatattvaṃ dhūmrākāraṃ        lakṣyaṃ karttavyam// \E
%atha vā nāsāgrād ārabhya ṣaḍaṃgulapramāṇaṃ    pavanatatvaṃ  dhūmrākāraṃ        lakṣyaṃ karttavyam \P
%atha vā nāsāgrād ārabhya ṣaḍaṃgulaṃ pramāṇaṃ ?bi?ṣi?īnāvarṇaṃ .. .. .. ..??.  lakṣyaṃ kartavyam/  \B
%\om \L
%atha vā nāsāgrād ābhya   ṣadaṃgulapramāṇaṃ    pavanatatvaṃ dhūmrākāraṃ        lakṣaṃ karttavyaṃ/  \N1
%atha vā nāsāgrād ābhya   ṣadaṃgulapramāṇaṃ    pavanatatvaṃ dhūmrākāraṃ        lakṣaṃ karttavyaṃ// \D
%atha vā nāsāgrārabhya    ṣadaṃgulapramāṇaṃ    pavanatatvaṃ dhūmrākāraṃ        lakṣaṇaṃ karttavyaṃ// \N2
%atha vā nāsāgrād ārabhya dvadaśaṃgulapramāṇaṃ pavanatatvaṃ dhūmrākāraṃ        lakṣyaṃ karttavyaṃ \U1
%atha vā nāsāgrād ārabhya ṣaḍaṃgulapramāṇaṃ    pavanatatvaṃ dhūmrākāraṃ        lakṣaṃ karttavyaṃ// \U2
%------------------------------
%Or, a six finger wide distance from the tip of the nose, the wind-element whose appearance is greyish shall be made the object of fixation. 
%------------------------------
\note[type=source, labelb=152, lem={ṣadaṃgulapramāṇaṃ}]{Ysv (PT): līlayā bhāvayel līnaṃ jyotiḥpūrṇaṃ mahāparam | atha vā tatra deveśi dhūmrākāraṃ ṣaḍaṅgulam |}
\app{\lem[wit={ceteri}]{atha vā}
      \rdg[wit={L}]{\om}}
    \app{\lem[wit={ceteri}]{nāsāgrād\skp{-}ārabhya}
      \rdg[wit={N1,D}]{nāsāgrād ābhya}
      \rdg[wit={N2}]{nāsāgrārabhya}
      \rdg[wit={L}]{\om}}
    \app{\lem[wit={ceteri}]{ṣaḍaṃgulapramāṇaṃ}
      \rdg[wit={B}]{ṣaḍaṃgulaṃ pramāṇaṃ}
      \rdg[wit={U2}]{dvadaśaṃgulapramāṇaṃ}
      \rdg[wit={L}]{\om}}
     \app{\lem[wit={E}]{pavanatattvaṃ}
       \rdg[wit={ceteri}]{°tatvaṃ}
       \rdg[wit={L}]{\om}
       \rdg[wit={B}]{\illeg}}
     \app{\lem[wit={ceteri}]{dhūmrākāraṃ}
       \rdg[wit={B}]{\illeg}
       \rdg[wit={L}]{\om}}
     \app{\lem[wit={ceteri}]{lakṣyaṃ}
       \rdg[wit={N1,D,U2}]{lakṣaṃ}
       \rdg[wit={N2}]{lakṣaṇaṃ}}
     \app{\lem[wit={ceteri}]{karttavyaṃ}
       \rdg[wit={L}]{\om}}/
%------------------------------
%\om \E
%\om \P
%\om \B
%\om \L
%atha vā nāsāgrād ārabhyā  ṣaḍaṃgulapramāṇām atiraktaṃ        tejo lakṣaṃ karttavyaṃ \N1
%atha vā nāsāgrād ārabhya  ṣaḍaṃgulapramāṇām atirattaṃ        tejo lakṣaṃ karttavyaṃ// \D
%atha vā nāsāgrād ārabhyaṃ ṣṭāṃgulapramāṇam atirakṭaṃ         tejo lakṣaṇaṃ kartavyaṃ// \N2
%atha    nāsāgrād ārabhyāṣṭaṃgulapramāṇam     itiriktaṃ       tejo lakṣyaṃ karttavyaṃ/ \U1
%atha vā nāsāgrād ārabhyaṃ        ṣṭagulapramāṇaṃ matiraktaṃ  teja lakṣyaṃ karttavyaṃ// \U2 
%------------------------------
% Or, an eight finger wide distance from the tip of the nose, the very red fire[-element] shall be made the object of fixation. 
%------------------------------
\note[type=testium, labelb=153, lem={ārabhyāṣṭaṃgula°}]{Ysv (PT): atha vāṣṭāṅgulaṃ raktaṃ nāsikopari lakṣayet |}
atha
\app{\lem[wit={ceteri}]{vā}
  \rdg[wit={U1}]{\om}}
nāsāgrā\skp{d-ā}\app{\lem[wit={U1}, alt={ārabhyāṣṭāṃgula°}]{\skm{d-ā}rabhyāṣṭaṃgulapramāṇa\skp{m-a}}
  \rdg[wit={N1}]{ārabhyā ṣaḍaṃgulapramāṇām}
  \rdg[wit={D}]{ārabhya ṣaḍaṃgulapramāṇām}
  \rdg[wit={N2}]{ārabhyaṃ ṣṭāṃgulapramāṇam}
  \rdg[wit={U2}]{ārabhyaṃ ṣṭagulapramāṇaṃ}
  \rdg[wit={ceteri}]{\om}
}\app{\lem[wit={N1,N2}, alt={atiraktaṃ}]{\skm{m-a}tiraktaṃ}
  \rdg[wit={D}]{atirattaṃ}
  \rdg[wit={U1}]{itiriktaṃ}
  \rdg[wit={U2}]{matiraktaṃ}
  \rdg[wit={ceteri}]{\om}}
\app{\lem[wit={ceteri}]{tejo}
  \rdg[wit={U2}]{teja°}
  \rdg[wit={ceteri}]{\om}}
\app{\lem[wit={U1,U2}]{lakṣyaṃ}
  \rdg[wit={N1,N2}]{lakṣaṃ}
  \rdg[wit={N2}]{lakṣaṇaṃ}
  \rdg[wit={ceteri}]{\om}}
karttavyaṃ/
%------------------------------
%\om \E
%\om \P
%\om \B
%\om \L
%atha vā nāsāgrād ārabhya daśāṃgulapramāṇaṃ śuklaṃ caṃcalam    udakaṃ lakṣya   karttavyaṃ/  \N1
%atha vā nāsāgrād ārabhya daśāṃgulapramāṇaṃ śuklaṃ caṃcalam    udakaṃ lakṣya   karttavyaṃ// \D
%atha vā nāsāgrād ārabhya daśāṃgulapramāṇaṃ śuklaṃ caṃdrākāram udakaṃ lakṣyaṃ  kartavyaṃ    \U1
%atha vā nāsāgrād ārabhya daśāṃgulapramāṇaṃ śuklaṃ caṃcalam    udakaṃ lakṣaṇaṃ kartavyaṃ//  \N2 [S.7 Verso, Zeile 1]
%atha vā nāsāgrād ārabhya daśāṃgulapramāṇaṃ śuklaṃ caṃcalam    udakaṃ lakṣaṃ   kartavyaṃ//  \U2
%------------------------------
%Or, a ten finger wide distance from the tip of the nose, the white water[-element] being fickle shall be made the object of fixation. 
%------------------------------
\note[type=philcomm, labelb=154, lem={daśāṃgulapramāṇaṃ}]{The instruction of a ten finger wide distance is absent in the surviving testimonia of the Ysv.}
\app{\lem[wit={ceteri}]{atha vā}
  \rdg[wit={E,P,B,L}]{\om}}
\app{\lem[wit={ceteri}, alt={nāsāgrād ārabhya}]{nāsāgrād-ārabhya}
  \rdg[wit={E,P,B,L}]{\om}}
\app{\lem[wit={ceteri}]{daśāṃgulapramāṇaṃ}
  \rdg[wit={E,P,B,L}]{\om}}
\app{\lem[wit={ceteri}]{śuklaṃ}
  \rdg[wit={E,P,B,L}]{\om}}
\app{\lem[wit={ceteri}]{caṃcalam}
  \rdg[wit={U1}]{caṃdrākāram}
  \rdg[wit={E,P,B,L}]{\om}}
\app{\lem[wit={ceteri}]{udakaṃ}
  \rdg[wit={E,P,B,L}]{\om}}
\app{\lem[wit={U1}]{lakṣyaṃ}
  \rdg[wit={N1,D}]{lakṣya}
  \rdg[wit={N2}]{lakṣaṇaṃ}
  \rdg[wit={U2}]{lakṣaṃ}
  \rdg[wit={ceteri}]{\om}}
\app{\lem[wit={ceteri}]{kartavyaṃ}
  \rdg[wit={ceteri}]{\om}}
%------------------------------
%atha vā nāsāgrād ārabhya tattvaṃ dvādaśāṃgulapramāṇaṃ   pītavarṇaṃ  pṛthvītattvaṃ lakṣyaṃ  karttavyam/ \E
%atha vā nāsāgrād ārabhya         dvādaśāṃgulapramāṇaṃ   pītavarṇaṃ  pṛthvītatvaṃ  lakṣyaṃ  karttavyaṃ \P
%atha vā nāsāgrād ārabhya         dvadaśāṃgulapramāṇaṃ   pītavarṇaṃ  pṛthvītatvaṃ  lakṣaṃ   kartavyaṃ// \B
%atha vā nāsāgrād ārabhya         dvādaśāṃgulapramāṇaṃ   pītavarṇaṃ  pṛthvītatvaṃ  lakṣaṃ   kartavyaṃ/  \L
%atha vā nāsāgrād ārabhya         dvadaśāṃgulapramāṇaṃ   pītavarṇṇaṃ prthvītatvaṃ  lakṣaṃ   karttavyaṃ/   \N1
%atha vā nāsāgrād ārabhya         dvadaśāṃgulapramāṇaṃ   pītavarṇṇaṃ prthvītatvaṃ  lakṣaṃ   karttavyaṃ/   \D
%atha vā nāsāgrād ārabhya         dvadaśāṃgulapramāṇaṃ   pītavarṇaṃ  prthvītatvaṃ  lakṣaṇaṃ karttavyaṃ//  \N2
%atha vā nāsāgrād ārabhya         dvādaśā aṃgulapramāṇaṃ pītavarṇaṃ  prthvītatvaṃ  lakṣyaṃ  karttavyaṃ   \U1
%atha vā nāsāgrād ārabhya         dvādaśāṃgulapramāṇaṃ   pītavarṇaṃ  pṛthvītatvaṃ  lakṣaṃ   karttavyaṃ//    \U2
%------------------------------
%Or, a twelve finger wide distance from the tip of the nose, the yellow-colored earth-element shall be made the object of fixation.  
%------------------------------
\note[type=source, labelb=155, lem={dvādaśāṃgulapramāṇaṃ}]{Ysv (PT): dvādaśāṅgulamānaṃ vā pṛthvītattvan tu pītabham | lakṣayed atha vā tatra koṭisūryasamaprabham | tejaḥ puñjaṃ mahākāśaṃ tattad dhyānāc chivo bhavet | ākāśamadhye ākāśoparito dṛṣṭis usthiram | kṛtvā dhyānād vinā sūryaṃ caṇḍasūryan tu paśyati | atha vā lakṣam etat tu karttur vahiḥ śivopari |}
atha vā nāsāgrād-ārabhya
\app{\lem[wit={ceteri}]{dvādaśāṃgulapramāṇaṃ}
  \rdg[wit={E}]{tattvaṃ dvādaśāṃgulapramāṇaṃ}
  \rdg[wit={U1}]{dvādaśā aṃgulapramāṇaṃ}}
pītavarṇaṃ pṛthvītattvaṃ
\app{\lem[wit={E,P,U1}]{lakṣyaṃ}
  \rdg[wit={N2}]{lakṣaṇaṃ}
  \rdg[wit={ceteri}]{lakṣaṃ}}
kartavyaṃ/
%------------------------------
%atha vā nāsāgrād ārabhya koṭisūryasamaprabhaṃ tejaḥ/ pūrṇam ākāśatattvaṃ lakṣyaṃ karttavyam/                 \E
%atha vā nāsāgrād ārabhya koṭisūryasamaprabhaṃ tejaḥpūrṇam   ākāśatatvaṃ lakṣyaṃ karttavyaṃ                      \P
%atha vā nāsāgrād ārabhya koṭisūryasamaprabhaṃ tejaḥ/ pūrṇam ākāśatatvaṃ lakṣaṃ kartavyaṃ//                   \B %%%%DSCN7161.JPG letzte 3 Zeilen!
%atha vā nāsāgrād ārabhya koṭisūryasamaprabhāṃ tejaḥpūrṇam   ākāśatatvaṃ lakṣaṃ karttavyaṃ//                   \L
%atha vā nāsāgrād ārabhya koṭisūryasamaprabhaṃ tejaḥpūrṇaṃ   ākāśatatvaṃ lakṣyaṃ karttavyaṃ/                   \N1
%atha vā nāsāgrād ārabhya koṭisūryasamaprabhaṃ tejaḥpūrṇaṃ   ākāśatatvaṃ lakṣyaṃ karttavyaṃ//                   \D
%atha vā nāsāgrād ārabhya koṭisūryasamaprabhaṃ tejaḥpūrṇa    ākāśatatvaṃ lakṣaṇaṃ karttavyaṃ//                    \N2
%atha vā nāsāgrād ārabhya koṭisūryasamaprabhaṃ tejaḥpūrṇaṃ   ākāśatatvaṃ lakṣyaṃ karttavyaṃ                     \U1
%atha vā nāsāgrād ārabhya koṭisūryasamaprabhaṃ tejaḥpūrṇaṃ   ākāśatatvaṃ lakṣaṃ karttavyaṃ//                    \U2
%------------------------------
%Or beginning at the tip of the nose\footnote{Given the clear instructions of the respective distance of the exercise in the previous sentences, it is surprising that this instruction is lacking the mention of the distance.} the space-element full of fire shining like ten million suns shall be made the object of fixation.  
%------------------------------
atha vā nāsāgrād ārabhya koṭisūrya\app{\lem[wit={ceteri}]{samaprabhaṃ}
  \rdg[wit={L}]{°prabhāṃ}}
\app{\lem[wit={ceteri}, alt={tejaḥpūrṇam}]{tejaḥpūrṇa\skp{m-ā}}
  \rdg[wit={E,B}]{tejaḥ | pūrṇaṃ}
  \rdg[wit={N1,D,U1,U2}]{pūrṇaṃ}
  \rdg[wit={N2}]{pūrṇa}
}\skm{m-ā}kāśatattvaṃ
\app{\lem[wit={E,P,N1,D,U1}]{lakṣyaṃ}
  \rdg[wit={B,L,U2}]{lakṣaṃ}
  \rdg[wit={N2}]{lakṣaṇaṃ}}
karttavyaṃ/
%------------------------------
%ākāśamadhye  ākāśopari    dṛṣṭiṃ kṛtvā             dhyānakāraṇāt// sūryaṃ vinā sūryasambandhinī  sahasrakiraṇapaṅktīḥ   paśyati/ \E
%             ākāśopari    dṛṣtiṃ kṛtvā             dhyānakaraṇāt   sūryaṃ vinā sūryasaṃbaṃdhīnīṃ sahasrakiraṇāvalīṃ     pati   \P
%             ākāśopari    dṛṣti  kṛtvā ākāśamadhye dhyānakaraṇāt// sūryaṃ vinā sūryasaṃbaṃdhīnī  sahasrakiraṇāvali      paśyatī// \B
%                                       ākāśamadhye dhyānakaraṇāt// sūryaṃ vinā sūryasaṃbaṃdhinī  sahasrakiraṇāvali      paśyati// \L
%ākāśamadhye  ākāśoparī vā dṛṣṭiṃ kṛtvā             dhyānakaraṇāt   sūryaṃ vinā sūryasaṃbaṃdhinī  sahasrāṇy api kīraṇāṇi paśyatī/     \N1
%ākāśamadhye  ākāśopari vā dṛṣṭiṃ kṛtvā             dhyānakaraṇāt   sūryaṃ vinā sūryasaṃbaṃdhinī  sahasrāṇapi   kīraṇāṇi paśyatī//     \D
%ākāśamadhye  ākāśopari vā dṛṣṭiṃ kṛtvā             dhyānakaraṇāt   sūrya  vinā sūryasaṃbaṃdhinī  sahasrāṇapi   kiraṇāṇi paśyate//     \N2
%ākāśamadhye  ākāśopari vā dṛṣṭiṃ kṛtvā             dhyānakaraṇāt   sūryaṃ vinā sūryasaṃbaṃdhinī  sahasrāṇy api kiraṇāṇi paśyaṃti     \U1
%ākāśamadhye  ākāśopari vā dṛṣṭiṃ kṛtvā             dhyānakaraṇāt// sūrya  vinā sūryasaṃbaṃdhinī  sahasrakiraṇāvaliṃ     paśyati//     \U2
%------------------------------ 
%After having fixed the gaze on the space[-element?] or above the space[-element?], due to the execution of meditation he sees the sun without the group of thousand rays related to the sun. 
%------------------------------
\app{\lem[wit={ceteri}]{ākāśamadhye}
  \rdg[wit={P,B,L}]{\om}}
\app{\lem[wit={ceteri}]{ākāśopari}
  \rdg[wit={N1}]{ākāśoparī}}
\app{\lem[wit={ceteri}]{vā}
  \rdg[wit={E,P,B,L}]{\om}}
\app{\lem[wit={ceteri}]{dṛṣṭiṃ}
  \rdg[wit={B}]{dṛṣṭi}
  \rdg[wit={L}]{\om}}
\app{\lem[wit={ceteri}]{kṛtvā}
  \rdg[wit={B}]{kṛtvā ākāśamadhye}
  \rdg[wit={L}]{ākāśamadhye}}
dhyānakāraṇāt
\app{\lem[wit={ceteri}]{sūryaṃ}
  \rdg[wit={N2, U2}]{sūrya}}
vinā
\app{\lem[wit={P}]{sūryasaṃbaṃdhīnīṃ}
  \rdg[wit={ceteri}]{sūryasaṃbaṃdhinī}}
\app{\lem[wit={P}]{sahasrakiraṇāvalīṃ}
  \rdg[wit={U2}]{sahasrakiraṇāvaliṃ}
  \rdg[wit={B,L}]{sahasrakiraṇāvali}
  \rdg[wit={E}]{sahasrakiraṇapaṅktīḥ}
  \rdg[wit={N1,U1}]{sahasrāṇy api kīraṇāṇi}
  \rdg[wit={D,N2}]{sahasrāṇapi kiraṇāṇi}}
\app{\lem[wit={E,L,U2}]{paśyati}
  \rdg[wit={B,N1,D}]{paśyatī}
  \rdg[wit={N2}]{paśyate}
  \rdg[wit={P}]{pati}
  \rdg[wit={U1}]{paśyaṃti}}/
%-----------------------------
%atha vā śivopari vṛddhaṃ  saptadaśāṃgulapramāṇaṃ  tejaḥpuṃjalakṣyaṃ     karttavyam/ \E
%\om \P
%atha vā śiroparir urdhvaṃ saptadaśāṃgulapramāṇaṃ  tejaḥpūṃjaṃ lakṣaṇaṃ  kartavyaṃ/ \B
%atha vā śiropari ūrdhva---saptadaśāṃgulapramāṇaṃ  tejaḥpūṃjaṃ lakṣaṃ    kartavyaṃ/ \L
%atha kā śiropari ūrddhvaṃ saptadaśāṃgulapramāṇaṃ  tejā  puṃjalakṣaṃ      karttavyaṃ/ \N1
%atha vā śiropari ūrddhvaṃ saptadaśāṃgulapramāṇaṃ  tejā  puṃjalakṣyaṃ     karttavyaṃ// \D
%atha vā śiropari ūrddhvaṃ saptadaśāṃgulaṃ parāṇaṃ tejaḥpuṃjalakṣaṇaṃ    kartavyaṃ// \N2
%atha vā śiropari ūrddhaṃ  saptadaśāṃgulapramāṇaṃ  tejaḥpuṃjakaṃ lakṣyaṃ kartavyaṃ \U1 %%%278.jpg
%atha vā śiropari ūrddhaṃ  saptadaśāṃgulapramāṇa---tejaḥpuṃjaṃ lakṣyaṃ   karttavyaṃ// \U2
%-----------------------------
%Or the mass of light situated seventeen fingers wide distance above the head shall be made the fixation object. 
%-----------------------------
\note[type=source, labelb=156, lem={saptadaśāṃgulapramāṇaṃ}]{Ysv (PT): ūrddhvaṃ saptadaśāṅgulyaṃ pramāṇaṃ tejasā prabham |}
\app{\lem[wit={ceteri}]{atha vā}
  \rdg[wit={N1}]{atha kā}
  \rdg[wit={P}]{\om}}
\app{\lem[type=emendation, resp=egoscr, alt={śiropary}]{śiropar\skp{y-ū}}
  \rdg[wit={ceteri}]{\korr śiropari}
  \rdg[wit={E}]{śivopari}
  \rdg[wit={B}]{śiroparir}
  \rdg[wit={P}]{\om}
}\app{\lem[wit={ceteri}, alt={ūrddhvaṃ}]{\skm{y-ū}rddhvaṃ}
  \rdg[wit={L}]{ūrdhva°}
  \rdg[wit={B}]{urdhvam}
  \rdg[wit={U1,U2}]{ūrddhaṃ}
  \rdg[wit={E}]{vṛddhaṃ}
  \rdg[wit={P}]{\om}}
\app{\lem[wit={ceteri}]{saptadaśāṃgulapramāṇaṃ}
  \rdg[wit={N2}]{saptadaśāṃgulaṃ parāṇaṃ}
  \rdg[wit={U2}]{saptadaśāṃgulapramāṇa°}
  \rdg[wit={P}]{\om}}
\app{\lem[wit={E}]{tejaḥpuṃjalakṣyaṃ}
  \rdg[wit={P}]{tejaḥpūṃjaṃ lakṣaṇaṃ}
  \rdg[wit={L}]{tejaḥpūṃjaṃ lakṣaṃ}
  \rdg[wit={N1}]{tejā puṃjalakṣaṃ}
  \rdg[wit={D}]{tejā puṃjalakṣyaṃ}
  \rdg[wit={N2}]{tejaḥpuṃjalakṣaṇaṃ}
  \rdg[wit={U1}]{tejaḥpuṃjakaṃ lakṣyaṃ}
  \rdg[wit={U2}]{tejaḥpuṃjaṃ lakṣyaṃ}}
karttavyaṃ/
\end{prose}
\end{ekdosis}
\ekdpb*{}
%%%%%%%%%%%%%%%%%%%%%%%%%%%%%%%%%%%%%%%%%%
%%%%%%%%%%%%%%%%%%%%%%%%%%%%%%%%%%%%%%%%%%
%%%%%%%%PAGEBREAK%%%%%%%PAGEBREAK%%%%%%%%%
%%%%%%%%%%%%%%%%%%%%%%%%%%%%%%%%%%%%%%%%%%
%%%%%%%%%%%%%%%%PAGEBREAK%%%%%%%%%%%%%%%%%
%%%%%%%%%%%%%%%%%%%%%%%%%%%%%%%%%%%%%%%%%%
%%%%%%%%PAGEBREAK%%%%%%%PAGEBREAK%%%%%%%%%
%%%%%%%%%%%%%%%%%%%%%%%%%%%%%%%%%%%%%%%%%%
%%%%%%%%%%%%%%%%%%%%%%%%%%%%%%%%%%%%%%%%%%
%%%%%%%%%%%%%%%%%%%%%%%%%%%%%%%%%%%%%%%%%%
%%%%%%%%%%%%%%%%%%%%%%%%%%%%%%%%%%%%%%%%%%
%%%%%%%%PAGEBREAK%%%%%%%PAGEBREAK%%%%%%%%%
%%%%%%%%%%%%%%%%%%%%%%%%%%%%%%%%%%%%%%%%%%
%%%%%%%%%%%%%%%%PAGEBREAK%%%%%%%%%%%%%%%%%
%%%%%%%%%%%%%%%%%%%%%%%%%%%%%%%%%%%%%%%%%%
%%%%%%%%PAGEBREAK%%%%%%%PAGEBREAK%%%%%%%%%
%%%%%%%%%%%%%%%%%%%%%%%%%%%%%%%%%%%%%%%%%%
%%%%%%%%%%%%%%%%%%%%%%%%%%%%%%%%%%%%%%%%%%
%%%%%%%%%%%%%%%%%%%%%%%%%%%%%%%%%%%%%%%%%%
%%%%%%%%%%%%%%%%%%%%%%%%%%%%%%%%%%%%%%%%%%
%%%%%%%%PAGEBREAK%%%%%%%PAGEBREAK%%%%%%%%%
%%%%%%%%%%%%%%%%%%%%%%%%%%%%%%%%%%%%%%%%%%
%%%%%%%%%%%%%%%%PAGEBREAK%%%%%%%%%%%%%%%%%
%%%%%%%%%%%%%%%%%%%%%%%%%%%%%%%%%%%%%%%%%%
%%%%%%%%PAGEBREAK%%%%%%%PAGEBREAK%%%%%%%%%
%%%%%%%%%%%%%%%%%%%%%%%%%%%%%%%%%%%%%%%%%%
%%%%%%%%%%%%%%%%%%%%%%%%%%%%%%%%%%%%%%%%%%
\begin{ekdosis}
  \begin{prose}
%-----------------------------
%atha vā dṛṣṭer agre tatparaṃ svarṇākāraṃ  pṛthvītattvaṃ  lakṣyaṃ kartavyam/ \E
%atha vā dṛṣṭer agne taptasvarṇavarṇakāraṃ pṛthvītatvaṃ   lakṣyaṃ \P
%atha vā dṛṣṭer agne taptasuvarṇavarṇa-----pṛthivītatvaṃ  lakṣaṃ kartavyaṃ/ \B
%atha vā dṛṣṭer agne taptasuvarṇavarṇa-----pṛthītatvaṃ    lakṣaṃ kartavyaṃ/ \L
%atha vā dṛṣṭer ag..?taptavarṇākāraṃ       pṛthvītatvaṃ   lakṣaṃ karttavyaṃ/ \N1
%atha vā dṛṣṭer agre taptavarṇākāraṃ       pṛthvītatvaṃ   lakṣaṃ karttavyaṃ// \D %%%p.10 beginning
%atha vā dṛṣṭer agre taptavarṇākāraṃ       pṛthvītatvaṃ   lakṣaṇaṃ karttavyaṃ/ \N2
%atha vā dṛṣṭer agre taptavarṇākāraṃ       pṛthvītatvaṃ   lakṣyaṃ karttavyaṃ \U1
%atha vā dṛṣṭer agre taptasvarṇavarṇākāraṃ pṛthvīṃ tatvaṃ lakṣaṃ karttavyaṃ// \U2
%-----------------------------
%Or at the uppermost part of the focal point the earth-element appearing in the color of molten gold shall be made the fixation object.  
%-----------------------------
\noindent
\note[type=source, labelb=157, lem={pṛthvītattvaṃ}]{Ysv (PT): ūrddhvaṃ saptadaśāṅgulyaṃ pramāṇaṃ tejasā prabham | athavā pṛthivītattvaṃ taptakāñcanasannibham | dṛṣṭiragre tu karttavyaṃ lakṣam etad yat ātmanām | uktānāṃ yasya kasyaiva ekaśaḥ karaṇaṃ priye | balīpalitahīnaḥ syādauṣadhena vinā tathā |}
atha vā dṛṣṭe\skp{r-a}\app{\lem[wit={ceteri}, alt={agre}]{\skm{r-a}gre}
  \rdg[wit={P,B,L}]{agne}}
\app{\lem[wit={U2}]{taptasvarṇavarṇākāraṃ}
  \rdg[wit={P}]{taptasvarṇavarṇakāraṃ}
  \rdg[wit={E}]{tatparaṃ svarṇākāraṃ}
  \rdg[wit={B,L}]{taptasuvarṇavarṇa}
  \rdg[wit={N1,N2,D,U1}]{taptavarṇākāraṃ}}
\app{\lem[wit={E}]{pṛthvītattvaṃ}
  \rdg[wit={P}]{pṛthvītatvaṃ}
  \rdg[wit={B}]{pṛthivītatvaṃ}
  \rdg[wit={L}]{pṛthītatvaṃ}
  \rdg[wit={N1,N2,D,N2}]{pṛthvītatvaṃ}
  \rdg[wit={N2}]{pṛthvīṃ tatvaṃ}}
\app{\lem[wit={E,P,U1}]{lakṣyaṃ}
  \rdg[wit={B,L,N1,D,U2}]{lakṣaṃ}
  \rdg[wit={N2}]{lakṣaṇaṃ}}
\app{\lem[wit={ceteri}]{karttavyaṃ}
  \rdg[wit={P}]{\om}}/
%-----------------------------
%uktānāṃ lakṣyāṇāṃ  madhye yasya kasyāpy ekasya lakṣyakaraṇāt     valitapalitā dūre bhavanti/ \E
%uktānāṃ lakṣaṇānāṃ madhye yasya kasyāpy ekasya lakṣyakaraṇāt     valitapalitādidūre bhavati \P
%uktānāṃ lakṣaṇaṃ   madhye yasya kasyāpi kasya  lakṣakaraṇāt//    valitaṃ palitādi dūre bhavatī/ \B
%uktānāṃ lakṣaṇaṃ   madhye yasya kasyāpi kasya  lakṣakaraṇāt//    valitaṃ palitādi dūre bhavati// \L
%uktānāṃ lakṣyaṇāṃ  madhye yasya kasyāpy ekasya lakṣasya karaṇāt  valitapalitādidūre bhavati \N1
%uktānāṃ lakṣyaṇaṃ  madhye yasya kasyāp--ekasya lakṣasya karaṇāt  valitapalitādidūre bhavati// \D
%uktānāṃ lakṣāṇā----madhye yasya lasyāpy elasya lakṣaṇasya karaṇātvalitapalitādidūre bhavati/ \N2
%uktānāṃ lakṣyaṇāṃ  madhye yasya kasyāpi kasya  lakṣyasya karaṇā  valitapalitādidūre bhavati \U1
%uktānāṃ lakṣāṃ     madhye yasya kasyāpy ekasya lakṣyakaraṇāt     valitapalitādidūre bhavaṃti// \U2
%-----------------------------
%From the execution of [the yoga of] fixation onto the middle of anyone of the discussed fixation objects wrinkles and grey hair etc. are removed. 
%-----------------------------
uktānāṃ 
\app{\lem[wit={E}]{lakṣyāṇāṃ}
  \rdg[wit={U1,N1}]{lakṣyaṇāṃ}
  \rdg[wit={D}]{lakṣyaṇaṃ}
  \rdg[wit={P}]{lakṣaṇānāṃ}
  \rdg[wit={B,L}]{lakṣaṇaṃ}
  \rdg[wit={N2}]{lakṣāṇā°}
  \rdg[wit={U2}]{lakṣāṃ}}
madhye yasya
\app{\lem[wit={ceteri},alt={kasyāpy}]{kasyā\skp{py-e}}
  \rdg[wit={B,L,U1}]{kasyāpi}
  \rdg[wit={D}]{kasyāp°}
  \rdg[wit={N2}]{lasyāpy}
}\app{\lem[wit={ceteri}, alt={ekasya}]{\skm{py-e}kasya}
  \rdg[wit={B,L,U1}]{kasya}
  \rdg[wit={N2}]{elasya}}
\app{\lem[wit={ceteri},alt={lakṣya°}]{lakṣya}
  \rdg[wit={B,L}]{lakṣa°}
  \rdg[wit={N1,D}]{lakṣasya}
  \rdg[wit={N2}]{lakṣaṇasya}
  \rdg[wit={U1}]{lakṣyasya}
}\app{\lem[wit={ceteri}, alt={°karaṇāt}]{karaṇāt}
  \rdg[wit={U1}]{karaṇā}}
\app{\lem[wit={E}]{valitapalitā dūre}
  \rdg[wit={B,L}]{valitaṃ palitādi dūre}
  \rdg[wit={ceteri}]{valitapalitādidūre}}
\app{\lem[wit={E,U2}]{bhavanti}
  \rdg[wit={B}]{bhavatī}
  \rdg[wit={ceteri}]{bhavati}}/
%-----------------------------
%aṃgarogāḥ  vinauṣadhaṃ dūrī bhavanti/  samagrāḥ śatravaḥ  svapne pya mitran   nāyāṃti/     \E
%aṃgirogā   vinauṣadhaṃ dūre bhavati    samagrāḥ śatravaḥ  svapne pya mitratām ayāṃti  \P  %%%7646.jpg Z.1 
%aṃgirogādi vinauṣadhaṃ dūro bhavatī    samagrāḥ śatrave   svapne pya mitratām ayāṃti//   \B
%aṃgirogādi vinauṣadhaṃ dūro bhavati    samagrāḥ śatravo   svapne pya mitratām ayāṃtī     \L
%aṃgarogā   vinauṣadhaṃ dūre bhavaṃti/  samagrāḥ śatravaḥ  svapin eva mityaṃ   nāyāti/     \N1
%aṃgarogā   vinauṣadhaṃ dūre bhavaṃti// samagrāḥ śatravaḥ  svacan eva mityaṃ   nāyāti//     \D
%aṃgarogā   vinauṣadhaṃ dūre bhavati//  samagrā  śatravaḥ  svapin evan nityaṃ  nāyāti//    \N2
%aṃgarogā   vinauṣadhaṃ dūre bhavati    samagrāḥ śatravaḥ  svapin eva mitevaṃ  naiyati    sahasravarṣaparyaṃtam āyuṣyaṃ varddhate \U1
%aṃgarogā   vinauṣadhaṃ dūre bhavaṃti// samagra  śatravaḥ  svapne pi mitratām  āyāṃti//  sahasravarṣam āyur varddhate// \U2 ā-yānti= von ā-√yā=in einen Zustand ~, in eine Lage ~, in ein Verhältniss kommen, ~ gerathen; theilhaftig werden, erlangen; mit Acc. 
%-----------------------------
%Diseases of the limbs are removed without medical herbs. All enemies become friends while sleeping. The lifespan increases up to 1000 years. 
%-----------------------------
\note[type=source, labelb=158, lem={aṅgarogā}]{Ysv (PT): sarvarogāṇi naśyanti mitravac ca vaśī ripuḥ | jīved varṣasahasran tu sarvalokeṣu pūjitaḥ | jihvāgre prabhaved vidyā vinā śāstrāvalokanāt |}
\app{\lem[wit={ceteri}]{aṅgarogā}
  \rdg[wit={E}]{aṃgarogāḥ}
  \rdg[wit={B,L}]{aṃgirogādi}}
vinauṣadhaṃ
\app{\lem[wit={ceteri}]{dūre}
  \rdg[wit={E}]{dūrī}
  \rdg[wit={B,L}]{dūro}}
\app{\lem[wit={E,N1,D,U2}]{bhavanti}
  \rdg[wit={P,L,N2,U1}]{bhavati}
  \rdg[wit={B}]{bhavatī}}/
\app{\lem[wit={ceteri}]{samagrāḥ}
  \rdg[wit={N2}]{samagrā}
  \rdg[wit={U2}]{samagra°}}
\app{\lem[wit={ceteri}]{svapne}
  \rdg[wit={N1,N2,U1}]{svapin}
  \rdg[wit={D}]{svacan}}
\app{\lem[wit={U2}]{'pi}
  \rdg[wit={E,P,B,L}]{pya}
  \rdg[wit={N1,D,U1}]{eva}
  \rdg[wit={N2}]{evan}}
\app{\lem[wit={P,B,L,U2}]{mitratām}
  \rdg[wit={E}]{mitran}
  \rdg[wit={N1,D}]{mityaṃ}
  \rdg[wit={N2}]{nityaṃ}
  \rdg[wit={U1}]{mitevaṃ}}
\app{\lem[wit={P,B}]{ayāṃti}
  \rdg[wit={L}]{ayāṃtī}
  \rdg[wit={N2}]{āyāṃti}
  \rdg[wit={E}]{nāyāṃti}
  \rdg[wit={N1,N2,D}]{nāyāti}
  \rdg[wit={U1}]{naiyati}}/
%-----------------------------STEMMAPOINT!!!!------------------------------------
% sahasravarṣam āyur bhavati/ \E
% sahasravarṣam āyur varddhate \P  %%%7646.jpg Z.1 
% sahasravarṣam āyur vardhate// \B
% sahasravarṣam āyur vardhate// \L
% sahasravarṣaparyaṃtam āyuṣaṃ varddhate/ \N1
% sahasravarṣaparyaṃtam āyuṣaṃ varddhate// \D
% sahasravarṣaparyaṃtam āyuṣaṃ vardhate// \N2
% sahasravarṣaparyaṃtam āyuṣyaṃ varddhate \U1
% sahasravarṣam āyur varddhate// \U2 ā-yānti= von ā-√yā=in einen Zustand ~, in eine Lage ~, in ein Verhältniss kommen, ~ gerathen; theilhaftig werden, erlangen; mit Acc. 
%-----------------------------
% The lifespan increases up to 1000 years. 
%-----------------------------
\app{\lem[wit={N1,N2,D,U1}]{sahasravarṣaparyaṃtam}
  \rdg[wit={E,P,B,L,U2}]{sahasravarṣam}}
\app{\lem[wit={N1,N2,D}]{āyuṣaṃ}
  \rdg[wit={U1}]{āyuṣyaṃ}
  \rdg[wit={E,P,B,L,U2}]{āyur}}
vardhate/
%-----------------------------
%apaṭhitaṃ śāstraṃ jihvāgreṇoccarati/  etādṛśaṃ phalaṃ bahutaraṃ bhavati// \E
%apaṭhitaṃ śāstraṃ jihvāgreṇoccarati   etādṛśaṃ mitratāmāyāṃti sahasravarṣamāyur varddhate apaṭhitaṃ śāstraṃ jihvāgreṇoccarati etādṛśaṃ phalaṃ bahutaraṃ bhavati \P
%apaṭhitaṃ śāstraṃ jihvāgreṇoccaratī/  etādṛśaṃ phalaṃ bahutaraṃ bhavatī// \B
%apaṭhitaṃ śāstraṃ jihvāgreṇoccarati   etādṛśaṃ phalaṃ bahutaraṃ bhavaṃtī// \L
%apaṭhitaṃ śāstraṃ jihvāgreṇoccarate// etādṛśaṃ bahutaraṃ phalaṃ bhavati// \N1
%apaṭhitaṃ śāstraṃ jihvāgreṇoccarate// etādṛśaṃ bahutaraṃ phalaṃ bhavati// \D
%apathitaṃ śāstraṃ jihvāgreṇoccarate// etādṛśaṃ bahutaraṃ phalaṃ bhavati// \N2
%apathitaṃ śāstraṃ jihvāgreṇoccarate   etādṛśyaṃ bahutaraṃ phalaṃ bhavati \U1
%apathitaṃ śāstraṃ jihvāgreṇoccarati// etādṛśaṃ phalaṃ bahutaraṃ phalaṃ bhavati// \U2
%-----------------------------
%Non-recitable? teachings are enunciated by the tip of the tongue [of the practitioner?]. An abundance of such results arise.  
%-----------------------------
\app{\lem[wit={ceteri}]{apaṭhitaṃ}
  \rdg[wit={N2,U1,U2}]{apathitaṃ}}
śāstraṃ
jihvāgreṇocca\app{\lem[wit={N1,N2,D,U1}, alt={°rate}]{rate}
  \rdg[wit={E,B,L,U2}]{°rati}
  \rdg[wit={B}]{°ratī}}/
\app{\lem[wit={ceteri}]{etādṛśaṃ}
  \rdg[wit={U1}]{etādṛśyaṃ}}
\app{\lem[wit={N1,N2,D,U1}]{bahutaraṃ phalaṃ}
  \rdg[wit={E,P,B,L,U2}]{phalaṃ bahutaraṃ}}\dd{}
\note[type=philcomm, labelb=159, lem={etādṛśaṃ}]{Witness P includes a dittography of the previous lines here and reads: \textit{etādṛśaṃ mitratāmāyāṃti sahasravarṣamāyur varddhate apaṭhitaṃ śāstraṃ jihvāgreṇoccarati etādṛśaṃ phalaṃ bahutaraṃ bhavati |}.}
\end{prose}
\end{ekdosis}
%-----------------------------
%idānīm anyataraṃ lakṣyaṃ kathyate/ \E
%idānīṃ aṃtaraṃ lakṣyaṃ   kathyate \P
%idānīṃ antaralakṣaṃ      kartavyaṃ// \B
%idānīṃ aṃtaralakṣaṃ      kartavyaṃ// \L
%idānīṃ antaralakṣyakaṃ   kathyate// \N1
%idānīṃ antaralakṣyaṃ     kathyate// \D
%idānīṃ aṇtaralakṣyaṇaṃ   kathyate// \N2
%idānīṃ aṇtaralakṣyaṇaṃ   kathyate \U1
%idānīm ataraṃ lakṣyaṃ    kathyate// \U2
%-----------------------------
%Now the inner fixation objects are taught. 
%-----------------------------
\begin{ekdosis}
  \ekddiv{type=ed}
       \bigskip
       \centerline{\textrm{\small{[Antaralakṣya]}}}
          \bigskip
          \begin{prose}
    \app{\lem[wit={E,U2},alt={idānīm}]{idānī\skp{m-a}}
      \rdg[wit={ceteri}]{idānīṃ}
}\app{\lem[wit={D}, alt={antaralakṣyaṃ}]{\skm{m-a}ntaralakṣyaṃ}
  \rdg[wit={E}]{anyataraṃ lakṣyaṃ}
  \rdg[wit={P}]{aṃtaraṃ lakṣyaṃ}
  \rdg[wit={B,L}]{antaralakṣaṃ}
  \rdg[wit={N1}]{antaralakṣyakaṃ}
  \rdg[wit={N2,U1}]{aṇtaralakṣyaṇaṃ}
  \rdg[wit={U2}]{ataraṃ lakṣyaṃ}}
\app{\lem[wit={ceteri}]{kathyate}
  \rdg[wit={B,L}]{kartavyaṃ}}/    
%-----------------------------
%mūlakandasthāne brahmadaṇḍotpannā nāḍī śvetavarṇā   brahmadaṇḍaparyantam   ekā brahmanāḍī varttate/ \E
%mūlakaṃdasthāne brahmānaṃḍād utpannā   śvetavarṇā   brahmaraṃdhraparyaṃtaṃ ekā brahmanāḍī varttate   \P
%mūlakaṃ sthāne  brahmānaṃḍād utpannā   śvetāvarṇā   brahmaraṃdhraparyaṃtaṃ ekā nāḍī       vartate/     \B
%mūlakaṃdasthāne brahmānaṃdād utpannā   śvetāvarṇā   brahmaraṃdhraparyaṃtaṃ ekanāḍī        vartate/     \L
%mūlakaṃdasthāne brahmadaṃḍa ityannā    śvetavarṇā   brahmaraṃdhraparyaṃtaṃ ekā brahmanāḍī varttate/ \N1
%mūlakaṃdasthāne brahmadaṃḍād utpannā   śvetavarṇā// brahmaraṃdhraparyaṃtaṃ ekā brahmanāḍī varttate// \D
%mūlakaṃdasthāne brahmadaṇḍad ūtpannā   śvetavarṇā   brahmaraṃdhraparyaṃtaṃ ekā brahmanāḍī varttate/ \N2
%mūlakaṃdasthāne brahmadaṇād ūtpannaḥ   śvetavarṇāṃ  brahmaraṃdhraparyaṃtaṃ ekā brahmanāḍī varttate \U1
%mūlakaṃdasthāne brahmadaṇḍād utpannā   śvetavarṇā   brahmaraṃdhraparyaṃtaṃ ekā brahmanāḍī varttate// \U2
%-----------------------------
%At the location of the root bulp rising from the staff of Brahma up to the aperture of Brahma exists the one white coloured Brahma channel. 
%-----------------------------
\note[type=source, labelb=160, lem={mūlakandasthāne}]{Ysv (PT): mūlakandotthatalato brahmanāḍīsamudbhavā | śvetavarṇā brahmarandhraparyantam eva tiṣṭhati | eṣā tu brahmarandhrākhyā tanmadhye varttate parā |}
\app{\lem[wit={ceteri}]{mūlakandasthāne}
  \rdg[wit={P}]{mūlakaṃ sthāne}}
\app{\lem[wit={ceteri}]{brahmadaṇḍād\skp{-}utpannā}
  \rdg[wit={E}]{brahmadaṇḍotpannā nāḍī}
  \rdg[wit={N1}]{brahmadaṃḍa ityannā}
  \rdg[wit={N2}]{brahmadaṇḍad ūtpannā}
  \rdg[wit={U1}]{brahmadaṇād ūtpannaḥ}}
 śvetavarṇā
\app{\lem[wit={ceteri}]{brahmaraṃdhraparyaṃtaṃ}
  \rdg[wit={E}]{brahmadaṇḍaparyantam}} 
\app{\lem[wit={ceteri}]{ekā brahmanāḍī}
  \rdg[wit={B}]{ekā nāḍī}
  \rdg[wit={L}]{ekanāḍī}}
vartate/
%-----------------------------
%brahmanāḍīmadhye kamalatantusamānākārā koṭisūryavidyutsamaprabhā ūrdhvaṃ calati/ \E
%brahmanāḍīmadhye kamalataṃ samānākārā  koṭisūryavidyutsamaprabhā ūrdhvaṃ calati \P
%brahmanāḍīmadhye kamalataṃtusamānākārā koṭisūryavidyutsabhāprabhā ūrdhvaṃ calati/ \B
%brahmanāḍīmadhye kamalataṃtusamānākārā koṭisūryavidyutsabhāprabhā ūrdhvaṃ calati/ \L
%brahmanāḍīmadhye kamalatantusamānākārā koṭisūryavidyutsamaprabhā ūrdhvaṃ calati/ \N1
%brahmanāḍīmadhye kamalataṃtusamānākārā koṭisūryavidyutsamaprabhā ūrdhvaṃ calati// \D
%\om                                                                              \N2
%brahmanāḍīmadhye kamalatantusamānākārā koṭisūryavidyutsamaprabhā rdhvaṃ ccalati  \U1
%brahmanāḍīmadhye kamalataṃtusamānākārā koṭisūryavidyutsamaprabhā// urdhvaṃ calati  \U2 %%%412.jpg 
%-----------------------------
%The interior of the Brahma channel, which equals a pale-red string shining like 10 million suns, goes upwards. 
%-----------------------------
\note[type=source, labelb=161, lem={kamalatantusamānākārā}]{Ysv (PT): padmatantusamākārā koṭisūryataḍitprabhā | calaty ūrddhaṃ mahāmūrttir asya dhyānād bhavec chivaḥ | aṇimādy aṣṭasiddhis tu samagreṇa prasīdati |}
\app{\lem[wit={ceteri}]{brahmanāḍīmadhye}
  \rdg[wit={N2}]{\om}}
\app{\lem[wit={ceteri}]{kamalatantusamānākārā}
  \rdg[wit={P}]{kamalataṃ samānākārā}
  \rdg[wit={N2}]{\om}}
koṭisūryavidyutsa\app{\lem[wit={ceteri},alt={°maprabhā}]{maprabhā}
  \rdg[wit={B,L}]{°bhāprabhā}
  \rdg[wit={N2}]{\om}}
\app{\lem[wit={ceteri}]{ūrdhvaṃ}
  \rdg[wit={U1}]{°rdhvaṃ}
  \rdg[wit={N2}]{urdhvaṃ}}
\app{\lem[wit={ceteri}]{calati}
  \rdg[wit={N2}]{\om}}/
%-----------------------------
%etādṛśy ekā mūrttir varttate/  tan    mūrter dhyānakāraṇāt      aṣṭamahāsiddhayo  'ṇimādayas   tasya                                                                               puruṣasya samīpam āgatya tiṣṭhanti// \E
%etādṛśy ekā mūrttir vartate    tasyā  mūrter dhyānakaraṇāt      aṣṭamahāsiddhayo   ṇimādyāḥ    aṇimā-mahimā-laghīmā-girimā-dure dīya vā            dure  stutvā parakāyapraveśītā   puruṣasya samīm   āgatya tiṣṭhaṃti \P
%etādṛśy ekā mūrttir varttate/  tasyā  mūrte  dhyānakaraṇāt//    aṣṭamahāsiddhayo// aṇimādyāḥ// aṇimā-mahimā-laghimā-girimā-dure vā yadi vā yadi vā dure  śrutvā parakāyāpraveśitā// puruṣasya samīpem āgatya tiṣṭhati// \B
%etādṛśy ekā mūrttir varttate/  tasyā  mūrter dhyānakaraṇāt//    aṣṭamahāsiddhayo   aṇimādyāḥ// aṇimā-mahimā-laghimā-garimā-dure vā yadi         vā ddure śrutvā parakāyāpraveśitā   puruṣasya samīpam āgatya tiṣṭhati// \L
%etādṛśī ekā mūrttir varttate/  tasyāḥ mūrtter dhyānakāraṇāt/    aṇimādīsiddhiḥ                                                                                                     puruṣasya samīpe? āgatya tiṣṭhanti// \N1
%etādṛśī ekā mūrttir varttate// tasyā  mūrtter dhyānakāraṇāt//   aṇimādyaṣṭasiddhiḥ                                                                                                 puruṣasya samīpe  āgatya tiṣṭhati// \D
%\om                            tasyā  mūrtter dhyānakaraṇāc                                                                                        \N2
%                                                                aṇimādyaṣṭasiddhiḥ                                                                                                 puruṣasya sāmīpe  āgatya tiṣṭhati \U1
%etādṛśy ekā mūrttir varttate// tasyā   mūrter dhyānakaraṇāt//   aṣṭamahāsiddhayo aṇimādyāḥ//                                                                                       puruṣasya samīpam āgamya tiṣṭhati// \U2
%-----------------------------
%A particular manifestation exists as such. Due to the execution of meditation on this manifestation, the eight great siddhis of humans beginning with aṇima etc. become established after one has entered into [the manufestation's] imminence. 
%-----------------------------
\app{\lem[wit={ceteri}]{etādṛśy\skp{-}ekā}
  \rdg[wit={N1,D}]{etādṛśī ekā}
  \rdg[wit={U1,N2}]{\om}}
\app{\lem[wit={ceteri}, alt={mūrtir}]{mūrti\skp{r-va}}
  \rdg[wit={U1,N2}]{\om}
}\app{\lem[wit={ceteri}, alt={vartate}]{\skm{r-va}rtate}
  \rdg[wit={U1,N2}]{\om}}/
\end{prose}
\end{ekdosis}
\ekdpb*{}
%%%%%%%%%%%%%%%%%%%%%%%%%%%%%%%%%%%%%%%%%%
%%%%%%%%%%%%%%%%%%%%%%%%%%%%%%%%%%%%%%%%%%
%%%%%%%%PAGEBREAK%%%%%%%PAGEBREAK%%%%%%%%%
%%%%%%%%%%%%%%%%%%%%%%%%%%%%%%%%%%%%%%%%%%
%%%%%%%%%%%%%%%%PAGEBREAK%%%%%%%%%%%%%%%%%
%%%%%%%%%%%%%%%%%%%%%%%%%%%%%%%%%%%%%%%%%%
%%%%%%%%PAGEBREAK%%%%%%%PAGEBREAK%%%%%%%%%
%%%%%%%%%%%%%%%%%%%%%%%%%%%%%%%%%%%%%%%%%%
%%%%%%%%%%%%%%%%%%%%%%%%%%%%%%%%%%%%%%%%%%
%%%%%%%%%%%%%%%%%%%%%%%%%%%%%%%%%%%%%%%%%%
%%%%%%%%%%%%%%%%%%%%%%%%%%%%%%%%%%%%%%%%%%
%%%%%%%%PAGEBREAK%%%%%%%PAGEBREAK%%%%%%%%%
%%%%%%%%%%%%%%%%%%%%%%%%%%%%%%%%%%%%%%%%%%
%%%%%%%%%%%%%%%%PAGEBREAK%%%%%%%%%%%%%%%%%
%%%%%%%%%%%%%%%%%%%%%%%%%%%%%%%%%%%%%%%%%%
%%%%%%%%PAGEBREAK%%%%%%%PAGEBREAK%%%%%%%%%
%%%%%%%%%%%%%%%%%%%%%%%%%%%%%%%%%%%%%%%%%%
%%%%%%%%%%%%%%%%%%%%%%%%%%%%%%%%%%%%%%%%%%
%%%%%%%%%%%%%%%%%%%%%%%%%%%%%%%%%%%%%%%%%%
%%%%%%%%%%%%%%%%%%%%%%%%%%%%%%%%%%%%%%%%%%
%%%%%%%%PAGEBREAK%%%%%%%PAGEBREAK%%%%%%%%%
%%%%%%%%%%%%%%%%%%%%%%%%%%%%%%%%%%%%%%%%%%
%%%%%%%%%%%%%%%%PAGEBREAK%%%%%%%%%%%%%%%%%
%%%%%%%%%%%%%%%%%%%%%%%%%%%%%%%%%%%%%%%%%%
%%%%%%%%PAGEBREAK%%%%%%%PAGEBREAK%%%%%%%%%
%%%%%%%%%%%%%%%%%%%%%%%%%%%%%%%%%%%%%%%%%%
%%%%%%%%%%%%%%%%%%%%%%%%%%%%%%%%%%%%%%%%%%
\begin{ekdosis}
  \begin{prose}
    \noindent
\app{\lem[wit={ceteri}]{tasyā}
  \rdg[wit={N1}]{tasyāḥ}
  \rdg[wit={E}]{tan}
  \rdg[wit={U1}]{\om}}
\app{\lem[wit={ceteri}, alt={mūrter}]{mūrte\skp{r-dhyā}}
  \rdg[wit={B}]{mūrte}
  \rdg[wit={U1}]{\om}
}\app{\lem[wit={ceteri}, alt={dhyāna°}]{\skm{r-dhyā}na}
  \rdg[wit={U1}]{\om}
}\app{\lem[resp=egoscr, type=emendation, alt={kāraṇād}]{kāraṇā\skp{d-a}}
  \rdg[wit={ceteri}]{\korr karaṇāt}
  \rdg[wit={N2}]{dhyānakaraṇāc°}
  \rdg[wit={U1}]{\om}
}\app{\lem[wit={U2}, alt={aṣṭamahāsiddhayo}]{\skm{d-a}ṣṭamahāsiddhayo}
  \rdg[wit={U1,D}]{aṇimādyaṣṭasiddhiḥ}
  \rdg[wit={N1}]{aṇimādīsiddhiḥ}
  \rdg[wit={E,P,B,L}]{aṣṭamahāsiddhayo}
  \rdg[wit={N2}]{\om}}
\app{\lem[wit={P}]{'ṇimādyāḥ}
  \rdg[wit={E}]{'ṇimādayas tasya}
  \rdg[wit={B,L,U2}]{aṇimādyāḥ}
  \rdg[wit={ceteri}]{\om}}
\note[type=philcomm, labelb=162, lem={'ṇimādyāḥ}]{Witnesses P, B and L add a incomplete list of eight supernatural powers here: \textit{aṇimāmahimālaghimāgarimādure vā yadi vā dure śrutvā parakāyāpraveśitā} | Since the list is incomplete and corrupt and stemmatically a later addition, I have decided not to include it into the edition's text.}
\app{\lem[wit={ceteri}]{puruṣasya}
  \rdg[wit={N2}]{\om}}
\app{\lem[wit={N1,D}]{samīpe}
  \rdg[wit={U1}]{sāmīpe}
  \rdg[wit={B}]{samīpem}
  \rdg[wit={E,L,U2}]{samīpam}
  \rdg[wit={P}]{samīm}
  \rdg[wit={N2}]{\om}}
\app{\lem[wit={ceteri}]{āgatya}
  \rdg[wit={U2}]{āgamya}
  \rdg[wit={N2}]{\om}}
\app{\lem[wit={E,P,N1}]{tiṣṭhanti}
  \rdg[wit={ceteri}]{tiṣṭhati}
  \rdg[wit={N2}]{\om}}/%
%[Aṇima (the ability to reduce size to the size of the smallest particle), Mahimā (the ability to expand one's body to an infinitely large size), Laghimā (the ability to become weightless or lighter than air), Garimā (the ability to become heavy or dense), Dūraśravaṇa (Hearing things far away), Dūradarśanam (Seeing things far away) Parakāya praveśitā: Entering the bodies of others.
%-----------------------------
%atha vā lalāṭopary ākāśamadhye śuklasadṛśasya tejaso dhyānakāraṇāt       śarīrasambandhinaḥ  kuṣṭhādayo rogā  naśyanti/    āyur vṛddhir bhavati/  \E
%atha vā lalāṭopari ākāśamadhye śuklasadṛśasya tejaso dhyānakāraṇāt       śarīrasaṃbaṃdhinaḥ  kuṣṭhādayo rogāḥ naśyaṃtī     āyur vṛddhir bhavati   \P  %%%7647.jpg
%atha vā lalāṭopari ākāśamadhye śuklasadṛśasya tejaso dhyānakāraṇāt//     charīrasambandhinaḥ kuṣṭhādayo rogā  naśyaṃtī//   āyur vṛddhir bhavatī   \B 
%atha vā lalāṭopari ākāśamadhye śuklasadṛśasya tejaso dhyānakāraṇāt       charīrasambandhinaḥ kuṣṭhādayo rogā  naśyaṃti//   āyur vṛddhir bhavati// \L
%atha vā lalāṭopari ākāśamadhye śuklasadṛśasya tejaso dhyānakāraṇāt       śarīrasambandhī     kuṣṭhādayo rogāḥ naśyaṃti/    āyur vṛddhir bhavati/  \N1
%atha vā lalāṭopari ākāśamadhye śuklasadṛśasya tejaso dhyānakāraṇāt       śarīrasaṃbaṃdhī     kuṣṭādayo  rogāḥ naśyaṃti//   āyur vṛddhir bhavati//  \D
%                                                                         charīrasaṃbaṃdhi----kuṣṭadayo  rogāḥ naśyaṃti     āyur vṛddi   bhavati/  \N2
%atha vā lalāṭoparī ākāśamadhye śuklasadṛśasya tejo   dhyānakāraṇāt       śarīrasambaṃdhī     kuṣṭhādayo rogā  naśyaṃti     āyur vṛddhir bhavati   \U1 %%%279.jpg
%atha vā lalāṭoparī ākāśamadhye śuklasadṛśasya tejaso dhyānakāraṇāt//     śarīrasambaṃdhinaḥ  kuṣṭhādayo rogā  naśyaṃti//   āyur vṛddhir bhavati//  \U2
%-----------------------------
%Or from the execution of meditation onto the bright light at the centre within the space at the forehead diseases related to the body beginning with leprosy vanish. Lifeforce increases.   
%-----------------------------
\note[type=source, labelb=163, lem={lalāṭopari}]{Ysv (PT): lalāṭopari vā dhyātvā candraṃ vā jyotir īśvaram | nāśayet kuṣṭharogādīn mahāyuṣmān śivaḥ paraḥ |}
\app{\lem[wit={ceteri}]{atha vā}
  \rdg[wit={N2}]{\om}}
\app{\lem[wit={E}, alt={lalāṭopary}]{lalāṭopa\skp{ry-ā}}
  \rdg[wit={ceteri}]{lalāṭopari}
  \rdg[wit={N2}]{\om}
}\app{\lem[wit={ceteri},alt={ākāśamadhye}]{\skm{ry-ā}kāśamadhye}
  \rdg[wit={N2}]{\om}}
\app{\lem[wit={ceteri}]{śuklasadṛśasya}
  \rdg[wit={N2}]{\om}}
\app{\lem[wit={ceteri}]{tejaso}
  \rdg[wit={N2}]{\om}}
\app{\lem[resp=egoscr, type=emendation, alt={dhyānakaraṇāc}]{dhyānakāraṇā\skp{c-cha}}
  \rdg[wit={ceteri}]{\korr dhyānakāraṇāt}
  \rdg[wit={N2}]{\om}
}\app{\lem[wit={B,L}, alt={śarīra°}]{\skm{c-cha}rīra}
  \rdg[wit={ceteri}]{śarīra°}
}\app{\lem[wit={E,P,B,L,U2}, alt={°sambandhinaḥ}]{sambandhinaḥ}
    \rdg[wit={N1,D,U1}]{°sambandhī}
    \rdg[wit={U2}]{saṃbaṃdhi}}
\app{\lem[wit={ceteri}]{kuṣṭhādayo}
  \rdg[wit={D,N2}]{kuṣṭādayo}}
\app{\lem[wit={ceteri}]{rogā}
  \rdg[wit={P,N1,D,N2}]{rogāḥ}}
\app{\lem[wit={ceteri}]{naśyanti}
  \rdg[wit={P,B}]{naśyaṃtī}}
āyur-vṛddhir-bhavati/ 
%-----------------------------
%          bhruvor madhye  tiriktavarṇasyātisthūlasya     tejaso dhyānakāraṇād bahulānāṃ   pārthivānāṃ tatpuruṣāṇāṃ ca vallabho bhavati/ jagadvallabho pi bhavati/      \E
%atha vā   bhruvor madhye  tiraktavarṇasyātisthūlasya     tejaso dhyānakaraṇāt   sakalānāṃ pārthivapuruṣāṇāṃ           vallabho bhavati          \P
%atha vā// bhruvor madhye 'tiraktavarṇasyātisthūlasya     tejaso dhyānaṃ karaṇāt-sakalānāṃ pārthivapuruṣāṇāṃ vallabho bhavati/         \B DSCN7163.jpg Z.1
%atha vā// bhruvor madhye 'tiraktavarṇasyātisthūlasya     tejaso dhyānakaraṇāt   sakalānāṃ pārthivapuruṣāṇāṃ vallabho bhavati/         \L
%atha vā   bhruvor madhye 'tiraktavarṇasyātisthūlasya     tejaso dhyānakaraṇāt-sakalānāṃ   pārthivapuruṣāṇāṃ vallabho bhavati/           \N1
%atha vā   bhruvor madhye 'tiraktavarṇasyātisthūlasya     tejaso dhyānakaraṇāt-sakalānā    pārthivapuruṣāṇāṃ vallabho bhavati             \D %%%p.10 verso
%atha vā   bhruvor madhye  tiraktavarṇasyātisthūlasya     tejaso dhyānakaraṇāt-sakālānāṃ   pārthivapuruṣāṇāṃ vallabho bhavati/             \N2
%atha vā   bhruvor madhye  tiraktavarṇasyātī sthalasya    tejaso dhyānakaraṇāt sakalānāṃ   pārthivapuruṣāṇāṃ vallabho bhavati/          \U1
%atha vā   bṛvor   madhye atiraktavarṇasya 'tisthūlasyaḥ  tejāso dhyānakaraṇāt sakalānāṃ   pārthivapuruṣāṇāṃ vallabho bhavati         \U2
%-----------------------------
%Or because of executing meditation on the middle of the eyebrows [of which there is] a very subtle and red colored light, he is one who is beloved among all royal people.    
%-----------------------------
\note[type=source, labelb=164, lem={bhruvor madhye}]{Ysv (PT): bhruvor madhye 'thavā dhyātvā arkantu teja īśvaram |  sthiradṛṣṭau rājapūjyo jīvanmuktaḥ śivo yathā | ātmānam ātmarūpaṃ hi dhyātvā yo niṣkriyo bhavet | nirāśīryatatattvo 'yaṃ itaro na nṛpasthitiḥ |}
\app{\lem[wit={ceteri}]{atha vā}
  \rdg[wit={E}]{\om}}
\app{\lem[wit={ceteri}, alt={bhruvor}]{bhruvo\skp{r-ma}}
  \rdg[wit={U2}]{bṛvor}
}\skm{r-ma}dhye
\app{\lem[wit={ceteri}, alt={'tirakta°}]{'tirakta}
  \rdg[wit={U2}]{atirakta°}
  \rdg[wit={E}]{tirikta°}
}\app{\lem[wit={ceteri}]{varṇasyātisthūlasya}
  \rdg[wit={U1}]{varṇasyātī sthalasya}
  \rdg[wit={U2}]{'tisthūlasyaḥ}}
tejaso
\app{\lem[wit={ceteri}, alt={dhyānakaraṇāt}]{dhyānakaraṇā\skp{t-sa}}
  \rdg[wit={B}]{dhyānaṃ karaṇāt}
  \rdg[wit={E}]{dhyānakāraṇād}
}\app{\lem[wit={ceteri}]{\skm{t-sa}kālānāṃ}
  \rdg[wit={D}]{sakalānā}
  \rdg[wit={E}]{bahulānāṃ}}
pārthi\app{\lem[wit={ceteri},alt={°vapuruṣāṇāṃ}]{vapuruṣāṇāṃ}
  \rdg[wit={E}]{°vānāṃ tatpuruṣāṇāṃ ca}}
vallabho\app{\lem[wit={ceteri}]{bhavati}
  \rdg[wit={E}]{bhavati | jagad vallabho pi bhavati}}/
%-----------------------------
%asya puruṣasyāvalokanena sarveṣāṃ dṛṣṭiḥ sthirā bhavati// \E
%taṃ  puruṣaṃ        pratisarveṣāṃ dṛṣṭiḥ sthirā bhavati  \P
%taṃ  puruṣaṃ        pratisarveṣāṃ dṛṣṭisthirā bhavatī// \B
%taṃ  puruṣa         pratisarveṣāṃ dṛṣṭisthirā bhavati// \L
%taṃ  puruṣaṃ dṛṣṭvā      sarveṣāṃ dṛṣṭisthirā bhavati// \N1
%taṃ  puruṣaṃ dṛṣṭvā      sarveṣāṃ dṛṣṭisthirā bhavati// \D
%taṃ  puruṣaṃ dṛṣṭā       sarveṣāṃ dṛṣṭisthirā bhavati// \N2
%taṃ  puruṣaṃ dṛṣṭvā      sarveṣāṃ dṛṣṭisthirā bhavati \U1
%taṃ  puruṣaṃ        pratisarveṣāṃ dṛṣṭisthirā bhavati// \U2
%----------------------------
%Having seen this person, everybody's gaze is fixed onto him. 
%-----------------------------
\app{\lem[wit={ceteri}]{taṃ}
  \rdg[wit={E}]{asya}}
\app{\lem[wit={N1,D,U1}]{puruṣaṃ dṛṣṭvā}
  \rdg[wit={N2}]{puruṣaṃ dṛṣṭā}
  \rdg[wit={P,B}]{puruṣaṃ}
  \rdg[wit={L}]{puruṣa°}
  \rdg[wit={E}]{puruṣasyāvalokanena}}
\app{\lem[wit={E,N1,D,N2,U1}]{sarveṣāṃ}
  \rdg[wit={ceteri}]{pratisarveṣāṃ}}
\app{\lem[wit={ceteri}]{dṛṣṭisthirā}
  \rdg[wit={E,P}]{dṛṣṭiḥ sthirā}}
\app{\lem[wit={ceteri}]{bhavati}
  \rdg[wit={B}]{bhavati}}\dd{}
\end{prose}
\end{ekdosis}
%-----------------------------
%-----------------------------
%-----------------------------
\begin{ekdosis}
  \ekddiv{type=ed}
 \bigskip
 \centerline{\textrm{\small{[The Ten Main Bodily Channels]}}}
  \bigskip
  \begin{prose}   
\note[type=source, labelb=165, lem={nāḍīnāṃ}]{SSP: atha nāḍīnāṃ daśadvārāṇi iḍā piṅgalā ca nāsādvārayor vahataḥ | gāndhārī hastijihvikā ca cakṣurdvārayor vahataḥ | pūṣā yaśasvinī ca karṇadvārayor vahataḥ | alambuṣā ānane vahati | kuhūr gudadvāre vahati | śaṃkhinī liṅgadvāre vahati | suṣumṇā madhyadeśe vahati | sā daṇḍamārgeṇa brahmarandhraparyantaṃ vahati | evaṃ daśanāḍyo daśadvāreṣu vahanti | anyāḥ sarvanāḍyo romakūpeṣu vahanti ||1.66||}    
\note[type=source, labelb=166, lem={nāḍīnāṃ}]{Ysv (PT): idānīṃ śṛṇu nāḍīnāṃ bhedaṃ vakṣyāmi siddhidam | meruvāhye iḍānāmnī piṅgalayā samanvitā | suṣumnā bhānumārgeṇa brahmadvārāvadhi sthitā | sarasvatī sugandhā tu gāndhārī hastijihvakā | jñātavyā karṇayormadhye netrayoś ca tathāntimā | pūṣā cālambuṣā ceti mūlasthā kutracit tathā | liṅgadvārādiḍāmārge brahmasthānāvadhi priye | nāḍyantaṃ pratilomeṣu sahasrāṇāṃ dvisaptatiḥ |}
%-----------------------------   
%idānīṃ śarīramadhye nāḍīnāṃ       bhedāḥ   kathyante  daśamukhyanāḍyaḥ/ \E
%idānīṃ śarīramadhye nāḍīnāṃ       bhedāḥ   kathyaṃte  daśamukhyānāḍyaḥ \P
%idānī  śarīramadhye nāḍī----------bhedaḥ   kathyate// daśamukhyenāḍyā \B
%idānī  śarīramadhye nāḍī----------bhedaḥ   kathyate// daśamukhyānāḍayas... \L
%idānīṃ śarīramadhye nāḍīnām aparo bhedaḥ   kathyate// daśamukhyanādhyaḥ/ \N1
%idānīṃ śarīramadhye nāḍīnām aparo bhedaḥ// kathyaṃte  daśamukhyānādhyaḥ// \D
%idānī  śarīramadhye nāḍīnām aparo bhedāḥ   kathyate// daśamukhyanāḍyaḥ// \N2
%idānīṃ śarīramadhye nāḍīnām aparo bhedāḥ   kathyate   daśamukhyanāḍyas \U1
%idānīṃ śarīramadhye nāḍīnaṃ       bhedaḥ   kathyate   daśamukhyanāḍyaḥ// \U2
%-----------------------------
%Now the divisions of channels within the body are explained. There are ten primary channels. 
%-----------------------------
\app{\lem[wit={ceteri}]{idānīṃ}
  \rdg[wit={L,B,N2}]{idānī}}
śarīramadhye 
\app{\lem[wit={ceteri}]{nāḍīnāṃ}
  \rdg[wit={B,L}]{nāḍī°}
  \rdg[wit={N1,N2,D,U1}]{nāḍīnām aparo}}
\app{\lem[wit={ceteri}]{bhedāḥ}
  \rdg[wit={B,L,N1,D}]{bhedaḥ}}
\app{\lem[wit={E,P,N2,U1}]{kathyante}
  \rdg[wit={ceteri}]{kathyate}}/
\app{\lem[wit={E,N2,U1,U2}]{daśamukhyanāḍyaḥ}
  \rdg[wit={P}]{daśamukhyānāḍyaḥ}
  \rdg[wit={B}]{daśamukhyenāḍyā}
  \rdg[wit={L}]{daśamukhyānāḍayas}
  \rdg[wit={N1,D}]{daśamukhyanādhyaḥ}}/
\end{prose}
\end{ekdosis}
\ekdpb*{}
%%%%%%%%%%%%%%%%%%%%%%%%%%%%%%%%%%%%%%%%%%
%%%%%%%%%%%%%%%%%%%%%%%%%%%%%%%%%%%%%%%%%%
%%%%%%%%PAGEBREAK%%%%%%%PAGEBREAK%%%%%%%%%
%%%%%%%%%%%%%%%%%%%%%%%%%%%%%%%%%%%%%%%%%%
%%%%%%%%%%%%%%%%PAGEBREAK%%%%%%%%%%%%%%%%%
%%%%%%%%%%%%%%%%%%%%%%%%%%%%%%%%%%%%%%%%%%
%%%%%%%%PAGEBREAK%%%%%%%PAGEBREAK%%%%%%%%%
%%%%%%%%%%%%%%%%%%%%%%%%%%%%%%%%%%%%%%%%%%
%%%%%%%%%%%%%%%%%%%%%%%%%%%%%%%%%%%%%%%%%%
%%%%%%%%%%%%%%%%%%%%%%%%%%%%%%%%%%%%%%%%%%
%%%%%%%%%%%%%%%%%%%%%%%%%%%%%%%%%%%%%%%%%%
%%%%%%%%PAGEBREAK%%%%%%%PAGEBREAK%%%%%%%%%
%%%%%%%%%%%%%%%%%%%%%%%%%%%%%%%%%%%%%%%%%%
%%%%%%%%%%%%%%%%PAGEBREAK%%%%%%%%%%%%%%%%%
%%%%%%%%%%%%%%%%%%%%%%%%%%%%%%%%%%%%%%%%%%
%%%%%%%%PAGEBREAK%%%%%%%PAGEBREAK%%%%%%%%%
%%%%%%%%%%%%%%%%%%%%%%%%%%%%%%%%%%%%%%%%%%
%%%%%%%%%%%%%%%%%%%%%%%%%%%%%%%%%%%%%%%%%%
%%%%%%%%%%%%%%%%%%%%%%%%%%%%%%%%%%%%%%%%%%
%%%%%%%%%%%%%%%%%%%%%%%%%%%%%%%%%%%%%%%%%%
%%%%%%%%PAGEBREAK%%%%%%%PAGEBREAK%%%%%%%%%
%%%%%%%%%%%%%%%%%%%%%%%%%%%%%%%%%%%%%%%%%%
%%%%%%%%%%%%%%%%PAGEBREAK%%%%%%%%%%%%%%%%%
%%%%%%%%%%%%%%%%%%%%%%%%%%%%%%%%%%%%%%%%%%
%%%%%%%%PAGEBREAK%%%%%%%PAGEBREAK%%%%%%%%%
%%%%%%%%%%%%%%%%%%%%%%%%%%%%%%%%%%%%%%%%%%
%%%%%%%%%%%%%%%%%%%%%%%%%%%%%%%%%%%%%%%%%%
\begin{ekdosis}
  \begin{prose}
    \noindent
%----------------------------- 
%tanmadhye dvayam       iḍā  piṃgalāsaṃjñakaṃ       nāsādvāre tiṣṭhati/ \E
%tanmadhye nāḍīdvayaṃ   idāṃ piṃgalāsaṃjñakaṃ       nāsādvāre tiṣṭhati  \P
%tanmadhye nāḍīdvayaṃ/  idāpiṃgalāsaṃjñīkāḥ         nāsādvāre tiṣṭhati//  \B
%tanmadhye nāḍīdvayaṃ   idāpiṃgalāsaṃjñīkāḥ         nāsādvāre tiṣṭhati//  \L
%tanmadhye nāḍīdvayam/  idāpiṃgalāsaṃjñakaṃ         nāsādvāre tiṣṭhati//  \N1
%tanmadhye nāḍīdvayaṃ   idāpiṃgalāsaṃjñakaṃ         nāsānāsādvāre tiṣṭhati//  \D
%tanmadhye nāḍīdvayam/  idānīṃ piṃgalāsaṃjñakaṃ     nāsādvāre tiṣṭhati//  \N2
%tanmadhye nāḍīdvayaṃ   idāpiṃgalāsaṃjñākaṃ         nāsādvāre tiṣṭhati  \U1
%tanmadhye nāḍidvayaṃ// idā// piṃgalā// saṃjñākaṃ// nāsādvāre tiṣṭhati//  \U2
%-----------------------------
%Among them the pair of channels designated Idā and the Piṅgalā exists at the entrance of the nose. 
%-----------------------------
tanmadhye
\app{\lem[resp=egoscr, type=emendation, alt={nāḍīdvayam}]{nāḍīdvaya\skp{m-i}}
  \rdg[wit={E}]{dvayam}
  \rdg[wit={ceteri}]{nāḍīdvayaṃ}
}\app{\lem[wit={E}, alt={iḍāpiṅgalā}]{\skm{m-i}ḍāpiṃgalā}
  \rdg[wit={ceteri}]{idā piṃgalā}
  \rdg[wit={N2}]{idānīṃ piṃgalā}
  \rdg[wit={P}]{idāṃ piṃgalā}
}\app{\lem[wit={ceteri}]{saṃjñakaṃ}
  \rdg[wit={U1,U2}]{saṃjñākaṃ}
  \rdg[wit={ceteri}]{saṃjñīkāḥ}}
\app{\lem[wit={ceteri}]{nāsādvāre}
  \rdg[wit={D}]{nāsānāsādvāre}}
tiṣṭhati/
%-----------------------------
%suṣumṇā    tālumārge   brahmadvāraparyantaṃ   vahati tiṣṭhati/ \E
%suṣumṇā    tālumārgeṇa brahmaraṃdhraparyanta--vahati tiṣṭhati... \P
%suṣumṇā    tālumārge   brahmaraṃdhraparyantaṃ vahatī tiṣṭhati... \B
%suṣumṇā    tālumārge   brahmaraṃdhraparyantaṃ vahati tiṣṭhati... \L
%suṣumṇā tu tālumārgeṇa brahmadvāraparyantaṃ   vahatī tiṣṭhati... \N1
%suṣumṇā tu tālumārgeṇa brahmadvāraparyantaṃ   vahatī tiṣṭhati    \D
%suṣumṇā tu tālumārge   brahmadvāraparyantaṃ   vahatī tiṣṭhati// \N2
%suṣumṇā tu tālumārgeṇa brahmadvāraparyantaṃ   vahati tiṣṭhati \U1
%suṣumṇā    tālumārgeṇa brahmadvāraparyantaṃ   vahati// \U2
%-----------------------------
%The river being the central channel leads from the palate to the door of Brahma.  
%-----------------------------
\app{\lem[wit={ceteri}]{suṣumṇā}
  \rdg[wit={N1,N2,D,U1}]{suṣumṇā tu}}
\app{\lem[wit={ceteri}]{tālumārgeṇa}
  \rdg[wit={E,B,L,N2}]{tālumārge}}
brahma\app{\lem[wit={ceteri}, alt={°dvāra°}]{dvāra}
  \rdg[wit={P,B,L}]{°raṃdhra°}
}paryantaṃ
\app{\lem[wit={U2}]{vahati}
  \rdg[wit={E,P,L,U1}]{vahati tiṣṭhati}
  \rdg[wit={ceteri}]{vahati tiṣṭhatī}}/
%-----------------------------
%        sarasvatī mukhamadhye tiṣṭhati/ \E
%        sarasvatī mukhamadhye tiṣṭhati  \P
%        sarasvatī mukhamadhye tiṣṭhatī/ \B
%        sarasvatī mukhamadhye tiṣṭhati/ \L
%        sarasvatī mukhamadhye varttate/ \N1
%        sarasvatī mukhamadhye varttate// \D
%        sarasvatī mukhamadhye varttate/ \N2
%        sarasvatī mukhamadhye varttate \U1
%ti sraḥ sarasvati mukhamadhye tiṣṭhati// \U2
%-----------------------------
%The Sarasvatī[-channel] exists at the centre of the face. 
%-----------------------------
\app{\lem[wit={ceteri}]{sarasvatī}
  \rdg[wit={U2}]{ti sraḥ sarasvati}}
mukhamadhye
\app{\lem[wit={N1,N2,D,U1}]{varttate}
  \rdg[wit={E,P,L,U2}]{tiṣṭhati}
  \rdg[wit={B}]{tiṣṭhatī}}/
%-----------------------------
%gāṃdhārīhastijihvākarṇayor            madhye  vahalyau  tiṣṭhataḥ/    \E
%gāṃdhārīhastinījihve karṇayor         madhye  vahatyau  tiṣṭhataḥ   \P
%gāṃdhārīhastījihve karṇa----------------------vahatyo   tiṣṭhati//                \B
%gāṃdhārīhastijihve karṇa----------------------vahatyo   tiṣṭhati...               \L
%gāṃdhārīhastinījihve karṇayor         madhye  vahatyau  tiṣṭhataḥ// \N1
%gāṃdhārīhastinījihve karṇayor         madhye  vahatyau  tiṣṭhataḥ// \D
%gāṃdhārīhastinījihve karṇayor         madhye  vahatyau  tiṣṭhataḥ// \N2
%gāṃdhādīharratījihvakarṇayor          madhye            tiṣṭhataḥ              \U1
%gāṃdhārī// hastinī// jihve// netrayor madhye  vahaṃtyaḥ//    \U2
%-----------------------------
%The two rivers Gāṃdhārī and Hastjihvā exist within the centre of the two ears. 
%-----------------------------
gāṃdhārī\app{\lem[wit={E}, alt={hastijihvākarṇayor}]{hastijihvākarṇayo\skp{r-ma}}
  \rdg[wit={P,N1,N2,D}]{hastinījihve karṇayor}
  \rdg[wit={B,L}]{hastījihve karṇa°}
  \rdg[wit={U1}]{harratījihvakarṇayor}
  \rdg[wit={U2}]{hastinī || jihve || netrayor}}
\app{\lem[wit={ceteri}, alt={madhye}]{\skm{r-ma}dhye}
  \rdg[wit={L,B}]{\om}}
\app{\lem[wit={P,N1,D,D}]{vahatyau}
  \rdg[wit={E}]{vahalyau}
  \rdg[wit={B,L}]{vahatyo}
  \rdg[wit={U2}]{vahaṃtyaḥ}}
\app{\lem[wit={ceteri}]{tiṣṭhataḥ}
  \rdg[wit={B,L}]{tiṣṭhati}
  \rdg[wit={U2}]{\om}}/
%-----------------------------
%pūṣālambusemā         netrayor madhye rvahalyā tiṣṭhataḥ/ \E
%pūṣālaṃbuse           netrayor madhye vahatyau tiṣṭataḥ \P
%pūṣoḍalabuṣe----------netra----madhye vahatyo  tiṣṭhati/ \B
%pūṣo ulabuso          netra----madhye vahatyo  tiṣṭhaṃti// \L
%pūṣāṃalaṃbuṣe         netrayor madhye vahatyo  tiṣṭhataḥ/ \N1
%pūṣāṃalaṃbuṣe         netrayor madhye vahatyau tiṣṭhataḥ// \D
%pūṣāṃalaṃbuṣe         netayor  madhye vahatyo  tiṣṭhataḥ/ \N2
%pūṣālaṃbuṣe           netayor  madhye vahatyau tiṣṭhataḥ \U1
%pūṣāya śakhinī// karṇayor      madhye vahatyo  tiṣṭhata// alaṃbuṣā// bhu?madhye vaṃhatyo tiṣṭhati// \U2
%-----------------------------
%The two rivers Pūṣā and Ālaṃbuṣā are situated at the center of the two eyes. 
%-----------------------------
\app{\lem[resp=egoscr, type=emendation, alt={pūṣālaṃbuṣānetrayor}]{pūṣālaṃbuṣānetrayo\skp{r-ma}}
  \rdg[wit={E}]{pūṣālambusemā netrayor}
  \rdg[wit={P}]{pūṣālaṃbuse netrayor}
  \rdg[wit={B}]{pūṣoḍalabuṣe netra°}
  \rdg[wit={L}]{pūṣo ulabuso netra°}
  \rdg[wit={N1,D}]{pūṣāṃalaṃbuṣe netrayor}
  \rdg[wit={N2}]{pūṣāṃalaṃbuṣe netayor}
  \rdg[wit={U1}]{pūṣālaṃbuṣe netayor}
  \rdg[wit={U2}]{pūṣāya śakhinī || karṇayor}
}\skm{r-ma}dhye
\app{\lem[wit={ceteri}]{vahatyau}
  \rdg[wit={E}]{rvahalyā}
  \rdg[wit={B,L,N1,N2,U2}]{vahatyo}}
\app{\lem[wit={E,N1,N2,D,U1}]{tiṣṭhataḥ}
  \rdg[wit={P}]{tiṣṭataḥ}
  \rdg[wit={B}]{tiṣṭhati}
  \rdg[wit={L}]{tiṣṭhaṃti}
  \rdg[wit={U2}]{tiṣṭhata || alaṃbuṣā || bhrumadhye vaṃhatyo tiṣṭhati ||}}
%-----------------------------
%śaṃkhinī liṃgadvārād ārabhye--ḍāmārgeṇa     brahmasthānaparyaṃtaṃ tiṣṭhatīti/     \E
%śaṃkhinī liṃgadvārād ārabhya iḍāmārgeṇa     brahmasthānaparyaṃtaṃ tiṣṭhati      \P   %%%%%%%7648.jpg
%śaṃkhinī liṃgadvārād ārabhya iḍāmārgeṇa     brahmasthānaparyaṃtaṃ tiṣṭhati/     \B
%śaṃkhinī liṃgadvārād ārabhya iḍāmārgeṇa     brahmasthānaparyaṃtaṃ tiṣṭhati//    \L 
%śaṃkhanī liṃgadvārād ārabhya iḍāmārgeṇa     brahmasthānaparyaṃtaṃ tiṣṭhati/     \N1
%śaṃkhinī liṃgadvārād ārabhya iḍāmārgeṇa     brahmasthānaparyaṃtaṃ tiṣṭhati//     \D
%śaṃkhinī liṃgadvārād ārabhya iḍānīṃ mārgeṇa brahmasthānaparyaṃtaṃ tiṣṭhati/ \N2
%śaṃkhinī liṃgadvārārabhya    iḍāmārgeṇa     brahmasthānaparyaṃtaṃ tiṣṭhati      \U1
%kuhū     liṃgadvārād ārabhya iḍāmārgeṇa     brahmasthānaparyaṃtaṃ tiṣṭhati// śāṃkhinī mūladvārād arabhya piṃgalamargeṇa brahmasthānaparyaṃtaṃ tiṣṭhati// \U2
%-----------------------------
%The Śaṃkhinī channel strechtes from the the beginning of the opening of the penis through the Iḍā-channel. 
%-----------------------------
\app{\lem[wit={U2}]{kuhū}
  \rdg[wit={ceteri}]{śaṃkhinī}
  \rdg[wit={N1}]{śaṃkhanī}}
\app{\lem[wit={ceteri}, alt={liṃgadvārād}]{liṃgadvārā\skp{d-ā}}
  \rdg[wit={U1}]{liṃgadvārā°}}
\app{\lem[wit={ceteri}, alt={ārabhye}]{\skm{d-ā}rabhye}
  \rdg[wit={ceteri}]{ārabhya}
}\app{\lem[wit={E}]{iḍāmārgeṇa}
  \rdg[wit={ceteri}]{iḍāmārgeṇa}
  \rdg[wit={N2}]{iḍānīṃ mārgeṇa}}
brahmasthānaparyaṃtaṃ 
\app{\lem[wit={ceteri}]{tiṣṭhati}
  \rdg[wit={E}]{tiṣṭhatīti}}/
\extra{śāṃkhinī mūladvārād-arabhya
\app{\lem[resp=egoscr, type=emendation, alt={piṃgalā}]{piṃgalā}
  \rdg[wit={U2}]{piṃgala°}}margeṇa brahmasthānaparyaṃtaṃ tiṣṭhati/}
\note[type=philcomm, labelb=177, lem={kuhū}]{I followed Witness U\textsubscript{2} and chose the reading \textit{kuhū} instead of \textit{śaṃkhinī} to arrive at the complete ten channels. Due to similar structure of the sentences describing the channels, the channel \textit{kuhū} dropped in the process of copying in all other witnesses except U\textsubscript{2}.}
%-----------------------------
%etādṛśa  nāḍyo daśasu dvāreṣu tiṣṭhanti/    \E
%etādṛṣā  nāḍyo daśasu dvāreṣu tiṣṭhaṃti      \P
%etādṛṣyā nāḍyo daśasu dvāreṣu tiṣṭhaṃti/    \B
%etādṛṣyā nāḍyo daśa   dvāreṣu    tiṣṭhaṃti/    \L 5876_15.jpg
%etādaśa  nāḍyo daśasu dvāreṣu tiṣṭhaṃti/    \N1
%etādaśa  nāḍyo daśasu dvāreṣu tiṣṭhaṃti//   \D
%etā            daśasu  dvāreṣu tiṣṭhaṃti/                \N2
%etādṛśa  nāḍyo daśasv adhāreṣu  tiṣṭhati    \U1
%etādaśa  nāḍyo daśaśoṣu dvāreṣu tiṣṭhaṃti// \U2 %%%413.jpg
%-----------------------------
%In such a way the channels are situated at the 10 openings. 
%-----------------------------
\app{\lem[wit={P}]{etādṛṣā}
  \rdg[wit={E,N1,D,U1,U2}]{etādṛśa}
  \rdg[wit={B,L}]{etādṛṣyā}
  \rdg[wit={N2}]{etā}}
\app{\lem[wit={ceteri}]{nāḍyo}
  \rdg[wit={N2}]{\om}}
\app{\lem[wit={ceteri}]{daśasu dvāreṣu}
  \rdg[wit={L}]{daśa dvāreṣu}
  \rdg[wit={U1}]{daśasv adhāreṣu}}
\app{\lem[wit={ceteri}]{tiṣṭhanti}
  \rdg[wit={U1}]{tiṣṭhati}}/
%-----------------------------
%anyā dvisaptatisahasraparimitā                      nāḍayo lomnāṃ mūleṣu sūkṣmarūpeṇa tiṣṭanti// \E
%anyā dvisaptatisahasraparimitā                      nāḍyo  lomnā  mūleṣu sūkṣmarūpeṇa tiṣṭaṃti      \P
%anyā dvisaptatīsahasraparimitā                      nāḍyo  lomnā  mūleṣu sūkṣmarūpeṇa tiṣṭaṃti// \B
%anyā dvisaptatisahasraparimitā                      nāḍyo  lomnā  mūleṣu sūkṣmarūpeṇa tiṣṭaṃti// \L
%anyā dvisaptatisahasraparamitā                      nāḍyā  lomnāṃ mūleṣu sūkṣmarūpeṇa tiṣṭaṃti// \N1
%anyā dvisaptatisahasraparamitā                      nāḍyā  lomnāṃ mūleṣu sūkṣmarūpeṇa tiṣṭaṃti// \D
%anyā dvisaptatrisahasraparimitā                     nāḍyā  lomnāṃ mūleṣu sūkṣmarūpeṇa tiṣṭaṃti// \N2
%anyā dvisaptatisahasraparimitāgryo                         lomnā  mūleṣu sūkṣmarūpeṇa tiṣṭaṃti \U1
%anyā hidaśonā dvisatyati sahasraḥ//71110// parimitā nādhyo lomnāṃ mūleṣu sūkṣmarūpeṇa tiṣṭaṃti// \U2
%-----------------------------
%The other channels measured as 72000 are situated with a subtle form at the roots of the hairs.
%-----------------------------
anyā
\app{\lem[wit={ceteri}]{dvisaptatisahasraparimitā}
  \rdg[wit={U1}]{dvisaptatisahasraparimitāgryo}
  \rdg[wit={U2}]{hidaśonā dvisatyati sahasraḥ || 71110 || parimitā}}
\app{\lem[wit={P,B,L}]{nāḍyo}
  \rdg[wit={E}]{nāḍayo}
  \rdg[wit={U2}]{nādhyo}
  \rdg[wit={U1}]{\om}}
\app{\lem[wit={E,N1,N2,D,U2}]{lomnāṃ mūleṣu} %%%lomnāṃ = gen pl neutrum v.loman
  \rdg[wit={P,B,L,U1}]{lomnā}}
sūkṣmarūpeṇa tiṣṭaṃti\dd{}
\end{prose}
\end{ekdosis}
%-----------------------------
%-----------------------------
%-----------------------------
\begin{ekdosis}
  \ekddiv{type=ed}
 \centerline{\textrm{\small{[The Ten Vitalwinds]}}}
 \bigskip
 \begin{prose}
%-----------------------------
%[p.36]
%idānīṃ śarīramadhye vāyavo daśa tiṣṭhanti/ \E
%idānīṃ śarīramadhye vāyavo daśa tiṣṭhaṃti  \P
%idānīṃ śarīramadhye .....\om               \B
%idānīṃ śarīramadhye .....\om               \L
%idānīṃ śarīramadhye vāyavas tiṣṭhaṃti/     \N1
%idānīṃ śarīramadhye vāyavas tiṣṭhaṃti//    \D
%idānīṃ śarīramadhye vāyavas tiṣṭhaṃti/     \N2
%idānīṃ śarīramadhye vāyavas tiṣṭhaṃti      \U1
%idānīṃ śarīramadhye vāyavo daśa ṣṭaṃti//   \U2
%-----------------------------
%Now ten vitalwinds are situated within the body.  
%-----------------------------
\note[type=source, labelb=178, lem={vāyavo}]{Ysv (PT): idānīṃ śṛṇu nāḍīnāṃ bhedaṃ vakṣyāmi siddhidam | meruvāhye iḍānāmnī piṅgalayā samanvitā | suṣumnā bhānumārgeṇa brahmadvārāvadhi sthitā | sarasvatī sugandhā tu gāndhārī hastijihvakā | jñātavyā karṇayor madhye netrayoś ca tathāntimā | pūṣā cālambuṣā ceti mūlasthā kutracit tathā | liṅgadvārādiḍāmārge brahmasthānāvadhi priye | nāḍyantaṃ pratilomeṣu sahasrāṇāṃ dvisaptatiḥ |}
idānīṃ śarīramadhye
\app{\lem[wit={E,P,U2}]{vāyavo}
  \rdg[wit={N1,N2,D,U1}]{vāyavas}
  \rdg[wit={B,L}]{\om}}
\app{\lem[wit={E,P,U2}]{daśa}
  \rdg[wit={ceteri}]{\om}}
\app{\lem[wit={ceteri}]{tiṣṭhanti}
  \rdg[wit={U2}]{ṣṭaṃti}
  \rdg[wit={B,L}]{\om}}/
%-----------------------------
%teṣāṃ nāmāni kāryāṇi kathyante/ \E
%teṣāṃ nāmāni kārmāṇi kathyante/ \P
%\om  \B
%\om \L
%teṣāṃ kāryāṇi kathyante/ \N1
%teṣāṃ kāryāṇi kathyaṃte/ \D
%teṣāṃ kāryāṇi kathyate/ \N2
%teṣāṃ kāryāṇi kathyate \U1 %%%280.jpg
%teṣāṃ kāryāṇi kathyate \U2
%-----------------------------
%their functions are taught. 
%-----------------------------
\app{\lem[wit={ceteri}]{teṣāṃ}
  \rdg[wit={B,L}]{\om}}
\app{\lem[wit={ceteri}]{kāryāṇi}
  \rdg[wit={E}]{nāmāni kāryāṇi}
  \rdg[wit={P}]{nāmāni kārmāṇi}
  \rdg[wit={L,B}]{\om}}
\app{\lem[wit={ceteri}]{kathyante}
  \rdg[wit={N2,U1,U2}]{kathyate}
  \rdg[wit={L,B}]{\om}}/
%-----------------------------
%prāṇavāyur hṛdayamadhye śvāsocchāsaṃ karoti/ \E
%prāṇavāyur hṛdayamadhye śvāsochāsaṃ karoti       \P
%------------------------śvāsośvaroti/               \B
%                        śvāsośvareti...             \L 
%prāṇavāyuhṛdayamadhye  utsvāsaprasvāsasaṃ karoti//   \N1
%prāṇavāyuhṛdayamadhye  utsvāsaprasvāsaṃ karotī   \D
%prāṇavāyuhṛdayamadhye ūrdhvaśvāsapraśvāsaṃ karoti// \N2
%prāṇavāyuhṛdayamadhye ūdhvasaprasase karoti \U1
%prāṇavāyuhṛdayamadhye   svāsochvāsaṃ karoti \U2
%-----------------------------
%The Prāṇa vitalwind is located in the middle of the heart and causes inhalation and exhalation. 
%----------------------------
\app{\lem[wit={E,P}, alt={prāṇavāyur}]{prāṇavāyu\skp{r-hṛ}}
  \rdg[wit={N1,N2,D,U1,U2}]{prāṇavāyu°}
  \rdg[wit={B,L}]{\om}
}\app{\lem[wit={ceteri},alt={hṛdayamadhye}]{\skm{r-hṛ}dayamadhye}
  \rdg[wit={B,L}]{\om}}
\app{\lem[type=emendation, resp=egoscr]{ucchvaśvāsapraśvāsaṃ}
  \rdg[wit={N1}]{\korr utsvāsaprasvāsasaṃ}
  \rdg[wit={N2}]{ūrdhvaśvāsapraśvāsaṃ}
  \rdg[wit={D}]{utsvāsaprasvāsaṃ}
  \rdg[wit={U1}]{ūdhvasaprasase}
  \rdg[wit={E}]{śvāsocchāsaṃ}
  \rdg[wit={P}]{śvāsochāsaṃ}
  \rdg[wit={B}]{śvāsośvaroti}
  \rdg[wit={L}]{śvāsośvareti}}/
\note[type=source, labelb=179, lem={prāṇavāyur}]{SSP: hṛdaye prāṇavāyur ucchvāsaniḥśvāsakārako hakārasakārātmakaś ca | gude tv apānavāyuḥ recakakumbhakapūrakaś ca | nābhau samānavāyuḥ dīpakaḥ pācakaś ca| kaṇṭhe vyānavāyuḥ śoṣaṇāpyāyanakārakaś ca | tālau udānavāyuḥ grasanavamanajalpakārakaś ca| nāgavāyuḥ sarvāṅgavyāpakaḥ mocakaś cālakaś ca | kūrmavāyuḥ cakṣuṣor unmeṣakārakaś ca| kṛkalaḥ udgārakaḥ kṣutkārakaś ca | devadatto mukhavijṛmbhakaḥ | dhanañjayo nādaghoṣakah ||1.67|| iti daśavāyvavalokanena piṇḍotpattiḥ naranārīrūpam |}
\end{prose}
\end{ekdosis}
\ekdpb*{}
%%%%%%%%%%%%%%%%%%%%%%%%%%%%%%%%%%%%%%%%%%
%%%%%%%%%%%%%%%%%%%%%%%%%%%%%%%%%%%%%%%%%%
%%%%%%%%PAGEBREAK%%%%%%%PAGEBREAK%%%%%%%%%
%%%%%%%%%%%%%%%%%%%%%%%%%%%%%%%%%%%%%%%%%%
%%%%%%%%%%%%%%%%PAGEBREAK%%%%%%%%%%%%%%%%%
%%%%%%%%%%%%%%%%%%%%%%%%%%%%%%%%%%%%%%%%%%
%%%%%%%%PAGEBREAK%%%%%%%PAGEBREAK%%%%%%%%%
%%%%%%%%%%%%%%%%%%%%%%%%%%%%%%%%%%%%%%%%%%
%%%%%%%%%%%%%%%%%%%%%%%%%%%%%%%%%%%%%%%%%%
%%%%%%%%%%%%%%%%%%%%%%%%%%%%%%%%%%%%%%%%%%
%%%%%%%%%%%%%%%%%%%%%%%%%%%%%%%%%%%%%%%%%%
%%%%%%%%PAGEBREAK%%%%%%%PAGEBREAK%%%%%%%%%
%%%%%%%%%%%%%%%%%%%%%%%%%%%%%%%%%%%%%%%%%%
%%%%%%%%%%%%%%%%PAGEBREAK%%%%%%%%%%%%%%%%%
%%%%%%%%%%%%%%%%%%%%%%%%%%%%%%%%%%%%%%%%%%
%%%%%%%%PAGEBREAK%%%%%%%PAGEBREAK%%%%%%%%%
%%%%%%%%%%%%%%%%%%%%%%%%%%%%%%%%%%%%%%%%%%
%%%%%%%%%%%%%%%%%%%%%%%%%%%%%%%%%%%%%%%%%%
%%%%%%%%%%%%%%%%%%%%%%%%%%%%%%%%%%%%%%%%%%
%%%%%%%%%%%%%%%%%%%%%%%%%%%%%%%%%%%%%%%%%%
%%%%%%%%PAGEBREAK%%%%%%%PAGEBREAK%%%%%%%%%
%%%%%%%%%%%%%%%%%%%%%%%%%%%%%%%%%%%%%%%%%%
%%%%%%%%%%%%%%%%PAGEBREAK%%%%%%%%%%%%%%%%%
%%%%%%%%%%%%%%%%%%%%%%%%%%%%%%%%%%%%%%%%%%
%%%%%%%%PAGEBREAK%%%%%%%PAGEBREAK%%%%%%%%%
%%%%%%%%%%%%%%%%%%%%%%%%%%%%%%%%%%%%%%%%%%
%%%%%%%%%%%%%%%%%%%%%%%%%%%%%%%%%%%%%%%%%%
\begin{ekdosis}
  \begin{prose}
    \noindent
%-----------------------------
%aśanapānecchā bhavati/   gudamadhye                                                            \E
%aśanapānechā  bhavati    gudamadhye 'pānāvāyus  tiṣṭhati    sa āṃkucanastaṃbhanaṃ   karoti     \P
%aśanapānechā  bhavati//  gudamadhye apānāvāyor  tiṣṭhatī    sa āṃkucanastaṃbhanaṃ   karotī/    \B
%aśanapānecha  bhavati//  gudamadhye apānāvāyo   tiṣṭhati    sa āṃkucanastaṃbhanaṃ   karotī/    \L
%asitapittecha bhavati/   guḍamadhye apānavāyu   tiṣṭhati    sa ākuṃcanasthaṃbhanaṃ  karoti/ /  \N2
%aśitapiteccha bhavati/   gudamadhye apānavāyus  tiṣṭhati/   sa ākuṃcanaṃ staṃbhanaṃ karoti/    \N1
%aśitapiteccha bhavati//  gudamadhye apānavāyus  tiṣṭhati/   sa ākuṃcanaṃ staṃbhanaṃ karoti//   \D
%asīte pitechā bhavati    gudamadhye apānavāyu   tiṣṭhati    sa ākuṃcanaṃ staṃbhanaṃ karoti     \U1
%aśanapānechā  bhavati//  gudamadhye apānāvāyo   tiṣṭhati//     āṃkucanastabhanaṃ    karoti/    \U2
%-----------------------------
%The wish for eating an drinking exists. At the center of the anus the Apāna-Vitalwind exists. He does contraction and checking. 
%-----------------------------
\app{\lem[wit={E}]{aśanapānecchā}
  \rdg[wit={P,B,L,U2}]{aśanapānechā}
  \rdg[wit={N1,D}]{aśitapiteccha}
  \rdg[wit={N2}]{asitapittecha}
  \rdg[wit={U1}]{asīte pitechā}}
bhavati/
gudamadhye
\app{\lem[type=emendation, resp=egoscr, alt={'pānavāyus}]{'pānavāyu\skp{s-ti}}
  \rdg[wit={N1,D}]{\korr apānavāyus}
  \rdg[wit={B}]{apānāvāyor}
  \rdg[wit={L,U2}]{apānāvāyo}
  \rdg[wit={N2,U1}]{apānavāyu}
  \rdg[wit={E}]{\om}
}\app{\lem[wit={ceteri}, alt={tiṣṭhati}]{\skm{s-ti}ṣṭhati}
  \rdg[wit={B}]{tiṣṭhatī}
  \rdg[wit={E}]{\om}}/
\app{\lem[wit={ceteri}]{sa}
  \rdg[wit={E,U2}]{\om}}
\app{\lem[wit={N1,D,U1}]{ākuṃcanaṃ staṃbhanaṃ}
  \rdg[wit={P,B,L,U2}]{āṃkucanastaṃbhanaṃ}
  \rdg[wit={E}]{\om}}
\app{\lem[wit={ceteri}]{karoti}
  \rdg[wit={B}]{karotī}
  \rdg[wit={E}]{\om}}/ 
%-----------------------------
%              samāno vāyur vartate/ sapta samagrā nāḍīḥ śoṣayati/  \E
%  nābhīmadhye samāno varttate       sa samagrā    nāḍīḥ  śoṣayati    \P
%  nābhīmadhye smānā  vartate/       sa samagrā    nāḍī    śoṣayati//  \B
%  nābhīmadhye samānā vartate        sa samagrā    nāḍī    śoṣayatī//  \L
%  nābhimadhye samāno varttate/      sa samāgraṃ   nādhyaṃ śoṣayati/   \N2
%  nābhimadhye samāno varttate/      sa samagraṃ   nādhyaṃ śoṣayati//  \N1
%  nābhimadhye samāno varttate//     sa samagraṃ   nādhyaṃ śoṣayati//  \D
%  nābhimadhye samāno varttate       sa samagrāṃ   nāḍīṃ śoṣayati      \U1
%  nābhipadmamadhye samāno vartate// sa samagrā    nāḍī śoṣayati        \U2
%-----------------------------
%At the center of the navel the Samāna[-vitalwind] exists. He causes to dry up all the channels.
%-----------------------------
\app{\lem[wit={ceteri}]{nābhimadhye}
  \rdg[wit={U2}]{nābhipadmamadhye}
  \rdg[wit={E}]{\om}}
\app{\lem[wit={ceteri}]{samāno}
  \rdg[wit={E}]{samāno vāyur}
  \rdg[wit={B}]{smānā}}
  vartate/
  \app{\lem[wit={ceteri}]{sa}
    \rdg[wit={E}]{sapta}}
  \app{\lem[wit={E,P,B,L,U2}]{samagrā}
    \rdg[wit={N1,N2,D,U1}]{samāgraṃ}}
  \app{\lem[wit={E,P}]{nāḍīḥ}
    \rdg[wit={B,L,U2}]{nāḍī}
    \rdg[wit={U1}]{nāḍīṃ}
    \rdg[wit={N1,N2,D}]{nādhyaṃ}}
  \app{\lem[wit={ceteri}]{śoṣayati}
    \rdg[wit={L}]{śoṣayatī}}/
%-----------------------------
%  nābhīmadhye samāno varttate       sa samagrā    nāḍīḥ  śoṣayati    \P
%  nābhīmadhye smānā  vartate/       sa samagrā    nāḍī    śoṣayati//  \B
%  nābhīmadhye samānā vartate        sa samagrā    nāḍī    śoṣayatī//  \L
%  nābhimadhye samāno varttate/      sa samāgraṃ   nādhyaṃ śoṣayati/   \N2
%  nābhimadhye samāno varttate/      sa samagraṃ   nādhyaṃ śoṣayati//  \N1
%  nābhimadhye samāno varttate//     sa samagraṃ   nādhyaṃ śoṣayati//  \D
%  nābhimadhye samāno varttate       sa samagrāṃ   nāḍīṃ śoṣayati      \U1
%  nābhipadmamadhye samāno vartate// sa samagrā    nāḍī śoṣayati        \U2
%-----------------------------
%At the center of the navel the Samāna[-vitalwind] exists. He causes to dry up all the channels.
%-----------------------------
\app{\lem[wit={ceteri}]{nābhimadhye}
  \rdg[wit={U2}]{nābhipadmamadhye}
  \rdg[wit={E}]{\om}}
\app{\lem[wit={ceteri}]{samāno}
  \rdg[wit={E}]{samāno vāyur}
  \rdg[wit={B}]{smānā}}
  vartate/
  \app{\lem[wit={ceteri}]{sa}
    \rdg[wit={E}]{sapta}}
  \app{\lem[wit={E,P,B,L,U2}]{samagrā}
    \rdg[wit={N1,N2,D,U1}]{samāgraṃ}}
  \app{\lem[wit={E,P}]{nāḍīḥ}
    \rdg[wit={B,L,U2}]{nāḍī}
    \rdg[wit={U1}]{nāḍīṃ}
    \rdg[wit={N1,N2,D}]{śoṣayati}
    \rdg[wit={B}]{śoṣayatī}}/
%-----------------------------
%tathā nāḍīśoṣaṇāt                      rucim  utpādayati/  vahniṃ dīpayati/ \E
%tathā nāḍīḥ pośayati                   rucim  utpādayati   vahniṃ dīpayatī \P
%tathā       pośayatī/ tathā poṣayatī// rucir  utpādayatī   vahnī  dīpayatī/ \B
%tathā       pośayatī                   rucim  utpādayatī   vahnī  dīpayatī... \L
%tathā nāḍīṃ pośayati/                  kvacit-utpādayati/  āgniṃ  dīpayati \N1
%tathā nāḍīṃ pośayati//                 kvacit-utpādayati// āgniṃ  dīpayati \D %%%p. 11 recto
%tathā nāḍīṃ pośayati/                  kvacit-utpādayati/  āgniṃ  dīpayati \N2
%tathā nāḍīṃ pośa iti                   rucim  utpādayati    agnīṃ  dīpayati \U1
%            ṣoṣayati                   rucim  utpādayati//  vahniṃ dīpayati// \U2
%-----------------------------
%In this way the channels are caused to thrive, beauty is caused to be generated and the fire is caused to light up.  
%-----------------------------
\app{\lem[wit={ceteri}]{tathā}
  \rdg[wit={U2}]{\om}}
\app{\lem[wit={P}]{nāḍīḥ}
  \rdg[wit={E}]{nāḍī}
  \rdg[wit={N1,N2,D,U1}]{nāḍīṃ}
  \rdg[wit={B,L,U2}]{\om}}
\app{\lem[type=emendation, resp=egoscr]{poṣayati}
  \rdg[wit={P,N1,N2,D,U1}]{\korr pośayati}
  \rdg[wit={B}]{pośayatī | tathā poṣayatī}
  \rdg[wit={L}]{pośayatī}
  \rdg[wit={U1}]{pośa iti}
  \rdg[wit={U2}]{ṣoṣayati}
  \rdg[wit={E}]{°śoṣaṇāt}}/
\app{\lem[wit={ceteri}, alt={rucim}]{ruci\skp{m-u}}
  \rdg[wit={B}]{rucir}
  \rdg[wit={N1,N2,D}]{kvacit}}
\app{\lem[wit={ceteri}, alt={utpādayati}]{\skm{m-u}tpādayati}
  \rdg[wit={P}]{utpādayatī}}/
\app{\lem[type=emendation, resp=egoscr]{agniṃ}
  \rdg[wit={N1,N2,D}]{\korr āgniṃ}
  \rdg[wit={U1}]{agnīṃ}
  \rdg[wit={E,P,U2}]{vahniṃ}
  \rdg[wit={B,L}]{vahnī}}
\app{\lem[wit={ceteri}]{dīpayati}
  \rdg[wit={P,B,L}]{dīpayatī}}/ 
%-----------------------------
%tālumadhye udāno vāyus-tiṣṭhati/   sa vāyuḥ ratnaṃ līlati/    pānīyaṃ pibati/  nāgavāyuḥ   sarva--śarīre varttate/  tasmād-vāyoḥ śarīraṃ cālayati/ śokam āpnoti// vivilaḥ        \E
%tālumadhye udāno vāyus-tiṣṭhati    sa vāyu  ratnaṃ gilati     pānīyaṃ pībati   nāgavāyuḥ   sakale śarīre varttate   tasmād-vāyo śarīraṃ calayati   śopham āpnoti  vikṛtaḥ        \P %%%7649.jpg
%tālumadhye udānavāyus-tiṣṭhati/    sa vāyur annaṃ  galayatī/  pānīyaṃ pibatī/  nāgavāyuḥ   sakala-śarīre varttate   tasmād-vāyoḥ// śarīre cālatī/  śokam āpnoti   vi??kru??taḥ// \B DSCN7163.JPG Z.11
%tālumadhye udānavāyus tiṣṭhati//   sa vāyur annaṃ  galayati// pānīyaṃ pibatī// nāgavāyu----sakala-śarīre vartate    tasmād vāyoḥ// śarīre cālayatī śokam āpnoti   vikutaḥ...     \L
%tālumadhye udānavāyus-tiṣṭhati/    sa vāyuḥ ratnaṃ śilati/    pānīyaṃ pibati/  nāgavāyuḥ   sakale śarīre varttate// tasmād-vāyoḥ śarīraṃ calati/                                 \N1
%tālumadhye udāno vāyus-tiṣṭhati//  sa vāyur annaṃ  gilati/    pānīyaṃ pibati   nānāgavāyuḥ sakale śarīre varttate// tasmād-vāyoḥ śarīraṃ calati//                                \D
%tālumadhye udānāni vāyus-tiṣṭhati/ sa vāyur-annaṃ  gīlati/    pānīyaṃ pibati/  nāgavāyuḥ   sakale śarīre varttate// tasmād-vāyoḥ śarīraṃ calati/                                 \N2
%tālumadhye udānavāyus-tiṣṭhati     sa vāyur-annaṃ  gilati     pānīyaṃ pibati   nāgavāyu    sakale śarīre varttate   tasmād-vāyoḥ śarīraṃ calati                                  \U1
%tālumadhye udāno vāyus-tiṣṭhati//   sa vāyur annaṃ  gilati//  pānīyaṃ pibati// nāgavāyuḥ   sakale śarīre varttate// tasmād-vāyoḥ śarīraṃ calayati śokam āpnoti vikṛtaḥ//         \U2
%-----------------------------
%Within the throat the Udāna-vitalwind is situated. This wind swallows food, [and] it drinks water. The Nāga-vitalwind exists in the entire body. Through the vitalwind the body is caused to move. 
%em. nāgavāyu = vyānavāyuḥ ....
%-----------------------------
tālumadhye
\app{\lem[wit={B,L,N1,U1}, alt={udānavāyus}]{udānavāyu\skp{s-ti}}
  \rdg[wit={E,P,D,U2}]{udāno vāyus}
  \rdg[wit={N2}]{udānāni vāyus}
}\skm{s-ti}ṣṭhati/
sa \app{\lem[wit={ceteri}, alt={vāyur}]{vāyu\skp{r-a}}
  \rdg[wit={E}]{vāyuḥ}
  \rdg[wit={P}]{vāyu}
}\app{\lem[wit={ceteri}, alt={annaṃ}]{skm{r-a}nnaṃ}
  \rdg[wit={E,P,N1}]{ratnaṃ}}
\app{\lem[wit={ceteri}]{gilati}
  \rdg[wit={E}]{līlati}
  \rdg[wit={B}]{galayatī}
  \rdg[wit={L}]{galayati}
  \rdg[wit={N1}]{śilati}}/
pānīyaṃ \app{\lem[wit={ceteri}]{pibati}
  \rdg[wit={P}]{pībati}
  \rdg[wit={B,L}]{pibatī}}/
\app{\lem[wit={ceteri}]{nāgavāyuḥ}
  \rdg[wit={L}]{nāgavāyu°}
  \rdg[wit={D}]{nānāgavāyuḥ}}
\app{\lem[wit={ceteri}]{sakale}
  \rdg[wit={B,L}]{sakala°}
  \rdg[wit={E}]{sarva°}}
śarīre varttate/ 
%\note[type=philcomm, labelb=s35.z3a, lem={nāgavāyu}]{Only nine of the promised ten vitalwinds are described here. The missing vitalwind is \textit{vyānavāyu}. The description of \textit{nāgavāyu} matches rather the \textit{vyānavāyu}. Witnesses E, P, B, L and U2 preserve a nonsensical fragment after the description of \textit{nāgavāyu}: śokam āpnoti vikṛtaḥ. Possibly the description of \textit{vyānavāyu} was lost due to an eyeskip of a scribe.}
tasmā\skp{d-vā}\app{\lem[wit={ceteri},alt={vāyoḥ}]{skm{d-vā}yoḥ}
  \rdg[wit={P}]{vāyo}}
\app{\lem[wit={ceteri}]{śarīraṃ}
  \rdg[wit={B,L}]{śarīre}}
\app{\lem[type=emendation, resp=egoscr]{calayati}
  \rdg[wit={E}]{\korr cālayati| śokam āpnoti || vivilaḥ}
  \rdg[wit={P}]{calayati śopham āpnoti vikṛtaḥ}
  \rdg[wit={B}]{cālatī | śokam āpnoti vikrutaḥ ||}
  \rdg[wit={L}]{cālayatī śokam āpnoti vikutaḥ}
  \rdg[wit={U2}]{calayati śokam āpnoti vikṛtaḥ ||}
  \rdg[wit={ceteri}]{calati}}/
%-----------------------------
%kūrmavāyur netramadhye tiṣṭhati/ nimeṣonmeṣaṃ karoti/ \E
%kūrmavāyur netramadhye           nimeṣonmeṣaṃ karoti \P
%kūrmavāyoḥ netramadhye           nimeṣonmeṣaṃ karotī/ \B
%kūrmavāyoḥ netramadhye           nimiṣonmeṣaṃ karotī... \L
%kūrmo vāyunetramadhye tiṣṭhati/  unmeṣaṃ nimeṣaṃ karoti/ \N1
%kūrmo vāyunetramadhye tiṣṭhati/  unmeṣaṃ nimeṣaṃ ca karoti// \D
%kūrmo vāyunetramadhye tiṣṭhati/  unmeṣaṃ nimeṣaṃ karoti/ \N2
%\om                                                     \U1
%kūrmavāyur netramadhye           nimiṣonmeṣaṃ karoti//            \U2
%-----------------------------
%The Kūrma-vitalwind exists within the eyes. It causes [the] opening and closing [of the eyes]. 
%-----------------------------
\app{\lem[wit={E,P,U2}, alt={kūrmavāyur}]{kūrmavāyu\skp{r-ne}}
  \rdg[wit={B,L}]{kūrmavāyoḥ}
  \rdg[wit={N1,N2,D}]{kūrmo vāyu}
  \rdg[wit={U1}]{\om}}
\skm{r-ne}tramadhye
\app{\lem[wit={E,N1,N2,D}]{tiṣṭhati}
  \rdg[wit={ceteri}]{\om}}/  
\app{\lem[wit={E,P,B,U2}]{nimeṣonmeṣaṃ}
  \rdg[wit={N1,N2}]{unmeṣaṃ nimeṣaṃ}
  \rdg[wit={D}]{unmeṣaṃ nimeṣaṃ ca}
  \rdg[wit={U1}]{\om}}
\app{\lem[wit={ceteri}]{karoti}
  \rdg[wit={B,L}]{karotī}
  \rdg[wit={U1}]{\om}}/
%-----------------------------
%kṛkalakartāvāyur  udgāraṃ karoti      \E
%kṛkalavāyur       udhāraṃ karoti      \P
%kṛkalavāyur       udhāraṃ karotī      \B
%kṛkalavāyur       uhāraṃ karotī        \L
%kṛkalavāyor       ūdgāro bhavati//    \N1
%kṛkalavāyor-------ūdgāto bhavati/      \D
%kṛkaravāyor-------ūdgāro bhavati/      \N2
%                                       \U1
%puṣkaravāyur      udgāraṃ karoti//    \U2
%-----------------------------
%From the Kṛkala-vitalwind gagging arises. 
%-----------------------------
\app{\lem[wit={N1,N2,D},alt={kṛkalavāyor}]{kṛkalavāyo\skp{r-u}}
  \rdg[wit={P,B,L}]{kṛkalavāyur}
  \rdg[wit={E}]{kṛkalakartāvāyur}
  \rdg[wit={U2}]{puṣkaravāyur}
  \rdg[wit={U1}]{\om}
}\app{\lem[type=emendation, resp=egoscr, alt={udgāro}]{\skm{r-u}dgāro}
  \rdg[wit={E,U2}]{udgāraṃ}
  \rdg[wit={P,B}]{udhāraṃ}
  \rdg[wit={L}]{uhāraṃ}
  \rdg[wit={N1,N2}]{ūdgāro}
  \rdg[wit={D}]{ūdgāto}
  \rdg[wit={U1}]{\om}}
\app{\lem[wit={N1,N2,D}]{bhavati}
  \rdg[wit={E,P,U2}]{karoti}
  \rdg[wit={B,L}]{karotī}
  \rdg[wit={U1}]{\om}}/
%-----------------------------
% devadattavāyoḥ  jṛmbhaṇaṃ bhavati/ dhanaṃjayavāyoḥ śabda utpadyate// \E
% devadattavāyor  jumbhā bhavati     dhanaṃjayavāyo  śabdāḥ utpadyete  \P
% devadattavāyor  jumbhā bhavaṃtī    dhanaṃjayavāyoḥ śabda utpadyate// \B
% devadattavāyor  jṛṃbhā bhavatī     dhanaṃjayavāyoḥ śabdaḥ utpadyate// \L
% devadattavāyor  jṛṃbha utpadyate// dhanaṃjayavāyo  śabda utpadyate// \N1
% devadattavāyor  jṛṃbha utpadyate// dhanaṃjayavāyo  śabda utpadyate// \D
% devadattavāyo   jṛṃbhotpadyate/    dhanaṃjayavāyo  śabdotpadyate// \N2
% devadattavāyor  jaṃbhā utpadyate   dhanaṃjayavāyoḥ sabta utpadyate \U1
% devadattavāyo   jṛṃbhā bhavati//   dhanaṃjayavāyoḥ śabda utpadyate// \U2
%-----------------------------
%From the Devadatta-vitalwind jawning arises. From the Dhanaṃjaya-vitalwind speech arises. 
%-----------------------------
\app{\lem[wit={ceteri}, alt={devadattavāyor}]{devadattavāyo\skp{r-jṛ}}
  \rdg[wit={E}]{devadattavāyoḥ}
  \rdg[wit={N2,U2}]{devadattavāyo}
}\app{\lem[wit={N1,D,U2},alt={jṛmbha}]{\skm{r-jṛ}mbha}
  \rdg[wit={E}]{jṛmbhaṇaṃ}
  \rdg[wit={P,B}]{jumbhā}
  \rdg[wit={L}]{jṛṃbhā}
  \rdg[wit={N2}]{jṛṃbho°}
  \rdg[wit={U1}]{jaṃbhā}}
\app{\lem[wit={N1,D,U2,U1}]{utpadyate}
  \rdg[wit={E,P,U2}]{bhavati}
  \rdg[wit={B}]{bhavaṃtī}
  \rdg[wit={L}]{bhavatī}}/
\app{\lem[wit={ceteri}]{dhanaṃjayavāyoḥ}
  \rdg[wit={P,N1,N2,D}]{dhanaṃjayavāyo}}
\app{\lem[wit={ceteri}]{śabda}
  \rdg[wit={P}]{śabdāḥ}
  \rdg[wit={L}]{śabdaḥ}
  \rdg[wit={N2}]{śabdo°}
  \rdg[wit={U1}]{sabta}}
utpadyate\dd{}
\end{prose}
\end{ekdosis}
\ekdpb*{}
%%%%%%%%%%%%%%%%%%%%%%%%%%%%%%%%%%%%%%%%%%
%%%%%%%%%%%%%%%%%%%%%%%%%%%%%%%%%%%%%%%%%%
%%%%%%%%PAGEBREAK%%%%%%%PAGEBREAK%%%%%%%%%
%%%%%%%%%%%%%%%%%%%%%%%%%%%%%%%%%%%%%%%%%%
%%%%%%%%%%%%%%%%PAGEBREAK%%%%%%%%%%%%%%%%%
%%%%%%%%%%%%%%%%%%%%%%%%%%%%%%%%%%%%%%%%%%
%%%%%%%%PAGEBREAK%%%%%%%PAGEBREAK%%%%%%%%%
%%%%%%%%%%%%%%%%%%%%%%%%%%%%%%%%%%%%%%%%%%
%%%%%%%%%%%%%%%%%%%%%%%%%%%%%%%%%%%%%%%%%%
%%%%%%%%%%%%%%%%%%%%%%%%%%%%%%%%%%%%%%%%%%
%%%%%%%%%%%%%%%%%%%%%%%%%%%%%%%%%%%%%%%%%%
%%%%%%%%PAGEBREAK%%%%%%%PAGEBREAK%%%%%%%%%
%%%%%%%%%%%%%%%%%%%%%%%%%%%%%%%%%%%%%%%%%%
%%%%%%%%%%%%%%%%PAGEBREAK%%%%%%%%%%%%%%%%%
%%%%%%%%%%%%%%%%%%%%%%%%%%%%%%%%%%%%%%%%%%
%%%%%%%%PAGEBREAK%%%%%%%PAGEBREAK%%%%%%%%%
%%%%%%%%%%%%%%%%%%%%%%%%%%%%%%%%%%%%%%%%%%
%%%%%%%%%%%%%%%%%%%%%%%%%%%%%%%%%%%%%%%%%%
%%%%%%%%%%%%%%%%%%%%%%%%%%%%%%%%%%%%%%%%%%
%%%%%%%%%%%%%%%%%%%%%%%%%%%%%%%%%%%%%%%%%%
%%%%%%%%PAGEBREAK%%%%%%%PAGEBREAK%%%%%%%%%
%%%%%%%%%%%%%%%%%%%%%%%%%%%%%%%%%%%%%%%%%%
%%%%%%%%%%%%%%%%PAGEBREAK%%%%%%%%%%%%%%%%%
%%%%%%%%%%%%%%%%%%%%%%%%%%%%%%%%%%%%%%%%%%
%%%%%%%%PAGEBREAK%%%%%%%PAGEBREAK%%%%%%%%%
%%%%%%%%%%%%%%%%%%%%%%%%%%%%%%%%%%%%%%%%%%
%%%%%%%%%%%%%%%%%%%%%%%%%%%%%%%%%%%%%%%%%%
\begin{ekdosis}
  \ekddiv{type=ed}
 \centerline{\textrm{\small{[Madhyalakṣya]}}}
 \bigskip
 \begin{prose}
   \noindent
%----------------------------
%\om                               \E
%idānī  madhyalakṣaṃ   kathyate      \P
%idānīṃ madhyalakṣaṇaṃ kathyate//  \B DSCN7164 Z.1
%idānīṃ madhye lakṣaṃ  kathyate//   \L
%idānīṃ madhyalakṣyaṃ  kathyate//   \N1
%idānīṃ madhyalakṣyaṃ  kathyate//   \D
%idānīṃ madhyalakṣaṇaṃ kathyate//  \N2
%idānīṃ madhyalakṣyaṃ  kathyate     \U1
%idānīṃ madhye lakṣyaṃ kathyate//  \U2
%-----------------------------
%Now the central fixation is taught. 
%-----------------------------
\note[type=source, labelb=180, lem={madhyalakṣyaṃ}]{Ysv (PT): idānīṃ madhyalakṣan tu kathyate siddhikārakam | śvetaṃ raktaṃ tathā pītaṃ dhūmrākāran tu nīlabham |}
\app{\lem[wit={ceteri}]{idānīṃ}
  \rdg[wit={P}]{idānī}
  \rdg[wit={E}]{\om}}
\app{\lem[wit={N1,D,U1}]{madhyalakṣyaṃ}
  \rdg[wit={B,N2}]{madhyalakṣaṇaṃ}
  \rdg[wit={P}]{madhyalakṣaṃ}
  \rdg[wit={L}]{madhye lakṣaṃ}
  \rdg[wit={U2}]{madhye lakṣyaṃ}
  \rdg[wit={E}]{\om}}
kathyate/
%-----------------------------SSP. S41!!! almost identical! 
%              aṃtha ca pītavarṇaṃ   raktavarṇaṃ vā dhūmrākāraṃ yan  nīlavarṇaṃ vā   agniśikhāsadṛśaṃ vidyutsamānaṃ   sūryamaṇḍalasadṛśaṃ     arddhacandrasadṛśaṃ jvalad  ākāśasamākāraṃ  \E
%śvetavarṇaṃ   atha     pītavarṇaṃ   raktaṃ vā      dhūmrākāraṃ yan  nīlavarṇaṃ vā   'gniśikhāsadṛśaṃ vidyutsamānaṃ   sūryamaṇdalasadṛśaṃ     arddhacaṃdrasadṛśaṃ jvalad  ākāśasamākāraṃ  \P
%śvetavaraṃ    atha     pītavarṇaṃ// rakta  vā      dhūmrākāraṃ yan  nīlavarṇaṃ vā// agniśikhāsadṛśaṃ vidyutsamānaṃ   sūryamaṇdalasadṛśaṃ/    ūrdhvacaṃdrasadṛśaṃ jvalad  ākāśasamākāraṃ// \B
%śvetavarṇaṃ   atha     pītavarṇaṃ   raktaṃ vā      dhūmrākāraṃ yan  nīlavarṇaṃ vā// agniśikhāsadṛśaṃ vidyutsamāne    sūryamaṇdalasadṛśaṃ//   ardhacaṃdrasadṛśaṃ  jvalad  ākāśasamākāra   \L
%śvetavarṇā/   atha vā  pītavarṇaṃ   raktaṃ vā      dhūmāra     va   nīlavarṇaṃ vā   agniśikhāsadṛśaṃ vidyutsamānaṃ   sūryamaṇdalaṃ sadṛśaṃ/  ūrdhvacaṃdrasadṛśaṃ jvalad  ākāśasamānakāraṃ//  \N1
%śvetavarṇaṃ// atha vā  pītavarṇaṃ   raktaṃ vā      dhūmākāro   vā   nīlavarṇaṃ vā   agniśikhāsadṛśaṃ vidyutsamānaṃ// sūryamaṇdalaṃ sadṛśaṃ// ūrdhvacaṃdrasadṛśaṃ jvalad  ākāśasamānakāraṃ//  \D
%śvetavarṇā    atha vā  pītavarṇa    raktavarṇa     dhūmravarṇa      nīlavarṇaṃ vā   agniśikhāsadṛśaṃ vidyutsamānaṃ   sūryamaṇdalasadṛśaṃ     ūrdhvacaṃdrasadṛśaṃ jvalad  ākāśasamānakāraṃ//  \N2
%svetavarṇaṃ   atha vā  pītavarṇaṃ   raktaṃ vā      dhūmrākāra  van  nīlavarṇaṃ vā   agniśikhāsadṛśaṃ vidyutsamānaṃ   sūryamaṇdalasadṛśaṃ     ārdhacaṃdrasadṛśaṃ  jalad---ā----samānākāraṃ \U1
%svatavarṇaṃ   atha vā  pītavarṇaṃ// raktaṃ vā      dhūmrākāraṃ yan  nīlavarṇaṃ vā   agniśikhāsadṛśaṃ vidyutsamānaṃ   sūryamaṇdalasadṛśaṃ     arddhacaṃdrasadṛśaṃ jvalad--ākāraṃ samākāraṃ \U2
%-----------------------------
%White-colored, or yellow-colored or red-coloured or smoke-coloured or blue-coloured, like the flame of fire, equal to a lightning, like the orb of the sun, like a half-moon, appearing like flaming space, ...  
%-----------------------------
\note[type=source, labelb=181, lem={agniśikhāsadṛśaṃ}]{Ysv (PT): agnijvālāsamānābhā vidyutpuñjasamaprabhā | ādityamaṇḍalākāramathavā candramaṇḍalam |}
\app{\lem[wit={ceteri}, alt={°śveta}]{śveta}
  \rdg[wit={U1}]{sveta°}
  \rdg[wit={U2}]{svata°}
  \rdg[wit={E}]{\om}
}śveta\app{\lem[wit={P,L,U1,U2}, alt={°varṇaṃ}]{varṇaṃ}
  \rdg[wit={P}]{°varaṃ}
  \rdg[wit={N1}]{°varṇā |}
  \rdg[wit={D}]{°varṇaṃ ||}
  \rdg[wit={E}]{\om}}
\app{\lem[wit={ceteri}]{atha}
  \rdg[wit={E}]{aṃtha}}
\app{\lem[wit={ceteri}]{vā}
  \rdg[wit={E}]{ca}
  \rdg[wit={P,B,L}]{\om}}
pīta\app{\lem[wit={ceteri}, alt={°varṇaṃ}]{varṇaṃ}
  \rdg[wit={B,U2}]{°varṇaṃ ||}
  \rdg[wit={N2}]{°varṇa}}
rakta\app{\lem[wit={E}, alt={°varṇaṃ}]{varṇaṃ}
  \rdg[wit={N2}]{°varṇa}
  \rdg[wit={ceteri}]{°ṃ}
  \rdg[wit={B}]{\om}}
\app{\lem[wit={ceteri}]{vā}
  \rdg[wit={N2}]{\om}}
\app{\lem[type=emendation, resp=egoscr]{dhūmravarṇaṃ}
  \rdg[wit={N2}]{\korr dhūmravarṇa}
  \rdg[wit={D}]{dhūmākāro}
  \rdg[wit={N1}]{dhūmāra}
  \rdg[wit={U1}]{dhūmrākāra}
  \rdg[wit={ceteri}]{dhūmrākāraṃ}}
\note[type=philcomm, labelb=182, lem={dhūmra°}]{Given the repetetive mentioning of colours compounded with °\textit{varṇaṃ} before and after the mentioning of \textit{dhūmra}°, and previous usage in the same compound it is highly likely that \textit{dhūmravarṇaṃ} was the original reading.}
\app{\lem[wit={D}]{vā}
  \rdg[wit={N1}]{va}
  \rdg[wit={U1}]{van}
  \rdg[wit={ceteri}]{yan}
  \rdg[wit={N2}]{\om}}
nīlavarṇaṃ
\app{\lem[wit={ceteri}]{vā}
  \rdg[wit={B,L}]{vā ||}}
\app{\lem[wit={P}, alt={°gni}]{'gni}
  \rdg[wit={ceteri}]{agni°}
}śikhāsadṛśaṃ vidyut\app{\lem[wit={ceteri},alt={°samānaṃ}]{samānaṃ}
  \rdg[wit={L}]{°samāne}
  \rdg[wit={D}]{°samānaṃ ||}}
sūryamaṇdala\app{\lem[wit={ceteri}, alt={°sadṛśaṃ}]{sadṛśaṃ}
  \rdg[wit={N1,D}]{°ṃ sadṛśaṃ}}
\app{\lem[wit={ceteri},alt={ardha°}]{ardha}
  \rdg[wit={U1}]{ārdha°}
  \rdg[wit={B,N1,N2,D}]{ūrdhva°}
}candrasadṛśaṃ
\app{\lem[wit={ceteri}, alt={jvalad°}]{jvala\skp{d-ā}}
  \rdg[wit={U1}]{jalad}
}\app{\lem[wit={ceteri}, alt={°ākāśa°}]{\skm{d-ā}kāśa}
  \rdg[wit={U1}]{°ā°}
  \rdg[wit={U2}]{°ākāraṃ}
}\app{\lem[wit={ceteri}, alt={°samākāraṃ}]{samākāraṃ}
  \rdg[wit={N1,N2,D,U1}]{°samānakāraṃ}
  \rdg[wit={U2}]{samakāraṃ}
  \rdg[wit={L}]{°samākāra}}/
%-----------------------------
%svaśarīraparimitaṃ      tejomanomadhye tathyaṃ kartavyam// \E
%svaśarīraparimitaṃ      tejomanomadhye lakṣyaṃ karttavyaṃ\P
%svaśarīraparimitaṃ      tejomanomadhye lakṣaṃ kartavyaṃ//  \B
%svaśarīraparimitaṃ      tejomanomadhye lakṣaṃ kartavyaṃ//  \L
%svaśarīraparimitaṃ      tejomanomadhye lakṣyaṃ karttavyaṃ//  \N1
%svaśarīraparimitaṃ      tejomanomadhye lakṣyaṃ karttavyaṃ// \D
%svaśarīraparimitaṃ      tejomanomadhye lakṣaṇaṃ karttavyaṃ//  \N2
%svaśarīraparimanomittaṃ tejomadhye     lakṣyaṃ karttavyaṃ  \U1
%svaśarīraparimitaṃ      tejomanomadhye lakṣaṃ kartavyaṃ//  \U2
%-----------------------------
%measured according to ones own body, the fixation shall be directed onto the center of the glowing mind.  
%-----------------------------
\note[type=source, labelb=183, lem={tejomanomadhye}]{Ysv (PT): jvaladākāśatulyaṃvā bhāvayed rūpamātmanaḥ | etaj jyotirmayaṃ dehaṃ manomadhye tu lakṣayet |}
svaśarīrapari\app{\lem[wit={ceteri},alt={°mitaṃ}]{mitaṃ}
  \rdg[wit={U1}]{°manomittaṃ}}
tejo\app{\lem[wit={ceteri},alt={°mano}]{mano}
  \rdg[wit={U1}]{\om}
}madhye
\app{\lem[wit={P,N1,D,U1}]{lakṣyaṃ}
  \rdg[wit={E}]{tathyaṃ}
  \rdg[wit={B,L,U2}]{lakṣaṃ}
  \rdg[wit={N2}]{lakṣaṇaṃ}}
kartavyaṃ/
%-----------------------------
%ekasmin lakṣye    kṛte sati manomadhye sthitasya malasya dāho bhavati/ \E [p.38]%
%etasmil lakṣye    kṛte sati manomadhye sthitasya         dāho bhavati \P
%etasmin lakṣe     kṛte satī manomadhye sthitasya malasya dāho bhavati \B
%etasmil lakṣe     kṛte satī manomadhye sthitasya malasya dāho bhavati... \L
%ekasmin lakṣye    kṛte sati manomadhye sthitasya malasya dāho bhavati/ \N1
%ekasmin lakṣye    kṛte sati manomadhye sthitasya malasya dāho bhavati// \D
%ekasmin lakṣaṇo   kṛte sati manomadhye sthitasya malasya dāho bhavati/ \N2
%etasmin na lakṣye kṛte satī manomadhye sthitasya malasya dāho bhavati \U1 %%%281.jpg
%etasmil lakṣe     kṛte satī manomadhye sthitasya malasya dāho bhavati// \U2
%-----------------------------
%While abiding in this fixation the burning of the impurity in the center of the mind arises. 
%-----------------------------
\note[type=source, labelb=184, lem={malasya}]{Ysv (PT): eteṣāñ ca kṛte lakṣe nānāduḥkhaṃ praṇaśyati | manas astu malo yāti mahānando bhavet tataḥ |}
\app{\lem[wit={P,L,U2},alt={etasmil}]{etasmi\skp{ll-a}}
  \rdg[wit={U1}]{etasmin}
  \rdg[wit={ceteri}]{ekasmin}
}\app{\lem[wit={ceteri},alt={lakṣye}]{\skm{ll-a}kṣye}
  \rdg[wit={B,L,U2}]{lakṣe}
  \rdg[wit={U1}]{na lakṣye}
  \rdg[wit={N2}]{lakṣaṇo}}
kṛte
\app{\lem[wit={ceteri}]{sati}
  \rdg[wit={B,L,U1,U2}]{satī}}
manomadhye sthitasya
\app{\lem[wit={ceteri}]{malasya}
  \rdg[wit={P}]{\om}}
dāho bhavati/ 
%-----------------------------
%manasaḥ   sattvaguṇaprakāśo   bhavati/     puruṣa ānandamayo bhūtvā tiṣṭhati//   \E 
%manasaḥ   sattvaguṇaḥ prakaṭo bhavati      puruṣa ānandamayo bhūtvā tiṣṭhati    \P   %%%7650.jpg
%manasaḥ// sattvaguṇo  prakaṭo  bhavati//   puruṣa ānandamayo bhūtvā tiṣṭhati// \B
%manasaḥ// sattvaguṇaḥ prakaṭo bhavati      puruṣa ānandamayo bhūtvā tiṣṭhati//  \L
%manasaḥ   sattvaguṇe  prakaṭo  bhavati/    puruṣa ānandamayo bhūtvā tiṣṭhati//    \N1
%manaḥ saḥ sattvaguṇo  prakaṭo  bhavati//   puruṣa ānandamayo bhūtvā tiṣṭhati// \D
%manasaḥ   sattvaguṇo  prakaṭo  bhavati/    puruṣa ānandamayo bhūtvā tiṣṭhati//    \N2
%manasaḥ   sattvaguṇo  prakaṭo  bhavati     puruṣa ānandamayo bhūtvā tiṣṭhati       \U1
%manasaḥ   satvaguṇaprakaśo    bhavati//    puruṣa ānandamayo bhūtvā tiṣṭhati//    \U2 %%414.jpg
%-----------------------------
%The Sattva-quality of the mind becomes revealed. After this has happend the person abides supreme bliss. 
%-----------------------------
mana\app{\lem[wit={ceteri},alt={°saḥ}]{saḥ}
  \rdg[wit={B,L}]{°saḥ ||}
  \rdg[wit={D}]{manaḥ saḥ}}
sattva\app{\lem[wit={B,D,N2,U1},alt={°guṇo}]{guṇo}
  \rdg[wit={N1}]{°guṇe}
  \rdg[wit={E,U2}]{°guṇa°}
  \rdg[wit={P,L}]{°guṇaḥ}}
\app{\lem[wit={ceteri}]{prakaṭo}
  \rdg[wit={E,U2}]{°prakāśo}}
bhavati/ puruṣa ānandamayo bhūtvā tiṣṭhati\dd{}
\end{prose}
\end{ekdosis}
%-----------------------------
%-----------------------------
%-----------------------------
\begin{ekdosis}
  \ekddiv{type=ed}
 \bigskip
 \centerline{\textrm{\small{[The Divisions of Space]}}}
 \bigskip
 \begin{prose}
%-----------------------------
%idānīm-ākāśabhedāḥ kathyante/ \E
%idānīm ākaśabhedāḥ kathyaṃte   \P
%idānīṃ ākaśabhedāḥ kathyaṃte/ \B
%idānīṃ ākaśabhedāḥ kathyate/  \L
%idānīṃ ākaśabhedāḥ kathyaṃte/ \N1
%idānīṃ ākaśabhedāḥ kathyaṃte// \D
%idānīṃ ākāśabhedāḥ kathyate/  \N2
%idānīṃ ākāśabhedāḥ kathyaṃte  \U1
%idānīm ākāśabhedāḥ kathyate// \U2
%-----------------------------
%Now the divisions of space are taught. 
%-----------------------------
\note[type=source, labelb=185, lem={ākaśabhedāḥ}]{kathyate tu devyadhunākāśaṃ pañcabhirlakṣaṇaiḥ | ākāśan tu mahākāśaṃ parākāśaṃ parātparam | tattvākāśaṃ sūryakāśamākāśaṃ pañcalakṣaṇam |}
\app{\lem[wit={E,P,U2},alt={idānīm}]{idānī\skp{m-ā}}
  \rdg[wit={ceteri}]{idānīṃ}
}\skm{m-ā}kāśabhedāḥ
\app{\lem[wit={ceteri}]{kathyante}
  \rdg[wit={L,N2,U2 }]{kathyate}}/
%-----------------------------SSP!
%te                          ākāśaḥ paramākāśaḥ mahākāśaḥ tattvākāśaḥ sūryākāśaḥ/    bāhyābhyantare nirmalaṃ nirākāram ākāśa---lakṣyaṃ  karttavyam/ \E
%teṣāṃ lakṣyāni ca kathyaṃte ākāśaḥ parākāśaḥ mahākāśaḥ tatvākāśaḥ sūryakāśaḥ        bāhyābhyaṃtare nirmalaṃ nirākāram ākāśaṃ  lakṣyaṃ  karttavyaṃ  \P
%                            ākāśaḥ paramākāśaḥ// mahākāśa// tattvākāśaḥ sūryākāśa// bāhyābhyaṃtaro nirmalaṃ nirākāram ākāśaṃ  lakṣaṃ   kartavyaṃ// \B
%                            ākāśaḥ paramākāśaḥ// mahākāśaḥ tattvākāśaḥ sūryākāśaḥ   bāhyābhyaṃtare nirmalaṃ nirākāram ākāśaṃ  lakṣaṃ   kartavyaṃ// \L
%teṣāṃ lakṣyāni  kathyate//  ākāśa, parākāśa,mahākāśa,tatvākāśa,sūryakāśa//          bāhyābhyaṃtare nirmalaṃ nirākāraṃ ākāśa---lakṣyaṃ  kartavyaṃ// \N1
%teṣāṃ lakṣyāṇi  kathyaṃte// ākāśa--parākāśamahākāśatatvākāśasūryakāśa               bāhyābhyaṃtare nirmalaṃ nirākāraṃ ākāśa---lakṣyaṃ  karttavyaṃ// \D   %%%p.11 verso
%teṣāṃ lakṣaṇāni kathyate//  ākāśa--parākāśamahākāśatatvākāśasūryakāśaḥ              bāhyābhyaṃtare nirmalaṃ nirākāraṃ ākāśa---lakṣaṇaṃ kartavyaṃ// \N2
%ṣāṃ   lakṣyāṇi  kathyaṃte   ākāśa--parākāśamahākāśatatvākāśasūryakāśa---------------bāhyābhyaṃtare nirmalaṃ nirākāraṃ ākāśa---lakṣyaṃ  karttavyaṃ  \U1
%teṣāṃ lakṣyāni  kathyaṃte// ākāśaḥ parākāśa// mahākāśaḥ// tatvākāśaḥ// sūryakāśaḥ// bāhyābhyaṃtare nirmalaṃ nirākāraṃ mākāśaṃ lakṣyaṃ  karttavyaṃ// \U2
%-----------------------------
%The fixations of them are taught: Space, beyond space, great space, space of reality, the space of the sun. The fixation onto the pure and formless space \textit{akāśa} shall be done internally as well as externally.  
%-----------------------------SSP!
\note[type=source, labelb=186, lem={ākāśaḥ}]{SSP: ākāśaṃ parākāśaṃ mahākāśaṃ tatvākaśaṃ sūryākāśamiti vyomapañcakam | bāhyābhyantare 'tyantaṃ nirmalaṃ nirākāraṃ ākāśaṃ lakṣayet |}
\app{\lem[wit={ceteri}]{teṣāṃ}
  \rdg[wit={E}]{te}
  \rdg[wit={U1}]{ṣaṃ}
  \rdg[wit={B,L}]{\om}}
\app{\lem[wit={ceteri}]{lakṣyāni}
  \rdg[wit={N2}]{lakṣaṇāni}
  \rdg[wit={B,E,L}]{\om}}
\app{\lem[wit={D,U1,U2}]{kathyante}
  \rdg[wit={P}]{ca kathyante}
  \rdg[wit={N1,N2}]{kathyate}
  \rdg[wit={B,E,L}]{\om}}/
\app{\lem[wit={B,E,L,P}]{ākāśaḥ}
  \rdg[wit={D,N1,N2,U1}]{ākāśa°}}\dd{}
\app{\lem[wit={B,E,L}]{paramākāśaḥ}
  \rdg[wit={P,U2}]{parākāśaḥ}
  \rdg[wit={N1}]{parākāśa}
  \rdg[wit={D,N2,U1}]{parākāśa°}}\dd{}
\app{\lem[wit={E,L,P,U2}]{mahākāśaḥ}
  \rdg[wit={B,N1}]{mahākāśa}
  \rdg[wit={ceteri}]{mahākāśa°}}\dd{}
\app{\lem[wit={B,E,L,U2}]{tattvakāśaḥ}
  \rdg[wit={N1}]{tatvakāśa}
  \rdg[wit={ceteri}]{tatvakāśa°}}\dd{}
\app{\lem[wit={B,E,L}]{sūryākāśaḥ}
  \rdg[wit={N2,P,U2}]{sūryakāśaḥ}
  \rdg[wit={N1}]{sūryakāśa}
  \rdg[wit={ceteri}]{sūryakāśa°}}\dd{}
bāhyābhyantare nirmalaṃ nirākāram
\app{\lem[wit={ceteri},alt={ākāśa°}]{ākāśa}
  \rdg[wit={U2}]{mākāśaṃ}
  \rdg[wit={P,B,L}]{ākāśaṃ}
}\app{\lem[wit={ceteri}, alt={°lakṣyaṃ}]{lakṣyaṃ}
  \rdg[wit={B,L}]{lakṣaṃ}
  \rdg[wit={N2}]{°lakṣaṇaṃ}}
kartavya\app{\lem[wit={E}]{kartavyam}
  \rdg[wit={ceteri}]{kartavyaṃ}}\dd{}
\end{prose}
\end{ekdosis}
\ekdpb*{}
%%%%%%%%%%%%%%%%%%%%%%%%%%%%%%%%%%%%%%%%%%
%%%%%%%%%%%%%%%%%%%%%%%%%%%%%%%%%%%%%%%%%%
%%%%%%%%PAGEBREAK%%%%%%%PAGEBREAK%%%%%%%%%
%%%%%%%%%%%%%%%%%%%%%%%%%%%%%%%%%%%%%%%%%%
%%%%%%%%%%%%%%%%PAGEBREAK%%%%%%%%%%%%%%%%%
%%%%%%%%%%%%%%%%%%%%%%%%%%%%%%%%%%%%%%%%%%
%%%%%%%%PAGEBREAK%%%%%%%PAGEBREAK%%%%%%%%%
%%%%%%%%%%%%%%%%%%%%%%%%%%%%%%%%%%%%%%%%%%
%%%%%%%%%%%%%%%%%%%%%%%%%%%%%%%%%%%%%%%%%%
%%%%%%%%%%%%%%%%%%%%%%%%%%%%%%%%%%%%%%%%%%
%%%%%%%%%%%%%%%%%%%%%%%%%%%%%%%%%%%%%%%%%%
%%%%%%%%PAGEBREAK%%%%%%%PAGEBREAK%%%%%%%%%
%%%%%%%%%%%%%%%%%%%%%%%%%%%%%%%%%%%%%%%%%%
%%%%%%%%%%%%%%%%PAGEBREAK%%%%%%%%%%%%%%%%%
%%%%%%%%%%%%%%%%%%%%%%%%%%%%%%%%%%%%%%%%%%
%%%%%%%%PAGEBREAK%%%%%%%PAGEBREAK%%%%%%%%%
%%%%%%%%%%%%%%%%%%%%%%%%%%%%%%%%%%%%%%%%%%
%%%%%%%%%%%%%%%%%%%%%%%%%%%%%%%%%%%%%%%%%%
%%%%%%%%%%%%%%%%%%%%%%%%%%%%%%%%%%%%%%%%%%
%%%%%%%%%%%%%%%%%%%%%%%%%%%%%%%%%%%%%%%%%%
%%%%%%%%PAGEBREAK%%%%%%%PAGEBREAK%%%%%%%%%
%%%%%%%%%%%%%%%%%%%%%%%%%%%%%%%%%%%%%%%%%%
%%%%%%%%%%%%%%%%PAGEBREAK%%%%%%%%%%%%%%%%%
%%%%%%%%%%%%%%%%%%%%%%%%%%%%%%%%%%%%%%%%%%
%%%%%%%%PAGEBREAK%%%%%%%PAGEBREAK%%%%%%%%%
%%%%%%%%%%%%%%%%%%%%%%%%%%%%%%%%%%%%%%%%%%
%%%%%%%%%%%%%%%%%%%%%%%%%%%%%%%%%%%%%%%%%%
\begin{ekdosis}
  \begin{prose}
    \noindent
%-----------------------------
%tataḥ paraṃ bāhyābhyantare  ṣvanandhakārasadṛśaṃ   parākāśaikyaṃ lakṣyaṃ  karttavyam// \E
%tataḥ paraṃ bāhyābhyantarai ghanāṃdhakāraṃ sadṛśa--parākāśasya   lakṣyaṃ  karttavyam \P
%tataḥ paraṃ bāhyābhyaṃtare  ghanāṃghakārasadṛśaḥ   parākāśa------lakṣaṃ   kartavyaṃ// \B
%tataḥ paraṃ bāhyābhyaṃtare        dhakārasadṛśaḥ   parākāśa------lakṣaṃ   kartavyaṃ... \L %%%%%%%%%%%%%%%%%ghana hier = dunkel, schwarz%%%% andhakāra=  finster, dunkel. Finsterniss
%tataḥ paraṃ bāhyābhyantare  ghanāṃdhakārasadṛśa----parākāśasya   lakṣyaṃ  kattavyam// \N1
%tataḥ paraṃ bāhyābhyantare  ghanāṃdhakārasadṛśa----parākāśasya   lakṣyaṃ  kattavyaṃ// \D
%tataḥ paraṃ bāhyābhyantare  ghanāṃdhakārasadṛśa----parākāśasya   lakṣaṇaṃ karttavyam// \N2
%tataḥ paraṃ bāhyābhyantare  ghanāṃdhakārasadṛśa----parākāśasya   lakṣyaṃ  karttavyaṃ \U1
%tataḥ       bāhyābhyantare  ghanāṃdhakārasadṛśaṃ   parākāśasya   lakṣaṃ   karttavyaṃ// \U2
%-----------------------------
%Moreover, the fixation of the beyond-space \textit{parākāśa} which is equal to dense darkness shall be done internally and externally.
%-----------------------------SSP!
\note[type=source, labelb=187, lem={parākāśasya}]{Ysv (PT): savāhyābhyantare nityaṃ nirākāśantu nirmalam | karttavyaṃ lakṣam ākāśaṃ sādhayet sādhanaṃ vinā | ghanāntarālasadṛśaṃ parākāśaṃ tathaiva ca |}
tataḥ
\app{\lem[wit={ceteri}]{paraṃ}
  \rdg[wit={U2}]{\om}}
\note[type=source, labelb=188, lem={parākāśasya}]{SSP: atha vā bāhyābhyantare 'tyantāndhakāranibhaṃ parākāśam avalokayet |}
\note[type=philcomm, labelb=189, lem={ghanāṃdhakāra°}]{Instead of extreme brightness as in the SSP, Rāmacandra conspicuously choose dense darkness to characterize his \textit{parākāśa}-visualization.}
bāhyābhyanta\app{\lem[wit={ceteri},alt={°re}]{re}
  \rdg[wit={P}]{°rai}}
\app{\lem[wit={ceteri},alt={ghanāndha°}]{ghanāndha}
  \rdg[wit={B}]{ghanāṃgha°}
  \rdg[wit={E}]{ṣvanandha°}
  \rdg[wit={L}]{dha°}
}\app{\lem[wit={ceteri},alt={°kāra°}]{kāra}
  \rdg[wit={P}]{°kāraṃ}
}\app{\lem[wit={ceteri},alt={°sadṛśa°}]{sadṛśa}
  \rdg[wit={E,U2}]{sadṛśaṃ}
  \rdg[wit={B,L}]{sadṛśaḥ}
}parākāśa\app{\lem[wit={ceteri},alt={°sya}]{sya}
  \rdg[wit={E}]{°ikyaṃ}
  \rdg[wit={B,L}]{°}}
lakṣ\app{\lem[wit={ceteri},alt={°yaṃ}]{yaṃ}
  \rdg[wit={B,L,U2}]{°aṃ}
  \rdg[wit={N2}]{°aṇaṃ}}
kartavyaṃ/
%-----------------------------
%tataḥ paraṃ pralayakālīna--jvalad-dāvā---nala-pūrṇaṃ  bāhyābhyantare, mahākāśalakṣyaṃ karttavyam/ \E
%tataḥ paraṃ pralayakālīna--jalad--vaḍavā-nala-pūrṇaṃ  bāhyābhyaṃtare  mahākāśaṃ lakṣyaṃ karttavyaṃ \P
%tataḥ paraṃ pralayakālīnaḥ jalad--vaḍavā-nala-pūrṇaṃ  bāhyābhyaṃtare  mahākāśalakṣaṃ kartavyaṃ// \B
%tataḥ paraṃ pralayakālīnaḥ jvalad-vaḍavā-nala-pūrṇaṃ  bāhyābhyaṃtare  mahākāśalakṣaṃ kartavyaṃ// \L
%tataḥ paraṃ pralayakālīna--jvalad-vṛddha-nala-pūrṇa---bāhyābhyaṃtare  mahākāśalakṣyaṃ karttavyaṃ// \N1 ?[S.9 verso letzte Zeile] 
%tataḥ paraṃ pralayakālīna--jvalad-dāvā---nala-pūrṇaṃ  bāhyābhyaṃtare  mahākāśaṃ lakṣaṃ karttavyaṃ// \D
%tataḥ paraṃ pralayakālīna--jvalad-vṛ-----nala-pūrṇa---bāhyābhyaṃtare  mahākāśalakṣaṃ karttavyaṃ// \N2
%tataḥ paraṃ pralayakālīta--jjala--vaḍavā-nala-pūrṇaṃ  bāhyābhyaṃtare  mahākāśaṃ lakṣyaṃ kartavyaṃ \U1
%tataḥ       pralayakālīna--jvalad-vaḍavā-nala-pūrṇa---bāhyābhyaṃtare  ghanāṃ dhakārasadṛśaṃ mahākāśasya lakṣaṃ karttavyaṃ \U2
%-----------------------------
%Moreover, the fixation of the great space (\textit{mahākāśa}) which is the plethora of the burning fire of the time of dissolution shall be done internally and externally. 
%-----------------------------SSP!
\note[type=source, labelb=190, lem={mahākaśa°}]{Ysv (PT): kalpāntāgnisamaṃ jyotir mahākāśaṃ smaret tathā |}
\note[type=testium, labelb=191, lem={mahākāśa°}]{SSP: bāhyābhyantare kālānalasaṃkāśaṃ mahākāśam avalokayet |}
tataḥ
\app{\lem[wit={ceteri}]{paraṃ}
  \rdg[wit={ceteri}]{U2}}
pralayakālī\app{\lem[wit={ceteri},alt={°na}]{na}
  \rdg[wit={B,L}]{°naḥ}
}\app{\lem[wit={ceteri},alt={°jvalad°}]{jvalad}
  \rdg[wit={P,B}]{°jalad°}
  \rdg[wit={U1}]{°jjala°}
}\app{\lem[wit={E,D},alt={°dāvā°}]{dāvā}
    \rdg[wit={B,L,P,U1,U2}]{°vaḍavā°}
    \rdg[wit={N1}]{°vṛddha°}
    \rdg[wit={N2}]{°vṛ°}
}nalapū\app{\lem[wit={ceteri},alt={°rṇaṃ}]{rṇaṃ}
  \rdg[wit={N1,N2,U2}]{°rṇa}}
bāhyābhyantare
\app{\lem[wit={ceteri},alt={mahākāśa°}]{mahākaśa}
  \rdg[wit={P,D,U1}]{mahākāśaṃ}
  \rdg[wit={U2}]{ghanāṃ dhakārasadṛśaṃ mahākāśasya}
}\app{\lem[wit={ceteri}, alt={°lakṣyaṃ}]{lakṣyaṃ}
  \rdg[wit={B,D,L,N2,U2}]{°lakṣaṃ}}
kartavvyaṃ/
%-----------------------------
%\om                                                                                                                         \E
%tataḥ paraṃ bāhyābhyaṃtare koṭidīpānāṃ prakāśaprāptau  yādṛśam aujvalyaṃ bhavati   tādṛśaṃ   tatvākāśaṃ lakṣyaṃ karttavyaṃ  \P
%tataḥ paraṃ bāhyābhyaṃtare koṭidīpānāṃ prakāśaprāpto   yādṛśam aujvalaṃ  bhavatī/  tādṛśaṃ   tatvāśa----lakṣaṃ kartavyaṃ//  \B
%tataḥ paraṃ bāhyābhyaṃtare koṭidīpānāṃ prakāśaprāpto   yādṛśam  ujvalaṃ  bhavatī/  tādṛśaṃ   tatvāśa----lakṣaṃ kartavyaṃ    \L  
%tataḥ paraṃ bāhyābhyaṃtare koṭidīpānāṃ prakāśaprāptau  yādṛśam aujvalyaṃ bhavati/  tādṛśaṃ   tatvākāśaṃ lakṣyaṃ kartavyaṃ// \N1
%tataḥ paraṃ bāhyābhyaṃtare koṭidīpānāṃ prakāśaprāptau  yādṛśam aujvalyaṃ bhavati// tādṛśaṃ   tatvākāśaṃ lakṣaṃ kartavyaṃ//  \D
%tataḥ paraṃ bāhyābhyaṃtare koṭidīpānāṃ prakāśaprāptau  yādṛśam aujvala   bhavati/  tādṛśaṃ   tatvākāśaṃ lakṣaṃ kartavyaṃ//  \N2
%tataḥ paraṃ bāhyābhyaṃtare koṭidīpānāṃ prakāśaprāptau  yādṛśam aujvalaṃ  bhavati   tādṛśaṃ   tatvākāśaṃ lakṣyaṃ kartavyaṃ   \U1
%tataḥ paraṃ bāhyābhyaṃtare koṭidīpānāṃ prakāśaprāptau  yādṛśem aujvalyaṃ bhavati   tādṛśaṃ// tatvākāśaṃ lakṣaṃ karttavyaṃ// \U2
%-----------------------------
%Moreover, for whom internally and externally the brightness of millions of blazing lights arises, he shall execute the fixation [directed onto] the reality-space (\textit{tattvakāśa}).   
%-----------------------------SSP!
tataḥ paraṃ bāhyābhyaṃtare koṭidīpānāṃ
\note[type=philcomm, labelb=192, lem={tataḥ \ldots kartavyaṃ}]{The whole sentence is omitted in E.}
\note[type=testium, labelb=193, lem={tattvākāśaṃ}]{SSP: bāhyābhyantare nijatatvakharūpaṃ tatvākāśam avalokayet |}
\note[type=source, labelb=194, lem={tattvākāśaṃ}]{Ysv (PT): koṭikoṭipradīpābhaṃ tattvākāśaṃ smaret tathā |}
prakāśaprā\app{\lem[wit={ceteri},alt={°ptau}]{ptau}
  \rdg[wit={B,L}]{°pto}}
yādṛśaṃ
\app{\lem[wit={ceteri}]{aujvalyaṃ}
  \rdg[wit={L}]{ujvalaṃ}}
bhava\app{\lem[wit={ceteri},alt={°ti}]{ti}
  \rdg[wit={B,L}]{°tī}}/
tādṛśaṃ
tattvā\app{\lem[wit={ceteri}, alt={°kāśaṃ}]{kāśaṃ}
  \rdg[wit={B,L}]{°śa°}}
\app{\lem[wit={P,N1,U1}]{lakṣyaṃ}
  \rdg[wit={B,D,L,N2,U2}]{lakṣaṃ}}
kartavyaṃ/
%-----------------------------
%tataḥ        bāhyābhyantare  prakāśa-mānayarsūsahitaṃ        sūryākāśaṃ lakṣyaṃ karttavyam/ \E [p.39]
%tataḥ paścād bāhyābhyaṃtare  prakāśa-māgasūryaṃ biṃbasahitaṃ sūryākāśalakṣyaṃ   karttavyaṃ ... \P
%      paccā  bāhyābhyaṃtare  prakāśa-mān sūryabiṃbasahita----sūryakāśalakṣaṃ    kartavyaṃ mataḥ ... \B
%      paccā  bāhyābhyaṃtare  prakāśa-mān sūryabiṃbasahita----sūryakāśalakṣaṃ    kartavyaṃ mataḥ ... \L 
%tataḥ paścāt bāhyābhyaṃtare  prakāśa-mānasūryabiṃbasahitaṃ   sūryakāśaṃ lakṣyaṃ karttavyaṃ// \N1
%tataḥ paścāt bāhyābhyaṃtare  prakāśa-mānasūryabiṃbasahitaṃ   sūryakāśaṃ lakṣyaṃ karttavyaṃ// \D
%tataḥ paścād     ābhyaṃtare  prakāśa-mānasūryabiṃbasahitaṃ   sūryakāśaṃ lakṣaṃ  karttavyaṃ// \N2
%tataḥ paścāt bāhyabhyaṃttare prakāśa-mānasūryabiṃbasāhitaṃ   sūryakāśaṃ lakṣyaṃ karttavyaṃ \U1
%tataḥ paścād bāhyābhyaṃtare  prakāśa-mānasūryabiṃbasāhitaṃ   sūryākāśaṃ lakṣyaṃ karttavyaṃ// \U2
%-----------------------------
%After that the fixation of the sun-space (\textit{sūryakāśa}) which is associated with sundisk's appearance of light shall be done internally and externally.   
%-----------------------------SSP!
\note[type=source, labelb=195, lem={sūryakāśaṃ}]{SSP: atha vā bāhyābhyantare sūryakoṭisadṛśaṃ sūryākāśam avalokayet |}
\note[type=source, labelb=196, lem={sūryakāśaṃ}]{Ysv (PT): sūryākāśaṃ tathā koṭisūryavindusamaṃ smaret | savāhyābhyantare caivamākāśaṃ lakṣayettu yaḥ | śivavadviharedviśve pāpapuṇyavivarjitaḥ | eteṣāñ caiva lakṣeṇa karmadvārā 'ghamāharet}
\app{\lem[wit={ceteri}]{tataḥ}
  \rdg[wit={B,L}]{\om}}
\app{\lem[wit={ceteri}, alt={paścād}]{paścā\skp{d-bā}}
  \rdg[wit={N1,N2,U1}]{paścāt}
  \rdg[wit={B,L}]{paccā}
  \rdg[wit={E}]{\om}}
\app{\lem[wit={ceteri},alt={bāhyā°}]{\skm{d-bā}hyā}
  \rdg[wit={N2}]{ā°}
}bhyaṃtare
prakāśa\app{\lem[wit={ceteri},alt={°māna°}]{māna}
  \rdg[wit={P}]{°māga°}
  \rdg[wit={B,L}]{°mān}
}\app{\lem[wit={ceteri},alt={°sūrya°}]{sūrya}
  \rdg[wit={E}]{°yarsū°}
  \rdg[wit={P}]{°sūryaṃ}
}\app{\lem[wit={ceteri},alt={°bimba°}]{bimba}
  \rdg[wit={E}]{\om}
}\app{\lem[wit={ceteri},alt={°sahitaṃ}]{sahitaṃ}
  \rdg[wit={B,L}]{°sahita°}}
sūryakā\app{\lem[wit={ceteri},alt={°śaṃ}]{śaṃ}
  \rdg[wit={B,L,P}]{°śa°}}
lakṣ\app{\lem[wit={ceteri}, alt={°yaṃ}]{yaṃ}
  \rdg[wit={B,L,N2}]{°aṃ}}
\app{\lem[wit={ceteri}]{kartavyaṃ}
  \rdg[wit={B,L}]{kartavyaṃ mataḥ}}/
%-----------------------------
%eteṣāṃ lakṣyāṇāṃ kāraṇāt   śarīraṃ rogāsaṃsargi    bhavati// \E
%eteṣāṃ lakṣāṇāṃ  karaṇāt   śarīre  rogasaṃsargo na bhavati \P %%%7651.jpg
%eteṣāṃ lakṣaṇaṃ  karaṇāt// śarīre  rogasaṃsargo na bhavatī/ \B
%eteṣāṃ lakṣaṃ    karaṇāt   śarīre  rogasaṃsargo na bhavati... \L
%eteṣāṃ lakṣyaṇāṃ karaṇāt   śarīra--rohasaṃsarge na bhavati/ \N1
%eteṣāṃ lakṣyāṇāṃ karaṇāt   śarīra--rohasaṃsargo na bhavati// \D
%eteṣāṃ lakṣāṇā---kāraṇāc---charīra-rogāsaṃsargo na bhavati// \N2
%eteṣāṃ lakṣyāṇāṃ karaṇāt   śarīra--rogāsaṃsargo na bhavati \U1
%eteṣāṃ lakṣyāṇāṃ karaṇāt// śarīre  rogāsaṃsargo na bhavati \U2
%-----------------------------
%From the execution of these fixations contact of diseases does not arise within the body. 
%-----------------------------
eteṣāṃ la\app{\lem[wit={ceteri},alt={°kṣyāṇāṃ}]{kṣyāṇāṃ}
  \rdg[wit={P}]{°kṣāṇāṃ}
  \rdg[wit={B}]{°kṣaṇaṃ}
  \rdg[wit={L}]{°kṣaṃ}
  \rdg[wit={N2}]{°kṣāṇā}}
\app{\lem[wit={N2}, alt={kāraṇāc}]{kāraṇā\skp{c-cha}}
  \rdg[wit={E}]{kāraṇāt}
  \rdg[wit={ceteri}]{karaṇāt}
}\app{\lem[wit={N2}, alt={charīre}]{\skm{c-cha}rīre}
  \rdg[wit={N1,D}]{śarīra°}
  \rdg[wit={B,P,L,U2}]{śarīre}
  \rdg[wit={E}]{°śarīraṃ}}
\app{\lem[wit={ceteri}, alt={rogāsaṃsargo}]{rogāsaṃsargo}
  \rdg[wit={E}]{rogāsaṃsargi}}
\app{\lem[wit={ceteri}]{na}
  \rdg[wit={E}]{\om}}
bhava\app{\lem[wit={ceteri}, alt={°ti}]{ti}
  \rdg[wit={B}]{°tī}}/
% -----------------------------
%tathā valitapalitaṃ   puṇyaṃ pāpaṃ    na bhavati//    \E
%tathā valitapalitaṃ   puṇyāṃ pāpaṃ ca na bhavati   \P
%tathā// valitapalitaṃ puṇyāṃ pāpaṃ ca na bhavatī// \B
%tathā valitaṃ palitaṃ puṇyāṃ pāpaṃ ca na bhavatī// \L
%tathā valitaṃ palitaṃ puṇyaṃ pāpaṃ ca na bhavati// \N1
%tathā valitaṃ palitaṃ puṇyaṃ pāpaṃ ca na bhavati// \D
%tathā valitaṃ palitaṃ puṇyaṃ pāpaṃ ca na bhavati// \N2
%tathā valitaṃ palitaṃ puṇyaṃ pāpaṃ ca na bhati \U1
%tathā valīpalitaṃ     puṇyaṃ pāpaṃ ca na bhavati \U2
%-----------------------------
%Thus wrinkles and grey hair, sin or merit does not arise. 
%-----------------------------
tathā
\app{\lem[wit={L,D,N1,N2}, alt={valitaṃ palitaṃ}]{valitaṃ palitaṃ}
  \rdg[wit={N2}]{valīpalitaṃ}
  \rdg[wit={B,E,P}]{valitapalitaṃ}}
pu\app{\lem[wit={ceteri},alt={°ṇyaṃ}]{ṇyaṃ}
  \rdg[wit={B,L}]{°ṇyāṃ}}
pāpaṃ
\app{\lem[wit={ceteri}]{ca}
  \rdg[wit={E}]{\om}}
na
\app{\lem[wit={ceteri}]{bhavati}
  \rdg[wit={B,L}]{bhavatī}
  \rdg[wit={U1}]{bhati}}/
\end{prose}
\end{ekdosis}
\begin{ekdosis}
\begin{tlg}
% -----------------------------
%          navacakraṃ kalādhāraṃ trilakṣyaṃ vyomapaṃcakam/ \E
%          navacakraṃ kalādhāraṃ trilakṣyaṃ vyomapaṃcakaṃ  \P
%śloka     navacakraṃ kalādhāraṃ trilakṣaṃ  vyomapaṃcakam/ \B
%//śloka// navacakraṃ kalādhāraṃ trilakṣaṃ  vyomapaṃcakam... \L %%%%%%%%%%%%GREP THIS%%%%%%%%%%%%% SSP 2.31!!!
%          navacakra--kalādhāraṃ trilakṣyaṃ vyomapaṃcakaṃ/ \N1
%          navacakra--kalādhāraṃ trilakṣyaṃ vyomapaṃcakaṃ// \D
%          navacakra--kalādhāraṃ trilakṣaṃ  vyomapaṃcakaṃ/ \N2
%          navacakraṃ kalādhāraṃ trilakṣyaṃ vyomapaṃcakaṃ \U1 %%%282.jpg
%          navacakraṃ kalādhāraṃ trilakṣyaṃ vyomapaṃcakaṃ// \U2
%-----------------------------
%The nine cakras, the sixteen Adhāras, the three lakṣyas and die five spaces. 
%-----------------------------
\tl{
\app{\lem[wit={ceteri}]{navacakraṃ}
  \rdg[wit={B,L}]{śloka navacakraṃ}
  \rdg[wit={D,N1,N2}]{navacakra°}}
kalādhāraṃ
\note[type=source, labelb=197, lem={navacakraṃ}]{SSP: navacakraṃ kalādhāraṃ trilakṣyaṃ vyomapañcakam | samyag etan na jānāti sa yogī nāmadhārakaḥ||2.31|| NT: ataḥ paraṃ pravakṣyāmi dhyānaṃ sūkṣmam anuttamam | ṛtucakraṃ svarādhāraṃ trilakṣyaṃ vyomapañcakam ||7.1||}
\note[type=testium, labelb=198, lem={navacakraṃ}]{Ysv (PT): navacakraṃ kalādhāraṃ trilakṣaṃ vyomapañcakam | svadehe yo na jānāti sa yogī nāmadhārakaḥ |}
\app{\lem[wit={ceteri}, alt={°kṣyaṃ}]{trilakṣyaṃ}
    \rdg[wit={B,L,N2}]{trilakṣaṃ}}
  vyomapaṃcakaṃ/}\\
%-----------------------------
%svadehe yo na jānāti sa yogī nāmadhārakaḥ//       \E
%svadehe yo na jānāti sa yogī nāmadhārakaḥ 1       \P
%svadehe yo na jānāti sa yogī nāmadhārakaḥ//1//    \B
%svadehe yo na jānāti sa yogī nāmadhārakaḥ//1//   \L
%samakriyā  na jānāti sa yogī nāmadhāraka//           \N1
%samakriyā  na jānāti sa yogī nāmadhārakaḥ//           \D
%samakriyā  na jānāti sa yogī nāmadhāraka//           \N2
%samakriyā  na jānāti sa yogī nāmadhārakaḥ            \U1
%svadehe yo na jānāti sa yogī nāmadhārakaḥ        \U2
%-----------------------------
%Who does not know [them?] within ones own body, he is only a Yogin by name. 
%-----------------------------
\tl{\app{\lem[wit={ceteri}]{svadehe yo}
  \rdg[wit={D,N1,N2,U1}]{samakriyā}} 
na jānāti sa yogī nāmadhārakaḥ\dd{}}
\end{tlg}
\end{ekdosis}
\ekdpb*{}
%%%%%%%%%%%%%%%%%%%%%%%%%%%%%%%%%%%%%%%%%%
%%%%%%%%%%%%%%%%%%%%%%%%%%%%%%%%%%%%%%%%%%
%%%%%%%%PAGEBREAK%%%%%%%PAGEBREAK%%%%%%%%%
%%%%%%%%%%%%%%%%%%%%%%%%%%%%%%%%%%%%%%%%%%
%%%%%%%%%%%%%%%%PAGEBREAK%%%%%%%%%%%%%%%%%
%%%%%%%%%%%%%%%%%%%%%%%%%%%%%%%%%%%%%%%%%%
%%%%%%%%PAGEBREAK%%%%%%%PAGEBREAK%%%%%%%%%
%%%%%%%%%%%%%%%%%%%%%%%%%%%%%%%%%%%%%%%%%%
%%%%%%%%%%%%%%%%%%%%%%%%%%%%%%%%%%%%%%%%%%
%%%%%%%%%%%%%%%%%%%%%%%%%%%%%%%%%%%%%%%%%%
%%%%%%%%%%%%%%%%%%%%%%%%%%%%%%%%%%%%%%%%%%
%%%%%%%%PAGEBREAK%%%%%%%PAGEBREAK%%%%%%%%%
%%%%%%%%%%%%%%%%%%%%%%%%%%%%%%%%%%%%%%%%%%
%%%%%%%%%%%%%%%%PAGEBREAK%%%%%%%%%%%%%%%%%
%%%%%%%%%%%%%%%%%%%%%%%%%%%%%%%%%%%%%%%%%%
%%%%%%%%PAGEBREAK%%%%%%%PAGEBREAK%%%%%%%%%
%%%%%%%%%%%%%%%%%%%%%%%%%%%%%%%%%%%%%%%%%%
%%%%%%%%%%%%%%%%%%%%%%%%%%%%%%%%%%%%%%%%%%
%%%%%%%%%%%%%%%%%%%%%%%%%%%%%%%%%%%%%%%%%%
%%%%%%%%%%%%%%%%%%%%%%%%%%%%%%%%%%%%%%%%%%
%%%%%%%%PAGEBREAK%%%%%%%PAGEBREAK%%%%%%%%%
%%%%%%%%%%%%%%%%%%%%%%%%%%%%%%%%%%%%%%%%%%
%%%%%%%%%%%%%%%%PAGEBREAK%%%%%%%%%%%%%%%%%
%%%%%%%%%%%%%%%%%%%%%%%%%%%%%%%%%%%%%%%%%%
%%%%%%%%PAGEBREAK%%%%%%%PAGEBREAK%%%%%%%%%
%%%%%%%%%%%%%%%%%%%%%%%%%%%%%%%%%%%%%%%%%%
%%%%%%%%%%%%%%%%%%%%%%%%%%%%%%%%%%%%%%%%%%
\begin{ekdosis}
  \ekddiv{type=ed}
  \centerline{\textrm{\small{[The Order of Cakras]}}}
  \bigskip
 \begin{prose}
   \noindent
%-----------------------------
%idānīṃ cakrāṇām anukramaḥ  kathyate/    \E
%idānīṃ cakrāṇām anukramaḥ  kathyate     \P
%idānīṃ cakrāṇām anukramaḥ//             \B
%idānīṃ cakrāṇām anukramaḥ//             \L 19.jpg 
%idānīṃ cakrāṇām anukrama   kathyaṃte/   \N1
%idānīṃ cakrāṇām anukramā   kathyaṃte//  \D
%idānīṃ cakrānām-anukramā   kathyaṃte/   \N2
%idānīṃ cakrānām anukramaḥ  kathyate     \U1
%idānīṃ cakrānām anukramaḥ  kathyate//   \U2
%-----------------------------
%Now the practice of the cakras is explained. 
%-----------------------------
idānīṃ cakrānām-\app{\lem[wit={ceteri}, alt={anukramaḥ}]{anukramaḥ}
  \rdg[wit={N1}]{anukrama}
  \rdg[wit={D,N2}]{anukramā}}
\app{\lem[wit={ceteri}]{kathyate}
  \rdg[wit={D,N1,N2}]{kathyaṃte}}/ \\
\note[type=source, labelb=199, lem={cakrāṇāṃ}]{SSP: atha piṇḍavicāraḥ kathyate piṇḍe navacakrāṇi |}
\note[type=philcomm, labelb=200, lem={cakrāṇāṃ}]{Even tough Rāmacandra's descriptions of the \textit{cakra}s are more brief in this section, they are certainly based on the respective passage in the SSP, since what follows in both texts is the description of the 16 \textit{ādhāra}s. Structurally it seems redundant of Rāmacandra to add another account of the ninefold \textit{cakra}-system.}
%-----------------------------
%ādhāre brahmacakram/    ādhāropari liṃgamūle sbādhiṣṭhānacakram/     nābhau maṇipūrakacakram/     hṛdaye anāhatacakram/     kaṇṭhasthāne viśuddhicakram/     \E
%ādhāre brahmacakraṃ 1   ādhāropari liṃgamūle svādhiṣṭhānacakram 2    nābhau maṇipūrakacakraṃ      hṛdaye 'nāhatacakraṃ 4    kaṃṭhasthāne viśuddhicakraṃ 5    \P
%ādhāro brahmacakram/    ādhāropari liṃgamūle svādhiṣṭhānacakraṃ//2// nābhau maṇipūrakacakram//3   hṛdaye anāhatacakram// 4  kaṇṭhasthāne viśuddhicakraṃ//    \B
%ādhāro brahmacakram//   ādhāropari liṃgamūle svādhiṣṭhānacakraṃ//2// nābhau maṇipūrakacakram//3// hṛdaye anāhatacakram//4// kaṇṭhasthāne viśuddhacakraṃ//    \L
%ādhāre brahmacakraṃ                liṃge     svādhiṣṭhānacakram/     nābhau maṇipūrakacakram/     hṛdaye viśuddhacakraṃ/    kaṇṭhasthāne anāhatacakraṃ/      \N1
%ādhāre brahmacakraṃ                liṃge     svādhiṣṭhānacakram//    nābhau maṇipūrakacakraṃ//    hṛdaye viśuddhacakraṃ//   kaṃṭhasthāne anāhatacakraṃ//     \D
%ādhāre brahmacakraṃ                liṃge     svādhiṣṭhānacakram//    nābhau maṇipūrakacakram/     hṛdaye viśuddhacakraṃ/    kaṇṭhasthāne anāhatacakraṃ       \N2
%ādhāre brahmacakraṃ                liṃge     svādhiṣṭhānacakraṃ      nābhau maṇipūrakacakraṃ      hṛdaye viśuddhacakraṃ     kaṇṭhasthāne anāhatacakraṃ       \U1
%ādhāre brahmacakraṃ//1// ādhāropariliṃgamūle svādhiṣṭhānacakraṃ//2// nābhau maṇipūrakacakraṃ//3// hṛdaye anāhatacakraṃ//4// kaṇṭhasthāne viśuddhacakraṃ//5// \U2
%-----------------------------
%At the pelvic floor there is the Brahmacakra. Above the pelvic floor at the root of the gender is the Svadiṣṭhānacakra. At the navel there is the Maṇipūrakacakra. At the heart the Anāhatacakra. Situated within the throat is the Viśuddhicakra. 
%-----------------------------
\note[type=source, labelb=201, lem={brahmacakram}]{SSP: ādhāre brahmacakraṃ tridhāvartaṃ bhagamaṇḍalākāram | tatra mūlakandaḥ | tatra śaktiṃ pāvakākārāṃ dhyāyet | tatraiva kāmarūpapīṭhaṃ sarvakāmaphalapradaṃ bhavati ||2.1||}
\app{\lem[wit={ceteri}]{ādhāre}
  \rdg[wit={B,L}]{ādhāro}}
\app{\lem[wit={B,E,L}]{brahmacakram}
  \rdg[wit={ceteri}]{brahmacakraṃ}} 1 \dd{} 
\app{\lem[wit={ceteri}]{ādhāropari}
  \rdg[wit={D,N1,N2,U1}]{\om}}
\app{\lem[wit={ceteri}]{liṅgamūle}
  \rdg[wit={D,N1,N2,U1}]{liṅge}}
\note[type=source, labelb=202, lem={svādhiṣṭhāna°}]{SSP: dvitīyaṃ svādhiṣṭhānacakram | tanmadhye paścimābhimukhaṃ liṅgaṃ pravālāṅkurasadṛśaṃ dhyāyet | tatraivoḍyānapīṭhaṃ jagadākarṣaṇaṃ bhavati ||2.2||}
\app{\lem[wit={E,D,P,N1,N2}]{svādhiṣṭhānacakram}
  \rdg[wit={ceteri}]{svādhiṣṭhānacakraṃ}} 2 \dd{}
\note[type=source, labelb=203, lem={maṇipūraka°}]{SSP:tṛtīyaṃ nābhicakraṃ pañcāvartaṃ sarpavat kuṇḍalākāram | tanmadhye kuṇḍalinīṃ śaktiṃ bālārkakoṭisannibhāṃ dhyāyet | sā madhyā śaktiḥ sarvasiddhidā bhavati ||2.3||}
nābhau \app{\lem[wit={E,P,L,N1,N2}]{maṇipūrakacakram}
  \rdg[wit={ceteri}]{maṇipūrakacakraṃ}} 3 \dd{} \\
\note[type=source, labelb=204, lem={anāhata°}]{SSP: caturthaṃ hṛdayacakram aṣṭadalakamalam adhomukhaṃ tanmadhye karṇikāyāṃ liṅgākārāṃ jyotīrūpām dhyāyet | saiva haṃsakalā sarvendriyavaśyā bhavati ||2.4||}
hṛdaye
\app{\lem[wit={P}, alt={'nāhata°}]{'nāhata}
  \rdg[wit={E,B,L,U2}]{anāhata°}
  \rdg[wit={ceteri}]{viśuddha°}
}\app{\lem[wit={E,B,L}]{cakram}
  \rdg[wit={ceteri}]{cakraṃ}} 4 \dd{}
kaṇṭhasthāne
\app{\lem[wit={E,P,B,L,U2}]{viśuddhicakram}
  \rdg[wit={ceteri}]{anāhatacakraṃ}} 5 \dd{} 
\note[type=source, labelb=205, lem={viśuddhi°}]{SSP: pañcamaṃ kaṇṭhacakraṃ caturaṅgulam | tatra vāma iḍā candranāḍī | dakṣiṇe piṅgalā sūryanāḍī | tanmadhye suṣumnāṃ dhyāyet | saiva anāhatakalā anāhatasiddhidā bhavati ||2.5||}
%-----------------------------
%ṣaṣṭhaṃ tālucakram/     bhruvor madhye ājñācakram/       brahmasthāne        kālacakram/     navamam         ākāśacakram/       etat--paraṃ śūnyam/              \E
%ṣaṣṭhaṃ tālucakraṃ 6    bhruvor madhye agnejacakraṃ 7    brahmasthāne        kālacakraṃ 8    navamaṃ         ākāśacakraṃ 8      tataḥ paraṃ śūnyaṃ               \P
%ṣaṣṭhaṃ tālucakre/6     bhruvor madhye ājñāyacakraṃ/     brahmasthāne        kālacakraṃ// 8  navamaṃ         ākāśacakraṃ/9      tat---paraṃ śūnyam/              \B
%ṣaṣṭha  tālucakre//6//  bhruvor madhye āgneyacakraṃ//7// brahmasthāne        kālacakraṃ//8// navamaṃ         ākāśacakraṃ//9//   tat---paraṃ śūnyam//             \L
%ṣaṣṭhaṃ tālucakram/     bhruvor madhye ājñācakram        brahmaraṃdhrasthāne kālacakraṃ/     navamaṃ         ākāśacakram/       tat---paramaśūnyaṃ/              \N1
%ṣaṣṭhaṃ tālucakraṃ//    bhruvor madhye ājñācakraṃ//      brahmaraṃdhrasthāne kālacakraṃ//    navamaṃ         ākāśacakram/       tat---paraṃ// tatparamaśūnyaṃ// \D
%ṣaṣṭhaṃ tālucakram/     bhruvor madhye ājñācakram        brahmaraṃdhrasthāne kālacakraṃ/     navama          ākāśacakram       tata---paraśūnyaṃ/               \N2
%ṣaṣṭhaṃ tālucakraṃ      bhruvor madhye ājñācakram        brahmaraṃdhrasthāne brahmacakraṃ    navamaṃ rattu?! ākāśacakram         tat--paraśūnyaṃ                \U1
%        tālucakra //6// bhruvor madhye ājñācakram//7//   brahmaraṃdhrasthāne kalācakraṃ//8//                 ākāśacakram ūrdhvaṃ tat--paraṃ śūnyaṃ//9//         \U2
%-----------------------------
%The sixth is the cakra of the palate. In the center of the eyebrows is the Ājñācakra. At the opening of Brahma is the Kālacakra. The ninth is the Ākāśacakra. It is supreme emptiness. 
%-----------------------------
\note[type=source, labelb=206, lem={tālu°}]{SSP: ṣaṣṭhaṃ tālucakram | tatrāmṛtadhārāpravāhaḥ | ghaṃṭikāliṅgaṃ mūlarandhraṃ rājadantaṃ śaṃkhinīvivaraṃ daśamadvāram | tatra śūnyaṃ dhyāyet | cittalayo bhavati ||2.6||}
\app{\lem[wit={ceteri}]{ṣaṣṭhaṃ}
  \rdg[wit={L}]{ṣaṣṭha°}}
\app{\lem[wit={E,N1,N2}]{tālucakram}
  \rdg[wit={D,P,U1}]{tālucakraṃ}
  \rdg[wit={B,L}]{tālucakre}
  \rdg[wit={U2}]{tālucakra}} 6 \dd{} \\
bhruvor madhye
\note[type=source, labelb=207, lem={ājñā°}]{SSP: aptamaṃ bhrūcakraṃ madhyamāṅguṣṭhamatram | tatra jñānanetraṃ dīpaśikhākāraṃ dhyāyet | tatra vāksiddhir bhavati ||2.7||}
\app{\lem[wit={ceteri}, alt={°ājñā}]{ājñā}
  \rdg[wit={P}]{agneja}
  \rdg[wit={L}]{āgneya}
  \rdg[wit={B}]{ājñāya}
}\app{\lem[wit={E,D,N1,N2,U1,U2}]{cakraṃ}
  \rdg[wit={B,D,P,L}]{cakram}}  7 \dd{}
\note[type=source, labelb=208, lem={brahmarandhra°}]{SSP: aṣṭamaṃ brahmarandhraṃ nirvāṇacakraṃ sūcikāgrabhedyam | tatra dhūmaśikhākāraṃ dhyāyet | tatra jālandharapīṭhaṃ mokṣapradaṃ bhavati ||2.8||}
brahma\app{\lem[wit={ceteri}, alt={°randhra°}]{randhra}
  \rdg[wit={B,E,L,P}]{\om}}sthāne
\app{\lem[wit={ceteri}, alt={°kāla}]{kāla}
  \rdg[wit={U1}]{brahma°}
}\app{\lem[wit={E}]{cakram}
  \rdg[wit={ceteri}]{cakraṃ}} 8 \dd{}
\note[type=source, labelb=209, lem={ākāśa°}]{SSP: navamam ākāśacakraṃ soḍaśadalakamalam ūrdhvamukham | tanmadhye karṇikāyāṃ trikūṭākārāṃ tadūrdhvaśaktiṃ tāṃ paramaśunyāṃ dhyāyet | tatraiva pūrṇagiripīṭhaṃ sarveṣṭasiddhir bhavati ||2.9|| iti navacakravicāraḥ ||}
\app{\lem[wit={E}, alt={navamam}]{navama\skp{m-ā}}
  \rdg[wit={N2}]{navama}
  \rdg[wit={U1}]{navamaṃ rattu}
  \rdg[wit={ceteri}]{navamaṃ}}
\skm{m-ā}kāśa\app{\lem[wit={E,D,N1,N2,U1,U2}]{cakram}
  \rdg[wit={B,L,P}]{cakraṃ}} \dd{} 9\\
\app{\lem[wit={B,L,D,N1,U1,U2}, alt={tat°}]{ta\skp{t-pa}}
  \rdg[wit={E}]{etat}
  \rdg[wit={P}]{tataḥ}
  \rdg[wit={N2}]{tata}
}\app{\lem[wit={N1},alt={°parama°}]{\skm{t-pa}rama}
  \rdg[wit={E,P,B,L,D,U2}]{°paraṃ}
  \rdg[wit={N2,U1}]{para°}
}\app{\lem[wit={B,E,L}, alt={°śūnyam}]{śūnyam}
  \rdg[wit={P,N1,N2,U1,U2}]{°śūnyaṃ}
  \rdg[wit={D}]{tatparamaśūnyaṃ}}\dd{}\\
\end{prose}
\end{ekdosis}
\ekdpb*{}
%%%%%%%%%%%%%%%%%%%%%%%%%%%%%%%%%%%%%%%%%%
%%%%%%%%%%%%%%%%%%%%%%%%%%%%%%%%%%%%%%%%%%
%%%%%%%%PAGEBREAK%%%%%%%PAGEBREAK%%%%%%%%%
%%%%%%%%%%%%%%%%%%%%%%%%%%%%%%%%%%%%%%%%%%
%%%%%%%%%%%%%%%%PAGEBREAK%%%%%%%%%%%%%%%%%
%%%%%%%%%%%%%%%%%%%%%%%%%%%%%%%%%%%%%%%%%%
%%%%%%%%PAGEBREAK%%%%%%%PAGEBREAK%%%%%%%%%
%%%%%%%%%%%%%%%%%%%%%%%%%%%%%%%%%%%%%%%%%%
%%%%%%%%%%%%%%%%%%%%%%%%%%%%%%%%%%%%%%%%%%
%%%%%%%%%%%%%%%%%%%%%%%%%%%%%%%%%%%%%%%%%%
%%%%%%%%%%%%%%%%%%%%%%%%%%%%%%%%%%%%%%%%%%
%%%%%%%%PAGEBREAK%%%%%%%PAGEBREAK%%%%%%%%%
%%%%%%%%%%%%%%%%%%%%%%%%%%%%%%%%%%%%%%%%%%
%%%%%%%%%%%%%%%%PAGEBREAK%%%%%%%%%%%%%%%%%
%%%%%%%%%%%%%%%%%%%%%%%%%%%%%%%%%%%%%%%%%%
%%%%%%%%PAGEBREAK%%%%%%%PAGEBREAK%%%%%%%%%
%%%%%%%%%%%%%%%%%%%%%%%%%%%%%%%%%%%%%%%%%%
%%%%%%%%%%%%%%%%%%%%%%%%%%%%%%%%%%%%%%%%%%
%%%%%%%%%%%%%%%%%%%%%%%%%%%%%%%%%%%%%%%%%%
%%%%%%%%%%%%%%%%%%%%%%%%%%%%%%%%%%%%%%%%%%
%%%%%%%%PAGEBREAK%%%%%%%PAGEBREAK%%%%%%%%%
%%%%%%%%%%%%%%%%%%%%%%%%%%%%%%%%%%%%%%%%%%
%%%%%%%%%%%%%%%%PAGEBREAK%%%%%%%%%%%%%%%%%
%%%%%%%%%%%%%%%%%%%%%%%%%%%%%%%%%%%%%%%%%%
%%%%%%%%PAGEBREAK%%%%%%%PAGEBREAK%%%%%%%%%
%%%%%%%%%%%%%%%%%%%%%%%%%%%%%%%%%%%%%%%%%%
%%%%%%%%%%%%%%%%%%%%%%%%%%%%%%%%%%%%%%%%%%
\begin{ekdosis}
  \ekddiv{type=ed}
 \centerline{\textrm{\small{[The sixteen Container]}}}
 \bigskip
 \begin{prose}
   \noindent
%-----------------------------
%idānīm ādhāracakrasya bhedāḥ kathyanta/   \E
%idānīm ādhāracakrasya bhedaḥ kathyate     \P
%idānīm ādhāracakrasya bhedā  kathyaṃte/    \B DSCN7165.jpg Z.3
%idānīm ādhāracakrasya bhedā  kathyaṃte//   \L
%idānīm ādhāracakrasya bhedaḥ kathyate/    \N1
%idānīṃ ādhāracakrasya bhedaḥ kathyate//   \D
%idānī  ādhāracakrasya bhedaḥ kathyaṃte/   \N2
%idānīṃ ādhāracakrasya bhedāḥ kathyaṃte    \U1
%idānīṃ ādhāracakrasya bhedāḥ kathyaṃte // \U2
%-----------------------------
%Now the  divisions of the container-\textit{cakra}s are taught.
%-----------------------------
\note[type=source, labelb=210, lem={ādhāracakrasya}]{SSP: atha ṣoḍaśādhārāḥ kathyante |}
\note[type=source, labelb=211, lem={ādhāracakrasya}]{Ysv (PT=YK): ṣoḍaśādhārabhedan tu śṛṇu devi viśeṣataḥ |}
\app{\lem[wit={ceteri}, alt={idānīm}]{idānī\skp{m-ā}}
  \rdg[wit={N2}]{idānī}
}\skm{m-ā}dhāracakrasya
\app{\lem[wit={ceteri}]{bhedāḥ}
  \rdg[wit={B,L}]{bhedā}}
\app{\lem[wit={ceteri}]{kathyante}
  \rdg[wit={E}]{kathyanta}
  \rdg[wit={N1,D}]{kathyate}}/ 
%-----------------------------
%pādayor aṃguṣṭhe  tejaso  lakṣyakāraṇāt              dṛṣṭiḥ sthirā bhavati/ \E
%pādayor aṃguṣṭhe  tejaso  lakṣyakaraṇāt              dṛṣṭiḥ sthirā bhavati  \P
%pādayor aṃguṣṭhai tejasaṃ lakṣaṃ kartavyaṃ kāraṇāt// dṛṣṭiḥ sthirā bhavati/ \B
%pādayor aṃguṣṭhe  tejasaṃ lakṣaṃ karttavyaṃ kāraṇāt  dṛṣṭiḥ sthirā bhavatī/ \L
%pādayor aṃguṣṭhe  tejaso  lakṣyakāraṇāt              dṛṣṭisthirā   bhavati/ \N1
%pādayor aṃguṣṭhe  tejaso  lakṣyakāraṇāt              dṛṣṭiḥ sthirā bhavati \D
%pādayor aṃguṣṭhe  tejaso  lakṣakāraṇāt               dṛṣṭisthirā   bhavati/ \N2
%pādayor aṃguṣṭhe  tejaso  lakṣyakāraṇāt              dṛṣṭisthirā   bhavati \U1
%pādayor aṃguṣṭhe  tejaso  lakṣyakāraṇāt              dṛṣṭisthirā   bhavati// \U2 %%%415.jpg
%-----------------------------
%From the execution of the fixation onto the light at the big toes of the feet stability of the gaze arises.
%-----------------------------
\note[type=source, labelb=212, lem={ādhāracakrasya}]{SSP: tatra prathamaḥ pādāṅguṣṭhādhāraḥ | tatrāgratas tejomayaṃ dhyāyet | dṛṣṭiḥ sthirā bhavati ||2.10|||}
\note[type=source, labelb=213, lem={ādhāracakrasya}]{Ysv (PT): aṅguṣṭhapādayos tejaḥ salakṣasthiradṛṣṭimān | pādāṅguṣṭhe ya ādhāraḥ prathamo [prathamaṃ (YK)] yogatattvataḥ | }
\app{\lem[type=conjecture, resp=egoscr]{prathamaḥ pādāṅguṣṭhādhāraḥ}
  \rdg[wit={ceteri}]{\conj \om}}\dd{}
\note[type=philcomm, labelb=214, lem={pādāṅguṣṭhādhāraḥ}]{Judging by the source and parallels as well as the introductory statements in the following \textit{ādhāra}s, as well as previous passages that must have been dropped in the text's transmission to me it seems more likely than not that originally the first \textit{ādhāra} was introduced, too.}
pādayo\skp{r-aṃ}\app{\lem[wit={ceteri}, alt={aṃguṣṭhe}]{\skm{r-aṃ}guṣṭhe}
  \rdg[wit={B}]{aṃguṣṭhai}}
\app{\lem[wit={ceteri}]{tejaso}
  \rdg[wit={B,L}]{tejasaṃ}}
\app{\lem[wit={ceteri}, alt={lakṣya°}]{lakṣya}
  \rdg[wit={N2}]{lakṣa°}
  \rdg[wit={B,L}]{lakṣaṃ kartavyaṃ}
}\app{\lem[wit={ceteri}, alt={°kāraṇāt}]{kāraṇāt}
  \rdg[wit={P}]{°karaṇāt}} 
 \app{\lem[wit={ceteri}]{dṛṣṭiḥ}
   \rdg[wit={N1,N2,U1,U2}]{dṛṣṭi°}}sthirā
 \app{\lem[wit={ceteri}]{bhavati}
   \rdg[wit={L}]{bhavatī}}/
%-----------------------------
%dvitīyo mūlādhāraḥ/  pādāṃguṣṭhasya mūle parapādasya  pārṣṇiḥ                                         sthāpyate tadāgniḥ prabalo bhavati/ \E
%dvitīyo mūlādhāraḥ   pādāṃguṣṭhasya mūle 'parapādasya dhāraḥ pādāṃduṣṭhasya mūleḥ paradādasya pārṣṇiḥ sthāpyate tadāgniḥ prabalo bhavati \P
%dvitīyo mūlādhāraḥ/  pādāṃguṣṭhasya mūle aparasya pādapārṣṇiḥ                                         syāpyate tadāgniḥ  prabalo bhavatī/ \B
%dvitīyo mūlādhāraḥ   pādāṃguṣṭhasya mūle aparasya pādapārṣṇīḥ                                         syāpyate tadāgniḥ  prabalo bhavatī/ \L
%dvitīyo mūlādhāraḥ/  pādāṃguṣṭhasya mūle aparapādasya pārṣṇiḥ                                         sthāpyate agniḥ    prabalo bhavati/   \N1
%dvitīyo mūlādhāraḥ// pādāṃguṣṭhasya mūle aparapādasya pārṣṇiḥ                                         sthāpyate agni-----prabalo bhavati//   \D  %%%p.12 recto
%dvitīyo mūlādhāraḥ   pādāṃguṣṭhasya mūle aparapādasya pārṣṇiḥ                                         sthāpyate/ \om                     \N2
%dvitīyo mūlādharaḥ   pādāṃguṣṭhasya mūle aparapādasya pārṣṇiḥ                                         sthāpyate agniṃ ---prabalo bhavati    \U1
%dvitīyo mūlādhare    pādāṃguṣṭhasya mūle 'parapādasya pārṣṇiḥ                                         sthāyyaṃte//                       \U2
%-----------------------------
%The root-container is the second [one]. The heel of the backfoot is caused to be placed at the root of the big toe. As a result the fire is strengthened. 
%-----------------------------
 \note[type=source, labelb=215, lem={mūlādhāraḥ}]{SSP: dvitīyo mūlādhāras taṃ vāmapādapārṣṇinā niṣpīḍya sthātavyam | tatrāgnidīpanaṃ bhavati ||2.11||}
%The second is the Mūlādhara which is to be pressend with the left heel. This enhances the bodily fire. 
 \note[type=source, labelb=216, lem={mūlādhāraḥ}]{Ysv (PT): dvitīyaṃ pādamūlan tu pādamūlaparaṃ [pādamūlaṃ paraṃ (YK)] sa vai | pādasya pārṣṇī [pārṣṇi (YK)] saṃsthāpya balavān prabhaven muniḥ | pādamūle 'thavā pādāṅguṣṭhamūlaṃ [pṛṣṭhe pādāṅguṣṭhe (YK)] vidhārayet ||}
%The second is the root of the foot. That root of the foot is truly superior. Having placed himself on the heel of the foot the Muni becomes powerful. He shall hold [the gaze?] at the root of the foot or at the back or at the big toe.
dvitīyo
\app{\lem[wit={ceteri}]{mūlādhāraḥ}
  \rdg[wit={U1}]{mūlādharaḥ}
  \rdg[wit={U2}]{mūlādhare}}\dd{}
pādāṃguṣṭhasya mūle
\app{\lem[wit={ceteri},alt={'para°}]{'para}
  \rdg[wit={N1,N2,D,U1}]{apara°}
  \rdg[wit={B,L}]{aparasya}
}\app{\lem[wit={ceteri}]{pādasya}
  \rdg[wit={B,L}]{pāda°}}
\app{\lem[wit={ceteri}]{pārṣṇiḥ}
  \rdg[wit={L}]{°pārṣṇīḥ}
  \rdg[wit={P}]{dhāraḥ pādāṃduṣṭhasya mūleḥ paradādasya pārṣṇiḥ}}
\app{\lem[wit={ceteri}]{sthāpyate}
  \rdg[wit={B,L}]{syāpyate}
  \rdg[wit={U2}]{sthāyyaṃte}}
\app{\lem[wit={ceteri}]{tadāgniḥ}
  \rdg[wit={N1}]{agniḥ}
  \rdg[wit={D}]{agni°}
  \rdg[wit={U2}]{\om}}
\app{\lem[wit={ceteri}]{prabalo}
  \rdg[wit={N2,U2}]{\om}}
\app{\lem[wit={ceteri}]{bhavati}
  \rdg[wit={B,L}]{bhavatī}
  \rdg[wit={N2,U2}]{\om}}/
%-----------------------------
%ekaḥ  pārṣṇir ādau  mūlādhāre  sthāpyate/     tasya pādasyāṃguṣṭhamūle      parasya  pādasya pārṣṇiḥ sthāpyate// tadagniḥ pradīpyate// \E [P.41]
%ekā   pārṣṇir ādau  mūlādhāre  sthāpyate      tasya pādasyāṃguṣṭhamūle     'parasya  pādasya pārṣṇiḥ sthāpyate   tadagnīḥ pradipyate \P
%ekā   pārṣṇir ādau  mūlādhāra  sthāpyate      tasya pādasyāṃguṣṭhamūle     aparasya  pādasya pārṣṇiḥ sthāpyate// tadagnīḥ pradipyate// \B
%ekā   pārṣṇir ādau  mūlādhārā  sthāpyate      tasya pādasyāṃguṣṭhamūle     aparasya  pādasya pārṣṇiḥ sthāpyate// tadāgnīḥ pradivyate// \L
%ekā   pārṣṇiḥ       mūladdhāre sthāpyate/     tasya pādasya aṃguṣṭhamūlaṃ/ aparasya  pādasya pārṣṇiḥ sthāpyaṃ agnir dāpyate?!/ \N1
%ekā   pārṣṇiḥ       mūlādhārai sthāpyate//    tasya pādasyāṃguṣṭhamūle//   aparasya  pādasya pārṣṇiḥ sthāpyaṃ// agnir dīpyate// \D
% \om -------------------------------------    tasya pādasyāṃguṣṭhamūle//   aparasya  pādasya pārṇisthāpyaṃ agni dīpate// \N2
%ekāṃ pārṣṇir mūlādhāra sthāpyate              tasya pādasya aṃguṣṭhamūlaṃ  aparasya          pārṣṇo sthāpyate agni dīpyate  \U1
% \om                                                                                                         tadagnīḥ pradipyate// \U2
%-----------------------------
%One heel is caused to be placed at the Root-container. The heel of the other foot is caused to be placed at the root of the big toe of this foot. The fire of it is caused to be kindled. 
%-----------------------------
\app{\lem[wit={ceteri}]{ekā}
  \rdg[wit={E}]{ekaḥ}
  \rdg[wit={U1}]{ekāṃ}}
\app{\lem[wit={U1},alt={pārṣṇiḥ}]{pārṣṇi\skp{r-mū}}
  \rdg[wit={N1,D}]{pārṣṇiḥ}
  \rdg[wit={B,E,L,P}]{pārṣṇir ādau}
  \rdg[wit={N2,U2}]{\om}
}\app{\lem[wit={ceteri},alt={mūlādhāre}]{\skm{r-mū}lādhāre}
  \rdg[wit={B,U1}]{mūlādhāra}
  \rdg[wit={L}]{mūlādhārā}
  \rdg[wit={D}]{mūlādhārai}
  \rdg[wit={N2,U2}]{\om}}
%-----------------------------
%tṛtīyaṃ gudādhārasthānaṃ   tanmadhye saṃkocavikāsākuṃcana--kāraṇāt pavanaḥ sthiro bhavati// \E
%tṛtīyaṃ gudādhārasthānaṃ   tanmadhye saṃkocavikāśākuṃcana--kāraṇāt pavanaḥ sthiro bhavati   \P
%tṛtīyaṃ gudādhārasthāne    tanmadhye saṃkocavikāśākuṃcana--kāraṇāt pavanaḥ sthiro bhavati// \B
%tṛtīyaṃ gudādhārasthānaṃ   tanmadhye saṃkocavikāśa ākuṃcanakāraṇāt pavanasthiro   bhavatī// \L
%tṛtīyaṃ gudādhārasthānaṃ   tanmadhye saṃkocavikāśākuṃcana--kāraṇāt pavanaḥ sthiro bhavati// \N1
%tṛtīyaṃ gudādhārasthānaṃ   tanmadhye saṃkocavikāśākuṃcanaṃ kāraṇāt pavanasthiro   bhavati// \D
%tṛtīyaṃ gudādhārasthānaṃ   taṃmadhye saṃkocavikāśākuṃcanaṃ kāraṇāt pavanasthiro   bhavati// \N2
%tṛtīyaṃ gudādhārasthānaṃ   taṃmadhye saṃkocavikāśā akuṃcanakāraṇāt pavanasthiro   bhavati \U1
%tṛtīya  gudādhārasthānaṃ// tanmadhye saṃkocavikāśākuṃcana--kāraṇāt pavanasthiro   bhavati// \U2
%-----------------------------
%The third is the place of the anus-container. From the execution of expansion and contraction a stable vitalwind arises.   
%-----------------------------
\note[type=source, labelb=217, lem={gudādhāra°}]{SSP: tṛtīyo gudādhāra taṃ vikāsasaṃkocanena nirākuñcayet | apānavāyuḥ sthiro bhavati ||2.12||}
\note[type=source, labelb=218, lem={gudādhāra°}]{Ysv (PT): tṛtīyantu gudādhāro [gudādhāre (YK)] gudasaṅkocanakriyā | vikāśākuñcanaṃ tasya sthiravāyau ca mṛtyujit |}
\app{\lem[wit={ceteri}]{tṛṭīyaṃ}
  \rdg[wit={U2}]{tṛtīya}}
gudādhāra\app{\lem[wit={ceteri},alt={°sthānaṃ}]{sthānaṃ}
  \rdg[wit={B}]{°sthāne}}\dd{}
tanmadhye
saṃkoca\app{\lem[wit={ceteri},alt={°vikāśā}]{vikāśā}
  \rdg[wit={L}]{°vikāśa°}
}\app{\lem[wit={ceteri},alt={°kuṃcana}]{kuṃcana}
  \rdg[wit={L}]{ākuṃcana}
  \rdg[wit={U1}]{akuṃcana}
  \rdg[wit={D,N2}]{kuṃcanaṃ}
}kāraṇāt-\app{\lem[wit={ceteri}]{pavanaḥ}
  \rdg[wit={D,U1,U2,N2}]{pavana°}}
sthiro
\app{\lem[wit={ceteri}]{bhavati}
  \rdg[wit={B}]{bhavatī}}/
%-----------------------------
%anyac ca/ puruṣasya maraṇaṃ na bhavati/ \E
%anu ca puruṣasya maraṇaṃ bhavati  \P
%anucarapuruṣasya maraṇaṃ bhavatī/ \B
%anucakrapuruṣasya maraṇaṃ bhavatī/ \L
%anū ca puruṣasya maraṇaṃ na bhavati ve?/ \N1
%anu ca puruṣasya maraṇaṃ na bhavati// \D
%anū ca puruṣasya maraṇaṃ na bhavati// \N2
%anu ca puruṣasya maraṇaṃ na bhavati  \U1
%anu ca puruṣasya maraṇaṃ na bhavati//  \U2
%-----------------------------
%Additionally death of the person does not arise. Additionally the person does not die.
%-----------------------------
\app{\lem[wit={D,P,U1,U2}]{anu ca}
  \rdg[wit={E}]{anyac ca}
  \rdg[wit={N1,N2}]{anūca}
  \rdg[wit={B}]{anucara°}
  \rdg[wit={L}]{anucakra°}}
puruṣasya maraṇaṃ
\app{\lem[wit={ceteri}]{na}
  \rdg[wit={B,P,L}]{\om}}
\app{\lem[wit={ceteri}]{bhavati}
  \rdg[wit={B,L}]{bhavatī}}/
%-----------------------------
%caturthaṃ liṃgādhāraṃ   tanmadhye/ liṃgasaṃkocanābhyāsāt  paścimadaṇḍamadhye prajñā nāḍī bhavati/  tanmadhye punar abhyāsakaraṇān manaḥ pavanayoḥ saṃcāro bhavati/ \E
%caturthaṃ liṃgādhāraṃ   tanmadhye  liṃgasaṃkocanābhyāsāt  paścīmadaṇḍamadhye vajñā nāḍī  bhavati   tanmadhye punar abhyāsakaraṇān manaḥ pavanayoḥ saṃcāro bhavati \P
%caturtha--liṃgādhāraṃ   tanmadhye  liṃgasaṃkocanābhyāsāt  paścīmadaṇḍamadhye vajñā nāḍī  bhavatī/  tanmadhye punar abhyāsakaraṇāt punaḥ pavanayo  saṃcāro bhavatī/     \B
%caturtha--liṃgādhāraṃ// tanmadhye  liṃgasaṃkocanābhyāsāt  paścamadaṇḍamadhye vajñā nāḍī  bhavatī// tanmadhye punar abhyāsakaraṇāt punaḥ pavanayo  saṃcāro bhavatī//     \L %%%%%%%%%%%20.jpg
%caturthaṃ liṃgādhāraṃ   tanmadhye/ liṃgasaṃkocanābhyāsāt/ paścimadaṇḍamadhye vajranāḍī   bhavati/  tanmadhye punaḥ abhyāsakaraṇāt manaḥpavanayoḥ saṃcāro bhavati/ \N1
%caturtha--liṃgādhāraṃ// tanmadhye/ liṃgasaṃkocanābhyāsāt//paścimadaṇḍamadhye vajrānāḍī   bhavati// tanmadhye punaḥ abhyāsakaraṇāt manaḥpavanayoḥ saṃcoro bhavati// \D
%caturthaṃ liṃgādhāraṃ   tanmadhye  liṃgasakoṇābhyāsāt//   paścimadaṇḍamadhye vajranāḍī   bhavati/  tanmadhye punar ābhyāsakaraṇāt manaḥpavanayoḥ saṃcāro bhavati// \N2
%caturthaṃ liṃgādhāraṃ   tanmadhye  liṃgasaṃkocanābhyāsāt  paścimadaṇḍamadhye vajranāḍī   bhavati   tanmadhye punar ābhyāsakaraṇāt manaḥpavanayoḥ saṃcāro bhavati    \U1    %%%283.jpg
%caturthaṃ liṃgādhāraṃ   tanmadhye  liṃgasaṃkocanābhyāsāt  paścimadaṇḍamadhye vajranāḍī   bhavati   tanmadhye punar ābhyāsakaraṇān manaḥpavanayoḥ saṃcāro bhavati//   \U2
%-----------------------------
%The fourth is the penis-container. Due to the execution of repeated practice of contracting the penis in the midst of therof, the adamantine channel appears in the middle of the staff of the back. From the repeated practice again [and again] the transition of both breath and mind into its center arises.  
%-----------------------------
\note[type=source, labelb=219, lem={liṃgādhāraṃ}]{SSP: caturtho meḍhrādhāraḥ | liṅgasaṃkocanena brahmagranthitrayaṃ bhitvā bhramaraguhāyāṃ viśramya tata ūrdhvamukhe bindustambhanaṃ bhavati| eṣā vajrolī prasiddhā ||2.13||}
\note[type=source, labelb=220, lem={liṃgādhāraṃ}]{Ysv (PT): liṅgādhāraṃ caturthan tu liṅgasaṅkocanan tu ca | liṅgasaṅkocanābhyāsāt paścimādaṇḍamadhyagaḥ | vajranāḍīti [vajrānāḍī tu (YK)] tanmadhye punar abhyasayaṃs [abhyasanan (YK)] tathā | sañcāro vāyumanasor atisañcāra iti [ratiṃ sañcarati (YK)] tridhā | granthitrayavibhedas [°bhedan (YK)] tu tadbhedo brahmamārgataḥ | brahmapadmo [°padme (YK)] vāyupūrṇo [°pūrṇe (YK)] bhūtvā tiṣṭhati yogirāṭ | vīryastambho bhavet tena sādhayet tu sadā yuvā | mūlādhāre brahmapadme ṣaṭpadme ca tathā tathā |}
\app{\lem[wit={ceteri}]{caturthaṃ}
  \rdg[wit={B,L,D}]{caturtha°}}
liṅgādhāraṃ \dd{}
tanmadhye
liṃga\app{\lem[wit={ceteri},alt={saṃkocanā°}]{saṃkocanā}
  \rdg[wit={N2}]{sakoṇā°}
}bhyāsāt
\app{\lem[wit={ceteri}, alt={paścima°}]{paścima}
  \rdg[wit={B,P}]{paścīma°}
  \rdg[wit={L}]{paścama°}
}daṇḍamadhye
\app{\lem[wit={ceteri}, alt={vajra°}]{vajra}
  \rdg[wit={B,P,L}]{vajñā}
  \rdg[wit={E}]{prajñā}
}nāḍī
\app{\lem[wit={ceteri}]{bhavati}
  \rdg[wit={B,L}]{bhavatī}}/
\end{prose}
\end{ekdosis}
\ekdpb*{}
%%%%%%%%%%%%%%%%%%%%%%%%%%%%%%%%%%%%%%%%%%
%%%%%%%%%%%%%%%%%%%%%%%%%%%%%%%%%%%%%%%%%%
%%%%%%%%PAGEBREAK%%%%%%%PAGEBREAK%%%%%%%%%
%%%%%%%%%%%%%%%%%%%%%%%%%%%%%%%%%%%%%%%%%%
%%%%%%%%%%%%%%%%PAGEBREAK%%%%%%%%%%%%%%%%%
%%%%%%%%%%%%%%%%%%%%%%%%%%%%%%%%%%%%%%%%%%
%%%%%%%%PAGEBREAK%%%%%%%PAGEBREAK%%%%%%%%%
%%%%%%%%%%%%%%%%%%%%%%%%%%%%%%%%%%%%%%%%%%
%%%%%%%%%%%%%%%%%%%%%%%%%%%%%%%%%%%%%%%%%%
%%%%%%%%%%%%%%%%%%%%%%%%%%%%%%%%%%%%%%%%%%
%%%%%%%%%%%%%%%%%%%%%%%%%%%%%%%%%%%%%%%%%%
%%%%%%%%PAGEBREAK%%%%%%%PAGEBREAK%%%%%%%%%
%%%%%%%%%%%%%%%%%%%%%%%%%%%%%%%%%%%%%%%%%%
%%%%%%%%%%%%%%%%PAGEBREAK%%%%%%%%%%%%%%%%%
%%%%%%%%%%%%%%%%%%%%%%%%%%%%%%%%%%%%%%%%%%
%%%%%%%%PAGEBREAK%%%%%%%PAGEBREAK%%%%%%%%%
%%%%%%%%%%%%%%%%%%%%%%%%%%%%%%%%%%%%%%%%%%
%%%%%%%%%%%%%%%%%%%%%%%%%%%%%%%%%%%%%%%%%%
%%%%%%%%%%%%%%%%%%%%%%%%%%%%%%%%%%%%%%%%%%
%%%%%%%%%%%%%%%%%%%%%%%%%%%%%%%%%%%%%%%%%%
%%%%%%%%PAGEBREAK%%%%%%%PAGEBREAK%%%%%%%%%
%%%%%%%%%%%%%%%%%%%%%%%%%%%%%%%%%%%%%%%%%%
%%%%%%%%%%%%%%%%PAGEBREAK%%%%%%%%%%%%%%%%%
%%%%%%%%%%%%%%%%%%%%%%%%%%%%%%%%%%%%%%%%%%
%%%%%%%%PAGEBREAK%%%%%%%PAGEBREAK%%%%%%%%%
%%%%%%%%%%%%%%%%%%%%%%%%%%%%%%%%%%%%%%%%%%
%%%%%%%%%%%%%%%%%%%%%%%%%%%%%%%%%%%%%%%%%%
\begin{ekdosis}
  \begin{prose}
    \noindent
tanmadhye punar-ābhyāsa\app{\lem[wit={E,P,U2}, alt={°karaṇān}]{karaṇā\skp{n-ma}}
  \rdg[wit={ceteri}]{karaṇāt}
}\app{\lem[wit={ceteri}, alt={manaḥ}]{\skm{n-ma}naḥ}
  \rdg[wit={B,L}]{punaḥ}}
\app{\lem[wit={ceteri}]{pavanayoḥ}
  \rdg[wit={B,L}]{pavanayo}}
\app{\lem[wit={ceteri}]{saṃcāro}
  \rdg[wit={D}]{saṃcoro}}
\app{\lem[wit={ceteri}]{bhavati}
  \rdg[wit={B,L}]{bhavatī}}/
%-----------------------------
%tayoḥ saṃcārān  madhye granthitrayaṃ truṭyati/  tatroṭanāt        pavano  brahmakamalamadhye pūrṇo bhūtvā tiṣṭhati/  tato vīryastambho bhavati/  puruṣaḥ sadaiva   yuvā      bhavati/ \E
%tayoḥ saṃcārān  madhye graṃthitrayaṃ truṭyati                                                                        tato vīryastaṃbho bhavati   puruṣaḥ saṃdaivaṃ yuve   prabhavati  \P
%tayo  saṃcārān  madhye granthitrayaṃ truṭyatī/  tatroṭanāt        pavano  brahmakamadhye     pūrṇā bhūtvā tiṣṭhati// tato vīryastambho bhavatī// puruṣaḥ sadaiva   yuvai     bhavatī/ \B
%tayoḥ saṃcārān  madhye graṃthitrayaṃ truṭayatī  tatroṭanāt        pavano  brahmakamadhye     pūrṇā bhūtvā tiṣṭhati// tato vīryastaṃbho bhavati   puruṣaḥ sadaiva   yuvaiva   bhavati// \L
%tayoḥ saṃcārān  madhye granthitrayaṃ truṭyati/  tattroṭanāt       pavanaḥ brahmakamalamadhye pūrṇo bhūtvā tiṣṭhati/  tato vīryastambho bhavati/  puruṣaḥ sadaiva   yuvā/e va bhavati// \N1 %truṭyati="zerbrechen"
%tayoḥ saṃcārāt  madhye graṃthitrayaṃ truṭyati// tata troṭanāt     pavanaḥ brahmakamalamadhye pūrṇo bhūtvā tiṣṭhati// tato vīryastambho bhavati// puruṣaḥ sadaiva   yuvaiva   bhavati// \D 
%tayoḥ saṃcārān  madhye granthitrayaṃ ... ..ti/  tata troṭanāt     pavanaḥ brahmakamalamadhye pūrṇo bhūtvā tiṣṭhati/  tato vīryastambho bhavati/  puruṣa  sadaiva   yurvaiva  bhavati// \N2
%tayoḥ saṃccārāt madhye graṃthitrayaṃ trudyati   tatroṭaṇāt        pavanaḥ brahmakamalamadhye pūrṇo bhūtvā tiṣṭhati   tato vīryastaṃbho bhavati/  puruṣaḥ sadaiva   yuvaivaṃ  bhavati \U1
%tayoḥ saṃccārān madhye graṃthitrayaṃ truṭyati// tattroṭaṇāt       pavanaḥ brahmakamalamadhye pūrṇo bhūtvā tiṣṭhati// tato vīryastaṃbho bhavati   puruṣaḥ sadaiva   vaibhavo  bhavati// \U2
%-----------------------------
%Caused by the transition of them both into the center the trinity of knots breaks. There from the breaking of that, the vitalwind after having filled up (the central channel?) resides in the center og the Brahma-lotus. Then virility and strength arises. The person becomes youthful forever. 
%-----------------------------
\app{\lem[wit={ceteri}]{tayoḥ}
  \rdg[wit={B}]{tayo}}
\app{\lem[wit={ceteri},alt={saṃcārān}]{saṃcārā\skp{n-ma}}
  \rdg[wit={D,U1}]{saṃcārāt}
}\skm{n-ma}dhye
granthitrayaṃ
\app{\lem[wit={ceteri}]{truṭyati}
  \rdg[wit={B}]{truṭyatī}
  \rdg[wit={L}]{truṭayatī}
  \rdg[wit={U1}]{trudyati}
  \rdg[wit={N2}]{ti}}/
\app{\lem[wit={N1,U2},alt={°tattroṭanāt}]{tattroṭanā\skp{t-pa}}
  \rdg[wit={B,E,L,U1}]{tatroṭanāt}
  \rdg[wit={D,N2}]{tata troṭanāt}
  \rdg[wit={P}]{\om}}
\app{\lem[wit={ceteri},alt={pavano}]{\skm{t-pa}vano}
  \rdg[wit={ceteri}]{pavanaḥ}}
brahma\app{\lem[wit={ceteri}, alt={°kamala°}]{kamala}
  \rdg[wit={B,L}]{°ka°}
  \rdg[wit={P}]{\om}
}madhye
\app{\lem[wit={ceteri}]{pūrṇo}
  \rdg[wit={B,L}]{pūrṇā}
  \rdg[wit={P}]{\om}}
bhūtvā tiṣṭhati/
tato vīryastambho bhavati/
\app{\lem[wit={ceteri}]{puruṣaḥ}
  \rdg[wit={N2}]{puruṣa}}
\app{\lem[wit={ceteri}]{sadaiva}
  \rdg[wit={P}]{saṃdaivaṃ}}
\app{\lem[wit={D,L}]{yuvaiva}
  \rdg[wit={E}]{yuvā}
  \rdg[wit={P}]{yuve}
  \rdg[wit={B}]{yuvai}
  \rdg[wit={N1}]{yuve va}
  \rdg[wit={N2}]{yurvaiva}
  \rdg[wit={U1}]{yuvaivaṃ}
  \rdg[wit={U2}]{yuvaivaṃ}}
\app{\lem[wit={ceteri}]{bhavati}
  \rdg[wit={B}]{bhavatī}
  \rdg[wit={P}]{prabhavati}}/
%-----------------------------
%paṃcama  udgīryāṇāṃ svādhiṣṭhānaṃ tatra bandhanān      malamūtrayor nāśo   bhavati/  \E
%paṃcamaṃ uḍḍīyāṇāṃ  svādhiṣṭhānaṃ tatra baṃdhadānān    malamūtrayor nāśo   bhavati   \P
%paṃcama  uḍḍiyānāṃ  svādhiṣṭhānaṃ tatra baṃdha dīyate/ malamūtrayor nāśo   bhavatī// \B
%paṃcamaṃ uḍḍiyānāṃ  svādhiṣṭhānaṃ tatra baṃdha dīyate/ mūlamūcayor  nāśo   bhavati// \L 
%paṃcamaṃ udyānaṃ                  tatra baṃdhanāt      malamūtrayor nāśe/o bhavati// \N1 [s.10, verso, z4]
%paṃcamaṃ udyāṇāṃ                  tatra vaṃdhanāt      malamūtrayor nāśo   bhavati// \D
%paṃcam   odyānaṃ                  tatra baṃdhanāt      malamūtrayor nāśo   bhavati/  \N2
%paṃcamaṃ uddyānaṃ                 tatra baṃdhadānāt    malamūtrayor nāśo   bhavati   \U1
%paṃcamaṃ uḍḍīyāṇaṃ  svādhiṣṭhānaṃ tatra badhadānān     malamūtrayor nāśo   bhavati// \U2
%-----------------------------
%The fifth is Udyāna. From performing \textit{bandha} there, urine and faeces disappear.  
%-----------------------------
\note[type=source, labelb=221, lem={udyānaṃ°}]{SSP: pañcame oḍīyāṇādhārayor bandhanān malamūtrasaṃkocanaṃ bhavati ||2.14|| *uḍyānā° etc. in various mss.}
\note[type=source, labelb=222, lem={udyānaṃ°}]{Ysv (PT): pañcamaṃ jaṭharādhāraṃ tadā bandhayati kramāt | mṛtyunā bhaṅgasiddho 'yaṃ [mṛtyunāmāṅgasiddho 'yaṃ (YK)] mṛtyor [mṛtyur (YK)] eva kṣayaṅkaraḥ | anena paścimād ūrddhaṃ [mṛtyunāmāṅgasiddho'yaṃ (YK)] vāyuḥ kuryād viśāladhīḥ | bandho 'yaṃ buddhimanasoḥ pañcamādhārakālajit |}
\app{\lem[wit={ceteri}]{paṃcamaṃ}
  \rdg[wit={B}]{paṃcama}
  \rdg[wit={N2}]{paṃcam}}
\app{\lem[wit={N1,D}]{udyānaṃ}
  \rdg[wit={N2}]{odyānaṃ}
  \rdg[wit={U1}]{uddyānaṃ}
  \rdg[wit={P,U2}]{uḍḍīyāṇāṃ svādhiṣṭhānaṃ}
  \rdg[wit={B,L}]{uḍḍiyānāṃ svādhiṣṭhānaṃ}
  \rdg[wit={E}]{udgīryāṇāṃ svādhiṣṭhānaṃ}} \dd{}
\note[type=philcomm, labelb=223, lem={udyānaṃ}]{Spellings for this component of the yogic body vary dramatically across yogic literature. Since this sentence is clearly based on the SSP and the prevelant variant of the component is *\textit{uḍyānā}° etc., the reading of N\textsubscript{1} seems to be original. B,E,L,P,U\textsubscript{2} add the expression \textit{svādhiṣṭhānaṃ}. Since this reading is absent in the source and parallels it seems to be a later addition.}
tatra
\app{\lem[wit={E}]{bandhanā\skp{n-ma}}
  \rdg[wit={U2}]{badhadānān}
  \rdg[wit={N1,N2}]{baṃdhanāt}
  \rdg[wit={D}]{vaṃdhanāt}
  \rdg[wit={U1}]{baṃdhadānāt}
  \rdg[wit={P}]{baṃdhadānān}
  \rdg[wit={B,L}]{baṃdha dīyate}
}\app{\lem[wit={ceteri},alt={malamūtrayor}]{\skm{n-ma}lamūtrayo\skp{r-nā}}
  \rdg[wit={L}]{mūlamūcayor}}
\skm{r-nā}śo
\app{\lem[wit={ceteri}]{bhavati}
  \rdg[wit={B}]{bhavatī}}/
%-----------------------------
%ṣaṣṭho nābhyādhāraḥ/    \E
%ṣaṣṭho nābhyādhāraḥ   tatra         prāṇavābhyāsād  nāhato   nāraḥ   svayam utpadyate / \P
%ṣaṣṭho nābhyādhāraḥ   tatra         prāṇavābhyāsād  anāhato  nādaḥ// svayam utpadyate// \B
%ṣaṣṭho nābhyādhāraḥ   tatra         prāṇavābhyāsād  anāhato  nādaḥ// svayam utpadyate... \L 
%ṣaṣṭho nābhyādhāraḥ/  tatra         praṇavābhyāsāt  anāhato  nādaḥ   svayam ūtpadyate/  \N1
%ṣaṣṭho nābhyādhāraḥ// tatra         prāṇavābhyāsāt  anāhato  nādaḥ// svayam utpadyate// \D
%ṣaṣṭho nābhyādhāraḥ   tatra         praṇavābhyāsāt  anāhato  tādaḥ   svayaṃ utpadyate/ \N2
%ṣaṣṭho nābhyādhāras   tatra         praṇavābhyāṃsad ānāhato  nadaḥ   svayam utpadyate   \U1
%ṣaṣṭho nābhyādhāre//  tatra         prāṇavābhyāsād  anohato  nādaḥ   svayam utpadyate// \U2
%-----------------------------
%The sixth is the navel-container. From repeated practice of \textit{praṇava}, the unstruck sound arises by itself. 
%-----------------------------
\note[type=source, labelb=224, lem={nābhyādhāraḥ}]{SSP: ṣaṣṭhe nābhyādhāra oṃkāram ekacittenoccārayet | nādalayo bhavati ||2.15||}
\note[type=source, labelb=225, lem={nābhyādhāraḥ}]{Ysv (PT): nābhyādhāro bhavet ṣaṣṭhas [ṣaṣṭhaṃ (YK)] tatra prāṇaṃ samabhyaset | svayam utpadyate nādo nādato muktidantataḥ [muktidaṇḍataḥ (YK)]|}
ṣaṣṭho
\app{\lem[wit={ceteri}]{nābhyādhāraḥ}
  \rdg[wit={U1}]{nābhyādhāras}
  \rdg[wit={U2}]{nābhyādhāre}}\dd{}
\app{\lem[wit={ceteri}]{tatra}
  \rdg[wit={E}]{\om}}
\app{\lem[wit={P,B,L,U2}]{prāṇavābhyāsā\skp{d-a}}
  \rdg[wit={P,B,L,U2}]{prāṇavābhyāsād}
  \rdg[wit={U1}]{prāṇavābhyāṃsad}
  \rdg[wit={E}]{\om}
}\app{\lem[wit={ceteri},alt={°anāhato}]{\skm{d-a}nāhato}
    \rdg[wit={P}]{nāhato}
    \rdg[wit={U1}]{ānāhato}
    \rdg[wit={U2}]{anohato}}
  \app{\lem[wit={ceteri}]{nādaḥ}
    \rdg[wit={P}]{nāraḥ}
    \rdg[wit={N2}]{tādaḥ}}
  \app{\lem[wit={ceteri}]{svaya\skp{m-u}}
    \rdg[wit={N2}]{svayaṃ}
}\app{\lem[wit={ceteri},alt={utpadyate}]{\skm{m-u}tpadyate}
  \rdg[wit={N1}]{ūtpadyate}}/
%-----------------------------
%                             tasmin sthāne prāṇavāyor  nirodhāt            ṣaḍapi kamalāny ūrdhvamukhāni             vikasaṃti// \E                                                                        \E
%saptamo hṛdayarūpadhāraḥ     tasmin sthāne prāṇavāyor  nirodhāt            ṣadapi kamalāny ūrdhvamukhāni             vikasaṃti  \P  %%%7653.jpg 
%                             tasmin sthāne prāṇavāyo   nirodhāt/           ṣaḍapi kamalāny ūrdhvamukhāni             vikasaṃti// \B
%saptamo hṛdayarūpadhāraḥ//   tasmin sthāne prāṇavāyor  nirodhāt            ṣadapi kamalāny ūrdhvamukhāni             vikasaṃti// \L
%saptamo hṛdayarūpa ādhāraḥ   tasmin sthāne prāṇavāyor  nirūṃdhanāt/        ṣadapi kamalāny ūrdhvamukhaṃ              vikasaṃti// \N1
%saptamo hṛdayarūpa ādhāraḥ// tasmin sthāne prāṇavāyor  nir???ūṃ???dhanāt// ṣadapi kamalāny ūrdhvamukhaṃ              vikasaṃti// \D
%saptamo hṛdayarūpādhāraḥ     tasmin sthāne prāṇavāyor  nirūṃdhanāt/        ṣadapi kamalāny ūrdhvemukhaṃ              vikasaṃti// \N2 %%%%%%%%%[S.9, recto, z.4]
%saptamo hṛdayarūpādhāraḥ     tasmin sthāne prāṇavāyor  nirūṃdhanāt         ṣadapi kamalāny ūrusyordha mukhaṃ bhavati vikasaṃti  \U1
%saptamo hṛdayādhāraḥ         tasmin sthāne prāṇavāyor  nirodhāt//          ṣadapi kamalāny ūrddhvamukhāni            vikasaṃti//  \U2
%-----------------------------
%The seventh is the container of the heart-form. 
%-----------------------------
\note[type=source, labelb=226, lem={hṛdayarūpadhāraḥ}]{SSP: saptame hṛdayādhāre prāṇaṃ nirodhayet | kamalavikāso bhavati ||2.16||}
\note[type=source, labelb=227, lem={hṛdayarūpadhāraḥ}]{Ysv (YK): saptamo hṛdayādhāras tasmin vāyunibandhanāt | ūrdhvavaktrāṇi [ūrddhaktrāṇi (YK)] padmāni vikasanti mahān bhavet ||26||}
\app{\lem[wit={ceteri}]{saptamo}
  \rdg[wit={E,B}]{\om}}
\app{\lem[wit={ceteri}]{hṛdaya}
  \rdg[wit={U2}]{hṛdayā°}
}\app{\lem[wit={N2,U1},alt={°rūpādhāraḥ}]{rūpādhāraḥ}
  \rdg[wit={L}]{°rūpadhāraḥ}
  \rdg[wit={D,N1}]{rūpa ādhāraḥ}
  \rdg[wit={U2}]{°dhāraḥ}
  \rdg[wit={E,P}]{\om}}\dd{}
tasmin-sthāne
\app{\lem[wit={ceteri}]{prāṇavāyo\skp{r-ni}}
  \rdg[wit={B}]{prāṇavāyo}
}\app{\lem[wit={ceteri},alt={nirodhāt}]{skm{r-ni}rodhā\skp{t-ṣa}}
  \rdg[wit={D,N1,N2,U1}]{nirūṃdhanāt}
}\app{\lem[wit={ceteri},alt={ṣad api}]{\skm{t-ṣa}dapi}
  \rdg[wit={B}]{ṣaḍapi}}   
kamalā\skp{ny-ū}\app{\lem[wit={ceteri},alt={ūrdhvamukhāni}]{\skm{ny-ū}rdhvamukhāni}
  \rdg[wit={D,N1,N2}]{ūrdhvamukhaṃ}
  \rdg[wit={U1}]{ūrusyordha mukhaṃ bhavati}}
vikasaṃti/
%-----------------------------
%aṣṭamaṃ kaṇṭhādhāraḥ/  tatra  jālaṃdharo bandho dīyate/  tasmin satīḍāyāṃ   piṃgalāyāṃ pavanaḥ sthiro bhavati/  \E %%[p.43]
%aṣṭamaḥ kaṃṭhādhāraḥ   tatra  jālaṃdharo baṃdho dīyate   tasmin satīḍāyāṃ   piṃgalāyāṃ pavanaḥ sthiro bhavataḥ  \P
%aṣṭame  kaṇṭhādhāraḥ/  tatra  jalaṃ baṃdho      dīyate   tasmin satīyāṃ     piṃgalāyāṃ pavanaḥ sthiro bhavatī/ \B  %%%%DSCN7166.jpg Z.3
%aṣṭame  kaṇṭhādhāraḥ/  tatra  jalaṃ baṃdho      dīyate   tasmin satīyāṃ     piṃgalāyāṃ pavanaḥ sthiro bhavatī// \L
%aṣṭamaḥ kaṇṭhādhāraḥ/  tatra  jālaṃdharo baṃdho dīyate/  tasmin sati iḍāyāṃ piṃgalāyāṃ pavanaḥ sthiro bhavati/ \N1
%aṣṭamaḥ kaṃṭhādhāraḥ// tatraḥ jālaṃdharo baṃdho dīyate// tasmin sati iḍāyāṃ piṃgalāyāṃ pavanasthiro bhavati// \D  %%%p.12 verso
%aṣṭama--kaṇṭhādhāraḥ/  tatra  jālaṃdharabandho  dīyate// tasmin satiśadāyāṃ piṃgalāyāṃ pavanaḥ sthiro bhavati/ \N2
%aṣṭamaḥ kaṇṭhādhāraḥ   tatra  jālaṃdharo bandho dīpyate  tasmin sati iḍāyāṃ piṃgalāyāṃ pavanaḥ sthiro bhavati \U1
%aṣṭamaḥ kaṇṭhādhāraḥ   tatra  jālaṃdharo bandho dīyate   tasmin sati piḍāyā piṃgalāyāṃ pavanaḥ sthiro bhavati// \U2
%-----------------------------
%The throat-support is the eighth. There the contraction of Jālaṃdhara is produced. While abiding therein the vitalwind in the Iḍā and Piṅgalā channel becomes stable.   
%-----------------------------
\note[type=source, labelb=228, lem={kaṇṭhādhāraḥ}]{SSP: aṣṭame kaṇṭhādhāre kaṇṭhamūlaṃ cibukena nirodhayet | iḍāpiṅgalayor vāyuḥ sthiro bhavati ||2.17||}
\note[type=source, labelb=229, lem={kaṇṭhādhāraḥ}]{Ysv (PT=YK):kaṇṭhādhāro 'ṣṭamas tatra kaṇṭhasaṅkocalakṣaṇaḥ | jālandharākhyo bandhaḥ syāt tasmin sati marud dṛḍhaḥ ||27||}
\app{\lem[wit={P,N1,D,U1,U2}]{aṣṭamaḥ}
  \rdg[wit={B,L}]{aṣṭame}
  \rdg[wit={N2}]{aṣṭama°}}
kaṇṭhādhāraḥ/
\app{\lem[wit={ceteri}]{tatra}
  \rdg[wit={D}]{tatraḥ}}
\app{\lem[wit={ceteri}]{jālaṃdharo}
  \rdg[wit={N2}]{jālaṃdhara°}
  \rdg[wit={B,L}]{jalaṃ}}
bandho
\app{\lem[wit={ceteri}]{dīyate}
  \rdg[wit={U1}]{dīpyate}}/
tasmin \app{\lem[wit={E,P}]{satīḍāyāṃ}
  \rdg[wit={B,L}]{satīyāṃ}
  \rdg[wit={N1,D,U1,U2}]{sati iḍāyāṃ}
  \rdg[wit={N2}]{satiśadāyāṃ}}
piṅgalāyāṃ
\app{\lem[wit={ceteri}]{pavanaḥ}
  \rdg[wit={D}]{pavana°}}
sthiro
\app{\lem[wit={ceteri}]{bhavati}
  \rdg[wit={B,L}]{bhavatī}}/
\end{prose}
\end{ekdosis}
\ekdpb*{}
%%%%%%%%%%%%%%%%%%%%%%%%%%%%%%%%%%%%%%%%%%
%%%%%%%%%%%%%%%%%%%%%%%%%%%%%%%%%%%%%%%%%%
%%%%%%%%PAGEBREAK%%%%%%%PAGEBREAK%%%%%%%%%
%%%%%%%%%%%%%%%%%%%%%%%%%%%%%%%%%%%%%%%%%%
%%%%%%%%%%%%%%%%PAGEBREAK%%%%%%%%%%%%%%%%%
%%%%%%%%%%%%%%%%%%%%%%%%%%%%%%%%%%%%%%%%%%
%%%%%%%%PAGEBREAK%%%%%%%PAGEBREAK%%%%%%%%%
%%%%%%%%%%%%%%%%%%%%%%%%%%%%%%%%%%%%%%%%%%
%%%%%%%%%%%%%%%%%%%%%%%%%%%%%%%%%%%%%%%%%%
%%%%%%%%%%%%%%%%%%%%%%%%%%%%%%%%%%%%%%%%%%
%%%%%%%%%%%%%%%%%%%%%%%%%%%%%%%%%%%%%%%%%%
%%%%%%%%PAGEBREAK%%%%%%%PAGEBREAK%%%%%%%%%
%%%%%%%%%%%%%%%%%%%%%%%%%%%%%%%%%%%%%%%%%%
%%%%%%%%%%%%%%%%PAGEBREAK%%%%%%%%%%%%%%%%%
%%%%%%%%%%%%%%%%%%%%%%%%%%%%%%%%%%%%%%%%%%
%%%%%%%%PAGEBREAK%%%%%%%PAGEBREAK%%%%%%%%%
%%%%%%%%%%%%%%%%%%%%%%%%%%%%%%%%%%%%%%%%%%
%%%%%%%%%%%%%%%%%%%%%%%%%%%%%%%%%%%%%%%%%%
%%%%%%%%%%%%%%%%%%%%%%%%%%%%%%%%%%%%%%%%%%
%%%%%%%%%%%%%%%%%%%%%%%%%%%%%%%%%%%%%%%%%%
%%%%%%%%PAGEBREAK%%%%%%%PAGEBREAK%%%%%%%%%
%%%%%%%%%%%%%%%%%%%%%%%%%%%%%%%%%%%%%%%%%%
%%%%%%%%%%%%%%%%PAGEBREAK%%%%%%%%%%%%%%%%%
%%%%%%%%%%%%%%%%%%%%%%%%%%%%%%%%%%%%%%%%%%
%%%%%%%%PAGEBREAK%%%%%%%PAGEBREAK%%%%%%%%%
%%%%%%%%%%%%%%%%%%%%%%%%%%%%%%%%%%%%%%%%%%
%%%%%%%%%%%%%%%%%%%%%%%%%%%%%%%%%%%%%%%%%%
\begin{ekdosis}
  \begin{prose}
    \noindent
%-----------------------------
%navamo ghaṃṭikādhāraḥ/   tatra jihvāgraṃ   lagnaṃ bhavati/    tato mṛtakalāyā     amṛtaṃ sravati/  tadamṛtapānāt             śarīramadhye rogasaṃcāro na bhavati/ \E
%navamo ghaṭikādhāraḥ     tatra jihvāgraṃ   lagnaṃ bhavati     tato mṛtakakalāyā   amṛta  sravati   tadamṛtapānāc            charīramadhye rogasaṃcāro na bhavati  \P
%navo   ghaṃṭikādhāraḥ//  tatra jihvāgraṃ   lagnaṃ bhavatī/    tato mṛtakalāyā     amṛtaṃ sravati/  tadamṛtakalāyāṃ amṛtapānīcharīramadhye rogasaṃcāro bhavatī/ \B
%navamo ghaṃṭādhāraḥ//    tatra jihvāgraṃ   lagnaṃ bhavati//   tato mṛtakalāyāṃ                        amṛtapānā-------------charīramadhye rogasaṃcāro bhavati// \L %eyeskip in line.. :(
%navamo ghaṃṭikādhāraḥ/   tatra jihvāgraṃ   lagnaṃ bhavati/    tato mṛtakalāyā     amṛtaṃ sravati/  tadamṛtapānāt             śarīramadhye rogasaṃcāro na bhavati/ \N1
%navamo ghaṃṭikādhāraḥ//  tatra jihvāyāgraṃ lagnaṃ bhavati//   tataḥ amṛtakalāyāḥ  amṛtaṃ sravati// tadamṛtapānāc           charīramadhye  rogasaṃcāro na bhavati// \D
%navamo ghaṃṭikādhāraḥ/   tatra jihvāgraṃ   lagnaṃ bhavati/    tato mṛtakalāyā     amṛtaṃ sravati/  tadamṛtapānāt             śarīramadhye rogasaṃcāro na bhavati/ \N2
%navamo ghaṃṭikādhāras    tatra juhvāyāṃ    lagnaṃ bhavati vā  tataḥ amṛtakalāyāḥ  amṛtaṃ sravati   tadamṛtapānāt            charīramadhye rogasaṃcāro na bhavati \U1
%navamo ghaṃṭikādhāraḥ    tatra jihvāgraṃ   lagnaṃ bhavati//   tato mṛtakalāyāḥ    amṛtaṃ sravati// tadamṛtapānā             charīramadhye rogasaṃcāro na bhavati// \U2
%-----------------------------
%The ninth is the container of the uvula. There the tip of the tongue becomes attached [to the uvula]. Then the nectar of immortality flows from the immortality-digit. From drinking the nectar of immortality diseases do not spread in the body. 
%-----------------------------
\note[type=source, labelb=230, lem={ghaṃṭikādhāraḥ}]{SSP: navame ghaṇṭikādhāre jihvāgraṃ dhārayet | amṛtakalā sravati ||2.18||}
\note[type=source, labelb=231, lem={ghaṃṭikādhāraḥ}]{Ysv (PT): navamo ghaṇṭikādhāras tatra jihvāgramagrataḥ [jihvāgrataḥ kṛte (YK)] | sampivatyamṛtaṃ tasmād yogajinmṛtyujitparaḥ |}
\app{\lem[wit={ceteri}]{navamo}
  \rdg[wit={B}]{navo}}
\app{\lem[wit={ceteri},alt={ghaṃṭikā°}]{ghaṃṭikā}
  \rdg[wit={P}]{ghaṭikā°}
  \rdg[wit={L}]{ghaṃṭā°}
}\app{\lem[wit={ceteri},alt={°dhāraḥ}]{dhāraḥ}
  \rdg[wit={U1}]{dhāras}}/
tatra
\app{\lem[wit={ceteri}]{jihvāgraṃ}
  \rdg[wit={D}]{jihvāyāgraṃ}
  \rdg[wit={U1}]{juhvāyāṃ}}
lagnaṃ
\app{\lem[wit={ceteri}]{bhavati}
  \rdg[wit={B}]{bhavatī}
  \rdg[wit={U1}]{bhavati vā}}
\app{\lem[wit={ceteri}]{tato}
  \rdg[wit={N1,U1}]{tataḥ}}
\app{\lem[wit={E,B,N1,N2}]{'mṛtakalāyā}
  \rdg[wit={P}]{mṛtakakalāyā}
  \rdg[wit={L}]{mṛtakalāyāṃ}
  \rdg[wit={D,U1}]{amṛtakalāyāḥ}}
\app{\lem[wit={ceteri}]{amṛtaṃ}
  \rdg[wit={P}]{amṛta}
  \rdg[wit={L}]{\om}}
\app{\lem[wit={ceteri}]{sravati}
  \rdg[wit={L}]{\om}}/
\app{\lem[wit={P,D},alt={tadamṛtapānāc}]{tadamṛtapānā\skp{c-cha}}
  \rdg[wit={E,N1,N2,U1}]{tadamṛtapānāt}
  \rdg[wit={B}]{tadamṛtakalāyāṃ amṛtapānī°}
  \rdg[wit={L}]{amṛtapānā}
  \rdg[wit={U2}]{tadamṛtapānā}
}\app{\lem[wit={ceteri},alt={charīra°}]{\skm{c-cha}rīra}
  \rdg[wit={E,N1,N2}]{śarīra°}
}madhye 
rogasaṃcāro
\app{\lem[wit={ceteri}]{na}
  \rdg[wit={B,L}]{\om}}
\app{\lem[wit={ceteri}]{bhavati}
  \rdg[wit={B}]{bhavatī}}/  
%-----------------------------
%daśamaṃ tālvādhāraḥ/  tanmadhye    vānaṃ dollahanaṃ      ca kṛtvā              laṃbikāpraveśe sati    tāluni magnā jihvā tiṣṭhati/ \E
%daśamas tālvādhāraḥ   tanmadhye  cālanaṃ dohanaṃ         ca kratvā             laṃbikāpraveśe śe sati tālumagnā    jihvā tiṣṭhati  \P %%%7654.jpg
%daśamaṃ stālvādhāraḥ/ tanmadhye  cālanaṃ dohanaṃ         ca kratvā             laṃbikāpraveśe sati    tālumagnā    jihvā tiṣṭhati/ \B
%daśamas tālvādhāraḥ// tanmadhye  cālanaṃ dohanaṃ         ca kṛtvā              laṃbikāpraveśe sati    tālumagnā    jihvā tiṣṭhati ... \L
%daśama  tālvādhāraḥ// tanmadhye  cānanaṃ dohanaṃ         ca kṛtvā              laṃbikāpraveśe grati   tāluni magnā jihvā tiṣṭhati/ \N1
%daśamas tālvādhāraḥ   tanmadhye  cānanaṃ dohanaṃ         ca kṛtvā              laṃbikāpravese grati   tāluni magnā jihvā tiṣṭhati// \D
%daśama  tālvādhāraḥ   tanmadhye  cālanaṃ dohanaṃ         ca kṛtvā              laṃbikāpraveśe grati   tālūni magnā                    \N2
%daśamas tālvādhāraḥ  staṃnmadhye cālanaṃ dohanaṃ         ca sva/sca? kṛtvā cālaṃ vikā praveśe sati    tālūni lagnā juhvā tiṣṭhati \U1 %%%284.jpg
%daśamas tālvādhāraḥ   tanmadhye  cālanaṃ dohanaṃ chedanaṃ ca kṛtvā             laṃbikāpraveśe sati    tāluni magnā jihvā tiṣṭhati// \U2 %%416.jpg
%-----------------------------
%The tenth is the container of the palate. After the moving and milking has been done therein while abiding at the door of the uvula, the tongue resides inserted within the palate.  
%-----------------------------
\note[type=source, labelb=232, lem={tālvādhāraḥ}]{SSP: daśame tālvādhāre tālvantar garbhe lambikāṃ cālanadohanābhyāṃ dīrghīkṛtvā viparītena praveśayet | kāṣṭhībhavati ||2.19 ||}
\note[type=source, labelb=233, lem={tālvādhāraḥ}]{Ysv (PT): daśamas tālukādhāras tatra jihvāgrataḥ kṛte | calane dohane caiva jihvā jaḍati lambitā | nāsikāprāptajihveyaṃ tālulagnā bhavet tataḥ [jāyeta lambitam (YK)] |}
\app{\lem[wit={ceteri},alt={daśamas}]{daśama\skp{s-tā}}
  \rdg[wit={E}]{daśamaṃ}
  \rdg[wit={B}]{daśamaṃs}
  \rdg[wit={N1,N2}]{daśama}
}\skm{s-tā}lvādhāraḥ/
\app{\lem[wit={ceteri}]{tanmadhye}
  \rdg[wit={U1}]{staṃnmadhye}}
\app{\lem[wit={ceteri}]{cālanaṃ}
  \rdg[wit={D}]{cānanaṃ}
  \rdg[wit={E}]{vānaṃ}}
\app{\lem[wit={ceteri}]{dohanaṃ}
  \rdg[wit={E}]{dollahanaṃ}
  \rdg[wit={U2}]{dohanaṃ chedanaṃ}}
ca \app{\lem[wit={ceteri}]{kṛtvā}
  \rdg[wit={B,L}]{kratvā}
  \rdg[wit={U1}]{sva kṛtvā}}
\app{\lem[wit={ceteri}]{laṃbikā}
  \rdg[wit={U1}]{cālaṃ vikā}
}praveśe
\app{\lem[wit={ceteri}]{sati}
  \rdg[wit={P}]{śe sati}
  \rdg[wit={D,N1,N2}]{grati}}
\app{\lem[wit={ceteri}]{tālunimagnā}
  \rdg[wit={N2,U1,U2}]{tālūnimagnā}
  \rdg[wit={B,P,L}]{tālumagnā}}
\app{\lem[wit={ceteri}]{jihvā}
  \rdg[wit={U1}]{juhvā}
  \rdg[wit={N2}]{\om}}
\app{\lem[wit={ceteri}]{tiṣṭhati}
  \rdg[wit={N2}]{\om}}/
%-----------------------------
%ekādaśo           jihvādhāraḥ/  tasmin   jihvāgreṇa manthanaṃ kriyate   tasmin  kṛte   timadhuraṃ  pānīyaṃ sravati/  tadā                            ca kavitva------cchandonāṭakādiviṣayajñānam utpadyate/ \E
%ekādaśo jihvātale jihvādhāraḥ   tasmin   jihvāgreṇa mathanaṃ  kriyate   tasmin  kṛte   timadhuraṃ  pānīyaṃ sravati   tathā                           ca kavitva------chaṃdonāṭakādiviṣayajñānam  utpadyate  \P
%ekādaśo jihvātale jihvādhāraḥ// tasmin   jihvāgreṇa manthanaṃ kṛtvā//   tasmiṃ  kṛte satimadhuraṃ  pānīyaṃ sravatī// tathā                              kvacitva-----cchaṃdonāṭakādiviṣayapānam  utpadyaṃte/ \B
%ekādaśo jihvātale jihvādhāraḥ// tasmin   jihvāgreṇa mathanaṃ  kṛtvā//   tasmiṃ  kṛte satimadhuraṃ  pānīyaṃ sravati// tathā                              kvacitva-----chaṃdonāṭakādiviṣayajñānam  utpadyate// \L
%ekādaśo           jihvādhāraḥ/  tasmin   jihvāgreṇa manthanaṃ kriyate/  tasmin  kṛte atimadhuraṃ   pānīyaṃ sravati/  tathā                           ca kavitva--gītacchaṃdanāṭakādiviṣaye jñānam utpadyate/ \N1
%ekādaśo jihvātale jihvādhāraḥ// tasmin   jihvāgreṇa mathanaṃ  kriyate// tasmiṃ  kṛte satimadhuraṃ  pānīyaṃ sravati// tathā                           ca kvacitta-----chaṃdanāṭakādiviṣayajñānam   utpadyate \D 
%                                         jihvāgreṇa manthanaṃ kriyate// tasmin  kṛte atimadhuraṃ   pānīyaṃ sravati// kaminnāsikā phatkāravat// tathā ca kavitvagīta--chaṃdanāṭakādiviṣaye jñānam  utpadyate/ \N2
%ekādaśā jihvātale jihvādhāraḥ  tasmin na jihvāgreṇa manthanaṃ kriyate   tasminn kṛte  timadhuraṃ   pānīyaṃ sravati   tathā                           ca kavitvagīta--chaṃdavacchaṃdanāḍīviṣayaṃ jñānānam utpadyate \U1
%ekādaśo jihvātale jihvādhāraḥ   tasmin   jihvāgreṇa manthanaṃ kriyate// tasminn kṛte 'timadhuraṃ   pānīyaṃ sravati// tathā                           ca kavitvaṃ     chaṃdonāṭakādiviṣayajñānam  utpadyate// \U2
%-----------------------------
%The eleventh is the tongue-container at the surface of the tontue. Within it the tip of the tongue has to be churned. While doing a sweet drink flows out. And in that manner the knowledge of areas like poetry, singing, metric and dance is generated. 
%----------------------------
\note[type=source, labelb=234, lem={jihvādhāraḥ}]{SSP: ekādaśe atha jihvādhāre tatra jihvāgraṃ dhārayet | sarvaroganāśo bhavati ||2.20||}
\note[type=source, labelb=235, lem={jihvādhāraḥ}]{Ysv (PT): ekādaśī [ekādaśo (YK)] bhavej jihvā talajādhāra īśvari | jihvāgramathane tasmin pānīyaṃ madhuraṃ bhavet | tatpīteṣu kavir gītijyotiś [gītir (YK)] chandovidāṃ [chandovidur (YK)] varaḥ |}
\app{\lem[wit={ceteri}]{ekādaśo}
  \rdg[wit={N2}]{\om}}
\app{\lem[wit={ceteri}]{jihvātale}
  \rdg[wit={E,N1,N2}]{\om}}
\app{\lem[wit={ceteri}]{jihvādhāraḥ}
  \rdg[wit={N2}]{\om}}/
\app{\lem[wit={ceteri}]{tasmin}
  \rdg[wit={U1}]{tasmin na}
  \rdg[wit={N2}]{\om}}
jihvāgreṇa
\app{\lem[wit={ceteri}]{manthanaṃ}
  \rdg[wit={D,L,P}]{mathanaṃ}}
\app{\lem[wit={ceteri}]{kriyate}
  \rdg[wit={B,L}]{kṛtvā}}/
tasmin-kṛte
\app{\lem[wit={ceteri}]{'timadhuraṃ}
  \rdg[wit={N1,N2}]{atimadhuraṃ}
  \rdg[wit={B,L,D}]{satimadhuraṃ}}
pānīyaṃ
\app{\lem[wit={ceteri}]{sravati}
  \rdg[wit={B}]{sravatī}}/ 
\app{\lem[wit={ceteri}]{tathā}
  \rdg[wit={E}]{tadā}
  \rdg[wit={N2}]{kamin nāsikā phatkāravat || tathā}}
\app{\lem[wit={ceteri}]{ca}
  \rdg[wit={B,L}]{\om}}
\app{\lem[wit={ceteri},alt={kavitva°}]{kavitva}
  \rdg[wit={B,L}]{kvacitva°}
  \rdg[wit={D}]{kvacitta°}
  \rdg[wit={U2}]{kavitvaṃ}
}\app{\lem[wit={N1,N2,U1},alt={°gīta°}]{gīta}
  \rdg[wit={ceteri}]{\om}
}\app{\lem[wit={E,P,B,L,U2},alt={°chando°}]{chando}
  \rdg[wit={U1}]{°chaṃdavacchaṃda°}
  \rdg[wit={ceteri}]{°chaṃda°}
}\app{\lem[wit={ceteri},alt={°nāṭakādi°}]{nāṭakādi}
  \rdg[wit={U1}]{°nāḍī°}} 
\app{\lem[wit={B,E,L,P,D,U2},alt={°viṣaya°}]{viṣaya}
  \rdg[wit={N1,N2}]{°viṣaye}
  \rdg[wit={U1}]{viṣayaṃ}}
\app{\lem[wit={ceteri},alt={jñānam}]{jñāna\skp{m-u}}
  \rdg[wit={U1}]{jñānānam}
}\app{\lem[wit={ceteri},alt={utpadyate}]{\skm{m-u}tpadyate}
  \rdg[wit={B}]{utpadyaṃte}}/
%----------------------------
%tadupari dvādaśadantayo   madhye   dantādhāraḥ/  tasmin sthāne jihvāyā  agraṃ  ghaṭīmātraṃ                    balātkāreṇa  sthāpyate/  tasmin  sati sādhakasya samagrā rogā naśyanti// \E %%%[p.44]
%tadupari dvādaśo daṃtayor madhye   daṃtādhāraḥ   tasmin sthāne jihvāyā  agraṃ  ghaṭīmātram ārghaghaṭīmātraṃ   bālātkāreṇa  sthāpyate   tasmin  sati sādhakasya samagrā rogā naśyaṃti \P
%tadupari dvādaśo daṃtayor madhye// daṃtādhāraḥ// tasmin sthāne jihvāyā  'agnaṃ ghaṭīmātram ārghaghaṭimātraṃ   bālākāreṇa   sthāpyate// tasmiṃ       sādhakasya samagrā rogā naśyaṃtī// \B
%tadupari dvādaśo daṃtayor madhye// daṃtādhāraḥ   tasmin sthāne jihvāyā  agnaṃ  ghaṭīmātram ārddhaghaṭimātraṃ  bālākāreṇa   sthāpyate// tasmiṃ       sādhakasya samagrā rogā naśyaṃti... \L
%tadupari dvādaśayor       madhye   daṃtādhāraḥ/  tasmin sthāne jihvāyā  agraṃ  ghaṭīmātraṃ arddhaghaṭimātraṃ  balātkāreṇa  sthāpyate// tasmin  sati sādhakasya samagrā rogā naśyaṃti// \N1
%tadupari dvādaśayor       madhye   daṃtādhāraḥ// tasmin sthāne jihvāyā  agraṃ  ghaṭīmātraṃ arddhaghaṭimātraṃ  balātkāreṇa  sthāpyate// tasmin  sati sādhakasya samagrā rogā naśyaṃti// \D
%tadupari dvādaśayor       madhye   daṃtādhāraḥ// tasmin sthāne jihvāyā   graṃ  ghaṭīmātraṃ arddhaghaṭimātraṃ  balātkāreṇa  sthāpyate// tasmin  sati sādhakasya samagrā rogā naśyanti \N2
%tadupari dvādaśo daṃtayor madhye   daṃtādhāraḥ   tasmin sthāne jihvāyāṃ agraṃ  ghaṭīmātram ārdhaghaṭikāmātraṃ bālātkāreṇa  sthāpyate   tasminn sati sādhakasya samagra rogā naśyaṃti \U1
%tadupari dvādaśor daṃtayo madhye   daṃtādhāraḥ   tasmin sthāne jihvāyā  agraṃ  ghaṭīmātram ārghaghaṭīmātraṃ   bālātkāreṇa  sthāpyate// tasmin  sati sādhakasya samagrā rogā naśyaṃti// \U2
%-----------------------------
%On top thereof is the twelfth, being the teeth-support, which is situated inbetween the teeth. At this place the tip of the tongue is to be positioned with force for the duration of one and a half \textit{ghāṭī}s (24+12 = 36 minutes). Abiding therein the diseases of the practitioner will entirely disappear!
%----------------------------
\note[type=source, labelb=236, lem={dantādhāraḥ}]{SSP: dvādaśe bhrūmadhyādhāre tatra candramaṇḍalaṃ dhyāyet śītalatāṃ yāti ||2.21||}
\note[type=source, labelb=237, lem={dantādhāraḥ}]{Ysv (PT): dantādhāro [dvandvādhāro (PT)] dvādaśeti sarvarogakṣayaṅkaraḥ [sarvarogaḥ (YK)] | dhārayed dantayor madhye jihvāgrañ ca balād api | dhṛtvārddhaghaṭikāmātraṃ sarvarogan [sarvarogāṃs (YK)] tu nāśayet |}
tadupari
\app{\lem[wit={P,B,L,U1},alt={dvādaśo daṃtayor}]{dvādaśo daṃtayo\skp{r-ma}}
  \rdg[wit={E}]{dvādaśadantayo}
  \rdg[wit={U2}]{dvādaśor daṃtayo}
  \rdg[wit={D,N1,N2}]{dvādaśayor}
}\skm{r-ma}dhye
dantādhāraḥ/
tasmin sthāne
\app{\lem[wit={ceteri}]{jihvāyā}
  \rdg[wit={U1}]{jihvāyāṃ}}
\app{\lem[wit={ceteri}]{agraṃ}
  \rdg[wit={B,L}]{agnaṃ}
  \rdg[wit={N2}]{graṃ}}
\app{\lem[wit={ceteri},alt={ghaṭīmātraṃ}]{ghaṭīmātra\skp{m-a}}
  \rdg[wit={D,N1,N2}]{ghaṭīmātraṃ}
}\app{\lem[type=emendation, resp=egoscr,alt={ardhagaṭīmātraṃ}]{\skm{m-a}rdhagaṭīmātraṃ}
  \rdg[wit={D,N1,N2}]{\korr arddhaghaṭimātraṃ}
  \rdg[wit={U1}]{ārdhaghaṭikāmātraṃ}
  \rdg[wit={P,U2}]{ārghaghaṭīmātraṃ}
  \rdg[wit={B}]{ārghaghaṭimātraṃ}
  \rdg[wit={L}]{ārddhaghaṭimātraṃ}
  \rdg[wit={E}]{\om}}
\app{\lem[wit={E,D,N1,N2}]{balātkāreṇa}
  \rdg[wit={P,U1,U2}]{bālātkāreṇa}
  \rdg[wit={B,L}]{bālākāreṇa}}
sthāpyate/
\app{\lem[wit={ceteri}]{tasmin}
  \rdg[wit={B,L}]{tasmiṃ}}
\app{\lem[wit={ceteri}]{sati}
  \rdg[wit={B,L}]{\om}}
sādhakasya samagrā rogā
\app{\lem[wit={ceteri}]{naśyanti}
  \rdg[wit={B}]{naśyaṃtī}}/
%----------------------------
%trayodaśo nāsikāgrādhāraḥ/ tasmin lakṣye kṛte sati manaḥ sthiraṃ bhavati/ \E
%trayodaśo nāsikāgrādhāraḥ  tasmiṃ lakṣye kṛte sati manaḥ sthiraṃ bhavati \P
%trayodaso nāsikādhāraḥ/    tasmin ḍraṣṭe kṛte      minasthire    bhavati/ \B
%trayodaso nāśikādhāraḥ     tasmin ḍraṣṭe kṛte      manaḥ sthiro  bhavati/ \L
%trayodaśo nāsikādhāraḥ/    tasmin lakṣe  kṛte sati manasthiraṃ   bhavati/ \N1
%trayodaśo nāsikādhāraḥ//   tasmin lakṣe  kṛte sati manasthiraṃ   bhavati \D
%trayodaśo nāsikādhāraḥ/    tasmin lakṣe  kṛte sati manasthiraṃ   bhavati/ \N2
%trayodaśo nāsikādhāraḥ     tasmiṃ lakṣye kṛte sati manasthiraṃ   bhavati \U1
%trayodaśo nāsikādhāraḥ     tasmil lakṣe  kṛte sati manasthiraṃ   bhavati// \U2
%-----------------------------
%The thirteenth is the nose-container. While making it into the fixation object the mind becomes stable. 
%----------------------------
\note[type=source, labelb=238, lem={nāsikādhāraḥ}]{SSP: trayodaśe nāsādhāre tasyāgraṃ lakṣayet manaḥ sthiraṃ bhavati ||2.22||}
\note[type=source, labelb=239, lem={nāsikādhāraḥ}]{Ysv (PT): nāsādhāras tato [tataḥ (YK)] jñeyo nāsālakṣas trayodaśaḥ [trayodaśa (YK)]| manaḥsthirakaro yas tu [sthiraṃ karoty eva (YK)] vāyusthirakaro [vāyuḥ (YK)] mahān |}
\app{\lem[wit={ceteri}]{ nāśikādhāraḥ}
  \rdg[wit={E,P}]{nāsikāgrādhāraḥ}}/
\app{\lem[type=emendation, resp=egoscr]{tasmil-lakṣye}
  \rdg[wit={U2}]{\korr tasmil lakṣe}
  \rdg[wit={E,P,U1}]{tasmiṃ lakṣye}
  \rdg[wit={D,N1,N2}]{tasmin lakṣe}
  \rdg[wit={B,L}]{tasmin ḍraṣṭe}}
kṛte
\app{\lem[wit={ceteri}]{sati}
  \rdg[wit={B,L}]{\om}}
\app{\lem[wit={E,P}]{manaḥ sthiraṃ}
  \rdg[wit={B}]{minasthire}
  \rdg[wit={L}]{manaḥ sthiro}
  \rdg[wit={ceteri}]{manasthiraṃ}}
bhavati/
\end{prose}
\end{ekdosis}
\ekdpb*{}
%%%%%%%%%%%%%%%%%%%%%%%%%%%%%%%%%%%%%%%%%%
%%%%%%%%%%%%%%%%%%%%%%%%%%%%%%%%%%%%%%%%%%
%%%%%%%%PAGEBREAK%%%%%%%PAGEBREAK%%%%%%%%%
%%%%%%%%%%%%%%%%%%%%%%%%%%%%%%%%%%%%%%%%%%
%%%%%%%%%%%%%%%%PAGEBREAK%%%%%%%%%%%%%%%%%
%%%%%%%%%%%%%%%%%%%%%%%%%%%%%%%%%%%%%%%%%%
%%%%%%%%PAGEBREAK%%%%%%%PAGEBREAK%%%%%%%%%
%%%%%%%%%%%%%%%%%%%%%%%%%%%%%%%%%%%%%%%%%%
%%%%%%%%%%%%%%%%%%%%%%%%%%%%%%%%%%%%%%%%%%
%%%%%%%%%%%%%%%%%%%%%%%%%%%%%%%%%%%%%%%%%%
%%%%%%%%%%%%%%%%%%%%%%%%%%%%%%%%%%%%%%%%%%
%%%%%%%%PAGEBREAK%%%%%%%PAGEBREAK%%%%%%%%%
%%%%%%%%%%%%%%%%%%%%%%%%%%%%%%%%%%%%%%%%%%
%%%%%%%%%%%%%%%%PAGEBREAK%%%%%%%%%%%%%%%%%
%%%%%%%%%%%%%%%%%%%%%%%%%%%%%%%%%%%%%%%%%%
%%%%%%%%PAGEBREAK%%%%%%%PAGEBREAK%%%%%%%%%
%%%%%%%%%%%%%%%%%%%%%%%%%%%%%%%%%%%%%%%%%%
%%%%%%%%%%%%%%%%%%%%%%%%%%%%%%%%%%%%%%%%%%
%%%%%%%%%%%%%%%%%%%%%%%%%%%%%%%%%%%%%%%%%%
%%%%%%%%%%%%%%%%%%%%%%%%%%%%%%%%%%%%%%%%%%
%%%%%%%%PAGEBREAK%%%%%%%PAGEBREAK%%%%%%%%%
%%%%%%%%%%%%%%%%%%%%%%%%%%%%%%%%%%%%%%%%%%
%%%%%%%%%%%%%%%%PAGEBREAK%%%%%%%%%%%%%%%%%
%%%%%%%%%%%%%%%%%%%%%%%%%%%%%%%%%%%%%%%%%%
%%%%%%%%PAGEBREAK%%%%%%%PAGEBREAK%%%%%%%%%
%%%%%%%%%%%%%%%%%%%%%%%%%%%%%%%%%%%%%%%%%%
%%%%%%%%%%%%%%%%%%%%%%%%%%%%%%%%%%%%%%%%%%
\begin{ekdosis}
  \begin{prose}
    \noindent
%----------------------------
%caturdaśo nāsāmūlādhāraḥ/         tasmin dṛṣṭeḥ            sthairyakāraṇāt   ṣaṣṭhe māsi svīyan tejaḥ pratyakṣaṃ bhavati/  tejasaḥ pratyakṣatve pārthivaṃ sakalaṃ bandhanaṃ tuṭyati/   \E
%caturdaśo nāsāmūlādhāro           tasmin dṛṣṭeḥ            sthairyakāraṇāt   ṣaṣṭhe māsi svīyaṃ tejaḥ pratyakṣaṃ bhavati   tejasaḥ pratyakṣatve pārthivaṃ sakalaṃ baṃdhanaṃ truṭyati/ \P %%%7654.jpg vorletzte Zeile
%caturdaśo nāso mūlādhāraḥ//       tasmin llakṣe krute satī sthairyakāraṇāt// ṣaṣṭhe māse svayaṃ tejaḥ pratyakṣaṃ bhavati// tejasaḥ pratyakṣatve pārthivaṃ sakalaṃ baṃdhanaṃ truṭayati/ \B
%caturdaśo nāso mūlādhāraḥ         tasmin lakṣe kṛte satī   sthairyakāraṇāt   ṣaṣṭhe māse svayaṃ tejaḥ pratyakṣaṃ bhavati// tejasaḥ pratyakṣatve pārthivaṃ sakalaṃ baṃdhanaṃ truṭayati/ \L
%caturdaśo nāsāmūle vāyvādhāraḥ/   tasmin dṛṣṭeḥ            sthairyakāraṇāt   ṣaṣṭhe māsi svīyaṃ tejaḥ pratyakṣaṃ bhavati/  tejasaḥ pratyakṣatve pārthivaṃ sakalaṃ baṃdhanaṃ trudyati/  \N1
%caturdaśo nāsāmūle vāyvādhāraḥ//  tasmin dṛṣṭeḥ            sthairyakāraṇāt   ṣaṣṭhe māsi svīyaṃ tejaḥ pratyakṣaṃ bhavati// tejasaḥ pratyakṣatve pārthivaṃ sakalaṃ baṃdhanaṃ trudyati// \D  %%%p.13 recto 
%caturdaśo nāsāmūle vāyvādhāraḥ??/ tasmin dṛṣṭeḥ            sthairyakāraṇāt   ṣaṣṭhe māsi svayaṃ tejaḥ pratyakṣaṃ bhavati   tejasaḥ pratyakṣatve pārthiva  sakalaṃ bandhanaṃ trudyati// \N2
%caturdaśo nāsāmūle vādhāraḥ       tasmiṃ na dṛṣṭeḥ         sthairyakāraṇāt   ṣaṣṭhe māse svīyaṃ tejaḥ pratyakṣaṃ bhavati   tejasaḥ pratyakṣatve pārthivaṃ sakalaṃ baṃdhanaṃ truṭyati   \U1
%caturdaśo nāsāmūlādhāraḥ          tasmin laṣṭhe?           sthairyakāraṇāt   ṣaṣṭhe māsi svayaṃ tejaḥ pratyakṣaṃ bhavati// tejasaḥ pratyakṣatve pārthivaṃ sakalaṃ baṃdhanaṃ truṭyati// \U2
%-----------------------------
%The fourteenth is the container of breath at the root of the nose. From the execution of stabilizing of the gaze onto this the light of one's own becomes perceptible within 60 months. He breaks the mundane with regard to direct perception of the light. 
%-----------------------------
\note[type=source, labelb=240, lem={nāsikādhāraḥ}]{SSP: caturdaśe nāsāmūle kapāṭādhāre dṛṣṭiṃ dhārayet | ṣaṇmāsāj jyotiḥpuñjaṃ paśyati ||2.23||}
\note[type=source, labelb=241, lem={nāsikādhāraḥ}]{Ysv (PT=YK): nāsāpuṭe sthirā dṛṣṭir ādhāro 'yaṃ caturdaśaḥ | kṛte 'smin svīyatejaḥ syāt pratyakṣaṃ ṣaṭtrimāsataḥ | pārthivaṃ truṭati kṣipraṃ pratyakṣaṃ svīyatejasā |}
caturdaśo
\app{\lem[wit={D,N1,N2}]{nāsāmūle vāyvādhāraḥ}
  \rdg[wit={U1}]{nāsāmūle vādhāraḥ}
  \rdg[wit={P}]{nāsāmūlādhāro}
  \rdg[wit={B,L}]{nāso mūlādhāraḥ}
  \rdg[wit={E,U2}]{nāsāmūlādhāraḥ}}
\app{\lem[wit={ceteri}]{tasmin}
  \rdg[wit={ceteri}]{tasmiṃ na}}
\app{\lem[wit={ceteri}]{dṛṣṭeḥ}
  \rdg[wit={U1}]{na dṛṣṭeḥ}
  \rdg[wit={B}]{llakṣe krute satī}
  \rdg[wit={L}]{lakṣe kṛte satī}
  \rdg[wit={U2}]{laṣṭhe}}
sthairyakāraṇāt
ṣaṣṭhe
\app{\lem[wit={B,L,U1}]{māse}
  \rdg[wit={ceteri}]{māsi}} 
\app{\lem[wit={ceteri}]{svīyaṃ}
  \rdg[wit={B,L,N2,U2}]{svayaṃ}}
tejaḥ pratyakṣaṃ bhavati/
tejasaḥ pratyakṣatve
\app{\lem[wit={ceteri}]{pārthivaṃ}
  \rdg[wit={N2}]{pārthiva}}
bandhanaṃ 
\app{\lem[wit={P,U2,U1}]{truṭyati}
  \rdg[wit={E}]{tuṭyati}
  \rdg[wit={B,L}]{truṭayati}
  \rdg[wit={N1,N2,D}]{trudyati}}/
%----------------------------
%pañcadaśo bhruvormadhyādhāras        tasmin dṛṣṭeḥ sthirīkaraṇāt    koṭikiraṇāḥ  sphuraṃti/ \E
%paṃcadaśo bhruvormadhyādhāraḥ        tasmin ḍṛṣṭeḥ sthirīkaraṇāt    koṭikiraṇāḥ  sphuraṃti  \P  %%%7655.jpg
%paṃcadaśo bhruvormadhye dhāraḥ//     tasmin ḍṛṣṭeḥ sthirikaraṇāt//  koṭikiriṇā   sphuraṃti// \B
%paṃcadaśo bhruvormadhye dhāraḥ//     tasmin ḍṛṣṭe  sthirīkaraṇāt//  koṭikiriṇā   sphuraṃti// \L
%pañcadaśo bhruvormadhye ādhāraḥ/      asmin dṛṣṭeḥ sthirīkaraṇāt    koṭikiraṇāni sphuraṃti/ \N1
%pañcadaśo bhruvormadhye ājñādhāraḥ// ..smin dṛṣṭeḥ sthirīkaraṇāt    koṭikiraṇāni sphuraṃti// \D
%pañcadaśo bhruvormadhye ādhāraḥ      tasmin dṛṣṭeḥ sthirīkaraṇāt    koṭikiraṇāni sphuraṃti/ \N2 [S.9]
%pañcadaśo bhruvormadhye ādhāra         asin na dṛṣṭeḥ sthirīkaraṇāt koṭikiraṇāni sphuraṃti \U1
%pañcadaśo bhruvormadhyādhāra         tasmin dṛṣṭisthirīkaraṇāt      koṭikiraṇaḥ  sphuraṃti// \U2
%-----------------------------
%The fifteenth container is situated in the middle of the eyebrows. Due to stabilized the gaze therein 10 million rays of light sparkle. 
%----------------------------
\note[type=source, labelb=242, lem={nāsikādhāraḥ}]{SSP: pañcadaśe lalāṭādhāre tatra jyotiḥpuñjaṃ lakṣayet | tejasvī bhavati ||2.24||}
\note[type=source, labelb=243, lem={nāsikādhāraḥ}]{Ysv (PT): pañcadaśo bhruvormadhye sthira [sthirā (YK)] dṛṣṭis tathā dhruvam | asmin dṛṣṭiḥ sthirā koṭiḥ [koṭi° (YK)] kiraṇāni sphuranti hi |}
pañcadaśo
\app{\lem[type=emendation, resp=egoscr]{bhruvormadhya ādhāraḥ}
  \rdg[wit={N1,N2}]{\korr bhruvormadhye ādhāraḥ}
  \rdg[wit={U1}]{bhruvormadhye ādhāra}
  \rdg[wit={L,B}]{bhruvormadhye dhāraḥ}
  \rdg[wit={U2}]{bhruvormadhyādhāra}
  \rdg[wit={P}]{bhruvormadhyādhāraḥ}
  \rdg[wit={E}]{bhruvormadhyādhāras}
  \rdg[wit={D}]{bhruvormadhye ājñādhāraḥ}}/
\app{\lem[wit={ceteri}]{tasmin}
  \rdg[wit={N1}]{asmin}
  \rdg[wit={D}]{smin}
  \rdg[wit={U1}]{asin}
}\app{\lem[wit={ceteri}]{ḍṛṣṭeḥ}
  \rdg[wit={L}]{ḍṛṣṭe}
  \rdg[wit={U1}]{na dṛṣṭeḥ}
  \rdg[wit={U2}]{dṛṣṭi°}}
sthirīkaraṇāt
koṭi\app{\lem[wit={D,N1,N2,U1}]{kiraṇāni}
  \rdg[wit={E,P}]{koṭikiraṇāḥ}
  \rdg[wit={U2}]{koṭikiraṇaḥ}
  \rdg[wit={B,L}]{koṭikiriṇā}}
sphuranti/
\note[type=philcomm, labelb=244, lem={kiraṇāni}]{The better group of witnesses D\textsubscript{1},N\textsubscript{1},N\textsubscript{2} and U\textsubscript{1} support the uncommon neuter from of \textit{kiraṇa}. This is also supported by the Ysv and was hence adopted.}
%----------------------------
%ṣoḍaśo  netrādhāraḥ/  ayam aṃgulyagreṇa cālyate/  tadabhyāsāt/ pṛthvīmadhye  yatkiṃcin  tejo  varttate/  \E   %%%p.45
%ṣoḍaśo  netrādhāraḥ   ayam aṃgulyagreṇa cālyate   tadabhyāsāt  pṛthvīmadhye  yatkiṃcit  tejo  vartate... \P
%ṣoḍaśo  netrā//       ayam aṃgulyagreṇa cālyate// tadabhyāsāt  pṛthivīmadhye yatkiṃcit  tejo  vartate//  \B %%%%%%%%%%%%%%%%DSCN7167.jpg Z. 1
%ṣoḍaśo  netrā//       ayam aṃgulyagreṇa cālyate// tadabhyāsāt  pṛthivīmadhye yatkiṃcit  tejo  vartate... \L
%ṣoḍaśaḥ netrādhāraḥ/  ayaṃ agulyagreṇa  cālyate/  tadabhyāsāt  pṛthvīmadhye  yatkiṃcit  tejaḥ varttate/  \N1
%ṣoḍaśaḥ netrādhāraḥ// ayaṃ agulyagreṇa  cālyate// tadabhyāsāt  pṛthvīmadhye  yatkiṃcit  tejaḥ varttate \D
%ṣoḍaśaḥ netrādhāraḥ/  ayaṃ aṃgugreṇa    cālyate/  tadabhyāsāt  pṛthvīmadhye  yatkiṃcit  tejaḥ varttate/  \N2
%ṣoḍaśo  netrādhāraḥ   ayaṃ aṃgulyagreṇa cālyate   tadābhyāsāt  pṛthvīmadhye  yatkiṃcit        vatate     \U1 %%%%%%%%%%%%%%%%%%285.jpg
%ṣoḍaśo  netrādhāraḥ   ayam aṃgulyagreṇa cālyate// tadabhyāsāt  pṛthivīmadhye yatkiṃcit// tejo vartate//  \U2
%-----------------------------
%The sixteenth is the eye-container. Without wavering, the gaze [ayam] is to be held at the tip of the finger without wavering. From practicing this on earth any energy exists [for him].   
%-----------------------------
\note[type=source, labelb=245, lem={netrādhāraḥ}]{SSP: avaśiṣṭe ṣoḍaśe brahmarandhram ākāśacakram | tatra śrīgurucaraṇāmbujayugmaṃ sadāvalokayet | ākāśavat pūrṇo bhavati ||2.25||}
\note[type=source, labelb=246, lem={netrādhāraḥ}]{Ysv (PT): netrādhāraḥ ṣoḍaśo 'yam aṅgulyagreṇa cālayet | pṛthvīmadhye tu yatkiñcid varttate [sarvajñaḥ prabhavastena varddhate (YK)] jaṭharānalaḥ | pratyakṣaṃ tad bhavet sarvaṃ tadābhyāsān na saṃśayaḥ |}
\app{\lem[wit={ceteri}]{ṣoḍaśo}
  \rdg[wit={D,N1,N2}]{ṣoḍaśaḥ}}
\app{\lem[wit={ceteri}]{netrādhāraḥ}
  \rdg[wit={L,B}]{netrā}}/
\app{\lem[wit={ceteri},alt={ayam}]{aya\skp{m-a}}
  \rdg[wit={D,N1,N2,U1}]{ayaṃ}
}\app{\lem[type=emendation, resp=egoscr, alt={aṅgulyagre na}]{\skm{m-a}ṅgulyagre na}
  \rdg[wit={ceteri}]{\korr aṅgulyagreṇa}
  \rdg[wit={N1,D}]{agulyagreṇa}
  \rdg[wit={N2}]{aṃgugreṇa}}
cālyate/
tadabhyāsāt
\app{\lem[wit={ceteri},alt={pṛthvī°}]{pṛthvī}
  \rdg[wit={L,B,U2}]{pṛthivī°}}madhye
yatkiṃcit
\app{\lem[wit={ceteri}]{tejo}
  \rdg[wit={D,N1,N2}]{tejaḥ}
  \rdg[wit={U1}]{\om}}
\app{\lem[wit={ceteri}]{vartate}
  \rdg[wit={U1}]{vatate}}/
%----------------------------
%tatsarvaṃ tejo   dṛṣṭiviṣayaṃ bhavati/  taddarśanāt  puruṣaḥ sarvajño  bhavati// \E
%tatsarvaṃ tejo   dṛṣṭiviṣayaṃ bhavati   tadarśanāt   puruṣaḥ sarvajño  bhavati     \P
%tatsarvaṃ tejo   dṛṣṭiviṣayaṃ bhavatī// taddarśanāt  puruṣaḥ sarvajño  bhavatī// \B
%tatsarvaṃ tejo   dṛṣṭiviṣayaṃ bhavati// taddarśanāt  puruṣaḥ sarvajño  bhavati// \L
%tatsarvvatejo    dṛṣṭiviṣayaṃ bhavati   taddarśanāt  puruṣaḥ sarvvajño bhavati// \N1
%tatsarvatejo     dṛṣṭiviṣayaṃ bhavati   taddarśanāt  puruṣaḥ sarvvajño bhavati// \D
%tatsarvatejo     dṛṣṭiviṣayaṃ bhavati   taddarśanāt  puruṣaḥ sarvajño  bhavati// \N2
%tatsarvaṃ tejo   dṛṣṭīviṣayaṃ bhavati   tatdarśaḥ    puruṣaḥ sarvajño  bhavati \U1
%tatsarvaṃ tajaso dṛṣṭiviṣayaṃ bhavati// taddarśanāt  puruṣaḥ sarvajño  bhavati// \U2
%-----------------------------
%The light of everying that is arises as the object of sight. From that sight the person becomes omniscient. 
%-----------------------------
\app{\lem[wit={D,N1,N2}]{tatsarvatejo}
  \rdg[wit={ceteri}]{tatsarvaṃ}}
dṛṣṭiviṣayaṃ
\app{\lem[wit={ceteri}]{bhavati}
  \rdg[wit={B}]{bhavatī}}
\app{\lem[wit={ceteri}]{taddarśanāt}
  \rdg[wit={P}]{tadarśanāt}
  \rdg[wit={U1}]{tatdarśaḥ}}
puruṣaḥ
sarvajño 
\app{\lem[wit={ceteri}]{bhavati}
  \rdg[wit={B}]{bhavatī}}/\\
\end{prose}
\end{ekdosis}
\ekdpb*{}
%%%%%%%%%%%%%%%%%%%%%%%%%%%%%%%%%%%%%%%%%%
%%%%%%%%%%%%%%%%%%%%%%%%%%%%%%%%%%%%%%%%%%
%%%%%%%%PAGEBREAK%%%%%%%PAGEBREAK%%%%%%%%%
%%%%%%%%%%%%%%%%%%%%%%%%%%%%%%%%%%%%%%%%%%
%%%%%%%%%%%%%%%%PAGEBREAK%%%%%%%%%%%%%%%%%
%%%%%%%%%%%%%%%%%%%%%%%%%%%%%%%%%%%%%%%%%%
%%%%%%%%PAGEBREAK%%%%%%%PAGEBREAK%%%%%%%%%
%%%%%%%%%%%%%%%%%%%%%%%%%%%%%%%%%%%%%%%%%%
%%%%%%%%%%%%%%%%%%%%%%%%%%%%%%%%%%%%%%%%%%
%%%%%%%%%%%%%%%%%%%%%%%%%%%%%%%%%%%%%%%%%%
%%%%%%%%%%%%%%%%%%%%%%%%%%%%%%%%%%%%%%%%%%
%%%%%%%%PAGEBREAK%%%%%%%PAGEBREAK%%%%%%%%%
%%%%%%%%%%%%%%%%%%%%%%%%%%%%%%%%%%%%%%%%%%
%%%%%%%%%%%%%%%%PAGEBREAK%%%%%%%%%%%%%%%%%
%%%%%%%%%%%%%%%%%%%%%%%%%%%%%%%%%%%%%%%%%%
%%%%%%%%PAGEBREAK%%%%%%%PAGEBREAK%%%%%%%%%
%%%%%%%%%%%%%%%%%%%%%%%%%%%%%%%%%%%%%%%%%%
%%%%%%%%%%%%%%%%%%%%%%%%%%%%%%%%%%%%%%%%%%
%%%%%%%%%%%%%%%%%%%%%%%%%%%%%%%%%%%%%%%%%%
%%%%%%%%%%%%%%%%%%%%%%%%%%%%%%%%%%%%%%%%%%
%%%%%%%%PAGEBREAK%%%%%%%PAGEBREAK%%%%%%%%%
%%%%%%%%%%%%%%%%%%%%%%%%%%%%%%%%%%%%%%%%%%
%%%%%%%%%%%%%%%%PAGEBREAK%%%%%%%%%%%%%%%%%
%%%%%%%%%%%%%%%%%%%%%%%%%%%%%%%%%%%%%%%%%%
%%%%%%%%PAGEBREAK%%%%%%%PAGEBREAK%%%%%%%%%
%%%%%%%%%%%%%%%%%%%%%%%%%%%%%%%%%%%%%%%%%%
%%%%%%%%%%%%%%%%%%%%%%%%%%%%%%%%%%%%%%%%%%
\begin{ekdosis}
  \ekddiv{type=ed}
   \centerline{\textrm{\small{[Aṣṭāṅgayoga]}}}
      \bigskip
      \begin{prose}
        \noindent
%----------------------------
%Note: Rāmacandra does not adopt the yāmas and niyāmas from the Yogasvarodaya! 
%----------------------------
%idānīm aṣṭāṃgayoga----vicāraḥ kathyate/  yamaniyamāsanaprāṇāyāmapratyāhāradhyānadhāraṇāsamādhir iti/  eteṣāṃ lakṣaṇāni kathyante/     \E
%idānīm aṣṭāṃgayogasya vicāraḥ kathyate   yamaniyamāsanaprāṇāyāmapratyāhāradhyānadhāraṇāsamādhir iti   eteṣāṃ lakṣaṇāni kathyaṃte  \P
%idānīm aṣṭāṃgayogasya vicāraḥ kathyate/  yamaniyamāsanaprāṇāyāmapratyāhāradhāraṇādhyānasamādhir iti/  eteṣāṃ lakṣaṇāni kathyaṃte/ \B
%idānīm aṣṭāṃgayogasya vicāraḥ kathyate/  yamaniyamāsanaprāṇāyāmapratyāhāradhāraṇādhyānasamādhir iti/  eteṣāṃ lakṣaṇāni kathyaṃte/ \L
%idānīm aṣṭāṃgayogasya vicāraḥ kathyate// yamaniyamāsanaprāṇāyāmapratyāhāradhyānadhāraṇāsamādhiyaḥ     eteṣāṃ lakṣaṇāni kathyaṃte/   \N1
%idānīm aṣṭāṃgayogasya vicāraḥ kathyate// yamaniyamāsanaprāṇāyāmapratyāhāradhyānadhāraṇāsamādhi//      eteṣāṃ lakṣaṇāni kathyaṃte//   \D
%idānīṃ aṣṭāṃgayogasya vicāraḥ kathyate// yamaniyamāsanaprāṇāyāmapratyāhāradhyānadhāraṇāsamādhiyaḥ     eteṣāṃ lakṣaṇāni kathyaṃte/   \N2
%idānīṃ aṣṭāṅgayogasya vicāraḥ kathyate// yamaniyamāsanaprāṇāyāmapratyāhāradhyānadhāraṇāsamādhi        eteṣāṃ lakṣaṇāni kathyate   \U1
%idānīṃ aṣṭāṅgayogasya vicāra  kathyate// yamaniyamāsanaprāṇāyāmapratyāhāradhyānadhāraṇāsamādhir iti// eteṣāṃ lakṣaṇāni kathyaṃte//   \U2
%-----------------------------
%Now the procedure of the eightfold yoga (\textit{aṣṭāṅgayoga})is explained: "Yama, niyama, āsana, prāṇāyāma, pratyāhāra, dhyāna, dhāraṇā and samādhi." Their characteristics will be explained.   
%----------------------------
\note[type=source, labelb=247, lem={aṣṭāṃga°}]{SSP:yamaniyamāsanaprāṇāyāmapratyāhāradhāraṇādhyānasamādhayoḥ 'ṣṭāṅgāni|}
\note[type=source, labelb=248, lem={aṣṭāṃga°}]{Ysv (PT=YK): idānīṃ yogamaṣṭāṅgaṃ śṛṇu lakṣaṇasaṃyutam | yamaś ca niyamaś caiva cāsanaṃ prāṇasaṃyamaḥ | pratyāhāro dhāraṇā ca samādhiś ca viśeṣataḥ | aṣṭāṅgayoga ebhis tu caiteṣāṃ lakṣaṇaṃ śṛṇu |}
\app{\lem[wit={ceteri},alt={idānīm}]{idānī\skp{m-a}}
  \rdg[wit={N2,U1,U2}]{idānīṃ}
}\app{\lem[wit={ceteri},alt={aṣṭāṅgayogasya}]{\skm{m-a}ṣṭāṅgayogasya}
  \rdg[wit={E}]{aṣṭāṃgayoga°}}
\app{\lem[wit={ceteri}]{vicāraḥ}
  \rdg[wit={U2}]{vicāra}}
kathyate/
yamaniyamāsanaprāṇāyāmapratyāhāra\app{\lem[wit={ceteri},alt={°dhyānadhāraṇāsamādhir iti}]{dhyānadhāraṇāsamādhir\skp{-}iti}
  \rdg[wit={B,L}]{dhāraṇādhyānasamādhir iti}
  \rdg[wit={N1,N2}]{dhyānadhāraṇāsamādhiyaḥ}
  \rdg[wit={D,U1}]{dhyānadhāraṇāsamādhi}}
eteṣāṃ lakṣaṇāni
\app{\lem[wit={ceteri}]{kathyante}
  \rdg[wit={U1}]{kathyate}}/
%----------------------------
%śāntiḥ/ ṣaṇṇām  indriyāṇāṃ jayaḥ/ svalpāhāraḥ/            nidrājayaḥ/      śītoṣṇajayaḥ/               ete yamāḥ/ \E
%śāṃtiḥ  ṣaṇāṃ   iṃdriyāṇāṃ jayaḥ       ahāraḥ svalpaḥ     nidrājayaḥ       śaityajayaḥ   uṣṇa?jayaḥ    ete yamāniyamāḥ ...\P
%śāntiḥ  ṣaṇāṃ   iṃdriṇāṃ   jayaḥ//     ahāraḥ svalpaḥ     nidrāyā jayaḥ//  śaityajayaḥ/  uṣṇājayaḥ// ya te yamaḥ// \B
%śāntiḥ  ṣaṇṇāṃ  iṃdriyāṇāṃ jayaḥ//     ahāraḥ// svalpaḥ// nidrāyāḥ jayaḥ/  śaityajayaḥ   uṣṇajayaḥ   ya te yamaḥ... \L
%śānti---ṣaṇṇāṃ  indriyāṇāṃ jayaḥ/      svalpāḥ            nidrājayaḥ/      śītyajayaḥ/   uṣṇajayaḥ/    ete yamāḥ/ \N1
%śāṃti---ṣaṇṇāṃ  indriyāṇāṃ jayaḥ//     āhāraḥ svalpāḥ     nidrājayaḥ//     śaityajayaḥ// uṣṇajayaḥ/    ete yamāḥ \D
%śānti---ṣaṇṇāṃ  indriyāṇāṃ jayaḥ/      ahāraḥ svalpāḥ     nidrājayaḥ/      śaityajayaḥ   uṣṇajayaḥ/    ete yamāḥ/ \N2
%śāntiḥ  ṣaṇṇām  iṃdriyāṇāṃ jayaḥ       āhāraḥ sajayaḥ     nidrājayaḥ       śaityajayaḥ   auṣṇājayaḥ    ete yamāḥ \U1
%śānti---śaṇa    iṃdriyāṇāṃ jayaḥ//     āhāraḥ svalpaḥ//   nidrāyāḥ jayaḥ// śaityajayaḥ// uṣṇājayaḥ//   ete yamāḥ// \U2 %%%417.jpg 
%----------------------------
%These are the Yāmas: Peace, conquer of the six senses, little food, conquer of sleep, conquer of cold and heat.
%----------------------------
\note[type=source, labelb=249, lem={ete yamāḥ}]{SSP:yama iti upaśamaḥ sarvendriyajayaḥ āhāranidrāśītavātātapajayaś caivaṃ śanaiḥ śanaisādhayet ||2.32||}
\note[type=source, labelb=250, lem={ete yamāḥ}]{Ysv (PT): śāntiḥ santoṣa āhāro nidrālpā [nidrālpaṃ (YK)] manaso damaḥ | śūnyāntaḥ karaṇañceti [karaṇaś ceti (YK)] yamā iti prakīrttitāḥ |}
\app{\lem[wit={ceteri}]{śāntiḥ}
  \rdg[wit={D,N1,N2,U2}]{śānti°}}\dd{}
\app{\lem[wit={E,U1},alt={ṣaṇṇām}]{ṣaṇṇā\skp{m-i}}
  \rdg[wit={D,L,N1,N2}]{ṣaṇṇāṃ}
  \rdg[wit={B,P}]{ṣaṇāṃ}
  \rdg[wit={U2}]{śaṇa}
}\app{\lem[wit={ceteri},alt={indriyāṇāṃ}]{\skm{m-i}ndriyāṇāṃ}
  \rdg[wit={B}]{iṃdriṇāṃ}}
jayaḥ\dd{}
\app{\lem[wit={U2}]{āhāraḥ svalpaḥ}
  \rdg[wit={E}]{svalpāhāraḥ}
  \rdg[wit={B,P}]{ahāraḥ svalpaḥ}
  \rdg[wit={L}]{ahāraḥ|| svalpaḥ ||}
  \rdg[wit={N1}]{svalpāḥ}
  \rdg[wit={N2}]{ahāraḥ svalpāḥ}
  \rdg[wit={D}]{āhāraḥ svalpāḥ}
  \rdg[wit={U1}]{āhāraḥ sajayaḥ}}\dd{}
\app{\lem[wit={ceteri}]{nidrājayaḥ}
  \rdg[wit={B}]{nidrāyā jayaḥ}
  \rdg[wit={L,U2}]{nidrāyāḥ jayaḥ}}\dd{}
\app{\lem[wit={ceteri}]{śaityajayaḥ}
  \rdg[wit={N1}]{śītyajayaḥ}
  \rdg[wit={E}]{śītoṣṇajayaḥ}}\dd{}
\app{\lem[wit={ceteri}]{uṣṇajayaḥ}
  \rdg[wit={B,U2}]{uṣṇājayaḥ}
  \rdg[wit={U1}]{auṣṇājayaḥ}
  \rdg[wit={E}]{\om}}\dd{}
\app{\lem[wit={ceteri}]{ete}
  \rdg[wit={B,L}]{ya te}}
\app{\lem[wit={ceteri}]{yamāḥ}
  \rdg[wit={P}]{yamāniyamāḥ}
  \rdg[wit={B,L}]{yamaḥ}}\dd{}
%----------------------------
%niyamāḥ   khalu       cāpalabhāvān nivārya  sthairye  sthāpyate/  ekāṃte sevanam/ prāṇimātre samābuddhiḥ/ audāsīnyaṃ   kasyāpi vastuna    icchā na karttavyā    yathā lābhasaṃtoṣaḥ/   \E
%          khalu       cāpalābhāvān nirvārya sthairye  sthāpyate   ekāṃta sevānaṃ  prāṇimātre samābuddhiḥ   udāsīnyaṃ   kasyāpi vastuna    icchā na kartavyā     yathā lābhasaṃtoṣaḥ    \P %%%7656.jpg
%          khalu       cāpalabhāvān nirvārya           sthāpyate//           ekāṃta sevānāṃ  prāṇimātre samābuddhiḥ   udāsīnyaṃ   kasyāpi vastunaḥ// icchā na kartavyā     yathā lābhasaṃtoṣaḥ/   \B
%          ḱhalu       cāpalabhāvān nirvārya           sthāpyate//           ekāṃtasevānāṃ   prāṇimātre samābuddhiḥ/  udāsīnyaṃ   kasyāpi vastunaḥ/  icchā na kartavyā     yathā lābhasaṃtoṣaḥ    \L
%niyamaḥ   khalu       capalabhāvān nivārya  sthairye  sthāpyate/  ekāṃte sevanam/ prāṇimātre samābuddhiḥ/  udāsīnya/   kasyāpi vastunaḥ   icchā na karttavyā//  yathā lābhasaṃtoṣaḥ/   \N1
%niyamaḥ   khalu manaḥ capalabhāvān nivārye            sthāpyate//           ekāṃtasevanaṃ// prāṇimātre samābuddhiḥ// udāsīnya//  kasyāpi vastunaḥ   icchā na karttavyā//  yathā lābhasaṃtoṣaḥ//  \D
%niyamaḥ   khalū manaḥ capalabhāvān nivārya  sthairye  sthāpyate   ekāṃtasevanam/  prāṇimātre samābuddhiḥ   udāsīnya    kasyāpi vastunaḥ   icchā na karttavyā/   yathā lābhasaṃtoṣaḥ    \N2
%niyamaḥ   khalū manaḥ capalabhāvān nivāraya sthairye  sthāpyate   ekāṃtasevanaṃ   prāṇimātre samābuddhi    udāsīnyāṃ   kasyāpi vastunaḥ   icchā na karttavyaṃ   yathā lābhasaṃtoṣaḥ    \U1
%niyamaḥ// khalū       cāpalābhāvān nivārya            sthāpyate// ekāṃtasevanaṃ// prāṇimātre samābuddhi//  udāsīnyaṃ// kasyāpi vastuna    icchā na karttavyaṃ// yathā lābhasaṃtoṣaḥ//  \U2
%----------------------------
%parameśvaranāma na vismaraṇīyam/  manomadhye      dainyaṃ    karttavyam/ iti niyamāḥ// \E
%parameśvaranāma na vismaraṇīyaṃ   manomadhye      dainyaṃ    kartavyaṃ   iti niyamāḥ\P %%%7656.jpg
%parameśvaranāma na vismaraṇīyaṃ   manomadhye      dainyaṃ    kartavyaṃ// iti niyamaḥ// \B
%parameśvaranāma na vismaraṇīyaṃ   manomadhye      dainyaṃ    karttavyaṃ/ iti niyamaḥ// \L
%parameśvaranāma----vismaraṇīyam/  manomadhye      dainyaṃ na karttavyam/ //[S.11] \N1
%parameśvaranāma----vismaraṇīyaṃ// manomadhye      dainyaṃ na karttavyaṃ// \D
%parameśvaranāma----vismanīyam/    manomadhye      dainyaṃ na karttavyam// // \N2 \em zu vismāra
%parameśvaraḥ nāma na vismaraṇīyaṃ mano            dainyaṃ na karttavyaṃ  \U1
%parameśvaraḥ nāma na vismaraṇaṃ// yaṃ mano madhye dainyaṃ na karttavyaṃ iti niyamaḥ//  \U2
%----------------------------
%Niyamās are truly: Keeping the mind from the state of unsteadiness [and] ground it in calmness, retreating to a lonely place, refraining from contact to animals, unchanging intellect, keeping equanimous one shall not crave for things, as well as being contend with what is given, never forgetting the name of the highest lord, one shall not bring the mind into depression. 
%----------------------------
\note[type=source, labelb=251, lem={niyamāḥ}]{SSP:niyama iti manovṛttīnāṃ niyamanam iti ekāntavāso niḥsaṃgataudāsīnyaṃ yathāprāptisaṃtuṣṭir vairasyaṃ gurucaraṇāvarūḍhatvam iti niyamalakṣaṇam ||2.33||}
\note[type=source, labelb=252, lem={niyamāḥ}]{Ysv (YK): tyaktvā dūre tu cāpalyaṃ [cāpalyantu dūre tyaktvā (Ysv)] manaḥ sthairyyaṃ vidhāya ca ||31|| ekatra melanaṃ nityaṃ prāṇāmātre na sāmabhiḥ [sā matiḥ (PT)] | sadodāsīnabhāvas tu sarvatrecchāvivarjitaḥ [°vivarjanam (PT)] ||32|| yathālābhena santuṣṭaḥ parameśvaramānasaḥ | mānadānaparityāga ete tu niyamā iti || 33||}
\app{\lem[wit={E}]{niyamāḥ}
  \rdg[wit={D,N1,N2,U1,U2}]{niyamaḥ}
  \rdg[wit={B,P,L}]{\om}}\dd{}
\app{\lem[wit={ceteri}]{khalu}
  \rdg[wit={N1,N2,U2}]{khalū}} 
\app{\lem[wit={D,N2,U1}]{manaḥ}
  \rdg[wit={ceteri}]{\om}}
\app{\lem[wit={B,E,L},alt={cāpala°}]{cāpala}
  \rdg[wit={P,U2}]{cāpalā°}
  \rdg[wit={D,N1,N2,U1}]{capala°}
}bhāvā\skp{n-ni}
\app{\lem[wit={ceteri},alt={nivārya}]{\skm{n-ni}vārya}
  \rdg[wit={D}]{nivārye}
  \rdg[wit={B,L,P}]{nirvārya}
  \rdg[wit={U1}]{nivāraya}}
\app{\lem[wit={ceteri}]{sthairye}
  \rdg[wit={B,L,D,U2}]{\om}}
sthāpyate\dd{}
%----------------------------
%āsanalakṣaṇaṃ     bahuṣu grantheṣu nirūpitam     asti    tenātra na nirūpyate/ \E
%āsanalakṣaṇaṃ     bahuṣu graṃtheṣu nirūpitam     asti    tenātra na nirūpyate \P
%āsanaṃ lakṣaṇāṃ   bahūgraṃtheṣu    nirūpyam      asti    tenātra    nirūpyate/       \B
%āsanalakṣaṇāṃ     bahūgraṃtheṣu    nirūpyam      asti    tenātra    nirūpyate//     \L
%āsanasya lakṣaṇaṃ bahūgraṃthe      nirūpitam/    ataḥ    atrāyaṃ    nirūpyate/   \N1
%āsanasya lakṣaṇaṃ bahūgraṃthe      nirūpitaṃ//   ataḥ    atratyaṃ   nirūpyate// \D %%%p. 13 verso
%āsanasya lakṣaṇaṃ bahugraṃthe      nirūpitam//   ataḥ    atrāyaṃ    nirūpyate/  \N2
%āsanasya lakṣaṇaṃ bahugraṃthe      nirūpitam tan attaḥ   atra    na nirūpyate  \U1
%āsanalakṣaṇaṃ tu  bahugraṃtheṣu    nirūpitam     asti//  tenātra    nirūpyate// \U2
%----------------------------
%The characteristic of posture has been discussed in many works. Because of that it will not be discussed here.  
%----------------------------
\note[type=source, labelb=253, lem={āsanasya}]{SSP: āsanam iti svasvarūpe samāsannatā | svastikāsanaṃ padmāsanaṃ siddhāsanam eteṣāṃ madhye yatheṣṭam ekaṃ vidhāya sāvadhānena sthātavyam ity āsanalakṣaṇam ||2.34||}
\note[type=source, labelb=254, lem={āsanasya}]{Ysv (YK): āsanāni ca tāvanti yāvanto jīvajantavaḥ |[om. YK]}
\app{\lem[wit={D,N1,N2,U1}]{āsanasya lakṣaṇaṃ}
  \rdg[wit={E,P,L}]{āsanalakṣaṇaṃ}
  \rdg[wit={U2}]{āsanalakṣaṇaṃ tu}
  \rdg[wit={B}]{āsanaṃ lakṣaṇāṃ}}
\app{\lem[wit={B,L,U2}]{bahūgrantheṣu}
  \rdg[wit={E,P}]{bahuṣu graṃtheṣu}
  \rdg[wit={D,N1,N2,U1}]{bahūgraṃthe}}
\app{\lem[wit={E,P,U2},alt={nirūpitam}]{nirūpita\skp{m-a}}
  \rdg[wit={D}]{nirūpitaṃ ||}
  \rdg[wit={N1,N2}]{nirūpitam |}
  \rdg[wit={B,L}]{nirūpyam}
  \rdg[wit={U1}]{nirūpitam tan}}
\app{\lem[wit={B,E,L,P,U2},alt={asti}]{\skm{m-a}sti}
  \rdg[wit={D,N1,N2,U1}]{ataḥ}}
\app{\lem[wit={U2}]{/}
  \rdg[wit={ceteri}]{\om}} 
\app{\lem[wit={B,E,L,P,U2}]{tenātra}
  \rdg[wit={N1,N2}]{atrāyaṃ}
  \rdg[wit={D}]{atratyaṃ}
  \rdg[wit={U1}]{atra}}
\app{\lem[wit={E,P,U1}]{na}
  \rdg[wit={ceteri}]{\om}}
nirūpyate/
%---------------------------
%prāṇāyāmas tu sukumāreṇa        sādhituṃ na śakyate   atas tasya nāmamātraṃ kathyate/ \E
%prāṇāyāmas tu sukumāreṇa        sādhituṃ na śakyate   atas tasya nāmamātraṃ kathyate  \P
%prāṇāyāmas tu kumāreṇa          sādhituṃ na śakyate// ataḥ       nāma       kathyate/ \B
%prāṇāyāmas tu kumāreṇa          sādhituṃ na śakyate// ataḥ       nāma       kathyate// \L
%prāṇāyāmas tu kūmāreṇa puruṣeṇa sādhituṃ na śakyate/  ataḥ tasya nāmamātraṃ kathitaṃ/ \N1
%prāṇāyāmas tu kūmāreṇa puruṣeṇa sādhituṃ na śakyate// ataḥ tasya nāmamātre  kathitaṃ// \D
%prāṇāyāmas tu kūmāreṇa puruṣeṇa sādhituṃ na śakyate// ata  tasya nāmamātre  kathitaṃ/ \N2
%prāṇāyāmas tu kūmāreṇa puruṣeṇa sādhituṃ na śakyate   atas tasya nāmamātre  kathitaṃ \U1
%prāṇāyāmas tu kūmāreṇa          sādhituṃ na śakyate// atā  tasya nāmamātraṃ  kathyate// \U2
%----------------------------
%Breath-control can't be practiced by young persons. That's why it is just mentioned by name.
%Practicing breath-control can't be done by a young person. 
%----------------------------
\note[type=source, labelb=255, lem={prāṇāyāmas}]{SSP: prāṇāyāma iti prāṇasya sthiratā recakapūrakakumbhakasaṃghaṭṭakaraṇāni catvāri prāṇāyāmalakaṇam ||2.35||}
\note[type=source, labelb=256, lem={prāṇāyāmas}]{Ysv (YK): prāṇāyāmas tridhā ceti bahudhā prathamaṃ śṛṇu | āsane prāṇasaṃyāme na śaktāḥ sukumārakāḥ | mahāpuṇyaprabhāveṇa śakyate tu mahātmanā | iḍāṃ śaśiprabhāṃ dhyātvā mandendunā [yathāśakti (YK)] tu pūrayet [tu kumbhayet (YK)] | pūrayitvā yathāśakti dhyānayogī tu kumbhayet [sentence om. (YK)] | mahājyotir mano [mayo (YK)] bhūtvā vāyuḥ [vāyu° (YK)] pūrṇakalevaraḥ | śaktitrāsantu santrāsya recayed vāyum arhitaḥ | piṅgalām arkavarṇān [°varṇaṃ (YK)] tu tyajed dhyātvā śanaiḥ śanaiḥ | ayaṃ pataṅgaḥ kāṣṭhāgnipratyāsena punaḥ punaḥ | kṛtvā kalevaraṃ śuddhaṃ kuryād yatnair mahātmanā | mano nivārya saṃsāre viṣayakārye [viṣayeṣu (YK)] tathaiva ca | manovikārabhavañ caiva [manovikārān sarvāś ca (YK)] tyaktvā śūnyamayo bhavet |}
%----------------------------
prāṇāyāmas-tu
\app{\lem[wit={E,P}]{sukumāreṇa}
  \rdg[wit={B,L,U2}]{kumāreṇa}
  \rdg[wit={D,N1,N2,U1}]{kūmāreṇa puruṣeṇa}}
sādhituṃ na śakyate/
\end{prose}
\end{ekdosis}
\ekdpb*{}
%%%%%%%%%%%%%%%%%%%%%%%%%%%%%%%%%%%%%%%%%%
%%%%%%%%%%%%%%%%%%%%%%%%%%%%%%%%%%%%%%%%%%
%%%%%%%%PAGEBREAK%%%%%%%PAGEBREAK%%%%%%%%%
%%%%%%%%%%%%%%%%%%%%%%%%%%%%%%%%%%%%%%%%%%
%%%%%%%%%%%%%%%%PAGEBREAK%%%%%%%%%%%%%%%%%
%%%%%%%%%%%%%%%%%%%%%%%%%%%%%%%%%%%%%%%%%%
%%%%%%%%PAGEBREAK%%%%%%%PAGEBREAK%%%%%%%%%
%%%%%%%%%%%%%%%%%%%%%%%%%%%%%%%%%%%%%%%%%%
%%%%%%%%%%%%%%%%%%%%%%%%%%%%%%%%%%%%%%%%%%
%%%%%%%%%%%%%%%%%%%%%%%%%%%%%%%%%%%%%%%%%%
%%%%%%%%%%%%%%%%%%%%%%%%%%%%%%%%%%%%%%%%%%
%%%%%%%%PAGEBREAK%%%%%%%PAGEBREAK%%%%%%%%%
%%%%%%%%%%%%%%%%%%%%%%%%%%%%%%%%%%%%%%%%%%
%%%%%%%%%%%%%%%%PAGEBREAK%%%%%%%%%%%%%%%%%
%%%%%%%%%%%%%%%%%%%%%%%%%%%%%%%%%%%%%%%%%%
%%%%%%%%PAGEBREAK%%%%%%%PAGEBREAK%%%%%%%%%
%%%%%%%%%%%%%%%%%%%%%%%%%%%%%%%%%%%%%%%%%%
%%%%%%%%%%%%%%%%%%%%%%%%%%%%%%%%%%%%%%%%%%
%%%%%%%%%%%%%%%%%%%%%%%%%%%%%%%%%%%%%%%%%%
%%%%%%%%%%%%%%%%%%%%%%%%%%%%%%%%%%%%%%%%%%
%%%%%%%%PAGEBREAK%%%%%%%PAGEBREAK%%%%%%%%%
%%%%%%%%%%%%%%%%%%%%%%%%%%%%%%%%%%%%%%%%%%
%%%%%%%%%%%%%%%%PAGEBREAK%%%%%%%%%%%%%%%%%
%%%%%%%%%%%%%%%%%%%%%%%%%%%%%%%%%%%%%%%%%%
%%%%%%%%PAGEBREAK%%%%%%%PAGEBREAK%%%%%%%%%
%%%%%%%%%%%%%%%%%%%%%%%%%%%%%%%%%%%%%%%%%%
\begin{ekdosis}
  \begin{prose}
    \noindent
\app{\lem[wit={E,P,U1},alt={atas tasya}]{atas\skp{-}tasya}
  \rdg[wit={D,N1}]{ataḥ tasya}
  \rdg[wit={N2}]{ata tasya}
  \rdg[wit={U2}]{atā tasya}
  \rdg[wit={B,L}]{ataḥ}}
\app{\lem[wit={E,P,N1,U2}]{nāmamātraṃ}
  \rdg[wit={D,N2,U1}]{nāmamātre}
  \rdg[wit={B,L}]{nāma}}
\app{\lem[wit={ceteri}]{kathyate}
  \rdg[wit={D,N1,N2,U1}]{kathitaṃ}}/
%----------------------------
%prāṇāyāmastridhā ceti bahudhā prathamaṃ śrṛṇu।
%āsane prāṇasaṃyāme na śaktāḥ sukumārakāḥ।। 2।।
%mahāpuṇyaprabhāveṇa śakyate tu mahātmamā।
%iḍāṃ śaśiprabhāṃ dhyātvā yathāśakti tu kumbhayet।। 3।।
%mahājyotirmayo bhūtvā vāyupūrṇakalevaraḥ।
%śaktitrāsantu saṃtrāsya recayedvāyumarhitaḥ।। 4।।
%piṅgalāmarkavarṇaṃ tu tyajed dhyātvā śanaiḥ śanaiḥ।
%ayaṃ pataṅgakāṣṭhāgnipratyāsena punaḥ punaḥ।। 5।।
%kṛtvā kalevaraṃ śuddhaṃ kuryād yatnairmahātmanā।
%mano nivārya saṃsāre viṣayeṣu tathaiva ca।। 6।।
%manovikārān sarvāśca tyaktvā śūnyamayo bhavet।
%pratyāhāro bhavatyeṣu sarvanindācamatkṛtaḥ।। 7।।
%dhyānaṃ ca dvividhaṃ proktaṃ sthūlasūkṣmavibhedataḥ।
%sthūlaṃ mantramayaṃ viddhi sūkṣmantu mantravarjitam।। 8।।
%ityetatkathitaṃ sarvaṃ yogasaṅketamuttamam।
%adhunā cāṣṭakumbhasya lakṣaṇaṃ śrṛṇu kathyate।। 9।।
%śītkāraṃ sūryabhedaṃ ca uhyāyī śītalī tathā।
%bhastrikā bhrāmarī mūrcchā kevalī cāṣṭa kumbhakāḥ।। 10।।
%----------------------------
%pratyāhāraḥ pratyato   manaḥ saṃsārān nivartyātmani   sthāpyate// manomadhye ye vikārā  utpadyante/  tepi nivāraṇīyāḥ/  anekacamatkāriṇī         buddhir utpadyate/  sāṃgopāṃgaṃ  \E XX! this one?[P.47]
%pratyāhāraḥ kathyate   manaḥ saṃsārān nivṛtyātmanī    sthāpyate   manomadhye ye vikāraḥ utpadyaṃte   tepi nivāraṇīyāḥ   anekacamatkāriṇi         buddhir utpadyataraṃ  \P
%pratyāhāraḥ kathyate// manaḥ saṃsārān nivṛtyātmanī    sthāpyate// manomadhye ye vikārā  utpadyaṃte   tepi nivāraṇīyā    anekacamatkāriṇī         buddhir utpadyate/  sāgopyā// \B 
%pratyāhāraḥ kathyate   manaḥ saṃsārān nivṛttyātmanī   sthāpyate// manomadhye ye vikārā  utpadyaṃte   tepi nivāraṇīyā    anekacamatkāriṇi         buddhir utpadyate   sāgopyā//  \L %%%%0023.jpg
%pratyāhāraḥ kathyate// manaḥ saṃsārān nivṛtya ātmani  sthāpyate/  manomadhye ye vikārā  utpadyante/  tepi nivāraṇīyāḥ/  anekacamatkārakarakāraṇī buddhi  utpadyate   sāṃgopyāḥ/ \N1
%pratyāhāraḥ kathyate// manaḥ saṃsārān nivṛtya ātmani  sthāpyate// manomadhye ye vikārāḥ utpadyaṃte// tepi nivāraṇīyāḥ// anekacamatkārakāraṇī     buddhi  utpadyate// sāṃgopyāḥ// \D
%pratyāhāraḥ kathyate// manaḥ saṃsārān nivṛtya ātmani                                                        vāraṇīyāḥ// anekacamatkārakarakāraṇī buddhi  utpadyate   sāgopyāḥ/  \N2
%pratyāhāraḥ kathyate   manaḥ saṃsārān nivṛtyātmanī    sthāpyate   manomadhye ye vikārā  utpadyaṃte   tepi nivāraṇīyaḥ   anekacamatkāriṇī         buddhir utpadyate   sāgaupyā \U1 %%%286.jpg
%pratyāhāraḥ kathyate// manaḥ saṃsārān nivṛtyātmanī    sthāpyate// manomadhye ye vikārā  utpadyaṃte   tepi nivāraṇīyaḥ// anekacamatkāriṇī         buddhir utpadyate// sāgopyā// \U2
%-----------------------------
%Pratyāhāra [however] is taught. The mind is supposed to be turn away from the cyclic existence and caused to abide in the self. Changes within the mind arise, but they are kept off. The manyfold admirations the intellect generates are well hidden.    
%----------------------------
\note[type=source, labelb=257, lem={pratyāhāraḥ}]{SSP: pratyāhāram iti caitanyataraṅgānāṃ pratyāharaṇaṃ yathā nānāvikāragrasanotpannavikārasyāpi nivṛttiḥ nirbhātīti pratyāhāralakṣaṇam ||2.36||}
\note[type=source, labelb=258, lem={pratyāhāraḥ}]{Ysv (YK): ayaṃ pataṅgakāṣṭhāgnipratyāsena punaḥ punaḥ ||5|| kṛtvā kalevaraṃ śuddhaṃ kuryād yatnair mahātmanā | mano nivārya saṃsāre viṣayeṣu tathaiva ca ||6|| manovikārān sarvāś ca tyaktvā śūnyamayo bhavet | pratyāhāro bhavaty eṣu sarvanindācamatkṛtaḥ ||7||}
pratyāhāraḥ
\app{\lem[wit={ceteri}]{kathyate}
  \rdg[wit={E}]{pratyato}}/
manaḥ saṃsārā\skp{n-ni}\app{\lem[type=emendation, resp=egoscr, alt={nivṛtyātmani}]{\skm{n-ni}vṛtyātmani}
  \rdg[wit={B,L,P,U1,U2}]{\korr nivṛtyātmanī}
  \rdg[wit={E}]{nivartyātmani}
  \rdg[wit={D,N1,N2}]{nivṛtya ātmani}}
\app{\lem[wit={ceteri}]{sthāpyate}
  \rdg[wit={N2}]{\om}}/
manomadhye ye
\app{\lem[wit={ceteri}]{vikārā}
  \rdg[wit={P}]{vikāraḥ}
  \rdg[wit={D}]{vikārāḥ}
  \rdg[wit={N2}]{\om}}
\app{\lem[wit={ceteri}]{utpadyante}
  \rdg[wit={N2}]{\om}}/
 anekacama\skp{t-kā}\app{\lem[type=emendation, resp=egoscr, alt={°kārīṇi}]{\skm{t-kā}rīṇi}
   \rdg[wit={B,E,L,P,U1,U2}]{\korr kāriṇī}
   \rdg[wit={N1,N2}]{kārakarakāraṇī}
   \rdg[wit={D}]{kārakāraṇī}}
 \app{\lem[wit={ceteri},alt={buddhir}]{buddhi\skp{r-ut}}
   \rdg[wit={D,N1,N2}]{buddhi}
 }\app{\lem[wit={ceteri},alt={utpadyate}]{\skm{r-ut}padyate}
   \rdg[wit={E,B,D,U2}]{utpadyate |}
   \rdg[wit={P}]{utpadyataraṃ}}
 \app{\lem[type=emendation, resp=egoscr]{saṃgopyāḥ}
   \rdg[wit={D,N1}]{\korr sāṃgopyāḥ}
   \rdg[wit={N2}]{sāgopyāḥ}
   \rdg[wit={B,L,U2}]{sāgopyā}
   \rdg[wit={U1}]{sāgaupyā}
   \rdg[wit={E}]{sāṃgopāṃgaṃ}}/ 
%----------------------------
% dhyānaṃ ca bahutaraṃ prāg uktam/ tenātra       nocyate// \E XX! this one?
%                      prāg uktam  tenātra       nocyate  \P
% dhyānaṃ ca bahutaraṃ prāg uktam  tenātra       nocyate// \B 
% dhyānaṃ ca bahutaraṃ prāg uktam  tenātra       nocyate// \L %%%%0023.jpg
% dhyānaṃ ca bahutaraṃ      uktam  tena atra     nocyate/ \N1
% dhyānaṃ ca bahutaraṃ      uktaṃ  tena atra     nocyate// \D
% dhyānaṃ ca bahuttaraṃ     uktam  tenātra       nocyate// \N2
% dhyānaṃ    bahutaraṃ      uktaṃ  tena atra  na ucyate \U1 %%%286.jpg
% dhyānaṃ    bahutaraṃ prāg uktaṃ  tenātra       nocyate// \U2
%-----------------------------
%Dhyāna has been taught many times before. Because of that is not discussed here.
%-----------------------------
\note[type=source, labelb=259, lem={dhyānaṃ}]{SSP: atha dhyānam iti || asti kaś cana paramādvaitasya bhāvaḥ sa eva ātmeti yathā yadyat sphurati tattat svarūpam eveti bhāvayet sarvabhūteṣu samadṛṣṭiś ceti dhyānalakṣaṇam ||2.38||}
\note[type=source, labelb=260, lem={dhyānaṃ}]{Ysv (YK): dhyānan tu dvividhaṃ proktaṃ sthūlasūkṣmavibhedataḥ | sthūlaṃ mantramayaṃ viddhi sūkṣmantu mantravarjjitam | samādhir niścalā buddhiḥ śvāsocchvāsādivarjitaḥ |}
\app{\lem[wit={ceteri}]{dhyānaṃ}
  \rdg[wit={P}]{\om}}
\app{\lem[wit={ceteri}]{ca}
  \rdg[wit={P,U1,U2}]{\om}}
\app{\lem[wit={ceteri}]{bahutaraṃ}
  \rdg[wit={P}]{\om}}
\app{\lem[wit={B,E,L,P,U2},alt={prāg}]{prā\skp{g-u}}
  \rdg[wit={D,N1,N2,U1}]{\om}
}\app{\lem[wit={D,U1,U2},alt={uktaṃ}]{\skm{g-u}ktaṃ}
  \rdg[wit={E}]{uktam |}
  \rdg[wit={ceteri}]{uktam}}
\app{\lem[wit={ceteri}]{tenātra}
  \rdg[wit={D,N1,U1}]{tena atra}}
\app{\lem[wit={ceteri}]{nocyate}
  \rdg[wit={U1}]{na ucyate}}\dd{}
\end{prose}
\end{ekdosis}
%%%%%%%%%%%%%%
%%%%%%%%%%%%%%
%%%%%%%%%%%%%%
%%%%%%%%%%%%%
%%%%%%%%%%%%%%% 
\begin{ekdosis}
  \ekddiv{type=ed}
        \bigskip
  \centerline{\textrm{\small{[Internal and External Universe]}}}
      \bigskip
 \begin{prose}
%----------------------------
%idānīṃ piṃḍa-brahmāṃḍayor  aikyam asti    tasmāt   brahmāṇḍamadhye ye padārthās te pi     piṃḍamadhye santīti    kathyante/  \E %[P.48]
%idānīṃ piṃḍa-brahmāṃḍayor  aikyam asti    tasmād   brahmāṃḍamadhye ye padārthās te        piṃḍamadhye saṃti      kathyate    \P
%idānīṃ piṃḍa-brahmāṃḍayor  ekyam  asti//  tasmā    brahmāṃḍamadhye ye padārthās te        piṃḍamadhye sati       kathyate//  \B %%%%%%%%%%%%DSCN7168.jpg Z.2
%idānīṃ piṃḍa-brahmāṃḍayor  aikyam asti//  tasmāt   brahmāṃḍamadhye ye padārthās te        piṃḍamadhye saṃ        kathyaṃte// \L
%idānīṃ piḍa--brahmāḍayoḥ   aikyam asti//  tasmāt   brahmāṇḍamadhye ye padārthāḥ te pi     piṃḍamadhye saṃti// te kathyante// \N1
%idānīṃ piḍa--brahmāḍayoḥ   aikyam asti//  tasmāt   brahmāṇḍamadhye ye padārthāḥ te pi     piṃḍamadhye saṃti/  te kathyaṃte// \D
%idānīṃ piṇḍa-brahmāḍayoḥ   ekam   asti/   tasmānte brahmāṇḍamadhye ye padārthā  te pi     piṇḍamadhye saṃti   te kathyante// \N2
%idānīṃ piṇḍa-brahmāḍayor   aikam  asti    tasmāt   brahmāṇḍamadhye ye padārthā  sarve pi  piṇḍamadhye saṃti      kathyate    \U1
%idānīṃ piṇḍa-brahmāḍayor   aikam  asti//  tasmād   brahmāṇḍamadhye ye padārthās tanmadhye piṇḍamadhye sati       kathyaṃte// \U2
%-----------------------------
%Now there is the identity of the external universe and the body. Because of that, the objects which exist in the external universe are also in the body. They are taught.  
%----------------------------
   \note[type=source, labelb=261, lem={piṇḍa°}]{Ysv (PT): piṇḍabrahmāṇḍayor aikyaṃ śṛṇv idānīṃ prayatnataḥ | brahmāṇḍe santi ye cāṇḍāḥ piṇḍamadhye 'pi te sthitāḥ |}
   \note[type=testium, labelb=262, lem={piṇḍa°}]{SSP: piṇḍamadhye carācarau yo jānāti sa yogī piṇḍasaṃvittir bhavati||}
   \note[type=philcomm, labelb=263, lem={piṇḍa°}]{This section is not found in the quotes from the Ysv of the YK.}
   idānīṃ \app{\lem[wit={ceteri},alt={piṇḍa°}]{piṇḍa}
  \rdg[wit={D,N1}]{piḍa°}
}\app{\lem[wit={B,E,L,P},alt={brahmāṇḍayor}]{brahmāṇḍayo\skp{r-ai}}
  \rdg[wit={ceteri}]{°brahmāḍayoḥ}
}\app{\lem[wit={ceteri},alt={aikyam}]{\skm{r-ai}kya\skp{m-a}}
  \rdg[wit={B}]{ekyam}
  \rdg[wit={N2}]{ekam}
}\skm{m-a}sti/
\app{\lem[wit={ceteri},alt={tasmāt}]{tasmā\skp{t-bra}}
  \rdg[wit={B}]{tasmā}
  \rdg[wit={N2}]{tasmānte}}
\skm{t-bra}hmāṇḍamadhye ye
\app{\lem[wit={ceteri},alt={padārthās}]{padārthā\skp{s-te}}
  \rdg[wit={D,N1}]{padārthāḥ}
  \rdg[wit={N2,U1}]{padārthā}
}\app{\lem[wit={ceteri}, alt={te 'pi}]{\skm{s-te} 'pi}
  \rdg[wit={B,L,P}]{te}
  \rdg[wit={U1}]{sarve pi}
  \rdg[wit={U2}]{tanmadhye}}
piṇḍamadhye
\app{\lem[wit={ceteri}]{santi}
  \rdg[wit={E}]{santīti}
  \rdg[wit={B,U2}]{sati}
  \rdg[wit={L}]{saṃ°}}
\app{\lem[wit={D,N1,N2}]{te}
  \rdg[wit={ceteri}]{\om}}
\app{\lem[wit={ceteri}]{kathyante}
  \rdg[wit={B,P,U1}]{kathyate}}/ 
%----------------------------
%padas   tale        talaṃ           varttate/ pādopari talātalaṃ varttate/                         gulphayor mahātalaṃ   varttate/ jaṃghāmadhye sutalaṃ varttate/  jānumadhye   vitalaṃ varttate/ ūrvormadhye'talaṃ varttate// \E %[P.48]
%pādayos tele        talaṃ           varttate  pādopari talātalaṃ vartate    pādopari talaṃ vartate gulphayor mahātalaṃ  varttate                                   jānumadhye   vitalaṃ           ūrvormadhye atalaṃ \P
%pādayas talās       talaṃ           vartate// pādopari talātalaṃ vartate/                          gulphayor mahātalaṃ  vartate//  jaṃghāmadhye stutalaṃ vartate// jānubhyāṃ    vitalaṃ vartate// ūrvo madhye atalaṃ vartate//\7168.jpg Z.2
%pādayos talās       talaṃ           vartate// pādopari talātalaṃ vartate//                         gulphayor mahātalaṃ  vartate//  jaṃghāmadhye sutalaṃ varttate// jānubhyāṃ    vitalaṃ vartate// ūrvormadhye atalaṃ vartate// \L
%padayor aṃguṣṭale   talaṃ           varttate/ tādupari talātalaṃ varttate/                         gulpho parimahātalaṃ varttate/  jaṃghāmadhye sutalaṃ/           jānvomadhye  vitalaṃ/          ūrvormadhye atalaṃ//         \N1
%padayor aṃguṣṭale   talaṃ           varttate/ tādupari talātalaṃ varttate//                        gulpho parimahātalaṃ varttate// jaṃghāmadhye sutalaṃ//          jānvormadhye vitalaṃ//         ūrvormadhye atalaṃ//         \D
%padayor aṃguṣṭale   talaṃ           varttate  tādupari talātalaṃ varttate/                         gulpho parimahātalaṃ varttate   jaṃghāmadhye sutalaṃ/           jānvomadhye  vitalaṃ/          ūrvormadhye atalaṃ//         \N2
%pādayor aṃguṣṭatale talaṃ ca        vartate   taduparī talātalaṃ varttate                          gulpho parimahātalaṃ varttate   jaṃghāmadhye sutalaṃ            jānvormadhye vitalaṃ           ūrvormadhye atalaṃ           \U1
%pādayoṃguṣṭatale    mūlaṃ rasātalāt vartate// pādopari talātalaṃ varttate//                        gulphayor  mahātalaṃ varttate// jaghāmadhye  sutalaṃ vartate//  jānumadhye   vitalaṃ//         ūrvormadhye atalaṃ//         \U2
%-----------------------------
%Talam exists at the base of the big toe[s] of the feet. On top of the feet exists Talātala. Mahātala exists at the two ankles. Sutala exists in the center of the lower part of the leg between ankle and knee. Vitala exists in the middle of the knee. Atala exists in the middle of the two thighs.   
%----------------------------
\note[type=source, labelb=264, lem={talaṃ}]{Ysv (PT): talaṃ pādāṅguṣṭhatale tasyopari talātalam | mahātalaṃ gulphayor madhye gulphopari rasātalam | sutalaṃ jaṅghayor madhye vitalaṃ jānumadhyakam | ūrvormadhye 'talaṃ proktaṃ saptapātālam īritam | talaṃ talātalañ ceti mahātalarasātalam | saptapātālam etat tu sutalaṃ vitalātalam |}
\note[type=testium, labelb=265, lem={talaṃ}]{SSP 3.2: kūrmaḥ pādatale vasati pātālaṃ pādāṅguṣṭhe talātalam aṅguṣṭhāgre mahātalaṃ pṛṣṭhe rasātalaṃ hulphe sutalaṃ jaṅghāyāṃ vitalaṃ jānvoḥ atalam urvor evaṃ saptapātālaṃ rudradevatādhipatye tiṣṭhati piṇḍamadhye krodharūpī bhāvaḥ sa eva kālāgnirudraḥ mahātalaṃ pādapṛṣthe}
\app{\lem[wit={ceteri},alt={pādayor}]{pādayo\skp{r-a}}
  \rdg[wit={E}]{padas}
  \rdg[wit={P,L}]{pādayos}
  \rdg[wit={B}]{pādayas}
  \rdg[wit={U2}]{pādayo°}
}\app{\lem[type=emendation, resp=egoscr,alt={aṅguṣṭatale}]{\skm{r-a}ṅguṣṭatale}
  \rdg[wit={U1}]{\korr aṃguṣṭatale}
  \rdg[wit={D,N1,N2}]{aṃguṣṭale}
  \rdg[wit={U2}]{°ṃguṣṭatale}
  \rdg[wit={B,L}]{tālas}
  \rdg[wit={P}]{tele}
  \rdg[wit={E}]{tale}}
\app{\lem[wit={ceteri}]{talaṃ}
  \rdg[wit={U1}]{talaṃ ca}
  \rdg[wit={U2}]{mūlaṃ rasātalāt}}
vartate/
\app{\lem[type=emendation, resp=egoscr]{tadupari}
  \rdg[wit={U1}]{\korr taduparī}
  \rdg[wit={D,N1,N2}]{tādupari}
  \rdg[wit={B,E,L,P,U2}]{pādopari}}
talātalaṃ
\app{\lem[wit={ceteri}]{vartate}
  \rdg[wit={P}]{vartate | pādopari talaṃ vartate}}/
\app{\lem[wit={B,E,L,P,U2},alt={gulphayor}]{gulphayo\skp{r-ma}}
  \rdg[wit={D,N1,N2,U1}]{gulpho}
}\app{\lem[wit={B,E,L,P,U2},alt={mahātalaṃ}]{\skm{r-m}ahātalaṃ}
  \rdg[wit={D,N1,N2,U1}]{parimahātalaṃ}}
vartate/
\end{prose}
\end{ekdosis}
\ekdpb*{}
%%%%%%%%%%%%%%%%%%%%%%%%%%%%%%%%%%%%%%%%%%
%%%%%%%%%%%%%%%%%%%%%%%%%%%%%%%%%%%%%%%%%%
%%%%%%%%PAGEBREAK%%%%%%%PAGEBREAK%%%%%%%%%
%%%%%%%%%%%%%%%%%%%%%%%%%%%%%%%%%%%%%%%%%%
%%%%%%%%%%%%%%%%PAGEBREAK%%%%%%%%%%%%%%%%%
%%%%%%%%%%%%%%%%%%%%%%%%%%%%%%%%%%%%%%%%%%
%%%%%%%%PAGEBREAK%%%%%%%PAGEBREAK%%%%%%%%%
%%%%%%%%%%%%%%%%%%%%%%%%%%%%%%%%%%%%%%%%%%
%%%%%%%%%%%%%%%%%%%%%%%%%%%%%%%%%%%%%%%%%%
%%%%%%%%%%%%%%%%%%%%%%%%%%%%%%%%%%%%%%%%%%
%%%%%%%%%%%%%%%%%%%%%%%%%%%%%%%%%%%%%%%%%%
%%%%%%%%PAGEBREAK%%%%%%%PAGEBREAK%%%%%%%%%
%%%%%%%%%%%%%%%%%%%%%%%%%%%%%%%%%%%%%%%%%%
%%%%%%%%%%%%%%%%PAGEBREAK%%%%%%%%%%%%%%%%%
%%%%%%%%%%%%%%%%%%%%%%%%%%%%%%%%%%%%%%%%%%
%%%%%%%%PAGEBREAK%%%%%%%PAGEBREAK%%%%%%%%%
%%%%%%%%%%%%%%%%%%%%%%%%%%%%%%%%%%%%%%%%%%
%%%%%%%%%%%%%%%%%%%%%%%%%%%%%%%%%%%%%%%%%%
%%%%%%%%%%%%%%%%%%%%%%%%%%%%%%%%%%%%%%%%%%
%%%%%%%%%%%%%%%%%%%%%%%%%%%%%%%%%%%%%%%%%%
%%%%%%%%PAGEBREAK%%%%%%%PAGEBREAK%%%%%%%%%
%%%%%%%%%%%%%%%%%%%%%%%%%%%%%%%%%%%%%%%%%%
%%%%%%%%%%%%%%%%PAGEBREAK%%%%%%%%%%%%%%%%%
%%%%%%%%%%%%%%%%%%%%%%%%%%%%%%%%%%%%%%%%%%
%%%%%%%%PAGEBREAK%%%%%%%PAGEBREAK%%%%%%%%%
%%%%%%%%%%%%%%%%%%%%%%%%%%%%%%%%%%%%%%%%%%
%%%%%%%%%%%%%%%%%%%%%%%%%%%%%%%%%%%%%%%%%%
  \begin{ekdosis}
    \begin{prose}
      \noindent
    \app{\lem[wit={ceteri},alt={jaṅghā°}]{jaṅghā}
  \rdg[wit={U2}]{jaghā°}
  \rdg[wit={P}]{\om}}madhye
\app{\lem[wit={ceteri}]{sutalaṃ}
  \rdg[wit={B}]{stutalaṃ}
  \rdg[wit={P}]{\om}}
\app{\lem[wit={B,E,L,U2}]{vartate}
  \rdg[wit={ceteri}]{\om}}/
\app{\lem[wit={D,U1}]{jānvormadhye}
  \rdg[wit={N1,N2}]{jānvomadhye}
  \rdg[wit={E,P,U2}]{jānumadhye}
  \rdg[wit={B,L}]{jānubhyāṃ}}
vitalaṃ
\app{\lem[wit={E,B,L}]{vartate}
  \rdg[wit={ceteri}]{\om}}/ 
ūrvormadhye
\app{\lem[wit={E}]{'talaṃ}
  \rdg[wit={ceteri}]{atalaṃ}} 
\app{\lem[wit={E,L,B}]{vartate}
  \rdg[wit={ceteri}]{\om}}/
\end{prose}
\end{ekdosis}
\begin{ekdosis}
    \bigskip
    \centerline{\textrm{\small{[Triad of Worlds]}}}
    \bigskip
     \begin{prose}
%----------------------------
%idānīṃ                    śarīramadhye lokatrayaṃ kathyate/  mūlādhāre bhūrlokaḥ/  liṃgāgre  bhuvarlokaḥ/  liṃgamadhye  svarlokaḥ//    \E
%idānīṃ                    piṃḍamadhye  lokatrayaṃ kathyate   mūlādhāre bhūrlokaḥ   liṃgāgre  bhuvarlokaḥ   liṃgamūle    svarlokaḥ      \P
%idānīṃ                    piḍopiri     lokatrayaṃ kathyate// mūlādhāre bhūrlokaḥ   liṃgāgre  bhuvarloka----liṃgamadhye  svarlokaḥ//    \B
%idānīṃ                    piṃḍopari    lokatrayaṃ kathyate// mūlādhāre bhūrlokaḥ// liṃgāgre  bhuvarloka----liṃgamadhye  svarlokaḥ//   \L
%idānīṃ                    piṃḍamadhye  lokatrayaṃ kathyate/  mūlādhāre bhūrlokaḥ/  liṃgamūle                            svarlokaḥ     \N1
%idānīṃ                    piṃḍamadhye  lokatrayaṃ kathyate// mūlādhāre bhūrlokaḥ// liṃgāgre  bhuvarlokaḥ// liṃgamadhye  svarlokaḥ//   \D
%idānīṃ                    piṃḍamadhye  lokatrayaṃ kathyate/  mūlādhāre bhūrlokaḥ   liṃgamūle                            svargalokaḥ// \N2
%idānīṃ upari tataṃ  lokaṃ piṃḍamadhye  lokatrayaṃ kathyate   mūlādhāre bhūrlokaḥ   liṃgāgre  bhuvarlokaḥ   liṃgamūle    svaravarlokaḥ \U1
%idānīṃ                    piṃḍamadhye  lokatrayaṃ kathyate// mūlādhāre bhūrlokaḥ// liṃgāgre  bhuvarlokaḥ// liṃgamūle    svarlokaḥ//   \U2
%-----------------------------
%Now the threefold world within the body is taught. The earthen world is situated at the Root-cotainer (\textit{mūladhāra}). The air world is at the root of the gender. In the center of the gender is the heavenly world. 
%----------------------------
\note[type=source, labelb=266, lem={piṇḍamadhye}]{Ysv\textsuperscript{PT}: idānīṃ piṇḍamadhye tu saptalokaṃ śṛṇu priye | mūlādhāre tu bhūrloko liṅgāgre tu bhuvas tataḥ | svarloko liṅgamūle tu merumūle mahas tathā |}
\note[type=testium, labelb=267, lem={bhūrlokaḥ}]{SSP 3.3: bhūrlokaṃ guhyasthāne bhuvarlokaṃ liṅgasthāne svarlokaṃ nābhisthāne evaṃ lokatraye indro devatā piṇḍamadhye sarvendriyaniyāmakaḥ sa evendraḥ||}
\app{\lem[wit={ceteri}]{idānīṃ}
  \rdg[wit={U1}]{idānīṃ upati tataṃ lokaṃ}}
\app{\lem[wit={ceteri}]{piṇḍamadhye}
  \rdg[wit={L}]{piṃḍopari}
  \rdg[wit={B}]{piḍopiri}
  \rdg[wit={E}]{śarīramadhye}}
lokatrayaṃ kathyate/ \\
mūlādhāre bhūrlokaḥ/
\app{\lem[wit={ceteri}]{liṅgāgre}
  \rdg[wit={N1,N2}]{liṃgamūle}}
\app{\lem[wit={D,E,P,U1,U2}]{bhuvarlokaḥ}
  \rdg[wit={B,L}]{bhuvarloka°}
  \rdg[wit={N1,N2}]{\om}}/
\app{\lem[wit={ceteri}]{liṅgamadhye}
  \rdg[wit={P,U1,U2}]{liṃgamūle}
  \rdg[wit={N1,N2}]{\om}}
\app{\lem[wit={ceteri}]{svarlokaḥ}
  \rdg[wit={N2}]{svargalokaḥ}
  \rdg[wit={U1}]{svaravarlokaḥ}}\dd{}
\end{prose}
\end{ekdosis}
\begin{ekdosis}
    \bigskip
    \centerline{\textrm{\small{[Tetrad of Worlds]}}}
    \bigskip
    \begin{prose}
%----------------------------
%idānīm   uparitanaṃ lokacatuṣka      kathyate/  pṛṣṭhadaṃḍāṃkure  maharlokaḥ/   daṇḍacchidramadhye janalokaḥ/  taddaṇḍanāḍīmadhye   tapolokaḥ/  daṇḍamalamadhye   satyalokaḥ/ \E
%idānīm   uparitanu--lokacatuṣkaṃ     kathyate   pṛṣṭhadaṃḍākūre   maharlokaḥ    daṃḍaschidramadhye janalokaḥ   taddaṃḍanālimadhye   tapolokaḥ   daṃḍakamalamadhye satyalokaḥ \P
%idānīm   uparitanu--lokaḥ catuṣṭayaṃ kathyate// daṃḍaṣṭaṭheṃskure maharlokā/    daṇḍachidramadhye  janaloka    taddaṃḍanālikāmadhye ..polokaḥ   daṇḍakamalamadhye satyalokaḥ// \B
%idānīm   uparitana--lokaḥ catuṣṭayaṃ kathyate// daṃḍaṣṭaṭheṃkure  maharlokaḥ/   daṇḍachidramadhye  janaloka    taddaṃḍatālikāmadhye tapolokaḥ   daṇḍakamalamadhye satyalokaḥ// \L
%idānīṃ   uparijanaṃ lokacatuṣkaṃ     kathyate/  pṛṣṭhadaṃḍāṃkure  maharllokaḥ/  daṇḍacchidramadhye janalokaḥ/  taddaṇḍanālī \om                                               \N1!!!!!!!!!!!!!!!important omission stemmapoint S.11 verso
%idānīṃ// uparitanaṃ lokacatuṣkaṃ     kathyate// pṛṣṭhadaṃḍāṃkure  maharlokaḥ    daṇḍachidramadhye  janalokaḥ// taddaṇḍanālamadhye   tapolokaḥ// daṇḍakamalamadhye satyalokaḥ// \D
%idānīṃ   uparijanaṃ lokacatuṣkaṃ     kathyate// pṛṣṭhadaṃḍākūle   maharllokaḥ/  uchidramadhye      janalokaḥ/  taddaṇḍanālī                                              \om  \N2 !!!!!!!!!!!!!!!!!!important omission stemmapoint
%idānīṃ   uparitanaṃ lokaṃ catuṣkaṃ   kathyate   pṛṣṭhadaṃḍāṃkure  maharlokaḥ    daṃḍasthitamadhye  janalokaḥ   taddaṇḍanāḍīmadhye   tapolokaḥ   daṇḍamalamadhye   satyalokaḥ \U1
%idānīṃ   uparitana--lokacatuṣkaṃ     kathyate// pṛṣṭhadaṃḍāṃkure  maharlokaḥ//  daṃḍachidramadhye  janalokaḥ// daṇḍanālimadhye      tapolokaḥ// daṇḍakamalamadhye satyalokaḥ// \U2
%-----------------------------
%Now the quadruplet of worlds will be taught. The great world is at the shoot of the staff of the spine. The world of men is in the centre of the cavity of the spine. In the centre of the tube of that spine is the world of heat?. In the center of the lotus of the spine
%-----------------------------
idānīṃ
\app{\lem[wit={D,E,U1}]{uparitanaṃ}
  \rdg[wit={L,U2}]{uparitana°}
  \rdg[wit={N1,N2}]{uparijanaṃ}
  \rdg[wit={P,B}]{uparitanu°}}
\app{\lem[wit={P,D,N1,N2,U2}]{lokacatuṣkaṃ}
  \rdg[wit={E}]{lokacatuṣka}
  \rdg[wit={B,L}]{lokaḥ catuṣṭayaṃ}
  \rdg[wit={U1}]{lokaṃ catuṣkaṃ}}
kathyate/ \\
\note[type=testium, labelb=268, lem={lokacatuṣkaṃ}]{SSP 3.4: daṇḍāṅkure maharlokaḥ daṇḍakuhare jano lokaḥ daṇḍanāle tapo lokaḥ mūlakamale satyalokaḥ evaṃ lokacatuṣṭaye brahmādidevatā piṇḍamadhye anekamānābhimānasvarūpī tiṣṭhati||}
\note[type=source, labelb=269, lem={lokacatuṣkaṃ}]{Ysv\textsuperscript{PT}: merucchidre janoloko merunāḍyāṃ tapas tathā | kamale marttyalokas tu iti lokaḥ pṛthak pṛthak | bhūrbhuvaḥsvarmahaś ceti janaś caiva tapas tathā | saptamaḥ satyalokas tu saptaloka iti smṛtaḥ | saptalokais tu pātālair bhuvanāni caturdaśa |}
\app{\lem[wit={ceteri}]{pṛṣṭhadaṇḍāṅkure}
  \rdg[wit={N2}]{pṛṣṭhadaṃḍākūle}
  \rdg[wit={P}]{pṛṣṭhadaṃḍākūre}
  \rdg[wit={B}]{daṃḍaṣṭaṭheṃskure}
  \rdg[wit={L}]{daṃḍaṣṭaṭheṃkure}}
 maha\skp{r-lo}\app{\lem[wit={ceteri},alt={°lokaḥ}]{\skm{r-lo}kaḥ}
   \rdg[wit={B}]{°lokā}}/
 \app{\lem[wit={ceteri}, alt={daṇḍachidra°}]{daṇḍachidra}
   \rdg[wit={P}]{daṃḍaschidra}
   \rdg[wit={U1}]{daṃḍasthita}
   \rdg[wit={U2}]{uchidra}}madhye
 \app{\lem[wit={ceteri}]{janalokaḥ}
   \rdg[wit={B,L}]{janaloka}}/
\app{\lem[wit={ceteri},alt={taddaṇḍa°}]{taddaṇḍa}
  \rdg[wit={U2}]{daṇḍa°}
}\app{\lem[wit={E,U1},alt={°nāḍīmadhye}]{nāḍīmadhye}
  \rdg[wit={P,U2}]{nālimadhye}
  \rdg[wit={B}]{nālikāmadhye}
  \rdg[wit={L}]{tālikāmadhye}
  \rdg[wit={B}]{nālamadhye}
  \rdg[wit={N1,N2}]{nālī}}
\note[type=philcomm, labelb=270, lem={nāḍīmadhye}]{At this point of the text a huge gab of approximately 25\% of the full text starts in the two important and most reliable witnesses of \textit{Yogatattvabindu}. The two Nepalese manuscripts N\textsubscript{1} and N\textsubscript{2} indicate a large gap in their template, which makes it more than clear that N\textsubscript{1} and N\textsubscript{2} stematically belong closely together. They are undoubtedly either direct copies of each other or copies of the same template. The omissions of the reading of N\textsubscript{1} and N\textsubscript{2} will not be recorded in the apparatus until after their gap.}
\app{\lem[wit={ceteri}]{tapolokaḥ}
  \rdg[wit={B}]{polokaḥ}}/ \\
 daṇḍa\app{\lem[wit={ceteri},alt={°kamalamadhye}]{kamalamadhye}
   \rdg[wit={E,U1}]{°malamadhye}}
 satyalokaḥ/ 
 \end{prose}
\end{ekdosis}
\ekdpb*{}
%%%%%%%%%%%%%%%%%%%%%%%%%%%%%%%%%%%%%%%%%%
%%%%%%%%%%%%%%%%%%%%%%%%%%%%%%%%%%%%%%%%%%
%%%%%%%%PAGEBREAK%%%%%%%PAGEBREAK%%%%%%%%%
%%%%%%%%%%%%%%%%%%%%%%%%%%%%%%%%%%%%%%%%%%
%%%%%%%%%%%%%%%%PAGEBREAK%%%%%%%%%%%%%%%%%
%%%%%%%%%%%%%%%%%%%%%%%%%%%%%%%%%%%%%%%%%%
%%%%%%%%PAGEBREAK%%%%%%%PAGEBREAK%%%%%%%%%
%%%%%%%%%%%%%%%%%%%%%%%%%%%%%%%%%%%%%%%%%%
%%%%%%%%%%%%%%%%%%%%%%%%%%%%%%%%%%%%%%%%%%
%%%%%%%%%%%%%%%%%%%%%%%%%%%%%%%%%%%%%%%%%%
%%%%%%%%%%%%%%%%%%%%%%%%%%%%%%%%%%%%%%%%%%
%%%%%%%%PAGEBREAK%%%%%%%PAGEBREAK%%%%%%%%%
%%%%%%%%%%%%%%%%%%%%%%%%%%%%%%%%%%%%%%%%%%
%%%%%%%%%%%%%%%%PAGEBREAK%%%%%%%%%%%%%%%%%
%%%%%%%%%%%%%%%%%%%%%%%%%%%%%%%%%%%%%%%%%%
%%%%%%%%PAGEBREAK%%%%%%%PAGEBREAK%%%%%%%%%
%%%%%%%%%%%%%%%%%%%%%%%%%%%%%%%%%%%%%%%%%%
%%%%%%%%%%%%%%%%%%%%%%%%%%%%%%%%%%%%%%%%%%
%%%%%%%%%%%%%%%%%%%%%%%%%%%%%%%%%%%%%%%%%%
%%%%%%%%%%%%%%%%%%%%%%%%%%%%%%%%%%%%%%%%%%
%%%%%%%%PAGEBREAK%%%%%%%PAGEBREAK%%%%%%%%%
%%%%%%%%%%%%%%%%%%%%%%%%%%%%%%%%%%%%%%%%%%
%%%%%%%%%%%%%%%%PAGEBREAK%%%%%%%%%%%%%%%%%
%%%%%%%%%%%%%%%%%%%%%%%%%%%%%%%%%%%%%%%%%%
%%%%%%%%PAGEBREAK%%%%%%%PAGEBREAK%%%%%%%%%
%%%%%%%%%%%%%%%%%%%%%%%%%%%%%%%%%%%%%%%%%%
\begin{ekdosis}
    \bigskip
    \centerline{\textrm{\small{[Four Lords of the Worlds]}}}
    \bigskip
    \begin{prose}
      \noindent
%----------------------------
%atha brahmāṇḍamadhye caturdaśa-lokāni sthānāni tānyapi piṃḍe varttante// \E
%atha brahmāṇḍamadhye caturdaśa-lokāsthānāni    tānyapi piḍe  varttate  \P
%atha brahmāṇḍamadhye caturdaśa-lokasthānānī    tānyapi piṃḍo vartate... \B
%atha brahmāṇḍamadhye caturdaśa-lokasthānānī    tānyapi piṃḍe vartate... \L
%\om                                                                 \N1
%atha brahmāṇḍamadhye catvāro   lokasvāminaḥ//  te pi piṃḍamadhye varttate \D %%%p. 14 recto
%\om                                                                 \N2
%atha brahmāṇḍamadhye catvāro   lokāḥ svāminaḥ  te pi piṃḍamadhye vartate \U1
%atha brahmāṇḍamadhye caturddaśalokāḥ stānāni// tānyapi piṃḍe vartate// \U2 %%418.jpg
%-----------------------------
%Now the locations of the fourteen worlds within the universe exist in the body.
%Now the four lords of the worlds of the external universe also exist in the internal universe.       
%-----------------------------
\note[type=source, labelb=270, lem={catvāro}]{Ysv\textsuperscript{PT}: atha brahmāṇḍamadhyasthāś catvāro lokapālakāḥ |}
\note[type=philcomm, labelb=271 , lem={catvaro}]{Only the reading of witness D and U\textsubscript{1} is plausible and has to be considered as \textit{lectio dificilior}. This is confirmed by the reading of the source text, the Ysv\textsuperscript{PT} introducing the \textit{lokapālakāḥ} which become rewritten by Rāmacandra to \textit{lokasvāminah̤}. In the transmission of the text within the E,N,L,P and U\textsubscript{2}-group this subject has not been properly understood and in order to fix it the passage was rewritten, which probably resulted in the introduction of the \textit{caturdaśalokāsthānāni}.}
atha brahmāṇḍamadhye
\app{\lem[wit={D,U1}]{catvāro}
  \rdg[wit={ceteri}]{caturdaśa°}}
\app{\lem[wit={D}]{lokasvāminaḥ}
  \rdg[wit={U1}]{lokāḥ svāminaḥ}
  \rdg[wit={P,B,L}]{°lokāsthānāni}
  \rdg[wit={U2}]{°lokāḥ stānāni}
  \rdg[wit={E}]{°lokāni sthānāni}}/
\app{\lem[wit={E,U1}]{te 'pi}
  \rdg[wit={ceteri}]{tānyapi}}
\app{\lem[wit={E,U1}]{piṇḍamadhye}
  \rdg[wit={B,E,L,U2}]{piṇḍe}
  \rdg[wit={P}]{piḍe}}
\app{\lem[wit={E}]{vartante}
  \rdg[wit={ceteri}]{vartate}}/
%----------------------------
%śarīramadhye  dvau kukṣī  dve sakthinī   vakṣaḥsthalaṃ   kaṃṭhamūlaṃ    kaṃṭhamadhyaṃ laṃbikāmūlaṃ   tāludvāraṃ tālumadhyaṃ     lalāṭamadhye   śṛṃgāṭikā    kapolamadhye   kamalinīmadhye   brahmaraṃdhra             kamalinya---strikūṭasthānam/ \E
%śarīramadhye  dvau kukṣī  dve sakṭhi??nī vakṣaḥ schalaṃ  kaṃṭhamūlaṃ    kaṃṭhamadhyaḥ laṃbikāmūlaṃ   tāludvāraṃ tālumadhye      lalāṭamadhyaṃ  śṛṃgāṭikā    kapolamadhye   kamalinīmadhye   brahmaraṃdhraṃ   ūrddhvaṃ kamalinyā   strikūṭasthānam \P %%%7658.jpg
%śarīramadhye//dvau kukṣau dve sakṭhinī   vakṣaḥsthalaṃ   kaṃṭhamūlaṃ    kamardhye     laṃbikāmūlaṃ   tāludvāraṃ tālumadhyaṃ     lalāṭamadhyaṃ//śṛṃgāṭikā//  kapolamadhye// kamalinīmadhyaṃ  brahmaraṃdhraṃ            kamalīnyāṃ  strikūṭasthānam// \B
%śarīramadhye  dvau kukṣau dve sakthinī   vakṣasthalaṃ    kaṃṭhamūle     kaṃṭhamadhyaṃ laṃbikāmūlaṃ   tāludvāraṃ tālamadhyaṃ     lalāṭamadhyaṃ  śṛṃgāṭikā    karālamadhye   kamalinīmadhyaṃ  brahmaraṃdhraṃ   ūrdhvaṃ  kamalīnyā   trikūṭasthānam... \L
%\om                                                                 \N1
%śarīramadhye  dvau kukṣīnau vartatte//   vakṣasthale//   kaṃṭhasya mūle kaṃṭhamadhye  laṃbikāyāmūle/ tāludvāre tālumadhye       lalāṭe//       śṛṃgāṭikāyāṃ kapolamadhye// kamalinīmadhye// brahmaraṃdhre//  ūrdhva---kamalinyaḥ//trikūṭasthāne// saptapātāle//\D
%\om                                                                 \N2
%śarīramadhye  dvau kukṣīṇau    varttate  vakṣasthale     kaṃṭhasya mūle kaṃṭhamadhye   laṃbikāyāmūle tāludvāre tālumadhye       lalāṭe         śṛṃgāṭikāyāṃ kapolamadhye   kamalinīmadhye   brahmaraṃdhre    urdhva---kamalinyaḥ  trikūṭasthāne... \U1  %%%%287.jpg
%śarīramadhye  dvau kukṣī  dve sakthinī// vakṣassthalaṃ/  kaṃṭhamūle//   kaṃṭhamadhyaḥ//laṃbikāmūlaṃ//tāludvāraṃ// tālumadhyaṃ// lalāṭamadhyaṃ// śrṛṃgāṭikā  kapolamadhye// kamalinīmadhye// brahmaraṃdhraṃ// urdhva---kamalinyās  trikūṭasthānam// \U2
%-----------------------------
%Within the body in the two cavities (1), within the two thighs (2), at the location of the chest (3), at the root of throat (4), in the center of the throat (5) at the root of the uvula (6) at the entrance of the palate (7) at the forehead (8) at the crossroad of the center of the cheecks (9), at the center of the lotuspond (10?), at the aperture of Brahman (11), at the place of the three peaks above the lotusses (14), in the seven hells (21) \ldots
%----------------------------
\note[type=source, labelb=272, lem={śarīramadhye}]{Ysv\textsuperscript{PT}: piṇḍamadhye tu tān jñātvā sarvasiddhīśvaro bhavet | indro brahmā viṣṇurīśaścatvāraś cātmadevatāḥ | mūlādhāre catuṣpatre gajārūḍho mahān iti | sṛṣṭikarttā ca tatraiva svādhiṣṭhāne mahān hariḥ | maṇipūre śūlapāṇiraṣṭasiddhīśvaro mahān | tāludvāre tālumadhye lalāṭe vakṣakaṇṭhake | śṛṅgāṭikā kapāle ca lambikā brahmarandhrake | navacakram ūrddhvacakrañ ca trikūṭety ekaviṃśatiḥ | brahmāṇḍāni vasantīti jñātavyāni prayatnataḥ |}
\note[type=source, labelb=273, lem={kukṣau}]{SSP 3.5: viṣṇulokaḥ kukṣau tiṣṭhati tatra viṣṇur devatā piṇḍamadhye aṇekavyāpārakārako bhavati| hṛdaye rudralokaḥ tatra rudro devatā piṇḍamadhye ugrasvarūpī tiṣṭhati | vakṣaḥsthale īśvaralokaḥ tatra īśvaro devatā piṇḍamadhye tṛptisvarūpī tiṣṭhati | kaṇṭhamadhye nīlakaṇṭho lokas tatra nīlakaṇṭho devatā piṇḍamadhye nityaṃ tiṣṭhati | tāludvāre śivalokas tatra śivo devatā piṇḍamadhye 'nupamasvarūpī tiṣṭhati| lambikāmūle bhairavalokas tatra bhairavo devatā piṇḍamadhye sarvottamasvarūpī tiṣṭhati | lalāṭamadhye 'nādilokas tatrānādidevatā piṇḍamadhye ānandaparāhantāsvarūpī tiṣṭhati | śṛṅgāre kulalokas tatra kuleśvaro devatā piṇḍamadhye ānandasvarūpī tiṣṭhati | śaṃkhamadhye nalinīsthāne akuleśvaro devatā piṇḍamadhye nirabhimānāvasthā tiṣṭhati | brahmarandhre parabrahmalokas tatra parabrahma devatā piṇḍamadhye paripūrṇadaśā tiṣṭhati | ūrdhvakamale parāparalokas tatra parameśvaro devatā piṇḍamadhye parāparabhāvas tiṣṭhati | trikūṭasthāne śaktilokas tatra parāśaktir devatā sarvasaṃ sarvakartṛtvāvasthā tiṣṭhati| evaṃ piṇḍamadhye saptapātālasahitaikaviṃśatibrahmāṇḍasthānavicāraḥ |}
śarīramadhye
\app{\lem[type=conjecture, resp=egoscr]{dvāyoḥ kukṣayoḥ}
  \rdg[wit={E,P,U2}]{\conj dvau kukṣī}
  \rdg[wit={B,L}]{dvau kukṣau}
  \rdg[wit={D}]{dvau kukṣīnau}
  \rdg[wit={U1}]{dvau kukṣīṇau}}\dd{}
\app{\lem[type=conjecture, resp=egoscr]{dvāyoḥ sakthinoḥ}
  \rdg[wit={E,L,U2}]{\conj dve sakthinī}
  \rdg[wit={P,B}]{dve sakṭhinī}
  \rdg[wit={D,U1}]{vartate}}\dd{}
\note[type=philcomm, labelb=273, lem={śarīramadhye}]{This passage which lists the 21 locations is very problematic. The accusatives preserved in E,N,L,P and U\textsubscript{2} are clearly an attempt to fix the text according to the rewriting of the previous \textit{caturdaśalokāsthānāni}-sentence, which is seen also in the limitation of the elements of the list in those witnesses from 21 to just 14. It is more likely that the locatives in D and \textsubscript{1} are original. Since the text promises to account for 21 locations which all seem to have been listed as locatives, my best guess is to conjecture two more locatives for the cavities (\textit{dvāyoḥ kukṣau}) and for the two thighs (\textit{dvāyoḥ sakthinoḥ}) in order to arrive at a grammatically correct text and to accept the reading for the final seven locations given as \textit{saptapālāle} which is only preserved in witness D.}
\app{\lem[type=emendation, resp=egoscr]{vakṣaḥsthale}
  \rdg[wit={D,U1}]{\korr vakṣasthale}
  \rdg[wit={E,B}]{vakṣaḥ sthalaṃ}
  \rdg[wit={P}]{vakṣaḥschalaṃ}
  \rdg[wit={U2}]{vakṣassthalaṃ}}
\app{\lem[wit={L,U2}]{kaṇṭhamūle}
  \rdg[wit={E,P,B}]{kaṃṭhamūlaṃ}
  \rdg[wit={D,U1}]{kaṃṭhasya mūle}}\dd{}
\app{\lem[wit={D,U1}]{kaṇṭhamadhye}
  \rdg[wit={B}]{kamardhye}
  \rdg[wit={E,L}]{kaṃṭhamadhyaṃ}
  \rdg[wit={P,U2}]{kaṃṭhamadhyaḥ}}\d{}
\app{\lem[type=emendation, resp=egoscr]{lambikāmūle}
  \rdg[wit={D,U1}]{\korr laṃbikāyā mūle} %laṃbikāyām mūle = locatives
  \rdg[wit={ceteri}]{laṃbikāmūlaṃ}}\dd{}
\app{\lem[wit={D,U1}]{tāludvāre}
  \rdg[wit={ceteri}]{tāludvāraṃ}}\dd{}
\app{\lem[wit={D,U1}]{tālumadhye}
  \rdg[wit={ceteri}]{tālumadhyaṃ}}\dd{}
\app{\lem[wit={D,U1}]{lalāṭe}
  \rdg[wit={E}]{lalāṭamadhye}
  \rdg[wit={ceteri}]{lalāṭamadhyṃ}}\dd{}
\end{prose}
\end{ekdosis}
\ekdpb*{}
%%%%%%%%%%%%%%%%%%%%%%%%%%%%%%%%%%%%%%%%%%
%%%%%%%%%%%%%%%%%%%%%%%%%%%%%%%%%%%%%%%%%%
%%%%%%%%PAGEBREAK%%%%%%%PAGEBREAK%%%%%%%%%
%%%%%%%%%%%%%%%%%%%%%%%%%%%%%%%%%%%%%%%%%%
%%%%%%%%%%%%%%%%PAGEBREAK%%%%%%%%%%%%%%%%%
%%%%%%%%%%%%%%%%%%%%%%%%%%%%%%%%%%%%%%%%%%
%%%%%%%%PAGEBREAK%%%%%%%PAGEBREAK%%%%%%%%%
%%%%%%%%%%%%%%%%%%%%%%%%%%%%%%%%%%%%%%%%%%
%%%%%%%%%%%%%%%%%%%%%%%%%%%%%%%%%%%%%%%%%%
%%%%%%%%%%%%%%%%%%%%%%%%%%%%%%%%%%%%%%%%%%
%%%%%%%%%%%%%%%%%%%%%%%%%%%%%%%%%%%%%%%%%%
%%%%%%%%PAGEBREAK%%%%%%%PAGEBREAK%%%%%%%%%
%%%%%%%%%%%%%%%%%%%%%%%%%%%%%%%%%%%%%%%%%%
%%%%%%%%%%%%%%%%PAGEBREAK%%%%%%%%%%%%%%%%%
%%%%%%%%%%%%%%%%%%%%%%%%%%%%%%%%%%%%%%%%%%
%%%%%%%%PAGEBREAK%%%%%%%PAGEBREAK%%%%%%%%%
%%%%%%%%%%%%%%%%%%%%%%%%%%%%%%%%%%%%%%%%%%
%%%%%%%%%%%%%%%%%%%%%%%%%%%%%%%%%%%%%%%%%%
%%%%%%%%%%%%%%%%%%%%%%%%%%%%%%%%%%%%%%%%%%
%%%%%%%%%%%%%%%%%%%%%%%%%%%%%%%%%%%%%%%%%%
%%%%%%%%PAGEBREAK%%%%%%%PAGEBREAK%%%%%%%%%
%%%%%%%%%%%%%%%%%%%%%%%%%%%%%%%%%%%%%%%%%%
%%%%%%%%%%%%%%%%PAGEBREAK%%%%%%%%%%%%%%%%%
%%%%%%%%%%%%%%%%%%%%%%%%%%%%%%%%%%%%%%%%%%
%%%%%%%%PAGEBREAK%%%%%%%PAGEBREAK%%%%%%%%%
%%%%%%%%%%%%%%%%%%%%%%%%%%%%%%%%%%%%%%%%%%
\begin{ekdosis}
  \begin{prose}
    \noindent    
\app{\lem[wit={D,U1}]{śṛṅgāṭikāyāṃ}
  \rdg[wit={ceteri}]{śṛṃgāṭikā}}
\app{\lem[type=conjecture, resp=egoscr]{kapālamadhye}
  \rdg[wit={L}]{\conj karālamadhye}
  \rdg[wit={ceteri}]{kapolamadhye}}
\app{\lem[wit={ceteri}]{kapolamadhye}
  \rdg[wit={L}]{karāla}}\dd{}
\app{\lem[wit={ceteri}]{kamalinīmadhye}
  \rdg[wit={B,L}]{kamalinīmadhyaṃ}}\dd{}
\app{\lem[wit={D,U1}]{brahmarandhre}
  \rdg[wit={E}]{brahmaraṃdhra°}
  \rdg[wit={ceteri}]{brahmaraṃdhraṃ}}\dd{}
\app{\lem[type=emendation, resp=egoscr]{ūrdhvakamalinyās\skp{-}trikūṭasthāne}
  \rdg[wit={U2}]{\korr urdhvakamalinyās trikūṭasthānam}
  \rdg[wit={U1}]{urdhvakamalinyaḥ trikūṭasthāne}
  \rdg[wit={D}]{ūrdhvakamalinyaḥ || trikūṭasthāne ||}
  \rdg[wit={L,P}]{ūrdhvaṃ kamalīnyā trikūṭasthānam}
  \rdg[wit={B}]{kamalīnyāṃ strikūṭasthānam}
  \rdg[wit={E}]{kamalinyas trikūṭasthānam}}\dd{}
\app{\lem[wit={D}]{saptapātāle}
  \rdg[wit={ceteri}]{\om}}\dd{} 
%----------------------------
%evam ekaviṃśatisthāne   ekaviṃśatibrahmāṃḍāni vasaṃti// \E
%evam ekaviṃśasthāneṣu   ekaviṃśabrahmāni vasaṃti \P
%ekam ekaṃ viṃśasthānek  ekaviṃśabrahmāḍānī vasaṃtī// \B
%ekam ekaṃ viṃśasthāneṣv ekaviṃśabrahmāḍānī vasaṃtī// \L %%%0024.jpg
%\om                                                                 \N1
%evaṃ ekaviṃśatisthāne   ekaviṃśatibrahmāṃḍāni vasaṃti// \D
%\om                                                                 \N2
%                        ekāviṃśatibrahmāṃḍāni vasaṃti \U1
%evam ekaviṃśasthān      ekaviṃśa---brahmāṃḍāni vasaṃti// \U2
%-----------------------------
%thus the 21 worlds reside in 21 locations.
%Thus they reside at the 21 worlds in the 21 locations. 
%----------------------------
\app{\lem[wit={ceteri},alt={evam}]{eva\skp{m-e}}
  \rdg[wit={D}]{evaṃ}}
\app{\lem[wit={P}, alt={ekaviṃśasthāneṣv}]{\skm{m-e}kaviṃśasthāneṣv}
  \rdg[wit={B}]{\korr viṃśasthānek°}
  \rdg[wit={L}]{ekaṃ viṃśasthāneṣv}
  \rdg[wit={E,D}]{ekaviṃśatisthāne}
  \rdg[wit={U2}]{ekaviṃśasthān}}
\app{\lem[wit={E,D,U1}]{ekaviṃśatibrahmāṃḍāni}
  \rdg[wit={B,P,L,U2}]{ekaviṃśabrahmāni}}
\app{\lem[wit={ceteri}]{vasanti}
  \rdg[wit={L,B}]{vasaṃtī}}/
\end{prose}
\end{ekdosis}
%
\begin{ekdosis}
   \smallskip
    \centerline{\textrm{\small{[Seven Islands]}}}
    \smallskip
    \begin{prose}
      \noindent
%----------------------------
%idānīṃ saptadvīpāni piṃḍamadhye kathyante// \E
%idānīṃ saptadvīpāni piṃḍamadhye kathyaṃte \P
%idānī  satyadvīpāni piṃḍamadhye kathyate// \B
%idānīṃ saptadvīpāni piṃḍamadhye kathyate \L
%\om                                                                 \N1
%idānīṃ saptadvīpāni piṃḍamadhye kathyaṃte// \D
%\om                                                                 \N2
%idānīṃ saptadvīpāni piṃḍamadhye kathyaṃte \U1
%idānīṃ saptadvīpāni piṃḍamadhye kathyaṃte// \U2
%-----------------------------
%----------------------------
\note[type=testium, labelb=274, lem={saptadvīpāni}]{SSP 3.7: majjāyaṃ jambūdvīpaḥ asthiṣu śaktidvīpaḥ śirāsu sūkṣmadvīpaḥ tvakṣu krauñcadvīpaḥ romasu gomayadvīpaḥ nakheṣu śvetadvīpaḥ māṃse (asthini) plakṣadvīpaḥ evaṃ saptadvīpāḥ |}
\note[type=source, labelb=275, lem={saptadvīpāni}]{Ysv\textsuperscript{PT}: sapta dvīpāni kathyante 'dhunā tāni śṛṇu priye  | jambūdvīpas tu majjāyāṃ śākadvīpas tu madhyamaḥ | śālmadvīpaḥ śiromadhye māṃsamadhye kuśas tathā | tvaci krauñco lomamadhye gomayadvīpa īritaḥ | nakhamadhye tathā śvetaḥ saptadvīpā vasundharā | jambūḥ śākastathā śālmaḥ kuśaḥ krauñcaś ca gomayaḥ | śvetaḥ sapteti khaṇḍāni saptakhaṇḍair vasundharā | guptāny etāni rūpāṇi dehamadhye sthirāṇi ca |}      
idānīṃ saptadvīpāni piṃḍamadhye
\app{\lem[wit={ceteri}]{kathyante}
  \rdg[wit={B,L}]{kathyate}}/ 
%----------------------------
%majjāmadhye jaṃbudvīpaḥ/  asthimadhye śākadvīpaḥ      śirāmadhye   śālmalidvīpaḥ/    \E
%majjāmadhye jaṃbūdvīpaḥ/  asthīmadhye śākadvīpaḥ      śirāmadhye   śālmalidvīpaḥ     \P
%majjāmadhye jaṃbudvīpaḥ/  astimadhye  śākaladvīpaḥ//  śirāmadhye   śākaladvīpaḥ//    \B
%majjāmadhye jaṃbudvīpaḥ   astimadhye  śākaladvīpaḥ    śarīramadhye śākadvīpaḥ...     \L
%\om                                                                                  \N1
%majjāmadhye jaṃbudvīpaḥ// asthimadhye śākadvīpaḥ      śiromadhye   śālmalidvīpaḥ//   \D
%\om                                                                                  \N2
%majjāmadhye jaṃbudvīpaḥ   astimadhye  śāktidvīpaḥ     śīromadhye   śālmalidvīpaḥ     \U1
%majjāmadhye jaṃbudvīpaḥ// astimadhye  śākadvīpaḥ//    śīromadhye   śālmalīdvīpaḥ//   \U2
%-----------------------------
%Within the marrow is the island of Jambu. Within the bones is the island of Śāka. In the head is the island of Śālmali. 
%-----------------------------
majjāmadhye
\app{\lem[wit={ceteri}]{jambu}
  \rdg[wit={P}]{jaṃbū}}dvīpaḥ\dd{}
\app{\lem[wit={E,D}]{asthi}
  \rdg[wit={P}]{asthī}
  \rdg[wit={B,L,U1,U2}]{asti}}madhye
\app{\lem[wit={E,D,P,U2}]{śākadvīpaḥ}
  \rdg[wit={B,L}]{śākaladvīpaḥ}
  \rdg[wit={U1}]{śāktidvīpaḥ}}\dd{}
\app{\lem[wit={D,U1,U2}]{śiromadhye}
  \rdg[wit={B,E,P}]{śirāmadhye}
  \rdg[wit={L}]{śarīramadhye}}
\app{\lem[wit={ceteri}]{śālmalidvīpaḥ}
  \rdg[wit={U2}]{śālmalīdvīpaḥ}
  \rdg[wit={B}]{śākaladvīpaḥ}
  \rdg[wit={L}]{śākadvīpaḥ}}\dd{}
%----------------------------
%māṃsamadhye kuśadvīpaḥ/  tvacāmadhye krauṃcadvīpaḥ/  śarīrasthalomamadhye gomedadvīpaḥ/  nakhamadhye puṣkaradvīpaḥ//  etāni dvīpāni         madhye tiṣṭhanti// \E [p.50]
%māṃsamadhye kuśadvīpaḥ   tvacāmadhye krauṃcadvīpaḥ   śarīrasya lomamadhye gomedadvīpaḥ   nakhamadhye puṣkaradvīpaḥ    etāni dvīpāni guptāni madhye tiṣṭhaṃti \P
%māṃsamadhye kuśadvīpaḥ   tvacāmadhye krauṃcadvīpaḥ// śarīrasya lomamadhye gomedadvīpaḥ// nakhamadhye puṣkaradvīpaḥ//  etāni dvīpāni guptāni madhye tiṣṭhaṃti// \B
%māṃsamadhye kuśadvīpaḥ   tvacāmadhye krauṃcadvīpaḥ   śarīrasya lomamadhye gomedadvīpaḥ  taravamadhye puṣkaradvīpaḥ    etāni dvīpāni guptāni madhye tiṣṭhaṃti// \L
%\om                                                                                                                                                          \N1
%māṃsamadhye kuśadvīpaḥ// tvacāmadhye krauṃcadvīpaḥ   śarīrasya lomamadhye gomayadvīpaḥ/  nakhamadhye  śvetadvīpaḥ/    etāni rūpaṇi  guptamadhye   tiṣṭhaṃti// \D
%\om                                                                                                                                                            \N2
%māṃsamadhye kuśadvīpaḥ   tvacāmadhye krauṃcadvīpaḥ   śarīrasya lomadhye   gomayadvīpaḥ   taravamadhye svetadvīpaḥ     etāni rūpāṇī  guptamadhye   tiṣṭhaṃti \U1
%māṃsamadhye kuśadvīpaḥ// tvacāmadhye krauṃcadvīpaḥ// śarīrasya lomadhye   gomedadvīpaḥ// nakhamadhye  puṣkaradvīpaḥ// etāni dvīpāni guptāni madhye tiṣṭhaṃti// \U2
%-----------------------------
%In the flesh is the island of Kuśa. Within the skin is the island of Krauñca. At the hairy line between chest and navel (\textit{loma}) is the island of Gomaya. In dthe nails is the island of Śveta. These islands are situated are hidden within. 
%----------------------------
māṃsamadhye kuśadvīpaḥ\dd{} tvacāmadhye krauṃcadvīpaḥ\dd{} śarīrasya 
\app{\lem[wit={ceteri}]{lomamadhye}
  \rdg[wit={U1,U2}]{lomadhye}}
\app{\lem[wit={D,U1}]{gomayadvīpaḥ}
  \rdg[wit={ceteri}]{gomedadvīpaḥ}}\dd{}
\app{\lem[wit={ceteri}]{nakhamadhye}
  \rdg[wit={L,U1}]{taravamadhye}}
\app{\lem[wit={D,U1}]{śvetadvīpaḥ}
  \rdg[wit={ceteri}]{puṣkaradvīpaḥ}}\dd{}
etāni \app{\lem[wit={ceteri}]{dvīpāni}
  \rdg[wit={D,U1}]{rūpaṇi}}
\app{\lem[wit={B,P,L,U2}]{guptāni}
  \rdg[wit={D,U1}]{gupta°}
  \rdg[wit={E}]{\om}}
madhye
tiṣṭhanti/
\end{prose}
\end{ekdosis}
%
\begin{ekdosis}
    \smallskip
    \centerline{\textrm{\small{[Seven Oceans]}}}
    \smallskip
    \begin{prose}
      \noindent
%----------------------------
%idānīṃ piṃḍamadhye saptasamudrāḥ kathyante// prasvedamadhye kṣārasamudraḥ/   \E
%idānīṃ piṃḍamadhye saptasamudrāḥ kathyaṃte   prasvedamadhye kṣārasamudraḥ    \P
%idānīṃ piṃḍamadhye samudrāḥ      kathyate//  prasvedamadhye kṣārasamudraḥ//  \B
%idānīṃ piṃḍamadhye samudrāḥ      kathyaṃte// prasvedamadhye sārasasamudraḥ// \L
%\om                                                                 \N1
%idānīṃ piṃḍamadhye saptasamudrāḥ kathyete//  prasvedamadhye kṣārasasamudra   \D
%\om                                                                 \N2
%idānīṃ piṃḍamadhye saptasamudrāḥ kathyaṃte      svedamadhye kṣārasasamudraḥ  \U1 %%%288.jpg
%idānīṃ piṃḍamadhye saptasamudrāḥ kathyaṃte// prasvedamadhye kṣārasāgaraḥ//   \U2
%-----------------------------
%Now the seven oceans within the body are taught. Within sweat is the salt ocean (1). 
%----------------------------
idānīṃ piṇḍamadhye
\app{\lem[wit={ceteri}]{saptasamudrāḥ}
  \rdg[wit={L,B}]{samidrāḥ}}
\app{\lem[wit={ceteri}]{kathyante}
  \rdg[wit={B}]{kathyate}
  \rdg[wit={D}]{kathyete}}/
\app{\lem[wit={ceteri}]{prasvedamadhye}
  \rdg[wit={U1}]{svedamadhye}}
\app{\lem[wit={ceteri}]{kṣārasamudraḥ}
  \rdg[wit={L}]{sārasasamudraḥ}
  \rdg[wit={U1}]{kṣārasasamudraḥ}
  \rdg[wit={U2}]{kṣārasāgaraḥ}}\dd{}
%----------------------------
%lalāṭamadhye kṣīraḥ samudraḥ/            vāṅmadhye                                 madhusamudraḥ/  kaphamadhye  dadhisamudraḥ/  medomadhye ghṛtasamudraḥ/  \E
%lālāmadhye   kṣīrasamudraḥ               vasāmadhye                                madhusamudraḥ   kaphamadhye  dadhisamudraḥ   medomadhye ghṛtasamudraḥ   \P
%lalāṭamadhye kṣīrasamudraḥ// raktamadhye vasāmadhye                                madasamudraḥ    kaphamadhye  dadhisamudraḥ// medomadhye ghṛtasamudraḥ// \B
%lalāṭamadhye kṣīrasamudraḥ// raktamadhye vasāmadhye                                madyasamudraḥ// kaphamadhye  dadhisamudraḥ// medamadhye ghṛtasamudraḥ// \L
%\om                                                                 \N1
%lalāṭamadhye kṣīrasamudraḥ/              vasāmadhye                                                             dadhisamudraḥ// medamadhye ghṛtasamudraḥ// \D
%\om                                                                                                                                                         \N2
%lalāṭamadhye kṣīrasamudraḥ               vasāmadhye                                                             dadhisamudraḥ   medamadhye ghṛtasamudraḥ   \U1 %%%288.jpg
%lalāṭamadhye kṣīrasamudraḥ//             vīryamadhye svāduḥ samudraḥ// majjāmadhye madhusamūdraḥ// kaphamadhye  dadhisamudraḥ// medamadhye ghṛtasamudraḥ// \U2
%-----------------------------
%Within the forehead is the milk ocean (2). Within the brain is the honey-ocean (3). In the phlegm is the sour milk ocean (4). In the fat is the butter ocean (5).  
%----------------------------
\app{\lem[wit={ceteri}]{lalāṭamadhye}
  \rdg[wit={P}]{lālāmadhye}}
\app{\lem[wit={ceteri}]{kṣīrasamudraḥ}
  \rdg[wit={E}]{kṣīraḥ samudraḥ}}\dd{}
\app{\lem[wit={ceteri}]{vasāmadhye}
  \rdg[wit={E}]{vāṅmadhye}
  \rdg[wit={U2}]{vīryamadhye svāduḥ samudraḥ || majjāmadhye}}
\app{\lem[wit={E,P}]{madhusamudraḥ}
  \rdg[wit={B}]{madasamudraḥ}
  \rdg[wit={L}]{madyasamudraḥ}
  \rdg[wit={U2}]{madhusamūdraḥ}}\dd{}
kaphamadhye dadhisamudraḥ\dd{}
\app{\lem[wit={ceteri}, alt={meda°}]{meda}
  \rdg[wit={B,E,P}]{medo°}}madhye ghṛtasamudraḥ\dd{}
%----------------------------
%                           rasamadhye   ikṣurasasamudraḥ// vīryamadhye svādusamudraḥ/                pādamadhye kūrmasthānam//   \E
%                           raktamadhye  ikṣurasasamudraḥ   vīryamadhye svādudakasamudraḥ             pādamadhye kūrmasthānam     \P
%                                        ikṣusamudraḥ/      vīryamadhye svādukasamudraḥ/  karmasthāna pādasamadhye/               \B
%                                        ikṣusamudraḥ//     vīryamadhye svādukasamudraḥ// karmasthāna pādamadhye                  \L
%\om                                                                                                                             \N1
%vasāmadhye madhusamudraḥ// raktamadhye  ikṣusamudraḥ//     vīryamadhye amṛtasamudraḥ/                pādamtale  kūrmasthānaṃ/    \D
%\om                                                                                                                             \N2
%vasāmadhye madhusamudraḥ   raktamadhye  ikṣurasamudraḥ     vīryamadhye mṛtasamudraḥ                  pādamadhye kūrmasthānaṃ     \U1 %%%288.jpg
%                           raktamadhye  ikṣurasamudraḥ//                                                                         \U2
%-----------------------------
%Within the forehead is the milk-ocean. Within the blood is the sugarcane ocean. Within the semen is the ocean of the nectar of immortality. Within the feet is the place of the turtle. 
%----------------------------
\app{\lem[wit={P,U1,U2}]{raktamadhye}
  \rdg[wit={D}]{vasāmadhye madhusamudraḥ || raktamadhye}
  \rdg[wit={U1}]{vasāmadhye madhusamudraḥ raktamadhye}
  \rdg[wit={E}]{rasamadhye}}
\app{\lem[wit={B,D,L}]{ikṣusamudraḥ}
  \rdg[wit={U1,U2}]{ikṣurasamudraḥ}
  \rdg[wit={E,P}]{ikṣurasasamudraḥ}}
\note[type=philcomm, labelb=278, lem={ikṣura°}]{Due to \textit{sandhi} \textit{akṣura°} would be exspected, but was probably misregarded for clarity.}
vīryamadhye
\app{\lem[wit={U1}]{'mṛtasamudraḥ}
  \rdg[wit={D}]{amṛtasamudraḥ}
  \rdg[wit={E}]{svādusamudraḥ}
  \rdg[wit={B,L}]{svādukasamudraḥ}
  \rdg[wit={P}]{svādudakasamudraḥ}}\dd{}
\end{prose}
\end{ekdosis}
\ekdpb*{}
%%%%%%%%%%%%%%%%%%%%%%%%%%%%%%%%%%%%%%%%%%
%%%%%%%%%%%%%%%%%%%%%%%%%%%%%%%%%%%%%%%%%%
%%%%%%%%PAGEBREAK%%%%%%%PAGEBREAK%%%%%%%%%
%%%%%%%%%%%%%%%%%%%%%%%%%%%%%%%%%%%%%%%%%%
%%%%%%%%%%%%%%%%PAGEBREAK%%%%%%%%%%%%%%%%%
%%%%%%%%%%%%%%%%%%%%%%%%%%%%%%%%%%%%%%%%%%
%%%%%%%%PAGEBREAK%%%%%%%PAGEBREAK%%%%%%%%%
%%%%%%%%%%%%%%%%%%%%%%%%%%%%%%%%%%%%%%%%%%
%%%%%%%%%%%%%%%%%%%%%%%%%%%%%%%%%%%%%%%%%%
%%%%%%%%%%%%%%%%%%%%%%%%%%%%%%%%%%%%%%%%%%
%%%%%%%%%%%%%%%%%%%%%%%%%%%%%%%%%%%%%%%%%%
%%%%%%%%PAGEBREAK%%%%%%%PAGEBREAK%%%%%%%%%
%%%%%%%%%%%%%%%%%%%%%%%%%%%%%%%%%%%%%%%%%%
%%%%%%%%%%%%%%%%PAGEBREAK%%%%%%%%%%%%%%%%%
%%%%%%%%%%%%%%%%%%%%%%%%%%%%%%%%%%%%%%%%%%
%%%%%%%%PAGEBREAK%%%%%%%PAGEBREAK%%%%%%%%%
%%%%%%%%%%%%%%%%%%%%%%%%%%%%%%%%%%%%%%%%%%
%%%%%%%%%%%%%%%%%%%%%%%%%%%%%%%%%%%%%%%%%%
%%%%%%%%%%%%%%%%%%%%%%%%%%%%%%%%%%%%%%%%%%
%%%%%%%%%%%%%%%%%%%%%%%%%%%%%%%%%%%%%%%%%%
%%%%%%%%PAGEBREAK%%%%%%%PAGEBREAK%%%%%%%%%
%%%%%%%%%%%%%%%%%%%%%%%%%%%%%%%%%%%%%%%%%%
%%%%%%%%%%%%%%%%PAGEBREAK%%%%%%%%%%%%%%%%%
%%%%%%%%%%%%%%%%%%%%%%%%%%%%%%%%%%%%%%%%%%
%%%%%%%%PAGEBREAK%%%%%%%PAGEBREAK%%%%%%%%%
%%%%%%%%%%%%%%%%%%%%%%%%%%%%%%%%%%%%%%%%%%
  \begin{ekdosis}
\begin{prose}
\app{\lem[wit={ceteri}]{pādamadhye}
  \rdg[wit={B}]{karmasthāna pādasamadhye}
  \rdg[wit={L}]{karmasthāna pādamadhye}
  \rdg[wit={D}]{pādamtale}}
\app{\lem[wit={ceteri}]{kūrmasthānam}
  \rdg[wit={B,L}]{\om}}\dd{}
\note[type=source, labelb=276, lem={saptasamudrāḥ}]{Ysv\textsuperscript{PT}: samudrāḥ sapta kathyante piṇḍamadhye vyavasthitāḥ | lavaṇekṣusurāsarpirdadhidugdhajalāntakāḥ | lavaṇaṃ svedamadhye tu ikṣūrakte madhu tvaci | sarpir medo vasā madhye dadhi kṣīraṃ lalāṭake | vīryamadhye 'mṛto jñeyaḥ pāde kūrmaḥ sthito mahān |}
\note[type=source, labelb=277, lem={saptasamudrāḥ}]{SSP 3.8: mūrte kṣārasamudraḥ lālāyāṃ kṣīrasamudraḥ kaphe dadhisamudraḥ medasi ghṛtasamudraḥ vasāyāṃ madhusamudraḥ rakte ikṣusamudraḥ śukre 'mṛtasamudraḥ evaṃ saptasamudrāḥ||}
\note[type=philcomm, labelb=277, lem={kūrmasthānam}]{All witnesses preserve the statement of \textit{kūrmasthānam}, except for witness U\textsubscript{2} which places the statement two sentences later right after the introduction of the \textit{navadvāra}. In both cases it seems completely out of context. It must stem from the description of its source text, the Ysv\textsuperscript{PT} in which the statement seems likewise out of place.}
\end{prose}
\end{ekdosis}
\begin{ekdosis}
  \bigskip
  \centerline{\textrm{\small{[Nine Continents]}}}
   \bigskip
    \begin{prose}
      \noindent
% ----------------------------      
%idānīṃ navadvāreṣu                                                                                 nāsikayoḥ kinnarakhaṃḍanaraharikhaṃḍauḥ netrayoḥ ketumāla bhadrāśvau/ karṇayoḥ hiraṇmayakhaṃḍa ramyakakhaṃḍau/ gude kurukhaṃḍaḥ   liṃge ilāvṛtakhaṇḍaḥ// \E  [p.51]
%idānīṃ navadvāreṣu     navakhaṃjani? kathyaṃte                              mukhe bharatakhaṃḍaḥ 1 nāsikayoḥ kinarakhaṃḍe 3                netrayoḥ ketumāla bhadrāśve 4 karṇayor hiraṇmayaramyaka khaṃdaḥ 5      gude kurukhaṃḍaḥ 6 liṃge ilāvṛtaḥ 7  \P 7659.jpg!!!
%idānīṃ                 navakhaṃḍāni  kathyaṃte/                             mukhe bharatakhaṃḍaḥ   nāsikayor madhye kināraharikhaṃḍā/      netrayo ketumāla bhadrāsve/   karṇayor hiraṇyamayaramyakhaṃḍaḥ/        gude kurukhaṃḍāḥ/  liṃge iḍṛttaṃ??/ \B DSCN7169.JPG Z.4
%idānīṃ                 navakhaṃḍāni  kathyaṃte//                            mukhe bharatakhaṃḍaḥ// nāsikayor madhye kinārasiṃhakhaṃḍā      netrayo ketumāla bhadrāsve//   karṇayor hiraṇyamayaramyakhaṃḍaḥ        gude kurukhaṃḍāḥ//  liṃge ilāvṛtaṃ// \L   0025.jpg
%\om                                                                 \N1
%idānīṃ navadvāramadhye navakhaṃḍāḥ   kathyaṃte//                                  bharatakhaṃḍaḥ/ kāśmīrakhaṃḍaḥ/ strīmaṃḍalakhaṃḍaḥ/     dvijakhaṃḍaḥ/ ekapādakhaṃḍaḥ/  rākṣasakhaṃḍaḥ  ghāṃdhārakhaṃḍaḥ// kaivarttakhaṃḍaḥ// garbhakhaṃḍaḥ// \D %%%p.14 verso
%\om                                                                 \N2
%idānīṃ navadvāramadhye navakhaṃḍāḥ   kathyate                                     bharatakhaṃḍaḥ  kāsmīrakhaṃḍaḥ strīmaṃḍalakhaṃḍaḥ   ???dvīttakhaṃḍaḥ yekapādakhaṃḍaḥ rākṣasakhaṃḍaḥ ghaṃdhārakhaṃḍaḥ  kaivartakhaṃḍaḥ   garbhakaṃḍhaḥ \U1
%idānīṃ navadvāreṣu     navakhaṃḍāṇi  kathyaṃte// pādamadhye kūrmasthānaṃ// mukhaṃ bhāratakhaṃḍaṃ// nāsikayoḥ// kinnara// harikhaṃḍa//       netrayoḥ// ketumāla// bhadraśve karṇayoḥ// hiraṇmaya// ramyakakaṃḍe// gudekurukhaṃḍaṃ// liṃge ulāvṛtaṃ// evaṃ navakhaṃḍāḥ//    \U2
%-----------------------------
%Now the nine continents the world within the nine doors are taught: Bharata (1), Kaśmīra (2), Strīmaṃḍala (3), Dvija (4), Ekapāda (5), Rākṣasa (6), Ghandhāra (7), Kaivartta (8) [and] Garbha (9). 
%----------------------------
\note[type=source, labelb=278, lem={navadvāra°}]{Ysv\textsuperscript{PT}: idānīn tu navadvāre nava khaṇḍāni saṃśṛṇu | pāyvādau bhārataṃ khaṇḍaṃ kāśmīraṃ trikamaṇḍalum | dvijakhaṇḍam ekapādaṃ khaṇḍaṃ vakṣye samaṇḍalam | kaivarttaṃ garttagāndhāraṃ navakhaṇḍam iti sthitam |}
\note[type=source, labelb=279, lem={navadvāra°}]{SSP 3.9: navakhaṇḍāḥ nava dvāreṣu vasanti| bhāratakhaṇḍaḥ kāśmīrakhaṇḍaḥ karparakhaṇḍaḥ śrīkhaṇḍaḥ śaṅkhakhaṇḍaḥ ekapādakhaṇḍaḥ gāndhārakhaṇḍaḥ kaivartakhaṇḍaḥ mahāmerukhaṇḍaḥ evaṃ navakhaṇḍāḥ|}
\note[type=philcomm, labelb=280, lem={navakhaṇḍāni}]{This is another highly problematic passage. We see complete divergence between the two main groups of manuscripts. Finanlly the \alpha -group represented by witnesses D and U\textsubscript{1} was chosen, since their readings can be found within the source texts. The \beta -group represented by B,E,L,P and U\textsubscript{2} seems to rewrite the passage by adding the names of the nine doors which are partially lacking in one of the sources, the Ysv\textsuperscript{PT} and missing entirely in the SSP. The \beta -group assigns the names of a competing system to the areas. The \beta -group situates the Bharatakhaṇḍa within the mouth (1), the Kinnaraharikhaṇḍa in the two nostrils (3), the Ketumālabhadrāśva[-khaṇḍa] in the eyes (5),  the Hiraṇyamayaramyakakhaṇḍa in the ears (7), the Kurukhaṇḍa at the Anus (8), and the Ilāvṛta[-khaṇḍa] at the gender (9).}% These are the well-known nine \textit{dvīpa}s or islands ruled by nine sons of Ṛṣabhadeva, which are the nine \textit{varṣa}s of Jambudvīpa, viz., Bhārata, Kinnara, Hari, Kuru, Hiraṇmaya, Ramyaka, Ilāvṛta, Bhadrāśva and Ketumāla.}
idānīṃ
\app{\lem[wit={E,U1}]{navadvāramadhye}
  \rdg[wit={E,P,U2}]{navadvāreṣu}
  \rdg[wit={B,L}]{\om}}
\app{\lem[wit={B,P,L,U2}]{navakhaṇḍāni}
  \rdg[wit={D,U1}]{navakhaṃḍāḥ}
  \rdg[wit={E}]{\om}}
\app{\lem[wit={ceteri}]{kathyante}
  \rdg[wit={U1}]{kathyate}}/
\app{\lem[wit={D,U1}]{bharatakhaṇḍaḥ}
  \rdg[wit={B,P,L}]{mukhe bharatakhaṃḍaḥ}
  \rdg[wit={U2}]{pādamadhye kūrmasthānaṃ || mukhaṃ bhāratakhaṃḍaṃ}
  \rdg[wit={E}]{\om}}\dd{}
\app{\lem[wit={D,U1}]{kāśmīrakhaṃḍaḥ}
  \rdg[wit={E}]{nāsikayoḥ kinnarakhaṃḍanaraharikhaṃḍauḥ}
  \rdg[wit={P}]{nāsikayoḥ kinarakhaṃḍe 3}
  \rdg[wit={B}]{nāsikayor madhye kināraharikhaṃḍā}
  \rdg[wit={L}]{nāsikayor madhye kinārasiṃhakhaṃḍā}
  \rdg[wit={U2}]{nāsikayoḥ || kinnara || harikhaṃḍa}}\dd{}
\app{\lem[wit={D,U1}]{strīmaṇḍalakhaṇḍaḥ}
  \rdg[wit={ceteri}]{\om}}\dd{}
\app{\lem[wit={D,U1}]{dvijakhaṇḍaḥ}
  \rdg[wit={E}]{netrayoḥ ketumāla bhadrāśvau}
  \rdg[wit={P}]{netrayoḥ ketumāla bhadrāśve 4}
  \rdg[wit={B,L}]{netrayo ketumāla bhadrāsve}
  \rdg[wit={U2}]{netrayoḥ || ketumāla || bhadraśve}}\dd{}
\app{\lem[wit={D}]{ekapādakhaṇḍaḥ}
  \rdg[wit={U1}]{yekapādakhaṃḍaḥ}
  \rdg[wit={ceteri}]{\om}}\dd{}
\app{\lem[wit={D,U1}]{rākṣasakhaṇḍaḥ}
  \rdg[wit={E}]{karṇayoḥ hiraṇmayakhaṃḍa ramyakakhaṃḍau}
  \rdg[wit={P}]{karṇayor hiraṇmayaramyaka khaṃdaḥ 5}
  \rdg[wit={B,L}]{karṇayor hiraṇyamayaramyakhaṃḍaḥ}
  \rdg[wit={U2}]{karṇayoḥ || hiraṇmaya || ramyakakaṃḍe}}\dd{}
\app{\lem[wit={D,U1}]{ghāndhārakhaṇḍaḥ}
  \rdg[wit={E}]{gude kurukhaṃḍaḥ}
  \rdg[wit={P}]{gude kurukhaṃḍaḥ 6}
  \rdg[wit={B,L}]{gude kurukhaṃḍāḥ}
  \rdg[wit={U2}]{gudekurukhaṃḍaṃ}}\dd{}
\app{\lem[wit={D,U1}]{kaivarttakhaṇḍaḥ}
  \rdg[wit={E}]{liṃge ilāvṛtakhaṇḍaḥ}
  \rdg[wit={P}]{liṃge ilāvṛtaḥ 7}
  \rdg[wit={B,L}]{ilāvṛtaṃ}
  \rdg[wit={U2}]{liṃge ulāvṛtaṃ}}\dd{}
\app{\lem[wit={D,U1}]{garbhakhaṇḍaḥ}
  \rdg[wit={U2}]{evaṃ navakhaṃḍāḥ}
  \rdg[wit={ceteri}]{\om}}\dd{}
\end{prose}
\end{ekdosis}
\ekdpb*{}
%%%%%%%%%%%%%%%%%%%%%%%%%%%%%%%%%%%%%%%%%%
%%%%%%%%PAGEBREAK%%%%%%%PAGEBREAK%%%%%%%%%
%%%%%%%%%%%%%%%%%%%%%%%%%%%%%%%%%%%%%%%%%%
%%%%%%%%%%%%%%%%PAGEBREAK%%%%%%%%%%%%%%%%%
%%%%%%%%%%%%%%%%%%%%%%%%%%%%%%%%%%%%%%%%%%
%%%%%%%%PAGEBREAK%%%%%%%PAGEBREAK%%%%%%%%%
%%%%%%%%%%%%%%%%%%%%%%%%%%%%%%%%%%%%%%%%%%
%%%%%%%%%%%%%%%%%%%%%%%%%%%%%%%%%%%%%%%%%%
%%%%%%%%%%%%%%%%%%%%%%%%%%%%%%%%%%%%%%%%%%
%%%%%%%%%%%%%%%%%%%%%%%%%%%%%%%%%%%%%%%%%%
%%%%%%%%PAGEBREAK%%%%%%%PAGEBREAK%%%%%%%%%
%%%%%%%%%%%%%%%%%%%%%%%%%%%%%%%%%%%%%%%%%%
%%%%%%%%%%%%%%%%PAGEBREAK%%%%%%%%%%%%%%%%%
%%%%%%%%%%%%%%%%%%%%%%%%%%%%%%%%%%%%%%%%%%
%%%%%%%%PAGEBREAK%%%%%%%PAGEBREAK%%%%%%%%%
%%%%%%%%%%%%%%%%%%%%%%%%%%%%%%%%%%%%%%%%%%
%%%%%%%%%%%%%%%%%%%%%%%%%%%%%%%%%%%%%%%%%%
%%%%%%%%%%%%%%%%%%%%%%%%%%%%%%%%%%%%%%%%%%
%%%%%%%%%%%%%%%%%%%%%%%%%%%%%%%%%%%%%%%%%%
%%%%%%%%PAGEBREAK%%%%%%%PAGEBREAK%%%%%%%%%
%%%%%%%%%%%%%%%%%%%%%%%%%%%%%%%%%%%%%%%%%%
%%%%%%%%%%%%%%%%PAGEBREAK%%%%%%%%%%%%%%%%%
%%%%%%%%%%%%%%%%%%%%%%%%%%%%%%%%%%%%%%%%%%
%%%%%%%%PAGEBREAK%%%%%%%PAGEBREAK%%%%%%%%%
%%%%%%%%%%%%%%%%%%%%%%%%%%%%%%%%%%%%%%%%%%
  \begin{ekdosis}
       \bigskip
       \centerline{\textrm{\small{[Eight Mountains]}}}
       \bigskip
    \begin{prose}
      \smallskip
      \noindent
%----------------------------
%idānīm             aṣṭamakulaparvatāḥ kathyante/  \E
%idānīm             aṣṭakulaparvatāḥ   kathyaṃte   \P
%idānīm             aṣṭamakulaparvatāḥ kathyaṃte// \B
%idānīm             aṣṭamakulaparvatāḥ kathyaṃte// \L
%\om                                               \N1
%idānīṃ piṃḍamadhye aṣṭakulaparvatāḥ   kathyaṃte// \D
%\om                                               \N2
%idānīṃ piṃḍamadhye aṣṭakulaparvatāḥ   kathyaṃte   \U1
%idānīm             aṣṭakulaparvatā    kathyaṃte// \U2
%-----------------------------
%Now the eight mointains within the body are taught. 
%----------------------------
\note[type=source, labelb=281, lem={aṣṭakulaparvatāḥ}]{Ysv\textsuperscript{PT}: idānīṃ parvatāś cāṣṭau kathyante śṛṇu yatnataḥ | merudaṇḍe sumerus tu pīṭhamadhye himālayaḥ | vāmaskandhe tathā dakṣe malayo mandarācalaḥ | vindhyas tu dakṣiṇe karṇe vāme maināka īśvari | lalāṭe madhyadeśe tu śrīśailaḥ parameśvari | tathā brahmakapāṭasthaḥ kailāsaḥ parvato mahān | sumerur himavān vindhyo malayo mandaras tathā | śrīśailo mainākaś ceti kailāso 'ṣṭau ca parvatāḥ | apare parvatāḥ sarveaṅgulīmadhyavāsinaḥ |}
\note[type=source, labelb=282, lem={aṣṭakulaparvatāḥ}]{SSP 3.10: meruparvato merukhaṇḍe vasati kailāso brahmakapāṭe vasati himālayaḥ pṛṣṭhe malayo vāmakandhare mandaro dakṣiṇakandhare vindhyo dakṣiṇakarṇe mainākaḥ vāmakarṇe śrīparvato lalāṭe evam aṣṭa kulaparvatāḥ anye upaparvatāḥ sarvāṅguliṣu vasanti ||}
  \app{\lem[wit={D,U1}]{idānīṃ}
  \rdg[wit={ceteri}]{idānīm}}
\app{\lem[wit={D,U1}]{piṇḍamadhye}
  \rdg[wit={ceteri}]{\om}}
\app{\lem[type=emendation, resp=egoscr]{'ṣṭakulaparvatāḥ}
  \rdg[wit={P,D,U1}]{\korr aṣṭakulaparvatāḥ}
  \rdg[wit={U2}]{aṣṭakulaparvatā}
  \rdg[wit={B,E,L}]{aṣṭamakulaparvatāḥ}}
kathyante/
%----------------------------  
%merudaṇḍamadhye merumaṃdaraḥ/   brahmakapāṭamadhye  kailāsaḥ/ \E
%merudaṇḍamadhye merumaṃdaraḥ    brahmakapāṭamadhye  kailāsaḥ  \P
%merudaṇḍamadhye merumaṃdaraḥ/   brahmakapāṭamadhye  kailāsaḥ/ \B
%merudaṇḍamadhye merumaṃdaraḥ/   brahmakapāṭamadhye  kailāsaḥ/ \L
%\om                                                                  \N1
%merudaṃḍamadhye merumparvataḥ// brahmakapāṭamadhye  kailāsaparvataḥ \D
%\om                                                                 \N2
%merudaṇḍamadhye merumparvattaḥ  brahmakapāṭamadhye  kailāsaparvataḥ \U1
%merudaṇḍamadhye merumaṃdaraḥ//  brahmakapāṭamadhye  kailāsaḥ// \U2 %%%419.jpg 
%-----------------------------
%Within the spine is mount Meru (1). Within the door of Bahman is mount Kailasa (2). 
%----------------------------
merudaṃḍamadhye
\app{\lem[type=emendation, resp=egoscr]{meruparvataḥ}
  \rdg[wit={D,U1}]{\korr merumparvataḥ}
  \rdg[wit={ceteri}]{merumaṃdaraḥ}}\dd{}
brahmakapāṭamadhye
\app{\lem[wit={D,U1}]{kailāsaparvataḥ}
  \rdg[wit={ceteri}]{kailāsaḥ}}\dd{}
%----------------------------
%pṛṣṭhamadhye   himācalaḥ/           vāmaskandhe malayācalaḥ/  dakṣiṇaskandhe mandarācalaḥ/  dakṣiṇakarṇe vindhyācalaḥ/  \E
%pṛṣṭhaṃ madhye himācalaḥ            vāmaskaṃdhe malayācalaḥ   dakṣiṇaskaṃdhe maṃdarācalaḥ   dakṣiṇakarṇe vindhyācalaḥ    \P
%pṛthvīmadhye  himācalaḥ/            vāmaskaṃdhe malayācalaḥ/  dakṣiṇaskaṃdhe maṃdarācalaḥ/  dakṣiṇakarṇe vindhyācalaḥ/ \B
%pṛthvīmadhye   himācalaḥ/           vāmaskaṃdhe malayācalaḥ/  dakṣiṇaskaṃdhe maṃdarācalaḥ/  dakṣiṇakarṇe viṃdhyācalaḥ/  \L
%\om                                                                 \N1
%paiṭimadhye    himācalaḥ// parvataḥ vāmaskaṃdhe malayācalaḥ   dakṣaṇaskaṃdhe maṃdarācalaḥ   dakṣaṇakarṇe viṃdhyācalaḥ  \D
%\om                                                                 \N2
%paiṭhamadhye   himācala----parvataḥ vāmaskaṃdhe malayācalaḥ   dakṣaṇaskaṃdhe maṃdarācalaḥ   dakṣaṇakarṇe viṃdhyācalaḥ  \U1
%pṛṣṭhamadhye   himācalaḥ//          vāmaskandhe malayācalaḥ// dakṣiṇaskandhe mandarācalaḥ// dakṣiṇakarṇe vindhyācalaḥ//  \U2
%-----------------------------
%Within the back is the Himālaya (3). Within the left shoulder the Malabar mountain (4). Within the right shoulder the mountain of Mandara (5). In the right ear the Vindhya mountain (6). 
%----------------------------
\app{\lem[wit={E,U2}]{pṛṣṭhamadhye}
  \rdg[wit={P}]{pṛṣṭhaṃ madhye}
  \rdg[wit={B,L}]{pṛthvīamadhye}
  \rdg[wit={D}]{paiṭimadhye}
  \rdg[wit={U1}]{paiṭhamadhye}}
\app{\lem[wit={ceteri}]{himācalaḥ}
  \rdg[wit={D}]{himācalaḥ || parvataḥ}
  \rdg[wit={U1}]{himācalaparvataḥ}}\dd{}
vāmaskaṃdhe malayācalaḥ\dd{}
dakṣiṇaskaṃdhe maṃdarācalaḥ\dd{}
dakṣaṇakarṇe viṃdhyācalaḥ\dd{}
%----------------------------
%vāmakarṇe mainākaḥ/  lalāṭamadhye śrīśailaḥ/   apare śailāḥ   hastayoḥ  pādayor aṃgulīnāṃ   mūleṣu varttaṃte// \E
%vāmakarṇe mainākaḥ   lalāṭamadhye śrīśailaḥ    apare śailā    hastayoḥ  pādayor aṃgulīnāṃ   mūleṣu varttaṃte   \P
%vāmakarṇe mainākaḥ/  lalāṭamadhye śrīśailāsaḥ/ apare śailā    hastayoḥ/ pādayor aṃguli------mūleṣu vartate//   \B
%vāmakarṇe mainākaḥ/  lalāṭamadhye śrīśailaḥ/   apare śailā    hastayoḥ/ pādayor aṃgulī------mūleṣu vartate/    \L
%\om                                                                 \N1
%vāmakarṇe mainākaḥ// lalāṭamadhye śrīśailaḥ//  apare parvatāḥ hastayoḥ  pādayor aṃgulīnāṃ   madhye vartatte//  \D
%\om                                                                 \N2 
%vāmakarṇe mainākaḥ   lalāṭamadhye śrīśailaḥ    apare parvatāḥ hastayoḥ  pādayor aṃgulībhyāṃ madhye parvate     \U1
%vāmakarṇe mainākaḥ   lalāṭamadhye śrīśailaḥ//  apare śailāḥ// hastayoḥ  pādayor aṃgulīnāṃ   mūleṣu vartaṃte//  \U2
%-----------------------------
%In the left ear the Maināka[-mountain] (7). Within the forehead Śrīśaila (8). Other mountains exist in the hands, feet, and toes.   
%----------------------------
vāmakarṇe mainākaḥ\dd{}
lalāṭamadhye
\app{\lem[wit={ceteri}]{śrīśailaḥ}
  \rdg[wit={B}]{śrīśailāsaḥ}}
apare \app{\lem[wit={D,U1}]{parvatāḥ}
  \rdg[wit={E,U2}]{śailāḥ}
  \rdg[wit={B,P,L}]{śailā}}
pādayo\skp{r-aṃ}\app{\lem[wit={E,P,D},alt={aṃgulīnāṃ}]{\skp{r-aṃ}gulīnāṃ}
  \rdg[wit={U1}]{aṃgulībhyāṃ}
  \rdg[wit={B,L}]{aṃguli°}}
\app{\lem[wit={ceteri}]{mūleṣu}
  \rdg[wit={D,U1}]{madhye}}
\app{\lem[wit={ceteri}]{vartante}
  \rdg[wit={B,L}]{vartate}
  \rdg[wit={U1}]{parvate}}\dd{}
\end{prose}
\end{ekdosis}
%%%%
%%%%
%%%%
\begin{ekdosis}
       \bigskip
       \centerline{\textrm{\small{[Nine Rivers]}}}
       \bigskip
    \begin{prose}
     \noindent
%----------------------------
%idānīṃ śarīramadhye navanāḍyas tiṣṭhanti   tanmadhye navanadīnāṃ     sthānāni   varttante/   \E
%idānīṃ śarīre       navanaḍyas tiṣṭhaṃti   tanmadhye navāṃnā nadīnāṃ sthānāni   vartaṃte     \P
%idānīṃ śarīre       navanaḍyas tiṣṭhanti// tanmadhye navānāṃ nadīnāṃ sthānāni   vartate/     \B
%idānīṃ śarīre       navanaḍyaḥ tiṣṭhaṃti/  tanmadhye navānāṃ nadīnāṃ sthānāni   vartaṃte/    \L
%\om                                                                 \N1
%idānīṃ śarīre       ṇavānāḍyas tiṣṭhati//  tanmadhye navānāṃ nadīnāṃ sthānāni   vartraṃte//  \D
%\om                                                                 \N2
%idānīṃ śarīre       ṇavānaḍyaḥ stiṣṭhaṃti  tanmadhye navānāṃ nadīnāṃ sthānāni   vartaṃte    \U1
%idānīṃ śarīramadhye navanāḍyas tiṣṭhati//  tanmadhye navānāṃ nadīnāṃ          nivarttaṃte// \U2
%-----------------------------
%Now within the body there are nine rivers. Within it the courses of the nine rivers exist. 
%----------------------------
\note[type=source, labelb=283, lem={navanāḍyas}]{SSP 3.11: pīnasā gaṅgā yamunā candrabhāgā sarasvatī | pipāsā śatarudrā ca śrīrātriś caiva narmadā evaṃ nava nadyo navanāḍīṣu vasanti}
\note[type=source, labelb=284, lem={navanāḍyas}]{Ysv\textsuperscript{PT}: śarīre navanāḍīsthā narmadā ca maheśvari | iḍāyāṃ yamunā devi piṅgalāyāṃ sarasvatī | suṣumnāyāṃ vahed gaṅgā cānyonyāsu ca nāḍiṣu | gaṅgā sarasvatī godā narmadā yamunā tathā | kāverī candrabhāgā ca vitastā ca iḍāvatī | dvisaptatisahasreṣu nadīnadaparisravaḥ |}
idānīṃ
\app{\lem[wit={ceteri}]{śarīre}
  \rdg[wit={E,U2}]{śarīramadhye}}
\app{\lem[wit={E,U2},alt={navanāḍyas}]{navanāḍya\skp{s-ti}}
  \rdg[wit={P,B,L}]{navanaḍyas}
  \rdg[wit={D}]{ṇavānāḍyas}
  \rdg[wit={U1}]{ṇavānaḍyaḥs}}
\app{\lem[wit={ceteri}]{tiṣṭhanti}
  \rdg[wit={D,U2}]{tiṣṭhati}}/
tanmadhye
\app{\lem[wit={ceteri}]{navānāṃ nadīnāṃ}
  \rdg[wit={E}]{navanadīnāṃ}}
sthānāni
\app{\lem[wit={ceteri}]{vartante}
  \rdg[wit={U2}]{nivartaṃte}
  \rdg[wit={B}]{vartate}}/
%----------------------------     
%gaṃgāyamune vitastā candrabhāgā sarasvatī vipāśā   śatahradā   irāvatī narmadā/   \E [p.52]
%gaṃgāyamunā vitastā caṃdrabhāgā sarasvatī vipāśā   śātahṛdā    irāvati narmmadā    \P
%gaṃgāyamunā vitastā caṃdrabhāgā sarasvatī vipāśā   śāśatahṛdā  irāvati narmadā/  \B
%gaṃgāyamunā vitastā caṃdrabhāgā sarasvati vipāśā   śatat hṛda  irāvati narmadā// \L
%\om                                                                 \N1
%gaṃgāyamunā vitastā caṃdrabhāgā sarasvatī/ vaipaśā śata hṛdā// irāvatī/ narmadā/ \D
%\om                                                                 \N2
%gaṃgāyamunā vitastā caṃdrabhāgā sarasvatī vaipaśā  śata hṛdā   airāvati narmadā \U1
%gaṃgāyamunā vitastā candrabhāgā sarasvatī vipāśā   śātadrumā//          narmadā   \U2
%-----------------------------
%Gaṅga, Yamuna, Vitastā, Candrabhāga, Sarasvatī, Vipāśā, Śatarudrā, Irāvati und Narmadā.  
%-----------------------------
gaṃgāyamunā vitastā caṃdrabhāgā
\app{\lem[wit={ceteri}]{sarasvatī}
  \rdg[wit={L}]{sarasvati}}/
\app{\lem[wit={ceteri}]{vipāśā}
  \rdg[wit={D,U1}]{vaipaśā}}
\app{\lem[type=emendation, resp=egoscr]{śatarudrā}
  \rdg[wit={P,D,U1}]{\korr śātahṛdā}
  \rdg[wit={E}]{śatahradā}
  \rdg[wit={B}]{śāśatahṛdā}
  \rdg[wit={U2}]{śātadrumā}}
\note[type=philcomm, labelb=285, lem={śatarudrā}]{Judging the spellings of geographical nomenclature of south asian rivers, the reading of Ysv\textsuperscript{PT} must be correct.}
\app{\lem[wit={E,D}]{irāvatī}
  \rdg[wit={P,B,L,U1}]{irāvati}
  \rdg[wit={U2}]{\om}}
narmadā/ 
%-----------------------------
%aparā    nadyo    nadāni       srotāṃsi   taṭākāni  vāpīkūpā---disaptati----sahasranāḍī----madhye tiṣṭhanti/  \E
%aparā    nadyo    nadānir jārā srotāṃsī   taṭānī    vāpīkūpā---dvisaptatī---sahasranāḍīnāṃ madhye tiṣṭaṃti    \P %7660.jpg
%aparā    nadyo    nadānir jñārāsty etāṃsī taṭānī    vāpīkūpā---dvisaptatī---sahasranāḍīnā--madhye tiṣṭaṃti/   \B
%aparā    nadyo    nadānir jñārāsty etāṃsi taṭāni    vāpīkūpā---dvisaptati---sahasranāḍīnāṃ madhye tiṣṭaṃti/   \L
%\om                                                                                                   \N1
%aparā    nadyopanadīnair   bhurasrota-----taṭāka----vāpikupāḥ  dvisaptati---sahasranāḍīnāṃ madhye tiṣṭaṃti/  \D
%\om                                                                                                   \N2
%gaṃḍakī  nadyūpanadīnair  bhurasrota------taḍaga    vāpīkūpa---dvisaptati---sahastranāḍī   madhye  tiṣṭhaṃṭī  \U1  %em zu upanadinirjhara!!! Wasserfälle! 
%aparā    nadyo   nadānir  jñārāsrotāsī    taṭhānī   vāpīkūpā---dvisaptati---sahasranāḍīnāṃ madhye tiṣṭaṃti//  \U2
%-----------------------------
%Other rivers, and waterfalls near the rivers, currents, lakes, ponds and wells are situated within the 72000 channels.
%-----------------------------
\note[type=source, labelb=285, lem={dvisaptati°}]{SSP 3.12: anyā upanadyaḥ kulyopakulyāḥ dvisaptatisahasranāḍīsu vasanti |}
\app{\lem[wit={ceteri}]{aparā}
  \rdg[wit={U1}]{gaṃḍakī}}
\app{\lem[type=emendation, resp=egoscr]{nadyopanadinirjharāḥ srotāṃsi}
  \rdg[wit={D}]{\korr nadyopanadīnairbhurasrota°}
    \rdg[wit={U1}]{nadyūpanadīnairbhurasrota°}
    \rdg[wit={P}]{nadyo nadānirjārā srotāṃsī}
    \rdg[wit={B,L}]{nadyo nadānirjñārāsty etāṃsī}
    \rdg[wit={U2}]{nadyo nadānirjñārāsrotāsī}
    \rdg[wit={E}]{nadyo nadāni srotāṃsi}}
\app{\lem[wit={E}]{taṭākāni}
    \rdg[wit={D}]{\rdg[wit={D}]{taṭāka}}
    \rdg[wit={P,B,L}]{taṭānī}
    \rdg[wit={D}]{taṭāka}
    \rdg[wit={U1}]{taḍaga}
    \rdg[wit={U2}]{taṭhānī}}
\app{\lem[wit={ceteri}]{vāpīkūpā}
  \rdg[wit={D}]{vāpikupāḥ}}
\app{\lem[wit={ceteri},alt={dvisaptati°}]{dvisaptati}
  \rdg[wit={B,P}]{dvisaptatī°}
  \rdg[wit={E}]{disaptati}
}\app{\lem[wit={ceteri}]{sahasranāḍīnāṃ}
  \rdg[wit={B}]{sahasranāḍīnā}
  \rdg[wit={E,U1}]{sahastranāḍī}}
madhye
\app{\lem[wit={ceteri}]{tiṣṭhanti}
  \rdg[wit={U1}]{tiṣṭhaṃṭī}
  \rdg[wit={ceteri}]{tiṣṭaṃti}}/
\end{prose}
\end{ekdosis}
\ekdpb*{}
%%%%%%%%%%%%%%%%%%%%%%%%%%%%%%%%%%%%%%%%%%
%%%%%%%%PAGEBREAK%%%%%%%PAGEBREAK%%%%%%%%%
%%%%%%%%%%%%%%%%%%%%%%%%%%%%%%%%%%%%%%%%%%
%%%%%%%%%%%%%%%%PAGEBREAK%%%%%%%%%%%%%%%%%
%%%%%%%%%%%%%%%%%%%%%%%%%%%%%%%%%%%%%%%%%%
%%%%%%%%PAGEBREAK%%%%%%%PAGEBREAK%%%%%%%%%
%%%%%%%%%%%%%%%%%%%%%%%%%%%%%%%%%%%%%%%%%%
%%%%%%%%%%%%%%%%%%%%%%%%%%%%%%%%%%%%%%%%%%
%%%%%%%%%%%%%%%%%%%%%%%%%%%%%%%%%%%%%%%%%%
%%%%%%%%%%%%%%%%%%%%%%%%%%%%%%%%%%%%%%%%%%
%%%%%%%%PAGEBREAK%%%%%%%PAGEBREAK%%%%%%%%%
%%%%%%%%%%%%%%%%%%%%%%%%%%%%%%%%%%%%%%%%%%
%%%%%%%%%%%%%%%%PAGEBREAK%%%%%%%%%%%%%%%%%
%%%%%%%%%%%%%%%%%%%%%%%%%%%%%%%%%%%%%%%%%%
%%%%%%%%PAGEBREAK%%%%%%%PAGEBREAK%%%%%%%%%
%%%%%%%%%%%%%%%%%%%%%%%%%%%%%%%%%%%%%%%%%%
%%%%%%%%%%%%%%%%%%%%%%%%%%%%%%%%%%%%%%%%%%
%%%%%%%%%%%%%%%%%%%%%%%%%%%%%%%%%%%%%%%%%%
%%%%%%%%%%%%%%%%%%%%%%%%%%%%%%%%%%%%%%%%%%
%%%%%%%%PAGEBREAK%%%%%%%PAGEBREAK%%%%%%%%%
%%%%%%%%%%%%%%%%%%%%%%%%%%%%%%%%%%%%%%%%%%
%%%%%%%%%%%%%%%%PAGEBREAK%%%%%%%%%%%%%%%%%
%%%%%%%%%%%%%%%%%%%%%%%%%%%%%%%%%%%%%%%%%%
%%%%%%%%PAGEBREAK%%%%%%%PAGEBREAK%%%%%%%%%
%%%%%%%%%%%%%%%%%%%%%%%%%%%%%%%%%%%%%%%%%%
\begin{ekdosis}
       \bigskip
       \centerline{\textrm{\small{[Stars and Vessels]}}}
       \bigskip
    \begin{prose}
       \noindent
%-----------------------------
%saptaviṃśatinakṣatrāṇi  dvisaptatikoṣṭhakābhyantare vasaṃti/    \E
%saptaviṃśatinakṣatrāṇi  dvisaptatikoṣṭhakāṃtrābhyaṃtare vasaṃti     \P
%saptaviṃśatinakṣatrāṇi/ dvisaptatīkoṣṭhākāṃtrābhyāṃtare vasaṃti//    \B
%saptaviṃśatinakṣatrāṇi  dvisaptatīkoṣṭākāṃtrābhyāṃtare vasaṃti    \L
%\om                                                                 \N1
%saptaviṃśatinakṣatrāṇi  dvisaptatikoṣṭhakāś cāṃtrābhyaṃtare vasṃati// \D
%\om                                                                 \N2
%saptaviṃśatinakṣatrāṇi  dvisaptatikoṣṭākāś  cāṃtrābhyaṃtare vasati    \U1 %%%289.jpg
%saptaviṃśatinakṣatrāṇi  dvisaptatikoṣṭhakāṃtarābhyaṃtare vasaṃti//    \U2
%-----------------------------
%Twentyseven stars and 72 vessels are residing inside the guts.   
%-----------------------------
saptaviṃśatinakṣatrāṇi
\app{\lem[wit={D}]{dvisaptatikoṣṭhakāś\skp{-}cāṃtrābhyantare}
  \rdg[wit={U1}]{dvisaptatikoṣṭākāś  cāṃtrābhyaṃtar}
  \rdg[wit={P}]{dvisaptatikoṣṭhakāṃtrābhyaṃtare}
  \rdg[wit={B}]{dvisaptatīkoṣṭhākāṃtrābhyāṃtare}
  \rdg[wit={L}]{dvisaptatīkoṣṭākāṃtrābhyāṃtare}
  \rdg[wit={U2}]{dvisaptatikoṣṭhakāṃtarābhyaṃtare}
  \rdg[wit={E}]{dvisaptatikoṣṭhakābhyantare}}
vasanti/
\end{prose}
\end{ekdosis}
%-----------------------------
%dvādaśa rāśayaḥ/  meṣaḥ vṛṣaḥ mithunaḥ karkaḥ siṃhaḥ kanyā tulā vṛściko dhanur makarakumbhamīnāḥ/ \E
%dvādaśa rāśayaḥ   meṣavṛṣamithūnaḥ karkasiṃhakanyātūlavṛścikadhanamakarakuṃbhamīna \P
%dvādaśa rāśayāḥ/  meṣavṛṣabhamithūnakarkasiṃhakanyātūlavṛścikadhanamakarakuṃbhamīnaḥ// \B
%dvādaśa rāśayaḥ   meṣavṛṣamithunakarkasiṃhakanyātūlavṛścikadhanamakarakuṃbhamīnaḥ// \L
%\om                                                                 \N1
%dvādaśa rāśayaḥ// meṣavṛṣamithunakarkasiṃhakanyātūlavṛścikadhanamakarakuṃbhamīna \D
%\om                                                                 \N2
%dvādaśa rāśayaḥ   meṣavṛṣamithunakarkasiṃhakanyātūlavṛścikadhanamakarakuṃbhamīna \U1
%dvādaśa rāśayaḥ// meṣa// vṛṣabha// mithuna// karka// siṃha// kanyā// tula// vṛścika// dhana// makara// kuṃbha// mīna// \U2
%-----------------------------
%The twelf zodiacal signs (rāśi) are: Aries, Taurus, Twins, Cancer, Lion, Virgo, Libra, Scrorpio, Sagittarius, Capricorn, Auqarius, and Fish.  
%-----------------------------
\begin{ekdosis}
       \bigskip
       \centerline{\textrm{\small{[Twelve Houses of the Zodiac]}}}
       \bigskip
    \begin{prose}
\note[type=source, labelb=286, lem={rāśayaḥ}]{Ysv\textsuperscript{PT}: itas tato dehamadhye ṛkṣaś ca saptaviṃśatiḥ | yogāś ca rāśayaś caiva grahāś ca tithayas tathā |}
\note[type=source, labelb=287, lem={rāśayaḥ}]{SSP 3.13: saptaviṃśatinakṣatrāṇi dvādaśa rāśayaḥ nava grahāḥ pañcadaśa tithayaḥ ete antarvalaye dvisaptati sahasra svahastakoṣṭheṣu vasanti|}
dvādaśa
\app{\lem[wit={ceteri}]{rāśayaḥ}
  \rdg[wit={B}]{rāśayāḥ}}\dd{}\\
\app{\lem[wit={E}]{meṣaḥ}
  \rdg[wit={U2}]{ meṣa ||}
  \rdg[wit={ceteri}]{meṣa°}}\dd{}
\app{\lem[wit={E}]{vṛṣaḥ}
  \rdg[wit={U2}]{vṛṣabha ||}
  \rdg[wit={ceteri}]{°vṛṣa°}}\dd{}
\app{\lem[wit={E}]{mithunaḥ}
  \rdg[wit={U2}]{mithuna ||}
  \rdg[wit={P}]{°mithūnaḥ}
  \rdg[wit={B}]{°mithūna°}
  \rdg[wit={ceteri}]{°mithuna°}}\dd{}
\app{\lem[wit={ceteri}]{karkaḥ}
  \rdg[wit={P}]{karka°}
  \rdg[wit={U2}]{karka ||}
  \rdg[wit={ceteri}]{°karka°}}\dd{}
\app{\lem[wit={E}]{siṃhaḥ}
  \rdg[wit={U2}]{siṃha ||}
  \rdg[wit={ceteri}]{°siṃha°}}\dd{}
\app{\lem[wit={E}]{kanyā}
  \rdg[wit={U2}]{kanyā ||}
  \rdg[wit={ceteri}]{°kanyā°}}\dd{}
\app{\lem[wit={E}]{tulā}
  \rdg[wit={U2}]{tula ||}
  \rdg[wit={ceteri}]{°tūla°}}\dd{}
\app{\lem[type=emendation, resp=egoscr]{vṛścikaḥ}
  \rdg[wit={E}]{\korr vṛściko}
  \rdg[wit={U2}]{vṛścika ||}
  \rdg[wit={ceteri}]{°vṛścika°}}
\app{\lem[type=emendation, resp=egoscr]{danuḥ}
  \rdg[wit={E}]{\korr dhanur}
  \rdg[wit={U2}]{dhana ||}
  \rdg[wit={ceteri}]{°dhana°}}
\app{\lem[type=emendation, resp=egoscr]{makaraḥ}
  \rdg[wit={U2}]{\korr makara ||}
  \rdg[wit={ceteri}]{°makara°}}
\app{\lem[type=emendation, resp=egoscr]{kumbhaḥ}
  \rdg[wit={U2}]{kuṃbha ||}
  \rdg[wit={ceteri}]{\korr °kumbha°}}\dd{}
\app{\lem[type=emendation, resp=egoscr]{mīnaḥ}
  \rdg[wit={E}]{°mīnāḥ}
  \rdg[wit={B,L}]{\korr mīnaḥ}
  \rdg[wit={U2}]{mīna ||}
  \rdg[wit={ceteri}]{°mīna}}\dd{}
\note[type=philcomm, labelb=286, lem={rāśayaḥ}]{In order to unify the various lists based on the previous usage of lists in the text the form of the list's item have been emenden to the nominativ case and double-\textit{daṇḍa}s were used to separate the items.}
\end{prose}
\end{ekdosis}
%%%%
%%%%
%%%%
\begin{ekdosis}
       \bigskip
       \centerline{\textrm{\small{[Nine Planets]}}}
       \bigskip
       \begin{prose}
         \noindent
%-----------------------------
%navagrahāḥ/  āditya--soma--maṃgala--budha-------guru----śukra---śani---rāhu--ketavaḥ/    paṃcadaśatithayo  tra     madhye vasaṃti// \E [P.53]
%navagrahaḥ   āditya--soma--maṃgala--budha--bṛhaspatiḥ --śukra---śaniḥ  rāhuḥ ketuḥ       paṃcadaśatithayo  tra   madhye vasaṃti \P
%navagrahāḥ// āditya--soma--maṃgala--budha--bṛhaspati----śukra---śani---rāhu--ketu//      paṃcadaśatithiḥ//  atra madhye vasaṃti// \B %%%DSCN7170.jpg Z.1
%navagrahāḥ// āditya--soma--maṃgala--budha--bṛhaspati----śukra---śani---rāhu--ketu        paṃcadaśatithayaḥ// atra madhye vasaṃti// \L  %%%0026.jpg
%\om                                                                                                                             \N1
%navagrahāḥ// āditya--soma/ maṃgala/ budha/ bṛhaspati/   śukra---śani --rāhu--ketu/       paṃcadaśatithayo  tra madhye vasaṃti \D %%%p.14 verso drittletzte Zeile
%\om                                                                                                                               \N2
%navagrahāḥ   āditya--soma--maṃgala--budha---bṛhaspati---śukra---śani---rāhu---ketu.h     paṃcadaśatithayo ātra madhye vasaṃti \U1
%navagrahāḥ/  ravi//caṃdra//maṃgala//budha// vṛhasyati// śukra// śanī// rāhu// ketuḥ//    padaśatithayo tra madhye tiṣṭhaṃti// \U2
%-----------------------------
%Nine Planets: Sonne, Mond, Mars, Merkur, Jupiter, Venus, Saturn,  Kopf des Schlangendämons (aufsteigender Mondknoten), Schwanz des Schlangendämons (absteigender Mondknoten). The fifteen lunar days reside within (the body?). 
%-----------------------------
\app{\lem[wit={ceteri}]{navagrahāḥ}
  \rdg[wit={P}]{navagrahaḥ}}\dd{}\\
\app{\lem[type=emendation, resp=egoscr]{ādityā}
  \rdg[wit={ceteri}]{\korr āditya°}
  \rdg[wit={U2}]{ravi ||}}\dd{}
\app{\lem[type=emendation, resp=egoscr]{somaḥ}
  \rdg[wit={ceteri}]{\korr °soma°}
  \rdg[wit={D}]{°soma |}
  \rdg[wit={U2}]{caṃdra ||}}\dd{}
\app{\lem[type=emendation, resp=egoscr]{maṅgalaḥ}
  \rdg[wit={D}]{\korr maṃgala |}
  \rdg[wit={U2}]{maṃgala ||}}\dd{}
\app{\lem[type=emendation, resp=egoscr]{budhaḥ}
  \rdg[wit={U2}]{\korr budha ||}
  \rdg[wit={D}]{budha |}
  \rdg[wit={ceteri}]{°budha°}}
\app{\lem[type=emendation, resp=egoscr]{bṛhaspatiḥ}
  \rdg[wit={P}]{\korr °bṛhaspatiḥ}
  \rdg[wit={D}]{bṛhaspati |}
  \rdg[wit={U2}]{vṛhasyati ||}
  \rdg[wit={ceteri}]{°bṛhaspati°}}\dd{}
\app{\lem[type=emendation, resp=egoscr]{śukraḥ}
  \rdg[wit={U2}]{\korr śukra ||}
  \rdg[wit={D}]{śukra°}
  \rdg[wit={ceteri}]{°śukra°}}\dd{}
\app{\lem[type=emendation, resp=egoscr]{śaniḥ}
  \rdg[wit={P}]{\korr °śaniḥ}
  \rdg[wit={U2}]{śanī ||}
  \rdg[wit={ceteri}]{°śani°}}\dd{}
\app{\lem[wit={P}]{rāhuḥ}
  \rdg[wit={U2}]{rāhu ||}
  \rdg[wit={ceteri}]{°rāhu°}}\dd{}
\app{\lem[wit={P,U1,U2}]{ketuḥ}
  \rdg[wit={E}]{ketavaḥ}
  \rdg[wit={ceteri}]{°ketu}}\dd{}\\
\app{\lem[wit={E,D,U1,P}]{pañcadaśatithayo}
  \rdg[wit={L}]{paṃcadaśatithayaḥ ||}
  \rdg[wit={B}]{paṃcadaśatithiḥ ||}
  \rdg[wit={U2}]{padaśatithayo}}
\app{\lem[wit={E,P,D,U2}]{'tra}
  \rdg[wit={B,L}]{atra}
  \rdg[wit={U1}]{ātra}}
  madhye
  \app{\lem[wit={ceteri}]{vasanti}
    \rdg[wit={U2}]{tiṣṭhaṃti}}/
  \ekdpb*{}
\end{prose}
\end{ekdosis}
%-----------------------------
%                     yathā samudramadhye laharī varttate/   tathā śarīramadhye kūrmmī nāma laharī bhavati/ \E
%                     yathā ....................................   sarīramadhye urmī   nāma laharī bhavati    \P
%                     yathā samudramadhye laharā vartate/    tathā śarīramadhye urmmī  nāma laharī bhavati/ \B
%                     yathā samudramadhye laharī vartate//   tathā śarīramadhye urmmī  nāma laharī bhavatī/ \L
%\om                                                                                                       \N1
%                     yathā samudramadhye laharī varttate/   tathā śarīramadhye ūrmī   nāma laharī bhavati/ \D
%\om                                                                 \N2
% pīṭhasya romamadhye yathā samudramadhye laharī vartate     tathā śarīramadhye urmi   nāma laharī bhavati  \U1
%                     yathā samudramadhye lahari varttate//  tathā śarīramadhye urmmī  nāma laharī bhavaṃti// \U2
%-----------------------------
%Just as the wave resides in the coean, so does the wave called Ūrmī resides in the body. 
%-----------------------------
\begin{ekdosis}
\begin{prose}
\note[type=source, labelb=287, lem={laharī}]{Ysv\textsuperscript{PT}:laharīṣu mīnamanī cāvāhanaṃ sthāpanaṃ tathā | sarvāṅgeṣu ca deveśi samagraṃ ṛkṣamaṇḍalam | trayastriṃśatkoṭay astu nivasanti ca devatāḥ |}
\note[type=source, labelb=288, lem={urmī}]{SSP 3.13: anekatāramaṇḍalaṃ ūrmiṣu vasanti | trayastriṃśatkoṭidevatā bāhuromakūpeṣu vasanti|}
\app{\lem[wit={ceteri}]{yathā}
  \rdg[wit={U1}]{pīṭhasya romamadhye yathā}}
\app{\lem[wit={ceteri}]{samudramadhye}
  \rdg[wit={P}]{\om}}
\app{\lem[wit={ceteri}]{laharī}
  \rdg[wit={B}]{laharā}
  \rdg[wit={P}]{\om}}
vartate/
\app{\lem[wit={ceteri}]{tathā}
  \rdg[wit={P}]{\om}}
śarīramadhye
\app{\lem[wit={D}]{ūrmī}
  \rdg[wit={ceteri}]{urmī}
  \rdg[wit={E}]{kūrmmī}}
nāma laharī
\app{\lem[wit={ceteri}]{bhavati}
  \rdg[wit={U2}]{bhavanti}}\dd{}
%-----------------------------
%ūrmyaś calās tataḥ            calanaṃ bhavati/                    tanmadhye samagraṃ tārāmaṇḍalaṃ varttate/trayastriṃśatkoṭidevatāḥ/ bāhuromamadhye vasaṃti/ \E
%ūrmyaś calāś cataḥ   śarīre   calanaṃ bhavati  dhāvanaṃ ca        tanmadyhe samagraṃ tārāmaṇḍalaṃ varttate trayastriṃśatkoṭyo devatāḥ bāhuromamadhye vasaṃti    \P
%ūrmmīś calāś cataḥ// śarire   calanaṃ bhavati/ dhāvanaṃ ca/       tanmadhye samagrāṃ tārāmaṇḍalaṃ vartate/ trayastriṃśatkoṭayo devatāḥ/ bāhuromamadhye vasaṃti// \B
%                                              dhāvanaṃ ca/        tanmadhye samagraṃ tārāmaṇḍalaṃ vartate/ trayastriṃśatkoṭayo devatāḥ/ bāhuromamadhye vasaṃti// \L
% \om                                                                 \N1
%tasyāḥ urmyaḥ  calācharīre    calanaṃ bhavati/ dhāvanaṃ bhavati// tanmadhye samagraṃ tārāmaṇḍalaṃ varttate trayastriśatkoṭyo devatā bāhuromamadhye vasaṃtī// \D %p.15 recto 
%\om                                                                 \N2
%tathā urmeś   calanāśarīre    calanaṃ bhavati  dhāvanaṃ bhavati   tanmadhye samagra--tārāmaṇḍalaṃ vartate  trayaḥ striśatakoṭī devatā bāhuromamadhye vasaṃtī \U1
%ūrmiyaś calāḥ// tataḥ śarīra--calanaṃ bhavati//dhāvanaṃ ca//      tanmadhye samagra--tārāmaṇḍalaṃ vartate//trayaḥ triṃśatkoṭyo devatāḥ// bāhuromamadhye vasaṃti// \U2
%-----------------------------
%Thus, because of the fluctuation of Ūrmī, movement arises in the body [and] running arises. Within it the entire cirlce of fixed stars exists. Thirty three crore divinities reside within the pores of the arms.   
%-----------------------------
\note[type=source, labelb=289, lem={samagraṃ}]{Ysv\textsuperscript{PT}: sarvāṅgeṣu ca deveśi samagraṃ ṛkṣamaṇḍalam | trayastriṃśatkoṭay astu nivasanti ca devatāḥ |}
\note[type=source, labelb=290, lem={devatāḥ}]{SSP 3.13: trayastriṃśatkoṭidevatā bāhuromakūpeṣu vasanti|}
\app{\lem[wit={U1},alt={tathā urmeś}]{tathā urme\skp{ś-ca}}
  \rdg[wit={D}]{tasyāḥ urmyaḥ}
  \rdg[wit={E}]{ūrmyaś calās}
  \rdg[wit={P}]{ūrmyaś calāś}
  \rdg[wit={B}]{ūrmmīś calāś}
  \rdg[wit={L}]{\om}
  \rdg[wit={U2}]{ūrmiyaś calāḥ ||}}
\app{\lem[type=emendation, resp=egoscr]{\skm{ś-ca}lanāccharīre}
  \rdg[wit={D}]{\korr calācharīre}
  \rdg[wit={U1}]{calanāśarīre}
  \rdg[wit={B}]{cataḥ || śarire}
  \rdg[wit={P}]{cataḥ śarīre}
  \rdg[wit={U2}]{tataḥ śarīra°}
  \rdg[wit={E}]{tataḥ}
  \rdg[wit={L}]{\om}}
calanaṃ bhavati/
\app{\lem[wit={D,U1}]{dhāvanaṃ bhavati}
  \rdg[wit={ceteri}]{dhāvanaṃ ca}
  \rdg[wit={E}]{\om}}/ 
tanmadhye
\app{\lem[wit={ceteri}]{samagraṃ}
  \rdg[wit={B}]{samagrāṃ}
  \rdg[wit={U1,U2}]{samagra°}}
tārāmaṇḍalaṃ vartate/
\app{\lem[wit={B,L}]{trayastriṃśatkoṭayo}
  \rdg[wit={P}]{trayastriṃśatkoṭyo}
  \rdg[wit={U2}]{trayaḥ triṃśatkoṭyo}
  \rdg[wit={U1}]{trayaḥ striśatakoṭī}
  \rdg[wit={D}]{trayastriśatkoṭyo}
  \rdg[wit={E}]{trayastriṃśatkoṭi°}}
\app{\lem[wit={D,U1}]{devatā}
  \rdg[wit={ceteri}]{devatāḥ |}}
bāhuromamadhye
\app{\lem[wit={ceteri}]{vasanti}
  \rdg[wit={D,U1}]{}vasaṃtī}/
%-----------------------------
%\om                                                                                                                            \E
%pṛṣṭaromamadhye     ṣaḍaśī   sahasra  divyatapasvinaḥ    pīṭhopapīṭhe  dvavoṣṭo        pariyāni   romāṇi tanmadhye vasaṃti    \P
%pṛṣṭīromamadhye     ṣaḍaśatī sahasra  divyatapasvinaḥ    mīṣṭhopapīṭher dvaiṣṭho       pariyāni   romāṇi tanmadhye vasaṃti/   \B
%pṛṣṭīromamadhye     ṣaḍaśatī sahasra  divyatapasvinaḥ    pīṭhopapīṭhe    dvaiṣṭhi        pariyā   romāṇi tanmadhye vasaṃti//  \L
%pṛṣṭīromamadhye     ṣaḍaśīti sahasra  divyatapasvino     pīṭhamahāpīṭhau urdhvapṛṣṭho  pariyāni   romāni tanmadhye saṃti      \U1
%pīṭhasya romamadhye ṣaḍaśīti sahasra  divyatapasvino     pīṭhamahāpīṭhau ūrddhva tuṣṭo pariyāni   romāṇi tanmadhye vasaṃti//  \D
%pṛṣṭaromamadhye     ṣaḍaśīti sahasra  divyatapasvinaḥ//  pīṭhopapīṭhordhva             pariyāti   romāṇi tanmadhye vaṃsaṃti// \U2 %%420.jpg pariyāṇa===surround! 
%\om                                                                 \N1
%\om                                                                 \N2
%-----------------------------
%Within the pores of the back there are 86000 (ṣaḍaśītisahasra) heavenly ascetics. Seats [of power] and great seats [of power] reside within the hair sourrounding the upper part of the back.   
%-----------------------------
\note[type=source, labelb=291, lem={pīṭhopapīṭhāṇi}]{Ysv\textsuperscript{PT}: tathā pīṭhāni sarvāṇi dehamadhye sthitāni ca }
\note[type=source, labelb=292, lem={pīṭhopapīṭhāṇi}]{SSP 3.13: anekapīṭḥopapīṭhakāḥ romakūpeṣu vasanti|}
\note[type=philcomm, labelb=293, lem={pṛṣṭīromamadhye \ldots vasanti}]{This sentence is \om in E.}
\app{\lem[wit={B,L,U1},alt={pṛṣṭī°}]{pṛṣṭī}
  \rdg[wit={P,U2}]{pṛṣṭa°}
  \rdg[wit={D}]{pīṭhasya}}romamadhye
\app{\lem[wit={D,U1,U2},alt={ṣaḍaśīti°}]{ṣaḍaśīti}
  \rdg[wit={B,L}]{ṣaḍaśatī°}
  \rdg[wit={P}]{ṣaḍaśī°}}sahasra
divya\app{\lem[wit={P,B,L,U2}]{tapasvinaḥ}
  \rdg[wit={U1,D}]{tapasvino}}/
\app{\lem[type=emendation, resp=egoscr]{pīṭhopapīṭhāṇi}
  \rdg[wit={P,L}]{\korr pīṭhopapīṭhe}
  \rdg[wit={B}]{mīṣṭhopapīṭher}
  \rdg[wit={D,U1}]{pīṭhamahāpīṭhau}
  \rdg[wit={U2}]{pīṭhopapīṭho°}}
\app{\lem[type=emendation, resp=egoscr]{ūrdhvapṛṣṭhe}
  \rdg[wit={U1}]{\korr urdhvapṛṣṭho}
  \rdg[wit={D}]{ūrddhva tuṣṭo}
  \rdg[wit={U2}]{ordhva}
  \rdg[wit={P}]{dvavoṣṭo}
  \rdg[wit={B}]{dvaiṣṭho}
  \rdg[wit={L}]{dvaiṣṭhi}}
\app{\lem[type=emendation, resp=egoscr]{pariyāṇe}
  \rdg[wit={B,D,P,U1,U2}]{\korr pariyāni}
  \rdg[wit={L}]{pariyā}}
\app{\lem[type=emendation, resp=egoscr]{romaṇi} %%em. zum Lokativ von roman 
  \rdg[wit={B,L,P,D,U2}]{\korr romāṇi}
  \rdg[wit={U1}]{romāni}}
tanmadhye
\app{\lem[wit={ceteri}]{vasanti}
  \rdg[wit={U1}]{santi}}/
%-----------------------------
%hṛdayaromamadhye takṣakaḥ mahānāgaḥ/               śaṃkhaḥ   takṣakaḥ/ vāsukiḥ/  anantaśeṣaḥ      ete nāga vasaṃti/       \E
%hṛdayaromamadhye takṣakamahānāga      karkoṭakaḥ   śaṃkhaḥ   pulakaḥ   vāsukiḥ   anaṃtaḥ  śoṣa    ete nāgā vasaṃti \P
%hṛdayaromamadhye takṣamā nāgaḥ        karkoṭaḥ     śaṃkhaḥ   pulikaḥ   vāsukī    ānaṃta   śoṣa    ete nāgā vasaṃti \U1
%hṛdayaromamadhye takṣakamahānāgaḥ//   karkoṭakaḥ/  śaṃkhaḥ/  pulika/   vāsukī/   ānaṃta/  śeṣā    ete nāgā vasaṃti  \D
%hṛdayaromamadhye takṣakaḥ mahānāgaḥ// karkoṭakaḥ// śaṃkhaḥ// kulakaḥ// vāsukiḥ// ānaṃta// śeṣaḥ// ete nāgā vasaṃti// \U2 %%420.jpg
%\om                                                        \B
%\om                                                                   \L
%\om                                                                 \N1
%\om                                                                 \N2
%-----------------------------
%Within the cavity of the heart these snake-demons reside: Takṣaka, Karkoṭaka, Śaṃkha, Pulaka, Vāsuki, Ānanta and Śeṣa. 
%-----------------------------
\note[type=source, labelb=294, lem={hṛdayaromamadhye}]{Ysv\textsuperscript{PT}: hṛdaye vyomamadhye tu anantādyāstu vāsukiḥ | udare vyomamadhye tu pare nāgā vasanti hi |}
\note[type=source, labelb=295, lem={hṛdayaromamadhye}]{SSP 3.13: kulanāgā vakṣasi vasanti |} %%%%ACHtung: LAUT LONAVLA nicht in allen Zeugen!
\note[type=philcomm, labelb=296, lem={hṛdayaromamadhye \ldots vasanti}]{This sentence is \om in B and L.}
hṛdayaromamadhye
\app{\lem[wit={D}]{takṣakamahānāgaḥ}
  \rdg[wit={E,U2}]{takṣakaḥ mahānāgaḥ}
  \rdg[wit={P}]{takṣakamahānāga}
  \rdg[wit={U1}]{takṣamā nāgaḥ}}\dd{}
\app{\lem[wit={D,P,U2}]{karkoṭakaḥ}
  \rdg[wit={U1}]{karkoṭaḥ}
  \rdg[wit={E}]{\om}}\dd{}
śaṅkhaḥ\dd{}
\app{\lem[wit={P}]{pulakaḥ}
  \rdg[wit={U1}]{pulikaḥ}
  \rdg[wit={D}]{pulika}
  \rdg[wit={U2}]{kulakaḥ}
  \rdg[wit={E}]{takṣakaḥ}}\dd{}
\app{\lem[wit={E,P,U2}]{vāsukiḥ}
  \rdg[wit={D,U1}]{vāsukī}}\dd{} 
\app{\lem[wit={P}]{anantaḥ}
  \rdg[wit={E}]{ananta°}
  \rdg[wit={U1}]{ānaṃta°}
  \rdg[wit={D,U2}]{ānanta}}\dd{}
\app{\lem[wit={U2}]{śeṣaḥ}
  \rdg[wit={E}]{°śeṣaḥ}
  \rdg[wit={P}]{śoṣa}
  \rdg[wit={U1}]{°śoṣa}
  \rdg[wit={D}]{śeṣā}}\dd{}
ete
\app{\lem[wit={ceteri}]{nāgā}
  \rdg[wit={E}]{nāga}}
vasanti/
%-----------------------------
%udararomamadhye apare  nāgā vasaṃti    guṇagandharvakinnarāpsaro vidyādharaguhyakāḥ/ \E
%udararomamadhye apare  nāgā vasaṃti    guṇagaṃdharvakinarā ...\P
%udararomamadhye apare  nāgā vasaṃti//  guṇagaṃdharvakinnarābharo vidyādharaguhyakāḥ... \B
%udararomamadhye apare  nāgā vasaṃti    guṇagaṃdharvakinnarābharo vidyādharaguhyakāḥ... \L
%\om                                                                 \N1
%\om                                                                 \N2
%udararomamadhye pare   nāgā vasaṃti    gaṇagaṃdharvakinnarapuruṣāpsarovidyādharaguhyaka \U1
%udararomamadhye/ apare nāgā vasaṃti//  gaṇagaṃdharvakiṃnarakiṃpuruṣa// apsarovidyādhāra/ guhyaka \D
%udararomamadhye apare  nāgā vasaṃti//  gaṃdhagaṃdharvakinnarāpsaro vidyādharaguhyakaḥ// \U2 %%420.jpg 
%-----------------------------
%Within the cavity of the belly reside other snakes, [as well as] Gaṇas, Gandharvas, Kinnaras, Apsaras, Vidyādharas and Guhyakas. 
%-----------------------------
\note[type=source, labelb=297, lem={udararomamadhye}]{Ysv\textsuperscript{PT}: udare vyomamadhye tu 'pare nāgā vasanti hi | gandharvakinnarāḥ śūrā vidyādharāpsarādayaḥ | anekatīrthavarṇāś ca guhyakāś ca vasanti hi |}
\note[type=source, labelb=298, lem={gandharva°}]{SSP 3.13: gandharvakinnarakiṃpuruṣā apsaraso gaṇā udare vasanti |}
udararoma\app{\lem[wit={ceteri},alt={°madhye}]{madhye}
  \rdg[wit={D}]{°madhye |}}
\app{\lem[wit={U1}]{'pare}
  \rdg[wit={ceteri}]{apare}}
nāgā
vasanti/
\app{\lem[type=emendation, resp=egoscr]{gaṇagandharvakinnarapsarovidyādharaguhyakāḥ}
  \rdg[wit={E}]{\korr guṇagandharvakinnarāpsaro vidyādharaguhyakāḥ}
  \rdg[wit={B}]{guṇagaṃdharvakinnarābharo vidyādharaguhyakāḥ}
  \rdg[wit={L}]{guṇagaṃdharvakinnarābharo vidyādharaguhyakāḥ}
  \rdg[wit={U1}]{gaṇagaṃdharvakinnarapuruṣāpsarovidyādharaguhyaka}
  \rdg[wit={D}]{gaṇagaṃdharvakiṃnarakiṃpuruṣa || apsarovidyādhāra | guhyaka}
  \rdg[wit={U2}]{gaṃdhagaṃdharvakinnarāpsaro vidyādharaguhyakaḥ}}/
%-----------------------------
%śarīramadhye              anekatīrthāni     vasaṃti/  aśrupātamadhye meghamaṇḍalaṃ vasati/   anaṃtāḥ siddhayo buddhayaś ca  prakāśamadhye varttante/ \E
%      madhye              nekatīrthā valī   vasaṃti/  aśrupātamadhye meghamaṇḍalaṃ vasati    anaṃtāḥ siddhayo buddhayaś ca  prakāśamadhye varttante \P
%śarīramadhye              anekatīrthāvalī   vasaṃtī// aśrupātamadhye meghamaṇḍala  vasaṃtī   anaṃtā  siddhayo buddhayac ca/ prakāśamadhye vartate/ \B
%śarīramadhye              anekatīrthāvalī   vasaṃtī// aśrupātamadhye meghamaṇḍalaṃ vasatī    anaṃtā  siddhayo buddhayaś ca  prakāśamadhye vartate// \L
%śarīmadhye   karmasthāne  nenekatīrthavallī vasaṃti// aśrupātamadhye meghamaṃḍalaṃ vasaṃti// anaṃtāḥ siddhayo buddhayaś ca  prakāśamadhye varttate// \D
%śarīramadhye marmasthāne  naikatīrthavallī  vasaṃtī   aśrupātamadhye meghamaṃḍalaṃ vasaṃti   anaṃtā  siddhayo budhayaś  ca  prakāśamadhye vartate \U1 %%%290.jpg
%śarīramadhye             'nekatīrthāvalī    vasatī//  aśrupātamadhye meghamaṃḍalaṃ vasati//  anaṃtā  siddhayo buddhayaś ca  prakāśamadhye vartante// \U2
%\om                                                                 \N1
%\om                                                                 \N2
%-----------------------------
%Within the body at the vulnerable place[s] many series of places of pilgrimage are located. Within the falling tears resides the circle of clouds. Within the light exists many siddhas and intelligences? 
%-----------------------------
\note[type=source, labelb=299, lem={meghamaṃḍalaṃ}]{Ysv\textsuperscript{PT}: anantasiddhayo buddhyā prakāśo varttate hṛdi | meghasya maṇḍalaṃ jñeyam aśrupāte tathaiva ca |}
\note[type=source, labelb=300, lem={meghamaṃḍalaṃ}]{SSP 3.13: anekameghāḥ aśrupāte vasanti | anekatīrthāni marmasthāne vasanti | anantasiddhāḥ matiprakaśe vasanti |}
\app{\lem[wit={ceteri}]{śarīramadhye}
  \rdg[wit={D}]{śarīmadhye}
  \rdg[wit={P}]{madhye}}
\app{\lem[wit={U1}]{marmasthāne}
  \rdg[wit={D}]{karmasthāne}
  \rdg[wit={ceteri}]{\om}}
\note[type=philcomm, labelb=301, lem={marmasthāne}]{Reading was adopted due to its presence in one of the sources.}
\app{\lem[wit={P,U2}]{'nekatīrthāvalī}
  \rdg[wit={B,L}]{anekatīrthāvalī}
  \rdg[wit={U1}]{naikatīrthavallī}
  \rdg[wit={D}]{nenekatīrthavallī}
  \rdg[wit={E}]{anekatīrthāni}}
vasanti/
    \end{prose}
  \end{ekdosis}
  \ekdpb*{}
%%%%%%%%%%%%%%%%%%%%%%%%%%%%%%%%%%%%%%%%%%
%%%%%%%%PAGEBREAK%%%%%%%PAGEBREAK%%%%%%%%%
%%%%%%%%%%%%%%%%%%%%%%%%%%%%%%%%%%%%%%%%%%
%%%%%%%%%%%%%%%%PAGEBREAK%%%%%%%%%%%%%%%%%
%%%%%%%%%%%%%%%%%%%%%%%%%%%%%%%%%%%%%%%%%%
%%%%%%%%PAGEBREAK%%%%%%%PAGEBREAK%%%%%%%%%
%%%%%%%%%%%%%%%%%%%%%%%%%%%%%%%%%%%%%%%%%%
%%%%%%%%%%%%%%%%%%%%%%%%%%%%%%%%%%%%%%%%%%
%%%%%%%%%%%%%%%%%%%%%%%%%%%%%%%%%%%%%%%%%%
%%%%%%%%%%%%%%%%%%%%%%%%%%%%%%%%%%%%%%%%%%
%%%%%%%%PAGEBREAK%%%%%%%PAGEBREAK%%%%%%%%%
%%%%%%%%%%%%%%%%%%%%%%%%%%%%%%%%%%%%%%%%%%
%%%%%%%%%%%%%%%%PAGEBREAK%%%%%%%%%%%%%%%%%
%%%%%%%%%%%%%%%%%%%%%%%%%%%%%%%%%%%%%%%%%%
%%%%%%%%PAGEBREAK%%%%%%%PAGEBREAK%%%%%%%%%
%%%%%%%%%%%%%%%%%%%%%%%%%%%%%%%%%%%%%%%%%%
%%%%%%%%%%%%%%%%%%%%%%%%%%%%%%%%%%%%%%%%%%
%%%%%%%%%%%%%%%%%%%%%%%%%%%%%%%%%%%%%%%%%%
%%%%%%%%%%%%%%%%%%%%%%%%%%%%%%%%%%%%%%%%%%
%%%%%%%%PAGEBREAK%%%%%%%PAGEBREAK%%%%%%%%%
%%%%%%%%%%%%%%%%%%%%%%%%%%%%%%%%%%%%%%%%%%
%%%%%%%%%%%%%%%%PAGEBREAK%%%%%%%%%%%%%%%%%
%%%%%%%%%%%%%%%%%%%%%%%%%%%%%%%%%%%%%%%%%%
%%%%%%%%PAGEBREAK%%%%%%%PAGEBREAK%%%%%%%%%
%%%%%%%%%%%%%%%%%%%%%%%%%%%%%%%%%%%%%%%%%%
\begin{ekdosis}
  \begin{prose}
    \noindent
%-----------------------------
%caṃdrasūryau dvayor netrayor madhye varttete/     anekavanaspatigulmalatātṛṇāni  jaṃghāromamadhye vasaṃti/ \E %%%[p.54]
%caṃdrasūryau dvayor netreyor madhye vartate       anekavanaspatigulmalatātṛṇāni  jaṃghāromamadhye vasaṃti \P
%caṃdrasūryo  dvayā--netrayo--madhye vartate//     anekavanaspatigulmalatātṛṇāni  jaṃghāroramadhye vasaṃti// \B
%caṃdrasūryo  dvayo  netrayor madhye vartate//     anekavanaspatigulmalatātṛṇāni  jaṃghāroramadhye vasaṃti... \L
%\om                                                                 \N1
%caṃdrasūryo  dvayor netrayor madhye varttate//    anaikavanaspatigulmatṛṇāni     jaṃghāromasthāne varttaṃte/ \D
%\om                                                                 \N2
%caṃdrasūryau        netradvaya      vasaṃti       anekavanaspatīgulmalatāni      jaṃghāromamadhye vasaṃti \U1
%caṃdrasūryau dvayo  netrayoḥ madhye pravartate//  anekavana/spatigulmalatātṛṇāni jaṃghāromamadhye vasati// \U2
%-----------------------------
%The sun and the moom exists within the two eyes. Many trees, bushes, creepers and grasses live within the hairs of the legs.  
%-----------------------------
\note[type=source, labelb=301, lem={candrasūryau}]{Ysv\textsuperscript{PT}: candrārkau netrayormadhye jaṅghā lomasu sākṣiṇaḥ | tṛṇagulmādikañcāpi viśvarūpaṃ smaret tataḥ |}
\note[type=source, labelb=302, lem={candrasūryau}]{SSP 3.13: candrasūryau netradvaye vasataḥ | anekavṛkṣalaṭāgulmatṛṇāni jaṅghāromakakūpasthāne vasanti |}
candra\app{\lem[wit={ceteri}, alt={°sūryau}]{sūryau}
  \rdg[wit={B,D,L}]{°sūryo}}
\app{\lem[wit={D,E,P},alt={dvayor}]{dvayo\skp{r-ne}}
  \rdg[wit={B}]{dvayā°}
  \rdg[wit={L,U2}]{dvayo}
  \rdg[wit={U1}]{\om}
}\app{\lem[wit={D,E},alt={netrayor}]{\skm{r-ne}trayo\skp{r-ma}}
  \rdg[wit={P}]{netreyor}
  \rdg[wit={B}]{netrayo}
  \rdg[wit={U2}]{netrayoḥ}
  \rdg[wit={U1}]{netradvaya}
}\app{\lem[wit={ceteri},alt={madhye}]{\skm{r-ma}dhye}
  \rdg[wit={U1}]{\om}}
\app{\lem[wit={ceteri}]{vartate}
  \rdg[wit={U2}]{pravartate}
  \rdg[wit={U1}]{vasaṃti}}/
\app{\lem[wit={B,E,L,P}]{anekavanaspatigulmalatātṛṇāni}
  \rdg[wit={D}]{anaikavanaspatigulmatṛṇāni}
  \rdg[wit={U1}]{anekavanaspatīgulmalatāni}
  \rdg[wit={U2}]{anekavana | spatigulmalatātṛṇāni}}
jaṅghā\app{\lem[wit={ceteri},alt={°roma°}]{roma}
  \rdg[wit={B,L}]{°rora°}}\app{\lem[wit={ceteri}]{madhye}
  \rdg[wit={D}]{sthāne}}
\app{\lem[wit={ceteri}]{vasanti}
  \rdg[wit={U2}]{vasati}
  \rdg[wit={D}]{varttaṃte}}/
%-----------------------------
%puruṣasya nṛtyadarśanāt gītaśravaṇāt/ vallabhavastuno  darśanāt/ yaḥ ānanda utpadyate saḥ svargalokaḥ               kathyate/ rogapīḍito durjanebhyaḥ puruṣasya yat duḥkham utpadyate   tadbahutaraṃ  narakaṃ kathyate// \E
%puruṣasya nṛtyadarśanāt gītaśravaṇāt  vallabhavastuno  darśanāt  ya  ānanda utpadyate     svargalokaḥ               kathyate  rogapīḍato durjjanebhya puruṣasya yaduḥkham   utpadyate   tadbahutaraṃ  narakaṃ kathyate \P
%puruṣasya nityadarśanāt gītaśravaṇāt/ vallabhavastuno  darśanāt/ yaḥ ānanda utpadyate     svargalokaḥ               kathyate  rogapīḍato durjanebhya  puruṣasya yat duḥkha  utpadyate// tadbahutaraṃ  narakaṃ kathyate// \B
%puruṣasya nityadarśanāt gītaśravaṇāt  vallabhavastuno  darśanāt  yaḥ ānanda utpadyate     svargalokaḥ               kathyate  rogapīḍano durjanebhya  puruṣasya yad duḥkhaṃ utpadyate// tadbahutaraṃ  narakaṃ kathyate// \L
%puruṣasya nṛtyadarśanād gītaśravaṇāt  vallabhavastuno  darśanāt  yaḥ ānanda utpadyate sa bahurānaṃdaḥ svargaphulaḥ? kathyate/ rogapīḍā   durjanebhyaḥ puruṣasya duḥkhaṃ     utpadyate// tat bahutaraṃ nakaṃ   kathyate/ \D
%puruṣasyāvādya   nṛtyodgītaśravaṇād  vallabhavasttuno  darśanād  yā  ānanda utpadyate sa bahurānaṃdaḥ svargaphalaḥ? kathyate  rogapīḍa   durjanebhyaḥ puruṣasya duḥkham     utpadyate       bahutaraṃ narakaṃ kathyate \U1
%puruṣasya darśanāt//   gītaśravaṇāt// vallabhavastuno  darśanāt//    ānanda utpadyate sa svargaloka                 kathyate//rogapīḍāto durjanebhyaḥ puruṣasya duḥkha      utpadyate    tadbahutaraṃ narakaṃ kathyate// \U2
%\om                                                                 \N1
%\om                                                                 \N2
%-----------------------------
%Of a person who generates joy because of enjoying the desired objects, because of hearing music [or] because of watching dance, he is said to belong to the people of heaven. Of a person who generates suffering, who is evil, plagued and sick, he is said to belong to [people] much worse than of them from hell.    
%-----------------------------
\note[type=source, labelb=303, lem={darśanāt}]{Ysv\textsuperscript{PT}: samagradarśanān muktaḥ svargabhogañ ca matsukham | tad etac cintayā yāti rogaśokavivarjjitaḥ |}
\note[type=source, labelb=304, lem={darśanāt}]{SSP 3.14: yat sukhaṃ tat svargaṃ yad duḥkhaṃ tan narakaṃ yat karma tad bandhanaṃ yan nirvikalpaṃ tan muktiḥ svasvarūpadaśāyāṃ nidrādau svātmajāgaraḥ śāntir bhavati | evaṃ sarvadeheṣu viśvasvarūpaḥ parameśvaraḥ paramātmā akhaṇḍasvabhāvena ghaṭe ghaṭe citsvarūpī tiṣṭhati ||3.14||}
\app{\lem[wit={ceteri}]{puruṣasya}
  \rdg[wit={U1}]{puruṣasyāvādya}}
\app{\lem[wit={D,E,P}]{nṛtyadarśanāt}
  \rdg[wit={D}]{nityadarśanād}
  \rdg[wit={U2}]{darśanāt ||}
  \rdg[wit={U1}]{nṛtyod°}}\dd{}
\app{\lem[wit={ceteri}]{gītaśravaṇāt}
  \rdg[wit={U1}]{gītaśravaṇād}}\dd{}
 vallabhavastuno
\app{\lem[wit={U1}]{darśanāt}
  \rdg[wit={U1}]{darśanād}}\dd{}
\app{\lem[wit={P}]{ya}
  \rdg[wit={U1}]{yā}
  \rdg[wit={B,D,E,L}]{yaḥ}
  \rdg[wit={U2}]{\om}}
ānanda utpadyate
\app{\lem[wit={E}]{saḥ}
  \rdg[wit={D,U1,U2}]{sa}}
\app{\lem[wit={B,E,L,P}]{svargalokaḥ}
  \rdg[wit={U2}]{svargaloka}
  \rdg[wit={D}]{bahurānaṃdaḥ svargaphulaḥ}
  \rdg[wit={U1}]{bahurānaṃdaḥ svargaphalaḥ}}
kathyate/
roga\app{\lem[wit={E},alt={°pīḍito}]{pīḍito}
  \rdg[wit={P,B}]{°pīḍato}
  \rdg[wit={U2}]{°pīḍāto}
  \rdg[wit={L}]{°pīḍano}
  \rdg[wit={D}]{°pīḍā}
  \rdg[wit={U1}]{°pīḍa}}
\app{\lem[wit={ceteri}]{durjanebhyaḥ}
  \rdg[wit={P,B,L}]{durjanebhya}}
puruṣasya
\app{\lem[wit={L}]{yad\skp{-}duḥkhaṃ}
  \rdg[wit={E}]{yat duḥkham}
  \rdg[wit={B}]{yat duḥkha}
  \rdg[wit={P}]{yaduḥkham}
  \rdg[wit={D,U1}]{duḥkhaṃ}
  \rdg[wit={U2}]{duḥkha}}
utpadyate/ 
\app{\lem[wit={ceteri}]{tad\skp{-}bahutaraṃ}
  \rdg[wit={D}]{tat bahutaraṃ}
  \rdg[wit={U1}]{bahutaraṃ}}
\app{\lem[wit={ceteri}]{narakaṃ}
  \rdg[wit={U1}]{nakaṃ}}
kathyate/
%-----------------------------
%                                                                                                      atha ca yatkarmakaraṇāt manomadhye śaṃkā na bhavati    tatkarma muktikāraṇam/ \E
%                                                                                                      atha ca yatkarmakaraṇān manomadhye śaṃkā na bhavati    tatkarma muktikāraṇam   \P %%%7662.jpg 
%                                                                                                      atha ca yatkarmakaraṇāt manobudhye śaṃkā na bhavati    tatkarma kamuktikāraṇam// \B
%                                                                                                      atha ca yatkarmakaraṇāt manobudhye śaṃkā na bhavati    tatkarma kamuktikāraṇam// \L
%                                                                                                     \om                                                                 \N1
%                                                                                                      atha ca yatkarmakaraṇāt manomadhye śaṃkā na bhaviti    tatkarma muktikāraṇaṃ// \D
%                                                                                                     \om                                                                  \N2
%atha ca yatkarmakaraṇāt sarveṣāṃ lokānāṃ svamanasī ca śubhaṃ na bharate tatkarma baṃdhanam ity ucyate atha ca yatkarmakaraṇāt manomadhye śaṃkā na bhavati    tatkarma muktikāraṇam \U1
%                                                                                                      atha ca yatkarmakaraṇān manomadhye śakā  na bhavaṃti// tatkarma muktikāraṇaṃ// \U2
%-----------------------------
%But the one whose own soul because of the execution of his action cannot bear well-being, his action binds, it is taught. But in whose mind because of the execution of action fear does not arise, his action causes liberation. 
%----------------------------
atha ca yatkarmakaraṇāt sarveṣāṃ lokānāṃ svamanasī ca śubhaṃ na bharate tatkarma baṃdhanam-ity-ucyate/
\note[type=source, labelb=305, lem={śaṅkā}]{Ysv\textsuperscript{PT}: tad etac cintayā yāti rogaśokavivarjjitaḥ | yatkarmā karmaṇā śaṅkā manomadhye bhavedvahiḥ | tatkarmakaraṇaṃ muktir ity āha bhagavān śivaḥ |}
\note[type=philcomm, labelb=306, lem={atha ca \ldots ity ucyate}]{This sentence is only preserved in witness U\textsubscript{1}.}
atha ca \app{\lem[wit={P,U2},alt={yatkarmakaraṇān}]{yatkarmakaraṇā\skp{n-ma}}
  \rdg[wit={ceteri}]{yatkarmakaraṇāt}
}\app{\lem[wit={ceteri},alt={manomadhye}]{\skm{n-ma}nomadhye}
  \rdg[wit={B,L}]{manobudhye}} 
\app{\lem[wit={ceteri}]{śaṅkā}
  \rdg[wit={U2}]{śakā}}
na
\app{\lem[wit={ceteri}]{bhavati}
  \rdg[wit={U2}]{bhavaṃti}}
tatkarma
\app{\lem[wit={ceteri}]{muktikāraṇaṃ}
  \rdg[wit={L,B}]{kamuktikāraṇam}}/
\end{prose}
\end{ekdosis}
\begin{ekdosis}
\ekddiv{type=ed}
    \centerline{\textrm{\small{[Attributes of the Rājayogic Body]}}}
    \bigskip
      \begin{prose}
%----------------------------
%idānīṃ rājayogāc charīre    yādṛśāni cihnāni bhavanti   tāni kathyante// \E
%idānī  rogayogācharīre     etādṛśāni cihnāni bhavaṃti   tāni kathyaṃte   \P
%idānī  rājayogāc charīre// etādṛśāni cihnāni bhavaṃti// tāni kathyaṃte// \B 7170.jpg end 7171.jpg beginning
%idānīṃ rājayogāc charīre   etādṛśāni cihnāni bhavaṃti   tāni kathyaṃte// \L
%\om                                                                     \N1
%idānīṃ rājayogāc charīre   etādṛśāni cihnāni bhavaṃti// tāni kathyaṃte// \D
%\om                                                                     \N2
%idānīṃ rājayogācharīre    etādṛśāni cihnāni bhavaṃti   tāni kathyaṃte   \U1
%idānī  rājayogāśarīre     etādṛśāni cihnāni bhavaṃti// tāni kathyaṃte// \U2
%-----------------------------
%Now such signs arise in the body from Rājayoga. They are about to be taught. 
%-----------------------------
\app{\lem[wit={ceteri}]{idānīṃ}
  \rdg[wit={B,P,U2}]{idānī}}
\app{\lem[wit={D,E,L}]{rājayogāc\skp{-}charīre}
  \rdg[wit={B}]{rājayogāc charīre ||}
  \rdg[wit={U1}]{rājayogācharīre}
  \rdg[wit={U2}]{rājayogāśarīre}
  \rdg[wit={P}]{rogayogācharīre}}
\app{\lem[wit={ceteri}]{etādṛśāni}
  \rdg[wit={E}]{yādṛśāni}}
cihnāni bhavanti/ tāni kathyante/
%-----------------------------
%sakalaroganāśaḥ   sakalapṛthvīṃ   paśyati/  tad anaṃtaraṃ              jñānam utpadyate// \E [p.55]
%sakalaroganāśaḥ   sakalāṃ pṛthvīṃ paśyati   tad aṃtaraṃ                jñānam utpadyate   \P
%sakalaroganāśaḥ   sakalapṛthvīṃ   paśyatī/  tad anaṃtaraṃ              jñānam utpadyate// \B
%sakalaroganāśaḥ   sakalapṛthvīṃ   paśyati/  tad anaṃtaraṃ              jñānam utpadyate// \L
%\om                                                                                       \N1
%sakalaroganāśaḥ   sakalapṛthvīṃ   paśyatī/  tad anaṃtaraṃ tatvaviṣayaṃ jñānam utpadyate/  \D %%%p. 15 verso 
%\om                                                                                       \N2
%sakalarogaḥ nāśaḥ sakalapṛthvīṃ   paśyati   tad anaṃtaraṃ tatvaviṣayaṃ jñānam utpadyate   \U1
%sakalaroganāśaḥ   sakalapṛthvīṃ   paśyati// tad anaṃtara---------------jñānam utpadyate// \U2
%-----------------------------
%All diseases are destroyed. He sees the entire earth. Then (tad anaṃtaraṃ) knowledge in the realm of reality is generated.   
%-----------------------------
\note[type=source, labelb=308, lem={tatvaviṣayaṃ jñānam}]{Ysv\textsuperscript{PT}: yasya darśanamātreṇa rogaśokavivarjitaḥ | paramānandacittaḥ syāt tapasvī caiva kīrttitaḥ | saptadvīpā bhaved dṛṣṭā tattvajñānaṃ tato bhavet | sarvabhāvaṃ vijānīyād vajradeho bhavet tathā | sarpadaṣṭe viṣaṃ na syāt kṣudhā nidrā tṛṣā tathā |}
\app{\lem[wit={ceteri}]{sakalaroganāśaḥ}
  \rdg[wit={U1}]{sakalarogaḥ nāśaḥ}}
\app{\lem[wit={ceteri}]{sakalapṛthvīṃ}
  \rdg[wit={P}]{sakalāṃ pṛthvīṃ}}
paśyati/
\app{\lem[wit={ceteri}]{tad\skp{-}anantaraṃ}
  \rdg[wit={P}]{tad aṃtaraṃ}
  \rdg[wit={U2}]{tad anaṃtara°}}
\app{\lem[wit={D,U1}]{tattvaviṣayaṃ}
  \rdg[wit={ceteri}]{\om}}
jñānam-utpadyate/
%-----------------------------
%samagrā  bhāṣā  jānāti/  tataḥ puruṣasya deho vajramayo bhavati/  sarpadaṃśena    maraṇaṃ na bhavati/   \E
%samagrāṃ bhāṣāṃ jānāti   tataḥ puruṣasya deho vajramayo bhavati   sarpadaṃśo      maraṇaṃ na bhavati    \P
%samagrā  bhāṣa  jānāti   tataḥ puruṣasya deho vajramayo bhavati// sarpadaṃśema    maraṇaṃ na bhavatī/   \B
%samagra  bhāṣā    jānāti tataḥ puruṣasya deho vajramayo bhavati// sarpadaṃśe      maraṇaṃ    bhavati//  \L
%\om                                                                                                     \N1
%samagrāṃ bhāṣāṃ jānāti/  tataḥ puruṣasya deho vajramayo bhavati/  sarpadaṃśe satī maraṇaṃ na bhavati/   \D
%\om                                                                                                     \N2
%samagrāṃ bhāṣāṃ jānāti   tataḥ puruṣasya deho vajramayo bhavati   sarpadaṃśe satī maraṇaṃ na bhavati    \U1
%samagrā bhāṣā   jānāti// tataḥ puruṣasya deho vajramayo bhavati// sarpadaṃśe      maraṇaṃ na vati//     \U2
%-----------------------------
%He knows the language completely. Because of that [Rājayoga?] the body of the human becomes indestructable. Death through a snake-bite does not arise. 
%-----------------------------
\app{\lem[wit={P,D,U1}]{samagrāṃ bhāṣāṃ}
  \rdg[wit={E,U2}]{samagrā bhāṣā}
  \rdg[wit={B}]{samagrā bhāṣa}
  \rdg[wit={L}]{samagra bhāṣā}}
jānāti/
tataḥ puruṣasya deho vajramayo bhavati
sarpa\app{\lem[wit={E},alt={°daṃśena}]{daṃśena}
  \rdg[wit={P}]{°daṃśo}
  \rdg[wit={B}]{°daṃśema}
  \rdg[wit={ceteri}]{°daṃśe}}
\app{\lem[wit={D,U1}]{satī}
  \rdg[wit={ceteri}]{\om}}
maraṇaṃ
\app{\lem[wit={ceteri}]{na}
  \rdg[wit={L}]{\om}}
\app{\lem[wit={ceteri}]{bhavati}
  \rdg[wit={B}]{bhavatī}
  \rdg[wit={U2}]{vati}}/
\end{prose}
\end{ekdosis}
\ekdpb*{}
%%%%%%%%%%%%%%%%%%%%%%%%%%%%%%%%%%%%%%%%%%
%%%%%%%%PAGEBREAK%%%%%%%PAGEBREAK%%%%%%%%%
%%%%%%%%%%%%%%%%%%%%%%%%%%%%%%%%%%%%%%%%%%
%%%%%%%%%%%%%%%%PAGEBREAK%%%%%%%%%%%%%%%%%
%%%%%%%%%%%%%%%%%%%%%%%%%%%%%%%%%%%%%%%%%%
%%%%%%%%PAGEBREAK%%%%%%%PAGEBREAK%%%%%%%%%
%%%%%%%%%%%%%%%%%%%%%%%%%%%%%%%%%%%%%%%%%%
%%%%%%%%%%%%%%%%%%%%%%%%%%%%%%%%%%%%%%%%%%
%%%%%%%%%%%%%%%%%%%%%%%%%%%%%%%%%%%%%%%%%%
%%%%%%%%%%%%%%%%%%%%%%%%%%%%%%%%%%%%%%%%%%
%%%%%%%%PAGEBREAK%%%%%%%PAGEBREAK%%%%%%%%%
%%%%%%%%%%%%%%%%%%%%%%%%%%%%%%%%%%%%%%%%%%
%%%%%%%%%%%%%%%%PAGEBREAK%%%%%%%%%%%%%%%%%
%%%%%%%%%%%%%%%%%%%%%%%%%%%%%%%%%%%%%%%%%%
%%%%%%%%PAGEBREAK%%%%%%%PAGEBREAK%%%%%%%%%
%%%%%%%%%%%%%%%%%%%%%%%%%%%%%%%%%%%%%%%%%%
%%%%%%%%%%%%%%%%%%%%%%%%%%%%%%%%%%%%%%%%%%
%%%%%%%%%%%%%%%%%%%%%%%%%%%%%%%%%%%%%%%%%%
%%%%%%%%%%%%%%%%%%%%%%%%%%%%%%%%%%%%%%%%%%
%%%%%%%%PAGEBREAK%%%%%%%PAGEBREAK%%%%%%%%%
%%%%%%%%%%%%%%%%%%%%%%%%%%%%%%%%%%%%%%%%%%
%%%%%%%%%%%%%%%%PAGEBREAK%%%%%%%%%%%%%%%%%
%%%%%%%%%%%%%%%%%%%%%%%%%%%%%%%%%%%%%%%%%%
%%%%%%%%PAGEBREAK%%%%%%%PAGEBREAK%%%%%%%%%
%%%%%%%%%%%%%%%%%%%%%%%%%%%%%%%%%%%%%%%%%%
\begin{ekdosis}
  \begin{prose}
    \noindent
%-----------------------------
%tataḥ puruṣasya bubhukṣā--pipāsā--nidrollatā------śītoṣṇatā bādhāṃ na kurvanti/ \E
%tataḥ puruṣasya bunnukṣā--pipāsā--nidrolmatā------śīta----tābādhā na kurvaṃti \P
%tatpuruṣasya    babhukṣā--pipāsā--nidrollatā------śīta------bādhā na kurvanti/ \B
%tatpuruṣasya    babhukṣā--pipāsā--nidroṣṇatā------śīta------bādhā na kurvanti... \L
%\om                                                                 \N1
%tataḥ puruṣasya bubhukṣā--pipāsā--nidrā/ uṣṇatā// śīta nā   bādhāṃ na kuroti???/ \D
%\om                                                                 \N2
% \om                                                         \U1
%tataḥ puruṣasya bubhukṣā--pipāsā--nidroṣṭṇatā-----śīta------bādhāṃ na kurvaṃti// \U2
%-----------------------------
%In consequence of that, for the person affliction of hunger, thirst, sleep, heat is inactive.    
%-----------------------------
\app{\lem[wit={ceteri}]{tataḥ}
  \rdg[wit={B,L}]{tat°}}
\note[type=philcomm, labelb=307, lem={tataḥ \ldots kurvanti}]{The sentence is \om in U\textsubscript{1}.}
puruṣasya
\app{\lem[wit={E,D,U2}]{bubhukṣā}
  \rdg[wit={P}]{bunnukṣā}
  \rdg[wit={B,L}]{babhukṣā}
}pipāsa\app{\lem[wit={L},alt={°nidroṣṇatā°}]{nidroṣṇatā}
  \rdg[wit={U2}]{°nidroṣṭṇatā°}
  \rdg[wit={D}]{nidrā |  uṣṇatā ||}
  \rdg[wit={E,B}]{nidrollatā}
  \rdg[wit={P}]{nidrolmatā}
}\app{\lem[wit={ceteri},alt={°śīta°}]{śīta}
  \rdg[wit={P}]{śītatā}
  \rdg[wit={E}]{śītoṣṇatā}
  \rdg[wit={D}]{śīta nā}
}\app{\lem[wit={P,B,L}]{bādhā}
  \rdg[wit={E,D,U2}]{bādhāṃ}}
na
\app{\lem[wit={ceteri}]{kurvanti}
  \rdg[wit={D}]{kuroti}}/
%-----------------------------
%vāksiddhir bhavati/  vidyatpāte           kācidbādhāpi na bhavati// \E
%vāksiddhir bhavati                                        \P
%vāksiddhir bhavatī/  vidyutpāte           kācidglānir na bhavati// \B
%vāksiddhir bhavati/  vidyutpāte           kācidglānir na bhavati// \L
%\om                                                                 \N1
%vāksiddhir bhavati/  vidyutpāte śarīre    na kiṃcid glānir bhavati/ \D
%\om                                                                 \N2
%                     vidyutpāte śarīre    kvācid glānir na bhavati  \U1
%vāksiddhir bhavati// vidyutpāte           kācid  dhānir na bhavati// \U2
%-----------------------------
%Perfection of speech arises. Within the moment of a thunderstroke any kind of fatigue does not arise in the body.  
%-----------------------------
\note[type=source, labelb=308, lem={vāksiddhiḥ}]{Ysv\textsuperscript{PT}: uṣṇatā śītatā ceti vāksiddhiḥ syān na saṃśayaḥ | vidyutpāte 'pi dehasya kvacid dhānir na jāyate |}
      vāksiddhir-bhavati/ vidyutpāte
    \app{\lem[wit={D,U1}]{śarīre}
      \rdg[wit={ceteri}]{\om}}
    \app{\lem[wit={U1},alt={kvācid glānir na}]{kvācid glānirna}
      \rdg[wit={B,L}]{kācid glānir na}
      \rdg[wit={D}]{na kiṃcid glānir}
      \rdg[wit={E}]{kācidbādhāpi}
      \rdg[wit={U2}]{kācid dhānir na}}
    bhavati/       
%-----------------------------
%tadanaṃtaraṃ  pavanarūṣī puruṣī bhavati/  samagrāṃ pṛthvīṃ dṛṣṭyā paśyati/   aṇimādyaṣṭasiddhir bhavati/ \E
%tadanaṃtaraṃ  pavanarūpī puruṣo bhavati   samagrāṃ pṛthvīṃ dṛṣṭyā paśyati    aṇimādyaṣṭasiddhir bhavati  \P
%tadanaṃtara   pavanarūpi puruṣo bhavati// samagrāṃ pṛthvī  dṛṣṭā paśyati/    aṇimādyāṣṭasiddhir bhavati/ \B scribe switches so much between i and ī
%tadanaṃtaraṃ  pavanarūpi puruṣo bhavati// samagrāṃ pṛthvīṃ dṛṣṭā paśyati//   aṇimādyāṣṭasiddhir bhavati// \L
%\om                                                                 \N1
%tadanaṃtaraṃ  pavanayopī puruṣo bhavati   samagrāṃ pṛthvīṃ dṛṣṭyā paśyati/   aṇimādyaṣṭasiddhir bhavati//  \D
%\om                                                                 \N2
%tadanaṃtaraṃ  pavanayogī puruṣo bhavati   samagrāṃ pṛthvīṃ dṛṣṭvā paśyati    aṇimādyāṣṭasiddhir bhavati  \U1 %%%291.jpg
%tadanaṃtaraṃ  pavanarūpī puruṣo bhavati// samagrāṃ pṛthvīṃ dṛṣṭvā paśyati//  aṇimāmahimāgarimāladhimā tathā prātikāmyamīśatvaṃ// viśītvaṃ// ity āṣṭasiddhayaḥ////  \U2
%-----------------------------
%Then the person becomes yogi of the wind. He sees the entire earth with a glance (dṛṣṭyā inst sg). The eight supernatural powers arise.  
%-----------------------------
\note[type=source, labelb=309, lem={pavanayogī}]{Ysv\textsuperscript{PT}: tato 'sau vāyuyogī syād dṛṣṭvā pṛthvīkulānvitaḥ | aṇimādy aṣṭasiddhiḥ syān mahāpadmodayas tathā | āgacchanti samīpe ca nidhayo nātra saṃśayaḥ |}
    tadanantaraṃ
    \app{\lem[wit={U1}]{pavanayogī}
      \rdg[wit={D}]{pavanayopī}
      \rdg[wit={P,U2}]{pavanarūpī}
      \rdg[wit={B,L}]{pavanarūpi}
      \rdg[wit={E}]{pavanarūṣī}}
    \app{\lem[wit={ceteri}]{puruṣo}
      \rdg[wit={E}]{puruṣī}} bhavati/
    samagrāṃ
    \app{\lem[wit={ceteri}]{pṛthvīṃ}
      \rdg[wit={B}]{pṛthvī}}
    \app{\lem[wit={D,E,P}]{dṛṣṭyā}
      \rdg[wit={B,L}]{dṛṣṭā}
      \rdg[wit={U1,U2}]{dṛṣṭvā}} paśyati/
    \app{\lem[wit={ceteri},alt={aṇimādyaṣṭasiddhir}]{aṇimādyaṣṭasiddhi\skp{r-bha}}
      \rdg[wit={U2}]{aṇimāmahimāgarimāladhimā tathā}}
    \app{\lem[wit={ceteri},alt={bhavati}]{\skm{r-bha}vati}
      \rdg[wit={U2}]{ prātikāmyamīśatvaṃ || viśītvaṃ || ity āṣṭasiddhayaḥ ||}}/ 
  \end{prose}
\end{ekdosis}
\begin{ekdosis}
%-----------------------------
%                                                                                                               mahāpadmādyā nava nidhyayaḥ samīpa āgacchanti/ \E
%śrīpadmaś ca mahāpadmaḥ saṃkho makarakachapa       kuṃdonukuṃda------nīlaś ca vijñeyā nidhayo nava-------------mamahāpadmā  dhānavanidhaya samīpe āgachaṃti \P     %%%7663.jpg 
%śrīpadmaś ca mahāpadmaṃ śaṃkho makarakacchapaḥ//   kuṃdonukuṃdoś  ca nīlaś-ca vajrayonī cīdātmakā// śrīnamaḥ   mahāpadmājñā navinidhyayaḥ//samipe āgacchatī//  nava nidhayaḥ samīpa āgacchanti/ \B
%śrīpadmaś ca mahāpadmaṃ śaṃkho makarakachapaḥ//    kuṃdonukuṃdoś  ca nīlaś ca vajrayonī cidātmakā// śrīnamaḥ   mahāpadmājñā nanidhyayaḥ//  samipe āgacchaṃti// -------------\om-------- \L
%                                                                                                               mahāpadmādyā nidhyayaḥ      samīpe āgacchaṃti// \D
%                                                                                                               mahāpadmādyā nava nidhapa   samīpe āgacchaṃti \U1
%   padmaś ca mahāpadmaś ca śaṃkho makarakachapaḥ// mukuṃdo kuṃdaś ca nīlaś ca vajrayo navanidhi//etādṛśaṃ                                  samīpe āgacchati// \U2 %%%421.jpg
%\om \N1
% \om \N2
%mahāpadmaśca padmaśca śaṅkho makarakacchapau | mukundakunda- nīlāśca kharvaśca nidhayo nava ---> Wisdomlib quote 
% -----------------------------
% 1. Padma (lotus) and 2. Mahāpadma (great lotus), 3. Śaṃkha (conch), 4. Makara (crocodile), 5. Kacchapa (turtle), 6. Mukunda (gem), 7. Kunda (jasmine), 8. Nīla (saphire) und 9. Kharva (another gem). The nine treasures beginning with the mahāpadma etc. return nearby.
%-----------------------------
\begin{tlg}
  \tl{\app{\lem[type=emendation, resp=egoscr]{mahāpadmaś-ca padmaś-ca}
    \rdg[wit={U2}]{\korr padmaś ca mahāpadmaś ca}
    \rdg[wit={P,B}]{śrīpadmaś ca mahāpadmaṃ}
    \rdg[wit={D,E,L,U1}]{\om}}
\app{\lem[wit={B,L,U2}]{śaṅkho}
  \rdg[wit={P}]{saṃkho}
  \rdg[wit={D,U1}]{\om}}
\app{\lem[type=emendation, resp=egoscr]{makarakacchapau}
  \rdg[wit={B,L,U2}]{\korr makarakachapaḥ}
  \rdg[wit={P}]{makarakachapa°}}}\\
\tl{\app{\lem[type=emendation, resp=egoscr,alt={mukundakundanīlāś ca}]{mukundakundanīlāś-ca}
  \rdg[wit={U2}]{\korr mukuṃdo kuṃdaś ca nīlaś ca}
  \rdg[wit={P}]{kuṃdonukuṃdanīlaś ca}
  \rdg[wit={B,L}]{kuṃdonukuṃdoś ca nīlaś ca}}
\app{\lem[type=emendation, resp=egoscr,alt={kharvaś ca nidhayo nava}]{kharvaś-ca nidhayo nava}
  \rdg[wit={P}]{\korr vijñeyāni dhayonava}
  \rdg[wit={B,L}]{vajrayonī cīdātmakā}
  \rdg[wit={U2}]{vajrayo navanidhi}}\dd{}1\hskip-2pt\dd{}}
\end{tlg}
\smallskip
\end{ekdosis}
\begin{ekdosis}
\begin{prose}
  \note[type=philcomm, labelb=310, lem={nidhayo nava}]{What must be meant here are the nine treasures of Kubera, mentioned in \textit{Śivapurāṇa} 2.3.15. I emenden according to the ``traditional'' list in circulation.}
  \app{\lem[wit={E,D,U1}]{mahāpadmādyā}
     \rdg[wit={B,L}]{mahāpadmājñā}
  \rdg[wit={P}]{mamahāpadmā}}
\app{\lem[wit={E}]{nava nidhyayaḥ}
  \rdg[wit={U1}]{nava nidhapa}
  \rdg[wit={D}]{nidhyayaḥ}
  \rdg[wit={L}]{nanidhyayaḥ ||}
  \rdg[wit={B}]{navinidhyayaḥ ||}
  \rdg[wit={P}]{dhānavanidhaya}}
\app{\lem[wit={E}]{samīpa}
  \rdg[wit={ceteri}]{samīpe}}
\app{\lem[wit={ceteri}]{āgacchanti}
  \rdg[wit={U2}]{āgacchati}
  \rdg[wit={B}]{āgacchatī ||  nava nidhayaḥ samīpa āgacchanti |}}/
%-----------------------------
%ākāśamadhye daśasu dikṣu        gamanāgamane bhavataḥ balaṃ bhavati/                                                                                     parameśvaraṃ samīpe paśyati/   karaṇe  haraṇe sāmarthyaṃ    bhavati// \E [P.56]
%ākāśamadhye daśasu dikṣu        gamanāgamanabalaṃ           bhavati       yatra loke gamanechā bhavati     tatra loke gacchati  ajñā sarvatra sphurati   parameśvaraṃ samīpe paśyati    karaṇe  haraṇe sāmarthyaṃ    bhavati \P
%ākāśamadhye daśasu dikṣu        gamanāgamanavallabhaṃ       bhavati//     yatra loke gamanechā bhavati/    yatra loke gacchati/ ajñā sarvatra sphurati// parameśvaraṃ samīpe paśyaṃtī/  karaṇe  haraṇe sāmarthyaṃ    bhavati// \B
%ākāśamadhye daśasu dikṣu        gamanāgamanavallabhaṃ       bhavati//     yatra loke gamanechā bhavati//  yatra loke gacchati// ajñā sarvatra sphurati   parameśvaraṃ samīpe paśyati    karaṇe  haraṇe sāmarthyaṃ    bhavati// \L%0028.jpg
%ākāśamadhye daśasu dikṣumadhye  gaṃmanāgamanabalaṃ          bhavatī/      yatra lo.. gamanechā bhavati    tatra loke gacchati/  ājñā sarvatra sphurati// parameśvaraṃ samīpe paśyati/   karaṇaṃ haraṇe .. ..marthyaṃ bhavati// \D
%ākāśa-------daśasu dikṣumadhye  gamanāgamanabalaṃ           bhavati       yatra loke gamanechā bhavatī    yatra loke gacchati   ājñā sarvatra sphurati   parameśvaraṃ samīpe paśyati    karaṇe  haraṇe ca sāmarthyaṃ bhavati \U1
%ākāśamadhye daśa   dikṣu        gamanāgamanabalam           bhavati//     yatra loke gamanechā bhavati//  tatra loke gacchati// ajñā sarvatra sphurati// parameśvaraṃ samipe paśyaṃti// karaṇe  taraṇe sāmarthyaṃ    bhavati// \U2
%\om                                                                 \N1
%\om                                                                 \N2
%-----------------------------
%Within the ten cardinal points in space the power over death and rebirth arises. Wherever [one] desires to go in the world, there he goes. Unlimited authority bursts forth everywhere. He sees the highest lord close by. In the making and in the taking away there is adequacy. 
%-----------------------------
\note[type=source, labelb=311, lem={gamanechā}]{Ysv\textsuperscript{PT}: yatrecchā gamanaṃ tatra svarge marttyerasātale | sphuraty ājñākhyaḥ sarvatra samīpe parameśvaraḥ | kāraṇe hāraṇe śakto rakṣaṇe'pi ca pārvati | ātmamadhye mano nityaṃ nirjane nivaset sudhīḥ | kṛtvātmamanasor aikyaṃ prāpnoti paramaṃ padam |}
\app{\lem[wit={ceteri}]{ākāśamadhye}
  \rdg[wit={U1}]{ākāśa°}}
\app{\lem[wit={ceteri}]{daśasu}
  \rdg[wit={U2}]{°daśa}}
\app{\lem[wit={ceteri}]{dikṣu}
  \rdg[wit={D,U1}]{dikṣumadhye}}
\app{\lem[wit={P,D,U1,U2}]{gamanāgamanabalaṃ}
  \rdg[wit={B,L}]{gamanāgamanavallabhaṃ}
  \rdg[wit={E}]{gamanāgamane bhavataḥ balaṃ}}
\app{\lem[wit={ceteri}]{bhavati}
  \rdg[wit={B}]{bhavatī}}/
yatra loke gamanechā
\app{\lem[wit={ceteri}]{bhavati}
  \rdg[wit={U1}]{bhavatī}}/
\app{\lem[wit={ceteri}]{tatra}
  \rdg[wit={B,P,U1}]{yatra}}
loke gacchati/
ajñā sarvatra sphurati/ parameśvaraṃ samīpe
\app{\lem[wit={ceteri}]{paśyati}
  \rdg[wit={B,U2}]{paśyaṃti}}/
\app{\lem[wit={ceteri}]{karaṇe}
  \rdg[wit={D}]{karaṇaṃ}}
\app{\lem[wit={ceteri}]{haraṇe}
  \rdg[wit={U2}]{taraṇe}}
\app{\lem[wit={ceteri}]{sāmarthyaṃ}
  \rdg[wit={U1}]{ca sāmarthyaṃ}
  \rdg[wit={D}]{....marthyaṃ}}
bhavati/
\end{prose}
\end{ekdosis}
%-----------------------------
%idaṃ gurubhakteḥ phalaṃ            ātmamadhye manaso viśrāma--karaṇamicchatā      puruṣeṇa sadguroḥ sevāṃ kṛtvā   sāvadhānaṃ manaḥ karaṇīyam/        abhyāsabalāt paramaprāptiḥ/  \E
%idaṃ gurubhaktaiḥ phalaṃ           ātmamadhye manaso viśrāma--karaṇamichatā       puruṣeṇa sadguroḥ sevāṃ kṛtvā   sāvadhānaṃ manaḥ karaṇīyaṃ         abhyāsabalāt paramaprāptiḥ \P
%idaṃ gurubhakteḥ  phalaṃ//         ātmamadhye manaso viśrāmaṃ karaṃṇaṃmicchatāṃ// puruṣeṇa sadguroḥ sevāṃ kṛtvā   sāvadhānaṃ manaḥ kṛtvā karaṇīyam// abhyāsabalāt paramaprāptiḥ//\B
%idaṃ gurubhakteḥ  phalaṃ//         ātmamadhye manaso viśrāmaṃ karaṇam icchatāṃ//  puruṣeṇa sadguroḥ sevāṃ kṛtvā   sāvadhānaṃ manaḥ kṛtvā karaṇīyaṃ...abhyāsabalāt// paramaprāptiḥ// \L
%idaṃ gurubhakteḥ  phalaṃ           ātmamadhye manaso viśrāma--karaṇam icchatā     puruṣeṇa sadguruḥ sevāṃ kṛ..    sāvadhānaṃ manaḥ karaṇīyaṃ/        abhyāsabalāt paramaprāptiḥ/\D
%idaṃ gurubhakteḥ  phalaṃ           ātmamadhye manaso viśrāma--karaṇam icchatā     puruṣeṇa sadguruḥ sevāṃ kṛtvā   sāvadhānaṃ manaḥ karaṇīyaṃ         abhyāsabalāt paramaprāptiḥ\U1
%idaṃ gurubhakteḥ  phalaṃ bhavati// ātmamadhye manaso viśrāme  karaṇam ichatā      puruṣeṇa sadguroḥ sevāṃ kṛtvā// māvadhānaṃ manaḥ karaṇīyaṃ//       abhyāsabalāt paramapadaprāptiḥ\U2
%\om                                                                 \N1
%\om                                                                 \N2
%-----------------------------
%This is the result of guru-devotion: Within the self there is the desire of the mind for the production of rest. The person - having done service to the guru - shall make the mind attentive. From the power of practice the highest place is reached.
%-----------------------------
\begin{ekdosis}
       \centerline{\textrm{\small{[Gurubhakti]}}}
       \bigskip
       \begin{prose}
         \noindent
idaṃ
\app{\lem[wit={ceteri}]{gurubhakteḥ}
  \rdg[wit={P}]{gurubhaktaiḥ}}
\app{\lem[wit={ceteri}]{phalaṃ}
  \rdg[wit={U2}]{phalaṃ bhavati}}/
ātmamadhye manaso
\app{\lem[wit={ceteri},alt={viśrāmakaraṇam}]{viśrāmakaraṇa\skp{m-i}}
  \rdg[wit={L}]{viśrāmaṃ karaṇam}
  \rdg[wit={B}]{viśrāmaṃ karaṃṇaṃm}
}\app{\lem[wit={ceteri},alt={icchatā}]{\skm{m-i}cchatā}
  \rdg[wit={B,L}]{icchatāṃ}}
puruṣeṇa
\app{\lem[wit={ceteri}]{sadguroḥ}
  \rdg[wit={D,U1}]{sadguruḥ}} 
sevāṃ
\app{\lem[wit={ceteri}]{kṛtvā}
  \rdg[wit={D}]{kṛ..}
  \rdg[wit={U2}]{kṛtvā ||}}
\app{\lem[wit={ceteri}]{sāvadhānaṃ}
  \rdg[wit={U2}]{māvadhānaṃ}}
manaḥ
\end{prose}
\end{ekdosis}
\ekdpb*{}
  \begin{ekdosis}
\begin{prose}
    \app{\lem[wit={ceteri}]{karaṇīyaṃ}
  \rdg[wit={L}]{kṛtvā karaṇīyaṃ}
  \rdg[wit={B}]{kṛtvā karaṇīyam ||}}
\app{\lem[wit={ceteri}]{abhyāsabalāt}
  \rdg[wit={L}]{abhyāsabalāt ||}}
\app{\lem[wit={ceteri}]{paramaprāptiḥ}
  \rdg[wit={U2}]{paramapadaprāptiḥ}}/
% -----------------------------
% tena      svaśiṣyamanasaḥ  svāsthyaṃ   karttavyam/  candrasūryyau yāvat piṃḍe  niścalau bhavataḥ//  \E %[p.57]
% tena      svasya manasaḥ   samarasyaṃ  karttavyam   caṃdrasūryau  yāvat piṃḍo  niścalo  bhavati     \P
% tena      svasya manasaḥ                                                                            \B %stemma point?! omission?!
% tena      svasya manasaḥ   samarasaṃ   karttavyaṃ   caṃdrasūrya---yāt   piṃḍo  niścalo  bhavati//   \L
% tena saha svasya manaḥ     samarasyaṃ  karttavyaṃ/  caṃdrasūryau  yāvit piṃde  niścalau bhavatiḥ//  \D
% tena saha svascha manaḥ                karttavyaṃ   caṃdrasūryau  yāvat piṃdau niścalo  bhavati     \U1
% tena      svasya manasaḥ   samarasyaṃ  karttavyaṃ// caṃdrasūrya---vat   piṃḍo  niścalo  bhavati//   \U2  %%%421verso.jpg
% \om                                                                \N1
%\om                                                                 \N2
%-----------------------------
%He shall create equanimity in his own mind. Just as the sun and moon [are unchangeable] in the same way and unchangeable body arises.
%----------------------------
\note[type=source, labelb=312, lem={caṃdrasūryau}]{Ysv\textsuperscript{PT}: candraḥ sūryaḥ sthiro yāvat tāvad dehasthitis tathā | tāvad ekaṃ samābhāṣya prāpnoti ca sadāgatiḥ | sa bhavet kavitā dhīrā niścalā śāntir eva ca | gurupādaprasādena tad aikyaṃ yāti siddhibhāk |}
\app{\lem[wit={ceteri}]{tena}
  \rdg[wit={D,U1}]{tena saha}}
\app{\lem[wit={P,B,L,U2}]{svasya manasaḥ}
  \rdg[wit={D}]{svasya manaḥ}
  \rdg[wit={U1}]{svascha manaḥ}
  \rdg[wit={E}]{svaśiṣyamanasaḥ}}
\app{\lem[wit={L}]{samarasaṃ}
  \rdg[wit={P,D,U2}]{samarasyaṃ}
  \rdg[wit={E}]{svāsthyaṃ}
  \rdg[wit={B,U1}]{\om}}
\app{\lem[wit={ceteri}]{karttavyaṃ}
  \rdg[wit={B}]{\om}}
\app{\lem[wit={E,P,U1}]{candrasūryau yāvat}
  \rdg[wit={D}]{caṃdrasūryau yāvit}
  \rdg[wit={L}]{caṃdrasūryayāt}
  \rdg[wit={U2}]{caṃdrasūryavat}
  \rdg[wit={B}]{\om}}
\app{\lem[wit={P,L,U2}]{piṃḍo}
  \rdg[wit={E,D}]{piṃḍe}
  \rdg[wit={U1}]{piṃḍau}
  \rdg[wit={B}]{\om}}
\app{\lem[wit={P,L,U1,U2}]{niścalo}
  \rdg[wit={D,E}]{niścalau}
  \rdg[wit={B}]{\om}}
\app{\lem[wit={ceteri}]{bhavati}
  \rdg[wit={E}]{bhavataḥ}
  \rdg[wit={D}]{bhavatiḥ}}/
\note[type=source, labelb=313, lem={samyaksvabhāva°}]{SSP 5.84: saṃvitkriyāvikaraṇodayacidvilāso viśrāntim eva bhajatāṃ svayam eva bhāti | graste svaveganicaye padapiṇḍam aikyaṃ satyaṃ bhavet samarasaṃ guruvatsalānām || 5.84 ||}
%A disciple enjoying the state of samvitkriyā, vikaraṇodaya, cidvilāsa and vishranti, becomes enlightened on his own. Those who are favourite to guru indeed enjoy merger with the Absolute, when pada and piṇḍa are identified on dissolution of one's mental activities. 
%----------------------------
%          samyak---svabhāva-kiraṇodaya---cidvilāsa--grastaṃ        svaśāṃti samatāṃ  svayam eva yāti/ \E %[p.57]
%          samyak---svabhāva-kiraṇodaya---cidvilāsa--grastaṃ        svaśāṃti manasā   svayam eva yāmi \P
%                                     samaradvilāsa//grastaṃ        svaśāṃti manasā   svam   eva śāṃti// \B %stemma point?! omission?!
% śloka    samyak---svabhāva-kiraṇodaya---cidvilāsa  grastaṃ        svaśāṃti manasā   svayam eva śāṃti... \L
% ślokaḥ// samyak---svabhāva-kiraṇodaya---cidvilāsaṃ/grastaṃ        svaśāṃti mavatāṃ  svayam eva yāti/ \D
% śloka    samyagaḥ svabhāva-kiraṇodaya---cidvilāsaṃ grastasamagraṃ saśāṃti  mahatāṃ  svayam eva yāti \U1
% ślokaḥ// samyak---svabhāva-karaṇotdṛdi--cidvilāsa--grastaṃ        svaśāṃti bhavatāṃ svayam eva yāti// \U2 %%%421verso.jpg
% \om                                                                \N1
%\om                                                                 \N2
%-----------------------------
%The complete inherent nature, the appearance of beams of light and play of the divine, completely posessed, inner peace in oneself, mightyness he reaches of his own accord. 
%-----------------------------
\app{\lem[wit={D,U2}]{ślokaḥ}
  \rdg[wit={L,U1}]{śloka}}\dd{}
\end{prose}
\end{ekdosis}
\begin{ekdosis}
\begin{tlg}
\tl{\app{\lem[wit={ceteri},alt={samyak°}]{samya\skp{k-sva}}
        \rdg[wit={U1}]{samyagaḥ}
}\skm{k-sva}bhāva\app{\lem[wit={ceteri},alt={°kiraṇodaya°}]{kiraṇodaya}
  \rdg[wit={U2}]{karaṇotdṛdi}
}\app{\lem[wit={ceteri},alt={°cidvilāsa°}]{cidvilāsa}
  \rdg[wit={B}]{samaradvilāsa ||}
  \rdg[wit={D}]{cidvilāsaṃ |}
  \rdg[wit={U1}]{cidvilāsaṃ}
}\app{\lem[type=emendation, resp=egoscr,alt={°grastasamagra°}]{grastasamagra}
  \rdg[wit={U1}]{\korr grastasamagraṃ}
  \rdg[wit={ceteri}]{grastaṃ}
}\app{\lem[wit={ceteri},alt={°svaśānti°}]{svaśānti}
  \rdg[wit={U1}]{saśāṃti}
}\app{\lem[wit={U1}]{mahatāṃ}
  \rdg[wit={U2}]{bhavatāṃ}
  \rdg[wit={D}]{mavatāṃ}
  \rdg[wit={E}]{samatāṃ}
  \rdg[wit={B,L,P}]{manasā}}
\app{\lem[wit={ceteri},alt={svayam}]{svaya\skp{m-e}}
  \rdg[wit={B}]{svam}} 
\skm{m-e}va
\app{\lem[wit={ceteri}]{yāti}
  \rdg[wit={P}]{yāmi}
  \rdg[wit={B,L}]{śāṃti}}}\\
\tl{
%-----------------------------
%graste svaveganicaye   padapiṃḍamaikyaṃ   satyaṃ bhavet samarasaṃ guruvatsalāṃ ca//1// \E
%graste svaveganicaye   padapiṃḍamaikyaṃ   satyaṃ bhavet samarasaṃ guruvatsalānāṃ 1  \P %%%7664.jpg
%graste svaveganicaye   padapiṃḍamaikyaṃ   sataṃ  bhavet samarasaṃ guruvatsalābhaṃ //1// \B
%graste svaveganicaye   padapiṃḍamaikyaṃ   satāṃ  bhavet samarasaṃ guruvatsalābhaṃ //1// \L
%\om                                                                 \N1
%graste svavegaṃ nicaye padapiḍamaikyaṃ    satyaṃ bhavet samarasaṃ-guruvatsalānāṃ//1//  \D
%\om                                                                 \N2
%graste svaveganiścaye  padapiṃḍamaikyaṃ   satyaṃ bhavet samarasaṃ-guruvatchalānāṃ 1  \U1
%grāme  sveraṃganicaye  yada piṃḍam aikyaṃ satyaṃ bhavet-samarasaṃ guruvatsalānāṃ// \U2
%-----------------------------
%Verschlungen eigene-schnellende Bewegung - Ansammlung -> Wenn die eigene Anhäufung [von Gedanken] ruckartig versiegt bei der Einswerdung von internen und externen Universum in Wahrhaftigkeit, welche bei Identifikation eintritt bei denen die von ganzer Seele dem Guru ergeben sind.
%-----------------------------
%Bei denen die dem Lehrer von ganzer Seele ergeben sind wird die kummulative Aktivität des eigenen Geistes ruckartig [vom Guru] genommen und die wahrhaftige Identifikation, die Einswerdung mit dem internen und externen Universum entsteht: die vollständige inhärente Natur, die Erscheinung von Lichtstrahlen, das göttliche Spiel, vollständige Verzückung, innerer Friede und Macht erreicht er wie von selbst.
%-----------------------------
\app{\lem[wit={ceteri}]{graste}
  \rdg[wit={U2}]{grāme}}
\app{\lem[wit={ceteri}]{svaveganicaye}
  \rdg[wit={D}]{svavegaṃ nicaye}
  \rdg[wit={U1}]{svaveganiścaye}
  \rdg[wit={U2}]{sveraṃganicaye}}
\app{\lem[wit={ceteri}]{padapiṃḍamaikyaṃ}
  \rdg[wit={D}]{padapiḍamaikyaṃ}
  \rdg[wit={U2}]{yada piṃḍam aikyaṃ}}
\app{\lem[wit={ceteri}]{satyaṃ}
  \rdg[wit={B}]{sataṃ}
  \rdg[wit={L}]{satāṃ}}
bhavet-samarasaṃ
\app{\lem[wit={D,P,U2}]{guruvatsalānāṃ}
  \rdg[wit={B,L}]{guruvatsalābhaṃ}
  \rdg[wit={E}]{guruvatsalāṃ ca}
  \rdg[wit={U1}]{guruvatchalānāṃ}}\dd{}1\hskip-2pt\dd{}}
\end{tlg}
\end{ekdosis}
\begin{ekdosis}
       \bigskip
       \centerline{\textrm{\small{[Avadhūta]}}}
       \bigskip
       \begin{prose}
         \noindent
%----------------------------TAGESZIEL(MINIMUM 12.12.2024). 
%idānīm avadhūtapuruṣasya lakṣaṇaṃ kathyate/ \E
%idānīm avadhūtapuruṣasya lakṣaṇaṃ kathyate \P
%idānīm avadhūtapuruṣasya lakṣaṇam āha/ \B DSCN7171.jpg last line
%idānīm avadhūtapuruṣasya lakṣaṇam āha// \L
%\om                                                                 \N1
%idānīm mavadhūtapuruṣasya lakṣaṇam kathyate// \D
%\om                                                                 \N2
%idānīm avadhūtapuruṣasya lakṣaṇam kathyate \U1
%idānīm avadhūtapuruṣasya lakṣaṇaṃ kathyate// \U2
%-----------------------------
%Now the characteristic of an Avadhūta-person is taught. 
%----------------------------
idānīm-avadhūtapuruṣasya
\app{\lem[wit={ceteri}]{lakṣaṇaṃ}
   \rdg[wit={B,L,D,U1}]{lakṣaṇam}}
\app{\lem[wit={ceteri}]{kathyate}
  \rdg[wit={B,L}]{āha}}/ 
       \end{prose}
     \end{ekdosis}
     \begin{ekdosis}
       \begin{tlg}
%----------------------------
%yasya haste  dhairyadaṇḍaḥ kharparaṃ  śūnyam āsanam/  yogaiśvaryeṇa saṃpannaḥ sovadhūta  udāhṛtaḥ//2// \E %%%SSP 6.10
%yasya haste  dhairyadaṇḍaḥ kharparaṃ  śūnyam āsanam   yogaiśvaryeṇa saṃpanna  sovadhūta  udāhṛtaḥ 2  \P
%yasya haste  dhairyadaṇḍaḥ kharparaṃ  śunyabhāsanam// yogaiśvaryai  saṃpannaḥ sovadhūtam udāhṛtaṃ// \B DSCN7172 Z.1
%yasya haste  dhairyadaṇḍaḥ kharparaṃ  śubhāsanam//    yogaiśvarye   saṃpannaḥ sovadhūtam udāhṛtaṃ// \L
%yasya haste  dhairyadaṇḍaḥ kharaparaṃ śūnyam ānasaṃ/  yogaiśvaryeṇa saṃpannaḥ sovadhūta  udāhṛtaḥ//2// \D
%yasya haste  dhairyadaṇḍaḥ kharaparaṃ śūnyanāmakaṃ    yogaiśvaryeṇa saṃpannaḥ sovadhūta  udāhṛtaḥ 2 \U1 %%%292.jpg
%yasya hastai dhairyadaṇḍaḥ kharparaṃ  śūnyam āsanaṃ// yogaiśvaryeṇa sapannaḥ  sovadhūta  udāhṛtaḥ//  \U2
%\om                                                                                            \N1
%\om                                                                                            \N2
%-----------------------------
%whose staff in the hand is [royal?]courage, whose begging bowl is the shine of emptiness. Furnished with the power of yoga, he is called an accomplished Avadhūta.  
%----------------------------
         \note[type=source, labelb=314, lem={yasya haste}]{SSP 6.10: yasya dhairyamayo daṇḍaḥ parākāśaṃ ca kharparaṃ | yogapaṭṭaṃ nijā śaktiḥ so 'vadhūto 'bhidhīyate ||6.10||}
         \noindent
\tl{yasya \app{\lem[wit={ceteri}]{haste}
    \rdg[wit={U2}]{hastai}}
dhairyadaṇḍaḥ \app{\lem[wit={ceteri}]{kharparaṃ}
  \rdg[wit={D,U1}]{kharaparaṃ}}
\app{\lem[wit={ceteri}, alt={śūnyam āsanaṃ}]{śūnyam\skp{-}āsanaṃ}
  \rdg[wit={B}]{śunyabhāsanam}
  \rdg[wit={U1}]{śūnyanāmakaṃ}}}\\
\tl{\app{\lem[wit={ceteri}]{yogaiśvaryeṇa}
  \rdg[wit={B}]{yogaiśvaryai}
  \rdg[wit={L}]{yogaiśvarye}}
\app{\lem[wit={ceteri}]{saṃpannaḥ}
  \rdg[wit={P}]{saṃpanna}
  \rdg[wit={U2}]{sapannaḥ}}
\app{\lem[wit={ceteri}]{sovadhūta}
  \rdg[wit={B,L}]{sovadhūtam}} 
\app{\lem[wit={ceteri}]{udāhṛtaḥ}
  \rdg[wit={B,L}]{udāhṛtaṃ}}\dd{}2\hskip-2pt\dd{}}
\end{tlg}
\begin{tlg}
%----------------------------           
%bhedābhedau yasya bhikṣā bharaṇaṃ  jāraṇaṃ tathā/   etādṛśopi  puraṣaḥ sovadhūta   udāhṛtaḥ//3//[p.58] SSP 6.11 \E
%bhedābhedau yasya bhikṣā bharaṇaṃ  jāgaraṃ tathā    etādṛśopi  puraṣaḥ sovadhūta   udāhṛtaḥ 3 \P
%bhedābhedau yasya bhikṣā bharaṇaṃ  jāraṇaṃ tathā//  tādṛśopi   puraṣaḥ sovadhūtam  udāhṛtaḥ//2// \B
%bhedābhedau yasya bhikṣā bharaṇaṃ  jāraṇaṃ tathā//  tādṛśopi   puraṣaḥ sovadhūtam  udāhṛtaḥ//2// \L
%bhedābhedau yasya bhikṣā bhakṣaṇaṃ jāraṇaṃ tathā//  etādṛśopi  puraṣaḥ sovadhūta   udāhṛtaḥ 3 \D
%bhedābhedau yasya bhikṣā bhakṣaṇaṃ jāraṇaṃ tathā    etādṛśopi  puraṣaḥ sovadhūta   udāhṛtaḥ 3 \U1
%bhedābhedo  yasya bhīkṣā bharaṇaṃ  jīraṇaṃ tathā//  etādṛśopi  puruṣaḥ sovadhūta   udāhṛtaḥ// \U2
%\om                                                                 \N1
%\om                                                                 \N2
%-----------------------------
%Whose alms are "difference and non-difference", whose dress is armor (jāgara!!!), such a person is called an Avadhūta. 
%----------------------------
  \note[type=source, labelb=315, lem={bhedābhedau}]{SSP 6.11: bhedābhedau svayaṃ bhikṣāṃ kṛtvā sāsvādane rataḥ | jāraṇaṃ tanmayībhāvaḥ so 'vadhūto 'bhidhīyate ||11||}
\tl{\app{\lem[wit={ceteri}]{bhedābhedau}
    \rdg[wit={U2}]{bhedābhedo}}
  yasya bhīkṣā
  \app{\lem[wit={ceteri}]{bharaṇaṃ}
    \rdg[wit={D,U1}]{bhakṣaṇaṃ}}
  \app{\lem[wit={P}]{jāgaraṃ}
    \rdg[wit={B,E,D,L,U1}]{jāraṇaṃ}
    \rdg[wit={U2}]{jīraṇaṃ}} tathā}\\
\tl{\app{\lem[wit={ceteri}]{etādṛśo 'pi}
    \rdg[wit={B,L}]{tādṛśopi}}
  puruṣaḥ 
\app{\lem[wit={ceteri}]{sovadhūta}
  \rdg[wit={B,L}]{sovadhūtam}}
udāhṛtaḥ\dd{}3\hskip-2pt\dd{}}
\end{tlg}
\begin{tlg}
%----------------------------       
%ātmā  hy akāro vijñeyo  vakāro bhavavāsanā/  dhūtaṃ  saṃtāpanaṃ  proktaṃ sovadhūto nigadyate// 4// \E
%ātmā  hy akāro vijñeyo  vakāro bhavavāsanā   dhūtas  tatkaṃpanaṃ proktaṃ sovadhūta nigadyate 3 \P%
%ātmāt dyukāro  vijñoyau vikāro bhavavāsanā// dhūtas  tatkaṃpanaṃ proktaṃ sovadhūta nigadyate// 3// \B
%ātmār dyukāro  vijñeyo  vikāro bhavavāsanā// dhūtas  tatkaṃpanaṃ proktaṃ sovadhūta nigadyate// 3// \L%%%0028.jpg last line
%ātmā  hy akāro vijñeyo  vakāro bhavavāsanā// dhūtasa tatkaṃpanaṃ proktaṃ sovadhūto nigadyate/ \D
%ātmai hy akāro vijñeyo  vakāro bhavavāsanā   dhūtas  tatkaṃpanaṃ proktaṃ sovadhūto nirucyate 4 \U1
%ā     hy akāro vijñeyo  vakāro bhavavāsanā   dhūtas  tatkaṃpanaṃ proktaṃ sovadhūto nigadyate// \U2
%\om                                                                 \N1
%\om                                                                 \N2
%-----------------------------
%The letter "a" is truly to be known as the self and the letter "va" as the impressions of existance. "Shaking off" is said to be his special vibration/weapon, he is called an Avadhūta. 
%-----------------------------
\tl{\app{\lem[wit={E,P,D}]{ātmā}
      \rdg[wit={B}]{ātmāt}
      \rdg[wit={L}]{ātmār}
      \rdg[wit={U1}]{ātmai}
      \rdg[wit={U2}]{ā}}
\app{\lem[wit={ceteri},alt={hy akāro}]{hy\skp{-}akāro}
  \rdg[wit={B,L}]{dyukāro}}
\app{\lem[wit={ceteri}]{vijñeyo}
  \rdg[wit={B}]{vijñoyau}}
\app{\lem[wit={ceteri}]{vakāro}
  \rdg[wit={B,L}]{vikāro}}
bhavavāsanā/}\\
\tl{\app{\lem[wit={ceteri},alt={dhūtas}]{dhūta\skp{s-ta}}
    \rdg[wit={E}]{dhūtaṃ}
    \rdg[wit={D}]{dhūtasa}}
  \app{\lem[wit={ceteri}]{tatkaṃpanaṃ}
    \rdg[wit={E}]{saṃtāpanaṃ}}
  proktaṃ
  \app{\lem[wit={ceteri}]{sovadhūto}
    \rdg[wit={B,L,P}]{sovadhūta}}
  \app{\lem[wit={ceteri}]{nigadyate}
    \rdg[wit={U1}]{nirucyate}}\dd{}4\hskip-2pt\dd{}} 
\end{tlg}
\end{ekdosis}
\ekdpb*{}
%%%%%%%%%%%%%%%%%%%%%%%%%%%%%%%%%%%%%%%%%%
%%%%%%%%PAGEBREAK%%%%%%%PAGEBREAK%%%%%%%%%
%%%%%%%%%%%%%%%%%%%%%%%%%%%%%%%%%%%%%%%%%%
%%%%%%%%%%%%%%%%PAGEBREAK%%%%%%%%%%%%%%%%%
%%%%%%%%%%%%%%%%%%%%%%%%%%%%%%%%%%%%%%%%%%
%%%%%%%%PAGEBREAK%%%%%%%PAGEBREAK%%%%%%%%%
%%%%%%%%%%%%%%%%%%%%%%%%%%%%%%%%%%%%%%%%%%
%%%%%%%%%%%%%%%%%%%%%%%%%%%%%%%%%%%%%%%%%%
%%%%%%%%%%%%%%%%%%%%%%%%%%%%%%%%%%%%%%%%%%
%%%%%%%%%%%%%%%%%%%%%%%%%%%%%%%%%%%%%%%%%%
%%%%%%%%PAGEBREAK%%%%%%%PAGEBREAK%%%%%%%%%
%%%%%%%%%%%%%%%%%%%%%%%%%%%%%%%%%%%%%%%%%%
%%%%%%%%%%%%%%%%PAGEBREAK%%%%%%%%%%%%%%%%%
%%%%%%%%%%%%%%%%%%%%%%%%%%%%%%%%%%%%%%%%%%
%%%%%%%%PAGEBREAK%%%%%%%PAGEBREAK%%%%%%%%%
%%%%%%%%%%%%%%%%%%%%%%%%%%%%%%%%%%%%%%%%%%
%%%%%%%%%%%%%%%%%%%%%%%%%%%%%%%%%%%%%%%%%%
%%%%%%%%%%%%%%%%%%%%%%%%%%%%%%%%%%%%%%%%%%
%%%%%%%%%%%%%%%%%%%%%%%%%%%%%%%%%%%%%%%%%%
%%%%%%%%PAGEBREAK%%%%%%%PAGEBREAK%%%%%%%%%
%%%%%%%%%%%%%%%%%%%%%%%%%%%%%%%%%%%%%%%%%%
%%%%%%%%%%%%%%%%PAGEBREAK%%%%%%%%%%%%%%%%%
%%%%%%%%%%%%%%%%%%%%%%%%%%%%%%%%%%%%%%%%%%
%%%%%%%%PAGEBREAK%%%%%%%PAGEBREAK%%%%%%%%%
%%%%%%%%%%%%%%%%%%%%%%%%%%%%%%%%%%%%%%%%%%
\begin{ekdosis}
  \begin{tlg}
%-----------------------------
%akārārtho jīvabhūto vakārārtho tha vāsanā/      etad dūyaṃ japaṃ kuryāt sovadhūta udāhṛtaḥ//5// \E
%ākārārtho jīvabhūto vikārārtho tha vāsanā       etad dvayaṃ yaṃ jayati yaḥ sovadhūta udāhṛtaḥ 4 \P
%ākārārtho jīvabhūto vikārādirsthor ya vāsanā//  etadvayaṃ yaḥ jānati sovadhūta udādhṛttā//4// \B
%ākārārtho jīvabhūto vikārādirsthor tha vāsanā// etadvayaṃ yaḥ jānati sovadhūta udādhṛtaḥ//4// \L
%akārārtho jīvabhūto vakārārtho tha vāsanā//     etadvayaṃ  jīyate yaḥ sovadhūta udāhṛtaḥ//4// \D
%akārārtho jīvabhūto vakārārtho yavāsanā         etad vayaṃ jīryate yaḥ sovadhūta udārataḥ 5 \U1
%akārārtho jīvabhūto vakārārtho yavāsanā//       etad vayaṃ jayati yaḥ sovadhūta udāhṛtaḥ// \U2
%\om                                                                 \N1
%\om                                                                 \N2
%-----------------------------
%The purpose of the letter "a" is being life, the purpose of the letter "va" then impressions. He who knows this couple,  he is declared to be an Avadhūta.  
%-----------------------------
\tl{akārārtho jīvabhūto
      \app{\lem[wit={ceteri}]{vakārārtho}
        \rdg[wit={B,L}]{vikārādirsthor}}
      \app{\lem[wit={ceteri}]{'tha}
        \rdg[wit={B,U1,U2}]{ya}} vāsanā/}\\
    \tl{\app{\lem[wit={P},alt={etad dvayaṃ}]{etad\skp{-}dvayaṃ}
        \rdg[wit={E}]{etad dūyaṃ}
        \rdg[wit={ceteri}]{etadvayaṃ}}
      \app{\lem[wit={B,L}]{yaḥ jānati}
        \rdg[wit={E}]{japaṃ kuryāt}
        \rdg[wit={P}]{yaṃ jayati yaḥ}
        \rdg[wit={D}]{jīyate yaḥ}
        \rdg[wit={U1}]{jīryate yaḥ}
        \rdg[wit={U2}]{jayati yaḥ}}
sovadhūta \app{\lem[wit={ceteri}]{udāhṛtaḥ}
  \rdg[wit={B}]{udādhṛttā}
  \rdg[wit={L}]{udādhṛtaḥ}
  \rdg[wit={U1}]{udārataḥ}}\dd{}5\hskip-2pt\dd{}} 
\end{tlg}
\bigskip
\end{ekdosis}
\begin{ekdosis}
  \begin{prose}
    \noindent
%-----------------------------
%yaḥ puruṣo dvitīyaṃ na paśyati   kevalaṃ svasvarūpaṃ paśyati  sovadhūtaḥ/ \E
%yaḥ puruṣo dvitīya  na paśyati   kevalaṃ svasvarūpaṃ paśyatī  sovadhūtaḥ/ \P
%yaḥ puruṣo dvitiyaṃ na paśyaṃtī  kevalaṃ svasvarūpaṃ paśyati  sovadhūtaḥ \B
%yaḥ puruṣo dvitiyaṃ na paśyati   kevalaṃ svasvarūpaṃ paśyati  sovadhūtaḥ// \L
%yaḥ puruṣo dvitiyaṃ na paśyati   kevalaṃ svasvarūpaṃ tiṣṭhati sovadhūtaḥ// \D
%yaḥ puruṣo dvitiyaṃ na paśyati   kevalaṃ svasvarūpaṃ tiṣṭhati sovadhūtaḥ \U1
%yaḥ puruṣo dvitīyaṃ na paśyati// kevalaṃ svasvarūpaṃ paśyati  sovadhūtaḥ// \U2
%\om                                                                 \N1
%\om                                                                 \N2
%-----------------------------
%The person who does not see an enemy, [but] sees the own essiantial nature alone, he is an Avadhūta.  
%-----------------------------
yaḥ puruṣo \app{\lem[wit={ceteri}]{dvitiyaṃ}
  \rdg[wit={P}]{dvitīya}}
na \app{\lem[wit={ceteri}]{paśyati}
  \rdg[wit={U2}]{paśyati ||}
  \rdg[wit={B}]{paśyaṃtī}}
 kevalaṃ svasvarūpaṃ \app{\lem[wit={ceteri}]{paśyati}
   \rdg[wit={D,U1}]{tiṣṭhati}}
 sovadhūtaḥ/
%-----------------------------
%atha vo yasya manaś caṃcalabhāvaṃ   na dadhāti  sovadhūtaḥ kathyate/ \E
%atha vā yasya manaś caṃcalabhāvaṃ   na dadhāti  sovadhūtaḥ kathyate  \P
%atha vā yasya manaś caṃcalaṃ bhāva  na dadhāti/ sovadhūtaḥ/ \B
%atha vā yasya manaś caṃcalaṃ bhāvaṃ na dadhāti  sovadhūtaḥ// \L
%\om                                                                 \N1
%atha cā yasya manaḥ caṃcala bhāvaṃ  na dadhāti/ sovadhūtaḥ kathyate/ \D
%\om                                                                 \N2
%atha cā yasya manaḥ caṃcala bhāve   na dadhāti  sovadhūtaḥ kathyate \U1
%atha vā yasya manaś caṃcalī bhāvaṃ  na dadhāti  sovadhūtaḥ kathyate//  \U2
%-----------------------------
%Or, whose mind does not create the unsteady state, he is said to be an Avadhūta.
%-----------------------------
atha \app{\lem[wit={ceteri}]{vā}
  \rdg[wit={E}]{vo}
  \rdg[wit={D,U1}]{cā}}
yasya
\app{\lem[wit={ceteri},alt={manaś}]{mana\skp{ś-ca}}
  \rdg[wit={D,U1}]{manaḥ}
}\app{\lem[wit={ceteri},alt={cañcala°}]{\skm{ś-ca}ñcala}
  \rdg[wit={B,L}]{caṃcalaṃ}
  \rdg[wit={U2}]{caṃcalī}}\app{\lem[wit={ceteri}]{bhāvaṃ}
  \rdg[wit={B}]{bhāva}
  \rdg[wit={U1}]{bhāve}}
na \app{\lem[wit={ceteri}]{dadhāti}
  \rdg[wit={B,D}]{dhadhāti |}}
\app{\lem[wit={ceteri}]{sovadhūtaḥ}
  \rdg[wit={B,L}]{sovadhūtaḥ |}}
\app{\lem[wit={ceteri}]{kathyate}
  \rdg[wit={B,L}]{\om}}/ 
\end{prose}
\begin{prose}
%-----------------------------
%yan na                 dṛśyate tad avyaktam ity ucyate/ \E
%yan na                 dṛśyate tad avyaktam ity ucyate \P
%atha vā kasyase panna iśyate  d    avyaktam ity ucyate/ \B
%atha vā kasyase panna dṛśyate d    avyaktam ity ucyate// \L
%yanma                 dṛśyate tad  avyaktam ity ucyate/ \D
%yan na                dṛśyate tad  avyaktam ity ucyate \U1
%                              tad  avyaktam ity ucyate \U2
% \om                                                                \N1
%\om                                                                 \N2
%-----------------------------
%What is not seen, it is said, that is the unmanifest .  
%-----------------------------
\app{\lem[wit={E,P,U1},alt={yan na}]{yan\skp{-}na}
  \rdg[wit={D}]{yanma}
  \rdg[wit={B,L}]{atha vā kasyase panna}
  \rdg[wit={U2}]{\om}}
\app{\lem[wit={ceteri}]{dṛśyate}
  \rdg[wit={B}]{iśyate}
  \rdg[wit={U2}]{\om}}
\app{\lem[wit={ceteri},alt={tad}]{ta\skp{d-a}}
  \rdg[wit={B,L}]{°d}
}\skm{d-a}vyaktam-ity-ucyate/
\end{prose}
\begin{prose}
%-----------------------------
%tad avyaktaṃ pratyakṣeṇa paśyati/ \E
%tad avyaktaṃ pratyakṣeṇa paśyati  \P
%tad avyaktaṃ pratyakṣeṇa yasyati / \B
%tad avyaktaṃ pratyakṣeṇa yasyati ... \L
%\om                                                                 \N1
%tad avyaktapratyakṣeṇa paśyati \D
%\om                                                                 \N2
%tad avyaktapratyakṣeṇa paśyatī \U1
%tad avyaktaṃ pratyakṣeṇa paśyati//  \U2
%-----------------------------
%He sees that unmanifest by means of direct perception, 
%-----------------------------
\app{\lem[wit={ceteri},alt={tad avyaktaṃ}]{tad\skp{-}avyaktaṃ}
  \rdg[wit={D,U1}]{tad avyakta°}}
pratyakṣeṇa \app{\lem[wit={ceteri}]{paśyati}
  \rdg[wit={B,L}]{yasyati}
  \rdg[wit={U1}]{paśyatī}}/
\end{prose}
\begin{prose}
  \noindent
%-----------------------------
%yatkiṃcid  ṛśyate tatsarvaṃ   grastāti muktam iti jñānaṃ paśyati/   \E
%yatkiṃcid dṛśyate tatatsarvaṃ grasati  muktam iti jñāyate           \P
%yatkiṃcid  ṛśyate tatsarvaṃ   gasati   muktam iti jñāyate           \B
%yatkiṃcid dṛśyate tatsarva    gasati   muktam iti jñāyate...        \L
%\om                                                                 \N1
%yatkiṃcit paśyati tatsarvaṃ   grasatī  muktam iti jñāyate           \D
%\om                                                                 \N2
%yatkiṃcit paśyati tatsarvaṃ   grasatī  muktam iti jñāyate           \U1
%yatkiṃcit dṛśyate tatsarvaṃ   grasaṃti muktim iti jñāyate//         \U2
%-----------------------------
%Whatever he sees, all that he completely encompasses. This is known to be liberation.  
%-----------------------------
\app{\lem[wit={D,U1,U2},alt={yatkiṃcit}]{yatkiṃci\skp{t-pa}}
  \rdg[wit={B,E,L,P}]{yatkiṃcid}
}\app{\lem[wit={D,U1}]{\skm{t-pa}paśyati}
  \rdg[wit={P,L,U2}]{dṛśyate}
  \rdg[wit={E,B}]{ṛśyate}}
\app{\lem[wit={ceteri}]{tatsarvaṃ}
  \rdg[wit={P}]{tatatsarvaṃ}
  \rdg[wit={L}]{tatsarva}}
\app{\lem[wit={P}]{grasati}
  \rdg[wit={D,U1}]{grasatī}
  \rdg[wit={U2}]{grasaṃti}
  \rdg[wit={E}]{grastāti}}
\app{\lem[wit={ceteri},alt={muktam}]{mukta\skp{m-i}}
  \rdg[wit={U2}]{muktim}}\skm{m-i}ti
\app{\lem[wit={ceteri}]{jñāyate}
  \rdg[wit={U2}]{jñāyate ||}
  \rdg[wit={E}]{jñānaṃ paśyati |}}
%-----------------------------
%sovadhūtaḥ kathyate/ \E [p.59]
%sovadhūtaḥ kathyate  \P  %%%7665.jpg
%sāvadhūtaḥ kathyate  \B
%sovadhūtaḥ kathyate  \L
%\om                                                                 \N1
%sovadhūtaḥ kathyate/  \D
%\om                                                                 \N2
%sovadhūtaḥ kathyate  \U1
%sovadhūtaḥ kathyaṃte//  \U2
%-----------------------------
%He is said the be an Avadhūta. 
%-----------------------------
\app{\lem[wit={ceteri}]{sovadhūtaḥ}
  \rdg[wit={P}]{sāvadhūtaḥ}}
\app{\lem[wit={ceteri}]{kathyate}
  \rdg[wit={U2}]{kathyaṃte}}/
\end{prose}
\end{ekdosis}
\begin{ekdosis}
  \begin{tlg}
%-----------------------------
%avadhūta tanuḥ somo nirākārapade sthitaḥ/  sarveṣāṃ darśanānāṃ ca svasvarūpaṃ prakāśyate// 1// \E [p.59]    %%%%%%%SSP 6.32 
%avadhūta tanu  somo nirākārapade sthiraḥ   sarveṣāṃ darśanānāṃ ca svasvarūpaṃ prakāśate 1  \P
%avadhūta tanuḥ somo nirākārapade sthita/   sarveṣāṃ darśanānāṃ ca svasvarūpaṃ prakāśate/ \B
%avadhūta tanu  somā nirākārapade sthitaḥ/  sarveṣāṃ darśanānāṃ ca svasvarūpaṃ prakāśate/ \L
%avadhūta tanu  somo nirākārapade sthitaḥ// sarveṣāṃ darśanānāṃ ca svasvarūpaṃ prakāśyate//1// \D
%āvadhūta tanuḥ somo nirākārapare sthita    sarveṣāṃ darśanānāṃ ca svasvarūpaṃ prakāśyate \U1
%avadhūta rutu? somo nirākārapade sthitaḥ// sarveṣāṃ darpaṇānāṃ ca svasvarūpaṃ prakāśyate// \U2 %%%422.jpg
%\om                                                                 \N1
%\om                                                                 \N2
%-----------------------------
%The sacrificer who is manifested as an Avadhūta, who is situated in the objectless state, he percieves all views in his own essential nature. 
%-----------------------------
\note[type=source, labelb=316, lem={avadhūtatanuḥ}]{SSP 6.32: avadhūtatanur yogī nirākārapade sthitaḥ | sarveṣāṃ darśanānāṃ ca svasvarūpaṃ prakāśate ||32||}
\tl{āvadhūta\app{\lem[wit={B,E,U1}]{tanuḥ}
  \rdg[wit={P,L,D}]{tanu}
  \rdg[wit={U2}]{rutu}}
\app{\lem[wit={L}]{somā}
  \rdg[wit={ceteri}]{somo}}
nirākārapade
\app{\lem[wit={ceteri}]{sthitaḥ}
  \rdg[wit={U1}]{sthita}}/}\\ 
\tl{sarveṣāṃ \app{\lem[wit={ceteri}]{darśanānāṃ}
    \rdg[wit={U2}]{darpaṇānāṃ}} ca svasvarūpaṃ 
\app{\lem[wit={B,L,P}]{prakāśate}
  \rdg[wit={ceteri}]{prakāśyate}}\dd{}6\hskip-2pt\dd{}}
\end{tlg}
\end{ekdosis}
%-----------------------------
%satyam ekam ajaṃ nityam anaṃtam   akṣayaṃ dhruvam/  jñātvā hy evaṃ   vaded   dhīmān satyavādī sa kathyate// 2// \E %%SSP 6.60
%satyam ekam ajaṃ nityam anaṃtam   akṣayaṃ dhruvaṃ   jñātvā hy evaṃ   vaded   dhīmān satyavādī sa kathyate 2 \P
%satyam ekam ajaṃ nityam anaṃtam   akṣayaṃ dhruvam/  jñātvā hy evaṃ   vaded   dhīmān satyavādī sa kathyate/ \B
%satyam ekam ajaṃ nityam anaṃtam   akṣayaṃ dhruvaṃ/  jñātvāt y evaṃ   vaded   dhīmān           sa kathyate/ \L
%satyam ekām ja   nityaṃ manaṃ tam akṣayaṃ dhruvaṃ/  jñātvā hy asta   vaded    dhīmān satyavādī sa kathyate//2// \D
%satyam ekām ajaṃ nityaṃ manaṃ tam akṣayaṃ dhruvaṃ   jñātvā hy astaṃ  vaded dhīmān    satyavādī sa kathyate 2 \U1
%satyam ekam ajaṃ nityaṃ manaṃ tam akṣayaṃ dhruvaṃ// jñātvā hy evaṃ   vadet   dhīmān  satyavādī    kathyate \U2
%\om \N1
%\om \N2
%-----------------------------
%One truth, unborn, eternal, infinite, imperishable [and] changeless, having realized it the wise shall proclaim it as such. He is said to be a speaker of truth. 
%-----------------------------
\begin{ekdosis}
  \begin{tlg}
    \note[type=source, labelb=317, lem={satyam}]{SSP 6.60: satyam ekam ajaṃ nityamanantaṃ cākṣayaṃ dhruvam | jñātvā yastu vaded dhīraḥ satyavādī sa kathyate ||60||}
\tl{satya\skp{m-e}\app{\lem[wit={ceteri},alt={ekam}]{\skm{m-e}ka\skp{m-a}}
  \rdg[wit={D,U1}]{ekām}
}\app{\lem[wit={ceteri},alt={ajaṃ}]{\skm{m-a}jaṃ}
  \rdg[wit={D}]{ja}
}\app{\lem[wit={B,E,L,P},alt={nityam}]{nitya\skp{m-a}}
  \rdg[wit={ceteri}]{nityaṃ}
}\app{\lem[wit={B,E,L,P},alt={anantam}]{\skm{m-a}nanta\skp{m-a}}
  \rdg[wit={D,U1,U2}]{manaṃ tam}
}\skm{m-a}kṣayaṃ dhruvaṃ/}\\
\tl{\app{\lem[wit={ceteri}]{jñātvā}
    \rdg[wit={L,D}]{jñātvāt}}
  \app{\lem[wit={ceteri},alt={hy evaṃ}]{hy\skp{-}evaṃ}
    \rdg[wit={D}]{hy ....}
    \rdg[wit={U1}]{hy astaṃ}}
  \app{\lem[wit={ceteri},alt={vaded}]{vade\skp{d-dhi}}
    \rdg[wit={U2}]{vadet}
}\skm{d-dhi}mān \app{\lem[wit={ceteri}]{satyavādī}
  \rdg[wit={L}]{\om}}
\app{\lem[wit={ceteri}]{sa}
  \rdg[wit={U2}]{\om}} kathyate\dd{}7\hskip-2pt\dd{}}
\bigskip
\end{tlg}
\end{ekdosis}
\begin{ekdosis}
  \begin{prose}
    \noindent
%-----------------------------
%yatkiṃcin      na     paśyati, sa    eko  hy evaṃ manaso vijānāti     nāśā na tādṛśaṃ padārthaṃ jñātvā kāle ceṣṭā bhavati/ sa satyavādī kathyate//   \E [p.60]
%yatkiṃcid    yena     paśyati  sa    ekaḥ   tasya manaso na jānāti na nāśo na tādṛśaṃ padārtha  jñātvā kāle ceṣṭā bhavati  sa satyavādī kathyate \P
%yatkiṃ       kena     paśyaṃti sa    ekaḥ/  tasya manaso jānātir   na nāśo na tādṛśaṃ padārthaṃ jñā    kāle ceṣṭā bhavati/ sa satyavādi kathyate//   \B
%yatkiṃ       kena     paśyaṃti sa    ekaḥ/  tasya manaso jānāti    na nāśo na tādṛśaṃ padārthaṃ jñā    kāle ceṣṭā bhavati/ sa satyavādi kathyate//     \L
%yatkiṃcid aikyena     paśyati  sa sa ekaḥ// tasya mano   jātitā    nāśo na    tādṛśāṃ padārthaṃ jñātvā kāla ceṣṭā bhavati/ sa satyavādī kathyate/ \D                                                              \N2
%yatkiṃcid aikena      paśyatī  sa    ekaḥ   tasya mano   jnānaṃti   tādṛśot   tādṛśāṃ padārthaṃ jñātvā kāla ceṣṭā bhavati  sa satyavādī kathyate \U1 %%%293.jpg
%\om                                                                                                                             \U2
%\om                                                                 \N1
%\om                                                                 \N2
%----------------------------- 
%Whatever he sees with unity, it is one. His mind knows. Having realized that there is neither destruction nor such a thing corresponding to that word, in this moment in time manner of life arises. He is said to be a speaker of truth.  
%-----------------------------
\note[type=philcomm, labelb=318, lem={yatkiṃcid \ldots satyavādī kathyate}]{This passage is \om in U\textsubscript{2}.}
\app{\lem[wit={D,P,U1},alt={yatkiṃcid}]{yatkiṃci\skp{d-ai}}
  \rdg[wit={E}]{yatkiṃcin}
  \rdg[wit={B,L}]{yatkiṃ}
}\app{\lem[wit={D},alt={aikyena}]{\skm{d-ai}kyena}
  \rdg[wit={U1}]{aikena}
  \rdg[wit={B,L}]{kena}
  \rdg[wit={P}]{yena}
  \rdg[wit={E}]{na}}
\app{\lem[wit={D,E,P}]{paśyati}
  \rdg[wit={U1}]{paśyatī}
  \rdg[wit={B,L}]{paśyaṃti}}
\app{\lem[wit={D}]{sa sa}
  \rdg[wit={ceteri}]{sa}}
\app{\lem[wit={ceteri}]{ekaḥ}
  \rdg[wit={E}]{eko}}/
\app{\lem[wit={ceteri}]{tasya}
  \rdg[wit={E}]{hy evaṃ}}
\app{\lem[wit={B,E,L,P}]{manaso}
  \rdg[wit={D,U1}]{mano}}
\app{\lem[wit={L}]{jānāti}
  \rdg[wit={E}]{vijānāti}
  \rdg[wit={P}]{na jānāti}
  \rdg[wit={B}]{jānātir}
  \rdg[wit={D}]{jātitā}
  \rdg[wit={U1}]{jnānaṃti}}
\app{\lem[wit={B,L,P}]{na nāśo na}
  \rdg[wit={D}]{nāśo na}
    \rdg[wit={E}]{nāśā na}
    \rdg[wit={U1}]{tādṛśot}}
     tādṛśāṃ \app{\lem[wit={ceteri}]{padārthaṃ}
       \rdg[wit={P}]{padārtha}}
     \app{\lem[wit={ceteri}]{jñātvā}
       \rdg[wit={B,L}]{jñā}}
     \app{\lem[wit={ceteri}]{kāle}
       \rdg[wit={D,U1}]{kāla}}
       ceṣṭā bhavati/ sa satyavādī kathyate/
     \end{prose}
   \end{ekdosis}
   \begin{ekdosis}
     \begin{tlg}
%-----------------------------
%vāsvare bhāsvare śaktiḥ   saṃkoco bhāsvare pi ca/   tayoḥ saṃyogakarttā   yaḥ           sa bhavet satyayogabhāk//3//     \E  %%SSP 6.64 %%%%%SELTSAMER SATZ! DISKUTIEREN! 
%vāsare  bhāsvare śaktiḥ   saṃkoco bhāsvare pi ca    tayoḥ saṃyogakarttā   yaḥ           sa bhavet satyayogabhāk 3        \P
%vāsvre  bhāsvare                           pi ca//        sayogaḥ  kartavyaḥ               bhavat satyayogabhāk//        \B
%vāsare  bhāskare                           pi ca//        saṃyogaḥ karttā yaḥ           sa bhavet satyayogabhāk//        \L
%vasare  bhāsvare śaktiḥ/  saṃkoco bhāsvare pi ca/   tayoḥ saṃyogakarttā   yaḥ//         sa bhavet satyayogabhāk//3//     \D
%vasare  bhāskare śaktiḥ   saṃkoco bhāskare pi ca          saṃyogakartā    yaḥ    saṃvit svabhāvāt satyayogabhāk          \U1
%vāsare  bhāsvare śaktiḥ// saṃkoco bhāsvare pi ca//  tayoḥ saṃyogakarttā   yaḥ           sa bhavet satyayogabhāk//        \U2
%\om                                                                                                                      \N1
%\om                                                                                                                      \N2
%-----------------------------
%Bei Tage ist Śakti in der Sonne und die Kontraktion ist ebenfalls in der Sonne. Wer die beiden miteinander vereint, der ist ein Proponent des Satyayoga.
%During the day Śakti is in the sun and the contraction is also in the sun. Whoever combines the two is a proponent of Satyayoga.
%-----------------------------
\tl{\note[type=source, labelb=319, lem={vāsare}]{SSP 6.64: prasaraṃ bhāsate śaktiḥ saṃkocaṃ bhāsate śivaḥ | tayor yogasya kartā yaḥ sa bhavet siddhayogiraṭ ||64||}
\app{\lem[wit={P,L,U2}]{vāsare}
  \rdg[wit={E}]{vāsvare}
  \rdg[wit={B}]{vāsvre}
  \rdg[wit={D,U1}]{vasare}}
\app{\lem[wit={ceteri}]{bhāsvare}
  \rdg[wit={L,U1}]{bhāskare}}
\app{\lem[wit={ceteri}]{śaktiḥ}
  \rdg[wit={D,U2}]{śaktiḥ |}
    \rdg[wit={B,L}]{\om}} 
\app{\lem[wit={ceteri}]{saṃkoco}
  \rdg[wit={B,L}]{\om}}
\app{\lem[wit={ceteri}]{bhāsvare}
  \rdg[wit={U1}]{bhāskare}
  \rdg[wit={B,L}]{\om}}
'pi ca/}\\
\tl{\app{\lem[wit={ceteri}]{tayoḥ}
  \rdg[wit={B,L,U1}]{\om}} 
\app{\lem[wit={ceteri}]{saṃyogakartā yaḥ}
  \rdg[wit={B}]{sayogaḥ kartavyaḥ}
  \rdg[wit={L}]{saṃyogaḥ karttā yaḥ}}
\app{\lem[wit={ceteri}]{sa bhavet}
  \rdg[wit={B}]{bhavat}
  \rdg[wit={U1}]{saṃvit svabhāvāt}}
satyayogabhāk\dd{}8\hskip-2pt\dd{}}
\end{tlg}
\end{ekdosis}
\ekdpb*{}
%%%%%%%%%%%%%%%%%%%%%%%%%%%%%%%%%%%%%%%%%%
%%%%%%%%PAGEBREAK%%%%%%%PAGEBREAK%%%%%%%%%
%%%%%%%%%%%%%%%%%%%%%%%%%%%%%%%%%%%%%%%%%%
%%%%%%%%%%%%%%%%PAGEBREAK%%%%%%%%%%%%%%%%%
%%%%%%%%%%%%%%%%%%%%%%%%%%%%%%%%%%%%%%%%%%
%%%%%%%%PAGEBREAK%%%%%%%PAGEBREAK%%%%%%%%%
%%%%%%%%%%%%%%%%%%%%%%%%%%%%%%%%%%%%%%%%%%
%%%%%%%%%%%%%%%%%%%%%%%%%%%%%%%%%%%%%%%%%%
%%%%%%%%%%%%%%%%%%%%%%%%%%%%%%%%%%%%%%%%%%
%%%%%%%%%%%%%%%%%%%%%%%%%%%%%%%%%%%%%%%%%%
%%%%%%%%PAGEBREAK%%%%%%%PAGEBREAK%%%%%%%%%
%%%%%%%%%%%%%%%%%%%%%%%%%%%%%%%%%%%%%%%%%%
%%%%%%%%%%%%%%%%PAGEBREAK%%%%%%%%%%%%%%%%%
%%%%%%%%%%%%%%%%%%%%%%%%%%%%%%%%%%%%%%%%%%
%%%%%%%%PAGEBREAK%%%%%%%PAGEBREAK%%%%%%%%%
%%%%%%%%%%%%%%%%%%%%%%%%%%%%%%%%%%%%%%%%%%
%%%%%%%%%%%%%%%%%%%%%%%%%%%%%%%%%%%%%%%%%%
%%%%%%%%%%%%%%%%%%%%%%%%%%%%%%%%%%%%%%%%%%
%%%%%%%%%%%%%%%%%%%%%%%%%%%%%%%%%%%%%%%%%%
%%%%%%%%PAGEBREAK%%%%%%%PAGEBREAK%%%%%%%%%
%%%%%%%%%%%%%%%%%%%%%%%%%%%%%%%%%%%%%%%%%%
%%%%%%%%%%%%%%%%PAGEBREAK%%%%%%%%%%%%%%%%%
%%%%%%%%%%%%%%%%%%%%%%%%%%%%%%%%%%%%%%%%%%
%%%%%%%%PAGEBREAK%%%%%%%PAGEBREAK%%%%%%%%%
%%%%%%%%%%%%%%%%%%%%%%%%%%%%%%%%%%%%%%%%%%
\begin{ekdosis}
  \begin{tlg}
%-----------------------------
%viśvānīta tayā         viśvam ekam eva virājate/  saṃyogo na sadā yasya siddhayogī sa gadyate//4//  \E   %%SSP 6.65 
%viśvānīta tayā         viśvam ekam eva virājate/  saṃyogo na sadā yasya siddhayogī sa kathyate 4    \P
%visvātitā tayā         viśvam ekam eva virājate/  saṃyogo na sadā yasya siddhayogī sa gadyate/      \B
%visvātitā tayā         viśvam ekam eva virājate/  saṃyogo na sadā yasya siddhayogī sa gadyate/      \L 0030.jpg
%viśvātīta ttayā        viśvam ekam eva virājate/  saṃyogena  sadā yasya siddhayogī sa gadyate//4//  \D
%viśvāso viśvātita tayā visvaṃ ekam eva virājate   saṃyogo na sadā yasya siddhayogī sa kathyate      \U1
%viśvātita tayā         viśvam ekam eva virājate// saṃyogo na sadā yasya siddhayogī sa gadyate//     \U2
%\om                                                                                                 \N1
%\om                                                                                                 \N2
%-----------------------------
%He is called a Siddhayogī for whom always by means of Yoga the universe as such shines forth as one by means of transcending the universe.
%-----------------------------
\tl{\note[type=source, labelb=320, lem={viśvātītaṃ}]{SSP 6.65: viśvātītaṃ yathā viśvam ekam eva virājate | saṃyogena sadā yas tu siddhayogī bhavet tu saḥ ||65|}
\app{\lem[type=emendation, resp=egoscr]{viśvātītaṃ}
  \rdg[wit={D,U2}]{\korr viśvātīta}
  \rdg[wit={B,L}]{visvātitā}
  \rdg[wit={E,P}]{viśvānīta}
  \rdg[wit={U1}]{viśvāso viśvātita}}
tayā viśvam-ekam-eva virājate/}\\
\tl{\app{\lem[wit={D}]{saṃyogena}
  \rdg[wit={ceteri}]{saṃyogo na}}
sadā yasya siddhayogī sa
\app{\lem[wit={ceteri}]{gadyate}
  \rdg[wit={P,U1}]{kathyate}}\dd{}9\hskip-2pt\dd{}}
\end{tlg}
\end{ekdosis}
\begin{ekdosis}
  \begin{tlg}
%-----------------------------
%sarvāsāṃ nijavṛttīnāṃ vismṛtīr bhajate ttu yaḥ/ sa bhavet siddhasiddhānto siddhayogī sa gadyate//5// \E [p.61] %This quote stems from the Siddhasiddhāntapaddhati 6.66
%sarvāsāṃ nijavṛtīnāṃ  vismṛtī  bhajate tu yaḥ   sa bhavet siddhasiddhāṃte siddhayogī sa gadyate 5    \P
%sarvāsāṃ bījavṛtīnāṃ  vismṛtī  bhajate tu yaḥ   sa bhavet siddhasiddhānte siddhayogī sa gadyate/     \B
%sarvāsāṃ bījavṛtīnāṃ  vismṛtīṃ bhajate tu yaḥ// sa bhavet siddhasiddhānte siddhayogī sa gadyate//    \L
%sarvāsāṃ \om                                                                                         \D
%sarvāsāṃ nijavṛtīnāṃ  vismṛtiṃ bhajate tu yaḥ   sa bhavet siddhasiddhāṃte siddhayogī sa gadyate      \U1
%sarvāsāṃ nijavṛttīnāṃ vismṛtiṃ bhajate tu yaḥ// sa bhavet siddhasiddhāṃte siddhayogī sa gadyate//    \U2
%\om                                                                                                  \N1
%\om                                                                                                  \N2
%-----------------------------
%He who assumes the state of oblivion of all the own inherent fluctiations [of the mind] he is called a Siddhayogin according to the doctrine of the Siddhas.  
%-----------------------------
\tl{\note[type=source, labelb=321, lem={sarvāsāṃ nijavṛtīnāṃ}]{SSP 6.66: sarvāsāṃ nijavṛttīnāṃ prasṛtir bhajate layam | sa bhavet siddhasiddhānte siddhayogī mahābalaḥ ||66||}
  sarvāsāṃ \note[type=philcomm, labelb=321, lem={nijavṛttīnāṃ \ldots gadyate}]{Starting after the first word of this verse there is a lenghty gap in D. Omissions will not be recorded. The reader will be notified once the evidence from D resumes.}
  \app{\lem[wit={ceteri}]{nijavṛttīnāṃ}
    \rdg[wit={B,L}]{bījavṛtīnāṃ}}
  \app{\lem[wit={U1,U2}]{vismṛtiṃ}
    \rdg[wit={L}]{vismṛtīṃ}
    \rdg[wit={B,P}]{vismṛtī}
    \rdg[wit={E}]{vismṛtīr}}
  bhajate tu yaḥ}\\
\tl{sa bhavet-siddha\app{\lem[wit={ceteri}]{siddhānte}
    \rdg[wit={E}]{siddhasiddhānto}}
siddhayogī sa gadyate\dd{}10\hskip-2pt\dd{}}
\end{tlg}
\end{ekdosis}
\begin{ekdosis}
  \begin{tlg}
%-----------------------------
%udāsīnaḥ sadā śānto brahmānandamayo pi ca/ yo bhavet siddhayogena siddhayogī sa kathyate//6// \E
%udāsīnaḥ sadā śānto brahmānandamayo pi ca/ yo bhavet siddhayogena siddhayogī sa kathyate 6 \P  %%%7666.jpg
%\om in \L 
%udāsīnaḥ sadā śānto mahānaṃdamayo pi ca/   yo bhavet siddhayogena siddhayogī sa kathyate// \B DSCN7173.JPG Z.1
%\om                                                                 \N1
%\om                                                                 \D
%\om                                                                 \N2
%udāsīna  sadā śānto mahānaṃdamayo pi ca    yo bhavet siddhayogena siddhayogī sa kathyate \U1
%udāsīnaḥ sadā śāṃto mahānaṃdamayā pi ca//  yo bhavet siddhayogena siddhayogī sa kathyate// \U2
%-----------------------------
%Wer durch die Praxis des Siddhayoga immer unbeteiligt, friedlich ist und einer ist, der aus großer Glückseeligkeit besteht der, so heißt es, ist ein Siddhayogin.
%One who is always indifferent, peaceful and one immersed in great bliss by means of Siddhayoga is said to be a Siddhayogin.
%-----------------------------
\tl{\note[type=source, labelb=322, lem={udāsīnaḥ}]{SSP 6.67: udāsīnaḥ sadā śāntaḥ svastho 'ntarnijabhāsakaḥ | mahānandamayo dhīraḥ sa bhavet siddhayogirāṭ ||67||}
\note[type=philcomm, labelb=323, lem={udāsīnaḥ \ldots kathyate}]{This verse is \om in L.}
\app{\lem[wit={ceteri}]{udāsīnaḥ}
  \rdg[wit={U1}]{udāsīna}}
sadā śānto
\app{\lem[wit={B,U1}]{mahānaṃdamayo}
  \rdg[wit={U2}]{mahānaṃdamayā}
  \rdg[wit={E,P}]{brahmānandamayo}}
'pi ca/}\\
\tl{yo bhavet siddhayogena siddhayogī sa kathyate\dd{}11\hskip-2pt\dd{}}
\end{tlg}
\end{ekdosis}
\begin{ekdosis}
    \bigskip
       \centerline{\textrm{\small{[Three Lotuses]}}}
       \bigskip
  \begin{tlg}
%-----------------------------
%adhunā kamalānāṃ tu śrṛṇu saṃketam adbhutam/    anekākārabhedotthaṃ kaṃ   svarūpātmakaṃ malam/     kamalaṃ tena vikhyātaṃ trividhaṃ tatra dehagam// 7// \E
%adhunā kamalānāṃ tu nuṣṛe saṃketam adbhutaṃ     anekākārabhedocchaṃ kaṃ   svarūpātmakaṃ malaṃ 7    kamalaṃ tena vikhyātaṃ vividhaṃ  tatra dehagaṃ       \P
%adhunā kamalānāṃ tu śṛṇu  saṃketam adbhutaṃ/    anekakārabhedochaṃ  kiṃ   svarūpātmakaṃ malaṃ//7// kamalaṃ tena vikhyātaṃ trividhaṃ tatra dehagam//     \B
%adhunā kamalānāṃ tu śṛṇu  saṃketam adbhutaṃ/    anekakārabhedātthaṃ kiṃ   svarūpātmakaṃ malaṃ//7// kamalaṃ tena vikhyātaṃ trividhaṃ tatra dehagaṃ//     \L
%adhunā kamalānāṃ tu śṛṇu  saṃketam adbhutaṃ     anekakārabhedotthaṃ       svasvarūpātmakaṃ malaṃ   kamalaṃ tena vikhyātaṃ trividhaṃ tena  dehagaṃ       \U1
%adhunā kamalānāṃ tu śṛṇu  saṃketam adbhutaṃ//   anekākārabhedotthaṃ kaḥ// svarūpātmakaṃ paraṃ//    kamalaṃ tena vikhyātaṃ trividhaṃ tatra dahagaṃ//     \U2
%\om                                                                 \N1
%\om                                                                 \D
%\om                                                                 \N2
%-----------------------------
%Now, carefully listen to the mysterious secret terminology of the buds of the lotus flower. Arising from the divisions of the manifold forms, the nature of the own true form is spotless (em zu nirmalam). Because of this the lotus flower is generally known as the threefold body of reality (em zu tattvadehakam) .  
%-----------------------------
\tl{\note[type=source, labelb=323, lem={adhunā kamalānāṃ}]{Ysv\textsuperscript{PT}: adhunā kamalānān tu śṛṇu saṅketam adbhutam | anekākārabhedotthaṃ kaṃ svarūpan tu nirmalam | kamalaṃ tena vikhyātaṃ trividhaṃ tattvadehakam |}
  adhunā kamalānāṃ tu
  \app{\lem[wit={ceteri}]{śṛṇu}
    \rdg[wit={P}]{nuṣṛe}} 
saṃketa\skp{m-a}\app{\lem[wit={ceteri},alt={adbhutaṃ}]{\skm{m-a}dbhutaṃ}
  \rdg[wit={E}]{adbhutam}}/}\\
\tl{\app{\lem[wit={E,U1}]{anekākārabhedotthaṃ}
  \rdg[wit={B,P}]{anekākārabhedocchaṃ}
  \rdg[wit={L}]{anekakārabhedātthaṃ}}
\app{\lem[wit={ceteri}]{kaṃ}
  \rdg[wit={B,L}]{kiṃ}
  \rdg[wit={U1}]{\om}}
\app{\lem[type=emendation, resp=egoscr]{svarūpan tu nirmalam}
  \rdg[wit={B,E,L,P}]{\korr svarūpātmakaṃ malam}
  \rdg[wit={U1}]{svasvarūpātmakaṃ malaṃ}
  \rdg[wit={U2}]{svarūpātmakaṃ paraṃ}}/}\\
\note[type=philcomm, labelb=324, lem={svarūpan tu nirmalam}]{Since the version of the fourth and sixth \textit{pāda} preserved in the witnesses of the \textit{Yogattavabindu} is not convincing content-wise, I decided to emend according to the source text.}
\tl{kamalaṃ tena vikhyātaṃ \app{\lem[wit={ceteri}]{trividhaṃ}
    \rdg[wit={P}]{vividhaṃ}}
  \app{\lem[type=emendation, resp=egoscr]{tattvadehakam}
    \rdg[wit={B,E,L,U2}]{\korr tatra dehagaṃ}
    \rdg[wit={U1}]{tena dehagaṃ}}\dd{}12\hskip-2pt\dd{}}
\end{tlg}
\end{ekdosis}
\bigskip
\begin{ekdosis}
  \begin{prose}
%-----------------------------
%                                  ādhārakamalam   asya kamalam iti    kaṃ kasmāt/  kamātmā             tasmāt kamalam iti saṃjñā         \E [p.62]                (em kam to kamalam?) 
%athādhaḥ kamalaṃ kathyate         ādhārakamalaṃ   asya kamalam iti saṃjñā kasmāt   kamātmasvarūpaṃ     sa ātmanaṃ  anekarūpaṃ            paśyati    \P
%athādhakamalaṃ   kathyate/        ārakamalaṃ      asya kamalam iti saṃjñā kasmāt--------masvarūpaṃ     sa ātmanaṃ  anarūpaṃ              paśyati//  \B
%athādhakamalaṃ   kathyate//       ādhārakamalaṃ   asya kamalam iti saṃjñā kasmāt   kāmātmasvarūpaṃ     sa ātmanaṃ  anarūpaṃ              paśyati//  \L
%athādhaḥ kamalaṃ kathyate         ādhārakamalaṃ   asya kamalam iti saṃjñā kasmāt   kaḥ ātmā            sa ātmanaṃ  anekarūpaṃ svarūpaṃ   paśyate    \U1 (em zu ātmānam) 
%athādhaḥ kamalaṃ kathyate//       ādhārakamalaṃ// asya kamalam iti saṃjñā kasmāt// ekam ātmasvarūpaṃ// sa ātmanaṃ  anekarūpaṃ            paśyati//  \U2
%\om                                                                 \N1
%\om                                                                 \D
%\om                                                                 \N2
%-----------------------------
%Now the lower Kamala is taught: the container-lotus. Why the technical term "Kamala"? Kamala is the own form of the self. One sees the self in various forms. 
%-----------------------------
\app{\lem[wit={P,U1,U2}]{athādhaḥ}
  \rdg[wit={B,L}]{athādha°}
  \rdg[wit={E}]{\om}}
\app{\lem[wit={ceteri}]{kamalaṃ}
  \rdg[wit={E}]{\om}}
\app{\lem[wit={ceteri}]{kathyate}
  \rdg[wit={E}]{\om}}/
\app{\lem[wit={ceteri}]{ādhārakamalaṃ}
  \rdg[wit={B}]{ārakamalaṃ}}\dd{}
asya kamalam-iti \app{\lem[wit={ceteri}]{saṃjñā}
  \rdg[wit={E}]{kaṃ}}
kasmāt/
\app{\lem[type=emendation, resp=egoscr,alt={kamalam ātmasvarūpaṃ}]{kamalam\skp{-}ātmasvarūpaṃ}
  \rdg[wit={E}]{\korr kamātmā tasmāt kamalam iti saṃjñā }
  \rdg[wit={P}]{kamātmasvarūpaṃ}
  \rdg[wit={B}]{masvarūpaṃ}
  \rdg[wit={L}]{kāmātmasvarūpaṃ}
  \rdg[wit={U1}]{kaḥ ātmā}
  \rdg[wit={U2}]{ekam ātmasvarūpaṃ ||}}
\app{\lem[wit={ceteri}]{sa ātmanaṃ}
  \rdg[wit={E}]{\om}} 
\app{\lem[wit={P,U2}]{anekarūpaṃ}
  \rdg[wit={U1}]{anekarūpaṃ svarūpaṃ}
  \rdg[wit={B,L}]{anarūpaṃ}
  \rdg[wit={E}]{\om}}
\app{\lem[wit={ceteri}]{paśyati}
  \rdg[wit={U1}]{paśyate}
  \rdg[wit={E}]{\om}}/
\end{prose}
\end{ekdosis}
\ekdpb*{}
%%%%%%%%%%%%%%%%%%%%%%%%%%%%%%%%%%%%%%%%%%
%%%%%%%%PAGEBREAK%%%%%%%PAGEBREAK%%%%%%%%%
%%%%%%%%%%%%%%%%%%%%%%%%%%%%%%%%%%%%%%%%%%
%%%%%%%%%%%%%%%%PAGEBREAK%%%%%%%%%%%%%%%%%
%%%%%%%%%%%%%%%%%%%%%%%%%%%%%%%%%%%%%%%%%%
%%%%%%%%PAGEBREAK%%%%%%%PAGEBREAK%%%%%%%%%
%%%%%%%%%%%%%%%%%%%%%%%%%%%%%%%%%%%%%%%%%%
%%%%%%%%%%%%%%%%%%%%%%%%%%%%%%%%%%%%%%%%%%
%%%%%%%%%%%%%%%%%%%%%%%%%%%%%%%%%%%%%%%%%%
%%%%%%%%%%%%%%%%%%%%%%%%%%%%%%%%%%%%%%%%%%
%%%%%%%%PAGEBREAK%%%%%%%PAGEBREAK%%%%%%%%%
%%%%%%%%%%%%%%%%%%%%%%%%%%%%%%%%%%%%%%%%%%
%%%%%%%%%%%%%%%%PAGEBREAK%%%%%%%%%%%%%%%%%
%%%%%%%%%%%%%%%%%%%%%%%%%%%%%%%%%%%%%%%%%%
%%%%%%%%PAGEBREAK%%%%%%%PAGEBREAK%%%%%%%%%
%%%%%%%%%%%%%%%%%%%%%%%%%%%%%%%%%%%%%%%%%%
%%%%%%%%%%%%%%%%%%%%%%%%%%%%%%%%%%%%%%%%%%
%%%%%%%%%%%%%%%%%%%%%%%%%%%%%%%%%%%%%%%%%%
%%%%%%%%%%%%%%%%%%%%%%%%%%%%%%%%%%%%%%%%%%
%%%%%%%%PAGEBREAK%%%%%%%PAGEBREAK%%%%%%%%%
%%%%%%%%%%%%%%%%%%%%%%%%%%%%%%%%%%%%%%%%%%
%%%%%%%%%%%%%%%%PAGEBREAK%%%%%%%%%%%%%%%%%
%%%%%%%%%%%%%%%%%%%%%%%%%%%%%%%%%%%%%%%%%%
%%%%%%%%PAGEBREAK%%%%%%%PAGEBREAK%%%%%%%%%
%%%%%%%%%%%%%%%%%%%%%%%%%%%%%%%%%%%%%%%%%%
\begin{ekdosis}
  \begin{prose}
%-----------------------------
%                                                               asyādhāraḥ   kamaladalasya   catuṣṭayaṃ bhavati/  \E [p.62]
%tadṛśanaṃ mala        ity ucyate   tasmāt kamalam iti saṃjñā   asyādhāraḥ   kamalasya                            \P
%tadṛśa             na ity ucyate// tasmāt kamalam iti saṃjñā/  asyādhāraḥ// kamalasya dalaṃ catuṣṭayaṃ bhavatī/  \B
%tadṛśa             na ity ucyate// tasmāt kamalam iti saṃjñāṃ  asyādhāraḥ// kamalasya dalaṃ catuṣṭayaṃ bhavatī/  \L
%tadṛśanaṃ kamala      iti kathyate tasmāt kamala  iti saṃjñā   asyādhāra----kamalasya dala--catuṣṭayaṃ bhavati   \U1
%tad darśanaṃ malaṃ//  ity ucyate// tasmāt kamalam iti saṃjñā// asyādhāra----kamalasya dala  catuṣṭayaṃ bhavati// \U2
%\om                                                                                                               \N1
%\om                                                                                                               \D
%\om                                                                                                               \N2
%-----------------------------
%Such is the Kamala, it is said. Because of that the technical designation is "Kamala". The container of the Kamala consists of four leaves. 
%-----------------------------
\note[type=source, labelb=323, lem={kamalasya dalaṃ catuṣṭayaṃ}]{Ysv\textsuperscript{PT}: tatrādhāraś catuṣpatre sattvarajastamodayaḥ | etad bhāvasthitaś cātmā sādhvasādhukaro bhavet | asmin sati sthire citte yamo vandīva gacchati |}
\app{\lem[type=emendation, resp=egoscr,alt={tadṛśanaṃ kamalam}]{tadṛśanaṃ kamala\skp{m-i}}
  \rdg[wit={U1}]{\korr tadṛśanaṃ kamala}
  \rdg[wit={E}]{ tadṛśanaṃ mala}
  \rdg[wit={B,L}]{tadṛśa na}
  \rdg[wit={U2}]{tad darśanaṃ malaṃ ||}
}\app{\lem[wit={ceteri},alt={ity ucyate}]{\skm{m-i}ty\skp{-}ucyate}
  \rdg[wit={U1}]{iti kathyate}}/
tasmāt
\app{\lem[wit={ceteri},alt={kamalam}]{kamala\skp{m-i}}
  \rdg[wit={U1}]{kamala}
}\skm{m-i}ti \app{\lem[wit={ceteri}]{saṃjñā}
  \rdg[wit={L}]{saṃjñāṃ}}
\app{\lem[wit={B,E,L,P}]{asyādhāraḥ}
  \rdg[wit={U1,U2}]{asyādhāra°}}\dd{}
\app{\lem[wit={B,L}]{kamalasya dalaṃ catuṣṭayaṃ}
  \rdg[wit={E}]{kamaladalasya}
  \rdg[wit={P}]{kamalasya}
  \rdg[wit={U1,U2}]{kamalasya dala°}}
catuṣṭayaṃ \app{\lem[wit={ceteri}]{bhavati}
  \rdg[wit={B,L}]{bhavatī}}/
%-----------------------------
%prathamaṃ sattvaguṇasya    dvitīyaṃ rājayogaya     tṛtīyaṃ tamoguṇaḥ     caturtho dale manas  tiṣṭhati/ \E
%                           dvitīyaṃ rājayogasya    tṛtīyaṃ tamoguṇasya   caturthe dalamenas   tiṣṭhati \P
%prathamaṃ sattvaguṇasya/   dvitīyaṃ rājoguṇaḥ/     tṛtīyaṃ tamoguṇ/                                  \B
%prathamaṃ satyaguṇasya//   dvitīyaṃ rājoguṇasya    tṛtīyaṃ tamoguṇaḥ     caturthe dale manas  tiṣṭhati// \L
%\om                                                                 \N1
%\om                                                                 \D
%\om                                                                 \N2
%prathamadalaṃ satvaguṇasya dvitīyaṃ rajoguṇa       tṛtīyaṃ tamoguṇasya   caturthe dalaṃ manaḥ stiṣṭhati \U1 %%%294.jpg
%prathamaṃ satvaguṇasya//   dvitīyaṃ rājoguṇasya // tṛtīyaṃ tamoguṇasya// caturthe dale manas  tiṣṭhati// \U2
%-----------------------------
%The first leave consists of the Sattva-quality, the second consists of the Rajas-quality, the third consists of the Tamas-quality and in the fourth leave the mind is situated. 
%-----------------------------
\note[type=philcomm, labelb=324, lem={caturthe \ldots karoti}]{Two sentences are \om in B and are not recorded in the apparatus.}
\app{\lem[wit={U1}]{prathamadalaṃ}
  \rdg[wit={B,E,L,U2}]{prathamaṃ}
  \rdg[wit={P}]{\om}}
\app{\lem[wit={ceteri}]{sattvaguṇasya}
  \rdg[wit={L}]{satyaguṇasya}}\dd{}
dvitīyaṃ \app{\lem[wit={L,U2}]{rājoguṇasya}
  \rdg[wit={P}]{rājayogasya}
  \rdg[wit={E}]{rājayogaya}
  \rdg[wit={B}]{rājoguṇaḥ}
  \rdg[wit={U1}]{rajoguṇa}}\dd{}
tṛtīyaṃ \app{\lem[wit={P,U1,U2}]{tamoguṇasya}
  \rdg[wit={E,L}]{tamoguṇaḥ}
  \rdg[wit={B}]{tamoguṇ}}\dd{}
\app{\lem[wit={ceteri}]{caturthe}
  \rdg[wit={E}]{caturtho}}
\app{\lem[wit={E,L,U2}]{dale mana\skp{s-ti}}
  \rdg[wit={P}]{dalam enas}
  \rdg[wit={U1}]{dalaṃ manaḥ}
}\app{\lem[wit={ceteri},alt={tiṣṭhati}]{\skm{s-ti}ṣṭhati}
  \rdg[wit={U1}]{stiṣṭhati}}/  
%-----------------------------
%etad dala-catuṣṭayaṃ ca saṃgād ātmā sādhu           karoti/            \E
%etad dala-catuṣṭaya     saṃgād ātmā sāvadhvasādhu   karoti             \P
%etad dala-catuṣṭayaṃ saṃjñāgid ātmā sādhu           karoti//           \L
%etac      catuṣṭaya---  saṃgād ātma sādhvasādhū      karoti             \U1
%etad dalacatuṣṭaya    saṃyogād ātmā sādhvasādhu      karoti//           \U2
%\om                                                                 \N1
%\om                                                                 \D
%\om                                                                 \N2
%\om                                                                 \B
%-----------------------------
%Because of the conflict of the four leaves the self acts good and bad.  
%-----------------------------
\app{\lem[wit={ceteri},alt={etad}]{eta\skp{d-da}}
  \rdg[wit={U1}]{etac}
}\app{\lem[wit={ceteri},alt={dala}]{\skm{d-da}la}
  \rdg[wit={U1}]{\om}
}\app{\lem[wit={E,L}]{catuṣṭayaṃ}
  \rdg[wit={P,U1,U2}]{catuṣṭaya°}}
\app{\lem[wit={P,U1},alt={saṃgād}]{saṃgā\skp{d-ā}}
  \rdg[wit={E}]{ca saṃgād}
  \rdg[wit={L}]{saṃjñāgid}
  \rdg[wit={U2}]{saṃyogād}
}\app{\lem[wit={ceteri},alt={ātmā}]{\skm{d-ā}tmā}
  \rdg[wit={U1}]{ātma}}
\app{\lem[wit={U2}]{sādhvasādhu}
  \rdg[wit={U1}]{sādhvasādhū}
  \rdg[wit={P}]{sāvadhvasādhu}
  \rdg[wit={E,L}]{sādhu}}
karoti/
%-----------------------------
%tasmin kamale niścalī kṛte sati puruṣasya samīpe maraṇaṃ na gacchati/  \E
%tasmin kamale niścalī kṛte sati puruṣasya samipe maraṇaṃ na gacchati   \P %%%7667.jpg
%tasmin kamale niccalī kṛte sati puruṣasya samipe maraṇaṃ na gacchati/  \B
%tasmin kamale niccalī kṛte sati puruṣasya samīpe maraṇaṃ na gacchati/  \L %%%0031.jpg
%tasmin kamale niścalī kṛte sati puruṣasya samīpe maraṇaṃ nāgacchati/  \U2
%\om                                                                   \N1
%\om                                                                   \D
%\om                                                                   \N2
%\om                                                                   \U1
%-----------------------------
%While having made the state within the Kamala motionless, the death of the person does not go near.
%-----------------------------
tasmin-kamale \app{\lem[wit={E,P,U2}]{niścalī}
  \rdg[wit={B,L}]{niccalī}}
kṛte sati puruṣasya samīpe maraṇaṃ \app{\lem[wit={ceteri}]{na gacchati}
  \rdg[wit={U2}]{nāgacchati}}/
%-----------------------------
%idānīṃ hṛyakamalabhedāḥ kathyaṃte/ \E
%idānīṃ hṛdayakamalasya bhedaḥ kathyate/ \P
%idānīṃ hṛdayakamalasya bhedaḥ kathyate/ \B
%idānīṃ hṛdayakamalasya bhedaḥ kathyate// \L
%\om                                                                 \N1
%\om                                                                 \D
%\om                                                                 \N2
%idānīṃ hṛdayakamalasya dvitīyo bhedaḥ kathyate \U1
%idānīṃ hṛdayakamalasya bhedāḥ kathyate// \U2
%-----------------------------
%Now the division of the heart-lotus is taught. 
%-----------------------------
\note[type=source, labelb=325, lem={hṛdayakamalasya}]{Ysv\textsuperscript{PT}: anāhato dvitīyaṃ yatkathyate śṛṇu śraddhayā | anāhate mahāpīṭhe caturasrasamanvitam | varttate 'ṣṭadalaṃ padmam adhovaktran tu satpuram |}
idānīṃ \app{\lem[wit={U1}]{hṛdayakamalasya dvitīyo bhedaḥ}
  \rdg[wit={B,L,P}]{hṛdayakamalasya bhedaḥ}
  \rdg[wit={U2}]{hṛdayakamalasya bhedāḥ}
  \rdg[wit={E}]{hṛyakamalabhedāḥ}}
\app{\lem[wit={ceteri}]{kathyate}
  \rdg[wit={E}]{kathyaṃte}}/
%-----------------------------
%asya dvādaśadalāni siddhapuruṣāḥ kathayaṃti/ \E
%asya dvādaśadalāni siddhapuruṣāḥ kathayaṃti/ \P
%asya dvādaśadalāni siddhapuruṣāḥ kathyaṃte/ \B
%asya dvādaśadalāni siddhapuruṣāḥ kathyaṃte// \L
%\om                                                                 \N1
%\om                                                                 \D
%\om                                                                 \N2
%asya dvādaśadalāni siddhapuruṣāḥ kathyaṃte \U1
%asya dvādaśadalāni siddhāḥ puruṣāḥ kathayaṃtī// \U2
%-----------------------------
%The accomplished persons teach twelve leaves of it. 
%-----------------------------
asya dvādaśadalāni \app{\lem[wit={ceteri}]{siddhapuruṣāḥ}
  \rdg[wit={U2}]{siddhāḥ puruṣāḥ}}
\app{\lem[wit={B,L,U1}]{kathyante}
  \rdg[wit={E,P}]{kathayaṃti}
  \rdg[wit={U2}]{kathayaṃtī}}/ 
% -----------------------------
%anuparṇa-------dalānām        aṣṭadalānāṃ madhya ekaṃ kaṭhinaṃ bhavati/ \E displaced..see below!!!! 
%tathā dviṣaṇā  dalanāmadhya                      ekaṃ kaṭhiṇaṃ bhavatī// \B
%tathā dviṣaṇāṃ dalānām aṣṭadalanāṃ madhye        ekaṃ kaṭhiṇaṃ bhavati \P
%\om                                               \E
%tathā dviṣaṇā  dalanāmadhya                      ekaṃ kaṭhiṇaṃ  bhavati// \L
%\om                                                                 \N1
%\om                                                                 \D
%\om                                                                 \N2
%tathāpi varṇadalānām    aṣṭadalā                 eva kaṭitaṃ bhavati \U1
%tathā  dviṣaṇāṃ dalānām aṣṭadalānāṃ madhye       ekaṃ kaṭhiṇaṃ bhavati// \U2
%-----------------------------
%Thus, the best among the leaves, it is said, arise as a unit within the eight leaves.?!  
%----------------------------
\note[type=philcomm, labelb=325, lem={tathā dviṣaṇāṃ \ldots}]{The next nine sentences are missing in E, but found at position in the course of E's textual evidence. In order to preserve important readings the evidence of E will be collated arranged according to the structure found in all other witnesses.}
\crazy{\app{\lem[wit={B,L,P,U2}]{tathā}
    \rdg[wit={U1}]{tathāpi}
    \rdg[wit={E}]{\om}}
\app{\lem[type=emendation, resp=egoscr,alt={viṣāṇam}]{viṣāṇa\skp{m-a}}
  \rdg[wit={P,U2}]{\korr dviṣaṇāṃ}
  \rdg[wit={B,L}]{dviṣaṇā}
  \rdg[wit={U1}]{varṇadalānām}
  \rdg[wit={E}]{anuparṇadalānām}
}\app{\lem[wit={P,U2},alt={aṣṭadalānāṃ}]{\skm{m-a}aṣṭadalānāṃ madhye}
  \rdg[wit={B,L}]{dalanāmadhya}
  \rdg[wit={E}]{aṣṭadalānāṃ madhya}}
\app{\lem[wit={ceteri}]{ekaṃ}
  \rdg[wit={U1}]{eva}}
\app{\lem[type=emendation, resp=egoscr]{kathitaṃ}
  \rdg[wit={B,E,P,L,U2}]{\korr kaṭhiṇaṃ}
  \rdg[wit={U1}]{kaṭitaṃ}}
bhavati/}
% ----------------------------
%tadaṣṭadalaṃ kamalaṃ hṛdaye tiṣṭhati/ te ubhaye hṛdaye tiṣṭhataḥ/ \E  displaced..see below!!!! 
%tadaṣṭadalaṃ kamalaṃ hṛdaye tiṣṭhati  te ubha hṛdaye tiṣṭhataḥ/   \B
%tadaṣṭadalaṃ kamalaṃ hṛdaye tiṣṭhati  te ubhe hṛdaye tiṣṭhataḥ    \P
%tadaṣṭadalaṃ kamalaṃ hṛdaye tiṣṭhati  te ubhe hṛdaye?! tiṣṭhataḥ/ \L 0031.jpg Z.3
%\om                                                                 \N1
%\om                                                                 \D
%\om                                                                 \N2
%tata aṣṭadalaṃ kamalaṃ hṛdaye tiṣṭhati   te ubhe pi kathyate \U1
%tad  aṣṭadalaṃ kamalaṃ hṛdaye tiṣṭhati// te ubha hṛdaye tiṣṭhataḥ// \U2
%-----------------------------
%This eight-leaved Kamala is situated in the heart. Both heaven and hearth are situated in the heart. 
%----------------------------
\app{\lem[wit={ceteri}]{tadaṣṭadalaṃ}
  \rdg[wit={U1}]{tata aṣṭadalaṃ}}
kamalaṃ hṛdaye tiṣṭhati/
\app{\lem[wit={P,L,U1}]{te ubhe}
  \rdg[wit={B,U2}]{te ubha}
  \rdg[wit={E}]{te ubhaye}}
\app{\lem[wit={ceteri}]{hṛdaye}
  \rdg[wit={U1}]{pi}}
\app{\lem[wit={ceteri}]{tiṣṭhataḥ}
  \rdg[wit={U1}]{kathyate}}/
% ----------------------------
% prathame dale śabdās tiṣṭhanti | dvitīyadale sparśaḥ | tṛtīye dale rūpaṃ tiṣṭhati / \E displaced..see below!!!!
% prathamadale/ śabdas tiṣṭhati/ dvitīyadale sparśa tiṣṭhati/ tritiyadale rūpaṃ tiṣṭhati/  \B
% prathamadale  śabdas tiṣṭhati  dvitīye dale sparśas tiṣṭhati tritīyadale rūpaṃ tiṣṭhati  \P
% prathamadale// śabdas tiṣṭhati// dvitīyadale sparśas tiṣṭhati// tritiyadale rūpaṃ tiṣṭhati// \L
% \om                                                                 \N1
%\om                                                                 \D
%\om                                                                 \N2
% prathame dale śabdaḥ stiṣṭhati dvitīye dale sparśaḥ tiṣṭhati tritīyadale rūpaḥ tiṣṭhati \U1
% prathamadalaśabdaṃ tiṣṭhati// dvitīyadale sparśas tiṣṭhati// tritīyadale rūpaṃ tiṣṭhati// \U2
%-----------------------------
%Speech is situated in the first leave. Touch is situated in the second leave. Form is situated in the third leave. 
%----------------------------
\note[type=source, labelb=326, lem={prathamadale}]{Ysv\textsuperscript{PT}: sparśaśabdarūparasagandhā buddhir manas tathā | ahaṅkāraḥ kramād ete tatrāṣṭadalasaṃsthitāḥ |}
\app{\lem[wit={E,U1}]{prathame dale}
  \rdg[wit={P}]{prathamadale}
  \rdg[wit={B,L}]{prathamadale |}
  \rdg[wit={U2}]{prathamadala°}}
\app{\lem[wit={ceteri},alt={śabdas}]{śabda\skp{s-ti}}
  \rdg[wit={U1}]{śabdaḥ}}
\app{\lem[wit={ceteri}]{\skm{s-ti}ṣṭhati}
  \rdg[wit={U1}]{stiṣṭhati}}\dd{}
\app{\lem[wit={P,U1}]{dvitīye dale}
  \rdg[wit={ceteri}]{dvitīyadale}}
\app{\lem[wit={ceteri},alt={sparśas}]{sparśa\skp{s-ti}}
  \rdg[wit={E,U1}]{sparśaḥ}}
\app{\lem[wit={ceteri},alt={tiṣṭhati}]{\skm{s-ti}tiṣṭhati}
  \rdg[wit={E}]{\om}}\dd{}
\app{\lem[wit={E}]{tṛtīye}
  \rdg[wit={B,L}]{tritiya°}
  \rdg[wit={P,U1,U2}]{tritīya°}}dale
\app{\lem[wit={ceteri}]{rūpaṃ}
  \rdg[wit={U1}]{rūpaḥ}}
tiṣṭhati\dd{}
% ----------------------------
%caturthe dale rasas tiṣṭhati/ paṃcame dale gandhaṃ tiṣṭhati/ paṣṭhadale cittaṃ tiṣṭhati/    saptame dale buddhis tiṣṭhati/ aṣṭame dale haṃkāras tiṣṭhati/ etad aṣṭadalamadhye pṛthivyākāro varttate/ \E displaced..see below!!!! 
%caturthadale  rasa tiṣṭhati/  paṃcamadale gaṃdha tiṣṭhati/    saṣṭhadale ciṃta tiṣṭhati/    saptamadale budhis tiṣṭhati/   aṣṭamadale ahaṃkāras tiṣṭhati/ etadaṣṭadalamadhye/ samagrapṛthivyākāro vartate/ \B
%caturthe dale rasas tiṣṭhati  paṃcamadale gaṃdha tiṣṭhati     saṣṭhadale cittaṃ tiṣṭhati    saptamadale buddhis tiṣṭhati   aṣṭame dale haṃkāras tiṣṭhati etadaṣṭadale madhye samagrapṛthivyākāro vartate \P
% caturthadale rasas tiṣṭhati// paṃcamadale gaṃdhas tiṣṭhati// saṣṭhadale ciṃtta tiṣṭhati//  saptamadale budhis tiṣṭhati//  aṣṭamadale ahaṃkāras tiṣṭhati// etad aṣṭadalamadhye// samagrapṛthivyākāro vartate// \L
% \om                                                                 \N1
%\om                                                                 \D
%\om                                                                 \N2
%caturthadale rasaḥ tiṣṭhati   paṃcame dale gaṃdhaḥ stiṣṭhati saṣṭhe dale cittaḥ stiṣṭhati   saptame dale budhiḥ tiṣṭhati    aṣṭame dale ahaṃkāraḥ tiṣṭhati etattatadalamadhye samagryā pṛthvākāro varttate \U1
%caturthadala-rasas tiṣṭhati// paṃcame dale gaṃdhas tiṣṭhati// saṣṭhe dale cittaṃ tiṣṭhati// saptame dale buddhis tiṣṭhati// aṣṭame dale ahaṃkāraḥ tiṣṭhati// etad aṣṭadalamadhye samagrapṛthivyākāro vartate// \U2
%-----------------------------
%Taste is sitaued in the fourth leave. Smell is situated in the fifth leave. The mental faculty is situated in the sixth leave. The intellect is situated in the seventh leave. The principle of individuation is situated in the eight leave. This form of the entire world exists within the eight leaves.  
%-----------------------------
\app{\lem[wit={E,P}]{caturthe dale}
  \rdg[wit={B,L,U1}]{caturthadale}
   \rdg[wit={U2}]{caturthadala°}}
\app{\lem[wit={ceteri},alt={rasas}]{rasa\skp{s-ti}}
  \rdg[wit={U1}]{rasaḥ}
}\skm{s-ti}ṣṭhati/ 
\app{\lem[wit={E,U1,U2}]{pañcame dale}
  \rdg[wit={ceteri}]{pañcamadale}}
\app{\lem[wit={ceteri},alt={gaṅdhas}]{gandha\skp{s-ti}}
  \rdg[wit={B,P}]{gaṃdha}
  \rdg[wit={U1}]{gaṃdhaḥ}
}\app{\lem[wit={ceteri},alt={tiṣṭhati}]{\skm{s-ti}ṣṭhati}
  \rdg[wit={U1}]{stiṣṭhati}}\dd{}
\app{\lem[wit={U1,U2}]{saṣṭhe dale}
  \rdg[wit={B,P,L}]{saṣṭhadale}
  \rdg[wit={U1,U2}]{saṣṭhe dale}
  \rdg[wit={E}]{paṣṭhadale}}
\app{\lem[wit={E,P,U2}]{cittaṃ}
  \rdg[wit={B}]{ciṃta}
  \rdg[wit={L}]{ciṃtta}
  \rdg[wit={U1}]{cittaḥ}}
\app{\lem[wit={ceteri}]{tiṣṭhati}
  \rdg[wit={U1}]{stiṣṭhati}}\dd{}
\app{\lem[wit={E,U1,U2}]{saptame dale}
  \rdg[wit={ceteri}]{saptamadale}}
\app{\lem[wit={ceteri},alt={buddhis}]{buddhi\skp{s-ti}}
  \rdg[wit={U1}]{budhiḥ}
}\skm{s-ti}ṣṭhati\dd{}
\app{\lem[wit={E,P,U1,U2}]{aṣṭame dale}
  \rdg[wit={B,L}]{aṣṭamadale}}
\app{\lem[wit={E,P}]{'haṃkāra\skp{s-ti}}
  \rdg[wit={B,L}]{ahaṃkāras}
  \rdg[wit={U1,U2}]{ahaṃkāraḥ}
}\skm{s-ti}ṣṭhati/
\app{\lem[wit={ceteri},alt={etad aṣṭadalamadhye}]{etad\skp{-}aṣṭadalamadhye}
  \rdg[wit={P}]{etad aṣṭadale madhye}
  \rdg[wit={U1}]{etat tatadalamadhye}}
\end{prose}
\end{ekdosis}
\ekdpb*{}
%%%%%%%%%%%%%%%%%%%%%%%%%%%%%%%%%%%%%%%%%%
%%%%%%%%PAGEBREAK%%%%%%%PAGEBREAK%%%%%%%%%
%%%%%%%%%%%%%%%%%%%%%%%%%%%%%%%%%%%%%%%%%%
%%%%%%%%%%%%%%%%PAGEBREAK%%%%%%%%%%%%%%%%%
%%%%%%%%%%%%%%%%%%%%%%%%%%%%%%%%%%%%%%%%%%
%%%%%%%%PAGEBREAK%%%%%%%PAGEBREAK%%%%%%%%%
%%%%%%%%%%%%%%%%%%%%%%%%%%%%%%%%%%%%%%%%%%
%%%%%%%%%%%%%%%%%%%%%%%%%%%%%%%%%%%%%%%%%%
%%%%%%%%%%%%%%%%%%%%%%%%%%%%%%%%%%%%%%%%%%
%%%%%%%%%%%%%%%%%%%%%%%%%%%%%%%%%%%%%%%%%%
%%%%%%%%PAGEBREAK%%%%%%%PAGEBREAK%%%%%%%%%
%%%%%%%%%%%%%%%%%%%%%%%%%%%%%%%%%%%%%%%%%%
%%%%%%%%%%%%%%%%PAGEBREAK%%%%%%%%%%%%%%%%%
%%%%%%%%%%%%%%%%%%%%%%%%%%%%%%%%%%%%%%%%%%
%%%%%%%%PAGEBREAK%%%%%%%PAGEBREAK%%%%%%%%%
%%%%%%%%%%%%%%%%%%%%%%%%%%%%%%%%%%%%%%%%%%
%%%%%%%%%%%%%%%%%%%%%%%%%%%%%%%%%%%%%%%%%%
%%%%%%%%%%%%%%%%%%%%%%%%%%%%%%%%%%%%%%%%%%
%%%%%%%%%%%%%%%%%%%%%%%%%%%%%%%%%%%%%%%%%%
%%%%%%%%PAGEBREAK%%%%%%%PAGEBREAK%%%%%%%%%
%%%%%%%%%%%%%%%%%%%%%%%%%%%%%%%%%%%%%%%%%%
%%%%%%%%%%%%%%%%PAGEBREAK%%%%%%%%%%%%%%%%%
%%%%%%%%%%%%%%%%%%%%%%%%%%%%%%%%%%%%%%%%%%
%%%%%%%%PAGEBREAK%%%%%%%PAGEBREAK%%%%%%%%%
%%%%%%%%%%%%%%%%%%%%%%%%%%%%%%%%%%%%%%%%%%
  \begin{ekdosis}
    \begin{prose}
\note[type=source, labelb=327, lem={saparyā pṛthag ākārā}]{Ysv\textsuperscript{PT}: saparyā pṛthag ākārā varttate tatra niścitam | dhyānād ātmaprakāśo 'sya prakāśaṃ kamalaṃ tataḥ |}
%Die Form Verehrung, Ehrenerweisung (Saparya), einzeln Form,Gestalt existiert, dort ist Gewissheit. Aufgrund von Meditation sein Licht des Selbst is der Kamala welcher leuchtet.?!   
\app{\lem[wit={B,P,L,U2}]{samagrapṛthivyākāro}
  \rdg[wit={U1}]{samagryā pṛthvākāro}
  \rdg[wit={E}]{pṛthivyākāro}} vartate/
%-----------------------------
% atha ca   tatkamalamadhye  mukhaṃ tiṣṭhati/  asya kamalasya nādāt prakāśo bhavati/  \E  displaced..see below!!!!     
% atha ca// tatkamalamadhye  mukhaṃ tiṣṭhati/  asya kamalasya dhyānākāśo bhavati/ \B
% atha ca   tatkamalamadhye  mukhaṃ tiṣṭhati   asya kamalasya dhyānākāśo bhavati \P
% atha ca   tatkamalamadhye  mukhaṃ tiṣṭhati// asya kamalasya dhyānākāśo bhavati// \L
% atha ca   tatkamalaṃ  adho mukhaṃ tiṣṭhati   asya kamalasya dhyānād ātmaprakāśo bhavati \U1
% atha ca   tatkamalamadhye  mukhaṃ tiṣṭhati// asya kamalasya dhyānād āt prakāśo bhavati// \U2
%\om                                                                 \N1
%\om                                                                 \D
%\om                                                                 \N2
%I don't understand the sudden mention of the form of the etire earth, and sudden mention of mukhaṃ... something is going on... 
%-----------------------------
%Furthermore within that lotus exists a face/entrance. Because of the meditation on that lotus the light of the self arises. 
%-----------------------------
\crazy{atha ca \app{\lem[wit={ceteri}]{tatkamalamadhye}
  \rdg[wit={U1}]{tatkamalaṃ}}
\app{\lem[wit={ceteri}]{mukhaṃ}
  \rdg[wit={U1}]{adhomukhaṃ}} tiṣṭhati/
asya kamalasya
\app{\lem[wit={U1},alt={dhyānād ātmaprakāśo}]{dhyānād\skp{-}ātmaprakāśo}
  \rdg[wit={B,P,L}]{dhyānākāśo}
  \rdg[wit={U2}]{dhyānād ātprakāśo}
  \rdg[wit={E}]{nādāt prakāśo}}
%\note[type=philcomm, labelb=328, lem={saparyā}]{Since the evidence of the manuscript's lack of meaningfulness of this passage, I decided to emend according to the source text.}
bhavati/}
%-----------------------------
%prakāśānaṃtaraṃ kamalam ūrdhvamukhaṃ bhavati | 
%prakāśād anaṃtara/ kamalaṃ mūrdhvaṃ mukhaṃ bhavati/ \B [DSCN7174.jpg Z.1]
%prakāśād anaṃtaraṃ kamalam ūrddhvamukhaṃ bhavati \P
%prakāśāvan aṃtaraṃ kamalam ūrdhvamukhaṃ bhavati// \L
%\om                                                                 \N1
%\om                                                                 \D
%\om                                                                 \N2
%prakāśād anaṃtaraṃ kamalam ūrdhvamukhaṃ bhavati \U1
%prakāśād anaṃtaraṃ kamalam ūrdhvamukhaṃ bhavati// \U2
%-----------------------------
%From the light immediately afterwards the upward-facing lotus arises. 
%-----------------------------
\note[type=source, labelb=328, lem={prakāśād}]{Ysv\textsuperscript{PT}: yathā sūryaprakāśena ūrddhvavaktraṃ prakāśitam | ātmadhyānāt sadā tatra āyur vṛddhir dine dine |}
\app{\lem[wit={ceteri},alt={prakāśād}]{prakāśā\skp{d-a}}
  \rdg[wit={L}]{prakāśāvan}
  \rdg[wit={E}]{prakāśā°}
}\app{\lem[wit={P,U1,U2},alt={anaṃtaraṃ}]{\skm{d-a}naṃtaraṃ}
  \rdg[wit={B}]{anaṃtara |}
  \rdg[wit={L}]{aṃtaraṃ}
  \rdg[wit={E}]{°naṃtaraṃ}}
\app{\lem[wit={ceteri},alt={kamalam}]{kamala\skp{m-ū}}
  \rdg[wit={B}]{kamalaṃ}}\app{\lem[wit={ceteri},alt={ūrdhvamukhaṃ}]{\skm{m-ū}ūrdhvamukhaṃ}
  \rdg[wit={B}]{mūrdhvaṃ mukhaṃ}} bhavati/
% -----------------------------
%tathā sūryaprakāśānantaraṃ    tadā saromadhye   kamalaṃ vikasati/ \E
%tathā sūryo prakāśānaṃtaraṃ/  tadā kamalamadhye kamalaṃ vikasati//  \B
%tathā sūryaprakāśānaṃtaraṃ    tadā kamalamadhye kamalaṃ visati      \P %%%7668.jpg 
%tathā sūryaprakāśānaṃtaraṃ    tadā kamalamadhye kamalaṃ vikasati//  \L
%yathā sūryaprakāśānaṃtaraṃ    tadā              kamalaṃ vikasati//  \U1 %%%295.jpg
%tathā sūryaprakāśād anaṃtaraṃ tadā   malamadhye kamalaṃ vikasati//    \U2
%\om                                                                 \N1
%\om                                                                 \D
%\om                                                                 \N2
%-----------------------------
%Thus, immediately afterwards, from the light which is like the sun then within the lotus a lotus blooms. 
%-----------------------------
\app{\lem[wit={ceteri}]{tathā}
  \rdg[wit={U1}]{yathā}}
\app{\lem[wit={U2},alt={sūryaprakāśād anaṃtaraṃ}]{sūryaprakāśād\skp{-}anaṃtaraṃ}
  \rdg[wit={B}]{sūryo prakāśānaṃtaraṃ |}
  \rdg[wit={E,P,L,U1}]{sūryaprakāśānaṃtaraṃ}}
tadā \app{\lem[wit={B,P,L}]{kamalamadhye}
  \rdg[wit={U2}]{malamadhye}
  \rdg[wit={E}]{saromadhye}
  \rdg[wit={U1}]{\om}}
kamalaṃ
\app{\lem[wit={ceteri}]{vikasati}
  \rdg[wit={P}]{visati}}/
%-----------------------------
%tathedam   apy ātmāprakāśānantaram  ūrdhvamukhaṃ    vikasati | tanmadhye  paramānandarūpābhūmir bhavati |
%tam        api ātmaprakāśanaṃtaraṃ  mūrdhvaṃ mukhaṃ vikasati/  tanmadhye  paramānaṃdarūpābhūmir  bhavati// \B
%tathe dam  api ātmaprakāśānaṃtaram  ūrdhvaṃ mukhaṃ  vikasati   tanmadhye  paramānaṃdarūpābhūmir  bhavati \P
%tam        api ātmaprakāśanaṃtaram  ūrdhvamukhaṃ    vikasati// tanmadhye  paramānaṃdarūpo bhūmir bhavati// \L
%tathā idam apy ātmaprakāśānataram   ūrdhvamukhaṃ    vikasati   tanmadhye  paramānaṃdarūpābhūmir bhavatī \U1
%tathedam   api ātmaprakāśānaṃtaram  ūrdhvamukhaṃ    vikasati// tanamadhye paramānaṃdasūpābhūmir  bhavati// \U2 %%%424.jpg
%\om                                                                 \N1
%\om                                                                 \D
%\om                                                                 \N2
%-----------------------------
%Thus immdiately after the light of the self the upward facing [one] blooms. Within it the place having the form of highest bliss arises.  
%-----------------------------
\app{\lem[wit={E,P,U2}]{tatheda\skp{m-a}}
  \rdg[wit={U1}]{tathā idam}
  \rdg[wit={B,L}]{tam}}\app{\lem[wit={E,U1},alt={apy}]{\skm{m-a}\skp{py-ā}}
  \rdg[wit={ceteri}]{api}}\app{\lem[wit={P,U2},alt={ātmaprakāśānaṃtaram}]{\skm{py-ā}tmaprakāśānaṃtara\skp{m-ū}}
  \rdg[wit={U1}]{ātmaprakāśānataram}
  \rdg[wit={E}]{ātmāprakāśānantaram}
}\app{\lem[wit={E,L,U1,U2},alt={ūrdhvamukhaṃ}]{\skm{m-ū}rdhvamukhaṃ}
  \rdg[wit={P}]{ūrdhvaṃ mukhaṃ}
  \rdg[wit={B}]{mūrdhvaṃ mukhaṃ}}
vikasati/
\app{\lem[wit={ceteri}]{tanmadhye}
  \rdg[wit={U2}]{tanamadhye}} 
paramānanda\app{\lem[wit={ceteri},alt={°rūpābhūmir}]{rūpābhūmi\skp{r-bha}}
  \rdg[wit={L}]{°rūpo bhūmir}}\app{\lem[wit={ceteri},alt={bhavati}]{\skm{r-bha}vati}
  \rdg[wit={U1}]{bhavatī}}/
% -----------------------------
%tasyāhaṃso ham   iti saṃjñā   tasyā madhye svātmano dhyānād dine dine hy āyur varddhate |
%tasyāhaṃso haṃsa iti saṃjñā// tasya madhye svātmano dhyād   dīne dine āyur vardhayati/ \B
%tasyāhaṃso haṃsa iti saṃjñā   tasyā madhye svātmano dhyānād dine dine āyur varddhate \P
%tasyāhaṃso haṃsa iti saṃjñā// tasya madhye svātmano dhyānād dine dine āyur varddhayati... \L
%\om                                                                 \N1
%\om                                                                 \D
%\om                                                                 \N2
%tasyāhaṃso haṃsa iti saṃjñā   tasyā madhye svātmanaḥ dhyānād dine dine āyur varddhati \U1
%tasyāhaṃso haṃsa iti saṃjñā// tasyā madhye svātmano dhyād dine dine āyur varddhati// \U2
%-----------------------------
%The technical designation of it is "I am he, he is I". Because of Meditation onto the own self which is situated within it [the Kamala], the force of life is caused to grow day by day.   
%-----------------------------
tasyāhaṃ so \app{\lem[wit={ceteri}]{'haṃ sa}
  \rdg[wit={E}]{ham}} iti saṃjñā\dd{}
\app{\lem[wit={P,U1,U2}]{tasyā}
  \rdg[wit={B,L}]{tasya}} madhye
\app{\lem[wit={ceteri}]{svātmano}
  \rdg[wit={U1}]{svātmanaḥ}}
\app{\lem[wit={ceteri},alt={dhyānād}]{dhyānā\skp{d-di}}
  \rdg[wit={B,U2}]{dhyād}
}\skm{d-d}ine dine \app{\lem[wit={ceteri},alt={āyūr}]{āyū\skp{r-va}}
  \rdg[wit={E}]{hy āyur}}\app{\lem[wit={B,L},alt={vardhayati}]{\skm{r-va}rdhayati}
  \rdg[wit={U1,U2}]{varddhati}
  \rdg[wit={E,P}]{varddhate}}/
%-----------------------------
%rogo dūre bhavati/ tathā dviṣaśaktis tṛtīyalokāṃtaḥ  samyak samudrā khecarī      \E
%rogā dūre bhavaṃti            śaktis tritayalokāṃta  samyak  mudrā ca khecarī         \P
%rogā dūro bhavati/            śaktis tritayo lokāṃta samyak  mudrā ca khecari/        \B
%rogā dūrā bhavaṃti//          śaktis tritayo lokāṃta samyak  mudrā ca khecarī/   \L
%rogā dūre bhavaṃti            śaktis trīvalī kṛtaṃ   samyak mudrā bhavati khecarī     \U1
%rogā dūre bhavaṃti//          śaktis tritayalokāṃtaḥ samyak mudrā ca khecarī//   \U2
%\om                                                                      \N1
%\om                                                                      \D
%\om                                                                      \N2
%-----------------------------
%Diseases are remote. Energy, the trinity of the inner worlds, the entire 
%-----------------------------
\note[type=source, labelb=329, lem={śaktis}]{Ysv\textsuperscript{PT}: śaktiprasannatā syāc ca rogaśokavivarjitaḥ | yasya mudrābhyāsaśālī samyak siddhā ca khecarī |} %Purificaton of the energy and freedom from diseases arises for one who is abundantly enganged in the practice of Mudrā. He is clearly becomes (samyak) a Siddha and a Sky-roamer. 
\note[type=philcomm, labelb=330, lem={śaktis}]{Evidence of witness E resumes at this point and synchronizes with the structure of the other witnesses.}
\app{\lem[wit={ceteri}]{rogā}
  \rdg[wit={E}]{rogo}}
\app{\lem[wit={ceteri}]{dūre}
  \rdg[wit={P}]{dūro}
  \rdg[wit={L}]{dūrā}}
\app{\lem[wit={ceteri}]{bhavanti}
  \rdg[wit={B,E}]{bhavati}}/
\crazy{\app{\lem[wit={U2}]{tritayalokāṃtaḥ}
  \rdg[wit={P}]{tritayalokāṃta°}
  \rdg[wit={E}]{tṛtīyalokāṃtaḥ}
  \rdg[wit={B,L}]{tritayo lokāṃta°}
  \rdg[wit={U1}]{trīvalī kṛtaṃ}}
samyak \app{\lem[wit={ceteri}]{mudrā}
  \rdg[wit={E}]{samudrā}}
\app{\lem[wit={P,L,U2}]{ca khecarī}
  \rdg[wit={B}]{ca khecari}
  \rdg[wit={U1}]{bhavati khecarī}
  \rdg[wit={E}]{khecarī}}/}
%-----------------------------
%cidānandādvayaś candracaṃdrikā veti nāmānvitaḥ//   paramātmanāsaha   raśmipuṃja-- -prakāśaḥ   prakāśānandayor aikyaṃ prakarttavyaṃ  nirantaraṃ   svayaṃ manasi mahājyotir ābhāti paramaṃ padam//  \E
%cidānaṃdādayaś  caḍriś cadrikā    cetanānvitāḥ     paramātmāsahāsūryaraśmipuṃja--prakāśakaḥ   prakāśānaṃdayor aikyaṃ prakartavyaṃ   niraṃtaraṃ   svayam agnir  mahājyotir ābhāti paramaṃ padaṃ    \P
%cidānaṃdādayoś  caḍrikā-------------cetanvitāḥ     paramātmāmahāsūryaraśmiyuṃja--prakāśakaḥ   prakāśānaṃdayor aikyaṃ prakartavyaṃ/  niraṃtaraṃ   svayam agnir  mahājyotir ābhāti paramapadam      \B
%cidānaṃdādayoś  caṃḍrīkā------------cetanvitāḥ     paramātmāmahāsūryaraśmipuṃja--prakāśakaḥ   prakāśānaṃdayor aikyaṃ prakartavyaṃ// niraṃtaraṃ   svayam agnir  mahājyotir ābhāti paramaṃ padaṃ//  \L
%cidānaṃdodayaṃś caṃdraḥś cetanāś caṃdrakānvitā     paramātmāmahāsūryaraśmipuṃjaḥ prakāśakaḥ   prakāśānaṃdayor aikyaṃ prakartavyaṃ   niraṃtaraṃ   svayam agnir  mahājyotiś abhāti paramaṃ padaṃ    \U1
%cidānaṃdādayaḥ  caṃdrāś ca   drikā cetanānvitaḥ//  paramātmāmahāsūryaraśmipuṃja--prakāśakaḥ// prakāśānaṃdayor aikyaṃ prakarttavyaṃ  niraṃtaraṃ// svayam agnir  mahājyotir ābhāti paraṃmapadaṃ//   \U2
%\om                                                                 \N1
%\om                                                                 \D
%\om                                                                 \N2
%-----------------------------
%The non-duality consisting of bliss and consciousness is consciousness endowed with illumination. The highest self, the great sun [and] the mass of rays of the sun is the light. Both bliss and light shall be brought into unity uninteruptedly. The own fire being the great light illumines the highest place. 
%-----------------------------
\note[type=source, labelb=331, lem={cidānandā°}]{Ysv\textsuperscript{PT}: cidānandamayaṃ cittaṃ cetanā candrikānvitā | paramātmā mahāsūryaḥ sūrya ekaḥ prakāśakaḥ | prakāśānandayor aikyaṃ karttavyañ ca nirantaram | dīptas tathā mahājyotīr avirbhāti paraṃ padam |}%The mental faculty consisting of bliss of consciousness is the soul endowed with illumination. The highest soul, the great sun [and the] sun are one light.  Uninteruptedly, both light and bliss shall be brought into unity. In this way the light and the great light illumine the highest place. 
\app{\lem[wit={P},alt={cidānandādayaś}]{cidānandādaya\skp{ś-ca}}
  \rdg[wit={U2}]{cidānaṃdādayaḥ}
  \rdg[wit={U1}]{cidānaṃdodayaṃś}
  \rdg[wit={B,L}]{cidānaṃdādayoś}
  \rdg[wit={E}]{cidānandādvayaś}
}\app{\lem[type=emendation, resp=egoscr, alt={candrikā°}]{\skm{ś-ca}ndrikā}
  \rdg[wit={L}]{\korr caṃḍrīkā°}
  \rdg[wit={B}]{caḍrikā}
  \rdg[wit={P}]{caḍriś cadrikā}
  \rdg[wit={E}]{candracaṃdrikā}
  \rdg[wit={U1}]{caṃdraḥś cetanāś}
  \rdg[wit={U2}]{caṃdrāś cadrikā}}
\app{\lem[type=emendation, resp=egoscr]{cetanānvitā}
  \rdg[wit={E}]{\korr veti nāmānvitaḥ}
  \rdg[wit={P}]{cetanānvitāḥ}
  \rdg[wit={B,L}]{cetanvitāḥ}
  \rdg[wit={U1}]{caṃdrakānvitā}
  \rdg[wit={U2}]{cetanānvitaḥ}}/
\app{\lem[wit={U1}]{paramātmāmahāsūryaraśmipuṃjaḥ}
  \rdg[wit={B,L,P,U2}]{paramātmāmahāsūryaraśmipuṃja°}
  \rdg[wit={E}]{paramātmanāsaharaśmipuṃja°}}
\app{\lem[wit={ceteri}]{prakāśakaḥ}
  \rdg[wit={E}]{prakāśaḥ}}/
prakāśānaṃdayor-aikyaṃ prakartavyaṃ/ 
niraṃtaraṃ svaya\skp{m-a}\app{\lem[wit={ceteri}, alt={agnir}]{\skm{m-a}gni\skp{r-ma}}
  \rdg[wit={E}]{manasi}
}\app{\lem[wit={ceteri},alt={mahājyotir}]{\skm{r-ma}hājyoti\skp{r-ā}}
  \rdg[wit={U1}]{mahājyotiś}
}\app{\lem[wit={ceteri},alt={ābhāti}]{\skm{r-ā}bhāti}
  \rdg[wit={U1}]{abhāti}}
\app{\lem[wit={E,P,L,U1}]{paramaṃ padaṃ}
  \rdg[wit={B}]{paramapadam}
  \rdg[wit={U2}]{paraṃmapadaṃ}}/
%-----------------------------
%sadoditamanaś   candraḥ sūryodayam avekṣate/ tena grasto manaś  candraḥ so pi lipyaḥ svayaṃ pade// \E
%sadoditaṃ manaś caṃdraḥ sūryodaya  ivekṣate  tena grasto manaś  caṃdraḥ so pi līnaḥ svayaṃ pade \P
%sadoditamanaś   cadraḥ  sūryodaya  ivekṣate  tena grasto manaḥ/ ścaṃdraḥ so   lina  syayaṃ pade \B
%sadoditamanaś   caṃdraḥ sūryodaya  ivekṣate//tena grasto manaś  caṃdraḥ so pi linaṃ svayaṃ pade... \L
%\om                                                                 \N1
%\om                                                                 \D
%\om                                                                 \N2
%sadoditamanaḥś  caṃdraḥ sūryodaye ca lakṣyate tena graste manaś  caṃdraḥ so pi linaṃ  svayaṃ pade \U1 %lakṣate geht auch, aber source hat ivekṣate 
%madohitaṃ manaś candraḥ sūryodaya ivekṣate//  tena graste manaś  candraḥ so pi lipyaḥ svayaṃ pade// \U2
%-----------------------------
%The constantly active mind being the moon perceives how the sun rises. Because of this, the mind, which is the moon, is devoured. Although it disappears in its own place. 
%-----------------------------
\note[type=source, labelb=331, lem={sadodita°}]{Ysv\textsuperscript{PT}: sadoditaṃ manaḥsūryaṃ candrajyotir ivekṣate |}
\app{\lem[wit={B,E,L},alt={sadoditamanaś}]{sadoditamana\skp{ś-ca}}
  \rdg[wit={U1}]{sadoditamanaḥś}
  \rdg[wit={P,U2}]{sadoditaṃ manaś}
}\app{\lem[wit={ceteri},alt={candraḥ}]{\skm{ś-ca}ndraḥ}
  \rdg[wit={B}]{cadraḥ}}
\app{\lem[wit={E},alt={sūryodayam}]{sūryodaya\skp{m-i}}
  \rdg[wit={B,P,L,U2}]{sūryodaya}
  \rdg[wit={U1}]{sūryodaye}
}\app{\lem[wit={ceteri},alt={ivekṣate}]{\skm{m-i}vekṣate}
  \rdg[wit={E}]{avekṣate}
  \rdg[wit={U1}]{ca lakṣyate}}
tena
\app{\lem[wit={ceteri}]{grasto}
  \rdg[wit={U1,U2}]{graste}}
\app{\lem[wit={ceteri},alt={manaś}]{mana\skp{ś-ca}}
  \rdg[wit={B}]{manaḥ |}
}\app{\lem[wit={ceteri},alt={candraḥ}]{\skm{ś-ca}ndraḥ}
  \rdg[wit={B}]{ścaṃdraḥ}}
so 'pi \app{\lem[wit={P}]{līnaḥ}
  \rdg[wit={B}]{lina}
  \rdg[wit={L,U1}]{linaṃ}
  \rdg[wit={E,U2}]{lipyaḥ}}
svayaṃ pade/
%-----------------------------
%padam eva mahānagnir yame grastaṃ  kalāmayam/   evaṃ candrārkavahnīnāṃ   saṃketaḥ   paramārthataḥ// \E [p.64]
%    m eva mahānagnir yena grastaṃ  kalāmayaṃ    evaṃ caṃdrārkavahnīnāṃ   saṃketaḥ   paramārthataḥ   \P
%padam eva mahānagnir sūryagrastaṃ  kalāmayam/   evaṃ caṃdrārkvavahnīnāṃ  saṃketanaṃ paramārthataḥ// \B
%padam eva mahānagniḥ sūryagrastaṃ  kalāmayaṃ//  evaṃ caṃdrārkavavahnīnāṃ saṃketanaṃ paramārthataḥ// \L
%padam eva mahānagnir yena grastaṃ  kalāmayaḥ    evaṃ caṃdrārkavatāṃ      saṃketaḥ   paramārthataḥ vā \U1
%padam eva mahānagnir yena grastaṃ  kalāmayaṃ//  evaṃ caṃdrārkavahnīnāṃ   saṃketaḥ   paramārthataḥ// \U2
%\om                                                                 \N1
%\om                                                                 \D
%\om                                                                 \N2
%-----------------------------
%The place, however, is the great fire by which that which is formed of the kalas is devoured. (padam= nom sg) Thus there is agreement of the fires and the beams of the moon with the highest reality.
%The place, however, made of digits is devoured by the sun, the great fire. Thus there is agreement of the fires and the beams of the moon with the highest reality.
%The place is devoured by that which is the great fire made of kalas. (U1) Thus there is agreement of the fires and the beams of the moon with the highest reality.
%The place, however is the great fire, being made of the digitd is devoured by the sun.
%----------------------------
\app{\lem[wit={ceteri},alt={padam}]{pada\skp{m-e}}
  \rdg[wit={P}]{m}
}\skm{m-e}va
\app{\lem[wit={ceteri},alt={mahānagnir}]{mahānagni\skp{r-ye}}
  \rdg[wit={L}]{mahānagniḥ}
}\app{\lem[wit={P,U1,U2}, alt={yena}]{\skm{r-ye}na}
  \rdg[wit={E}]{yame}
  \rdg[wit={B,L}]{sūrya°}}
grastaṃ
\app{\lem[wit={ceteri}]{kalāmayaṃ}
  \rdg[wit={U1}]{kalāmayaḥ}}/
evaṃ 
\app{\lem[wit={E,P,U2}]{candrārkavahnīnāṃ}
  \rdg[wit={L}]{caṃdrārkavavahnīnāṃ}
  \rdg[wit={B}]{caṃdrārkvavahnīnāṃ}
  \rdg[wit={U1}]{caṃdrārkavatāṃ}}
\end{prose}
\end{ekdosis}
\ekdpb*{}
%%%%%%%%%%%%%%%%%%%%%%%%%%%%%%%%%%%%%%%%%%
%%%%%%%%PAGEBREAK%%%%%%%PAGEBREAK%%%%%%%%%
%%%%%%%%%%%%%%%%%%%%%%%%%%%%%%%%%%%%%%%%%%
%%%%%%%%%%%%%%%%PAGEBREAK%%%%%%%%%%%%%%%%%
%%%%%%%%%%%%%%%%%%%%%%%%%%%%%%%%%%%%%%%%%%
%%%%%%%%PAGEBREAK%%%%%%%PAGEBREAK%%%%%%%%%
%%%%%%%%%%%%%%%%%%%%%%%%%%%%%%%%%%%%%%%%%%
%%%%%%%%%%%%%%%%%%%%%%%%%%%%%%%%%%%%%%%%%%
%%%%%%%%%%%%%%%%%%%%%%%%%%%%%%%%%%%%%%%%%%
%%%%%%%%%%%%%%%%%%%%%%%%%%%%%%%%%%%%%%%%%%
%%%%%%%%PAGEBREAK%%%%%%%PAGEBREAK%%%%%%%%%
%%%%%%%%%%%%%%%%%%%%%%%%%%%%%%%%%%%%%%%%%%
%%%%%%%%%%%%%%%%PAGEBREAK%%%%%%%%%%%%%%%%%
%%%%%%%%%%%%%%%%%%%%%%%%%%%%%%%%%%%%%%%%%%
%%%%%%%%PAGEBREAK%%%%%%%PAGEBREAK%%%%%%%%%
%%%%%%%%%%%%%%%%%%%%%%%%%%%%%%%%%%%%%%%%%%
%%%%%%%%%%%%%%%%%%%%%%%%%%%%%%%%%%%%%%%%%%
%%%%%%%%%%%%%%%%%%%%%%%%%%%%%%%%%%%%%%%%%%
%%%%%%%%%%%%%%%%%%%%%%%%%%%%%%%%%%%%%%%%%%
%%%%%%%%PAGEBREAK%%%%%%%PAGEBREAK%%%%%%%%%
%%%%%%%%%%%%%%%%%%%%%%%%%%%%%%%%%%%%%%%%%%
%%%%%%%%%%%%%%%%PAGEBREAK%%%%%%%%%%%%%%%%%
%%%%%%%%%%%%%%%%%%%%%%%%%%%%%%%%%%%%%%%%%%
%%%%%%%%PAGEBREAK%%%%%%%PAGEBREAK%%%%%%%%%
%%%%%%%%%%%%%%%%%%%%%%%%%%%%%%%%%%%%%%%%%%
\begin{ekdosis}
  \begin{prose}
    \noindent
\app{\lem[wit={ceteri}]{saṃketaḥ}
  \rdg[wit={B,L}]{saṃketanaṃ}}
\app{\lem[wit={ceteri}]{paramārthataḥ}
  \rdg[wit={U1}]{paramārthataḥ vā}}/
%----------------------------
%idānīṃ yogasiddher    anaṃtaram    etādṛśaṃ jñānam utpadyate/ \E
%idānīṃ yogasiddhe     naranaṃtaraṃ etādṛśaṃ jñānam utpadyate \P %%%7669.jpg 
%idānīṃ yo yogasiddhar anaṃtaraṃ/   etādṛśaṃ jñānam utpadyate \B
%idānīṃ yogasiddhar    anaṃtaraṃ    etādṛśaṃ jñānaṃ utpadyate... \L
%\om                                                                 \N1
%\om                                                                 \D
%\om                                                                 \N2
%idānīṃ yogasiddhar    anaṃtaraṃ etādṛśa--jñānam utpadyate... \U1
%idānīṃ yogasiddher    anaṃtaraṃ etādṛśaṃ jñānam utpadyate// \U2
%-----------------------------
%Now, immediately afterwards such knowledge of the accomplishment of yoga is generated. 
%----------------------------
idānīṃ \app{\lem[wit={L,U1},alt={yogasiddhar}]{yogasiddha\skp{r-a}}
  \rdg[wit={B}]{yo yogasiddhar}
  \rdg[wit={E,U2}]{yogasiddher}
  \rdg[wit={P}]{yogasiddhe}
}\app{\lem[wit={ceteri},alt={anaṃtaraṃ}]{\skm{r-a}nantaraṃ}
  \rdg[wit={B}]{anaṃtaraṃ |}
  \rdg[wit={P}]{naranaṃtaraṃ}}
\app{\lem[wit={ceteri}]{etādṛśaṃ}
  \rdg[wit={U1}]{etādṛśa}} 
\app{\lem[wit={ceteri},alt={jñānam}]{jñāna\skp{m-u}}
  \rdg[wit={L}]{jñānaṃ}
}\skm{m-u}tpadyate/
\end{prose}
\end{ekdosis}
\begin{ekdosis}
\begin{tlg}
%----------------------------
%yadā nāsti svayaṃ karttā kāraṇaṃ   na kulākulam// avyaktaṃ na paraṃ tattvam anāmā      vidyate tadā//1//  \E SSP 1.4 
%yadā nāsti svayaṃ kartā  kāraṇaṃ   na kulākulam   avyaktaṃ na paraṃ tatvam  anāmā      vidyate tadā/  \P
%yadā nāsti svayaṃ kartā  kāraṇaṃ   na kulākulam// avyaktaṃ na param         anāmā      vidyate tadā/  \B
%yadā nāsti svayaṃ kartā  kāraṇaṃ   na kulākulam// avyaktaṃ na param         anāmā      vidyate tadā...  \L
%padā nāsti svayaṃ karttā kāraṇaṃ   na kulākulam   avyaktaṃ na paraṃ tatvaṃ  manā bhā?? vidyate tadā...  \U1
%yadā nāsti svayaṃ karttā kāraṇaṃ// na kulākulaṃ// avyaktaṃ na paraṃ tatvam  anāmā      vidyate tadā// \U2
%\om                                                                 \N1
%\om                                                                 \D
%\om                                                                 \N2
%-----------------------------
%When the creator himself, the cause, the "Kula (Śakti) and Akula (Śiva)" not exist, but then the unmanifest, the supreme tattva, the nameless one exists. 
%----------------------------
\note[type=source, labelb=332, lem={yadā nāsti}]{Ysv\textsuperscript{PT}: yadā nāsti svayaṃkartā kāraṇaṃ na kulākulam | avyaktaṃ ca paraṃ brahma anāmā vidyate tadā ||1.4||}
  \tl{\app{\lem[wit={ceteri}]{yadā}
      \rdg[wit={U1}]{padā}}
    nāsti svayaṃ karttā \app{\lem[wit={ceteri}]{kāraṇaṃ}
      \rdg[wit={U2}]{kāraṇaṃ ||}} na
    \app{\lem[wit={ceteri}]{kulākulam}
      \rdg[wit={U2}]{kulākulaṃ}}/\\}
    \tl{avyaktaṃ na
    \app{\lem[wit={ceteri}]{paraṃ}
      \rdg[wit={B,L}]{para°}}
    \app{\lem[wit={E,P,U2},alt={tattvam}]{tattva\skp{m-a}}
      \rdg[wit={U1}]{tatvaṃ}
      \rdg[wit={P}]{tatva°}
      \rdg[wit={B,L}]{\om}
}\app{\lem[wit={B,E,L,P,U2},alt={anāma}]{\skm{m-a}nāmā}
      \rdg[wit={U1}]{manā bhā}}
    vidyate tadā\dd{}1\hskip-2pt\dd{}}
\end{tlg}
\end{ekdosis}
\bigskip
\begin{ekdosis}
  \begin{prose}
%----------------------------  
%anāmā ekaḥ kaścitpuruṣo varttate/ \E
%anāmā ekaḥ kaścitpuruṣo varttate \P
%anāmā ekapuruṣo varttate/ \B
%anāmā ekapuruṣo vartate// \L
%\om                                                                 \N1
%\om                                                                 \D
%\om                                                                 \N2
%anāmay eka  kāścitpuruṣo varttate \U1 %%%296.jpg
%anāmā  ekaḥ kaścitpuruṣo varttate// \U2
%-----------------------------
%The nameless, [the] one, [the] unspecified consciousness exists. 
%----------------------------
\app{\lem[wit={ceteri}]{anāmā}
  \rdg[wit={U1}]{anāmay}}
\app{\lem[wit={E,P,U2}]{ekaḥ}
  \rdg[wit={B,L,U1}]{eka°}}
\app{\lem[wit={ceteri}]{kaścitpuruṣo}
  \rdg[wit={B,L}]{°puruṣo}}
vartate/
%----------------------------
%anāmnaś ca parāvaraḥ    parātparaḥ  paraṃ padaṃ   paramapadāt  paraṃ  śūnyaṃ    śūnyān niraṃjanam    \E
%anāmnaḥ    parāvaraḥ    parāvarāt   paramapadaṃ   paramapadāt  paramaśūnyaṃ     śūnyān niraṃjanaḥ     \P
%anāmnaś ca parāvarā-----parāvarāt   paraṃ pada/   paramapadāt  paramaśūnyaṃ     śūnyā niraṃjanaṃ/    \B
%anāmnaś ca parāvarā-----parāvarāt   paraṃ padaṃ   paramapadāt  paramaśūnya------śūnyā niraṃjanaṃ/    \L
%anāthaḥ    parāvaraś ca parāvarāt   paraṃ padaṃ   paramapadāt  paramaṃ śūnyaṃ   śūnyā niraṃjanaḥ      \U1
%anāmnaś ca parāvaraḥ//  parāvarāt   paraṃ padaṃ// paramapadāt  paramaṃ śūnyaṃ// śūnyān niraṃjanaṃ//  \U2
%\om                                                                 \N1
%\om                                                                 \D
%\om                                                                 \N2
%-----------------------------
%It is nameless and all encompassing. From being all-encompassing [it is] the highest place. From the highest place [it is] the highest emptiness. From the emptiness [it is] immacule. The five qualities of the nameless are: 
%----------------------------
\app{\lem[wit={ceteri},alt={anāmnaś ca}]{anāmnaś\skp{-}ca}
  \rdg[wit={P}]{anāmnaḥ}
  \rdg[wit={U1}]{anāthaḥ}}
\app{\lem[wit={E,P,U2}]{parāvaraḥ}
  \rdg[wit={U1}]{parāvaraś ca}
  \rdg[wit={B,L}]{parāvarā°}}/
\app{\lem[wit={ceteri}]{parāvarāt}
  \rdg[wit={E}]{parātparaḥ}}
\app{\lem[wit={E,L,U1,U2}]{paraṃ padaṃ}
  \rdg[wit={P}]{paramapadaṃ}
  \rdg[wit={B}]{paraṃ pada}}/
paramapadāt
\app{\lem[wit={U1,U2}]{paramaṃ śūnyaṃ}
  \rdg[wit={B,P}]{paramaśūnyaṃ}
  \rdg[wit={L}]{paramaśūnya}}
\app{\lem[wit={E,U2},alt={śūnyān niraṃjanaṃ}]{śūnyān\skp{-}niraṃjanaṃ}
  \rdg[wit={B,L}]{śūnyā niraṃjanaṃ}
  \rdg[wit={P,U1}]{śūnyā niraṃjanaḥ}}/ 
%----------------------------
% anāmnaḥ paṃcaguṇāsteṣv  anutattvam      akhaṇḍatvam           avayavatvam ananyatvaṃ               ceti / \E
% anāmnaḥ paṃcaguṇāḥ     anutpannatvaṃ   akhaṇḍatvaṃ            anupamatvaṃ   ananyatvaṃ             ceti \P
% amnaḥ   paṃcaguṇāḥ/     anutpanatvaṃ/   akhaṃḍatvaṃ/                         anatvaṃ                cetiḥ/\B
% anāmnaḥ paṃcaguṇāḥ//     anutpanatvaṃ//  akhaṃḍatvaṃ//          anupamatvaṃ   anatvaṃ                ceti \L
% anāmnāḥ paṃcaguṇāḥ      anutpannatvaṃ   akhaṇḍatvaṃ ācalatvaṃ  anupamatvaṃ   ananyastvaṃ            ceti  \U1
% anāmnaḥ paṃcaguṇāḥ//     anutpannatvaṃ// akhaṃḍatvaṃ//          anupamatvaṃ// ananyatvaṃ nirmalatvaṃ ceti// \U2
%\om                                                                 \N1
%\om                                                                 \D
%\om                                                                 \N2
%-----------------------------
%Unbornness, indivisibility, immobility, unquealled and uniqueness.
%----------------------------
\app{\lem[wit={ceteri}]{anāmnaḥ}
  \rdg[wit={B}]{amnaḥ}}
\app{\lem[wit={ceteri}]{pañcaguṇāḥ}
  \rdg[wit={E}]{paṃcaguṇās}}/ 
\app{\lem[wit={ceteri}]{anutpannatvaṃ}
  \rdg[wit={E}]{teṣv anutattvam}}\dd{}
akhaṇḍatvaṃ\dd{}
\app{\lem[wit={U1}]{ācalatvaṃ}
  \rdg[wit={ceteri}]{\om}}\dd{}
\app{\lem[wit={U1}]{ācalatvaṃ}
  \rdg[wit={ceteri}]{\om}}\dd{}
\app{\lem[wit={ceteri}]{anupamatvaṃ}
  \rdg[wit={E}]{avayavatvam}
  \rdg[wit={B}]{\om}}\dd{}
\app{\lem[wit={E,P}]{ananyatvaṃ}
  \rdg[wit={U2}]{ananyatvaṃ nirmalatvaṃ}
  \rdg[wit={U1}]{ananyastvaṃ}
  \rdg[wit={B,L}]{anatvaṃ}}
\app{\lem[wit={E,P,L,U1,U2}]{ceti}
  \rdg[wit={B}]{cetiḥ}}/
%----------------------------
%tadaṣṭadalaṃ kamalaṃ hṛdaye tiṣṭhati/ \E
%\om \P
%\om \B
%\om \L
%\om                                                                 \N1
%\om                                                                 \D
%\om                                                                 \N2
%\om \U1
%\om \U2
%-----------------------------
%
%----------------------------
%te ubhaye hṛdaye tiṣṭhataḥ/ \E
%\om \P
%\om \B
%\om \L
%\om                                                                 \N1
%\om                                                                 \D
%\om                                                                 \N2
%\om \U1
%\om \U2
%-----------------------------
%
%----------------------------
%prathame dale śabdās tiṣṭhanti/ \E
%\om \P
%\om \B
%\om \L
%\om                                                                 \N1
%\om                                                                 \D
%\om                                                                 \N2
%\om \U1
%\om \U2
%-----------------------------
%
%----------------------------
%dvitīyadale sparśaḥ/ \E
%\om \B
%\om \L
%\om                                                                 \N1
%\om                                                                 \D
%\om                                                                 \N2
%\om \U1
%\om \U2
%-----------------------------
%
%----------------------------
%tṛtīye dale rūpaṃ tiṣṭhati/ \E
%\om \P
%\om \B
%\om \L
%\om                                                                 \N1
%\om                                                                 \D
%\om                                                                 \N2
%\om \U1
%\om \U2
%-----------------------------
%
%----------------------------
%caturthe dale rasastiṣṭhati/ \E
%\om \B
%\om \L
%\om                                                                 \N1
%\om                                                                 \D
%\om                                                                 \N2
%\om \U1
%\om \U2
%-----------------------------
%
%----------------------------
%paṃcame dale gandhaṃ tiṣṭhati/ \E
%\om \P
%\om \B
%\om \L
%\om                                                                 \N1
%\om                                                                 \D
%\om                                                                 \N2
%\om \U1
%\om \U2
%-----------------------------
%
%----------------------------
%paṣṭhadale cittaṃ tiṣṭhati/ \E
%\om \P
%\om \B
%\om \L
%\om                                                                 \N1
%\om                                                                 \D
%\om                                                                 \N2
%\om \U1
%\om \U2
%-----------------------------
%
%----------------------------
%saptame dale buddhis tiṣṭhati/ \E
%\om \P
%\om \B
%\om \L
%\om                                                                 \N1
%\om                                                                 \D
%\om                                                                 \N2
%\om \U1
%\om \U2
%-----------------------------
%
%----------------------------
%aṣṭame dale haṃkāras tiṣṭhati/ \E
%\om \P
%\om \B
%\om \L
%\om                                                                 \N1
%\om                                                                 \D
%\om                                                                 \N2
%\om \U1
%\om \U2
%-----------------------------
%
%----------------------------
%etad aṣṭadalamadhye pṛthivyākāro varttate/ \E
%\om \P
%\om \B
%\om \L
%\om                                                                 \N1
%\om                                                                 \D
%\om                                                                 \N2
%\om \U1
%\om \U2
%-----------------------------
%
%----------------------------
%atha ca tatkamalamadhye mukhaṃ tiṣṭhati/ \E
%\om \P
%\om \B
%\om \L
%\om                                                                 \N1
%\om                                                                 \D
%\om                                                                 \N2
%\om \U1
%\om \U2
%-----------------------------
%
%----------------------------
%asya kamalasya nādāt prakāśo bhavati// \E
%\om \P
%\om \B
%\om \L
%\om                                                                 \N1
%\om                                                                 \D
%\om                                                                 \N2
%\om \U1
%\om \U2
%-----------------------------
%
%----------------------------
%prakāśānaṃtaraṃ kamalam ūrdhvamukhaṃ bhavati/ \E
%\om \P
%\om \B
%\om \L
%\om                                                                 \N1
%\om                                                                 \D
%\om                                                                 \N2
%\om \U1
%\om \U2
%-----------------------------
%
%----------------------------
%tathā sūryaprakāśānantaraṃ tadā saromadhye kamalaṃ [p.66] vikasati/ \E
%\om \P
%\om \B
%\om \L
%\om                                                                 \N1
%\om                                                                 \D
%\om                                                                 \N2
%\om \U1
%\om \U2
%-----------------------------
%
%----------------------------
%tathedam apy ātmā prakāśānantaram ūrdhvamukhaṃ vikasati/ \E
%\om \P
%\om \B
%\om \L
%\om                                                                 \N1
%\om                                                                 \D
%\om                                                                 \N2
%\om \U1
%\om \U2
%-----------------------------
%
%----------------------------
%tanmadhye paramānandarūpā bhūmir bhavati/ \E
%\om \P
%\om \B
%\om \L
%\om                                                                 \N1
%\om                                                                 \D
%\om                                                                 \N2
%\om \U1
%\om \U2
%-----------------------------
%
%----------------------------
%tasyāhaṃ soham iti saṃjñā tasyā madhye svātmano dhyānād dine dine hy āyur varddhate/ \E
%\om \P
%\om \B
%\om \L
%\om \N1
%\om                                                                 \D
%\om \N2
%\om \U1
%\om \U2
%-----------------------------
%
%----------------------------
%rogo dūre bhavati// \E
%\om \P
%\om \B
%\om \L
%\om                                                                 \N1
%\om                                                                 \D
%\om                                                                 \N2
%\om \U1
%\om \U2
%-----------------------------
%
%----------------------------
%guṇāḥ kartṛtvaṃ jñātṛtvam abhyāsatvaṃ kalatvaṃ sarvajñatvaṃ prakāśasya guṇāḥ sakalaḥ niṣkalaḥ sarvaiḥ saha samatā viśrāṃtiḥ tata etādṛśam utpadyate/ \E
%\om \P
%\om \B
%\om \L
%\om                                                                 \N1
%\om                                                                 \D
%\om                                                                 \N2
%\om \U1
%\om \U2
%-----------------------------
%
%----------------------------
%ādyaḥ ātmā ātmana ākāśaḥ ākāśādvāyuḥ vāyostejaḥ tejaso jalaṃ jalāt pṛthvī/ \E
%\om \P
%\om \B
%\om \L
%\om                                                                 \N1
%\om                                                                 \D
%\om                                                                 \N2
%\om \U1
%\om \U2
%-----------------------------
%
%----------------------------
%atrātmanaḥ pañcaguṇāḥ agrāhyaḥ, anantaḥ, avācyaḥ, agocaraḥ,[p.67] aprameyaś ca ākāśasya pañcaguṇāḥ/ praveśaḥ niṣkramaṇaṃ, chiṃdraṃ, śabdādhāraḥ, bhrāṃtinilayatvam/ \E
%\om \P
%\om \B
%\om \L
%\om                                                                 \N1
%\om                                                                 \D
%\om                                                                 \N2
%\om \U1
%\om \U2
%-----------------------------
%
%----------------------------
%mahāvāyoḥ pañcaguṇāḥ/ calanaṃ śeṣasaṃcāraḥ, sparśaḥ, dhūmravarṇatā, tejaḥ saṃcaraḥ tejasaḥ pañcaguṇāḥ/ dahanaṃ, jvālarūpaṃ, uṣṇatā, rakto varṇaḥ// \E
%\om \P
%\om \B
%\om \L
%\om                                                                 \N1
%\om                                                                 \D
%\om                                                                 \N2
%\om \U1
%\om \U2
%-----------------------------
%
%----------------------------
%apāṃ paṃca guṇāḥ/ pravāhaḥ śithilatā dravaḥ madhuratā śvetavarṇaḥ/ pṛthivyāḥ paṃca guṇāḥ/ [p.68] sthūlatā sākāratā kaṭhinatā gandhavattā pītavarṇatā \E
%\om \P
%\om \B
%\om \L
%\om                                                                 \N1
%\om                                                                 \D
%\om                                                                 \N2
%\om \U1
%\om \U2
%-----------------------------
%
%---------------------------- 
%%%%%%%%%%%%%%%%%%%%%%%%%%%%check \B a few lines up!!!%%%%%%%%%% section \om in B and L%%%%%%%%%%%%%%%%%%%
%-----------------------------
\note[type=philcomm, labelb=333, lem={anupamatvaṃ}]{At this point of the text in witness E the passage which was previously omitted suddenly reappears. Since the order of the text is common to all manuscripts, we have to assume that the folios of the template of E were mistakenly swapped by the editor. Additionally five more sentences, which appear in later in the course of the text in all other witnesses are displaced in E. In this case, too, the arrangement of the sentences in witness E appears to be the result of an inadvertent transposition of the respective folios of the original manuscript. Thus, there are two gaps within E, in comparison to the other witnesses, which are patched together and relocated within E. All readings will be recorded in the critical apparatus with the proviso that they are arranged according to the textual structure found in all other witnesses.}
%This is the gap: anuparṇadalānām aṣṭadalānāṃ madhya ekaṃ kaṭhinaṃ bhavati | tadaṣṭadalaṃ kamalaṃ hṛdaye tiṣṭhati | te ubhaye hṛdaye tiṣṭhataḥ | prathame dale śabdās tiṣṭhanti | dvitīyadale sparśaḥ | tṛtīye dale rūpaṃ tiṣṭhati | caturthe dale rasas tiṣṭhati | paṃcame dale gandhaṃ tiṣṭhati | paṣṭhadale cittaṃ tiṣṭhati | saptame dale buddhis tiṣṭhati | aṣṭame dale haṃkāras tiṣṭhati | etad aṣṭadalamadhye pṛthivyākāro varttate | atha ca tatkamalamadhye mukhaṃ tiṣṭhati | asya kamalasya nādāt prakāśo bhavati | prakāśānaṃtaraṃ kamalam ūrdhvamukhaṃ bhavati | tathā sūryaprakāśānantaraṃ tadā saromadhye kamalaṃ vikasati | tathedam apy ātmā prakāśānantaram ūrdhvamukhaṃ vikasati | tanmadhye paramānandarūpā bhūmir bhavati | tasyāhaṃ soham iti saṃjñā tasyā madhye svātmano dhyānād dine dine hy āyur varddhate | rogo dūre bhavati | guṇāḥ kartṛtvaṃ jñātṛtvam abhyāsatvaṃ kalatvaṃ sarvajñatvaṃ prakāśasya guṇāḥ sakalaḥ niṣkalaḥ sarvaiḥ saha samatā viśrāṃtiḥ tata etādṛśam utpadyate | ādyaḥ ātmā ātmana ākāśaḥ ākāśād vāyuḥ vāyos tejaḥ tejaso jalaṃ jalāt pṛthvī | atrātmanaḥ pañcaguṇāḥ agrāhyaḥ anantaḥ avācyaḥ agocaraḥ aprameyaś ca ākāśasya pañcaguṇāḥ | praveśaḥ niṣkramaṇaṃ chiṃdraṃ śabdādhāraḥ bhrāṃtinilayatvam | mahāvāyoḥ pañcaguṇāḥ | calanaṃ śeṣasaṃcāraḥ, sparśaḥ, dhūmravarṇatā, tejaḥ saṃcaraḥ tejasaḥ pañcaguṇāḥ/ dahanaṃ, jvālarūpaṃ, uṣṇatā, rakto varṇaḥ || apāṃ paṃca guṇāḥ | pravāhaḥ śithilatā dravaḥ madhuratā śvetavarṇaḥ | pṛthivyāḥ paṃca guṇāḥ | sthūlatā sākāratā kaṭhinatā gandhavattā pītavarṇatā avayavatvam ananyatvaṃ ceti |}
%----------------------------
%parāvarasya paṃca guṇāḥ - niścalatvaṃ   niṣkarmatvaṃ paripūrṇaṃtvaṃ  vyāpakatvam  akalatvaṃ                  ceti/   \E
%                                        nirmalatvaṃ  paripūṇatvaṃ    vyāpakatvaṃ  akalatvaṃ                  ceti    \P
%parāvarasya paṃcaguṇāḥ/   niścalatvaṃ   nirmalatvaṃ/ paripūrṇatvaṃ/  vyāpakatvaṃ/ akalaṃtvaṃ                 ceti/   \B
%parāvarasya paṃcaguṇāḥ//  niścalatvaṃ   nirmalatvaṃ  paripūrṇatvaṃ   vyāpakatvaṃ  akalatvaṃ                  ceti    \L
%parāvarasya paṃcaguṇāḥ    niścalatvaṃ   nirmalatvaṃ  paripūrṇatvaṃ   vyāpakatvaṃ   prakāśatvaṃ                       \U1
%parāvarasya paṃcaguṇā//   niścalatvaṃ//              paripūrṇatvaṃ// vyāpakatvaṃ// akalatvaṃ// nirvikāratvaṃ ceti//  \U2  %425.jpg
%\om                                                                                                                                              \N1
%\om                                                                                                                                              \D
%\om                                                                                                                                              \N2
%-----------------------------   
%The five qualitiers of all-encompassing [are]: immobility, purity, completeness, pervasiveness, partlessness. 
%----------------------------
\app{\lem[wit={ceteri}]{parāvarasya}
  \rdg[wit={P}]{\om}}
\app{\lem[wit={ceteri}]{pañcaguṇāḥ}
  \rdg[wit={U2}]{paṃcaguṇā}
  \rdg[wit={P}]{\om}}/
\app{\lem[wit={ceteri}]{niścalatvaṃ}
  \rdg[wit={P}]{\om}}\dd{}
\app{\lem[wit={ceteri}]{nirmalatvaṃ}
  \rdg[wit={E}]{niṣkarmatvaṃ}
  \rdg[wit={U2}]{\om}}\dd{}
\app{\lem[wit={ceteri}]{paripūrṇatvaṃ}
  \rdg[wit={P}]{paripūṇatvaṃ}}\dd{}
vyāpakatvaṃ\dd{}
\app{\lem[wit={E,L,P}]{akalatvaṃ}
  \rdg[wit={B}]{akalaṃtvaṃ}
  \rdg[wit={U1}]{prakāśatvaṃ}
  \rdg[wit={U2}]{akalatvaṃ || nirvikāratvaṃ}}
\app{\lem[wit={ceteri}]{ceti}
  \rdg[wit={U1}]{\om}}/
%----------------------------
%paramapadasya paṃcaguṇāḥ   nityaṃ             nirantaraṃ             nirākāraṃ    nirniketanaṃ niścalatvaṃ ceti/  \E
%paramapadasya paṃcaguṇāḥ   nityaṃ             niraṃtaraṃ                          nirniketanaṃ             ceti   \P
%paramapadasya paṃcaguṇāḥ/  niś..              ....raṃga              nirākāraṃ    nirniketunaṃ             ceti/  \B
%paramapadasya paṃcaguṇāḥ// nitya--------------niraṃstaga-------------nirākāraṃ    nirviketunaṃ             ceti// \L  0033.jpg
%paramapadasya paṃcaguṇāḥ   nityānija----------niraṃtara--------------nirākāra     nimilaketanā            \U1
%paramapadasya paṃcaguṇāḥ// nityaṃ//           nirantarā//            nirākārā//   nirniketanaṃ//           ceti/  \U2  %425.jpg
%\om                                                                                             \N1
%\om                                                                                              \D
%\om                                                                                              \N2
%-----------------------------
%The five qualities of the supreme place [are]: permanent, immanent, identical, without form and without place. 
%----------------------------
\note[type=source, labelb=334, lem={paramapadasya pañcaguṇāḥ}]{SSP 1.17: niṣkalatvam aṇutaratvam acalatvam asaṃkhyatvam anādhāratvam iti pañcaguṇaṃ paramapadam ||1.17||}
\note[type=source, labelb=335, lem={paramapadasya pañcaguṇāḥ}]{Ysv\textsuperscript{PT}: ete pañcaguṇopetāḥ kathyante tadguṇaṃ yathā | nirguṇatvaṃ nirmalatvaṃ paripūrṇatvam eva ca | vyāpakatvaṃ kevalatvaṃ ānandasya guṇā iti | nirākāratvanityatvanijatvañ ca nirañjanam | nirṇiketanatā ceti tatpadasyeti tadguṇāḥ |}
paramapadasya pañcaguṇāḥ/
\app{\lem[wit={E,P,U2}]{nityaṃ}
  \rdg[wit={L}]{nitya°}
  \rdg[wit={U1}]{nityā°}
  \rdg[wit={B}]{niś..}}\dd{}
\app{\lem[type=emendation, resp=egoscr]{nijaṃ}
  \rdg[wit={U1}]{\korr °nija°}
  \rdg[wit={ceteri}]{\om}}\dd{}
\app{\lem[wit={E,P}]{niraṃtaraṃ}
  \rdg[wit={U2}]{nirantarā}
  \rdg[wit={U1}]{°niraṃtara°}
  \rdg[wit={L}]{°niraṃstaga°}
  \rdg[wit={B}]{°....raṃga°}}
\app{\lem[wit={B,E,L}]{nirākāraṃ}
  \rdg[wit={U1}]{nirākāra}
  \rdg[wit={U2}]{nirākārā}}\dd{}
\app{\lem[wit={B,P,U2}]{nirniketanaṃ}
  \rdg[wit={U1}]{nimilaketanā}
  \rdg[wit={E}]{nirniketanaṃ niścalatvaṃ}}
\app{\lem[wit={ceteri}]{ceti}
  \rdg[wit={U1}]{\om}}/
\end{prose}
\end{ekdosis}
\ekdpb*{}
%%%%%%%%%%%%%%%%%%%%%%%%%%%%%%%%%%%%%%%%%%
%%%%%%%%PAGEBREAK%%%%%%%PAGEBREAK%%%%%%%%%
%%%%%%%%%%%%%%%%%%%%%%%%%%%%%%%%%%%%%%%%%%
%%%%%%%%%%%%%%%%PAGEBREAK%%%%%%%%%%%%%%%%%
%%%%%%%%%%%%%%%%%%%%%%%%%%%%%%%%%%%%%%%%%%
%%%%%%%%PAGEBREAK%%%%%%%PAGEBREAK%%%%%%%%%
%%%%%%%%%%%%%%%%%%%%%%%%%%%%%%%%%%%%%%%%%%
%%%%%%%%%%%%%%%%%%%%%%%%%%%%%%%%%%%%%%%%%%
%%%%%%%%%%%%%%%%%%%%%%%%%%%%%%%%%%%%%%%%%%
%%%%%%%%%%%%%%%%%%%%%%%%%%%%%%%%%%%%%%%%%%
%%%%%%%%PAGEBREAK%%%%%%%PAGEBREAK%%%%%%%%%
%%%%%%%%%%%%%%%%%%%%%%%%%%%%%%%%%%%%%%%%%%
%%%%%%%%%%%%%%%%PAGEBREAK%%%%%%%%%%%%%%%%%
%%%%%%%%%%%%%%%%%%%%%%%%%%%%%%%%%%%%%%%%%%
%%%%%%%%PAGEBREAK%%%%%%%PAGEBREAK%%%%%%%%%
%%%%%%%%%%%%%%%%%%%%%%%%%%%%%%%%%%%%%%%%%%
%%%%%%%%%%%%%%%%%%%%%%%%%%%%%%%%%%%%%%%%%%
%%%%%%%%%%%%%%%%%%%%%%%%%%%%%%%%%%%%%%%%%%
%%%%%%%%%%%%%%%%%%%%%%%%%%%%%%%%%%%%%%%%%%
%%%%%%%%PAGEBREAK%%%%%%%PAGEBREAK%%%%%%%%%
%%%%%%%%%%%%%%%%%%%%%%%%%%%%%%%%%%%%%%%%%%
%%%%%%%%%%%%%%%%PAGEBREAK%%%%%%%%%%%%%%%%%
%%%%%%%%%%%%%%%%%%%%%%%%%%%%%%%%%%%%%%%%%%
%%%%%%%%PAGEBREAK%%%%%%%PAGEBREAK%%%%%%%%%
%%%%%%%%%%%%%%%%%%%%%%%%%%%%%%%%%%%%%%%%%%
  \begin{ekdosis}
\begin{prose}
%----------------------------
%śūnyasya   pañcaguṇāḥ – līnatā   ghūrṇatā   mūrchā   unmanībhāvaḥ   alasatvaṃ  ceti/  \E
%śunyasya   paṃcaguṇāḥ   līnatā   pūrṇatā    murchā   unmanībhāvaḥ   alasatvaṃ  ceti   \P
%śūnyasya   paṃcaguṇāḥ/  līnatāḥ  pūrṇatā    murchā   unmabhāvaḥ/    ālasyatvaṃ ceti/  \B
%śūnyasya   paṃcaguṇāḥ// līnatāḥ  pūrṇatā    murchā   unmanībhāvaḥ   ālasyatvaṃ ceti   \L  0033.jpg
%ti anasya  paṃcaguṇāḥ// līnatā// pūrṇatā//  mūrchā// unmanībhāva/   alasatvaṃ  ceti   \N1
%ti anyasya paṃcaguṇāḥ// līnatā// pūrṇatā//  mūrchā// unmanībhāva/   alasatvaṃ  ceti   \N2
%śūnyaḥsya  paṃcaguṇā    līnatā   pūrṇatā    mūrchā   unmanībhāva    alasatvaṃ         \U1
%śūnyasya   pañcaguṇāḥ// līnatā// ghūrṇatā// mūrchā// unmanībhāvaḥ// alasatvaṃ ceti//  \U2  %425.jpg
%\om D
%-----------------------------
%The five qualities of emptiness [are]: absorption, completeness, swooning, the state without mind and inactivity. 
%-----------------------------
\note[type=source, labelb=336, lem={śūnyasya pañcaguṇāḥ}]{Ysv\textsuperscript{PT}: līnatāśīrṇatāmūrcchātoyamaṇḍalatā iti | guṇāḥ pañca samākhyātāḥ śūnyasya paramasya vai |}
\note[type=source, labelb=337, lem={śūnyasya pañcaguṇāḥ}]{SSP 1.18: līnatā pūrṇatā unmanī lolatā mūrcchatā iti pañcaguṇaṃ śūnyam ||1.18||}
\note[type=philcomm, labelb=337+, lem={śūnyasya pañcaguṇāḥ}]{This point marks the end of the huge gap in witnesses N\textsubscript{1} and N\textsubscript{2}.}
\app{\lem[wit={B,E,L,U2}]{śūnyasya}
  \rdg[wit={P}]{śunyasya}
  \rdg[wit={U1}]{śūnyaḥsya}
  \rdg[wit={N1}]{ti anasya}
  \rdg[wit={N2}]{ti anyasya}}
\app{\lem[wit={ceteri}]{pañcaguṇāḥ}
  \rdg[wit={U1}]{paṃcaguṇā}}/
\app{\lem[wit={ceteri}]{līnatā}
  \rdg[wit={B,L}]{līnatāḥ}}\dd{}
\app{\lem[wit={ceteri}]{pūrṇatā}
  \rdg[wit={E,U2}]{ghūrṇatā}}\dd{}
\app{\lem[wit={ceteri}]{mūrchā}
  \rdg[wit={B,L,P}]{murchā}}\dd{}
\app{\lem[wit={E,P,L,U2}]{unmanībhāvaḥ}
  \rdg[wit={N1,N2,U1}]{unmanībhāva}
  \rdg[wit={B}]{unmabhāvaḥ}}\dd{}
\app{\lem[wit={ceteri}]{alasatvaṃ} %%%in the sense of inactivity!! 
  \rdg[wit={B,L}]{ālasyatvaṃ}}
\app{\lem[wit={ceteri}]{ceti}
  \rdg[wit={U1}]{\om}}/
%-----------------------------
%niraṃjanasya  paṃcaguṇāḥ   satyā  saha        bhāvā   sattā   svarūpatā samatā ceti/ \E
%niraṃjanasya  paṃcaguṇāḥ   satyaḥ sahaḥ    svabhāvaḥ  satta---svarūpatāḥ             \P
%niraṃjanasya  paṃcaguṇāḥ/  satyaḥ saha/    svabhāvaḥ  sata----svarūpatā/             \B
%niraṃjanasya  paṃcaguṇāḥ   satyaḥ saha---- svabhāvaḥ  sata----svarūpatā...           \L  0033.jpg
%niraṃjanasya  paṃcaguṇāḥ// satya/ sahaja/  svabhāva---sattā/  svarūpatā/             \N1
%niraṃjanasya/ paṃcaguṇāḥ// satya/ sahaja/  svabhāva---sattā/  svarūpatā/             \N2
%niraṃjanasya  paṃcaguṇāḥ   satya  sahaja-- svabhāva---sattā---svarūpatā              \U1
%niraṃjanasya  paṃcaguṇaḥ   satya// saha//  svabhāva// sattā// svarūpatā        ceti/ \U2  %425.jpg
%\om D
%-----------------------------
%The five qualities of the immacule [are]: truth, naturality, self-existence, beingness and peculiarity. 
%-----------------------------
\note[type=source, labelb=338, lem={niraṃjanasya  pañcaguṇāḥ}]{SSP 1.19: satyatvaṃ sahajatvaṃ samarasatvaṃ sāvadhānatvaṃ sarvagatvam iti pañcaguṇaṃ nirañjanam||1.19||}
\note[type=source, labelb=339, lem={niraṃjanasya  pañcaguṇāḥ}]{Ysv\textsuperscript{PT}: svabhāvaṃ sahajaṃ satyaṃ śāntiḥ śāntisvarūpataḥ | iti | nirañjanaguṇāḥ pañca etajjñānī maheśvaraḥ |}
niraṃjanasya
\app{\lem[wit={ceteri}]{pañcaguṇāḥ}
  \rdg[wit={U2}]{paṃcaguṇaḥ}}/
\app{\lem[wit={B,L,P}]{satyaḥ}
  \rdg[wit={N1,N2,U1,U2}]{satya}
  \rdg[wit={E}]{satyā}}\dd{}
\app{\lem[type=emendation, resp=egoscr]{sahajaḥ}
  \rdg[wit={N1,N2,U1}]{\korr sahaja}
  \rdg[wit={P}]{sahaḥ}
  \rdg[wit={E}]{saha°}
  \rdg[wit={B,L,U2}]{saha}}
\app{\lem[wit={B,L,P}]{svabhāvaḥ}
  \rdg[wit={N1,N2,U1,U2}]{svabhāva°}
  \rdg[wit={E}]{bhāvā}}\dd{}
\app{\lem[wit={ceteri}]{sattā}
  \rdg[wit={P}]{satta°}
  \rdg[wit={B,L}]{sata°}}\dd{}
\app{\lem[type=emendation, resp=egoscr]{svarūpataḥ}
  \rdg[wit={P}]{\korr svarūpatāḥ}
  \rdg[wit={E}]{svarūpatā samatā}
  \rdg[wit={ceteri}]{svarūpatā}}
\app{\lem[wit={E,U2}]{ceti}
  \rdg[wit={ceteri}]{\om}}\dd{}
\end{prose}
\end{ekdosis}
\ekdpb*{}
%%%%%%%%%%%%%%%%%%%%%%%%%%%%%%%%%%%%%%%%%%
%%%%%%%%PAGEBREAK%%%%%%%PAGEBREAK%%%%%%%%%
%%%%%%%%%%%%%%%%%%%%%%%%%%%%%%%%%%%%%%%%%%
%%%%%%%%%%%%%%%%PAGEBREAK%%%%%%%%%%%%%%%%%
%%%%%%%%%%%%%%%%%%%%%%%%%%%%%%%%%%%%%%%%%%
%%%%%%%%PAGEBREAK%%%%%%%PAGEBREAK%%%%%%%%%
%%%%%%%%%%%%%%%%%%%%%%%%%%%%%%%%%%%%%%%%%%
%%%%%%%%%%%%%%%%%%%%%%%%%%%%%%%%%%%%%%%%%%
%%%%%%%%%%%%%%%%%%%%%%%%%%%%%%%%%%%%%%%%%%
%%%%%%%%%%%%%%%%%%%%%%%%%%%%%%%%%%%%%%%%%%
%%%%%%%%PAGEBREAK%%%%%%%PAGEBREAK%%%%%%%%%
%%%%%%%%%%%%%%%%%%%%%%%%%%%%%%%%%%%%%%%%%%
%%%%%%%%%%%%%%%%PAGEBREAK%%%%%%%%%%%%%%%%%
%%%%%%%%%%%%%%%%%%%%%%%%%%%%%%%%%%%%%%%%%%
%%%%%%%%PAGEBREAK%%%%%%%PAGEBREAK%%%%%%%%%
%%%%%%%%%%%%%%%%%%%%%%%%%%%%%%%%%%%%%%%%%%
%%%%%%%%%%%%%%%%%%%%%%%%%%%%%%%%%%%%%%%%%%
%%%%%%%%%%%%%%%%%%%%%%%%%%%%%%%%%%%%%%%%%%
%%%%%%%%%%%%%%%%%%%%%%%%%%%%%%%%%%%%%%%%%%
%%%%%%%%PAGEBREAK%%%%%%%PAGEBREAK%%%%%%%%%
%%%%%%%%%%%%%%%%%%%%%%%%%%%%%%%%%%%%%%%%%%
%%%%%%%%%%%%%%%%PAGEBREAK%%%%%%%%%%%%%%%%%
%%%%%%%%%%%%%%%%%%%%%%%%%%%%%%%%%%%%%%%%%%
%%%%%%%%PAGEBREAK%%%%%%%PAGEBREAK%%%%%%%%%
%%%%%%%%%%%%%%%%%%%%%%%%%%%%%%%%%%%%%%%%%%
\begin{ekdosis}
  \ekddiv{type=ed}
  \bigskip
    \centerline{\textrm{\small{[Generation of the Body]}}}
    \bigskip
    \begin{prose}
      \noindent
%-----------------------------
%idānīṃ piṃḍotpattiḥ kathyate//[p.69] \E
%idānīṃ piṃḍotpattiḥ kathyate  \P
%idānīṃ piṃḍotpattiṃ kathyate/ \B
%idānīṃ piṃḍotpattiṃ kathyate/ \L
%idānīṃ piṃḍotpattiḥ kathyate/ \N1
%idānīṃ piṃḍotpatti  kathyate// \N2 %%%%S.15 of 20 transcription stopped here 
%\om                                                                 \D
%idānīṃ piṃḍotpatti  kathyate   \U1
%idānīṃ piṃḍotpattiḥ kathyate  \U2
%-----------------------------
%Now the generation of the body is taught. 
%-----------------------------
\note[type=source, labelb=340, lem={piṃḍotpattiḥ}]{Ysv\textsuperscript{PT}: vidyotpattis tadānīn tu kathyate śṛṇu yatnataḥ | ānandaparamātmeti paramānanda ekataḥ | prabodhaparamānandacittotpattiprabodhavān | cidudayāt prakāśaś ca eṣāṃ pañca tathaiva ca |}
idānīṃ
\app{\lem[wit={ceteri}]{piṇḍotpattiḥ}
  \rdg[wit={N2,U1}]{piṃḍotpatti}
  \rdg[wit={B,L}]{piṃḍotpattiṃ}}
kathyate/ 
%-----------------------------
%anāditaḥ   paramātmā   paramātmanaḥ   paramānaṃdaḥ   paramānaṃdāt prabodhaḥ   prabodhāc cidudayaḥ   cidudayāt          prakāśaḥ/ \E
%anāditaḥ   paramātmā   paramātmanaḥ   paramānaṃdaḥ   paramānaṃdāt prabodhaḥ   prabodhāc cidudayaḥ   vidudayāt          prakāśaḥ \P %%%7670.jpg 
%anāditaḥ   paramātmā   paramātmanaḥ/  paramānaṃdaḥ/  paramānaṃdāt prabodhaḥ   prabodhāc cidudaya----viduyāt            prakāśaḥ/ \B
%anāditaḥ   paramātmā   paramātmanaḥ   paramānaṃdaḥ   paramānaṃdāt prabodhaḥ   prabodhāc cidudaya    cidudayāt          prakāśaḥ... \L
%anāditaḥ   paramātmā/  paramātmanaḥ   paramānaṃdaḥ/  paramānaṃdāt prabodhaḥ/  prabodhāc ciddayaḥ    ciddayacidudayāt   prakāśaḥ// \N1
%anāditaḥ   paramātmā/  paramātmanaḥ   paramānaṃdaḥ/  paramānaṃdāt prabodhaḥ/  prabodhāc ciddayaḥ    cidudayacidudayāt  prakāśaḥ \N2
%anāditaḥ   paramātmā   paramātmanaḥ   paramānaṃdaḥ   paramānaṃdāt prabodhaḥ             cittayaḥ    citta--------------prakāśaḥ  \U1
%anāditaḥ// paramātmā// paramātmanaḥ// paramānaṃdaḥ// paramānaṃdāt prabodhaḥ// prabodhā  cidudayaḥ// cidudayāt          prakāśaḥ// \U2 XX
%\om                                                                 \D
%-----------------------------
%From without beginning the supreme self, from the supreme self supreme bliss, from supreme bliss awakening, from awakening manifestation of spirit/consciousness, from manifestation of spirit light. 
%-----------------------------
anāditaḥ paramātmā/
paramātmanaḥ paramānandaḥ/
paramānaṃdāt-prabodhaḥ/ \app{\lem[wit={ceteri},alt={prabodhāc}]{prabodhā\skp{c-ci}}
  \rdg[wit={U2}]{prabodhā}
  \rdg[wit={U1}]{\om}}\app{\lem[wit={E,P,U2},alt={cidudayaḥ}]{\skm{c-ci}dudayaḥ}
  \rdg[wit={B,L}]{cidudaya°}
  \rdg[wit={U1}]{cittayaḥ}
  \rdg[wit={N1,N2}]{ciddayaḥ}}/
\app{\lem[wit={E,L,U2},alt={cidudayāt}]{cidudayā\skp{t-pra}}
  \rdg[wit={P}]{vidudayāt}
  \rdg[wit={B}]{viduyāt}
  \rdg[wit={N1}]{ciddayacidudayāt}
  \rdg[wit={N2}]{cidudayacidudayāt}
  \rdg[wit={U1}]{citta°}}\skm{t-pra}kāśaḥ/
%-----------------------------
%tatra paramātmanaḥ   paṃcaguṇāḥ –  akṣayaḥ,             abhedyaḥ,  acchedyaḥ,  adāhyaḥ, avināśī/ \E
%tatra paramātmanaḥ   paṃcaguṇāḥ    akṣayaḥ              abhedyaḥ              aṣṭadyaḥ avināśī   \P
%tatra paramātmanaḥ   paṃcaguṇāḥ/   akṣayaḥ/             abhedyaḥ/  avināśī/   adāhyaḥ/  \B
%tatra paramātmanaḥ   paṃcaguṇāḥ//  akṣayaḥ/             abhedyaḥ/  avināśī/   adāhyaḥ/  \L
%tatra paramātmanaḥ   paṃcaguṇāḥ//  akṣayaḥ/             abhedyaḥ/  acchedyaḥ/ adāhyaḥ/ avināśī/ // \N1
%tatra paramātmanaḥ   paṃcaguṇāḥ//  akṣayaḥ              abhedyaḥ   acchedyaḥ  adāhyaḥ  avināśī // \N2
%\om                                                                 \D
%tatra paramātmanaḥ   paṃcaguṇāḥ    akṣayyaḥ  avadyha    abhedyaḥ               ādṛṣya  avināsī /  \U1
%tatra paramātmanaḥ// paṃcaguṇāḥ//  akṣayaḥ//            abhedyaḥ// achedyaḥ/  adāhyaḥ  avināsaḥ// \U2
%-----------------------------
%There [are] the five qualities of the supreme self: imperishable, indivisible, uncuttable, unburnable, indestructible. 
%-----------------------------
\note[type=source, labelb=340, lem={paramātmanaḥ pañcaguṇāḥ}]{Ysv\textsuperscript{PT}: avināśyo 'kṣayo 'bhedo 'dāhyo hyakhādya eva ca | ete pañca guṇāḥ proktā anādo nādavairiṇā | kiraṇasphūrttivisphūrttiharṣavat paramātmanā | tetu pañca prakāreṇa guṇāḥ pañca prakīrttitāḥ |}
tatra paramātmanaḥ pañcaguṇāḥ/
\app{\lem[wit={ceteri}]{akṣayaḥ}
  \rdg[wit={U1}]{akṣayyaḥ avadyha}}\dd{}
abhedyaḥ\dd{}
\app{\lem[wit={ceteri}]{acchedyaḥ}
  \rdg[wit={B,L}]{avināśī}
  \rdg[wit={P,U1}]{\om}}\dd{}
\app{\lem[wit={ceteri}]{adāhyaḥ}
  \rdg[wit={P}]{aṣṭadyaḥ}
  \rdg[wit={U1}]{ādṛṣya}}
\app{\lem[wit={ceteri}]{avināśī}
  \rdg[wit={U1}]{avināsī}
  \rdg[wit={U2}]{avināsaḥ}
  \rdg[wit={B,L}]{\om}}\dd{}
%-----------------------------
%paramānaṃdasya paṃcaguṇāḥ – sphuraṇaḥ, kiraṇaḥ, visphuraṇaḥ, ahaṃtā,  harṣavattvam/ \E
%paramānaṃdasya paṃcaguṇā    sphuraṇaḥ  kiraṇaḥ  visphuraṇaḥ  ahaṃtā   harṣatatvaṃ \P
%paramānaṃdasya paṃcaguṇāḥ/  sphuraṇa---kiraṇa   visphuriṇa   ahaṃtā   harṣavatvaṃ/  \B
%paramānaṃdasya paṃcaguṇāḥ/  sphuraṇa/  kiraṇā// visphura/    ahaṃtā// harṣavatvaṃ/  \L
%paramānaṃdasya paṃcaguṇāḥ// sphuraṇa---kiraṇa   visphuriṇa   ahaṃtā   harṣavatvaṃ/ \N1
%paramānaṃdasya paṃcaguṇāḥ// sphuraṇa/  kiraṇa   visphura     ahaṃtā   harṣavatvaṃ/ \N2
%paramānaṃdasya paṃcaguṇāḥ   sphuraṇaḥ  kiraṇaḥ  visphuraḥ    ahaṃtā   dhairyatva \U1 
%paramānaṃdasya paṃcaguṇāḥ// sphuraṇa// kiraṇa// visphuraṇa// ahaṃtā// harṣavārttvaṃ// \U2
%\om                                                                 \D
%-----------------------------
%The five qualities of the supreme bliss [are]: vibration, beam of light, quiver, I-ness, joyful excitement. 
%-----------------------------
\note[type=source, labelb=341, lem={paramānandasya pañcaguṇāḥ}]{Ysv\textsuperscript{PT}: kiraṇasphūrttivisphūrttiharṣavat paramātmanā | tetu pañca prakāreṇa guṇāḥ pañca prakīrttitāḥ |}
paramānaṃdasya
\app{\lem[wit={ceteri}]{pañcaguṇāḥ}
  \rdg[wit={P}]{paṃcaguṇā}}/
\app{\lem[wit={E,P,U1}]{sphuraṇaḥ}
  \rdg[wit={ceteri}]{sphuraṇa}}\dd{}
\app{\lem[wit={E,P,U1}]{kiraṇaḥ}
  \rdg[wit={ceteri}]{kiraṇa}}\dd{}
\app{\lem[wit={E,P}]{visphuraṇaḥ}
  \rdg[wit={U1}]{visphuraḥ}
  \rdg[wit={B,N1}]{visphuriṇa}
  \rdg[wit={L,N2,U1}]{visphura}}\dd{}
ahaṃtā\dd{}
\app{\lem[wit={E}]{harṣavattvam}
  \rdg[wit={B,L,P,N1,N2}]{harṣavatvaṃ}
  \rdg[wit={U2}]{harṣavārttvaṃ}
  \rdg[wit={U1}]{hairyatva}}\dd{}
%-----------------------------
%prabodhasya paṃca guṇāḥ –layaḥ,  ullāsaḥ,  vibhāsaḥ,  vicāraḥ,  prabhā/   \E
%prabodhasya paṃcaguṇāḥ   layaḥ   ullāsā    vibhāsā    vicāraḥ   prabhā    \P
%prabodhasya paṃcaguṇāḥ/  layā    ullāsā    vibhāsā    vicāraḥ/  abhā         \B
%prabodhasya paṃcaguṇāḥ/  laya/   ullāsa/   vibhāsa/   vicāra/   prabhā/   \L
%prabodhasya paṃcaguṇāḥ/  laya/   ullāsā    vibhāsā    vicāraḥ/           \N1
%prabodhasya paṃcaguṇāḥ   laya/   ullāsā/   vibhāsā/   vicāra             \N2
%\om                                                                      \D
%   bodhasya paṃcaguṇāḥ           ullāsā    vibhāsā    vicāra    samādhi     \U1 %%%296 Ende.... weiter mit 297!!
%prabodhasya paṃcaguṇāḥ// layaḥ// ullāsaḥ// vibhāsaḥ// vicāraḥ// prabhā// \U2 
%-----------------------------
%The five qualities of awakening [are]: absoprtion, becoming visible, light, roving?, light.
%-----------------------------
\note[type=source, labelb=342, lem={prabodhasya paṃcaguṇāḥ}]{Ysv\textsuperscript{PT}: vicāraś ca prabhollāsā vibhāvaś ca layas tathā | prabodhasya guṇāḥ pañca kīrttyante tena hetunā |}
\app{\lem[wit={ceteri}]{prabodhasya}
  \rdg[wit={U1}]{bodhasya}} paṃcaguṇāḥ/
\app{\lem[wit={E,P,U2}]{layaḥ}
  \rdg[wit={L,N1,N2}]{laya}
  \rdg[wit={B}]{layā}
  \rdg[wit={U1}]{\om}}\dd{}
\app{\lem[wit={E,U2}]{ullāsaḥ}
  \rdg[wit={ceteri}]{ullāsā}}\dd{}
\app{\lem[wit={E,U2}]{vibhāsaḥ}
  \rdg[wit={ceteri}]{vibhāsā}}\dd{}
\app{\lem[wit={B,E,P,N1,U2}]{vicāraḥ}
  \rdg[wit={L,N2,U1}]{vicāra}}
\app{\lem[wit={E,P,U2}]{prabhā}
  \rdg[wit={B}]{abhā}
  \rdg[wit={U1}]{samādhi}
  \rdg[wit={N1,N2}]{\om}}
\app{\lem[wit={E,P,N1,N2,U2}]{cidudayasya}
  \rdg[wit={U1}]{udadayasya}
  \rdg[wit={L}]{cidudadayasya}
  \rdg[wit={B}]{vihṛdayasya}}
\app{\lem[wit={ceteri}]{pañcaguṇāḥ}
  \rdg[wit={P,U2}]{paṃcaguṇā}}/
%----------------------------
%       cidudayasya    paṃcaguṇāḥ    kartṛtvaṃ   jñātṛtvam  abhyāsatvaṃ   kalatvaṃ     sarvajñatvaṃ    \E %TRANSPOSED
%       cidudayasya    paṃcaguṇā     kartṛtve    jñātṛtvaṃ  abhyāsatvaṃ   kalanaṃtvaṃ  sarvajñatvaṃ    \P
%       vihṛdayasya    paṃcaguṇāḥ//  katutvaṃ    jñātṛtvaṃ  abhyāsatvaṃ/  kalanatvaṃ   saṃvajñatvaṃ/   \B
%       cidudadayasya  paṃcaguṇāḥ// akartutvaṃ   jñātṛtvaṃ  abhyāsatvaṃ   kalanatvaṃ   saṃvajñatvaṃ/   \L
%       cidudayasya    paṃcaguṇāḥ/   kartṛtvaṃ/  jñātṛtvaṃ/ abhyāsatvaṃ/  kalanaṃtvaṃ/ sarvajñatvaṃ    \N1
%       cidudayasya    paṃcaguṇāḥ/   kartṛtvaṃ/  jñātvaṃ/     ...satvaṃ/  kalanātvaṃ/  sarvajñatvaṃ//  \N2
%\om                                                                                               \D
%       udadayasya     paṃcaguṇāḥ    katṛtvaṃ    jñānatvaṃ   abhyāsatvaṃ  kalyana------sarvaśatvaṃ     \U1
%       cidudayasya    paṃcaguṇā//   kartṛtvaṃ// jñātṛtvaṃ// abhāsatvaṃ// kalanatvaṃ// sarvajñatvaṃ//  \U2
%-----------------------------
%The five qualities of manifestation of spirit [are]: creatorship, knowership, accumulation of skills-ship,  
%----------------------------
\note[type=philcomm, labelb=341, lem={cidudayasya pañcaguṇāḥ}]{Immediately after the words \textit{cidudayasya paṃca°} there is a gap within witness E in comparison to all other witnesses. This gap however, as it turned out, is simply displaced due to conflation of folios of the template of witness E. Thus, I included its readings in the critical edition according to the original structure of the text as preserved in all other witnesses, instead of abandoning all of its readings in a footnote.}
\note[type=source, labelb=342, lem={cidudayasya pañcaguṇāḥ}]{Ysv\textsuperscript{PT}: abhyāsakartṛkamanāḥ sarvatattvaprabhā tathā | cidudayasya pañceti guṇā jñeyā viśeṣataḥ |}
\app{\lem[wit={ceteri}]{cidudayasya}
  \rdg[wit={U1}]{udadayasya}
  \rdg[wit={L}]{cidudadayasya}
  \rdg[wit={B}]{vihṛdayasya}
  \rdg[wit={U1}]{udadayasya}}
\app{\lem[wit={ceteri}]{pañcaguṇāḥ}
  \rdg[wit={P,U2}]{paṃcaguṇā}}/
%----------------------------
%prakāśasya      guṇāḥ   sakalaḥ niṣkalaḥ  sarvaiḥ saha  samatā   viśrāṃtiḥ   tata             etādṛśam         utpadyate/   \E %TRANSPOSED  
%prakāśasya paṃcaguṇāḥ   sakala  niṣkvalā  saṃbodhanā    samatā   viśrāṃtiḥ   tat              etādṛśaṃ  jñānam utpadyate//  \P
%prakāśasya paṃcaguṇāḥ/  sakala/ niṣkvala/ saṃbodhana----samatā   viśrāṃti/   tat              etādraśaṃ jñānaṃ mutpadyate/  \B
%prakāśasya paṃcaguṇāḥ// sakala/ niṣkvala/ saṃbodhana----samatā   viśrāṃti    tat              etādraśaṃ jñānam utpadyate//  \L
%prakāśasya paṃcaguṇāḥ/  sakala/ nikala/   saṃbodhana/   samatā/  viśrāṃti/   tata             etādṛśaṃ  jñānam upadyate/    \N1
%prakāśasya paṃcaguṇāḥ// sakala/ nikala/   saṃbodhana/   samaṃtā/ viśrāṃti//  tata             etādṛśaṃ  jñānam upadyate/    \N2
%\om                                                                                                                         \D
%prakāśasya paṃcaguṇāḥ   sakalā            saṃbodhana----samatā   viśrāṃti    tataḥ ***viśvāsa etādṛśaṃ  jñānam utpadyate*** \U1 %***INSERTION***
%prakāśasya paṃcaguṇaḥ// sakalā  tidvasā   saṃbodhanaṃ// samatā// viśrāṃtiḥ// tataḥ//          etādṛśyaṃ jñānam utpadyate//  \U2
%-----------------------------
%The five qualities of light [are]: consisting of parts, not consisting of parts, recognition, uniformity, tranquility. Because of that reliable knowledge is generated. 
%----------------------------
\note[type=source, labelb=343, lem={prakāśasya paṃcaguṇāḥ}]{Ysv\textsuperscript{PT}: bodhanaṃ samayatvañ ca vismṛtiḥ sakalaprabhā | prakāśasya guṇāḥ pañcacaite jñānakarāḥ śubhāḥ | etaj jñāne tataś caiṣāṃ jñānam utpadyate mahat |}
prakāśasya
\app{\lem[wit={ceteri}]{pañcaguṇāḥ}
  \rdg[wit={E}]{guṇāḥ}}/
\app{\lem[wit={E}]{sakalaḥ}
  \rdg[wit={U1,U2}]{sakalā}
  \rdg[wit={ceteri}]{sakala}}\dd{}
\app{\lem[wit={E}]{niṣkalaḥ}
  \rdg[wit={P}]{niṣkvalā}
  \rdg[wit={B,L}]{niṣkvala}
  \rdg[wit={N2}]{nikala}
  \rdg[wit={U2}]{tidvasā}
  \rdg[wit={U1}]{\om}}\dd{}
\app{\lem[wit={U2}]{saṃbodhanaṃ}
  \rdg[wit={P}]{saṃbodhanā}
  \rdg[wit={E}]{sarvaiḥ saha}
  \rdg[wit={ceteri}]{saṃbodhana}}
\app{\lem[wit={ceteri}]{samatā}
  \rdg[wit={N2}]{samaṃtā}}\dd{}
\app{\lem[wit={E,P,U2}]{viśrāṃtiḥ}
  \rdg[wit={ceteri}]{viśrāṃti}}\dd{}
\app{\lem[wit={E,N1,N2}]{tata}
  \rdg[wit={B,L,P}]{tat}
  \rdg[wit={U1,U2}]{tataḥ}}
\app{\lem[wit={ceteri}]{etādṛśaṃ}
  \rdg[wit={B,L}]{etādraśaṃ}
  \rdg[wit={U2}]{etādṛśyaṃ}}
\app{\lem[wit={ceteri},alt={jñānam}]{jñāna\skp{m-u}}
  \rdg[wit={E}]{\om}
}\skm{m-u}tpadyate/
%----------------------------
%ādyaḥ   ātmā ātmana      ākāśaḥ    ākāśād           vāyuḥ   vāyos   tejaḥ   tejaso jalaṃ  jalāt     pṛthvī / atrātmanaḥ pañcaguṇāḥ       agrāhyaḥ   anantaḥ   avācyaḥ  agocaraḥ   aprameyaś ca   \E %TRANSPOSED 
%ādyaḥ// ādhyād ātmā      ātmana    ākāśaḥ   ākāśād  vayuḥ   vāyos   tejaḥ   tejaso jalaṃ  jalāt     pṛthivī  ātrātmanaḥ paṃcaguṇāḥ       agrāhyaḥ   anaṃtaḥ   avācyaḥ  agocaraḥ   aprameyaś ca   \P
%ādyaḥ/  ādhyāt manaṃ/    ākāśaḥ/   ākāśād           vayoḥ/  vāyos   tejaḥ/  tejaso jalaṃ/ jalāt     pṛthvī/  ādyātmanaḥ paṃcaguṇāḥ/                                                             \B
%ādyaḥ// ādhyāt manaḥ/    ākāśaḥ/   ākāśād           vayuḥ/  vāyos   tejaḥ   tejaso jalaṃ  jalāt     pṛthvī   ādyātmanaḥ paṃcaguṇāḥ       agrāhya    anaṃtaḥ   avācyaḥ  agocaraḥ   aprameyaś ca   \L
%ādyaḥ/  ādhyād ātmā/     ātmanaḥ/  ākāśaḥ/  ākāśat  yavanḥ   pavanāt tejaḥ/  tejaso--------dakāt     pṛthvī/  tatra ātmanaḥ/ paṃcaguṇāḥ/  agrāhyaḥ/  anaṃtaḥ/  avācyaḥ/ agocaraḥ/  aprameyaś ca/  \N1
%ādya/   ādhyād ātmā/     ātmana/   ākāśa/   ākāśat  yavak?   pavanāt tejaḥ/  tejaso        udakāt    pṛthvī// tatrātmanaḥ paṃcaguṇāḥ//    agrāhya/   anaṃtaḥ/  avācya/  agocaraḥ/  aprameyaś ca/  \N2
%                         ātmanaḥ   ākāśaḥ   ākāśāt  pavanaḥ  pavanāt tejaḥ   tejaḥ sa udakaṃ udakāt  pṛthvī   tatra ātmanaḥ paṃcaguṇāḥ    agrāhyaḥ   anaṃtaḥ   avācyaḥ  agocaraḥ   aprameyaś ca   \U1
%ādyaḥ// ādhyā// dātmā//  ātmanaḥ// ākāśaḥ// ākāśād  vayuḥ//  vāyos   tejaḥ// tejasor jalaṃ//   jalāt pṛthvī// ātmanaḥ paṃcaguṇaḥ//        agrāhyaḥ// anaṃtaḥ// avācyā// agocaraḥ// aprameyaś ca// \U2
%\om                                                                                                                                                                                        \D
%-----------------------------
%It is unparalelled. From of being unparalled self arises. Because there is a self, space arises. Because there is space, wind arises. From wind light arises. Because there is light water arises. From water the world arises. The self has five qualities: not tangible, infinite, unexpressable, not in the range of ordinary existance and unfathomable.
%-----------------------------
\note[type=source, labelb=344, lem={ādyaḥ}]{Ysv\textsuperscript{PT}: ākāśāt pavano vāyos tejas tejasa eva ca | jalaṃ jalāt tathā pṛthvī eṣāṃ pañcaguṇās tathā | agocarād vayānantagrāhyam eṣāṃ tathātmanaḥ |}
\app{\lem[wit={ceteri}]{ādyaḥ}
  \rdg[wit={N2}]{adya}
  \rdg[wit={U1}]{\om}}\dd{}
\app{\lem[wit={ceteri},alt={ādhyād}]{ādhyā\skp{d-ā}}
  \rdg[wit={E}]{ātmā}
  \rdg[wit={U2}]{ādhyā}
\rdg[wit={U1}]{\om}
}\app{\lem[wit={P,N1,N2},alt={ātmā}]{\skm{d-ā}tmā}
  \rdg[wit={E}]{ātmana}
  \rdg[wit={U2}]{dātmā}
  \rdg[wit={B}]{manaṃ}
  \rdg[wit={L}]{manaḥ}}\dd{}
\end{prose}
\end{ekdosis}
\ekdpb*{}
%%%%%%%%%%%%%%%%%%%%%%%%%%%%%%%%%%%%%%%%%%
%%%%%%%%PAGEBREAK%%%%%%%PAGEBREAK%%%%%%%%%
%%%%%%%%%%%%%%%%%%%%%%%%%%%%%%%%%%%%%%%%%%
%%%%%%%%%%%%%%%%PAGEBREAK%%%%%%%%%%%%%%%%%
%%%%%%%%%%%%%%%%%%%%%%%%%%%%%%%%%%%%%%%%%%
%%%%%%%%PAGEBREAK%%%%%%%PAGEBREAK%%%%%%%%%
%%%%%%%%%%%%%%%%%%%%%%%%%%%%%%%%%%%%%%%%%%
%%%%%%%%%%%%%%%%%%%%%%%%%%%%%%%%%%%%%%%%%%
%%%%%%%%%%%%%%%%%%%%%%%%%%%%%%%%%%%%%%%%%%
%%%%%%%%%%%%%%%%%%%%%%%%%%%%%%%%%%%%%%%%%%
%%%%%%%%PAGEBREAK%%%%%%%PAGEBREAK%%%%%%%%%
%%%%%%%%%%%%%%%%%%%%%%%%%%%%%%%%%%%%%%%%%%
%%%%%%%%%%%%%%%%PAGEBREAK%%%%%%%%%%%%%%%%%
%%%%%%%%%%%%%%%%%%%%%%%%%%%%%%%%%%%%%%%%%%
%%%%%%%%PAGEBREAK%%%%%%%PAGEBREAK%%%%%%%%%
%%%%%%%%%%%%%%%%%%%%%%%%%%%%%%%%%%%%%%%%%%
%%%%%%%%%%%%%%%%%%%%%%%%%%%%%%%%%%%%%%%%%%
%%%%%%%%%%%%%%%%%%%%%%%%%%%%%%%%%%%%%%%%%%
%%%%%%%%%%%%%%%%%%%%%%%%%%%%%%%%%%%%%%%%%%
%%%%%%%%PAGEBREAK%%%%%%%PAGEBREAK%%%%%%%%%
%%%%%%%%%%%%%%%%%%%%%%%%%%%%%%%%%%%%%%%%%%
%%%%%%%%%%%%%%%%PAGEBREAK%%%%%%%%%%%%%%%%%
%%%%%%%%%%%%%%%%%%%%%%%%%%%%%%%%%%%%%%%%%%
%%%%%%%%PAGEBREAK%%%%%%%PAGEBREAK%%%%%%%%%
%%%%%%%%%%%%%%%%%%%%%%%%%%%%%%%%%%%%%%%%%%
  \begin{ekdosis}
    \begin{prose}
      \noindent
\app{\lem[wit={N1,U1,U2}]{ātmanaḥ}
    \rdg[wit={P,N2}]{ātmana}
    \rdg[wit={B,E,L}]{ākāśaḥ}}
\app{\lem[wit={P,N1,U1,U2}]{ākāśaḥ}
    \rdg[wit={E,B,L}]{ākāśād}
    \rdg[wit={N2}]{ākāśa}}\dd{}
\app{\lem[wit={P,U2},alt={ākāśād}]{ākāśā\skp{d-va}}
  \rdg[wit={N1,N2,U1}]{ākāśāt}
  \rdg[wit={N2}]{ākāśa}
  \rdg[wit={E,B,L}]{\om}
}\app{\lem[wit={E},alt={vayuḥ}]{\skm{d-va}yuḥ}
  \rdg[wit={L,P,U2}]{vayuḥ}
  \rdg[wit={B}]{vayoḥ}
  \rdg[wit={U1}]{pavanaḥ}
  \rdg[wit={N2}]{yavak}
  \rdg[wit={N1}]{yavanḥ}}\dd{}
\app{\lem[wit={ceteri},alt={vāyos}]{vāyo\skp{s-te}}
  \rdg[wit={N1,N2,U1}]{pavanāt}}\skm{s-te}jaḥ\dd{}
\app{\lem[wit={ceteri}]{tejaso}
  \rdg[wit={U1}]{tejaḥ sa}
  \rdg[wit={U2}]{tejasor}}
\app{\lem[wit={ceteri}]{jalaṃ}
  \rdg[wit={U1}]{udakaṃ}
  \rdg[wit={N1,N2}]{\om}}\dd{}
\app{\lem[wit={ceteri},alt={jalāt}]{jalā\skp{t-pṛ}}
  \rdg[wit={U1,N2}]{udakāt}
  \rdg[wit={N1}]{°dakāt}}
\app{\lem[wit={ceteri},alt={pṛthvī}]{\skp{t-pṛ}thvī}
  \rdg[wit={P}]{pṛthivī}}\dd{}
\app{\lem[wit={N2}]{tatrātmanaḥ}
  \rdg[wit={N1,U1}]{tatra ātmanaḥ}
  \rdg[wit={E}]{atrātmanaḥ}
  \rdg[wit={P}]{ātrātmanaḥ}
  \rdg[wit={B,L}]{ādyātmanaḥ}
  \rdg[wit={U2}]{ātmanaḥ}}
\app{\lem[wit={ceteri}]{pañcaguṇāḥ}
  \rdg[wit={U2}]{paṃcaguṇaḥ}} 
\app{\lem[wit={ceteri}]{agrāhyaḥ}
  \rdg[wit={L,N2}]{agrāhya}
  \rdg[wit={B}]{\om}}\dd{}
anantaḥ\dd{}
\app{\lem[wit={ceteri}]{avācyaḥ}
  \rdg[wit={N2}]{avācya}
  \rdg[wit={U2}]{avācyā}
  \rdg[wit={B}]{\om}}
\app{\lem[wit={ceteri}]{agocaraḥ}
  \rdg[wit={B}]{\om}}\dd{}
\app{\lem[wit={ceteri},alt={aprameyaś ca}]{aprameyaś\skp{-}ca}
  \rdg[wit={B}]{\om}}\dd{}
%----------------------------
%ākāśasya pañcaguṇāḥ/  praveśaḥ niṣkramaṇaṃ   chiṃdraṃ   śabdādhāraḥ   bhrāṃtinilayatvam/    \E %TRANSPOSED
%ākāśasya paṃcaguṇāḥ   praveśaḥ niśkrumāṇaṃ   chidraṃ    śabdadhāraḥ   bhrāṃtinilayatvaṃ     \P
%                      preveśaḥ niśkru?māṇaṃ/ chidraṃ/   śabdadhāraṃ/  bhrāṃtinilayatvaṃ/    \B
%ākāśapaṃcaguṇāḥ       preveśaḥ niśrumāṇaṃ    chidraṃ    śabdadhāraṃ   bhrāṃtinilayatvaṃ     \L
%ākāśasya paṃcaguṇāḥ/  praveśaḥ niśkrumāṇaṃ/  cchidraṃ/  śabdadhāraḥ   bhrāṃtinilayatvaṃ//  \N1
%ākāśasya paṃcaguṇāḥ/  praveśaḥ niśkrumāṇaṃ/  cchidraḥ   śabdadhāraḥ   bhrāṃtinilayatvaṃ//  \N2
%\om                                                                 \D
%ākāśasya paṃcaguṇāḥ   preveśaḥ nikrumāṇaḥ    chidraṃ    śabdadhāraṃ   bhrāṃte nijatvaṃ      \U1
%ākāśasya paṃcaguṇaḥ// praveśa--niśkraṇaṃ     chidraṃḥ// śabdādhāraḥ// bhrāṃtinilayatvaṃ//   \U2
%-----------------------------
%The five qualities of space [are]: penetration, being without effect, openness, carrier of sound, container of movement.   
%----------------------------
\app{\lem[wit={ceteri}]{ākāśasya}
  \rdg[wit={L}]{ākāśa°}
  \rdg[wit={B}]{\om}}
\app{\lem[wit={ceteri}]{pañcaguṇāḥ}
  \rdg[wit={B}]{\om}}/
\app{\lem[wit={ceteri}]{praveśaḥ}
  \rdg[wit={U2}]{praveśa°}}\dd{}
\app{\lem[type=emendation, resp=egoscr]{niṣkrāmaṇaṃ}
  \rdg[wit={E}]{\korr niṣkramaṇaṃ}
  \rdg[wit={P,B,N1,N2}]{niśkrumāṇaṃ}
  \rdg[wit={U1}]{nikrumāṇaḥ}
  \rdg[wit={U2}]{niśkraṇaṃ}}\dd{}
chidraṃ\dd{}
\app{\lem[wit={ceteri}]{śabdadhāraḥ}
  \rdg[wit={L,U1}]{śabdadhāraṃ}}\dd{}
\app{\lem[wit={ceteri}]{bhrāṃtinilayatvaṃ}
  \rdg[wit={U1}]{bhrāṃte nijatvaṃ}}\dd{} 
%----------------------------
% mahāvāyoḥ pañcaguṇāḥ/        calanaṃ   śeṣa-----saṃcāraḥ, sparśaḥ,   dhūmravarṇatā,  \E %TRANSPOSED
% mahāvāyoḥ paṃcaguṇāḥ         calanaṃ   śoṣaḥ    saṃcāraḥ  sparśaḥ    dhūmravarṇatā   \P
% mahāvoyoḥ paṃcaguṇāḥ/                                                                \B
% mahāvāyoḥ paṃcaguṇāḥ         calanaṃ   śoṣaḥ    saṃcāraḥ  sparśa     dhūmravarṇatā   \L
% mahāvāyoḥ paṃcaguṇāḥ/        calanaṃ/  śoṣaḥ/   saṃcāraḥ/ sparśaḥ/   dhūmravarṇatā/  \N1
% mahāvāyoḥ paṃcaguṇāḥ//       calana/   śoṣaḥ    saṃcāraḥ  sparśaḥ    dhūmravarṇatā/  \N2
%\om                                                                 \D
% mahāvāyor        guṇāḥ    pracālanā    śoṣaḥ    saṃcāraḥ                                 nirodhanaṃ prasaraṇaṃ vaḥ                            \U1
% mahāvāyoḥ// paṃcaguṇaḥ//    calanaṃ// śoṣaṇaṃ// saṃcāraḥ// sparśaḥ// dhūmravarṇatā// \U2
%-----------------------------
%The five qualities of the great wind [are]: movement, wither (verdorren, aber kann auch Hauch, Lebenskraft sein), passage, touch, essence/form of smoke.    
%----------------------------
\note[type=source, labelb=345, lem={mahāvāyoḥ pañcaguṇāḥ}]{Ysv\textsuperscript{PT}:sañcāraścālanaṃ śeṣe pañcadhūmrābhamambare |}
\app{\lem[wit={ceteri}]{mahāvāyoḥ}
  \rdg[wit={U1}]{mahāvāyor}}
\app{\lem[wit={ceteri}]{pañcaguṇāḥ}
  \rdg[wit={U1}]{guṇāḥ}}/
\app{\lem[wit={ceteri}]{calanaṃ}
  \rdg[wit={U1}]{pracālanā}
  \rdg[wit={B}]{\om}}\dd{}
\app{\lem[wit={ceteri}]{śoṣaḥ}
  \rdg[wit={E}]{śeṣa°}
  \rdg[wit={U2}]{śoṣaṇaṃ}}\dd{}
\app{\lem[wit={ceteri}]{saṃcāraḥ}
  \rdg[wit={B,U1}]{\om}}\dd{}
\app{\lem[wit={ceteri}]{sparśaḥ}
  \rdg[wit={L}]{sparśa}
  \rdg[wit={B,U1}]{\om}}\dd{}
\app{\lem[wit={ceteri}]{dhūmravarṇatā}
  \rdg[wit={U1}]{nirodhanaṃ prasaraṇaṃ vaḥ}
  \rdg[wit={B}]{\om}}\dd{}
%----------------------------
%tejaḥ saṃcaraḥ tejasaḥ pañcaguṇāḥ/  dahanaṃ,  jvālarūpaṃ,     uṣṇatā, rakto varṇaḥ//   \E [P.70] %TRANSPOSED
%               tejasaḥ paṃcaguṇāḥ   dahanaṃ   jvālārūpaṃ      uṣṇatā--rakto  varṇāḥ     \P
%                                    dahanaṃ   jvālārūpaṃ/     uṣṇatā--rakto  varṇaḥ/      \B
%               tejasaḥ paṃcaguṇāḥ   dahanaṃ   jvālārūpaṃ/     uṣṇatā--rakta--varṇaḥ       \L
%               tejasaḥ paṃcaguṇāḥ/  dahanaṃ/  jvālārūpaṃ/     uṣṇatā/ rakto/ varṇaḥ/      \N1
%               tejasaḥ paṃcaguṇāḥ// dahanaṃ/  jvālārūpaṃ/     uṣṇatā/ rakta--varṇaḥ//    \N2
%\om                                                                                      \D
%\om                                                                                      \U1
%               tejasaḥ paṃcaguṇaḥ// dahanaṃ// jvālā// rūpaṃ// uṣṭṇatā// raktavarṇāḥ//   \U2
%-----------------------------
%The five qualities of light [are]: burning, flame shaped, heat, red-coloured, brightness.  
%-----------------------------
\note[type=source, labelb=346, lem={tejasaḥ pañcaguṇāḥ}]{Ysv\textsuperscript{PT}: uṣṇaprakāśaraktābhajvālādāhas tu tejasā | prakāśād eva śaithilyam adhutā śvetataj jale | sthūlasākārakāṭhinyagandhaṃ pātamṛdau tathā |}
\app{\lem[wit={ceteri}]{tejasaḥ}
  \rdg[wit={B,U1}]{\om}}
\app{\lem[wit={ceteri}]{pañcaguṇāḥ}
  \rdg[wit={U2}]{paṃcaguṇaḥ}
  \rdg[wit={B,U1}]{\om}}/
\app{\lem[wit={ceteri}]{dahanaṃ}
  \rdg[wit={U1}]{\om}}\dd{}
\app{\lem[wit={ceteri}]{jvālārūpaṃ}
  \rdg[wit={U2}]{jvālā || rūpaṃ}
  \rdg[wit={U1}]{\om}}\dd{}
\app{\lem[wit={ceteri}]{uṣṇatā}
  \rdg[wit={U2}]{uṣṭṇatā}
  \rdg[wit={U1}]{\om}}\dd{}
\app{\lem[wit={L,N2}]{raktavarṇaḥ}
  \rdg[wit={U2}]{raktavarṇāḥ}
  \rdg[wit={E}]{rakto varṇaḥ}
  \rdg[wit={B}]{rakto varṇaḥ}
  \rdg[wit={P}]{rakto varṇāḥ}
  \rdg[wit={N1}]{rakto | varṇaḥ}
  \rdg[wit={U1}]{\om}}\dd{}
\app{\lem[type=conjecture, resp=egoscr]{prakāśaḥ}
  \rdg[wit={ceteri}]{\conj \om}}\dd{}
\note[type=philcomm, labelb=347, lem={prakāśaḥ}]{Since all witnesses preserve only four qualities of light, I conjectured the existence of a the fifth in accordance with the source, the Ysv\textsuperscript{PT}.}
%----------------------------
%apāṃ paṃcaguṇāḥ | pravāhaḥ   śithilatā  dravaḥ madhuratā        śvetavarṇaḥ |   \E [P.70] %TRANSPOSED
%apāṃ paṃcaguṇāḥ   pravāha----śithilatā  dravaḥ madhurasatā      śvetavarṇāḥ      \P
%apa--paṃcaguṇāḥ/  pravāhaḥ   śithatā    dravaḥ madhuradatā      śvetavarṇāḥ/     \B
%apa--paṃcaguṇāḥ   pravāhaḥ   śithilatā  dravaḥ madhurasatā      śvetavarṇāḥ/     \L
%āpo  paṃcaguṇāḥ// pravāha/   śithilatā/ drava/ madhurarasatā/   śvetavarṇtā/     \N1
%āpo  paṃcaguṇāḥ// pravāha/   śithilatā/ drava--madhura/ rasatā/ śvetavarṇtā//    \N2
%\om                                                                                                                                                    \D
%\om                                                                                                                                                    \U1
%apāṃ paṃcaguṇāḥ// pravāhaḥ// śithilatā  dravaḥ// madhuratā//     svetavarṇaḥ//    \U2
%-----------------------------
%The five qualities of water [are]: flow, flabbiness, fluidness, lovely liquid tastefulness, transparent colour.
%-----------------------------
\app{\lem[wit={E,P,U2}]{apāṃ}
  \rdg[wit={L,B}]{apa°}
  \rdg[wit={N1,N2}]{āpo}
  \rdg[wit={U1}]{\om}}
\app{\lem[wit={ceteri}]{pañcaguṇāḥ}
  \rdg[wit={U1}]{\om}}
\app{\lem[wit={B,E,L,U2}]{pravāhaḥ}
  \rdg[wit={P}]{pravāha°}
  \rdg[wit={N1,N2}]{pravāha}
  \rdg[wit={U1}]{\om}}\dd{}
\app{\lem[wit={ceteri}]{śithilatā}
  \rdg[wit={B}]{śithatā}
  \rdg[wit={U1}]{śithilatā}}
\app{\lem[wit={ceteri}]{dravaḥ}
  \rdg[wit={N1,N2}]{drava}
\rdg[wit={U1}]{\om}}\dd{}
\app{\lem[wit={N1}]{madhurarasatā}
  \rdg[wit={N2}]{°madhura | rasatā}
  \rdg[wit={L,P}]{madhurasatā}
  \rdg[wit={B}]{madhuradatā}
  \rdg[wit={E,U2}]{madhuratā}
  \rdg[wit={U1}]{\om}}\dd{}
\app{\lem[wit={E,U2}]{śvetavarṇaḥ}
  \rdg[wit={P,B,L}]{śvetavarṇāḥ}
  \rdg[wit={N1,N2}]{śvetavarṇtā}
  \rdg[wit={U1}]{\om}}\dd{}
%-----------------------------
%pṛthivyāḥ paṃcaguṇāḥ | sthūlatā   sākāratā  kaṭhinatā  gandhavattā    pītavarṇatā     \E [P.70] TRANSPOSED 
%pṛthivyā   guṇāpaṃca// sthulatā// sakāratā/ kathinatā/ gaṃdhavatta/   pītavarṇaḥ      \N1 (S.12 verso Z.1)
%pṛthivyā   guṇāpaṃca// syūlatā/   sākāratā/ kathinatā/                pītavarṇaḥ/     \N2
%pṛthivyā   guṇāḥ       sthalatā   sākāratā  kaṭhiṇatā  gaṃdhavettā    pītavarṇā       \U1
%pṛthivyāḥ paṃcaguṇāḥ// sthūlatā//           kaṭhiṇatā  gaṃdhavatā//   pītavarṇatā//   \U2
%\om                                                                                      \P
%\om                                                                                      \B
%\om                                                                                      \L
%\om                                                                                      \D
%-----------------------------
%The five qualities of earth [are]: grossness, shapeliness, hardness, smelliness [and] yellowness, 
%-----------------------------
\note[type=philcomm, labelb=348, lem={pṛthivyāḥ pañcaguṇāḥ}]{The passage of the five qualities of earth is missinf in P,B,L. Omissions are not recorded in the \textit{apparatus criticus}.}
\app{\lem[wit={E,U2}]{pṛthivyāḥ}
  \rdg[wit={N1,N2,U1}]{pṛthivyā}}
\app{\lem[wit={E,U2}]{pañcaguṇāḥ}
  \rdg[wit={N1,N2}]{guṇāpaṃca}
  \rdg[wit={U1}]{guṇāḥ}}/
\app{\lem[wit={E,U2}]{sthūlatā}
  \rdg[wit={N1}]{sthulatā}
  \rdg[wit={N2}]{syūlatā}
  \rdg[wit={U1}]{sthalatā}}\dd{}
\app{\lem[wit={ceteri}]{sākāratā}
  \rdg[wit={U2}]{\om}}\dd{}
\app{\lem[wit={E}]{kaṭhinatā}
  \rdg[wit={N1,N2}]{kathinatā}
  \rdg[wit={U1,U2}]{kaṭhiṇatā}}
\app{\lem[wit={E,U1}]{gandhavattā}
  \rdg[wit={N1}]{gaṃdhavatta}
  \rdg[wit={U1}]{gaṃdhavettā}
  \rdg[wit={N2}]{\om}}\dd{}
\app{\lem[wit={E,U2}]{pītavarṇatā}
  \rdg[wit={N1,N2}]{pītavarṇaḥ}
  \rdg[wit={U1}]{pītavarṇā}}\dd{}
%-----------------------------
%        śarīramadhye     paṃca mahābhūtāni//           teṣāṃ guṇāḥ kathyante    \E [P.70] TRANSPOSED until  avayavatvam ananyatvaṃ
% idānīṃ śarīramadhye     paṃca mahāsūtāni? kathyate/   teṣāṃ guṇāḥ kathyate/    \N1 (S.12 verso Z.1)
% idānīṃ śarīramadhye     paṃca mahābhūtāni kathyate//  teṣā  guṇāḥ kathyate//    \N2
% idānīṃ śrīramadhye      paṃca āpaguṇaḥ mahāsveravarṇa tāvāt                   \U1
% atha   śarīrasya madhye paṃca mahābūtāni//            teṣāṃ guṇāḥ kathyaṃte//  \U2
%\om                                                                                                                                                                      \P
%\om                                                                                                                                                                      \B
%\om                                                                                                                                                                      \L
%\om                                                                                                                                                                      \D
%-----------------------------
%Now the five great elements situated within the body are taught. The qualities of them are taught:
%-----------------------------
\note[type=philcomm, labelb=349, lem={śarīramadhye}]{At this point witness E resynchronizes with the textual structure of all other witnesses.}
\note[type=source, labelb=350, lem={pañca mahābhūtāni}]{Ysv\textsuperscript{PT}: mahābhūtāni pañceti dehamadhye 'dhunā śṛṇu | mahābhūtāni pañceti pṛthvītejo marut khakam |}
\app{\lem[wit={N1,N2,U1}]{idānīṃ}
  \rdg[wit={U2}]{atha}
  \rdg[wit={E,P,B,L}]{\om}}
\app{\lem[wit={E,N1,N2}]{śarīramadhye}
  \rdg[wit={U1}]{śrīramadhye}
  \rdg[wit={U2}]{śarīrasya madhye}
  \rdg[wit={P,B,L}]{\om}}
pañca\app{\lem[wit={E,N2,U2},alt={°mahābhūtāni}]{mahābhūtāni}
  \rdg[wit={N1}]{mahāsūtāni}
  \rdg[wit={U1}]{āpaguṇaḥ mahāsveravarṇa}
  \rdg[wit={P,B,L}]{\om}}
\app{\lem[wit={N1,N2}]{kathyate}
  \rdg[wit={ceteri}]{\om}}
\app{\lem[wit={E,N1,U2}]{teṣāṃ}
  \rdg[wit={N2}]{teṣā}
  \rdg[wit={U1}]{tāvāt}
  \rdg[wit={P,B,L}]{\om}}
\app{\lem[wit={E,N1,N2,U2}]{guṇāḥ}
  \rdg[wit={ceteri}]{\om}}
\app{\lem[wit={E,U2}]{kathyante}
  \rdg[wit={N1,N2}]{kathyate}
  \rdg[wit={ceteri}]{\om}}/
\end{prose}
\end{ekdosis}
\ekdpb*{}
%%%%%%%%%%%%%%%%%%%%%%%%%%%%%%%%%%%%%%%%%%
%%%%%%%%PAGEBREAK%%%%%%%PAGEBREAK%%%%%%%%%
%%%%%%%%%%%%%%%%%%%%%%%%%%%%%%%%%%%%%%%%%%
%%%%%%%%%%%%%%%%PAGEBREAK%%%%%%%%%%%%%%%%%
%%%%%%%%%%%%%%%%%%%%%%%%%%%%%%%%%%%%%%%%%%
%%%%%%%%PAGEBREAK%%%%%%%PAGEBREAK%%%%%%%%%
%%%%%%%%%%%%%%%%%%%%%%%%%%%%%%%%%%%%%%%%%%
%%%%%%%%%%%%%%%%%%%%%%%%%%%%%%%%%%%%%%%%%%
%%%%%%%%%%%%%%%%%%%%%%%%%%%%%%%%%%%%%%%%%%
%%%%%%%%%%%%%%%%%%%%%%%%%%%%%%%%%%%%%%%%%%
%%%%%%%%PAGEBREAK%%%%%%%PAGEBREAK%%%%%%%%%
%%%%%%%%%%%%%%%%%%%%%%%%%%%%%%%%%%%%%%%%%%
%%%%%%%%%%%%%%%%PAGEBREAK%%%%%%%%%%%%%%%%%
%%%%%%%%%%%%%%%%%%%%%%%%%%%%%%%%%%%%%%%%%%
%%%%%%%%PAGEBREAK%%%%%%%PAGEBREAK%%%%%%%%%
%%%%%%%%%%%%%%%%%%%%%%%%%%%%%%%%%%%%%%%%%%
%%%%%%%%%%%%%%%%%%%%%%%%%%%%%%%%%%%%%%%%%%
%%%%%%%%%%%%%%%%%%%%%%%%%%%%%%%%%%%%%%%%%%
%%%%%%%%%%%%%%%%%%%%%%%%%%%%%%%%%%%%%%%%%%
%%%%%%%%PAGEBREAK%%%%%%%PAGEBREAK%%%%%%%%%
%%%%%%%%%%%%%%%%%%%%%%%%%%%%%%%%%%%%%%%%%%
%%%%%%%%%%%%%%%%PAGEBREAK%%%%%%%%%%%%%%%%%
%%%%%%%%%%%%%%%%%%%%%%%%%%%%%%%%%%%%%%%%%%
%%%%%%%%PAGEBREAK%%%%%%%PAGEBREAK%%%%%%%%%
%%%%%%%%%%%%%%%%%%%%%%%%%%%%%%%%%%%%%%%%%%
\begin{ekdosis}
  \begin{prose}
    \noindent
%----------------------------
% tatra pṛthivyā       guṇāḥ –          asthi-māṃsa--nāḍī-lomāni vāk/   \E [P.70]
%       pṛthivyāḥ paṃcaguṇāḥ kathyaṃte  asthi-māṃsaṃ nāḍī lomāni vākṛt- \P
%       pṛthvīyā  paṃcaguṇāḥ/ athyate/  asthi-māṃsa--nāḍī-------tvak lomāni/ \B
%       pṛthvīyā  paṃcaguṇāḥ kathyaṃte/ asthi-māṃsa--nāḍī-------tvak lomāni  \L
% tatra pṛthivyā       guṇāḥ//          asthi/ māṃsa/ nāḍī/ lomāni/ tvak/ \N1 (S.12 verso Z.1)
% tatra pṛthivyā       guṇāḥ//          asthi/ māṃsa/ nāḍī/ lomāni/ tvak/  \N2
%\om                                                                 \D
%\om                                                                \U1
% tatra pṛthvyā        guṇāḥ//          asti// māṃsa// nāḍī// lomāni// tvakḥ// \U2
%-----------------------------
%Therein the five qualities are of the earth-element are: bone, flesh, channels, hair [and] skin. 
%-----------------------------
\note[type=source, labelb=351, lem={pṛthivyāḥ pañcaguṇāḥ}]{Ysv\textsuperscript{PT}: eteṣāñ ca tathā pañcaguṇasthānaṃ śṛṇu priye | asthi māṃsaṃ loma nāḍī tvak ceti pṛthivīguṇāḥ |}
\note[type=testium, labelb=352, lem={pṛthivyāḥ pañcaguṇāḥ}]{SSP 1.37: asthimāṃsatvaṅnāḍīromāṇīti pañcaguṇā bhūmiḥ |}
\app{\lem[wit={E,N1,N2}]{tatra}
  \rdg[wit={ceteri}]{\om}}
\app{\lem[wit={B,E,L,N1,N2,U2}]{pṛthvīyā}
  \rdg[wit={P}]{pṛthivyāḥ}
  \rdg[wit={ceteri}]{\om}}
\app{\lem[wit={E,N1,N2,U2}]{guṇāḥ}
  \rdg[wit={L,P}]{paṃcaguṇāḥ kathyaṃte}
  \rdg[wit={B}]{paṃcaguṇāḥ | athyate |}
  \rdg[wit={ceteri}]{\om}}/ 
\app{\lem[wit={B,E,L,P,N1,N2}]{asthi}
  \rdg[wit={U2}]{asti}
  \rdg[wit={ceteri}]{\om}}\dd{}
\app{\lem[wit={P}]{māṃsaṃ}
  \rdg[wit={ceteri}]{māṃsa}
  \rdg[wit={U1}]{\om}}\dd{}
\app{\lem[wit={E,P,N1,N2,U2}]{lomāni}
  \rdg[wit={B,L}]{tvak}
  \rdg[wit={ceteri}]{\om}}\dd{}
\app{\lem[wit={N1,N2}]{tvak}
  \rdg[wit={U2}]{tvakḥ}
  \rdg[wit={E}]{vāk}
  \rdg[wit={P}]{vākṛt}
  \rdg[wit={ceteri}]{\om}}\dd{}
%----------------------------
%tatrodakaguṇāḥ-   lālā,   mūtraṃ,  śuklaṃ, raktaṃ, prasvedaḥ/      \E [P.70]
%tatrodakaguṇāḥ    lālā----muvaṃ    śukraṃ  raktaṃ  prasvedaḥ       \P
%tatrodakaguṇāḥ/   lāla----mutra----śukraṃ  raktaṃ  prasvedaḥ/      \B
%tatrodakaguṇāḥ    lāla----mutra----śukraṃ  raktaṃ  prasvedaḥ//     \L
%netrodake guṇāḥ// lālā----mutraṃ/  śukraṃ/ raktaṃ/ prasvedaḥ/      \N1
%netrodakaguṇāḥ//  lālā/   mūtraṃ/  śukraṃ/         prasvedaḥ//     \N2
%\om                                                                \D
%                  lālā----mutraṃ   śukraṃ  raktaṃ      svedaḥ      \U1
%tatrodakaguṇaḥ    lallā// mūtraṃ// śukraṃ//raktaṃ// prasvedaḥ//    \U2
%-----------------------------
%Therein the qualities of the water-element are: saliva, urine, semen, blood and sweat.  
%-----------------------------
\note[type=source, labelb=353, lem={tatrodakaguṇāḥ}]{Ysv\textsuperscript{PT}: kṣudhātṛṣṇālasyanidrā glāniś ca pañca vāriṇaḥ | rogo lajjā bhayodvegau dhāraṇā ca marud guṇāḥ |}
\note[type=testium, labelb=354, lem={tatrodakaguṇāḥ}]{SSP 1.38: lālā mūtraṃ śukraṃ śoṇitaṃ sveda iti pañcaguṇā āpaḥ |}
\app{\lem[wit={ceteri}]{tatrodakaguṇāḥ}
  \rdg[wit={N1}]{netrodake guṇāḥ}
  \rdg[wit={N2}]{netrodakaguṇāḥ}
  \rdg[wit={U1}]{\om}}
\app{\lem[wit={ceteri}]{lālā}
  \rdg[wit={B,L}]{lāla°}}\dd{}
\app{\lem[wit={E,N2,U2}]{mūtraṃ}
  \rdg[wit={N1,U1}]{mutraṃ}
  \rdg[wit={B,L}]{°mutra°}
  \rdg[wit={P}]{°muvaṃ}}\dd{}
\app{\lem[wit={ceteri}]{śukraṃ}
  \rdg[wit={E}]{śuklaṃ}}
\app{\lem[wit={ceteri}]{raktaṃ}
  \rdg[wit={N2}]{\om}}\dd{}
\app{\lem[wit={ceteri}]{prasvedaḥ}
  \rdg[wit={U1}]{svedaḥ}}\dd{}
%----------------------------
%tejaso guṇāḥ-  kṣudhā   tṛṣā   nidrā   glāniḥ ālasyam/   \E [P.70]
%tejaso guṇāḥ   kṣudhā   tṛṣā   nidrā   glāniḥ ālasyaṃ    \P
%tejaso guṇāḥ/  kṣudhāṃ  tṛṣā   nidrā   glāni  ālasyaṃ//  \B
%tejaso guṇāḥ/  kṣudhā   tṛṣā   nidrā   glāni  ālasyaṃ//  \L
%tejaso guṇāḥ// kṣudhā/  tṛṣā/  nidrā/  glāni/ alasyaṃ//  \N1
%tejaso guṇāḥ// kṣudhā/  tṛṣā/  nidrā/  glāni/ ālasyaṃ//  \N2
%\om                                                                                                                      \D
%tejaso guṇāḥ   kṣudhā   tṛṣā   nidrā   glāni   ālasya    \U1
%tejaso guṇaḥ// kṣudhā// tṛṣā// nidrā// glāni// ālasyaṃ// \U2
%-----------------------------
%The qualities of the fire-element: hunger, thirst, sleep, exhaustion, sloth.   
%-----------------------------
\note[type=testium, labelb=355, lem={tejaso guṇāḥ}]{SSP 1.39: kṣudhā tṛṣā nidrā kāntir ālasyam iti pañcaguṇaṃ tejaḥ |}
tejaso \app{\lem[wit={ceteri}]{guṇāḥ}
  \rdg[wit={U2}]{guṇaḥ}}/
\app{\lem[wit={ceteri}]{kṣudhā}
  \rdg[wit={B}]{kṣudhāṃ}}\dd{}
tṛṣā\dd{}
nidrā\dd{}
\app{\lem[wit={E,P}]{glāniḥ}
  \rdg[wit={ceteri}]{glāni}}\dd{}
\app{\lem[wit={ceteri}]{ālasyaṃ}
  \rdg[wit={U1}]{ālasya}}\dd{}
%-----------------------------
%vāyor guṇāḥ - dhāvanaṃ  majjanaṃ   nirodhanaṃ    prasāraṇam   ākuṃcanaṃ  ceti/   \E
%vāyor guṇāḥ   dhāvanaṃ  majjanaṃ   nirodhanaṃ    prasāraṇaṃ   ākuṃcanaṃ  ceti    \P
%vāyo  guṇāḥ/  dhāvanaṃ  majjanaṃ   nirodhanaṃ/   prasāraṇaṃ/  ākuṃcanaṃ  ceti/   \B
%vāyor guṇāḥ// dhāvanaṃ  majjanaṃ   nirodhanaṃ    prasāraṇaṃ   ākuṃcanaṃ  ceti... \L
%vāyor guṇāḥ/  dhāvanaṃ/ majjanaṃ/  nirodhanaṃ/   prasāraṇaṃ/  ākuṃcanaṃ/ ceti//  \N1
%vāyo  guṇāḥ/  dhāvanaṃ/ majana/    virodhana/    praśaraṇāṃ/  ākūrcana   ceti//  \N2
%%\om                                                                             \D
%vāyu  guṇā    dhāvanaṃ  mano----------rodhanaṃ   prasāraṇaṃ   ākuṃcanaṃ  ceti    \U1
%vāyo  guṇaḥ// dhāvanaṃ// majjanaṃ// nirodhanaṃ// prasāraṇaṃ// ākuṃcanaṃ//        \U2
%-----------------------------
%The qualities of the wind-element are: wash off, marrow, confinement, stretch out and contraction. 
%----------------------------
\note[type=testium, labelb=356, lem={vāyor guṇāḥ}]{SSP 1.40: dhāvanaṃ plavanaṃ prasāraṇaṃ ākuñcanaṃ nirodhanam iti pañcaguṇo vayuḥ |}
\app{\lem[wit={ceteri},alt={vāyor}]{vāyo\skp{r-gu}}
  \rdg[wit={B,N2,U2}]{vāyo}
  \rdg[wit={U1}]{vāyu}
}\app{\lem[wit={ceteri},alt={guṇāḥ}]{\skm{r-gu}ṇāḥ}
  \rdg[wit={U1}]{guṇā}}/
dhāvanaṃ\dd{}
\app{\lem[wit={ceteri}]{majjanaṃ}
  \rdg[wit={N2}]{majana}
  \rdg[wit={U1}]{mano°}}\dd{}
\app{\lem[wit={ceteri}]{nirodhanaṃ}
  \rdg[wit={U1}]{°rodhanaṃ}
  \rdg[wit={N2}]{virodhana}}\dd{}
prasāraṇam\dd{}
\app{\lem[wit={ceteri}]{ākuñcanaṃ}
  \rdg[wit={N2}]{ākūrcana}}
\app{\lem[wit={ceteri}]{ceti}
  \rdg[wit={U2}]{\om}}\dd{}
%----------------------------
%ākāśasya guṇāḥ – rāga-dveṣau      bhayaṃ   lajjā   mohaḥ/  \E
%ākāśasya guṇāḥ   rāga-dveṣaḥ      bhayaṃ   lajjā   mohaḥ   \P
%ākāśasya guṇāḥ/  rāga-dveṣ--------bhayaṃ   lajjā   moha/   \B
%ākāśasya guṇāḥ// rāga-dveṣ--------bhayaṃ   lajjā   moha//  \L
%ākāsasya guṇāḥ/  rāga-dveṣo/      bhayaṃ/  lajjā/  mohaḥ/  \N1
%ākāsasya guṇāḥ// rāga/ dveṣau/    bhayaṃ/  lajjā/  moha/   \N2
%\om                                                        \D CHECK!!!!
%ākāśasya guṇaḥ   rāgadveṣau       bhayaṃ   lajjā   mohā    \U1
%ākāśasya guṇāḥ// rāgaḥ// dveṣaḥ// bhayaṃ// lajjā// mohaḥ// \U2
%-----------------------------
%The qualities of the space-element are: attachment, aversion, fear, shame and confusion. 
%----------------------------
\note[type=testium, labelb=357, lem={ākāsasya guṇāḥ}]{SSP 1.41: rāgo dveṣo bhayaṃ lajjā moha iti pañcaguṇa ākaśaḥ |}
ākāśasya \app{\lem[wit={ceteri}]{guṇāḥ}
  \rdg[wit={U1}]{guṇaḥ}}/
\app{\lem[wit={U2}]{rāgaḥ}
  \rdg[wit={ceteri}]{rāga}}\dd{}
\app{\lem[wit={P,U2}]{dveṣaḥ}
  \rdg[wit={N1}]{°dveṣo}
  \rdg[wit={E}]{°dveṣau}
  \rdg[wit={U1}]{dveṣau}
  \rdg[wit={B,L}]{dveṣ°}}\dd{}
bhayaṃ\dd{}
lajjā\dd{}
\app{\lem[wit={E,P,N1,U2}]{mohaḥ}
  \rdg[wit={B,L,N2}]{moha}
  \rdg[wit={U1}]{mohā}}\dd{}
%----------------------------
%tad anaṃtaram ekādaśī   kā buddhir utpadyate/ \E
%tad anaṃtaram ekādṛśy  ekā buddher utpadyate  \P
%tad anaṃtaraṃ metādaśī     buddhir utpadyate/ \B
%tad anaṃtaraṃ etādaśī      buddhir utpadyate.. \L
%tad anaṃtaraṃ etādṛśā  ekā buddhir utpadyate/ \N1
%tad anaṃtaraṃ etādṛśī  ekā buddhir utpadyate/ \N2
%\om                                                                 \D
%tad anaṃtaraṃ etādaśī  ekā buddhir utpadyate.. \U1
%tad anaṃtaram etādṛśy  ekā buddhir utpadyate// \U2
%-----------------------------
%After that only the intellect arises. 
%----------------------------
\note[type=source, labelb=358, lem={buddhir utpadyate}]{Ysv\textsuperscript{PT}: etaj jñānenaiva teṣāṃ buddhir utpadyate śubhā | yadyapi sargakāṇḍe pṛthvyāder guṇā uktās tathāpy etaj jñānenety anena kāryakāraṇabhāvadarśanāya punar ucyante |}
tad \app{\lem[wit={E,P,U2},alt={anantaram}]{anantara\skp{m-e}}
  \rdg[wit={ceteri}]{anaṃtaraṃ}
}\app{\lem[wit={U2,P},alt={etādṛśy}]{\skm{m-e}tādṛ\skp{śy-e}}
  \rdg[wit={N2}]{etādṛśī}
  \rdg[wit={N1}]{etādṛśā}
  \rdg[wit={L,U1}]{etādaśī}
  \rdg[wit={E}]{ekādaśī}
  \rdg[wit={B}]{metādaśī}
}\app{\lem[wit={ceteri},alt={ekā}]{\skm{śy-e}kā}
  \rdg[wit={E}]{kā}
  \rdg[wit={B,L}]{\om}}
\app{\lem[wit={ceteri},alt={buddhir}]{buddhi\skp{r-u}}
  \rdg[wit={P}]{buddher}}\skm{r-u}tpadyate/
%----------------------------
%mano buddhyahaṃkārāś   cittaṃ caitanyaṃ ceti/     ete paṃcaprakārā    aṃtaḥkaraṇasya/ \E
%mano buddhir aṃhaṃkāraś  cittaṃ caitanyaṃ ceti    ete paṃcāpiprakārā  aṃtaḥkaraṇasya  \P %%%7672.jpg 
%mano buddhir ahaṃkāra/  ścittaṃ caitanyaṃ ceti/   ete paṃcāpiprakāra/ aṃtaḥkarṇsya    \B
%mano buddhir ahaṃkāraś   cittaṃ caitanyaṃ ceti... ete paṃcāpiprakārāḥ aṃtaḥkarṇsya  \L
%mano buddhir ahaṃkāra    cittaṃ           ceti/   ete paṃcāpiprakārā  aṃtaḥkaraṇasya/ \N1
%mano buddhir ahaṃkāra    cittaṃ           ceti//  ete paṃcāprakārā    aṃtakaraṇasya//  \N2
%\om                                                                                   \D
%mano buddhir ahaṃkāraś   cittaṃ           ceti... ete paṃcāpiprakārā  aṃtaḥkarṇva    \U1
%mano buddhir ahaṃkāraḥ// cittaṃ cautanyaṃ ceti//  ete paṃcaprakāraḥ   aṃtaḥkaraṇasya \U2
%-----------------------------
%The mind, the intellect, the ego, the spirit and consciousness. These are the five modes of the internal organ. 
%----------------------------
\note[type=source, labelb=359, lem={mano buddhir}]{Ysv\textsuperscript{PT}: mano buddhir ahaṅkāraś cittaṃ caitanyameva ca | ete pañcaprakārāś ca antaḥkaraṇasambhavāḥ |}
\note[type=testium, labelb=360, lem={mano buddhir}]{SSP 1.42: mano buddhir ahaṅkāraś cittaṃ caitanyam ity antaḥkaraṇapañcakam |}
mano \app{\lem[wit={ceteri},alt={buddhir}]{buddhi\skp{r-a}}
  \rdg[wit={E}]{buddhy}
}\app{\lem[wit={B,L,U1},alt={ahaṃkāraś}]{\skm{r-a}haṃkāra\skp{ś-ci}}
  \rdg[wit={E}]{ahaṃkārāś}
  \rdg[wit={U2}]{ahaṃkāraḥ ||}
  \rdg[wit={B}]{ahaṃkāra | ś}
  \rdg[wit={N1,N2}]{ahaṃkāra}
}\skm{ś-ci}ttaṃ \app{\lem[wit={B,E,L,P,U2}]{caitanyaṃ}
  \rdg[wit={ceteri}]{\om}} ceti/
ete \app{\lem[wit={E}]{pañcaprakārā}
  \rdg[wit={N2}]{paṃcāprakārā}
  \rdg[wit={U2}]{paṃcaprakāraḥ}
  \rdg[wit={P}]{paṃcāpiprakārā}
  \rdg[wit={B}]{paṃcāpiprakāra |}
  \rdg[wit={L}]{paṃcāpiprakārāḥ}
  \rdg[wit={N1,U1}]{paṃcāpiprakārā}}
\app{\lem[wit={ceteri}]{aṃtaḥkaraṇasya}
  \rdg[wit={N2}]{aṃtakaraṇasya}
  \rdg[wit={B,L}]{aṃtaḥkarṇsya}
  \rdg[wit={U1}]{aṃtaḥkarṇva}}/ 
%----------------------------
%manasaḥ ye ca guṇāḥ  saṃkalpa---vikalpa---mūrkhatvā---lasatā  mananaṃ ceti//     \E [p.71]
%manasaḥ paṃcaguṇāḥ   saṃkalpa---vikalpa---mūrkhatva---jaḍatā  mananaṃ ceti       \P
%manasaḥ paṃcaguṇāḥ   saṃkalpa---vikalpa---mūrkhatva---jaḍatā  mananaṃ ceti/      \B
%manasaḥ paṃcaguṇāḥ   sakalpa----vikalpa---mūrkhatva---jaḍatā  mananaṃ ceti...    \L
%manasaḥ paṃcaguṇāḥ   saṃkalpa/  vikalpaḥ/ mūrṣatvaṃ/  jaḍatā/ mananaṃ ceti/ ete paṃcāpiprakārā aṃtaḥ karaṇasya ma ****doubling****\N1
%manasaḥ paṃcaguṇāḥ   saṃkalpaḥ  vikalpa   mūrkhatvaṃ/ jaḍatā/ mananaṃ ceti//     \N2
%\om                                                                              \D
%manasaḥ paṃcaguṇāḥ   saṃkalpa---vikalpa---mūrṣatvaṃ   jaḍatā  mananaṃ ceti vā... \U1 %%%298.jpg
%manasaḥ paṃcaguṇaḥ// saṃkalpa// vikalpa// mūrkhatva// jaḍatā  mananaṃ ceti//     \U2
%-----------------------------
%The five qualities of the mind are: 
%----------------------------
\note[type=source, labelb=361, lem={manasaḥ pañcaguṇāḥ}]{SSP 1.43: saṃkalpo vikalpo mūrcchā jaḍatā mananam iti pañcaguṇaṃ manaḥ}
manaṣaḥ \app{\lem[wit={ceteri}]{pañcaguṇāḥ}
  \rdg[wit={E}]{ye ca guṇāḥ}}/
\app{\lem[wit={N2}]{saṃkalpaḥ}
  \rdg[wit={L}]{sakalpa}
  \rdg[wit={ceteri}]{saṃkalpa}}\dd{}
\app{\lem[wit={N1}]{vikalpaḥ}
  \rdg[wit={ceteri}]{vikalpa}}\dd{}
\app{\lem[wit={N2}]{mūrkhatvaṃ}
  \rdg[wit={N1,U1}]{mūrṣatvaṃ}
  \rdg[wit={E}]{mūrkhatvā}
  \rdg[wit={ceteri}]{mūrkhatva}}
\app{\lem[wit={ceteri}]{jaḍatā}
  \rdg[wit={E}]{lasatā}}\dd{}
mananaṃ \app{\lem[wit={ceteri}]{ceti}
  \rdg[wit={U1}]{ceti vā}
  \rdg[wit={N1}]{ceti ete paṃcāpiprakārā aṃtaḥkaraṇasya ma}}/
%----------------------------
%buddheḥ  paṃcaguṇāḥ/  viveko   vairāgyaṃ   śāntiḥ   santoṣaḥ   kṣamā ceti/  \E
%buddheḥ  paṃcaguṇāḥ   vivekaḥ  vairāgya    śāṃtiḥ   saṃtoṣaḥ   kṣamā ceti   \P
%buddhe   paṃcaguṇāḥ/  viveka---vairāgya----śāntiḥ   santoṣaḥ   kṣamā ceti/  \B
%buddheḥ  paṃcaguṇāḥ/  viveka---vairāgya----śāntiḥ   santoṣaḥ   kṣamā ceti   \L
%buddheḥ  paṃcaguṇāḥ/  vivekaḥ/ vairāgya/   śāntiḥ/  santoṣaḥ/  kṣamā ceti/  \N1
%                      vivekaḥ  vairāgya    śāntiḥ   santoṣa    kṣamā ceti// \N2
%                                                                     ceti/  \D
%                      viveka---vairāgya----śāntiḥ   saṃtoṣaḥ   kṣamā vā     \U1
%                      viveko// vairāgyaṃ// śāntiḥ// santoṣāḥ// kṣamā ceti// \U2
%-----------------------------
%The five qualities of the intellect are: differentitation, equanimity, peace, contentment and patience. 
%-----------------------------
\note[type=source, labelb=362, lem={buddheḥ pañcaguṇāḥ}]{SSP 1.44: viveko vairāgyaṃ śāntiḥ santoṣaḥ kṣameti pañcaguṇā buddhiḥ |}
\note[type=source, labelb=363, lem={buddheḥ pañcaguṇāḥ}]{Ysv\textsuperscript{PT}: mananāmananaṃ jñeyaṃ buddhyādipañca pañca tu | vivekaśāntisantoṣakṣamāvairāgyateti ca | ete pañcaguṇā buddher ahaṅkāraguṇān śṛṇu |}
\app{\lem[wit={E,L,P,N1}]{buddheḥ}
  \rdg[wit={B}]{buddhe}
  \rdg[wit={ceteri}]{\om}}
\app{\lem[wit={B,E,L,P,N1}]{pañcaguṇāḥ}
  \rdg[wit={ceteri}]{\om}}/
\app{\lem[wit={P,N1,N2}]{vivekaḥ}
  \rdg[wit={E,U2}]{viveko}
  \rdg[wit={B,L,U1}]{viveka}}\dd{}
\app{\lem[wit={E,U2}]{vairāgyaṃ}
  \rdg[wit={ceteri}]{vairāgya}}\dd{}
śāntiḥ\dd{}
\app{\lem[wit={ceteri}]{santoṣaḥ}
  \rdg[wit={N2}]{santoṣa}
  \rdg[wit={U2}]{santoṣāḥ}}\dd{}
kṣamā
\app{\lem[wit={ceteri}]{ceti}
  \rdg[wit={U1}]{vā}}\dd{} 
\note[type=philcomm, labelb=362, lem={ceti | ahaṃkārasya \ldots}]{Witness D resumes its evidence from here.}
\end{prose}
\end{ekdosis}
\ekdpb*{}
%%%%%%%%%%%%%%%%%%%%%%%%%%%%%%%%%%%%%%%%%%
%%%%%%%%PAGEBREAK%%%%%%%PAGEBREAK%%%%%%%%%
%%%%%%%%%%%%%%%%%%%%%%%%%%%%%%%%%%%%%%%%%%
%%%%%%%%%%%%%%%%PAGEBREAK%%%%%%%%%%%%%%%%%
%%%%%%%%%%%%%%%%%%%%%%%%%%%%%%%%%%%%%%%%%%
%%%%%%%%PAGEBREAK%%%%%%%PAGEBREAK%%%%%%%%%
%%%%%%%%%%%%%%%%%%%%%%%%%%%%%%%%%%%%%%%%%%
%%%%%%%%%%%%%%%%%%%%%%%%%%%%%%%%%%%%%%%%%%
%%%%%%%%%%%%%%%%%%%%%%%%%%%%%%%%%%%%%%%%%%
%%%%%%%%%%%%%%%%%%%%%%%%%%%%%%%%%%%%%%%%%%
%%%%%%%%PAGEBREAK%%%%%%%PAGEBREAK%%%%%%%%%
%%%%%%%%%%%%%%%%%%%%%%%%%%%%%%%%%%%%%%%%%%
%%%%%%%%%%%%%%%%PAGEBREAK%%%%%%%%%%%%%%%%%
%%%%%%%%%%%%%%%%%%%%%%%%%%%%%%%%%%%%%%%%%%
%%%%%%%%PAGEBREAK%%%%%%%PAGEBREAK%%%%%%%%%
%%%%%%%%%%%%%%%%%%%%%%%%%%%%%%%%%%%%%%%%%%
%%%%%%%%%%%%%%%%%%%%%%%%%%%%%%%%%%%%%%%%%%
%%%%%%%%%%%%%%%%%%%%%%%%%%%%%%%%%%%%%%%%%%
%%%%%%%%%%%%%%%%%%%%%%%%%%%%%%%%%%%%%%%%%%
%%%%%%%%PAGEBREAK%%%%%%%PAGEBREAK%%%%%%%%%
%%%%%%%%%%%%%%%%%%%%%%%%%%%%%%%%%%%%%%%%%%
%%%%%%%%%%%%%%%%PAGEBREAK%%%%%%%%%%%%%%%%%
%%%%%%%%%%%%%%%%%%%%%%%%%%%%%%%%%%%%%%%%%%
%%%%%%%%PAGEBREAK%%%%%%%PAGEBREAK%%%%%%%%%
%%%%%%%%%%%%%%%%%%%%%%%%%%%%%%%%%%%%%%%%%%
\begin{ekdosis}
  \begin{prose}
    \noindent
%-----------------------------
%ahaṃkārasya paṃcaguṇāḥ/  ahaṃ  mameti etasya duḥkhaṃ   svataṃtratā/      \E
%ahaṃkārasya paṃcaguṇāḥ                                                  \P
%ahaṃkārasya paṃcaguṇāḥ/                                                 \B
%ahaṃkārasya paṃcaguṇāḥ//                                                \L
%ahaṃkārasya paṃcaguṇāḥ/  ahaṃ/ mama / etasya duḥkhaṃ/  svatantratā/     \N1
%ahaṃkārasya paṃcaguṇāḥ// ahaṃ/ mama/  etasya duḥkhaṃ/  svataṃtratā/     \N2
%ahaṃkārasya paṃcaguṇāḥ/  ahaṃ/ mama/  etasya duḥkhaṃ/  svataṃtratāḥ/    \D
%ahaṃkārasya paṃcaguṇāḥ         samā   etasya           svasvataṃ tratā   \U1
%ahaṃkārasya paṃcaguṇaḥ// ahaṃ mama    etasya duḥkhaṃ// svataṃtratāḥ//  \U2
%-----------------------------
%The five qualities of the ego are: [Sense of] I, [Sense of] mine, the suffering of this, 
%-----------------------------
\note[type=source, labelb=363, lem={ahaṃkārasya pañcaguṇāḥ}]{SSP 1.45: abhimānaṃ madīyaṃ mama sukhaṃ mama duḥkhaṃ mamedam iti pañcaguṇo 'haṅkāraḥ |}
\note[type=source, labelb=364, lem={ahaṃkārasya pañcaguṇāḥ}]{Ysv\textsuperscript{PT}: ahambhāvamahañcādiyugāntaṃ hiṃsanaṃ tathā |}
\crazy{ahaṃkārasya \app{\lem[wit={ceteri}]{pañcaguṇāḥ}
  \rdg[wit={U2}]{paṃcaguṇaḥ}}/
\app{\lem[wit={ceteri}]{ahaṃ}
  \rdg[wit={B,L,P,U1}]{\om}}\dd{} 
\app{\lem[wit={ceteri}]{mama}
  \rdg[wit={U1}]{samā}
  \rdg[wit={B,L,P}]{\om}}\dd{}
\app{\lem[wit={ceteri}]{etasya}
  \rdg[wit={B,L,P}]{\om}}
\app{\lem[wit={ceteri}]{duḥkhaṃ}
  \rdg[wit={B,L,P,U1}]{\om}}\dd{}
\app{\lem[wit={E,N1,N2}]{svatantratā}
  \rdg[wit={U1}]{svasvataṃtratā}
  \rdg[wit={D,U2}]{svataṃtratāḥ}
  \rdg[wit={P,B,L}]{\om}}\dd{}}
\note[type=philcomm, labelb=365, lem={ahaṃkārasya pañcaguṇāḥ}]{Only four out of five qualities are mentioned. According to the sources the options for the missing item of the list are: \textit{abhimāna}, \textit{madīya} or \textit{hiṃsana}.}
%-----------------------------
%cittasya paṃcaguṇāḥ/ dhṛtiḥ    smṛtiḥ/ rāgadveṣau             matiḥ/  \E
%                     dhṛtiḥ    smṛtiḥ  rāgadveṣa--------------matiḥ   \P
%                     dhṛti-----smṛti---rāgadveṣam               iti/  \B
%                     dhṛti-----smṛti---rāgadveṣa------------- bhīti/  \L
%cittasya paṃcaguṇāḥ/ dhṛtiḥ/   smṛtiḥ/                 tyāgaḥ matih   \N1
%cittasya paṃcaguṇāḥ/ dhṛtiḥ    smṛtiḥ                  tyāgaḥ matiḥ// \N2
%cittasya paṃcaguṇāḥ/ dhṛtiḥ/   smṛtiḥ/                 tyāgaṃ mati    \D
%cittasya        naḥ  vṛddhiḥ                           tyāgaḥ matiḥ   \U1
%cittasya paṃcaguṇaḥ// dhṛtiḥ// smṛtiḥ// rāgaḥ// dveṣaḥ//      matiḥ// \U2
%-----------------------------
%The five qualities of the spirit are: will, memory, attachment, aversion, intention/opinion. 
%-----------------------------
\note[type=source, labelb=366, lem={cittasya pañcaguṇāḥ}]{SSP 1.48: matir dhṛtiḥ smṛtis tyāgaḥ svīkāra iti pañcaguṇaṃ cittam |}
\note[type=source, labelb=367, lem={cittasya pañcaguṇāḥ}]{Ysv\textsuperscript{PT}: vṛttiḥ smṛtir matis tyājyaṃ nirāśaṃ caitikā guṇāḥ | niḥspṛhatā dveṣatā dhairyaṃ vimarṣacintanaṃ tathā |}
\note[type=philcomm, labelb=368, lem={cittasya pañcaguṇāḥ}]{Despite the mention of \textit{rāga} and \textit{dveṣa} in the \beta -group, both qualities are absent in the sources for this passage and both items were already picked up previously for the \textit{guṇa}s of \textit{akāśa}. This indicates that the \alpha -group with the mention of \textit{tyāgaḥ} is closer to the original and \textit{svīkāra} was lost in transmission. For this reason I opted to conjecture \textit{svīkāra} into the list. Other possibilities for the missing item according to the sources would be \textit{vṛttiḥ} and \textit{nirāśaṃ}, but since the author was clearly more orientated towards the SSP for this sentence those options seem unlikely.}
\app{\lem[wit={ceteri}]{cittasya}
  \rdg[wit={B,L,P}]{\om}}
\app{\lem[wit={ceteri}]{pañcaguṇāḥ}
  \rdg[wit={U1}]{naḥ}
  \rdg[wit={B,L,P}]{\om}}/
\app{\lem[wit={ceteri}]{dhṛtiḥ}
  \rdg[wit={B,L}]{dhṛti°}
  \rdg[wit={U1}]{vṛddhiḥ}}\dd{}
\app{\lem[wit={ceteri}]{smṛtiḥ}
  \rdg[wit={B,L}]{°smṛti°}
  \rdg[wit={U1}]{\om}}\dd{}
\app{\lem[type=emendation, resp=egoscr]{svīkāraḥ}
  \rdg[wit={E}]{\korr rāgadveṣau}
  \rdg[wit={P}]{rāgadveṣa°}
  \rdg[wit={B}]{rāgadveṣam}
  \rdg[wit={L}]{°rāgadveṣa°}
  \rdg[wit={U2}]{rāgaḥ || dveṣaḥ}
  \rdg[wit={N1,N2,D,U1}]{\om}}
\app{\lem[wit={N1,N2,U1}]{tyāgaḥ}
  \rdg[wit={D}]{tyāgaṃ}
  \rdg[wit={ceteri}]{\om}}\dd{}
\app{\lem[wit={ceteri}]{matiḥ}
  \rdg[wit={D}]{mati}
  \rdg[wit={B}]{iti}
  \rdg[wit={L}]{bhīti}}\dd{}
%-----------------------------
%caitanyasya paṃcaguṇāḥ/   ārṣaṃ    vimarśaḥ   dhairyaṃ   ciṃtanaṃ  nispṛhatvam/\E
%caitanyasya guṇāḥ         harṣaḥ   vimar..    dhairyaṃ   ciṃtanaṃ  nispṛhatvaṃ\P
%caitanyasya guṇāḥ/        harṣa----vimarśa----dhairyaṃ   ciṃtanaṃ  nispṛhatvaṃ/ \B
%caitanyasya guṇāḥ/        harṣa----vimarśa----dhairyaṃ   ciṃtanaṃ  nispṛhatvaṃ...\L
%caitanyasya guṇāḥ paṃca// harṣaḥ,  vimarśaḥ,  dhairyaṃ,  ciṃtanaṃ/ nispṛhatvaṃ// \N1
%caitanyasya guṇāḥ paṃca// harṣa    vimarśa    dhairyaṃ   ciṃtanaṃ  nispṛhatvaṃ\N2
%caitanyasya guṇāḥ/        harṣaḥ   vimarśaḥ   dhairyaṃ   ciṃtanaṃ/ nispṛhatvaṃ/ \D
%caitanyasya guṇāḥ         harṣaḥ   vimarśaḥ   dhairyaṃ   cetanā    nispṛhatvaṃ   \U1
%caitanya    paṃcaguṇaḥ//  harṣaḥ// vimarṣaḥ// dhairyaṃ// cetanaṃ// nispṛhatvaṃ \U2
%-----------------------------
%The five qualities of consciousness are: excitement, reflection, understanding, thinking, desirelessness.
%-----------------------------
\note[type=source, labelb=369, lem={caitanyasya pañcaguṇāḥ}]{SSP 1.47: vimarśaḥ śīlanaṃ dhairyaṃ cintanaṃ nispṛhatvam iti pañcaguṇaṃ caitanyam |}
\note[type=testium, labelb=370, lem={caitanyasya pañcaguṇāḥ}]{Ysv\textsuperscript{PT}: niḥspṛhatā dveṣatā dhairyaṃ vimarṣacintanaṃ tathā |}
caitanyasya
\app{\lem[wit={E,U2}]{pañcaguṇāḥ}
  \rdg[wit={N1,N2}]{guṇāḥ paṃca}
  \rdg[wit={ceteri}]{guṇāḥ}}/
\app{\lem[wit={P,N1,D,U1,U2}]{harṣaḥ}
  \rdg[wit={B,L,N2}]{harṣa°}
  \rdg[wit={E}]{ārṣaṃ}}\dd{}
\app{\lem[wit={ceteri}]{vimarśaḥ}
  \rdg[wit={B,L,N2}]{°vimarśa°}
  \rdg[wit={P}]{vimar..}}\dd{}
dhairyaṃ\dd{}
\app{\lem[wit={ceteri}]{ciṃtanaṃ}
  \rdg[wit={U1}]{cetanā}
  \rdg[wit={U2}]{cetanaṃ}}\dd{}
nispṛhatvaṃ\dd{}
\end{prose}
\end{ekdosis}
\begin{ekdosis}
  \ekddiv{type=ed}
    \bigskip
    \centerline{\textrm{\small{[The Pentad of the Kula]}}}
    \bigskip
    \begin{prose}
%-----------------------------
%ataḥ    paraṃ  kulapaṃcakasya bhedāḥ kathyante/ \E
%ataḥ    paraṃ  kulapaṃcakasya bhedāḥ kathyaṃte  \P
%ataḥ    paraṃ/ kulapaṃcakasya bhedā  kathyaṃte/ \B
%ataḥ    paraṃ  kulapaṃcakasya bhedāḥ kathyaṃte/ \L
%tad anaṃtaraṃ  kulapaṃcakasya bhedāḥ kathyaṃte/ \N1
%tad anaṃtaraṃ  kulapaṃcakasya bhedāḥ kathyate/ \N2
%tad anaṃtaraṃ  kulapaṃcakasya bhedāḥ kathyaṃte// \D
%tad anaṃtaraṃ  kulapaṃcakasya bhedāḥ kathyaṃte \U1
%aṃtaḥ   paraṃ  kulapaṃcakasya bhedā  kathyaṃte// \U2
%-----------------------------
%Immediately after the divisions of the pentad of the Kūla are taught. 
%-----------------------------
\note[type=source, labelb=371, lem={kulapañcakasya}]{SSP 1.48: sattvaṃ rajas tamaḥ kālo jīva iti kulapañcakam |}
\app{\lem[wit={D,N1,N2,U2},alt={tad anaṃtaraṃ}]{tad\skp{-}anaṃtaraṃ}
  \rdg[wit={ceteri}]{ataḥ paraṃ}}
kulapañcakasya \app{\lem[wit={ceteri}]{bhedāḥ}
  \rdg[wit={B,U2}]{bhedā}}
\app{\lem[wit={ceteri}]{kathyante}
  \rdg[wit={N2}]{kathyate}}\dd{} 
%-----------------------------
%sattvaṃ  rajaḥ   tamaḥ   kālaḥ   jīvanam/ \E
%satvaṃ   rajaḥ   tamaḥ   kālaḥ   jīvanam \P
%satvaṃ   rajas   tamaḥ   kā------jīvanaṃ/ \B
%satvaṃ   rajas   tamaḥ   kāla----jīvanaṃ \L
%satva/   raja/   tamaḥ/  kālaḥ/  jīvanaṃ// \N1
%satva----raja----tama----kāla----jīvanaṃ// \N2
%satvaṃ// rajaḥ/  tamaḥ/  kālaḥ/  jīvanaṃ/ \D
%satva----raja----tama----kāla----jīvanaṃ vā \U1
%satvaṃ// rajaḥ// tamaḥ// kālaḥ// jīvanaṃ//  \U2
%-----------------------------
%Sattva, Rajas, tamas, time and the soul. 
%-----------------------------
\app{\lem[wit={ceteri}]{sattvaṃ}
  \rdg[wit={N1,N2,U1}]{satva}}\dd{}
\app{\lem[wit={ceteri}]{rajaḥ}
  \rdg[wit={B,L}]{rajas}
  \rdg[wit={N1,N2,U1}]{raja}}\dd{}
\app{\lem[wit={ceteri}]{tamaḥ}
  \rdg[wit={N2,U1}]{tama}}\dd{}
\app{\lem[wit={ceteri}]{kālaḥ}
  \rdg[wit={L,N2,U1}]{kāla}
  \rdg[wit={B}]{kā}}\dd{}
\app{\lem[wit={ceteri}]{jīvanaṃ}
  \rdg[wit={E,P}]{jīvanam}}\dd{}
%-----------------------------
%tatra    sattva----guṇāḥ/  dayā   dharmaḥ   kṛpā   bhaktiḥ   śraddhā ceti/  \E
%tatra    satvasya  guṇāḥ   dayā   dharmaḥ   kṛpā   bhaktiḥ   śraddhā ceti   \P
%tatrasya satva-----guṇāḥ   dayāḥ  dharma----kṛpā   bhakti----śraddhā ceti/  \B
%tatra    satva-----guṇāḥ// dayāḥ  dharma----kṛpā   bhakti----śraddhā ceti/  \L
%tatra    satvasya  guṇāḥ// dayā,  dharma,   kṛpā/  bhaktiḥ/  śraddhā ceti// \N1
%tatra    satvasya  guṇāḥ// dayā   dharma    kṛpā   bhakti    śraddhā ceti// \N2
%tatra    sattvasya guṇāḥ/  dayā   dharmaḥ   kṛpā// bhaktiḥ/  śraddhā ceti/  \D
%tatra    satvasya  guṇāḥ   dayā   dharma----kṛpā---bhaktiḥ   śraddhā        \U1
%tatra    satvasya  guṇāḥ// dayā// dharmaḥ// kṛpā// bhaktiḥ// śraddhā ceti// \U2
%-----------------------------
%In the case of Sattva the qualities are: compassion, religious duty, pity, devotion and confidence. 
%-----------------------------
\note[type=source, labelb=372, lem={sattvasya guṇāḥ}]{Ysv\textsuperscript{PT}: citter guṇās trayo jīvaguṇān śṛṇu maheśvari | āsthā śraddhā kṛpā bhaktiḥ satyaṃ sattvaguṇā iti |}
\note[type=source, labelb=373, lem={sattvasya guṇāḥ}]{SSP 1.49: dayā dharmaḥ kriyā bhaktiḥ śraddheti pañcaguṇaṃ sattvam |}
\app{\lem[wit={ceteri}]{tatra}
  \rdg[wit={B}]{tatrasya}}
\app{\lem[wit={ceteri}]{sattvasya}
  \rdg[wit={E,B,L}]{sattva}}
guṇāḥ/
\app{\lem[wit={ceteri}]{dayā}
  \rdg[wit={B,L}]{dayāḥ}}\dd{}
\app{\lem[wit={ceteri}]{dharma}
  \rdg[wit={E,P,U2}]{dharmaḥ}}\dd{}
kṛpā\dd{}
\app{\lem[wit={ceteri}]{bhaktiḥ}
  \rdg[wit={B,L,N2}]{bhakti}}\dd{}
śraddhā
\app{\lem[wit={ceteri}]{ceti}
  \rdg[wit={U1}]{\om}}
%-----------------------------
%rajaso guṇāḥ/ tyāgaḥ/  bhogaḥ   śṛṃgāraḥ   svārthaḥ/  vastusaṃgrahaś ceti// \E [p.72]
%rajaso guṇāḥ  tyāgaḥ   bhedaḥ   śṛṃgāraḥ   svārthaḥ   vastusaṃgrahaḥ \P
%rajaso guṇāḥ/ tyāgaḥ/  bhogaḥ   śṛṃgāraḥ   svārtha----vastunā saṃgrahaḥ// \B
%rajaso guṇāḥ  tyāgaḥ   bhogaḥ   śṛṃgāraḥ   svārtha----vastunāṃ saṃgrahaḥ// \L
%rajaso guṇāḥ//tyāgaḥ/  bhogaḥ/  śṛṃgāraḥ/  svārthaḥ/  vastusaṃgrahaḥ/ \N1
%rajaso guṇāḥ//tyāga    bhoga    śṛṃgāraḥ   svārtha    vastusaṃgrahaḥ// \N2
%rajaso guṇāḥ/ tyāgaḥ   bhogaḥ   śṛṃgāraḥ   svārthaḥ/  vastusaṃgrahaḥ \D
%rajaso guṇāḥ  tyāgaḥ            śṛṃgāraḥ   svārtha----vastusaṃgrahaḥ \U1
%rajo   guṇaḥ//tyāgaḥ// bhogaḥ// śṛṃgāraḥ// svārthaḥ// vastusaṃgrahaḥ// \U2
%-----------------------------
%The qualities of Rajas are: renunciation, enjoyment, sexuality, self-interest and accumulation of possessions. 
%-----------------------------
\note[type=source, labelb=374, lem={rajaso guṇāḥ}]{SSP 1.50: dānaṃ bhogaḥ śṛṅgāro vastugrahaṇaṃ svārthasaṃgrahaṇam iti pañcaguṇaṃ rajaḥ |}
\note[type=source, labelb=375, lem={rajaso guṇāḥ}]{Ysv\textsuperscript{PT}: tyāgo bhogaś ca śraddhā ca sārthavastuspṛhā tathā | raso pañcaguṇāḥ caite tāmasasya guṇān śṛṇu |}
\app{\lem[wit={ceteri}]{rajaso}
  \rdg[wit={U2}]{rajo}}
guṇāḥ/
\app{\lem[wit={ceteri}]{tyāgaḥ}
  \rdg[wit={N2}]{tyāga}}\dd{}
\app{\lem[wit={ceteri}]{bhogaḥ}
  \rdg[wit={N2}]{bhoga}
  \rdg[wit={P}]{bheda}
  \rdg[wit={U1}]{\om}}\dd{}
śṛṇgāraḥ\dd{}
\app{\lem[wit={ceteri}]{svārthaḥ}
  \rdg[wit={B,L,N2,U1}]{svārtha}}\dd{}
\app{\lem[wit={ceteri}]{vastusaṃgrahaḥ}
  \rdg[wit={L}]{vastunāṃ saṃgrahaḥ}
  \rdg[wit={B}]{vastunā saṃgrahaḥ}
  \rdg[wit={E}]{vastusaṃgrahaś ceti}}\dd{}
\end{prose}
\end{ekdosis}
\ekdpb*{}
%%%%%%%%%%%%%%%%%%%%%%%%%%%%%%%%%%%%%%%%%%
%%%%%%%%PAGEBREAK%%%%%%%PAGEBREAK%%%%%%%%%
%%%%%%%%%%%%%%%%%%%%%%%%%%%%%%%%%%%%%%%%%%
%%%%%%%%%%%%%%%%PAGEBREAK%%%%%%%%%%%%%%%%%
%%%%%%%%%%%%%%%%%%%%%%%%%%%%%%%%%%%%%%%%%%
%%%%%%%%PAGEBREAK%%%%%%%PAGEBREAK%%%%%%%%%
%%%%%%%%%%%%%%%%%%%%%%%%%%%%%%%%%%%%%%%%%%
%%%%%%%%%%%%%%%%%%%%%%%%%%%%%%%%%%%%%%%%%%
%%%%%%%%%%%%%%%%%%%%%%%%%%%%%%%%%%%%%%%%%%
%%%%%%%%%%%%%%%%%%%%%%%%%%%%%%%%%%%%%%%%%%
%%%%%%%%PAGEBREAK%%%%%%%PAGEBREAK%%%%%%%%%
%%%%%%%%%%%%%%%%%%%%%%%%%%%%%%%%%%%%%%%%%%
%%%%%%%%%%%%%%%%PAGEBREAK%%%%%%%%%%%%%%%%%
%%%%%%%%%%%%%%%%%%%%%%%%%%%%%%%%%%%%%%%%%%
%%%%%%%%PAGEBREAK%%%%%%%PAGEBREAK%%%%%%%%%
%%%%%%%%%%%%%%%%%%%%%%%%%%%%%%%%%%%%%%%%%%
%%%%%%%%%%%%%%%%%%%%%%%%%%%%%%%%%%%%%%%%%%
%%%%%%%%%%%%%%%%%%%%%%%%%%%%%%%%%%%%%%%%%%
%%%%%%%%%%%%%%%%%%%%%%%%%%%%%%%%%%%%%%%%%%
%%%%%%%%PAGEBREAK%%%%%%%PAGEBREAK%%%%%%%%%
%%%%%%%%%%%%%%%%%%%%%%%%%%%%%%%%%%%%%%%%%%
%%%%%%%%%%%%%%%%PAGEBREAK%%%%%%%%%%%%%%%%%
%%%%%%%%%%%%%%%%%%%%%%%%%%%%%%%%%%%%%%%%%%
%%%%%%%%PAGEBREAK%%%%%%%PAGEBREAK%%%%%%%%%
%%%%%%%%%%%%%%%%%%%%%%%%%%%%%%%%%%%%%%%%%%
\begin{ekdosis}
  \begin{prose}
    \noindent
%-----------------------------
%tamaso guṇāḥ   vivādaḥ   kalahaḥ   śokaḥ   baṃdhaḥ   vañcanam/         \E
%tamaso guṇāḥ   vivādaḥ   kalahaḥ   śokaḥ   baṃdhaḥ   vaṃcanaṃ          \P
%tamaso guṇāḥ/  vivādaḥ   kalaha----śoka/   baṃdha----vaṃcanaṃ/         \B
%tamo  guṇāḥ//  vivādaḥ   kalaha----śokaiḥ  baṃdha----vaṃcanaṃ          \L
%tamaso guṇāḥ// vivādaḥ   kalahaṃ/  śokaḥ/  baṃdhaḥ/  vaṃcanaṃ/         \N1
%tamo  guṇāḥ//  vivāda    kalahaṃ   śoka    vidha vā  vaṃcanaṃ smṛtaṃ// \N2
%tamaso guṇāḥ   vivādaḥ/  kalahaṃ/  śokaḥ/  baṃdhaḥ/  vaṃcanaṃ/          \D
%tamaso  guṇaḥ  vivādaḥ   kalaha----śoka    baṃdha----vaṃcanā            \U1
%tamo guṇaḥ//   viṣādaḥ// kalahaḥ// śokaḥ// baṃdhaḥ// caṃcalaṃ ceti//    \U2  %%%427.jpg 
%-----------------------------
%The qualities of Tamas are: conflict, struggle, grief, bond, cheating 
%-----------------------------
\note[type=source, labelb=376, lem={tamaso guṇāḥ}]{SSP 1.51: vivādaḥ kalahaḥ śoko baṃdho vañcanam iti pañcaguṇaṃ tamaḥ |}
\note[type=testium, labelb=377, lem={tamaso guṇāḥ}]{Ysv\textsuperscript{PT}: pramodaḥ svādakalahau vivādo bhrāntivarddhanam | vañcanañ ca tathā śokas tāmasasya guṇā ime |}
\app{\lem[wit={ceteri}]{tamaso}
  \rdg[wit={L,N2,U2}]{tamo}}
\app{\lem[wit={ceteri}]{guṇāḥ}
  \rdg[wit={U2}]{guṇaḥ}}/
\app{\lem[wit={ceteri}]{vivādaḥ}
  \rdg[wit={N2}]{vivāda}}\dd{}
\app{\lem[wit={E,P,U2}]{kalahaḥ}
  \rdg[wit={D,N1,N2}]{kalahaṃ}
  \rdg[wit={B,L,U1}]{kalaha}}\dd{}
\app{\lem[wit={D,E,P,N1,U2}]{śokaḥ}
  \rdg[wit={B,N2,U1}]{śoka}
  \rdg[wit={L}]{śokaiḥ}}\dd{}
\app{\lem[wit={ceteri}]{bandhaḥ}
  \rdg[wit={B,L,U1}]{baṃdha}
  \rdg[wit={N2}]{vidha vā}}\dd{}
\app{\lem[wit={ceteri}]{vañcanam}
  \rdg[wit={N2}]{vaṃcanaṃ smṛtaṃ}
  \rdg[wit={U1}]{vaṃcanā}
  \rdg[wit={U2}]{caṃcalaṃ ceti}}\dd{}
%-----------------------------
%              kālasya   guṇāḥ   kalanā   kalma  ṣaṃbhrāntiḥ   prasādaḥ   unmādaḥ/       \E
%              kālasya   guṇāḥ   kalanā   kalpaḥ    bhrāṃtiḥ   prasādaḥ   unmādaḥ        \P
%              kālasya   guṇāḥ/  kalanā   kalpanā   bhrāṃti----pramādaḥ   unmādaḥ/       \B
%              kālasya   guṇāḥ// kalanā   kalpanā   bhrāṃtiḥ   pramādaḥ   unmādaḥ...     \L
%tad anaṃtaraṃ kālasya   guṇāḥ// kalanā/  kalpanā/  bhrāṃtiḥ/  pramādaḥ/  unmādaḥ/       \N1
%tad anaṃtaraṃ kālasya   guṇāḥ// \om                                                     \N2
%tad anaṃtaraṃ kāraṇasya guṇāḥ/  kalanā/  kalpanā/  bhrāṃtiḥ/  pramādaḥ/  unmādaḥ/       \D
%tad anaṃtaraṃ kāla------guṇāḥ   kalanā   kalpanā   bhrāṃti    pramādaḥ   unmādaḥ        \U1
%              kālasya   guṇāḥ// kalanā// kalpanā// bhrāṃtiḥ// pramādaḥ// unmādaś ceti// \U2
%-----------------------------
%Furthermore the qualities of time are: inciting, arranging, moving around, neglience [and] mental disorder. 
%-----------------------------
\note[type=source, labelb=378, lem={kālasya guṇāḥ}]{SSP 1.52: kalanā kalpanā bhrāntiḥ pramādo 'nartha iti pañcaguṇaḥ kālaḥ |}
\app{\lem[wit={D,N1,N2,U1},alt={tad anaṃtaraṃ}]{tad\skp{-}anaṃtaraṃ}
  \rdg[wit={ceteri}]{\om}}
\app{\lem[wit={ceteri}]{kālasya}
  \rdg[wit={U1}]{kāla°}
  \rdg[wit={D}]{kāraṇasya}}
guṇāḥ/
\app{\lem[wit={ceteri}]{kalanā}
  \rdg[wit={N2}]{\om}}\dd{}
\app{\lem[wit={ceteri}]{kalpanā}
  \rdg[wit={P}]{kalpaḥ}
  \rdg[wit={E}]{kalma°}
  \rdg[wit={N2}]{\om}}\dd{}
\app{\lem[wit={ceteri}]{bhrāntiḥ}
  \rdg[wit={B,U1}]{bhrāṃti°}
  \rdg[wit={E}]{ṣaṃbhrāntiḥ}
  \rdg[wit={N2}]{\om}}\dd{}
\app{\lem[wit={ceteri}]{pramādaḥ}
  \rdg[wit={E,P}]{prasādaḥ}
  \rdg[wit={N2}]{\om}}\dd{}
\app{\lem[wit={ceteri}]{unmādaḥ}
  \rdg[wit={U2}]{unmādaś ceti}
  \rdg[wit={N2}]{\om}}\dd{}
%-----------------------------
%jīvasya guṇāḥ   jāgradavasthā    svapnāvasthā   suṣuptāvasthā   turīyāvasthā/     turīyātītāvasthā     \E
%jīvasya guṇāḥ   jāgradavasthā    svapnāvasthā   suṣuptāvasthā   turīyāvasthā      turīyātītāvasthā    \P %%%7673.jpg 
%jīvasya guṇāḥ/  jāgravadasthāḥ   svapnāvasthā   suṣupta---------turyāvasthā/      turiyā/ tītāvasthā    \B
%jīvasya guṇāḥ// jāgradavasthā    svapnāvasthā   suṣupti---------turyāvasthā//     turiyātītāvasthā//    \L
%jīvasya guṇāḥ/  jāgravadasthā/   svapnāvasthā/  suṣuptavasthā/  turīyāvasthā/     turiyātītāvasthā//    \N1
%                jāgradavadasthā/ svapnāvasthā/  suṣumṇāvasthā/  turīyāvasthā/     turiyātītāvasthā//  \N2
%jīvasya guṇā//  jāgradavasthā/   svapnāvasthā/  suṣuptavasthā/  turīyāvayāvasthā/ turiyātītāvasthā//\D
%jīvasya guṇāḥ   jāgṛdavasthā     svapnāvasthā   suṣuptāvasthā   turyāvasthā       turiyātītāvasthā// kaivalyā \U1
%jīvasya guṇaḥ// jāgradavasthā//  svapnāvasthā// suṣuptāvasthā// turīyāvasthā//    turiyātītāvasthā//\U2
%-----------------------------
%The qualities of the living soul are: the state of waking, the state of sleeping, the state of deep sleep, the state of liberation, 
%-----------------------------
\note[type=source, labelb=379, lem={jīvasya guṇāḥ}]{SSP 1.53: jāgrat svapnaḥ suṣuptis turyaṃ tūryātītam iti pañcāvasthāguṇo jīvaḥ |}
\note[type=testium, labelb=380, lem={jīvasya guṇāḥ}]{Ysv\textsuperscript{PT}: svapnajāgratsuṣuptāni caitanyaṃ jīvakā guṇāḥ | etādṛśi sati tattvaṃ caitanyāt tad bhaved iti |}
\app{\lem[wit={ceteri}]{jīvasya}
  \rdg[wit={N2}]{\om}}
\app{\lem[wit={ceteri}]{guṇāḥ}
  \rdg[wit={D}]{guṇā}
  \rdg[wit={U2}]{guṇaḥ}
  \rdg[wit={N2}]{\om}}/
\app{\lem[wit={E,P,L,D,U2}]{jāgradavasthā}
  \rdg[wit={B}]{jāgravadasthāḥ}
  \rdg[wit={N1}]{jāgravadasthā}
  \rdg[wit={N2}]{jāgradavadasthā}
  \rdg[wit={U1}]{jāgṛdavasthā}}\dd{}
svapnāvasthā\dd{}
\app{\lem[wit={ceteri}]{suṣuptāvasthā}
  \rdg[wit={B}]{suṣupta°}
  \rdg[wit={L}]{suṣupti°}}\dd{}
\app{\lem[wit={ceteri}]{turīyāvasthā}
  \rdg[wit={D}]{turīyāvayāvasthā}
  \rdg[wit={B,L,U1}]{turyāvasthā}}\dd{}
\app{\lem[wit={ceteri}]{turīyātītāvasthā}
  \rdg[wit={B}]{turiyā | tītāvasthā}
  \rdg[wit={U1}]{turiyātītāvasthā || kaivalyā}}\dd{}
%-----------------------------
%tad anaṃtaram etādṛśam eka--jñānam utpadyate/ \E
%tad anaṃtaram etādṛśam eka--jñānam utpadyate  \P
%tad anaṃtaram etādṛśam ekaṃ jñānam utpadyate/ \B
%tad anaṃtaram etādṛśam ekaṃ jñānam utpadyate// \L 0036.jpg
%tad anaṃtaram etādṛśam ekaṃ jñānam utpadyate/ \N1
%tad anaṃtaram etādṛśam eka--jñānam utpadyate// \N2
%tad anaṃtaraṃ etādṛśam ekaṃ jñānam utpadyate/ \D
%tad anaṃtaram etādṛśaṃ ekaṃ jñānam utpadyate \U1
%tad anaṃtaram etādṛśom ekaṃ jñānam utpadyate// \U2
%-----------------------------
%Furthermore such unique knowledge is generated:
%-----------------------------
tad-anaṃtara\skp{m-e}\app{\lem[wit={ceteri},alt={etādṛśam}]{\skm{m-e}tādṛśa\skp{m-e}}
  \rdg[wit={U2}]{etādṛśom}
}\app{\lem[wit={ceteri},alt={ekaṃ}]{\skm{m-e}kaṃ}
  \rdg[wit={E,P,N2}]{eka}}
jñānam-utpadyate/
%-----------------------------
%                                            icchāyāḥ paṃcaguṇāḥ/  unmany  avasthā/   vāṃchā   cittaṃ veṣṭanam vibhramaḥ/  \E
%icchā    kriyā   māyā   prakṛtiḥ   vācāḥ    ichāyā   paṃcaguṇāḥ   unmaya   vāsanā    vāṃcha   vittaṃ   ceṣṭa \P
%icchā    kriyā   māyā   prakṛtī    vācāḥ/   ichāyā   paṃcaguṇāḥ   unmayā   vāsanā/   vāṃcha   krittaṃ  ccoṣṭhā/ \B
%icchā    kriyā   māyā   prakṛti    vācyaḥ// ichāyā   paṃcaguṇāḥ// unmany  avāsanā//  vāṃcha   cittaṃ   ceṣṭa \L
%icchāyāḥ         māyā/  prakṛtiḥ/  vāca//   icchayāḥ paṃcaguṇāḥ// unmany/  vāsanā/   vāṃcchā/ caittaṃ/ ceṣṭā//  \N1
%icchā/   kriyā/  māyā/  prakṛtiḥ/  vāca//   icchayā  paṃcaguṇāḥ// unmany/  vāsanā/   vāṃcchā/ caittaṃ/ ceṣṭā//  \N2
%icchā    kriyā   māyā/  prakṛtiḥ// vāca//   icchayāḥ paṃcaguṇāḥ// unmany/  vāsana/   vāṃchā/  caita/   ceṣṭā// \D
%ichā     kriyā   māyā   prakṛti----vāca     ichāyāḥ  paṃcaguṇāḥ   unmany                                         \U1
%icchā//  kriyā// māyā// prakṛtiḥ// bhāvaḥ// icchāyāḥ paṃcaguṇāḥ// unmanyaṃ vāsanāḥ// vāṃchā// cittaḥ// ceṣṭāḥ//  \U2
%-----------------------------
%Desire, action, illusion, nature, speech. The five qualities of desire are: madness, mental imprint, wish, thinking, activity. 
%-----------------------------
\note[type=source, labelb=381, lem={icchā}]{SSP 1.54: icchā kriyā māyā prakṛtir vāg iti vyaktaśaktipañcakam |}
\note[type=source, labelb=382, lem={icchā}]{Ysv\textsuperscript{PT}: prakṛtīcchā kriyā māyā vacaḥ pañca guṇā iti |}
\note[type=source, labelb=383, lem={icchayāḥ pañcaguṇāḥ}]{SSP 1.55: unmādo vāsanā vāñchā cintā ceṣṭeti pañcaguṇecchā |}
\note[type=source, labelb=384, lem={icchayāḥ pañcaguṇāḥ}]{Ysv\textsuperscript{PT}: āśātṛṣṇāspṛhākāṅkṣāmithyāntaṃ prakṛter iti | unmādo vāsanā vāñchā cekṣitā ca guṇāḥ priye |}
\app{\lem[wit={ceteri}]{icchā}
  \rdg[wit={N1}]{icchāyāḥ}
  \rdg[wit={E}]{\om}}\dd{}
\app{\lem[wit={ceteri}]{kriyā}
  \rdg[wit={E,N1}]{\om}}\d{}
\app{\lem[wit={ceteri}]{māyā}
  \rdg[wit={E}]{\om}}\dd{}
\app{\lem[wit={ceteri}]{prakṛtiḥ}
  \rdg[wit={P}]{prakṛtī}
  \rdg[wit={U1}]{prakṛti°}
  \rdg[wit={E}]{\om}}\dd{}
\app{\lem[type=emendation, resp=egoscr]{vācā}
  \rdg[wit={D,N1,N2,U1}]{\korr vāca}
  \rdg[wit={P,B}]{vācāḥ}
  \rdg[wit={L}]{vācyaḥ}
  \rdg[wit={U2}]{bhāvaḥ}
  \rdg[wit={E}]{\om}}\dd{}
\app{\lem[wit={E,N1,D,U1,U2}]{icchayāḥ}
  \rdg[wit={B,L,P}]{ichāyā}
  \rdg[wit={N2}]{icchayā}}/
\app{\lem[type=conjecture, resp=egoscr]{unmādaḥ}
  \rdg[wit={D,E,L,N1,N2,U1}]{\conj unmany}
  \rdg[wit={P}]{unmaya}
  \rdg[wit={B}]{unmayā}
  \rdg[wit={U2}]{unmanyaṃ}}
\note[type=philcomm, labelb=385, lem={unmādaḥ}]{Since the reading preserved in all witnesses in clearly corrupted, I conjectured according to the sources.}
\app{\lem[wit={ceteri}]{vāsanā}
  \rdg[wit={L}]{avāsanā}
  \rdg[wit={U2}]{vāsanāḥ}
  \rdg[wit={E}]{avasthā}
  \rdg[wit={U1}]{ichā kriyā māyā prakṛti vāca ichāyāḥ paṃcaguṇāḥ unmany}}\dd{}
\app{\lem[wit={ceteri}]{vāñchā}
  \rdg[wit={B,L,P}]{vāṃcha}
  \rdg[wit={U1}]{\om}}\dd{}
\app{\lem[wit={E,L}]{cittaṃ}
  \rdg[wit={N1,N2}]{caittaṃ}
  \rdg[wit={D}]{caita}
  \rdg[wit={B}]{krittaṃ}
  \rdg[wit={P}]{vittaṃ}
  \rdg[wit={U1}]{\om}}\dd{}
\app{\lem[wit={N1,N2,D}]{ceṣṭā}
  \rdg[wit={P,L}]{ceṣṭa}
  \rdg[wit={U2}]{ceṣṭāḥ}
  \rdg[wit={B}]{ccoṣṭhā}
  \rdg[wit={E}]{veṣṭanam vibhramaḥ}
  \rdg[wit={U1}]{\om}}\dd{}
%-----------------------------
%kriyāyāḥ paṃcaguṇāḥ/   smaraṇaṃ   udyamaḥ   udvegaḥ/  kāryaniścayaḥ/  satkulācāratvam//  \E
%kriyāyāḥ paṃcaguṇāḥ/   smaraṇaṃ   udyamaḥ   udvega----kāryaniścayaḥ   satkulācāratvaṃ    \P
%kriyāyā  paṃcaguṇāḥ/   smaraṇaṃ   udyamaḥ   udvega----kāryaniścayaḥ   satkulācāratvaṃ/.... [DSCN7177.JPG, Z.1] \B
%kriyāyā  paṃcaguṇāḥ/   smaraṇaṃ   udyamaḥ   udvega----kāryaniścayaḥ   satkulācāratvaṃ... \L
%kriyāyāḥ paṃcaguṇāḥ/   smaraṇaṃ/  udyamaḥ/  udvegaḥ/  kārya/ niścayaḥ/satkulācāratvam//  \N1
%kriyāyā  paṃcaguṇāḥ/   smaraṇaṃ/  udyama/   udvega/   kāryaniścayaḥ   satkulācāratvam//  \N2
%kriyāyāḥ paṃcaguṇāḥ/   smaraṇaṃ/  udyamaḥ   udvegaḥ/  kāryaniścayaḥ/  satkulācāratvaṃ/   \D
%                                                                                         \U1
%kriyāyāḥ paṃcaguṇaḥ//  smaraṇaṃ// udyamaḥ// udvegaḥ// kāryaniścayaḥ// satkulācāratvaṃ//  \U2
%-----------------------------
%The five qualities of action are: memory, effort, agitation, decision about the activity, adherence to the conduct of the real family. 
%-----------------------------
\note[type=source, labelb=385, lem={kriyāyāḥ pañcaguṇāḥ}]{SSP 1.56: smaraṇam udyogaḥ kāryaṃ niścayaḥ svakulācāra iti pañcaguṇā kriyā |}
\note[type=testium, labelb=386, lem={kriyāyāḥ pañcaguṇāḥ}]{śaraṇaṃ satkulācāraḥ kāryaniścaya ucyate | idaṃ yogarahasyañ ca na vācyaṃ mūrkhasannidhau |}
\note[type=philcomm, labelb=387, lem={kriyāyāḥ pañcaguṇāḥ}]{This list and the three lists that follow are \om in U\textsubscript{1}. The omissions will not be recorded in the \textit{apparatus criticus}.}
\app{\lem[wit={ceteri}]{kriyāyāḥ}
  \rdg[wit={B,L,N2}]{kriyāyā}}
pañcaguṇāḥ/
smaraṇaṃ\dd{}
\app{\lem[wit={ceteri}]{udyamaḥ}
  \rdg[wit={N2}]{udyama}}\dd{}
\app{\lem[wit={D,E,N1,U2}]{udvegaḥ}
  \rdg[wit={B,L,P,N2}]{udvega}}\dd{}
\app{\lem[wit={ceteri}]{kāryaniścayaḥ}
  \rdg[wit={N1}]{kārya | niścayaḥ}}\dd{}
satkulācāratvaṃ\dd{}
%-----------------------------
%māyāyāḥ paṃcaguṇāḥ/  mada----mātsaryādayaḥ/        kīrtiḥ      asatyabhāvāḥ/   \E [S.73]
%māyāyāṃ      guṇāḥ   madaḥ   mātsaryaṃ   daṃbhaḥ   kīrtiḥ      asatyabhāvaḥ    \P
%māyāyāḥ paṃcaguṇāḥ/  madaḥ   mātsarya----raṃbhaḥ/  kīrtiḥ      asatyabhāvaḥ/   \B
%māyāyā  paṃcaguṇāḥ   madaḥ   mātsarya----raṃbhaḥ   kīrtiḥ      asatyabhāvaḥ... \L
%māyāyā       guṇāḥ/  madaḥ/  mātsaryaḥ/  daṃbhaḥ/  kīrtiś ca// asatyabhāvaḥ//  \N1
%māyāyā       guṇāḥ/  mada/   mātsarya/   daṃbha/   kīrtiś ca// asatyabhāvaḥ//  \N2
%māyāyā       guṇā//  madaḥ/  mātsaryaḥ/  daṃbhaḥ/  kīrtiś ca/  asatyabhāvaḥ/   \D
%\om \U1
%māyāyāḥ      guṇāḥ// madaḥ// mātsaryaṃ// daṃbhaḥ// kīrttiḥ//   asatyabhāvaḥ//  \U2
%-----------------------------
%The qualities of illusion are: intoxication, envy, fraud, fame, the state of untruth. 
%-----------------------------
\note[type=source, labelb=388, lem={māyāyāḥ pañcaguṇāḥ}]{SSP 1.57: mado mātsaryaṃ dambhaḥ kṛtrimatvam asatyam iti pañcaguṇā māyā |}
\app{\lem[wit={E,B,U2}]{māyāyāḥ}
  \rdg[wit={P}]{māyāyāṃ}
  \rdg[wit={ceteri}]{māyāyā}}
\app{\lem[wit={B,E,L}]{pañcaguṇāḥ}
  \rdg[wit={P,N1,N2,U2}]{guṇāḥ}
  \rdg[wit={D}]{guṇā}}/
\app{\lem[wit={ceteri}]{madaḥ}
  \rdg[wit={E,N2}]{mada}}\dd{}
\app{\lem[wit={N1,D}]{mātsaryaḥ}
  \rdg[wit={P,U2}]{mātsaryaṃ}
  \rdg[wit={B,L,N2}]{mātsarya}
  \rdg[wit={E}]{mātsaryādayaḥ}}\dd{}
\app{\lem[wit={ceteri}]{daṃbhaḥ}
  \rdg[wit={B,L}]{raṃbhaḥ}
  \rdg[wit={N2}]{daṃbha}}\dd{}
\app{\lem[wit={ceteri}]{kīrtiḥ}
  \rdg[wit={D,N1,N2}]{kīrtiś ca}}\dd{}
\app{\lem[wit={ceteri}]{asatyabhāvaḥ}
  \rdg[wit={E}]{asatyabhāvāḥ}}\dd{}
\end{prose}
\end{ekdosis}
\ekdpb*{}
%%%%%%%%%%%%%%%%%%%%%%%%%%%%%%%%%%%%%%%%%%
%%%%%%%%PAGEBREAK%%%%%%%PAGEBREAK%%%%%%%%%
%%%%%%%%%%%%%%%%%%%%%%%%%%%%%%%%%%%%%%%%%%
%%%%%%%%%%%%%%%%PAGEBREAK%%%%%%%%%%%%%%%%%
%%%%%%%%%%%%%%%%%%%%%%%%%%%%%%%%%%%%%%%%%%
%%%%%%%%PAGEBREAK%%%%%%%PAGEBREAK%%%%%%%%%
%%%%%%%%%%%%%%%%%%%%%%%%%%%%%%%%%%%%%%%%%%
%%%%%%%%%%%%%%%%%%%%%%%%%%%%%%%%%%%%%%%%%%
%%%%%%%%%%%%%%%%%%%%%%%%%%%%%%%%%%%%%%%%%%
%%%%%%%%%%%%%%%%%%%%%%%%%%%%%%%%%%%%%%%%%%
%%%%%%%%PAGEBREAK%%%%%%%PAGEBREAK%%%%%%%%%
%%%%%%%%%%%%%%%%%%%%%%%%%%%%%%%%%%%%%%%%%%
%%%%%%%%%%%%%%%%PAGEBREAK%%%%%%%%%%%%%%%%%
%%%%%%%%%%%%%%%%%%%%%%%%%%%%%%%%%%%%%%%%%%
%%%%%%%%PAGEBREAK%%%%%%%PAGEBREAK%%%%%%%%%
%%%%%%%%%%%%%%%%%%%%%%%%%%%%%%%%%%%%%%%%%%
%%%%%%%%%%%%%%%%%%%%%%%%%%%%%%%%%%%%%%%%%%
%%%%%%%%%%%%%%%%%%%%%%%%%%%%%%%%%%%%%%%%%%
%%%%%%%%%%%%%%%%%%%%%%%%%%%%%%%%%%%%%%%%%%
%%%%%%%%PAGEBREAK%%%%%%%PAGEBREAK%%%%%%%%%
%%%%%%%%%%%%%%%%%%%%%%%%%%%%%%%%%%%%%%%%%%
%%%%%%%%%%%%%%%%PAGEBREAK%%%%%%%%%%%%%%%%%
%%%%%%%%%%%%%%%%%%%%%%%%%%%%%%%%%%%%%%%%%%
%%%%%%%%PAGEBREAK%%%%%%%PAGEBREAK%%%%%%%%%
%%%%%%%%%%%%%%%%%%%%%%%%%%%%%%%%%%%%%%%%%%
\begin{ekdosis}
  \begin{prose}
    \noindent
%-----------------------------
%prakṛteḥ  paṃcaguṇāḥ   āśā   tṛṣṇā   spṛhā   kāṃkṣā  mithyātvam/  \E
%prakṛter       guṇāḥ   āśā   tṛṣṇā   spṛhā   bhikṣā  mithyātvaṃ   \P
%prakṛte        guṇāḥ/  āśā   tṛṣṇā   spṛhā   kāṃkṣā  mithyātvaṃ/  \B
%prakṛte        guṇāḥ/  āśā   tṛṣṇā   spṛhā   kāṃkṣā  mithyātvaṃ// \L
%prakṛte        guṇāḥ/  āśā/  tṛṣṇā/  spṛhā/  kāṃkṣā/ mithyātvaṃ// \N1
%prakṛte        guṇāḥ// āśā/  tṛṣṇā/  spṛhā/  kāṃkṣā/ mithyātvaṃ// \N2
%prakṛte        guṇāḥ/  āśā   tṛṣṇā// spṛhā/  kākṣā   mithyātvaṃ// \D
%\om \U1
%prakṛter       guṇāḥ// āśā// tṛṣṇā// spṛhā// kāṃkṣā  mithyātvaṃ// \U2
%-----------------------------
%The five qualities of nature are: space, thrirst, desire, striving [and] infatuation. 
%-----------------------------
\note[type=source, labelb=389, lem={prakṛteḥ pañcaguṇāḥ}]{SSP 1.58: āśā tṛṣṇā spṛhā kāṃkṣā mithyeti pañcaguṇā prakṛtiḥ |}
\app{\lem[wit={E}]{prakṛteḥ}
  \rdg[wit={P,U2}]{prakṛter}
  \rdg[wit={ceteri}]{prakṛte}}
\app{\lem[wit={E}]{pañcaguṇāḥ}
  \rdg[wit={ceteri}]{guṇāḥ}}/
āśā\dd{}
tṛṣṇā\dd{}
spṛhā\dd{}
\app{\lem[wit={ceteri}]{kāṃkṣā}
  \rdg[wit={D}]{kākṣā}
  \rdg[wit={P}]{bhikṣā}}\dd{}
mithyātvaṃ\dd{}
%-----------------------------
%vācāyā paṃcaguṇāḥ/  parā   paśyantī   madhyamā   vaikharī/  mātṛkā  \E
%vācāyā      guṇāḥ   parā   paśyaṃtī   madhyamā   vaikharī   mātṛkā  \P
%vācāyā paṃcaguṇāḥ/  parā   paśyanti   madhyamā   vaikharī   mātṛkā  \B
%vācāyā paṃcaguṇāḥ/  parā   paśyanti   madhyamā   vaikharī   mātṛkā  \L
%vācāyā      guṇāḥ/  parā/  paśyanti,  madhyamā,  vaikharī/  mātṛkā  \N1 %usage of comma!
%vācāyā      guṇāḥ/  parā/  paśyanti/  madhyamā/  vaikharī/  mātṛkā  \N2
%vācā        guṇāḥ// parā/  paśyantī   madhyamā/  vaikharī/  mātṛkā  \D
%\om                                              vaikharī   mātṛkā  \U1
%vācaḥ  paṃcaguṇaḥ// parā// paśyaṃti// madhyamā// vaikharī// mātṛkāḥ//  \U2
%-----------------------------
%The five qualities of speech are: Parā, Paśyantī, Madhyamā, Vaikharī [and] Mātṛkā. 
%-----------------------------
\note[type=source, labelb=390, lem={vācāyā pañcaguṇāḥ}]{SSP 1.59: parā paśyantī madhyamā vaikharī mātṛketi pañcaguṇā vāk | iti vyaktiśaktipañcaviṃśatiguṇāḥ |}
\app{\lem[wit={ceteri}]{vācāyā}
  \rdg[wit={D}]{vācā}
  \rdg[wit={U2}]{vācaḥ}}
\app{\lem[wit={B,E,L}]{pañcaguṇāḥ}
  \rdg[wit={U2}]{pañcaguṇaḥ}
  \rdg[wit={ceteri}]{guṇāḥ}}/
parā\dd{}
\app{\lem[wit={ceteri}]{paśyantī}
  \rdg[wit={B,L,N1,N2,U2}]{paśyanti}}\dd{}
madhyamā\dd{}
vaikharī\dd{}
\app{\lem[wit={ceteri}]{mātṛkā}
  \rdg[wit={U2}]{mātṛkāḥ}}\dd{}
\end{prose}
\end{ekdosis}
\begin{ekdosis}
   \ekddiv{type=ed}
    \bigskip
    \centerline{\textrm{\small{[Karma, Kāma, Moon, Sun and Fire]}}}
    \bigskip
    \begin{prose}
%-----------------------------
%tad anaṃtaram  etādṛśaṃ jñānam utpadyate/  karma----kāraḥ/  candraḥ/  sūryaḥ/  agniḥ   etat paṃcakaṃ pratyakṣaṃ karttavyam \E
%tad anaṃtaraṃ  etādṛśaṃ jñānam utpadyate   karma----kāma----candra----sūryaḥ   āgniḥ   etat paṃcakaṃ pratyakṣaṃ karttavyaṃ \P
%tad anaṃtaraṃ metādṛśaṃ jñānam utpadyate/  karma----kāma----candra----sūryaḥ   āgniḥ/  etat paṃcakaṃ pratyakṣaṃ karttavyaṃ/ \B
%tad anaṃtaram  etādṛśaṃ jñānam utpadyate// karma----kāma----candra----sūryaḥ   āgniḥ// enat paṃcakaṃ pratyakṣaṃ karttavyaṃ/ \L
%tad anaṃtaraṃ  etādṛśaṃ jñānam utpādyate// karmma,  kāmaḥ/  candraḥ/  sūryaḥ/  āgniḥ/  etat paṃcakaṃ pratyakṣaṃ karttavyaṃ/  \N1
%tad anaṃtaraṃ  etādṛśaṃ jñānam utpādyate// karma----kāma----candra----sūrya    agni    etat paṃcakaṃ pratyakṣaṃ karttavyaṃ// \N2
%tad anaṃtaraṃ  etādṛśaṃ jñānam utpādyate/  karmma   kāmaḥ/  candraḥ/  sūryaḥ/  āgniḥ/  etat paṃcakaṃ pratyakṣaṃ karttavyaṃ/ \D %%%p.17 recto 
%tad anaṃtaraṃ  etādṛśaṃ jñānam utpadyate   karma----kāma----caṃdra----sūrya    agnī    etat paṃcakaṃ pratyakṣaṃ karttavyaṃ  \U1
%tad anaṃtaram  etādṛśaṃ jñānam utpadyate// karmaḥ// kāmaḥ// candraḥ// sūryaḥ// agniḥ// etat paṃcakaṃ pratyakṣaṃ karttavyaṃ// \U2
%-----------------------------
%Following knowledge about such things is generated: Karma, desire, moon, sun, and fire. The direct perceotion of this pentat shall be done. 
%----------------------------
\note[type=source, labelb=391, lem={pañcakaṃ pratyakṣaṃ}]{SSP 1.60: karmaḥ kāmaś candraḥ sūryo 'gnir iti pratyakṣakaraṇapañcakam}
ta\skp{d-a}\app{\lem[wit={E,L,U2},alt={anantaram}]{\skm{d-a}nantara\skp{m-e}}
  \rdg[wit={ceteri}]{anaṃtaraṃ}
}\skm{m-e}tādṛśaṃ jñāna\skp{m-u}\app{\lem[wit={ceteri},alt={utpadyate}]{\skm{m-u}tpadyate}
  \rdg[wit={D,N1,N2}]{utpādyate}}/
\app{\lem[wit={ceteri}]{karma}
  \rdg[wit={U2}]{karmaḥ}}\dd{}
\app{\lem[wit={ceteri}]{kāmaḥ}
  \rdg[wit={B,L,P,N2,U1}]{kāma}}\dd{}
\app{\lem[wit={E,N1,U2}]{candraḥ}
  \rdg[wit={ceteri}]{candra}}\dd{}
\app{\lem[wit={ceteri}]{sūryaḥ}
  \rdg[wit={N2,U1}]{sūrya}}\dd{}
\app{\lem[wit={E,U2}]{agniḥ}
  \rdg[wit={N2}]{agni}
  \rdg[wit={U1}]{agnī}
  \rdg[wit={ceteri}]{āgniḥ}}\dd{}
etat-pañcakaṃ pratyakṣaṃ karttavyaṃ/
%----------------------------
%tatra karmaṇaḥ paṃcaguṇāḥ                                                               \E
%tatra karmaṇā  paṃcaguṇāḥ   śubhaṃ             yaśaḥ   apakīrttiḥ  iṣṭaphalasādhānaṃ    \P
%tatra karmaṇā  paṃcaguṇāḥ/  śubhaṃ   aśubhaṃ   yaśaḥ   apakīrtiḥ/  iṣṭaphalasādhānaṃ/   \B
%tatra karmaṇāṃ paṃcaguṇāḥ   śubhaṃ   aśubhaṃ   yaśaḥ   apakīrtiḥ   iṣṭaphalasādhānaṃ    \L
%tatra karmaṇaḥ paṃcaguṇāḥ// śubhaṃ/  aśubhaṃ/  yaśaḥ/  apakīrttiḥ/ iṣṭaphalasādhanaṃ//  \N1
%tatra karmaṇa  paṃcaguṇāḥ// śubhaṃ/  aśubhaṃ/  yasa/   apakīrtti/  iṣṭaphalasādhanaṃ//  \N2
%tatra karmaṇaḥ paṃcaguṇāḥ// śubha/   aśubhaṃ   yaśaḥ/  apakīrttiḥ/ iṣṭaphalasādhanaṃ/   \D %%%p.17 recto 
%tatra karnaṇaḥ paṃcaguṇāḥ   śubha    aśubha    yaśaḥ   āvakīrtiḥ   iṣṭaphalasādhanaṃ    \U1
%tatra karmaṇaḥ paṃcaguṇāḥ// śubhaṃ// aśubhaṃ// yaśaḥ// apakīrtiḥ// iṣṭaphalasādhānaṃ//  \U2
%-----------------------------
%Among those, the five qualities of Karma are: salvation, calamity, honour, disgrace [and] bringing about the desired result.
%----------------------------
\note[type=source, labelb=392, lem={karmaṇaḥ pañcaguṇāḥ}]{SSP 1.61: śubham aśubhaṃ yaśo 'pakīrtir adṛṣṭaphalasādhanam iti pañcaguṇaṃ karma |}
tatra
\app{\lem[wit={ceteri}]{karmaṇaḥ}
  \rdg[wit={B,P}]{karmaṇā}
  \rdg[wit={N2}]{karmaṇa°}
  \rdg[wit={L}]{karmaṇāṃ}}
pañcaguṇāḥ/
\app{\lem[wit={ceteri}]{śubhaṃ}
  \rdg[wit={D,U1}]{śubha}
  \rdg[wit={E}]{\om}}\dd{}
\app{\lem[wit={ceteri}]{aśubhaṃ}
  \rdg[wit={U1}]{°aśubha°}
  \rdg[wit={E,P}]{\om}}\dd{}
\app{\lem[wit={ceteri}]{yaśaḥ}
  \rdg[wit={N2}]{yasa}
  \rdg[wit={E}]{\om}}\dd{}
\app{\lem[wit={ceteri}]{apakīrtiḥ}
  \rdg[wit={N2}]{apakīrtti}
  \rdg[wit={U1}]{āvakīrtiḥ}
  \rdg[wit={E}]{\om}}\dd{}
\app{\lem[wit={ceteri}]{iṣṭaphalasādhānaṃ}
  \rdg[wit={E}]{\om}}\dd{} 
%----------------------------
%kāmasya  guṇāḥ   ratiḥ   prītiḥ/  krīḍā   kāmanā   anustutā// \E
%kāmasya  guṇāḥ   ratiḥ   prītiḥ   krīḍā   kāmanāḥ  anurattutā \P
%kāmasya  guṇāḥ   ratiḥ   prītiḥ   krīḍā   kāminā   anustuttā// \B
%kāmasya  guṇāḥ   ratiḥ   prītiḥ   krīḍā   kāminy   anuraktatā...\L
%kāmasya  guṇāḥ/  ratiḥ   prīti,   krīḍā/  kāmanā/  anuratā// \N1
%kāmasya  guṇāḥ// rati    prīti    krīḍā/  kāmanā/  anurajā// \N2
%kāmasya  guṇāḥ/  ratiḥ   prīti    krīḍā/  kāmanā/  anuratā/ \D %%%p.17 recto 
%kāmasya  guṇāḥ   rati    prīti    kriḍā   kāmanā   ānuratā \U1
%kāmaḥsya guṇāḥ// ratiḥ// prītiḥ// krīḍā// kāmanā// anuraktā// \U2
%-----------------------------
%The qualities of desire are: lust, satisfaction, play, horny, falling in love. 
%----------------------------
\note[type=source, labelb=393, lem={kāmasya guṇāḥ}]{SSP 1.62: ratiḥ prītiḥ krīḍā kāmanāturateti pañcaguṇaḥ kāmaḥ |}
\app{\lem[wit={ceteri}]{kāmasya}
  \rdg[wit={U2}]{kāmaḥsya}}
guṇāḥ/
\app{\lem[wit={ceteri}]{ratiḥ}
  \rdg[wit={N2,U1}]{rati°}}\dd{}
\app{\lem[wit={ceteri}]{prītiḥ}
  \rdg[wit={D,N1,N2,U1}]{°prīti°}}\dd{}
krīḍā\dd{}
\app{\lem[wit={ceteri}]{kāmanā}
  \rdg[wit={P}]{kāmanāḥ}
  \rdg[wit={B}]{kāminā}
  \rdg[wit={L}]{kāminy}}\dd{}
\app{\lem[wit={D,N1}]{anuratā} %%%%CONSIDER \om to source āturatā
  \rdg[wit={U1}]{ānuratā}
  \rdg[wit={N2}]{anurajā}
  \rdg[wit={L}]{anuraktatā}
  \rdg[wit={P}]{anurattutā}
  \rdg[wit={B,E}]{anustutā}}\dd{}
%-----------------------------
%idānīṃ caṃdrasya ṣoḍaśakalāḥ kathyante/ \E [p.74]
%idānīṃ caṃdrasya ṣoḍaśakalāḥ kathyaṃte \P %%%7674.jpg 
%idānīṃ caṃdrasya ṣoḍaśakalāḥ kathyate/ \B
%idānīṃ caṃdrasya ṣoḍaśa      kathyate... \L
%idānīṃ caṃdrasya śodaśakalāḥ kathyaṃte/ \N1
%idānīṃ caṃdrasya ṣodaśakalāḥ kathyaṃte/ \N2
%idānīṃ caṃdrasya śodaśakalāḥ kathyaṃte/ \D
%idānīṃ caṃdrasya śodaśakalāḥ kathyaṃte \U1
%idānīṃ caṃdrasya saptadaśakalā vartaṃte// tasyānāmāni// ṣoḍaśakalā kathyaṃte// \U2
%-----------------------------
%Now the sixteen digits of the moon are taught. 
%----------------------------
idānīṃ candrasya
\app{\lem[wit={ceteri}]{ṣodaśakalāḥ}
  \rdg[wit={L}]{ṣoḍaśa}
  \rdg[wit={U2}]{saptadaśakalā}}
\app{\lem[wit={ceteri}]{kathyante}
  \rdg[wit={B,L}]{kathyate}
  \rdg[wit={U2}]{vartaṃte || tasyānāmāni || ṣoḍaśakalā kathyaṃte ||}}/
%----------------------------
%dallolā    kallolinī   uścalinī    unmādinī   taraṃgiṇī   poṣayaṃtī  laṃpaṭā   laharī   lolā    lelihānā    prasarantī   pravṛttiḥ   plavantī   pravāhā       saumyā    prasannā//  \E
%hallolā    kallolinī   uścalinī    unmādinī   taraṃgiṇī   poṣayaṃtī  laṃpaṭā   laharī   lolā    lelihānā    prasaraṃtī   pravṛttiḥ   sravaṃtī   pravāhā       saumyā    prasannā    \P
%dullālā    kallolinī   ucaṃlini    unmādinī   taraṃgiṇī   poṣāyaṃtī  lapaṃṭāḥ  laharī   lolā    lelihā/     prasarantī/  pravṛttī    sravaṃtī   mavāhā        somyā     prasannā//  \B
%hullātvā   kallolini   uchaṃlini   unmādanī   taraṃgiṇī   poṣāyaṃtī  lapaṭāḥ   lahari   lolā    lelihā      prasaraṃtī// prakṛtī     sravaṃtī   mavāhā        somyā     prasannā... \L
%hallolā/   kallolinī/              unmādinī/  taraṃgiṇī/  poṣayanti, lapaḍā    laharī/  lolā/   lelihānā/   prasaraṃtī/  pravṛttiḥ/  sravaṃtī/  pravāhā/      saumyā,   prasannā,   \N1
%hallolā/   kalloli/                unmādinī/  taraṃgiṇī/  poṣayaṃti/ lapaḍā    laharī/  lolā/   lelihānā/   prasaraṃtī/  pravṛttiḥ   sravaṃtī   pravāhā//     saumyā    prasannā    \N2
%hallolā/   kallolinī/              unmādinī/  taraṃgiṇī/  poṣayanti  lapaḍā    laharī/  lolā/   lelihānā/   prasaraṃtī/  pravṛttiḥ   sravaṃtī/  pravāhā/      saumyā    prasannā    \D
%hallolā    kallolini   uchalanī    unmādani   taraṃgiṇī   poṣayanī   laṃpaṭā   laharī   lolā    lelihanā    prasaraṃti   pravṛtiḥ    sravaṃtī   pravaṃtī śvāḥ saumya    prasannā    \U1
%hallolāḥ// kallolinī// ucchṛlinī// unmādinī// taraṃgiṇī// poṣayati// laṃpaṭā// laharī// lolāḥ// lelihānāḥ// prasaraṃti// pravṛttiḥ// sravaṃti// pravāhāḥ//    saumyāḥ// prasannāḥ// \U2
%----------------------------
%Ullola (she who is violently moving), Kallolinī (she who is surging), Uccalantī (she who is springing), Unmādinī (she who is intoxicating), Taraṅginī (she who is waving), Poṣayanti (she who is nourishing), Laṃpaṭā (she who is lustful), Laharī (she who is billow), Lolā (she who oscilating), Lelihānā (she who is darting out), Prasarantī (she who is spreading), Pravṛttiḥ (she who is appearing), Sravantī (she who flows), Pravāhā (she who is pulling), Saumyā (she who is dedicated to Soma), Prasannā (she who is pleasing).   
%----------------------------
\note[type=source, labelb=394, lem={ullolā}]{SSP 1.63: ullolā kallolinī uccalantī unmādinī taraṃgiṇī śoṣiṇī alampaṭā pravṛttiḥ laharī lolā lelihānā prasarantī pravāhā saumyā prasannā plavantī | evaṃ candrasya ṣoḍaśa kalāḥ | saptadaśī kalā nivṛttiḥ | sā 'mṛtakalā |}
\app{\lem[type=emendation, resp=egoscr]{ullolā}
  \rdg[wit={D,P,N1,N2,U1}]{\korr hallolā}
  \rdg[wit={U2}]{hallolāḥ}
  \rdg[wit={L}]{hullātvā}
  \rdg[wit={B}]{dullālā}
  \rdg[wit={E}]{dallolā}}\dd{}
\app{\lem[wit={ceteri}]{kallolinī}
  \rdg[wit={U1}]{kallolini}
  \rdg[wit={N2}]{kalloli}}\dd{}
\app{\lem[type=emendation, resp=egoscr]{uccalantī}
  \rdg[wit={E,P}]{\korr uścalinī}
  \rdg[wit={B}]{ucaṃlini}
  \rdg[wit={L}]{uchaṃlini}
  \rdg[wit={U1}]{uchalanī}
  \rdg[wit={U2}]{ucchṛlinī}
  \rdg[wit={D,N1,N2}]{\om}}\dd{}
\app{\lem[wit={ceteri}]{unmādinī}
  \rdg[wit={U1}]{unmādani}}\dd{}
\app{\lem[wit={E,P}]{poṣayaṃtī}
  \rdg[wit={D,N1,N2}]{poṣayanti}
  \rdg[wit={B,L}]{poṣāyaṃtī}
  \rdg[wit={U1}]{poṣayanī}
  \rdg[wit={U2}]{poṣayati}}\dd{}
\app{\lem[wit={E,P,U1,U2}]{laṃpaṭā}
  \rdg[wit={B}]{lapaṃṭāḥ}
  \rdg[wit={L}]{lapaṭāḥ}
  \rdg[wit={D,N1,N2}]{lapaḍā}}\dd{}
laharī\dd{}
\app{\lem[wit={ceteri}]{lolā}
  \rdg[wit={U2}]{lolāḥ}}\dd{}
\app{\lem[wit={ceteri}]{lelihānā}
  \rdg[wit={U2}]{lelihānāḥ}
  \rdg[wit={B,L}]{lelihā}}\dd{}
\app{\lem[wit={ceteri}]{prasarantī}
  \rdg[wit={U1,U2}]{prasaraṃti}}\dd{}
\app{\lem[wit={ceteri}]{pravṛttiḥ}
  \rdg[wit={B}]{pravṛttī}
  \rdg[wit={L}]{prakṛtī}}\dd{}
\app{\lem[wit={ceteri}]{sravantī}
  \rdg[wit={U2}]{sravaṃti}
  \rdg[wit={E}]{plavantī}}\dd{}
\app{\lem[wit={ceteri}]{pravāhā}
  \rdg[wit={U2}]{pravāhāḥ}
  \rdg[wit={B,L}]{mavāhā}
  \rdg[wit={U1}]{pravaṃtī śvāḥ}}\dd{}
\app{\lem[wit={ceteri}]{saumyā}
  \rdg[wit={U2}]{saumyāḥ}
  \rdg[wit={U1}]{saumya}
  \rdg[wit={B,L}]{somyā}}\dd{}
\app{\lem[wit={ceteri}]{prasannā}
  \rdg[wit={U2}]{prasannāḥ}}\dd{}
%----------------------------
%candrasya saptadaśamī kalā varttate   tasyā  nāma nivṛtti----sametākalā        kathyate/   \E
%candrasya saptadaśī   kalā varttate   tasya  nāma nivṛtti----sametākalā        kathyate    \P
%candrasya saptadaśamī kalā varttate   tasyā  nāma nivṛtti----sametākalā        kathyate/   \B
%candrasya saptadaśī   kalā varttate// tasyā  nāma nivṛtti----sametakalā        kathyate/   \L
%caṃdrasya saptadaśī   kalā varttate   tasyā  nāma naivṛttiḥ  sā mṛtakalā        kathyate//  \N1
%caṃdrasya saptadaśī   kalā varttate// tasyā  nāma naivṛttiḥ  sā mṛtakalā        kathyate//  \N2
%caṃdrasya saptadaśī   kalā varttate   tasyā  nāma naivṛttaiḥ sā mṛtakalā        kathyate/   \D
%candrasya saptadaśī   kā   varttate   tasyā  nāma   nivṛttiḥ sā mṛta            kathyate    \U1
%caṃdrasya saptadṛśī   kalā vartate//  tasyāḥ nāmāni// vṛttiḥ sametaḥ// kalāḥ// kathyaṃte// \U2 %%%428.jpg 
%-----------------------------
%A seventeenth digit of the moon exists. Its name is Nivṛtti (inactivity), it is taught to be the the digit of the nectar of immortality. 
%-----------------------------
candrasya
\app{\lem[wit={ceteri}]{saptadaśī}
  \rdg[wit={U2}]{saptadṛśī}
  \rdg[wit={B,E}]{saptadaśamī}}
\app{\lem[wit={ceteri}]{kalā}
  \rdg[wit={U1}]{kā}}
vartate/
\app{\lem[wit={ceteri}]{tasyā}
  \rdg[wit={P}]{tasya}
  \rdg[wit={U2}]{tasyāḥ}}
\app{\lem[wit={ceteri}]{nāma}
  \rdg[wit={U2}]{nāmāni ||}}
\app{\lem[wit={U1}]{nivṛttiḥ}
  \rdg[wit={B,E,L,P}]{nivṛtti}
  \rdg[wit={N1,N2}]{naivṛttiḥ}
  \rdg[wit={D}]{naivṛttaiḥ}
  \rdg[wit={U2}]{vṛttiḥ}}
\app{\lem[wit={D,N1,N2}]{sā 'mṛtakalā}
  \rdg[wit={U1}]{sā mṛta}
  \rdg[wit={U2}]{sametaḥ || kalāḥ ||}
  \rdg[wit={B,E,L,P}]{sametakalā}}
\app{\lem[wit={ceteri}]{kathyate}
  \rdg[wit={U2}]{kathyante}}/
\end{prose}
\end{ekdosis}
\ekdpb*{}
%%%%%%%%%%%%%%%%%%%%%%%%%%%%%%%%%%%%%%%%%%
%%%%%%%%PAGEBREAK%%%%%%%PAGEBREAK%%%%%%%%%
%%%%%%%%%%%%%%%%%%%%%%%%%%%%%%%%%%%%%%%%%%
%%%%%%%%%%%%%%%%PAGEBREAK%%%%%%%%%%%%%%%%%
%%%%%%%%%%%%%%%%%%%%%%%%%%%%%%%%%%%%%%%%%%
%%%%%%%%PAGEBREAK%%%%%%%PAGEBREAK%%%%%%%%%
%%%%%%%%%%%%%%%%%%%%%%%%%%%%%%%%%%%%%%%%%%
%%%%%%%%%%%%%%%%%%%%%%%%%%%%%%%%%%%%%%%%%%
%%%%%%%%%%%%%%%%%%%%%%%%%%%%%%%%%%%%%%%%%%
%%%%%%%%%%%%%%%%%%%%%%%%%%%%%%%%%%%%%%%%%%
%%%%%%%%PAGEBREAK%%%%%%%PAGEBREAK%%%%%%%%%
%%%%%%%%%%%%%%%%%%%%%%%%%%%%%%%%%%%%%%%%%%
%%%%%%%%%%%%%%%%PAGEBREAK%%%%%%%%%%%%%%%%%
%%%%%%%%%%%%%%%%%%%%%%%%%%%%%%%%%%%%%%%%%%
%%%%%%%%PAGEBREAK%%%%%%%PAGEBREAK%%%%%%%%%
%%%%%%%%%%%%%%%%%%%%%%%%%%%%%%%%%%%%%%%%%%
%%%%%%%%%%%%%%%%%%%%%%%%%%%%%%%%%%%%%%%%%%
%%%%%%%%%%%%%%%%%%%%%%%%%%%%%%%%%%%%%%%%%%
%%%%%%%%%%%%%%%%%%%%%%%%%%%%%%%%%%%%%%%%%%
%%%%%%%%PAGEBREAK%%%%%%%PAGEBREAK%%%%%%%%%
%%%%%%%%%%%%%%%%%%%%%%%%%%%%%%%%%%%%%%%%%%
%%%%%%%%%%%%%%%%PAGEBREAK%%%%%%%%%%%%%%%%%
%%%%%%%%%%%%%%%%%%%%%%%%%%%%%%%%%%%%%%%%%%
%%%%%%%%PAGEBREAK%%%%%%%PAGEBREAK%%%%%%%%%
%%%%%%%%%%%%%%%%%%%%%%%%%%%%%%%%%%%%%%%%%%
\begin{ekdosis}
  \begin{prose}
    \noindent
%-----------------------------
%idānīṃ sūryasya kalāḥ        kathyante/        \E
%idānīṃ sūryasya dvādaśakalāḥ kathyaṃte  \P
%idānīṃ sūryasya dvādaśakalā  kathyate/   \B
%idānīṃ sūryasya dvādaśakalā  kathyate//  \L
%idānīṃ sūryasya dvādaśakalā  kathyaṃte/  \N1
%idānīṃ sūryasya dvādaśakalā  kathyate/   \N2
%idānīṃ sūryasya dvādaśakalā  kathyaṃte/  \D
%idānīṃ sūryasya dvādaśakalā  kathyaṃte   \U1
%idānīṃ sūryasya dvādaśakalāḥ kathyaṃte// \U2
%-----------------------------
%Now the twelve digits of the sun are taught. 
%-----------------------------
\note[type=source, labelb=395, lem={sūryasya dvādaśakalāḥ}]{SSP 1.64: tāpinī grāsikā ugrā ākuñcinī śoṣiṇī prabodhinī smarā ākarṣiṇī tuṣṭivardhinī urmirekhā kiraṇavatī prabhāvatīti dvādaśa kalāḥ sūryasya | trayodaśī svaprakāśatā nijakalā |}
idānīṃ sūryasya
\app{\lem[wit={P,U2}]{dvādaśakalāḥ}
  \rdg[wit={B,D,L,N1,N2,U1}]{dvādaśakalā}
  \rdg[wit={E}]{kalāḥ}}
\app{\lem[wit={ceteri}]{kathyante}
  \rdg[wit={B,L,N2}]{kathyate}}/ 
%-----------------------------
%tapanī   grāsakā   ugrā   akocanī   śoṣaṇī   prabodhinī   ghasmarā   ākarṣiṇī     tuṣṭivarddhinī    kūrmīreṣā      kiraṇavatī   prabhavati     sūryasya trayodaśī kalā vidyate/  \E %[p.75]
%tāpanī   grāsaka   ugra   ākocanī   śoṣiṇī   prabodhinī   ghasmarā   ākarṣayaṃtī  tuṣṭivarddhinī    kurmmīrekhā    kiraṇāvatī   prabhūtavatī   sūryasya trayodaśī kalā vidyate   \P
%tāpani   grāsaka   ugrā   ākocanī/  śoṣaṇī   prabodhanī   ghasmarā/  ākarṣayaṃtī/ tuṣṭivardhanī/     ūrmmirekhā    kīrṇāvatī    prabhavati/    sūryasya trayodaśī kalā vidyate/  \B
%tāpani   grāsaka   ugrā   ākocanī   śoṣaṇī   prabodhanī   ghasmarā// ākarṣayaṃtī  tuṣṭivardhanī      ūrmmirekhā    kīrṇāvitī    prabhutavati// sūryasya trayodaśi kalā vidyate// \L
%tapanī/  grāsakā/  ugrā/  ākuṃcanī/ śoṣaṇī/  prabodhinī/  ghasmarā/  ākarṣayaṃtī/ tuṣṭi, varddhanī/  ūrmmirekhā//  kiraṇāvatī/  prabhutavatī/  sūryasya trayodaśī kalā vidyate/  \N1 %%%S.13
%tapanī/  grāsakā/  ugrā/  ākuṃcanī/ śoṣaṇī/  prabodhinī/  ghasmara/  ākarṣayaṃtī/ tuṣṭivarddhanī/    ūrmmirekhā//  kiraṇāvatī/  prabhutavatī// sūryasya trayodaśī kalā vidyate// \N2
%tapanī/  grāsakā/  ugrā/  ākuṃcanī/ śoṣaṇī/  prabodhinī/  ghasmarā/  ākarṣayaṃtī/ tuṣṭivarddhanī     ūrmmirekhā/   kiraṇāvatī/  prabhutavatī/  sūryasya trayodaśī kalā vidyate/  \D
%tapani   grāsakā   ugrā   ākuṃcanī  śoṣaṇī   prabodhinī   ghasmarā   ākarṣayaṃti  tuṣṭivardhanī      ūrmirekhā     kīrṇavatī    prabhutavatī   sūryasya trayodaśī kalā vidyate   \U1
%tapanī// grāsakā// ugrā// akocanī// śoṣaṇī// prabodhinī// ghasmarā// ākarṣayatī// tuṣṭiḥ varddhanī// ūrmī// rekhā  kiraṇavatī// prabhūtavatī// sūryasya trayodaśī kalā vidyate// \U2
%-----------------------------
%Tāpinī (she who is heating), Grāsikā (she who is seizing), Ugrā (she who is fierce), Ākuñcinī (she who is contracting), Śoṣiṇī (she who is desiccating), Prabodhinī (she who is awakening), Ghasmarā (she who is voracious), Ākarṣiṇī (she who is attracting), Tuṣṭivarddhinī (she who is satisfying), Ūrmirekhā (she who is a row of waves), Kiraṇavatī (she who is readiating), Prabhāvatī (she who is shining). The thirteenth digit of the sun is to be known.   
%-----------------------------
\app{\lem[type=emendation, resp=egoscr]{tāpinī}
  \rdg[wit={P}]{\korr tāpanī}
  \rdg[wit={B,L}]{tāpani}
  \rdg[wit={D,E,N1,N2,U2}]{tapanī}
  \rdg[wit={U1}]{tapani}}\dd{}
\app{\lem[type=emendation, resp=egoscr]{grāsikā}
  \rdg[wit={ceteri}]{\korr grāsakā}
  \rdg[wit={B,L,P}]{grāsaka}}
ugrā\dd{}
\app{\lem[type=emendation, resp=egoscr]{ākuñcinī}
  \rdg[wit={D,N1,N2,U1}]{\korr ākuṃcanī}
  \rdg[wit={B,L,P}]{ākocanī}
  \rdg[wit={U2}]{akocanī}}\dd{}
\app{\lem[wit={P}]{śoṣiṇī}
  \rdg[wit={ceteri}]{śoṣaṇī}}\dd{}
prabodhinī\dd{}
ghasmarā\dd{}
\app{\lem[wit={E}]{ākarṣiṇī}
  \rdg[wit={U2}]{ākarṣayatī}
  \rdg[wit={U1}]{ākarṣayaṃti}
  \rdg[wit={ceteri}]{ākarṣayaṃtī}}\dd{}
\app{\lem[wit={E,P}]{tuṣṭivardhinī}
  \rdg[wit={B,L}]{tuṣṭivardhanī}
  \rdg[wit={N1}]{tuṣṭi, varddhanī}
  \rdg[wit={D,N2}]{tuṣṭi varddhanī}
  \rdg[wit={U2}]{tuṣṭiḥ varddhanī}}\dd{}
\app{\lem[wit={ceteri}]{ūrmirekhā}
  \rdg[wit={E}]{kūrmīreṣā}
  \rdg[wit={P}]{kurmmīrekhā}
  \rdg[wit={U2}]{ūrmī || rekhā}}\dd{}
\app{\lem[wit={E,U2}]{kiraṇavatī}
  \rdg[wit={D,P,N1,N2}]{kiraṇāvatī}
  \rdg[wit={B,L}]{kīrṇāvatī}
  \rdg[wit={U1}]{kīrṇavatī}}\dd{}
\app{\lem[type=emendation, resp=egoscr]{prabhāvatī}
  \rdg[wit={B,E}]{\korr prabhavati}
  \rdg[wit={P,U2}]{prabhūtavatī}
  \rdg[wit={L}]{prabhutavati}
  \rdg[wit={ceteri}]{prabhutavatī}}\dd{}
sūryasya trayodaśī kalā vidyate/
%-----------------------------
%tasya nāma    nijakalā---svaprakāśā ca// \E
%tasya nāma    nijakalā---svaprakāśā ca  \P
%tasya nāmaḥ   nijakalā---svaprakāśā ca/ \B
%tasya nāma    nijakalā---svaprakāśā ca... \L
%tasya saṃjñā  nijakalāṃ  svaprakāśā ca// \N1
%tasya saṃjñā  nijakalāṃ  svaprakāśā ca// \N2
%tasyāḥ saṃjñā nijakalāṃ  svaprakāśā ca/ \D
%tasyāḥ saṃjñā nijakalā---svaprakāśā ca \U1
%tasyā nāmāni  nijakalā// svaprakāśā ca// \U2
%-----------------------------
%Her technical designation is the inherent digit, und [the] self-luminous. 
%-----------------------------
\app{\lem[wit={D,U1}]{tasyāḥ}
  \rdg[wit={U2}]{tasyā}
  \rdg[wit={ceteri}]{tasya}}
\app{\lem[wit={D,N1,N2,U1}]{saṃjñā}
  \rdg[wit={E,L,P}]{nāma}
  \rdg[wit={B}]{namaḥ}
  \rdg[wit={U2}]{nāmāni}}
\app{\lem[wit={ceteri}]{nijakalā}
  \rdg[wit={D,N1,N2}]{nijakalāṃ}}
svaprakāśā ca/
%-----------------------------
%idānīm agnisaṃbaṃdhinyo daśakalāḥ   kathyante/ \E
%idānīm agnisaṃbaṃdhinyo daśakalāḥ   kathyaṃte/ \P DSCN7675.jpg
%idānīm agnīsaṃbaṃdhini  daśakalā    kathyaṃte/ \B
%idānīm agnīsaṃbaṃdhini  daśakalā    kathyaṃte// \L
%idānīm agnīsaṃbaṃdhinī  daśakalāḥ   kathyaṃte/ \N1
%idānīm agnīsaṃbaṃdhinī  daśakalā    kathyaṃte/ \N2
%idānīm agnīsaṃbaṃdhinī  daśakalāḥ   kathyaṃte/ \D
%idānīm agnīsaṃbaṃdhinīṃ dvādaśakalā kathyaṃte \U1
%idānīṃ agnisaṃbaṃdhinī  daśakalāḥ// kathyaṃte// \U2
%-----------------------------
%Now the ten digits, which are related to the fire are taught. 
%-----------------------------
\note[type=source, labelb=396, lem={agnisaṃbandhinyo}]{SSP 1.65: dīpikā rājikā jvalanī visphuliṃginī pracaṇḍā pācikā raudrī dāhikā rāgiṇī śikhāvatī ity agner daśa kalāḥ | ekādaśī kalā jyotiḥ |}
\app{\lem[wit={ceteri},alt={idānīm}]{idānī\skp{m-a}}
  \rdg[wit={U2}]{idānīṃ}}\app{\lem[wit={E,P},alt={agnisaṃbaṃdhinyo}]{\skm{m-a}gnisaṃbaṃdhinyo}
  \rdg[wit={ceteri}]{agnīsaṃbaṃdhinī}
  \rdg[wit={U1}]{agnīsaṃbaṃdhinīṃ}} kathyante/
%-----------------------------
%dīpikā            jvālā    visphuliṃginī   pracaṃḍā   pācikā   raudrī   dāhikā   rāvaṇī/  śikhāvatī/  agner ekādaśī nijakalā   jyotiḥ   saṃjñā varttate// \E
%dīpikā            jvālā    visphuliṃginī   pracaṃḍā   pāvakā   raudrī   dāhakā   rāvaṇī   śikhāvatī   agner ekādaśī nijakalā   jyotiḥ   saṃjñā varttate   \P
%dīpikā            jvālā    visphuliṃginī   pracaṃḍā/  pāvakā   raudrī   dāhaka   rāvaṇī   śikhāvatī/  agne  ekādaśi nijakalā   jyotiḥ/  saṃjñā vartate//  \B
%dīpikā            jvālā    visphuliṃginī   pracaṃḍā// pāvakā   raudrī   dāhaka   rāvaṇī   śikhāvatī   agne  ekādaśi nijakalā   jyotiḥ// saṃjñā vartate//  \L
%dīpikā/  jārakā/  jvālā/   visphuliṃginī/  pracaṃḍā/  pācakā/  raudrī   dāhakā/  rāvaṇi/  śikhāvatī/  agner ekādaśi nijakalā   jyoti    saṃjñakā//        \N1
%dīpikā/  jārakā/  jvālā/   visphuliṃginī/  pracaṇḍā/  pācakā/  raudrī/  dāhakā/  rāvaṇi/  śikhāvatī(/ agner ekādaśi nijakalā   jyoti    saṃjñakā//        \N2
%dīpikā/  jārakā/  jvālā/   visphuliṃginī/  pracaṃḍā/  pācakā   raudrī   dāhakā/  rāvaṇi/  śikhāvatī/  agner ekādaśī nijakalā   jyoti    saṃjñakā/         \D
%dīpikār  jakā     jvālāviḥ visphuliṃginī   pracaṃḍā// pāvakā   raudrī   dāhaka   rāvaṇī   śikhāvatī   agne  ekādaśi nijakalā   jyotiḥ// saṃjñā vartate//  \U1 300.jpg
%dīpikā//          jvālā//  visphuliṃginī// pracaṃḍā// pāvakā// raudrī// dāhakā// rāvaṇī// śikhāvatī// agner ekādaśī nijakalā// jyotiḥ// saṃjñā varttate// \U2
%-----------------------------
%Dīpikā (she who is kindling), Rājikā (she shp is resplendent), Visphuliṅginī (she who is sparkling), Pracaṇḍā (she who is furious), Pācikā (she who is cooking), Raudrī (she who is violent), Dāhakā (she who is inflaming), Rāgiṇī (she who is colouring), Śikhāvatī (she who is flaming). Light is the technical designation for the eleventh inherent digit of the fire.   
%----------------------------
\app{\lem[wit={ceteri}]{dīpikā}
  \rdg[wit={U1}]{dīpikar}}\dd{}
\app{\lem[type=emendation, resp=egoscr]{rājikā}
  \rdg[wit={D,N1,N2}]{\korr jārakā}
  \rdg[wit={U1}]{jakā}
  \rdg[wit={ceteri}]{\om}}\dd{}
\app{\lem[type=emendation, resp=egoscr]{jvalanī}
  \rdg[wit={U1}]{\korr jvālāviḥ}
  \rdg[wit={ceteri}]{jvālā}}\dd{}
visphuliṅginī\dd{}
pracaṇḍā\dd{}
\app{\lem[wit={E}]{pācikā}
  \rdg[wit={D,N1,N2}]{pācakā}
  \rdg[wit={ceteri}]{pāvakā}}\dd{}
raudrī\dd{}
\app{\lem[wit={E}]{dāhikā}
  \rdg[wit={D,P,N1,N2,U2}]{dāhakā}
  \rdg[wit={B,L,U1}]{dāhaka}}\dd{}
\app{\lem[type=emendation, resp=egoscr]{rāgiṇī}
  \rdg[wit={B,E,L,P,U2}]{\korr rāvaṇī}
  \rdg[wit={ceteri}]{rāvaṇi}}\dd{}
śikhāvatī\dd{}
\app{\lem[wit={ceteri},alt={agner}]{agne\skp{r-e}}
  \rdg[wit={B,L,U1}]{agne}}\app{\lem[wit={D,E,P,U2},alt={ekādaśī}]{\skm{r-e}kādaśī}
  \rdg[wit={ceteri}]{ekādaśi}}  
nijakalā jyotiḥ
\app{\lem[wit={ceteri}]{saṃjñā}
  \rdg[wit={D,N1,N2}]{saṃjñakā}}
\app{\lem[wit={ceteri}]{vartate}
  \rdg[wit={D,N1,N2}]{\om}}\dd{}
\end{prose}
\end{ekdosis}
\begin{ekdosis}
   \ekddiv{type=ed}
    \bigskip
    \centerline{\textrm{\small{[The Magnificence of Yoga]}}}
    \bigskip
    \begin{prose}
%----------------------------
%idānīṃ yogasya māhātmyaṃ kathyate/ \E
%idānīṃ yogasya māhātmyaṃ kathyate \P
%idānī  yogasya māhātmaṃ  kathyate/ \B
%idānīṃ yogasya māhātmaṃ  kathyate// \L
%idānīṃ yogasya māhātmyaṃ kathyate// \N1
%idānīṃ yogasya māhātmya  kathyate// \N2
%idānīṃ yogasya māhātmyaṃ kathyate/ \D
%idānīṃ yasya   māhātmyaṃ kathyate \U1
%idānīṃ yogasya māhātmyaṃ kathyaṃte// \U2
%-----------------------------
%Now the magnificence of yoga is taught. 
%-----------------------------
\note[type=source, labelb=397, lem={yogasya māhātmyaṃ}]{Ysv\textsuperscript{PT}: idānīṃ yogamāhātmyaṃ kathyate yad bhavet tataḥ |} %%%also SSP 5.55-5.59!! 
\app{\lem[wit={ceteri}]{idānīṃ}
  \rdg[wit={B}]{idānī}}
\app{\lem[wit={ceteri}]{yogasya}
  \rdg[wit={U1}]{yasya}}
\app{\lem[wit={ceteri}]{māhātmyaṃ}
  \rdg[wit={B,L}]{māhātmaṃ}
  \rdg[wit={N2}]{māhātmya}}
\app{\lem[wit={ceteri}]{kathyate}
  \rdg[wit={U2}]{kathyaṃte}}/ 
%-----------------------------
%guror anugrahāt   śāstrasya paṭhanāt   ācārakaraṇāt  vedāṃtarahasya śravaṇāt   dhyānakaraṇāt                 upavāsakaraṇāt   caturaśītyāsanesādhanāt    vairāgyasyotpatteḥ nairāśye karaṇāt ...      \E [P.76]
%guror anugrahāt   śāstrasya paṭhanāt   ācārakaraṇāt  vedāṃtarahasya śravaṇāt                                                  caturaśītyāsanasādhanāt    vairāgyasyotpattaḥ nairāśya karaṇāt          \P
%guru  anugrahāt/  śāstrasya paṭhanāt/  ācārakaraṇāt/ vedāṃtarahasya śravaṇāt/  dhyānakaraṇāt/                upavāsakaraṇāt/  caturāśītyāsanasādhanāt/   vairāgyasyotpatte/ nairāśa karaṇāt/ ... \B
%guru  agrahāt     śāstrasya paṭhanāt   ācārakaraṇāt  vedāṃtarahasya śravaṇāt   dhyānakaraṇāt                 upavāsakaraṇāt   caturāśītyāsanasādhanāt    vairāgyasyotpatteḥ nairāśya karaṇāt ... \L %%%0038.jpg
%guror anugrahāt/  śāstrasya paṭhanāt/  ācārakaraṇāt/ vedāntarahasya śravaṇāt/  dhyānakaraṇāt/ layasādhanāt/  upavāsakaraṇāt/  caturaśīti āsanasādhanāt/  vairāgyotpatteḥ/     vairāgyakaraṇāt//  \N1
%guror anugrahāt/  śāstrasya paṭhanāt/  ācārakaraṇāt/ vedāntarahasya śravaṇāt/  dhyānakaraṇāt/ layasādhanāt/  upavāsakaraṇāt/  caturaśīti āsanasādhanāt/  vairāgyasyotpatteḥ   vairāgyakaraṇāt/  \N2
%guror anugrahāt/  śāstrasya paṭhanāt/  ācārakaraṇāt/ vedāṃtarahasya śravaṇāt/  dhyānakaraṇāt  layasādhanāt/  upavāsakaraṇāt/  caturaśīti āsanasādhanāt/  vairāgyotpatteḥ/     vairāgyakaraṇāt/  \D
%guror anugrahāt   śāstrasya paṭhanāt   ācārakaraṇāt  vedāṃtarahasya śravaṇāt   dhyānakaraṇāt  layasādhanāt   upavāsakaraṇāt   caturaśīti āsanasādhanāt   vairāgyotpatte       vairāgyakaraṇāt  \U1
%guror anugrahāt// śāstrasya paṭhanāt// ācārakathanāt vedāṃtarahasya śravaṇāt// dhyānakaraṇāt//               upavāsakaraṇāt// caturāśītyāsanasādhanāt//  vairāgyasyotpatteḥ// vairāgyakaraṇāt//      \U2
%-----------------------------
%Because of grace of the teacher, because of studying the teaching, because of execution of good conduct, because of hearing the secret of Vedānta, because of execution of meditation, because of practicing dissolution, because of the execution of fasting, because of practising 84 āsanas, because of the generation of equanimity, because of executing equanimity, 
%-----------------------------
\note[type=source, labelb=398, lem={guror anugrahāt}]{Ysv\textsuperscript{PT}: guror anugrahāc chāstrapāṭhād ācāratas tathā | vedāntārtharahasyārthasarvajñānādupāsanāt | āsanād dhāraṇād dhyānāl layaṣaṭkarmasādhanāt | āsanāc caturaśītivairāgyatyāgasambhavāt |}
\app{\lem[wit={ceteri},alt={guror}]{guro\skp{r-a}}
  \rdg[wit={B,L}]{guru}
}\app{\lem[wit={ceteri},alt={anugrahāt}]{\skm{r-a}nugrahāt}
  \rdg[wit={L}]{agrahāt}}\dd{}
śāstrasya paṭhanāt\dd{}
\app{\lem[wit={ceteri}]{ācārakaraṇāt}
  \rdg[wit={U2}]{ācārakathanāt}}\dd{}
vedāntarahasya śravaṇāt\dd{}
\app{\lem[wit={ceteri}]{dhyānakaraṇāt}
  \rdg[wit={P}]{\om}}\dd{}
\app{\lem[wit={D,N1,N2,U1}]{layasādhanāt}
  \rdg[wit={ceteri}]{\om}}\dd{}
\app{\lem[wit={ceteri}]{upavāsakaraṇāt}
  \rdg[wit={P}]{\om}}\dd{}
\app{\lem[wit={B,L,P,U2}]{caturaśītyāsanasādhanāt}
  \rdg[wit={E}]{caturaśītyāsane sādhanāt}
  \rdg[wit={D,N1,N2,U1}]{caturaśīti āsanasādhanāt}}\dd{}
\app{\lem[wit={E,L,N2,U2}]{vairāgyasyotpatteḥ}
  \rdg[wit={B}]{vairāgyasyotpatte}
  \rdg[wit={P}]{vairāgyasyotpattaḥ}
  \rdg[wit={N1,D}]{vairāgyotpatteḥ}
  \rdg[wit={U1}]{vairāgyotpatte}}\dd{}
\app{\lem[wit={ceteri},alt={vairāgya°}]{vairāgya}
  \rdg[wit={P,L}]{nairāśya}
  \rdg[wit={B}]{nairāśa°}
  \rdg[wit={E}]{nairāśye}}karaṇāt\dd{}
%----------------------------
%haṭhayogasya karaṇāt   iḍāpiṃgalayoḥ   pavanadhāraṇāt          mahāmudrādidaśamudrāsādhanāt                 maunakaraṇāt   vanavāsāt  bahutarakleśakaraṇāt     bahukāla------yaṃtramaṃtrādisādhanāt  tapaḥ karaṇāt... \E
%haṭhayogasya karaṇāt   iḍāpiṃgalayoḥ   pavanadhāraṇāt          mahāmudrādidaśamudrāsādhanāt                 maunakaraṇāt   vanavāsāt  bahutarakleśakaraṇāt     bahutarakālaṃ yaṃtramaṃtrādisādhanāt  tapaḥ karaṇāt... \P
%haṭayogasya karaṇāt/                                           mahāmudrādidaśamudrāsādhanāt/                maunakaraṇāt/  vanavāsāt/ bahutarakleśakaraṇāt/    bahukāla------yaṃtramaṃtrādisādhanāt/ tapakaraṇāt/... \B
%haṭayogasya karaṇāt//  iḍāpiṃgalayoḥ   pāvanāpāvadhyānakaraṇāt mahāmudrādidaśamudrāsādhanāt                 maunakaraṇāt   vanavāsāt  bahutarakleśakaraṇāt     bahutarakāla--maṃtrayaṃtrādisādhanāt  tapakaraṇāt... \L
%haṭhayogakaraṇāt/      iḍāpiṃgalayoḥ   pāvanādhāraṇāt/         mahāmudrādidaśamudrāsādhanāt//               maunakaraṇāt/  vane vāsāt,bahutararakleśakaraṇāt// bahutarakālaṃ yaṃtrayaṃtrādisādhanāt  tapakaraṇāt/ \N1
%haṭhayogakaraṇāt/      iḍāpiṃgalayāḥ   pavanādhāraṇāt/         mahāmudrādidaśamudrāsādhanāt/                maunakaraṇād---vane vāsātabahutarakleśakaraṇāt/    bahutarakālaṃ yaṃtrayaṃtrādisādhanāt  tapakaraṇāt/ \N2
%haṭhayogakaraṇāt/      iḍāpiṃgalayoḥ   pāvanādhāraṇāt/         mahāmudrādidaśamudrādi daśamūdrasādhanāt//   maunakaraṇāt   vane vāsāt bahutararakleśakaraṇāt/  bahutarakālaṃ yaṃtrayaṃtrādisādhanāt  tapakaraṇāt/ \D %%%p.17 verso
%haṭayogasya  karaṇāt   iḍāpiṃgalayāḥ   pavanadharaṇāt          mahāmudrāsādhanāt                            maunakaraṇāt   vane vāsāt bahutarakleśakaraṇāt     bahutarakāla--maṃtrayaṃtrādisādhanāt  tapakaraṇāt \U1
%haṭhayogasya karaṇāt// iḍāpiṃgalayoḥ// pavanādhānākaraṇāt//    mahāmudrādidaśamudrāsādhanāt//               maunakaraṇāt// vanavāsāt//bahutarakleśakaraṇāt//   bahutarakāla--yaṃtramaṃtrādisādhanāt//tapaḥ karaṇāt// \U2
%-----------------------------
%because of doing haṭhayoga, because of holding the breath of the Iḍā- and Piṅgalā-channels, because of practicing the ten seals [like] the great-seal etc., because of [the observation of] silence, because of dwelling in the forest, because of the execution of many defilements?!, because of practicing Mantra and Yantra for a long time, because of austerities,  
%-----------------------------
\note[type=source, labelb=399, lem={haṭhayogasya karaṇāt}]{Ysv\textsuperscript{PT}: haṭhayogād varauṣadhyāt mudrāsādhanamānataḥ | vanavāsād bahukleśāt tathā mantrādisādhanāt |}
\app{\lem[wit={ceteri}, alt={haṭha°}]{haṭha}
  \rdg[wit={B,L,U1}]{haṭa°}}\app{\lem[wit={ceteri}]{yogasya}
  \rdg[wit={N1,N2,D}]{yoga°}}karaṇāt\dd{}   
\app{\lem[wit={ceteri}]{iḍāpiṅgalayoḥ}
  \rdg[wit={N2,U1}]{iḍāpiṃgalayāḥ}}
\app{\lem[wit={E,P,U1}]{pavanadhāraṇāt}
  \rdg[wit={D,N1}]{pāvanādhāraṇāt}
  \rdg[wit={N2}]{pavanādhāraṇāt}
  \rdg[wit={U2}]{pavanādhānākaraṇāt}
  \rdg[wit={L}]{pāvanāpāvadhyānakaraṇāt}
  \rdg[wit={B}]{\om}}\dd{}
\app{\lem[wit={ceteri}]{mahāmudrādidaśamudrāsādhanāt}
  \rdg[wit={U1}]{mahāmudrāsādhanāt}
  \rdg[wit={D}]{mahāmudrādidaśamudrādi daśamūdrasādhanāt}}\dd{}
\app{\lem[wit={ceteri}]{maunakaraṇāt}
  \rdg[wit={N2}]{maunakaraṇād}}\dd{}
\app{\lem[wit={ceteri}]{vanavāsāt}
  \rdg[wit={D,N1,U1}]{vane vāsāt}
  \rdg[wit={N2}]{vane vāsāta°}}\dd{}
bahutarakleśakaraṇāt\dd{}
\end{prose}
\end{ekdosis}
\ekdpb*{}
%%%%%%%%%%%%%%%%%%%%%%%%%%%%%%%%%%%%%%%%%%
%%%%%%%%PAGEBREAK%%%%%%%PAGEBREAK%%%%%%%%%
%%%%%%%%%%%%%%%%%%%%%%%%%%%%%%%%%%%%%%%%%%
%%%%%%%%%%%%%%%%PAGEBREAK%%%%%%%%%%%%%%%%%
%%%%%%%%%%%%%%%%%%%%%%%%%%%%%%%%%%%%%%%%%%
%%%%%%%%PAGEBREAK%%%%%%%PAGEBREAK%%%%%%%%%
%%%%%%%%%%%%%%%%%%%%%%%%%%%%%%%%%%%%%%%%%%
%%%%%%%%%%%%%%%%%%%%%%%%%%%%%%%%%%%%%%%%%%
%%%%%%%%%%%%%%%%%%%%%%%%%%%%%%%%%%%%%%%%%%
%%%%%%%%%%%%%%%%%%%%%%%%%%%%%%%%%%%%%%%%%%
%%%%%%%%PAGEBREAK%%%%%%%PAGEBREAK%%%%%%%%%
%%%%%%%%%%%%%%%%%%%%%%%%%%%%%%%%%%%%%%%%%%
%%%%%%%%%%%%%%%%PAGEBREAK%%%%%%%%%%%%%%%%%
%%%%%%%%%%%%%%%%%%%%%%%%%%%%%%%%%%%%%%%%%%
%%%%%%%%PAGEBREAK%%%%%%%PAGEBREAK%%%%%%%%%
%%%%%%%%%%%%%%%%%%%%%%%%%%%%%%%%%%%%%%%%%%
%%%%%%%%%%%%%%%%%%%%%%%%%%%%%%%%%%%%%%%%%%
%%%%%%%%%%%%%%%%%%%%%%%%%%%%%%%%%%%%%%%%%%
%%%%%%%%%%%%%%%%%%%%%%%%%%%%%%%%%%%%%%%%%%
%%%%%%%%PAGEBREAK%%%%%%%PAGEBREAK%%%%%%%%%
%%%%%%%%%%%%%%%%%%%%%%%%%%%%%%%%%%%%%%%%%%
%%%%%%%%%%%%%%%%PAGEBREAK%%%%%%%%%%%%%%%%%
%%%%%%%%%%%%%%%%%%%%%%%%%%%%%%%%%%%%%%%%%%
%%%%%%%%PAGEBREAK%%%%%%%PAGEBREAK%%%%%%%%%
%%%%%%%%%%%%%%%%%%%%%%%%%%%%%%%%%%%%%%%%%%
\begin{ekdosis}
  \begin{prose}
\app{\lem[wit={D,P,N1,N2}]{bahutarakālaṃ}
  \rdg[wit={L,U1,U2}]{bahutarakāla°}
  \rdg[wit={B,E}]{bahukāla°}}
\app{\lem[wit={B,D,E,P,N1,N2,U2}]{yantramantrādisādhanāt}
  \rdg[wit={L,U1}]{maṃtrayaṃtrādisādhanāt}}\dd{}
\app{\lem[wit={ceteri},alt={tapa°}]{tapa}
  \rdg[wit={E,P,U2}]{tapaḥ}}karaṇāt\dd{}
%-----------------------------
%bahutarārpaṇadānāt                                     āśramācārapālanāt   saṃnyāsagrahaṇāt  ṣaḍdarśanagrahaṇāt   śiromuṃḍanāt   anyopāyakaraṇāt   yogatattvaṃ na  prāpyate// \E
%bahutarakleśakaraṇāt bahutarakaraṇāt bahutatārthadānāt āśramācārapālanāt   saṃnyāsagrahaṇāt  ṣaṭdarśanagrahaṇāt                                    yogatatvaṃ  na  prāpyate \P %%%7676.jpg
%bahutarārthādānāt/                                     āśramācārapālanāt/  sanyāsagrahaṇāt/  ṣaḍdarśanagrahaṇāt/  śiromuṃḍanāt/  anyopāyakaraṇāt/  yogatattvaṃ nna prāpyate/ \B
%bahutarārthādānāt                                      āśramācārapālanāt   sanyāsagrahaṇāt   ṣaḍdarśanagrahaṇāt   śiromuṃḍanāt   anyopāyakaraṇāt   yogatattvaṃ nna prāpyate \L
%bahutarārthadānāt/ tīrthasevokaraṇāt/                  āśramācārapālanāt/  saṃnyāsagrahaṇāt/ ṣaṭdarśanagrahaṇāt/  siromuṃḍanāt// anyopāyakaraṇāt/  yogatatvaṃ  na  prāpyate/ \N1
%bahutarārthadānāt/ tīrthasevākaraṇāt/                  āśramācārapālanāt/  saṃnyāsagrahaṇāt/ ṣaṭdarśanagrahaṇāt/  siromaṃḍanāt / anyopāyakaraṇāt// yogatatvaṃ  na  prāpyate/ \N2
%bahutarārthadānāt// tīrthasevākaraṇāt/                 āśramācārapālanāt/  saṃnyāsagrahaṇāt/ ṣaṭdarśanagrahaṇāt   siromuṃḍanāt/  anyopāyakaraṇāt/  yogatatvaṃ  na  prāpyate/ \D
%bahuttarārthadānāt niyamakaraṇāt                       āśramācyārapālanāt  sanyāsagrahaṇāt   ṣaḍdarśanagrahaṇāt   siromuṃḍanāt   anyopāyakaraṇāt   yogatatvaṃ  na  prāpyate \U1
%bahutarārthadānāt//                                    āśramācārapālanāt// sanyāsagrahaṇāt// ṣaṭdarśanagrahaṇāt// siromuṃḍanāt// anyopāyakaraṇāt// yogatatvaṃ  na  prāpyate// \U2 %%%429.jpg 
%-----------------------------
%because of giving up a lot of possession, because of frequenting places of pilgrimage, because of protection of the habit of the stages of life, because of undertaking renunciation, because of grasping the six philosophies, because of shaving the head, because of the execution of other means, the reality of yoga is not attained. 
%-----------------------------
\note[type=source, labelb=400, lem={bahutarārthadānāt}]{Ysv\textsuperscript{PT}: bahudānatapastīrthasevanād dānaśikṣaṇāt | sandhyātrayagraheṇātha ṣaḍadarśagrahaṇāt tathā | śiromuṇḍagato nyāsād yogatattvañ ca vidyate |}
\app{\lem[wit={ceteri}]{bahutarārthādānāt}
  \rdg[wit={E}]{bahutarārpaṇadānāt}
  \rdg[wit={P}]{bahutarakleśakaraṇāt bahutarakaraṇāt bahutatārthadānāt}}\dd{}
\app{\lem[wit={D,N2}]{tīrthasevākaraṇāt}
  \rdg[wit={N1}]{tīrthasevokaraṇāt}
  \rdg[wit={U1}]{niyamakaraṇāt}
  \rdg[wit={ceteri}]{\om}}\dd{}
\app{\lem[wit={ceteri}]{āśramācārapālanāt}
  \rdg[wit={U1}]{āśramācyārapālanāt}}\dd{}  
saṃnyāsagrahaṇāt\dd{}
\app{\lem[wit={B,E,L,U1}]{ṣaḍdarśanagrahaṇāt}
  \rdg[wit={ceteri}]{ṣaṭdarśanagrahaṇāt}}\dd{}
\app{\lem[wit={ceteri}]{siromuṃḍanāt}
  \rdg[wit={N2}]{siromaṃḍanāt}
  \rdg[wit={P}]{\om}}\dd{}
\app{\lem[wit={ceteri}]{anyopāyakaraṇāt}
  \rdg[wit={P}]{\om}}\dd{}\\
yogatattvaṃ na prāpyate/\\
%-----------------------------
%[p.76]
%sa tu yogaḥ gurusevayā prāpyate/ \E
%\om                              \P
%sa tu yogo  gurusevayā prāpyate/ \B
%sa tu yogo  gurusevayā prāpyate/ \L
%sa tu yogo  gurusevayā prāpyate śrī// \N1
%sa tu yogo  gurusevayā prāpyate// \N2
%sa tu yogo  gurusevayā prāpyate/ \D
%sa tu yogo  gurusevayā prāpyate \U1
%sa tu yogo  gurusevayā prāpyate// \U2
%-----------------------------
%The [reality of] yoga is truly attained by frequenting the teacher. 
%-----------------------------
\note[type=philcomm, labelb=401, lem={sa tu \ldots prāpyate}]{The senctence is \om in P.}
sa tu yogo gurusevayā prāpyate/
\end{prose}
\end{ekdosis}
%-----------------------------
%gurukṛpātaḥ pātrāṇāṃ    dṛḍhānāṃ  satyavādinām/  kathanād dṛṣṭipātād vā    sāṃnidhyād avalokanāt/ \E
%gurudṛkpātapātrāṇāṃ     dṛḍhānāṃ  satyavādinām   kathanāt dṛṣṭipātād vā    sāṃnidhyād avalokanāt \P
%gurudṛk/ pāt/ patrāṇāṃ  dṛḍhānāṃ  satyavādinām/  kathanād viṣapātād  vā     sānidhyāt  dyavatrokanāt//1//  \B
%gurudṛkpāt patrāṇāṃ               satyavādinām// kathanād viṣapānād  vā     sānnitdhy--avalokanāt//1//  \L
%gurudṛkpātapātrāṇāṃ     dṛḍhānāṃ  satyavādinām/  kathanād dṛṣṭipātād vā,   sānidhyād dhyavalokanāt//1//  \N1
%gurudṛkpātapātrāṇāṃ     dṛḍhānāṃ  satyavādinām/  kathanād dṛṣṭipātād vā    sānidhyād dhyavalokanāt//1//   \N2
%gurudṛkpātapātrāṇo      dṛḍhānāṃ  satyavādināṃ/  kathanād dṛṣṭipātād vā    sānidhyād dyavalokanāt//1// \D
%gurudakpātrāṇāṃ         dṛḍhānāṃ  satyavāridinām    ?upayādṛṣtipātād vā    sānidhyāty avalokanāt   \U1
%gurudṛkpātāpātrāṇāṃ     dṛḍhānāṃ  satyavādinām   kathanā  dṛṣṭipātād vā    sāṃnidhyād ddhyāvalokanāt//  \U2
%-----------------------------
%Among the firm, the truthful [and] among those worthy of the teacher's gaze, caused by [the teachers] narration or caused by [the teachers] glance, caused by the [mere] proximity [to the teacher, or] caused by looking at [the teacher] 
%-----------------------------
\begin{ekdosis}
  \begin{tlg}
\tl{      
\note[type=testium, labelb=402, lem={sa tu yogo}]{Ysv\textsuperscript{PT}: gurupādodakaṃ śiṣṭasevinā satyavādinā | kanyāstrādidṛṣṭipātaharṣagativivarttanāt |}
\note[type=source, labelb=403, lem={gurudṛkpāta°}]{SSP 5.61: gurudṛkpātanāt prāyo dṛḍhānāṃ satyavādināṃ sā sthitir jāyate |}
\app{\lem[wit={P,N1,N2,U2}]{gurudṛkpātapātrāṇāṃ}
  \rdg[wit={L}]{gurudṛkpāt patrāṇāṃ}
  \rdg[wit={B}]{gurudṛk | pāt | patrāṇāṃ}
  \rdg[wit={U1}]{gurudakpātrāṇāṃ}
  \rdg[wit={D}]{gurudṛkpātapātrāṇo}
  \rdg[wit={E}]{gurukṛpātaḥ pātrāṇāṃ}}
\app{\lem[wit={ceteri}]{dṛḍhānāṃ}
  \rdg[wit={L}]{\om}}
\app{\lem[wit={ceteri}]{satyavādinām}
  \rdg[wit={U1}]{satyavāridinām}}/}\\
\tl{
\app{\lem[wit={ceteri},alt={kathanād}]{kathanā\skp{d-dṛ}}
  \rdg[wit={U1}]{upayā°}
}\app{\lem[wit={ceteri},alt={dṛṣṭipātād}]{\skm{d-dṛ}ṣṭipātā\skp{d-vā}}
  \rdg[wit={B}]{viṣapātād}
  \rdg[wit={L}]{viṣapānād}}\skm{d-vā}
\app{\lem[wit={P,E,U2},alt={sāṃnidhyād}]{sāṃnidhyā\skp{d-a}}
  \rdg[wit={B}]{sānidhyāt}
  \rdg[wit={L}]{sānnitdhy}
  \rdg[wit={D,N1,N2}]{sānidhyād}
  \rdg[wit={U1}]{sānidhyāty}
}\app{\lem[wit={E,L,P,U1},alt={avalokanāt}]{\skm{d-a}valokanāt}
  \rdg[wit={B}]{dyavatrokanāt}
  \rdg[wit={N1,N2}]{dhyavalokanāt}
  \rdg[wit={U2}]{dhyāvalokanāt}
  \rdg[wit={D}]{dyavalokanāt}}\dd{}1\hskip-2pt\dd{}}
\end{tlg}
\end{ekdosis}
%-----------------------------
%sadguruprasādāt   samyak paramaṃ padaṃ pāpyate/   ata evaṃ vacaḥ proktaṃ na guror adhikaṃ param//1//      \E
%prasādāsya  guroḥ samyak prāpyate paramaṃ padaṃ   ata eva  vacaḥ proktaṃ na guror adhikaṃ paraṃ         \P
%prasāt   sadguroḥ saṃyak prāpyate paramaṃ padaṃ/  ara eva  vacaḥ proktaṃ na guror adhikaṃ paraṃ 2          \B
%prasādāt sadguroḥ saṃyak prāpyate paramaṃ padaṃ// ata eva  vacaḥ proktaṃ na guror adhikaṃ paraṃ//2//  \L
%prasādāt sadguroḥ saṃyak prāpyate paramaṃ padaṃ/  ata eva  vacaḥ proktaṃ na guror adhikaṃ paraṃ//2//   \N1 S.13 verso
%prasādāt sadguroḥ saṃyak prāpyate paramaṃ padaṃ/  ata eva  vacaḥ proktaṃ na guror adhikaṃ paraṃ//2//  \N2
%prasādāt sadguroḥ saṃyak prāpyate paramaṃ padaṃ// ata eva  vacaḥ proktaṃ na guror adhikaṃ paraṃ//2    \D
%prasādāt sadguroḥ saṃyak prāpyate paramaṃ padaṃ// ata eva  vacaḥ proktaṃ na guror adhikaṃ paraṃ//     \U1
%prasādāt sadguroḥ saṃyak prāpyate paramaṃ padaṃ// ata eva  vacaḥ proktaṃ na guror adhikaṃ paraṃ//     \U2
%-----------------------------
%caused by the favor of the good teacher, truely one [of those] attains the highest place. For this very reason the advice is stated: There is nothing greater than the guru.   
%-----------------------------
\begin{ekdosis}
  \begin{tlg}
    \tl{
\note[type=testium, labelb=404, lem={prasādāt\ldots}]{Ysv\textsuperscript{PT}: prasādāt sadguroḥ samyak prāpnoti paramaṃ padam | na guror adhikaṃ tattvaṃ yat tasmāt paramaṃ padam |}
\note[type=source, labelb=405, lem={prasādāt\ldots}]{SSP 5.62: ata eva śivenoktam na guror adhikaṃ na guror adhikaṃ na guror adhikaṃ |}
\app{\lem[wit={ceteri}]{prasādāt\skp{-}sadguroḥ}
  \rdg[wit={E}]{sadguruprasādāt}
  \rdg[wit={P}]{prasādāsya guroḥ}
  \rdg[wit={B}]{prasāt sadguroḥ}}
samyak
\app{\lem[wit={ceteri}]{prāpyate paramaṃ padaṃ}
  \rdg[wit={E}]{paramaṃ padaṃ pāpyate}}/}\\
\tl{
  \app{\lem[wit={ceteri}]{ata eva}
    \rdg[wit={E}]{ata evaṃ}}
  vacaḥ proktaṃ na guror-adhikaṃ
  \app{\lem[wit={ceteri}]{paraṃ}
    \rdg[wit={E}]{param}}\dd{}2\hskip-2pt\dd{}}
\end{tlg}
\end{ekdosis}
%-----------------------------
%vāṅmātrād bodha dṛkpātād yaḥ karoti śamaṃ kṣaṇāt/  prasphuṭad bhrāṃtihṛttoṣaṃ       svacchaṃ vaṃde guruṃ param// 2// \E
%vāṅmātrād vātha dṛkpātād yaḥ karoti śamaṃ kṣaṇāt   prasphuṭad bhrāṃtihṛttoṣaṃ       svacchaṃ vaṃde guruṃ param      \P
%vāṅmātrād vātha dṛkpītād yaḥ karoti śamaṃ kṣaṇāt/  prasphaṭad bhātihatoṣaṃ          svachaṃ vaṃde guruṃ param// 3// \B
%vāṅmātrād vātha dṛkpātād yaḥ karoti śamaṃ kṣaṇāt// prasphaṭad bhāti hatoṣaṃ         svachaṃ vaṃde guruṃ param// 3// \L
%vāṅmātrād vātha dṛkpātād yaḥ karoti śamaṃ kṣaṇāt// prasphaṭat bhrāṃti hatddoṣaṃ    svacchaṃ vade karaṃ parī [<-oder->] parāṃ// 3// \N1
%vāṅmātrād vātha dṛkpātād yaḥ karoti śasaṃ kṣaṇāt// prasphaṭa--bhrāṃti haddoṣaṃ-----tvacchaṃ vedakaraṃ paraṃ// 3// \N2 S.11
%vāṅmātrād vātha dṛkpātād yaḥ karoti śamaṃ kṣaṇāt/  prasphuṭat bhrāṃti hṛddoṣaṃ???   svachaṃ vedakakaraṃ paraṃ// 3/ \D
%vāṅmātrād vātha dṛkpātād yaḥ karoti samaṃ kṣaṇāt// prasphuṭad bhrāṃti ittoṣaṃ?      svachaṃ vaṃde guruṃ paraṃ//\U2
%\om \U1
%-----------------------------
%Who immediately makes peace of mind from his mere utterance (\textit{vāṅmātrād}) or by his mere glance (\textit{vāṅmātrād}), who makes the peace of the soul by crushing doubt, I bow in front of the teacher who is pure, who is supreme.
%Who immediately makes peace of mind from his mere utterance (\textit{vāṅmātrād}) or by his mere glance (\textit{vāṅmātrād}), I bow in front of the teacher who is pure, supreme [and] appeases the soul for those who are full of doubt.
%-----------------------------
\begin{ekdosis}
  \begin{tlg}
    \tl{
      % A teacher is the one who instantly imparts knowledge of the self through initiating one in the highest place by his mere utterance or by mere glance or by initiation or vedha.!?
\note[type=source, labelb=406, lem={vāṅmātrād}]{SSP 5.64: vāṅmātrād vātha dṛkpātāt yaḥ karoti ca tatkṣaṇāt | prasphuṭaṃ śāmbhavaṃ vedhaṃ svasaṃvedyaṃ paraṃ padam |}
\note[type=philcomm, labelb=407, lem={vāṅmātrād \ldots paraṃ}]{The sentence is \om in U\textsubscript{1}. It is the beginning of a larger gap in U\textsubscript{1}. Omissions will not be recorded. The reader will be informed once the evidence resumes.}
vāṅmātrā\skp{d-vā}\app{\lem[wit={ceteri},alt={vātha}]{\skm{d-vā}tha}
  \rdg[wit={E}]{bodha}}
\app{\lem[wit={ceteri},alt={dṛkpātād}]{dṛkpātā\skp{d-yaḥ}}
  \rdg[wit={B}]{dṛkpītād}
}\skm{d-yaḥ} karoti
\app{\lem[wit={ceteri}]{śamaṃ}
  \rdg[wit={N2}]{śasaṃ}} kṣaṇāt/}\\
\tl{
\app{\lem[type=emendation, resp=egoscr,alt={prasphuṭa°}]{prasphuṭa}
  \rdg[wit={N2}]{\korr prasphaṭa°}
  \rdg[wit={B,L}]{prasphaṭad}
  \rdg[wit={N1}]{prasphaṭat}
  \rdg[wit={E,P,U2}]{prasphuṭad}
  \rdg[wit={D}]{prasphuṭat}
}\app{\lem[wit={ceteri},alt={°bhrānti°}]{bhrānti}
  \rdg[wit={B,L}]{°bhāti°}
}\app{\lem[wit={E,P}]{hṛttoṣaṃ}
  \rdg[wit={B,L}]{hatoṣaṃ}
  \rdg[wit={N1}]{hatddoṣaṃ}
  \rdg[wit={N2}]{haddoṣaṃ}
  \rdg[wit={D}]{hṛddoṣaṃ}
  \rdg[wit={U2}]{ittoṣaṃ}}
\app{\lem[wit={ceteri}]{svacchaṃ}
  \rdg[wit={N2}]{tvacchaṃ}}
\app{\lem[wit={B,E,L,P,U2}]{vande}
    \rdg[wit={N1}]{vade}
    \rdg[wit={N2,D}]{veda°}}
\app{\lem[wit={B,E,L,P,U2}]{guruṃ}
  \rdg[wit={N1}]{karaṃ}
  \rdg[wit={N2}]{°karaṃ}
  \rdg[wit={D}]{vedakakaraṃ}}
\app{\lem[wit={ceteri}]{paraṃ}
  \rdg[wit={N1}]{parāṃ}}\dd{}3\hskip-2pt\dd{}}
\end{tlg}
\end{ekdosis}
\ekdpb*{}
%%%%%%%%%%%%%%%%%%%%%%%%%%%%%%%%%%%%%%%%%%
%%%%%%%%PAGEBREAK%%%%%%%PAGEBREAK%%%%%%%%%
%%%%%%%%%%%%%%%%%%%%%%%%%%%%%%%%%%%%%%%%%%
%%%%%%%%%%%%%%%%PAGEBREAK%%%%%%%%%%%%%%%%%
%%%%%%%%%%%%%%%%%%%%%%%%%%%%%%%%%%%%%%%%%%
%%%%%%%%PAGEBREAK%%%%%%%PAGEBREAK%%%%%%%%%
%%%%%%%%%%%%%%%%%%%%%%%%%%%%%%%%%%%%%%%%%%
%%%%%%%%%%%%%%%%%%%%%%%%%%%%%%%%%%%%%%%%%%
%%%%%%%%%%%%%%%%%%%%%%%%%%%%%%%%%%%%%%%%%%
%%%%%%%%%%%%%%%%%%%%%%%%%%%%%%%%%%%%%%%%%%
%%%%%%%%PAGEBREAK%%%%%%%PAGEBREAK%%%%%%%%%
%%%%%%%%%%%%%%%%%%%%%%%%%%%%%%%%%%%%%%%%%%
%%%%%%%%%%%%%%%%PAGEBREAK%%%%%%%%%%%%%%%%%
%%%%%%%%%%%%%%%%%%%%%%%%%%%%%%%%%%%%%%%%%%
%%%%%%%%PAGEBREAK%%%%%%%PAGEBREAK%%%%%%%%%
%%%%%%%%%%%%%%%%%%%%%%%%%%%%%%%%%%%%%%%%%%
%%%%%%%%%%%%%%%%%%%%%%%%%%%%%%%%%%%%%%%%%%
%%%%%%%%%%%%%%%%%%%%%%%%%%%%%%%%%%%%%%%%%%
%%%%%%%%%%%%%%%%%%%%%%%%%%%%%%%%%%%%%%%%%%
%%%%%%%%PAGEBREAK%%%%%%%PAGEBREAK%%%%%%%%%
%%%%%%%%%%%%%%%%%%%%%%%%%%%%%%%%%%%%%%%%%%
%%%%%%%%%%%%%%%%PAGEBREAK%%%%%%%%%%%%%%%%%
%%%%%%%%%%%%%%%%%%%%%%%%%%%%%%%%%%%%%%%%%%
%%%%%%%%PAGEBREAK%%%%%%%PAGEBREAK%%%%%%%%%
%%%%%%%%%%%%%%%%%%%%%%%%%%%%%%%%%%%%%%%%%%
\begin{ekdosis}
  \begin{tlg}
%-----------------------------
%samyag ānandajananaḥ  sadguruḥ sobhidhīyate/   nimeṣārddhaṃ vā tatpādaṃ    yad vākyād avalokanāt// 3// \E[p.78]
%samyag ānandajananaḥ  sadguruḥ sobhidhīyate    nimiṣārddhaṃ vā tatpādaṃ    yad vākyād avalokanāt       \P
%samyag ānandajananaḥ/ sadguruḥ sobhidhīyate/   nimeṣārddhā  vā tatpāda     yad vākyād avalokanāt// 3//  \B
%samyag ānandajananaḥ  sadguruḥ sobhidhīyate//  nimeṣārddhā  vā tatpāda     yad vākyād avalokanāt// 3// \L %%%%0039.jpg
%samyag ānandajananaṃ  sadguruḥ sobhidhīyate/   nimeṣārddhaṃ ca    pādaṃ vā yad vākyād avalokanāt/ \N1
%samyag ānandajananaṃ  sadguruḥ sobhidhīyate/   nimiṣārddhaṃ ca    pādaṃ vā yad vākyād avalokanāt  \N2
%samyag ānandajananaṃ  sadguruḥ sobhidhīyate/   nimeṣārddhaṃ ca    pādaṃ vā yad vākyād avalokanāt/ \D
%samyag ānandajananaḥ  sadguruḥ  sobhidhīyate// nimeṣārddhaṃ vā tatpādaṃ    yad vākyād avalokanāt// \U2
%\om \U1
%-----------------------------
%He, who is the progenitor of absolute bliss, is known to be the true teacher. Because of a glance for just half a wink on the [teacher's] feet [or] by [just] talking about [it], ...  
%-----------------------------
\tl{
  \note[type=source, labelb=408, lem={samyag}]{SSP 5.64cd-5.65ab: samyag ānandajanakaḥ sadguruḥ so 'bhidhīyate | nimiṣārdhārdhapātād vā yad vā pādāvalokanāt |}
  \note[type=testium, labelb=409, lem={samyag}]{Ysv\textsuperscript{PT}: nimeṣārddhena tasyaiva ājñāpālanato bhavet | mahānandaśataprāptis tasmai śrīgurave namaḥ |}
samyag-ānanda\app{\lem[wit={B,E,L,P,U2},alt={°jananaḥ}]{jananaḥ}
  \rdg[wit={D,N1,N2}]{jananaṃ}} sadguruḥ sobhidhīyate/}\\
\tl{
  \app{\lem[wit={ceteri}]{nimeṣārddhaṃ}
    \rdg[wit={P,N2}]{nimiṣārddhaṃ}
    \rdg[wit={B,L}]{nimeṣārddhā}}
  \app{\lem[wit={ceteri}]{vā}
    \rdg[wit={D,N1,N2}]{ca}}
  \app{\lem[wit={E,P,U2}]{tatpādaṃ}
    \rdg[wit={B,L}]{tatpāda}
    \rdg[wit={D,N1,N2}]{pādaṃ vā}}
yad-vākyād-avalokanāt\dd{}4\hskip-2pt\dd{}}
\end{tlg}
\end{ekdosis}
%-----------------------------
%svātmā  sthiratvam  āyāti   tasmai śrīgurave  namaḥ/    nānāviplava--viśrāntiḥ  kathanāt  kurute tataḥ// 4// \E
%svātmā/ sthiratvam  āyāti   tasmai śrīgurave  namaḥ     nānāvikalpaḥ viśrāṃtiḥ  kathanāt  kurute tu yaḥ      \P einziges daṇḍa im Text! 
%svātmā  sthiraṃtvam āyāti   tasmai śrīgurave  namaḥ/    nānāvikalpa--viśrāṃti   kathanāt/ kurute tu yaḥ/   \B
%svātmā  sthiratvam  āyāti   tasmai śrīgurubho namaḥ//   nānāvikalpa--viśrāṃti   kathanāt  kurute tu yaḥ//   \L
%svātmā  sthiratvam  āyāti   tasmai śrīgurave  namaḥ/    nānāvikalpa--viśrāṃtiṃ  kathanāt  kurute tu saḥ/   \N1
%svātmā  sthiratvam  āyāti   tasmai śrīgurave  namaḥ/    nānāvikalpa--viśrāṃti   kathanāt  kurute tu saḥ//5// \N2
%svātmā  sthiratvam  āyāti   tasmai śrīgurave  namaḥ//   nānāvikalpaṃ viśrāṃtiṃ  kathanāt  kurute tu saḥ/      \D
%svātmā  sthiratvam  āyāti   tasmai śrīguru namo namaḥ// nānāvikalpaviśrāntiṃ    kathanāt  kurute tu yaḥ// \U2
%\om \U1
%-----------------------------
%... the own self goes into stability, homage to that teacher, who brings all doubts to stop because of [his] advice.   
%-----------------------------
\begin{ekdosis}
\begin{tlg}
\tl{
  \note[type=source, labelb=410, lem={svātmā}]{SSP 5.65cd-5.66ab: svātmānaṃ sthiram ādhatte tasmai śrīgurave namaḥ | nānāvikalpaviśrāntiṃ kathayā kurute tu yaḥ |}
 \note[type=testium, labelb=411, lem={nānāvikalpa°}]{Ysv\textsuperscript{PT}: nānāvikalpavibhrāntināśañca kurute tu yaḥ | sadguruḥ sa tu vijñeyo na tu vairaprakalpakaḥ |}
svātmā sthiratvam-āyāti tasmai
\app{\lem[wit={ceteri}]{śrīgurave}
  \rdg[wit={L}]{śrīgurubho}
  \rdg[wit={U2}]{śrīguru namo}} namaḥ/}\\
\tl{
\app{\lem[wit={N1,U2}]{nānāvikalpaviśrāntiṃ}
  \rdg[wit={D}]{nānāvikalpaṃ viśrāṃtiṃ}
  \rdg[wit={E}]{nānāviplavaviśrāntiḥ}
  \rdg[wit={P}]{nānāvikalpaḥ viśrāṃtiḥ}
  \rdg[wit={B,L}]{nānāvikalpaviśrāṃti}
  \rdg[wit={N2}]{nānāvikalpaviśrāṃti}}
\app{\lem[wit={ceteri},alt={kathanāt}]{kathanā\skp{t-ku}}
  \rdg[wit={B}]{kathanāt |}
}\skm{t-ku}rute
\app{\lem[wit={B,L,P,U2}]{tu yaḥ}
  \rdg[wit={E}]{tataḥ}
  \rdg[wit={D,N1,N2}]{tu saḥ}}/}\\
%-----------------------------
%sadguruḥ sa tu vijñeyo na tu vai priyajalpakaḥ// 5// \E
%sadguruḥ sa tu vijñeyo na tu vipriyajalpakaḥ      5 \P
%sadguruḥ sa tu vijño   nnu   viprāyajalākaḥ// 5//        \B
%sadguruḥ sa tu vijño   nnu   viprāyajalākaḥ// 5//        \L
%sadguruḥ sa tu vijñeyo na tu vipriyajalpakaḥ//       \N1
%sadguruḥ sa tu vijñeyo na tu vipriyajalpakaḥ//6//    \N2
%sadguruḥ sa tu vijñeyo na tu vipriyajalpakaḥ/       \D
%sadguruḥ sa tu vijñeyo na tu vipriyajalpakaḥ// \U2
%\om \U1
%-----------------------------
%He is known to be true teacher, not an unpleasant disputant.  
%-----------------------------
\tl{
  \note[type=source, labelb=410, lem={sadguruḥ}]{SSP 5.66cd: sadguruḥ sa tu vijñeyo na tu mithyāviḍambakaḥ||5.67||}
  sadguruḥ sa tu
  \app{\lem[wit={ceteri}]{vijñeyo}
    \rdg[wit={B,L}]{vijño}}
  \app{\lem[wit={ceteri}]{na tu}
    \rdg[wit={B,L}]{nnu}}
\app{\lem[wit={ceteri}]{vipriyajalpakaḥ}
  \rdg[wit={B,L}]{viprāyajalākaḥ}
  \rdg[wit={E}]{vai priyajalpakaḥ}}\dd{}5\hskip-2pt\dd{}}
\end{tlg}
\end{ekdosis}
\bigskip
\begin{ekdosis}
  \begin{prose}
%-----------------------------
%ata eva paramapadasya prāpty arthaṃ sadguruḥ sevyaḥ sarvadā           yaḥ puruṣaḥ satyavādī bhavati/ \E
%ata eva paramapadasya prāpty arthaṃ sadguruḥ sevyaḥ sarvadā           yaḥ puruṣaḥ satyavādī bhavati  \P %%%7677.jpg
%ata eva paramapada----prāpty arthaṃ sadguruḥ sevya--sarvadā/          yaḥ puruṣaḥ satyavādī bhavati/ \B
%ata eva paramapada----prāpty arthaṃ sadguruḥ sevya--sarvadā//         yaḥ puruṣaḥ satyavādī bhavatī// \L
%ata eva paramapada----prāpty arthaṃ sadguruḥ        sarvadā vaṃdyaḥ?/ yaḥ puruṣaḥ satyavādī bhavati/ \N1
%ata eva paramapada----prāpty arthaṃ sadguruḥ        sarvadā vaṃdyaḥ/  yaḥ puruṣaḥ satyavādī bhavati/ \N2 S. 11 verso Zeile 6. 
%ata eva paramapada----prāpty arthaṃ sadguruḥ        sarvadā vaṃdyaḥ/  yaḥ puruṣaḥ satyavādī bhavati \D
%ata eva paramapada----prāpty arthaṃ sadguruḥ sevyaḥ sarvadā           yaḥ puruṣaḥ satyavādī bhavati// \U2
%\om \U1
%-----------------------------
%Hence the true teacher is always is to be praised to attain the highest place. That person is a speaker of truth. 
%-----------------------------
\note[type=source, labelb=411, lem={ata eva}]{SSP 5.67: ata eva paramapadaprāpty arthaṃ sa sadguruḥ sadā vandanīyaḥ |}
\note[type=testium, labelb=412, lem={ata eva}]{Ysv\textsuperscript{PT}: ata eva maheśāni sadguruḥ śiva āditaḥ | satyavādī ca sacchīlo gurubhakto dṛḍhavrataḥ |}
ata eva
\app{\lem[wit={ceteri},alt={paramapadaprāpty}]{paramapadaprāpt\skp{y-a}}
  \rdg[wit={E,P}]{paramapadasya prāpty}
}arthaṃ sadguruḥ
\app{\lem[wit={D,N1,N2}]{sarvadā vandyaḥ}
  \rdg[wit={E,P,U2}]{sevyaḥ sarvadā}
  \rdg[wit={B,L}]{sevyasarvadā}}/
yaḥ puruṣaḥ satyavādī bhavati/ 
%-----------------------------
%niraṃtaraṃ gurusevā tatparo bhavati/ \E [p.79]
%niraṃtara--gurusevā rato bhavati   \P
%niraṃtaraṃ gurusevā taro bhavati/   \B
%niraṃtaraṃ gurusevā rato bhavati...   \L
%niraṃtaraṃ gurusevā rato bhavati   \N1
%niraṃtaraṃ gurusevā rato bhavati   \N2
%niraṃtaraṃ gurusevā rato bhava   \D
%niraṃtaraṃ gusevā   rato bhavati// \U2
%\om \U1
%-----------------------------
%Uninterrupted devotion for frequenting the teacher arises. 
%-----------------------------
\app{\lem[wit={ceteri}]{nirantaraṃ}
  \rdg[wit={P}]{niraṃtara°}}
\app{\lem[wit={ceteri}]{gurusevā}
  \rdg[wit={U2}]{gusevā}}
\app{\lem[wit={ceteri}]{rato}
  \rdg[wit={B}]{taro}
  \rdg[wit={E}]{tatparo}}
\app{\lem[wit={ceteri}]{bhavati}
  \rdg[wit={D}]{bhava}}/
%-----------------------------
%yasya manasi pāpaṃ na bhavati/ \E
%yasya manasi pāpaṃ na bhavati  \P
%yasya manasi pāpa nna bhavati/ \B
%yasya manasi pāpaṃ nna bhavati/ \L
%yasya manasi pāpaṃ nna bhavati/ \N1
%yasya manasi pāpaṃ na bhavati/ \N2
%yasya manasi pāpaṃ nna bhavati/ \D
%\om \U1
%yasya manasi pāpa na bhavati// \U2
%-----------------------------
%He becomes one in whose mind evil does not arise. 
%-----------------------------
yasya manasi
\app{\lem[wit={ceteri}]{pāpaṃ}
  \rdg[wit={B}]{pāpa}}
na bhavati/ 
%-----------------------------
%svācārarataḥ   snānādiśīlo bhavati/ \E
%svācārarataḥ   snānodiśīlo bhavati \P
%svācāraratāḥ   snānādiśīlo bhavatī/ \B
%svācāraratāḥ   snānādiśīlo bhavatī/ \L
%svasyācārarato snānādiśīlo bhavati/ \N1
%svasyācārarato snānādiśīlo bhavati/ \N2
%svasyācārarato snānādiśīlo bhavati/ \D
%\om \U1
%svācārataḥ// snānādiśīlo bhavati// \U2
%-----------------------------
%Being someone who is devoted to good habits, habits such as ceremonial bathing etc. arise. 
%-----------------------------
\note[type=source, labelb=413, lem={svācārarataḥ}]{Ysv\textsuperscript{PT}: svalpācāraratātmā yo dānādiśīlasaṃyutaḥ | kāpaṭyalobhavinyāsau mahāvaṃśasamudbhavaḥ |}
\app{\lem[wit={E,P}]{svācārarataḥ}
  \rdg[wit={B,L}]{svācāraratāḥ}
  \rdg[wit={U2}]{svācārataḥ ||}
  \rdg[wit={D,N1,N2}]{svasyācārarato}}
snānādiśīlo bhavati/ 
%-----------------------------
%kāpaṭyaṃ na bhavati   yasya vaṃśaparaṃparā jñāyate/ \E
%kāpaṭyaṃ na bhavati   yasya vaṃśaparaṃparā jñāyate  \P
%kāpaṭyaṃ bhavati/     yasya vaṃśaparaṃparā jñāyate/ \B
%kāpaṭyaṃ na bhavati/  yasya vaṃśaparaṃparā jñāyate... \L
%kāpaṭyaṃ nāsti/       yasya vaṃśaparaṃparā jñāyate/ \N1
%kāpaṭyaṃ nāsti/       yasya vaṃśaparaṃparā jñāyate/ \N2
%kāpaṭyaṃ nāsti/       yasya parāparaṃparā  jñāyate/ \D
%\om \U1
%kāpaṭyaṃ na bhavati// yasya vaṃśaparaṃ parā jñāyate// \U2
%-----------------------------
%Deceiving does not arise. His noble race is recognized by him.
%-----------------------------
kāpaṭyaṃ \app{\lem[wit={E,P,L,U2}]{na bhavati}
  \rdg[wit={B}]{bhavati}
  \rdg[wit={D,N1,N2}]{nāsti}}/
yasya \app{\lem[wit={ceteri}]{vaṃśaparaṃparā}
  \rdg[wit={D}]{parāparaṃparā}}
jñāyate/ 
%-----------------------------
%etādṛśasya sadguroḥ saṃgatiḥ karttavyā  tena puruṣasya manaḥ śāṃtiṃ prāpnoti/ \E
%etādṛśasya sadguroḥ saṃgati  karttavyā       puruṣasya manaḥ śāṃtiṃ prāpnoti  \P
%etādṛśasya sadguroḥ saṃgatī  karttavyā/      puruṣasya manaḥ śāṃti  prāpnoti/ \B
%etādṛśasya sadguroḥ saṃgatī  karttavyā       puruṣasya manaḥ śāṃti  prāpnoti... \L
%etādṛśasya sadguroḥ saṃgatiḥ  kattavyāḥ      puruṣasya manaḥ śāṃtiṃ  prāpnoti/ \N1
%etādṛśasya sadguroḥ saṃgati   karttavyāḥ     puruṣasya manaḥ śāṃtiṃ  prāpnoti// \N2
%etādṛśasya sadguroḥ saṃgatiḥ  kattavyāḥ/     puruṣasya manaḥ śāṃtiṃ  prāpnoti/ \D
%etādṛśasya guroḥ    saṃgatiḥ  karttavyā//    puruṣasya mano śāṃtiṃ  prāpnoti// \U2
%\om \U1
%-----------------------------
%One shall associate with such a true teacher. The mind of such a person attains peace.
%-----------------------------
\note[type=source, labelb=414, lem={etādṛśasya}]{Ysv\textsuperscript{PT}: īdṛśaḥ sadgurustasya saṅgatau yatnavān bhavet | tad eva manasaḥ śāntiṃ prāpnoti paramaṃ padam |}
etādṛśasya
\app{\lem[wit={ceteri}]{sadguroḥ}
  \rdg[wit={U2}]{guroḥ}}
  \app{\lem[wit={E,D,N1,U2}]{saṃgatiḥ}
    \rdg[wit={P,N2}]{saṃgati}
    \rdg[wit={B,L}]{saṃgatī}}
  \app{\lem[wit={B,E,L,P,U2}]{karttavyā}
    \rdg[wit={D,N1}]{kattavyāḥ}
    \rdg[wit={N2}]{karttavyāḥ}}
  \app{\lem[wit={E}]{tena}
    \rdg[wit={ceteri}]{\om}}
  puruṣasya
  \app{\lem[wit={ceteri}]{manaḥ}
    \rdg[wit={U2}]{mano}}
  \app{\lem[wit={ceteri}]{śāntiṃ}
    \rdg[wit={B,L}]{śāṃti}}
  prāpnoti/
%-----------------------------
%atha ca yasya manomadhye sthira ānanda utpadyate  so pi sadguruḥ kathyate/ \E
%atha ca yasya manomadhye sira   ānanda utpadyate  so pi sadguruḥ kathyate \P
%atha ca yasya manomadhye sīraḥ  ānaṃda utpadyate  so pi sadguruḥ kathyate... \B
%atha ca yasya manomadhye sīraḥ  ānaṃda utpadyate  so pi sadguruḥ kathyate \L
%atha ca yasya manomadhye sthira ānanda utpadyate  so pi sadguruḥ kathyate \N1
%atha ca yasya manomadhye sthīrānaṃda   utpadyate/ so pi sadguruḥ kathyate// \N2
%atha ca yasya manomadhye sthira ānaṃda utpadyate/ so pi sadguruḥ kathyate/ \D
%atha ca       manomadhye sthira ānanda utpadyate//so pi sadguruḥ kathyate// \U2
%\om \U1
%-----------------------------
%And he in whose mind arises steady bliss is also called a true teacher. 
%-----------------------------
  atha ca \app{\lem[wit={ceteri}]{yasya}
    \rdg[wit={U2}]{\om}}
  manomadhye
  \app{\lem[wit={E,D,N1,N2}]{sthira}
    \rdg[wit={B,L}]{sīraḥ}
    \rdg[wit={P}]{sira}
    \rdg[wit={N2}]{sthīrā°}}ānanda utpadyate/ so 'pi sadguruḥ kathyate/ 
%-----------------------------
% \om                                                                                                           \E
%atha ca                ghaṭikārdhaṃ   ghaṃṭikā   caturthāṃśo       vā  yasya pārśvam upaviṣṭe satyatādṛṣo bhāvo manomadhya utpadyate    \P
%atha ca                ghaṭikārdhaṃ   ghaṃṭikā   caturthāṃśo       vā  yasya pārśvam upaviṣṭe satyatādṛṣo bhāvo manomadhya uppapadyate/ \B
%atha ca                ghaṭikārdhaṃ   ghaṭikā    caturthaṃśo       vā  yasya pārśvam upaviṣṭe satyetādṛṣo bhāvo manomadhya upapadyate/  \L
%atha ca ghaṭī .. .. mo ghaṭikārddhaṃ  ghaṭikāyāḥ caturtho ḍaṃśo    vā  yasya pārśvam upaviṣṭe satyetādṛśo bhavo manomadhye utpadyate/  \N1 
%atha ca ghaṭimātra-----ghaṃṭikārddhaṃ ghaṭikā----caturtho daṃśo    vā  yasya pārśvem upaviṣṭe satyatādṛṣo .. .. manomadhye utpadyate/ \N2
%atha ca ghaṭīṃ mātraṃ  ghaṭikārddhaṃ  ghaṭikāyāḥ caturtho  aṃśo    vā  yasya pārśvam upaviṣṭe satyetādṛṣo bhāvo manomadhye utpadyate/ \D %%%%p.18 verso
%atha                   ghaṭikā        ghaṭikā    caturthāṃśo       vā  yasya pārśvam upaviṣṭe satyatādṛṣo bhāvo manomadhye utpadyate//    \U2
%\om \U1
%-----------------------------
%And then such a state of reality is generated of one who is seated at the side of him [the teacher] for a \textit{ghaṭikā}\footnote{[1/60th part of a day (24 minutes). siehe Wörterbuch], half a \textit{ghaṭikā} or a quarter part of a \textit{ghaṭikā}. 
%-----------------------------
\note[type=philcomm, labelb=415, lem={atha ca ghaṭīmātraṃ \ldots utpadyate}]{E omits this sentence.}
atha
\app{\lem[wit={ceteri}]{ca}
  \rdg[wit={U2}]{\om}} 
\app{\lem[type=emendation, resp=egoscr]{ghaṭimātraṃ}
  \rdg[wit={N2}]{\korr ghaṭimātra°}
  \rdg[wit={D}]{ghaṭīṃ mātraṃ}
  \rdg[wit={N1}]{ghaṭī....mo}
  \rdg[wit={ceteri}]{\om}}
\app{\lem[wit={B,L,P,N1,D}]{ghaṭikārdhaṃ}
  \rdg[wit={N2}]{°ghaṃṭikārddhaṃ}
  \rdg[wit={U2}]{ghaṭikā}}
\app{\lem[wit={L,N2,U2},alt={ghaṭikā°}]{ghaṭikā}
  \rdg[wit={N1,D}]{ghaṭikāyāḥ}
  \rdg[wit={B,P}]{ghaṃṭikā°}}\app{\lem[wit={B,P,L,U2}]{caturthāṃśo}
  \rdg[wit={N1}]{caturtho ḍaṃśo}
  \rdg[wit={N2}]{caturtho daṃśo}
  \rdg[wit={D}]{caturtho aṃśo}}
vā yasya pārśvaṃ upaviṣṭe satyatādṛṣo bhāvo manomadhye \app{\lem[wit={ceteri}]{utpadyate}
  \rdg[wit={B,L}]{uppapadyate}}/
%-----------------------------
%\om                           \E
%gatvā vanamadhye  sthiyate  gṛhaṃ tyajyate so pi sadguruḥ kathyate    \P
%gatvā vanamadhye/ sthiyate/ gṛhaṃ tyajyate so pi sadguruḥ kathyate/\B
%gatvā vanamadhye sthīyate... gṛhaṃ tyajyate so pi sadguruḥ kathyate...\L
%gatvā vanamadhye sthīyate/ gṛhaṃ tyajyate \N1
%gatvā vanamadhye sthīyate/ gṛhaṃ tyajyate \N2
%gatvā vanamadhye sthīyate/ gṛhaṃ tyajyate \D
%\om \U1
%gatvā vanamadhye  sthīyate// gṛhaṃ tyajyate// so pi sadguruḥ kathyate//  \U2
%-----------------------------
%A true teacher is taught to be one that has left the house and went into the forest,
%-----------------------------
\note[type=philcomm, labelb=416, lem={gatvā \ldots tyajyate \ldots kathyat e}]{E omits both sentences.}
gatvā vanamadhye sthīyate/ gṛhaṃ tyajyate
\app{\lem[wit={ceteri}]{so}
  \rdg[wit={D,N1,N2}]{\om}} 
\end{prose}
\end{ekdosis}
\ekdpb*{}
%%%%%%%%%%%%%%%%%%%%%%%%%%%%%%%%%%%%%%%%%%
%%%%%%%%PAGEBREAK%%%%%%%PAGEBREAK%%%%%%%%%
%%%%%%%%%%%%%%%%%%%%%%%%%%%%%%%%%%%%%%%%%%
%%%%%%%%%%%%%%%%PAGEBREAK%%%%%%%%%%%%%%%%%
%%%%%%%%%%%%%%%%%%%%%%%%%%%%%%%%%%%%%%%%%%
%%%%%%%%PAGEBREAK%%%%%%%PAGEBREAK%%%%%%%%%
%%%%%%%%%%%%%%%%%%%%%%%%%%%%%%%%%%%%%%%%%%
%%%%%%%%%%%%%%%%%%%%%%%%%%%%%%%%%%%%%%%%%%
%%%%%%%%%%%%%%%%%%%%%%%%%%%%%%%%%%%%%%%%%%
%%%%%%%%%%%%%%%%%%%%%%%%%%%%%%%%%%%%%%%%%%
%%%%%%%%PAGEBREAK%%%%%%%PAGEBREAK%%%%%%%%%
%%%%%%%%%%%%%%%%%%%%%%%%%%%%%%%%%%%%%%%%%%
%%%%%%%%%%%%%%%%PAGEBREAK%%%%%%%%%%%%%%%%%
%%%%%%%%%%%%%%%%%%%%%%%%%%%%%%%%%%%%%%%%%%
%%%%%%%%PAGEBREAK%%%%%%%PAGEBREAK%%%%%%%%%
%%%%%%%%%%%%%%%%%%%%%%%%%%%%%%%%%%%%%%%%%%
%%%%%%%%%%%%%%%%%%%%%%%%%%%%%%%%%%%%%%%%%%
%%%%%%%%%%%%%%%%%%%%%%%%%%%%%%%%%%%%%%%%%%
%%%%%%%%%%%%%%%%%%%%%%%%%%%%%%%%%%%%%%%%%%
%%%%%%%%PAGEBREAK%%%%%%%PAGEBREAK%%%%%%%%%
%%%%%%%%%%%%%%%%%%%%%%%%%%%%%%%%%%%%%%%%%%
%%%%%%%%%%%%%%%%PAGEBREAK%%%%%%%%%%%%%%%%%
%%%%%%%%%%%%%%%%%%%%%%%%%%%%%%%%%%%%%%%%%%
%%%%%%%%PAGEBREAK%%%%%%%PAGEBREAK%%%%%%%%%
%%%%%%%%%%%%%%%%%%%%%%%%%%%%%%%%%%%%%%%%%%
  \begin{ekdosis}
    \begin{prose}
      \noindent
\app{\lem[wit={B,L,P,U2}]{'pi sadguruḥ kathyate}
  \rdg[wit={D,N1,N2}]{\om}}/ 
%-----------------------------
%kasyāpi duḥkhaṃ na dīyate/ \E
%kasyāpi duḥkhaṃ na dīyate  \P
%kasyāpi duḥkhaṃ na dīyate/ \B
%kasyāpi duḥkhaṃ na dīyate... \L
%kasyāpi duḥkhaṃ na dīyate  \N1
%kasyāpi duḥkhaṃ na dīyate//  \N2
%kasyāpi duḥkhaṃ na dīyate//  \D
%\om \U1
%kasyāpi duḥkhaṃ na dīyate/ \U2
%-----------------------------
% one who does not harm anyone,  
%-----------------------------
kasyāpi duḥkhaṃ na dīyate/
%-----------------------------
%prāṇimātreṇa saha maitrī kriyate  kasyāpi doṣaṃ na kathayati         so pi sadguruḥ kathyate// \E
%prāṇimātreṇa saha maitrī kriyate  kasyāpi doṣo  na prakāśyate        so pi sadguruḥ kathyate  \P %%%7678.jpg
%prāṇimātreṇa saha maitrī krīyate/ kasyāpi doṣau na prākāśate/        so pi sadguruḥ kathyate/  \B
%prāṇimātreṇa saha maitrī krīyate  kasyāpi doṣo  na prākāśate         so pi sadguruḥ kathyate \L
%prāṇimātreṇa saha maitrī krīyate/ kasyāpi doṣo  na prākāśyate/  yena so pi sadguruḥ kathyate// \N1
%prāṇimātreṇa saha maitrī yate/    kasyāpi doṣaṃ na prakāśyate// yena so pi sadguruḥ kathyate// \N2
%prāṇimātreṇa saha maitrī krīyate/ kasyāpi doṣo  na prākāśyate/  yena so pi sadguruḥ kathyate/ \D
%\om \U1
%prāṇimātre   saha maitrī kriyate// kasyāpi doṣo na prakāśyate//      so pi sadguruḥ kathyate// \U2 %%%430.jpg
%-----------------------------
%, one who practices loving kindness towards living beings, one who will not expose anyone's badness, is said to be a true teacher.
%-----------------------------
\app{\lem[wit={ceteri}]{prāṇimātreṇa}
  \rdg[wit={U2}]{prāṇimātre}}
saha maitrī \app{\lem[wit={ceteri}]{krīyate}
  \rdg[wit={N2}]{yate}}/
kasyāpi
\app{\lem[wit={E,N2}]{doṣaṃ}
  \rdg[wit={P,L,N1,D,U2}]{doṣo}
  \rdg[wit={B}]{doṣau}}
na
\app{\lem[wit={ceteri}]{prākāśyate}
  \rdg[wit={B,L}]{prākāśate}
  \rdg[wit={E}]{kathayati}}/
\app{\lem[wit={B,E,L,P,U2}]{so}
  \rdg[wit={D,N1,N2}]{yena so}}
'pi sadguruḥ kathyate/
\end{prose}
\end{ekdosis}
\begin{ekdosis}
%-----------------------------
%ajñātakulaśīlānāṃ yatīnāṃ brahmacāriṇām/   upadeśaṃ na  gṛhṇīyād anyathā narakaṃ dhruvam// \E %[p.80]
%ajñātakulaśīlānāṃ yatīnāṃ brahmacāriṇām    upadeśo  na  gṛhṇīyād anyathā narakaṃ dhruvam 1 \P
%ajñānakulaśilānāṃ yatīnāṃ brahmacāriṇām    upadeśa  na  gṛhītyāsthā/ yadānyathā na narakaṃ dhruvaṃ// 1// \B
%ajñānakulaśilānāṃ yatīnāṃ brahmacāriṇām//  upadeśaṃ na  gṛṇhīyād anyathā narakaṃ dhruvaṃ// 1// \L  0040.jpg
%ajñānakulaśīlānāṃ, yatīnāṃ brahmacāriṇām/  upadeśaṃ na  gṛhnīyāt anyathā narakaṃ dhruvaṃ// \N1 %%%S.14 
%ajñātakulaśīlānāṃ yatīnāṃ brahmacāriṇām//  upadeśaṃ na  gṛhnīyāt anyathā narakaṃ dhruvaṃ//1// \N2 
%ajñānakulaśīlānāṃ yatīnāṃ brahmacāriṇām/   upadeśaṃ na  gṛhnīyāt anyathā narakaṃ dhruvaṃ/ \D
%\om \U1
%ajñātakulaśīlānāṃ yatīnāṃ brahmacāriṇām    upadeśo na gṛhṇīyāt// anyathā narakaṃ dhruvaṃ// \U2
%-----------------------------
%Man soll nicht die Lehre der Zölibatäre, der Asketen und von denen deren Abstammung und Charakter unbekannt sind akzeptieren, andernfalls [käme man] in Folge dessen für immer in die Hölle.
%----------------------
\begin{tlg}
  \tl{ajñānakulaśīlānāṃ yatīnāṃ brahmacāriṇām/}\\
  \tl{\app{\lem[wit={ceteri}]{upadeśaṃ}
      \rdg[wit={P,U2}]{upadeśo}}
    na
    \app{\lem[wit={E,P,L},alt={gṛṇhīyād}]{gṛṇhīyā\skp{d-a}}
      \rdg[wit={B}]{gṛhītyāsthā |}
      \rdg[wit={ceteri}]{gṛhnīyāt}
}\app{\lem[wit={ceteri},alt={anyathā}]{\skm{d-a}nyathā}
  \rdg[wit={B}]{yadānyathā}}
\app{\lem[wit={ceteri}]{narakaṃ}
  \rdg[wit={B}]{na narakaṃ}}
\app{\lem[wit={ceteri}]{dhruvaṃ}
  \rdg[wit={E,P}]{dhruvam}}\dd{}4\hskip-2pt\dd{}} 
\end{tlg}
\end{ekdosis}
\bigskip
\begin{ekdosis}
  \begin{prose}
%----------------------
%yasya vacasi manasi dhṛte sati  svātmanaḥ parameśvarasyaikyaṃ bhavati/ \E
%yasya vacasi manasi dhṛte sati  svātmanaḥ parameśvarasyaikyaṃ bhavati  \P
%yasya vacasi manasi dhṛte sati  svātmanaḥ parameśvarasakyaṃ   bhavati/ \B
%yasya vacasi manasi dhṛte sati  svātmanaḥ parameśvarasakyaṃ   bhavati// \L
%yasya vacasi manasi dhṛte sati/ svātmanaḥ parameśvarasyaikyaṃ bhavati/ \N1
%yasya vacasi manasi dhṛte sati/ svātmanaḥ parameśvarasyaikaṃ  bhavati/ \N2
%yasya vacasi manasi dhṛte sati/ svātmanaḥ parameśvarasyaikyaṃ bhavati/ \D
%\om \U1
%yasya cavi          dhṛte sati  svātmanaḥ parameśvarasyaikyaṃ bhavati// \U2
%-----------------------------
%Unitiy of the supreme deity and the own self arises, for one who abides calm in mind and speech.  
%-----------------------------
yasya
\app{\lem[wit={ceteri}]{vacasi}
  \rdg[wit={U2}]{cavi}}
\app{\lem[wit={ceteri}]{manasi}
  \rdg[wit={U2}]{\om}}
dhṛte
\app{\lem[wit={ceteri}]{sati}
  \rdg[wit={D,N1,N2}]{sati |}}
svātmanaḥ
\app{\lem[wit={E,P,N1,D,U2}]{parameśvarasyaikyaṃ}
  \rdg[wit={N2}]{parameśvarasyaikaṃ}
  \rdg[wit={B,L}]{parameśvarasakyaṃ}}
bhavati/ 
%-----------------------------
%etādṛśo manomadhye niścayo bhavati/ \E
%etādṛśo manomadhye niścayo bhavati \P
%etādṛśo manomadhye niścayo bhavati/ \B
%etādṛśo manomadhye niścayo bhavati// \L
%etādṛśo manomadhye niścayo bhavati \N1
%etādṛśo manomadhye niścayo bhavati \N2
%etādṛśo manomadhye niścayo bhavati/ \D
%etādṛśo manomadhye niścayo bhavati// \U2
%\om \U1
%-----------------------------
%Such is the determination in the mind 
%-----------------------------
etādṛśo manomadhye niścayo bhavati/ 
%-----------------------------
%taṃ sadguruṃ vijānīyāt vikalpa etādaśo  yathā samudramadhye mahattaraṃ kallolāḍambaram/ prapaṃce vāsanā   tādṛśī yathodakamadhye mahattaraṃgāḥ/    \E
%taṃ sadguruṃ jānīyāt   vikalpa etādṛśo  yathā samudramadhye mahattara--kallolāḍambaraṃ  prapaṃca-vāsanā  etādṛśī yathodakamadhye mahattarati        \P
%taṃ sadguruṃ jānīyāt   vikalpa etādṛśo  yathā samudramadhye mahattara--kallolāḍambara   prapaṃca-vāsanā  etādṛśī yathodakamadhye mahattarati ....   \B
%taṃ sadguruṃ jānīyāt   vikalpa etādṛśo  yathā samudramadhye mahattara--kallolāḍambara   prapaṃca-vāsanā  etādṛśī yathodakamadhye mahattarati ....   \L
%taṃ sadguruṃ jānīyāt// vikalpa etādṛśo  yathā samudramadhye mahattara--kallolāḍambaraḥ/ prapaṃca-vāsanā  etādṛśī yathodakamadhye mahattarati        \N1
%taṃ sadguruṃ jānīyāt// vikalpa etādṛśaṃ yathā samudramadhye mahattara--kallolāḍambaraḥ  prapaṃca-vāsanā  etādṛśī yathodakamadhye mahattarati        \N2 %%S.12
%taṃ sadguruṃ jānīyāt/  vikalpa etādṛśo  yathā samudramadhye mihattara--kallolāḍambaraḥ/ prapaṃca-vāsanā  etādṛśī yathodakamadhye mahattarati        \D
%taṃ sadguruṃ jānīyāt// vikalpa etādṛśo  yathā samudramadhye mahattara--kallolāḍambaraṃ  prapaca vāsanā// etādṛśī yathodakamadhye mahattarī          \U2
%\om \U1
%-----------------------------
%E: One should know this true teacher. The changing thought is like the roar of waves within the ocean. The manifold mental imprints are like the ripples in the water...
%----------------------------
taṃ sadguruṃ
\app{\lem[wit={ceteri}]{jānīyāt}
  \rdg[wit={E}]{vijānīyāt}}/
vikalpa
\app{\lem[wit={ceteri}]{etādṛśo}
  \rdg[wit={N2}]{etādṛśaṃ}}
yathā samudramadhye
\app{\lem[wit={ceteri},alt={mahattara°}]{mahattara}
  \rdg[wit={D}]{mihattara}
  \rdg[wit={E}]{mahattaraṃ}
}kallolā\app{\lem[wit={ceteri},alt={°ḍambaraḥ}]{ḍambaraḥ}
  \rdg[wit={B,L}]{°ḍambara°}
  \rdg[wit={E,P,U2}]{°ḍambaraṃ}}
\app{\lem[wit={ceteri},alt={prapañca°}]{prapañca}
  \rdg[wit={U2}]{prapaca}}vāsanā 
\app{\lem[wit={ceteri}]{etādṛśī}
  \rdg[wit={E}]{tādṛśī}}
yathodakamadhye
\app{\lem[wit={E}]{mahattaraṅgāḥ}
  \rdg[wit={U2}]{mahattarī}
  \rdg[wit={ceteri}]{mahattarati}}/ 
%----------------------------
%tādṛśasya saṃsārasāgarasya   yaḥ svavākyanāvā paraṃ pāraṃ  prāpayati/  sa sadguruḥ kathyate// \E
%tādṛśya   saṃsārā---rṇāvād   yo          nāvā paraṃ        prāpayati   sa sadguruḥ kathyate  \P
%tādṛśāt   saṃsārā---rṇavavād yau         nāvā paraṃ        prāpayati/  sa sadguruḥ kathyate// \B
%tādṛśāt   saṃsārā---rṇavād   yau         nāvā paraṃ        prāpayati   sa sadguruḥ kathyate// \L
%tādṛśāt   saṃsārāt arṇavād  yo           nāvaraṃ           prāpayati/  sa sadguruḥ kathyate// \N1 %%%S.14 Z. 3
%tādṛśāt   saṃsārāt arṇavād  yo           nāvaraṃ           prāpayati/  sa sadguruḥ kathyate// \N2
%tādṛśāt   saṃsārāt aṛṇavād   yo          nāvā paraṃ        prāpayati/     sadguruḥ kathyate/ \D
%tādṛśāt   saṃsārā---rṇavād   yo          nāvā pāraṃ pāraṃ  prāpayati// sa sadguruḥ kathyate// \U2
%\om \U1
%-----------------------------
%He who causes to navigate/convey the boat from such an ocean of Saṃsāra to the other shore is called a true teacher. 
%-----------------------------
\app{\lem[wit={ceteri}]{tādṛśāt}
  \rdg[wit={E}]{tādṛśasya}}
\app{\lem[wit={P,L,U2}]{saṃsārārṇavā\skp{d-yo}}
  \rdg[wit={B}]{saṃsārārṇavavād}
  \rdg[wit={D,N1,N2}]{saṃsārāt arṇavād}}
\app{\lem[wit={ceteri},alt={yo}]{\skm{d-yo}}
  \rdg[wit={B,L}]{yau}
  \rdg[wit={E}]{yaḥ}}
\app{\lem[wit={B,L,P,D,U2}]{nāvā}
  \rdg[wit={N1,N2}]{nāvaraṃ}
  \rdg[wit={E}]{svavākyanāvā}}
\app{\lem[wit={E}]{paraṃ pāraṃ}
  \rdg[wit={U2}]{pāraṃ pāraṃ}
  \rdg[wit={B,L,P,D}]{paraṃ}
  \rdg[wit={N1,N2}]{\om}}
prāpayati/
\app{\lem[wit={ceteri}]{sa}
  \rdg[wit={D}]{\om}}
sadguruḥ kathyate/ 
%-----------------------------
%yasya puruṣasya mano  'khaṇḍe paramapade līnaṃ bhavati/ [p.81]     \E
%yasya puruṣasya mano  'khaṃḍe       pade līnaṃ bhavati/                 \P
%yasya puruṣasya manaḥ akhaṃḍe       pade līnaṃ bhavatī/ DSCN7180.JPG   \B
%yasya puruṣasya manaḥ akhaṃḍe       pade līnaṃ bhavati/                \L
%yasya puruṣasya mano   khaṃḍe   parapada-līna bhavati/                \N1
%yasya puruṣasya mano   khaṃḍe   paramada-lita bhavati/                \N2
%yasya puruṣasya mano   khaṃḍe   parapada-līnaṃ bhavati/                \D
%yasya puruṣasya mano  'khaṇḍe   parapade līnaṃ bhavati     \U2
%\om \U1
%----------------------------
%The mind of the person is absorbed into the indivisible supreme place.  
%-----------------------------
yasya puruṣasya
\app{\lem[wit={ceteri}]{mano}
  \rdg[wit={B,L}]{manaḥ}}
\app{\lem[wit={ceteri}]{'khaṇḍe}
  \rdg[wit={B,L}]{akhaṃḍe}}
\app{\lem[wit={E}]{paramapade}
  \rdg[wit={D,N1}]{parapada°}
  \rdg[wit={N2}]{paramada°}
  \rdg[wit={U2}]{parapade}}
\app{\lem[wit={ceteri}]{līnaṃ}
  \rdg[wit={N1}]{°līna}
  \rdg[wit={N2}]{°lita}}
\app{\lem[wit={ceteri}]{bhavati}
  \rdg[wit={B}]{bhavatī}}/  
%-----------------------------
%yaḥ puruṣaḥ svakulaṃ     trividhāt   tāpān nivartya  parame  muktipade  rakṣati/ \E
%yaḥ puruṣaḥ svīyaṃ kulaṃ trividhāt   tapān nivartta--parama--muktipade  rakṣati \P %%%7579.jpg
%yaḥ puruṣaḥ svikulaṃ     trividhaṃ/  tāpān nivartya  para----muktipade  rakṣati/ \B
%yaḥ puruṣaḥ sviyaṃ kulaṃ trividhat   āpān nivartya   para----muktipade  rakṣati... \L
%yaḥ puruṣaḥ svīyaṃ kūlaṃ trividhāt   tāpān nivarttya parama--muktipade  rakṣati/ \N1
%yaḥ puruṣa  svīyaṃ kūlaṃ trividhāt   tāpān nivartya  paramamamuktipade  rakṣati/ \N2
%yaḥ puruṣaḥ svīyaṃ kūlaṃ trividhāt---tāpān mivarttya parama--muktipade  rakṣati/ \D
%yaḥ puruṣa  svīyaṃ kulaṃ trividhat---āpān nivartya   parama--muktipakṣe rakṣati// \U2
%\om \U1
%-----------------------------
%The person situated in the place of supreme liberation who turned away from the threefold misery (adhyātmika, adhibhautika, adhidaivika) protects the own kula (lineage? noble family? tribe?). 
%-----------------------------
yaḥ \app{\lem[wit={ceteri}]{puruṣaḥ}
  \rdg[wit={N2,U2}]{puruṣa}}
\app{\lem[wit={ceteri}]{svīyaṃ kūlaṃ}
  \rdg[wit={B}]{svikulaṃ}
  \rdg[wit={E}]{svakulaṃ}}
\app{\lem[wit={E,D,P,N1,N2},alt={trividhāt}]{trividhā\skp{t-tā}}
  \rdg[wit={L,U2}]{trividhat}
  \rdg[wit={B}]{trividhaṃ |}
}\app{\lem[wit={ceteri},alt={tāpān}]{\skm{t-tā}pā\skp{n-ni}}
  \rdg[wit={L,U2}]{āpān}}\skm{n-ni}vartya
\app{\lem[wit={P,D,N1}]{paramamuktipade}
  \rdg[wit={E}]{parame muktipade}
  \rdg[wit={N2}]{paramamamuktipade}
  \rdg[wit={B,L}]{paramuktipade}
  \rdg[wit={U2}]{paramamuktipakṣe}} rakṣati/
%-----------------------------
%etādṛśasya puruṣasya śravaṇād   darśanāt  samagravighnā naśyanti/    \E
%etādṛśā    puruṣasya śravaṇā    darśanāt  samagravighnā naśyaṃti     \P
%etādṛśā    puruṣasya śravaṇāt   darśanāt/ samagravighnā na naśyaṃtī/ \B
%etādṛśā    puruṣasya śravaṇāt   darśanāt  samagravighnā na naśyaṃti... \L
%etādṛśa/   puruṣaṃ   śravaṇād   darśanāt  samagravighnā naśyaṃti/ \N1
%etādṛśaṃ   puruṣaṃ   śravaṇād   darśanāt  samagravighnā naśyaṃti/ \N2
%etādṛśa----puruṣaṃ   śravaṇād   darśanāt  samagravighnā naśyaṃti/ \D
%etādṛśaṃ   puruṣaṃ   śravaṇād   darśanāt  samagraviśvaś ca vaśāṃ? bhavati \U1
%etādṛśa    puruṣasya śravaṇāt// darśanāt  samagravighna naśyaṃti//    \U2
%-----------------------------
%From hearing [or] from seeing about such a person all obstacles are destroyed. 
%-----------------------------
\note[type=philcomm, labelb=417, lem={etādṛśaṃ \ldots}]{U\textsubscript{1} resumes from here.}
\app{\lem[wit={ceteri}]{etādṛśaṃ}
  \rdg[wit={D,U2}]{etādṛśa}
  \rdg[wit={N1}]{etādṛśa |}
  \rdg[wit={B,L,P}]{etādṛśā}
  \rdg[wit={E}]{etādṛśasya}}
\app{\lem[wit={D,N1,N2,U1}]{puruṣaṃ}
  \rdg[wit={ceteri}]{puruṣasya}}
\app{\lem[wit={ceteri},alt={śravaṇād}]{śravaṇā\skp{d-da}}
  \rdg[wit={B,L}]{śravaṇāt}
  \rdg[wit={U2}]{śravaṇāt ||}
  \rdg[wit={P}]{śravaṇā}
}\app{\lem[wit={ceteri},alt={darśanāt}]{\skm{d-da}rśanā\skp{t-sa}}
  \rdg[wit={B}]{darśanāt |}
}\skm{t-sa}magra\app{\lem[wit={ceteri}]{vighnā}
\rdg[wit={U1}]{viśvaś ca vaśāṃ}}
\app{\lem[wit={ceteri}]{naśyanti}
  \rdg[wit={L}]{na naśyaṃti}
  \rdg[wit={B}]{na naśyaṃtī}
  \rdg[wit={U1}]{bhavati}}/ 
%-----------------------------
%dine dine kalyāṇaṃ  bhavati/ niṣkalaṃkā buddhir utpadyate/    \E
%dine dine kalyāṇaṃ  bhavati  niṣkalaṃkā buddhir utpadyate     \P
%dine dine kalyāṇaṃ  bhavati  niṣkalaṃkā buddhir utpadyate/    \B
%dine dine kalyāṇaṃ  bhavati  niṣkalaṃkā buddhir utpadyate...  \L
%dine dine kalyāṇaṃ  bhavati/ niṣkalaṃ   buddhir utpadyate//     \N1
%dine dine kalyāṇaṃ  bhavati/ niṣkalaṃ   buddhir utpadyate//     \N2
%dine dine kalyāṇaṃ  bhavati/ niṣkalaṃkā buddhir utpadyate/    \D
%dine      kalyāṇāṃ  bhavatīr niṣkalaṃkā buddhir utpadyate     \U1
%dine dine kalyāṇaṃ  bhavati/ niṣkalaṃko buddhir utpadyate// \U2
%-----------------------------
%Day by day prosperity arises. A flawless intellect arises. 
%-----------------------------
\app{\lem[wit={ceteri}]{dine dine}
  \rdg[wit={U1}]{dine}}
\app{\lem[wit={ceteri}]{kalyāṇaṃ}
  \rdg[wit={U1}]{kalyāṇāṃ}}
\app{\lem[wit={ceteri}]{bhavati}
  \rdg[wit={U1}]{bhavatīr}}
\app{\lem[wit={ceteri}]{niṣkalaṃkā}
  \rdg[wit={N1,N2}]{niṣkalaṃ}
  \rdg[wit={U2}]{niṣkalaṃko}}
buddhir-utpadyate/
\end{prose}
\end{ekdosis}
\begin{ekdosis}
   \ekddiv{type=ed}
    \bigskip
    \centerline{\textrm{\small{[The Secret of the Scriptures of Yoga]}}}
    \bigskip
    \begin{prose}
%-----------------------------
%idaṃ yogaśāstrasya rahasyaṃ samastaśāstraprame            yasya manaḥ   yathāṃdhakārasya    madhye dīpa----tejaḥ praviśati/        \E
%idaṃ yogaśāstrasya rahasyaṃ samagraśāstramadhye           yasya manaḥ   yathāṃdhakārasya    madhye dīpasya tejaḥ praviśati   \P
%idaṃ yogaśāstrarahasyaṃ/    samastaśāstramadhye     manaḥ yasya mana    yathāṃdhakārasya    madhye dīpasya tejaḥ praviśyati... \B
%idaṃ yogaśāstrarahasyaṃ//   samastaśāstramadhye     mano  yasya mana    yathāṃdhakārasya    madhye dīpasya tejaḥ praviśyati...  \L
%idaṃ yogaśāstrarahasyaṃ/    samagraśāstramadhye           yasya mana/   yathāṃdhakāras      madhye dīpasya tejaḥ praviśati/     \N1
%idaṃ yogaśāstrarahasya      samagraśāstramadhye                                                                               \N2
%idaṃ yogaśāstrarahasya/     samagraśāstramadhye           yasya mana/   yathāṃdhakāramadhye        dīpasya tejaḥ praviśati/     \D
%idaṃ yogaśāstreṣu rahasyaṃ  samagraśāstramadhye           yasya manaḥ   yathāṃdhakārasya    madhye dīpasya tejaḥ praviśyati...  \U1
%idaṃ yogaśāstrarahasyaṃ     samagraśāstramadhye           yasya manaḥ// yathāṃdhakārasya    madhye dīpasya tejaḥ vipraśati//        \U2
%-----------------------------
%This is the secret of the scriptures of Yoga within all scriptures of yoga. Just as he whose mind is like the light of a lamp that enters into the midst of darkness ...  
%-----------------------------
      idaṃ
      \app{\lem[wit={B,L,N1,U2}]{yogaśāstrarahasyaṃ}
        \rdg[wit={D,N2}]{yogaśāstrarahasya}
        \rdg[wit={U1}]{yogaśāstreṣu rahasyaṃ}
        \rdg[wit={E,P}]{yogaśāstrasya rahasyaṃ}}
      samagraśāstramadhye/
      \app{\lem[wit={ceteri}]{yasya}
        \rdg[wit={U2}]{\om}} 
\app{\lem[type=emendation, resp=egoscr]{mano}
  \rdg[wit={E,P,U1,U2}]{\korr manaḥ}
  \rdg[wit={ceteri}]{mana}
  \rdg[wit={N2}]{\om}}
\app{\lem[wit={ceteri}]{yathāndhakārasya}
  \rdg[wit={N1}]{yathāṃdhakāras}
  \rdg[wit={D}]{yathāṃdhakāra°}
  \rdg[wit={N2}]{\om}}
\app{\lem[wit={ceteri}]{madhye}
  \rdg[wit={N2}]{\om}} 
\app{\lem[wit={ceteri}]{dīpasya}
  \rdg[wit={E}]{dīpa°}
  \rdg[wit={N2}]{\om}}
\app{\lem[wit={ceteri}]{tejaḥ}
  \rdg[wit={N2}]{\om}}
\app{\lem[wit={D,E,P,N1}]{praviśati}
  \rdg[wit={B,L,U1}]{praviśyati}
  \rdg[wit={U2}]{vipraśati}
  \rdg[wit={N2}]{\om}}/ 
%-----------------------------
%tathā śāstramadhye mano praviśati/     \E
%tathā śāstramadhye manaḥ praviśati     \P
%tathā.... \om                          \B
%tathā.... \om                          \L
%tathā śāstramadhye tasya manaḥ praviśati/     \N1
%                   tasya manaḥ praviśati/     \N2
%tathā śāstramadhye tasya manaḥ praviśati/     \D
%tathā.... \om                          \U1
%yathā śāstramadhye mano praviśati//     \U2
%-----------------------------
%similarly his mind enters into the teaching. %%%%1+2=He whose mind enters into the scriptures as the light of a lamp enters into darkness.
%-----------------------------
\app{\lem[wit={ceteri}]{tathā}
  \rdg[wit={U2}]{yathā}
  \rdg[wit={N2}]{\om}}
\app{\lem[wit={ceteri}]{śāstramadhye}
  \rdg[wit={B,L,N2,U1}]{\om}}
\app{\lem[wit={D,N1,N2}]{tasya manaḥ}
  \rdg[wit={P}]{manaḥ}
  \rdg[wit={E,U2}]{mano}
  \rdg[wit={B,L,U1}]{\om}}
\app{\lem[wit={ceteri}]{praviśati}
  \rdg[wit={B,L,U1}]{\om}}/
%-----------------------------
%yasya rājño madhye        kalaho nāsti/ \E
%yasya rājño manomadhye    kapaṭaṃ nāsti \P
%yasya rājño madhye manasi kapaṭaṃ nāsti/ \B
%yasya rājño madhye manasi kapaṭaṃ nāsti... \L
%yasya rājño manomadhye    kapaṭaṃ nāsti/ \N1
%yasya rājño manomadhye    kapaṭaṃ nāsti  \N2
%yasya rājño manomadhye    kapaṭaṃ nāsti/ \D
%      rājño manomadhye    kapaṭaṃ nāsti \U1
%yasya rājño manomadhye    kapaṭaṃ nāsti// \U2
%-----------------------------
%Deceit does not exists in the mind of such a king. 
%-----------------------------
\app{\lem[wit={ceteri}]{yasya}
  \rdg[wit={U1}]{\om}}
rājño
\app{\lem[wit={ceteri}]{manomadhye}
  \rdg[wit={B,L}]{madhye manasi}
  \rdg[wit={E}]{madhye}}
\app{\lem[wit={ceteri}]{kapaṭaṃ}
  \rdg[wit={E}]{kalaho}} nāsti/
\end{prose}
\end{ekdosis}
\ekdpb*{}
%%%%%%%%%%%%%%%%%%%%%%%%%%%%%%%%%%%%%%%%%%
%%%%%%%%PAGEBREAK%%%%%%%PAGEBREAK%%%%%%%%%
%%%%%%%%%%%%%%%%%%%%%%%%%%%%%%%%%%%%%%%%%%
%%%%%%%%%%%%%%%%PAGEBREAK%%%%%%%%%%%%%%%%%
%%%%%%%%%%%%%%%%%%%%%%%%%%%%%%%%%%%%%%%%%%
%%%%%%%%PAGEBREAK%%%%%%%PAGEBREAK%%%%%%%%%
%%%%%%%%%%%%%%%%%%%%%%%%%%%%%%%%%%%%%%%%%%
%%%%%%%%%%%%%%%%%%%%%%%%%%%%%%%%%%%%%%%%%%
%%%%%%%%%%%%%%%%%%%%%%%%%%%%%%%%%%%%%%%%%%
%%%%%%%%%%%%%%%%%%%%%%%%%%%%%%%%%%%%%%%%%%
%%%%%%%%PAGEBREAK%%%%%%%PAGEBREAK%%%%%%%%%
%%%%%%%%%%%%%%%%%%%%%%%%%%%%%%%%%%%%%%%%%%
%%%%%%%%%%%%%%%%PAGEBREAK%%%%%%%%%%%%%%%%%
%%%%%%%%%%%%%%%%%%%%%%%%%%%%%%%%%%%%%%%%%%
%%%%%%%%PAGEBREAK%%%%%%%PAGEBREAK%%%%%%%%%
%%%%%%%%%%%%%%%%%%%%%%%%%%%%%%%%%%%%%%%%%%
%%%%%%%%%%%%%%%%%%%%%%%%%%%%%%%%%%%%%%%%%%
%%%%%%%%%%%%%%%%%%%%%%%%%%%%%%%%%%%%%%%%%%
%%%%%%%%%%%%%%%%%%%%%%%%%%%%%%%%%%%%%%%%%%
%%%%%%%%PAGEBREAK%%%%%%%PAGEBREAK%%%%%%%%%
%%%%%%%%%%%%%%%%%%%%%%%%%%%%%%%%%%%%%%%%%%
%%%%%%%%%%%%%%%%PAGEBREAK%%%%%%%%%%%%%%%%%
%%%%%%%%%%%%%%%%%%%%%%%%%%%%%%%%%%%%%%%%%%
%%%%%%%%PAGEBREAK%%%%%%%PAGEBREAK%%%%%%%%%
%%%%%%%%%%%%%%%%%%%%%%%%%%%%%%%%%%%%%%%%%%
\noindent
\begin{ekdosis}
  \begin{prose}
%-----------------------------
%yasmin dṛṣṭe deśikatrāso na bhavati/ \E
%yasmin dṛṣṭe deśikasya trāso na bhavati  \P
%yasmiṃ dṛṣṭe deśikasya trāso na bhavati/ \B
%yasmiṃ dṛṣṭe deśikasya trāso na bhavati \L
%yasmiṃ dṛṣṭe deśakasya trāso na bhavati/  \N1
%yasmin dṛṣṭe deśakasya trāso na bhavati/  \N2
%yasmiṃ dṛṣṭe deśakasya trāso na bhavati/  \D
%yasmiṃ dṛṣṭe darśakasya trāso na bhavati \U1
%yasmin dṛṣṭe deśikasya trāso na bhavati//  \U2
%-----------------------------
%Fear does not arise in the sight of the teacher. 
%-----------------------------
\app{\lem[wit={ceteri},alt={yasmin}]{yasmi\skp{n-dṛ}}
  \rdg[wit={B,L,N1,D,U1}]{yasmiṃ}
}\skm{n-dṛ}ṣṭe
\app{\lem[wit={ceteri}]{deśakasya}
  \rdg[wit={U1}]{darśakasya}
  \rdg[wit={E}]{deśika°}}
trāso na bhavati/
%-----------------------------
%tasya manaḥ śuddhaṃ bhavati/  \E
%tasya manaḥ śuddhaṃ bhavati   \P
%tasya manaḥ śuddhaṃ bhavati/  \B
%tasya manaḥ śuddhaṃ bhavati   \L
%tasya manaḥ śuddhaṃ bhavati/  \N1
%tasya manaḥ śuddhaṃ bhavati/  \N2
%tasya manaḥ śuddhaṃ bhavati/  \D
%yasya manaḥ śuddhaṃ bhavati   \U1
%tasya manaḥ śuddhaṃ bhavati//  \U2
%-----------------------------
%His mind becomes pure. 
%-----------------------------
tasya manaḥ śuddhaṃ bhavati/
%-----------------------------
%yasya pṛthvyāṃ  vītir   bhavati/ \E
%yasya pṛthivyāṃ kīrttir bhavati \P
%yasya pṛthvyāṃ  kīrtir  bhavati/ \B
%yasya pṛthivyāṃ kīrtir  bhavati \L
%yasya pṛthivī   kīrttir bhavati/ \N1
%yasya pṛthivī   kīrttir bhavati/ \N2
%yasya pṛthivī   kīrttir bhavati/ \D
%      pṛithīvī  kīrti   bhavati \U1
%yasya pṛthvyāṃ  kītīr   bhavati// \U2
%-----------------------------
%Of whom fame arises on earth, 
%-----------------------------
\app{\lem[wit={ceteri}]{yasya}
  \rdg[wit={U1}]{\om}}
\app{\lem[wit={P,L}]{pṛthivyāṃ}
  \rdg[wit={B,E,U2}]{pṛthvyāṃ}
  \rdg[wit={D,N1,N2}]{pṛthivī}
  \rdg[wit={U1}]{pṛithīvī}}
\app{\lem[wit={ceteri}]{kīrtir}
  \rdg[wit={E}]{vītir}
  \rdg[wit={U1}]{kīrti}
  \rdg[wit={U2}]{kītīr}} bhavati/
%-----------------------------
%yasya manomadhye satpuruṣasya vaco viśvāso bhavati/           yo rājā sadānaṃdarūpo bhavati// \E
%yasya manomadhye satpuruṣavacanaviśvāso    bhavati            yo rājā sadānaṃdapūrṇo bhavati   \P
%yasya manomadhye satpuruṣavacanaviśvāso    bhavati            yo rājā sadānaṃdapūrṇo bhavati/  \B
%yasya manomadhye satpuruṣavacanaviśvāso    bhavati            yo rājā sānaṃdapūrṇo bhavati   \L
%yasya manomadhye satpuruṣavacanaviśvāso    bhavati/           yo rājā sadānaṃdapūrṇo bhavati   \N1
%yasya manomadhye satpuruṣavacanaḥ viśvāso  bhavati/           yo rājā sadānaṃdapūrṇo bhavati   \N2
%yasya manomadhye satpuruṣavacanaviśvāso    bhavati/           yo rājā sadānaṃdapūrṇo bhavati/   \D
%yasya manomadhye satpuruṣasya vacanaviśvabhyāso    bhavati    yo rājā sadānaṃdapūrṇo bhavati   \U1
%yasya manomadhye satpuruṣavacanaviśvāso           bhavati//   yo rājā sadānaṃdapūrṇo bhavati// \U2
%-----------------------------
%of whom the confidence of speech of a wise man arises within the mind, who is a prince full of permanent bliss,  
%-----------------------------
yasya manomadhye
\app{\lem[wit={ceteri}]{satpuruṣavacanaviśvāso}
  \rdg[wit={N2}]{satpuruṣavacanaḥ viśvāso}
  \rdg[wit={E}]{satpuruṣasya vaco viśvāso}
  \rdg[wit={U1}]{vacanaviśvabhyāso}} bhavati/
yo rājā
\app{\lem[wit={ceteri}]{sadānandapūrṇo}
  \rdg[wit={E}]{sadānaṃdarūpo}
  \rdg[wit={L}]{sānaṃdapūrṇo}} bhavati/ 
%-----------------------------
%yasya pārśve pratyakṣam aneka--manohārivastūni tiṣṭhaṃti/ \E
%yasya pārśve pratyakṣam anekaṃ manohārivastu bhavati  \P
%yasya pārśve pratyakṣam aneka--manohārivastu bhavati/ \B
%yasya pārśve pratyakṣam aneka--manohārivastu bhavati \L
%yasya pārśve pratyakṣam anekaṃ manohārivastu bhavati/ \N1
%yasya pārśve pratyakṣam anekaṃ manohārivastu bhavati// \N2
%yasya pārśve pratyakṣam anekaṃ manohārivastu bhavati/ \D
%yasya pārśve pratyakṣam anekaṃ manohārivastu bhavati \U1
%yasya pārśve pratyakṣam anekaṃ manohārivastu bhavati// \U2 %%%431
%-----------------------------
%E: Next to him in front of his eyes manifold beautiful things arise.
%of whom in front of the eyes manifold beautiful things arise, 
%-----------------------------
yasya pārśve pratyakṣa\skp{m-a}\app{\lem[wit={ceteri},alt={anekaṃ}]{\skm{m-a}nekaṃ}
  \rdg[wit={B,L,E}]{aneka°}}
\app{\lem[wit={E}]{manohārivastūni}
  \rdg[wit={ceteri}]{manohārivastu}}
\app{\lem[type=emendation, resp=egoscr]{bhavanti}
  \rdg[wit={E}]{\korr tiṣṭhaṃti}
  \rdg[wit={ceteri}]{bhavati}}/
%-----------------------------
%etādṛśasya rājña idaṃ yogarahasyaṃ   kathanīyam/ \E
%etādṛśasya rājño ye   yogarahasyaṃ   kathanīyaṃ \P
%etādṛśasya rājño ye   thogarahasyaṃ  kathyaniyaṃ/ \B
%etādṛśasya rājño yad  yogarahasyaṃ   kathyanīyaṃ// \L
%etādṛśasya rājño 'gre yogarahasyaṃ   karttavyaṃ// \N1
%etādṛśasya rājño 'gre yogarahasyaṃ   karttavyaṃ// \N2
%etādṛśasya rājño gre  yogarahasyaṃ   karttavya/ \D
%etādṛśasya rājño gre  yogarahasyaṃ   karttavyaṃ \U1
%etādṛśasya rājño ye   yogarahasyaṃ   kathyate// \U2
%-----------------------------
%This secret of Yoga of such a prince is the foremost secret of yoga that has to be told. 
%-----------------------------
etādṛśasya
\app{\lem[wit={ceteri}]{rājño}
  \rdg[wit={E}]{rājña}}
\app{\lem[wit={D,N1,N2,U1}]{'gre}
  \rdg[wit={B,P,U2}]{ye}
  \rdg[wit={L}]{yad}
  \rdg[wit={E}]{idaṃ}}
\app{\lem[wit={ceteri}]{yogarahasyaṃ}
  \rdg[wit={B}]{thogarahasyaṃ}}
\app{\lem[wit={N1,N2,U1}]{karttavyaṃ}
  \rdg[wit={D}]{karttavya}
  \rdg[wit={E,P}]{kathanīyam}
  \rdg[wit={B,L}]{kathyaniyaṃ}
  \rdg[wit={U2}]{kathyate}}/
\end{prose}
\end{ekdosis}
\begin{ekdosis}
  \begin{tlg}
  \note[type=source, labelb=418, lem={yogarahasyaṃ}]{Ysv\textsuperscript{PT}: idaṃ yogarahasyañca na vācyaṃ mūrkhasannidhau || yogadeśas tu tatraiva ||}
%-----------------------------
%na snehān na bhayān na lobhān na mohān na dhanād   balān    na maitrībhāvān  nau dāryān na sauṃdaryān na sevanāt/ \E
%na snehān no bhayāl lno?                                             bhāvān  no  dānān  na sau daryān na sevanāt  \P %%%7680.jpg
%ni śnehān nā bhayāl lobhān    na mohān na dhanād   balāt/   na maitrībhāvān  no  dānāt  na sauṃdaryān na sevanāt/ \B
%ni śnehān nā bhayāl lobhān    na mohān na nadhanād balāta// na maitrībhāvān  no  dānāt  na sauṃdayan  ni sevanāt// \L
%na śnehān a  bhayāl obhān     na mohān na dhanād   balāt/   na maitrībhāvān  na  dāsān  na sauṃdaryān na sevanāt/ \N1
%na śnehān a  bhayāl obhān     na mohān na dhanād   balāt//  na maitrībhāva   nā  dānān  na saudaryān  na sevanāt// 1 \N2
%na śnehān a  bhayāl lobhān    na mohān na dhanād   balāt/   na maitrī                                                \D
%na śnehān na bhayān lobhān    na mohān na dhanāt   balāt    na maitrībhāvān  na  dāsān  na sauṃdaryān na sevatā \U1
%na snehān na bhayāl lon       na mohān na dhanād   balāt//  na maitrībhāvān  no  dānān  na sauṃdaryān na sevanāt// \U2
%-----------------------------
%Not because of love, not because of fear, not because of greed, not because of gift, not because of friendship, not because of hostility, not because of nobility, not because of service, 
%-----------------------------
  \tl{\app{\lem[wit={ceteri}]{na}
      \rdg[wit={B,L}]{ni}}
    \app{\lem[wit={E,P,U2},alt={snehān}]{snehā\skp{n-na}}
      \rdg[wit={ceteri}]{śnehān}
    }\app{\lem[wit={E,P,U2},alt={na}]{\skm{n-na}}
      \rdg[wit={B,L}]{nā}
      \rdg[wit={D,N1,N2}]{a}
    }\app{\lem[wit={ceteri},alt={bhayāl}]{bhayā\skp{l-lo}}
      \rdg[wit={E,U1}]{bhayān}
    }\app{\lem[wit={B,L,D,U1},alt={lobhān}]{\skm{l-lo}bhā\skp{n-na}}
      \rdg[wit={N1,N2}]{obhān}
      \rdg[wit={P}]{lno}
      \rdg[wit={U2}]{lon}
    }\app{\lem[wit={ceteri},alt={na}]{\skm{n-na}}
      \rdg[wit={P}]{\om}
    }\app{\lem[wit={ceteri},alt={mohān}]{mohā\skp{n-na}}
      \rdg[wit={P}]{\om}
    }\app{\lem[wit={ceteri},alt={na}]{\skm{n-na}}
      \rdg[wit={P}]{\om}
    }\app{\lem[wit={ceteri}]{dhānā\skp{d-ba}}
      \rdg[wit={L}]{na dhanād}
      \rdg[wit={P}]{\om}
    }\app{\lem[wit={ceteri},alt={balāt}]{\skm{d-ba}lāt}
      \rdg[wit={B}]{balāta}
      \rdg[wit={P}]{\om}}/}\\
  \tl{\app{\lem[wit={ceteri}]{na}
      \rdg[wit={P}]{\om}}
    \app{\lem[wit={ceteri},alt={maitrībhāvān}]{maitrībhāvā\skp{n-na}}
      \rdg[wit={N2}]{maitrībhāva}
      \rdg[wit={D}]{maitrī}
      \rdg[wit={P}]{bhāvān}
    }\note[type=philcomm, labelb=418, lem={maitrī \ldots}]{A lenghty omission starts in D after the word \textit{maitrī}. The single omissions will not be recorded in the critical apparatus. The reader will be informed once the evidence of D resumes.}\app{\lem[wit={N1,U1},alt={na}]{\skm{n-na}}
      \rdg[wit={B,L,P,U2}]{no}
      \rdg[wit={E}]{nau}
      \rdg[wit={N2}]{nā}
      \rdg[wit={D}]{\om}}
    \app{\lem[wit={N1,U1},alt={dāsān}]{dāsā\skp{n-na}}
      \rdg[wit={P}]{dānān}
      \rdg[wit={E}]{dāryān}
      \rdg[wit={B,L}]{dānāt}
      \rdg[wit={N2,U2}]{dānān}
      \rdg[wit={D}]{\om}
    }\app{\lem[wit={ceteri},alt={na}]{\skm{n-na}}
      \rdg[wit={D}]{\om}
    }\app{\lem[wit={ceteri},alt={sauṃdaryān}]{sauṃdaryā\skp{n-na}}
      \rdg[wit={P,N2}]{saudaryān}
      \rdg[wit={L}]{sauṃdayan}
      \rdg[wit={D}]{\om}
    }\app{\lem[wit={ceteri},alt={na}]{\skm{n-na}}
      \rdg[wit={L}]{ni}
      \rdg[wit={D}]{\om}}
    \app{\lem[wit={ceteri}]{sevanāt}
      \rdg[wit={U1}]{sevatā}}\dd{}1\hskip-2pt\dd{}}
\end{tlg}
\end{ekdosis}
\bigskip
\begin{ekdosis}
  \begin{prose}
%-----------------------------
%sāmānyāgre     yogo na kathanīyaḥ/ \E
%sāmānyād agre  yogo na kathanīyaḥ \P
%sāmānyāgre     yogo na kathaniyaṃ/ \B
%sāmānyāgre     yogo na kathanīyaṃ// \L
%sāmānyād agre  yogo na kathanīyaḥ/ \N1
%sāmānyād agre  yogo na kanīyaḥ/ \N2
%\om \D
%sāmānyāgre     yogo na kathanīyaḥ \U1
%sāmānyād agre  yogo na kathanīyaḥ \U2
%-----------------------------
%shall yoga be taught in front of everyone. 
%-----------------------------
\app{\lem[wit={P,N1,N2,U2}]{sāmānyād\skp{-}agre}
  \rdg[wit={B,E,L,U1}]{sāmānyāgre}}
yogo na
\app{\lem[wit={E,P,N1,U1,U2}]{kathanīyaḥ}
  \rdg[wit={B}]{kathaniyaṃ}
  \rdg[wit={L}]{kathanīyaṃ}
  \rdg[wit={N2}]{kanīyaḥ}}/ 
%-----------------------------
%yaḥ paraniṃdā  rato bhavati/ \E
%yaḥ paraniṃdā  rato bhavati \P
%yaḥ paraniṃdāṃ      karoti/ \B
%yaḥ paraniṃdāṃ      karoti \L
%yaḥ paraniṃdā  rato bhavati \N1
%yaḥ paraniṃdā  rato bhavati \N2
%    paraniṃdāṃ rato bhavati \U1
%yaḥ paraniṃdā  rato bhavati// \U2
%\om \D
%-----------------------------
%He, who loves it to blame others, 
%-----------------------------
\app{\lem[wit={ceteri}]{yaḥ}
  \rdg[wit={U1}]{\om}}
\app{\lem[wit={ceteri}]{paranindā}
  \rdg[wit={B,L,U1}]{paraniṃdāṃ}}
\app{\lem[wit={ceteri}]{rato}
  \rdg[wit={B,L}]{\om}}
\app{\lem[wit={ceteri}]{bhavati}
  \rdg[wit={B,L}]{karoti}}/  
%-----------------------------
%durācāro bhavati/ \E
%durācāro bhavati  \P
% \om              \B
% \om              \L
%dūrācāro bhavati/ \N1
%dūrācāro bhavati/ \N2
%\om \D
%dūrācāro bhavati  \U1
%dūrācāro bhavati//  \U2
%-----------------------------
%who is behaving badly, 
%-----------------------------
dūrācāro bhavati/
\note[type=philcomm, labelb=419, lem={dūrācāro bhavati}]{The sentence is omitted in B and L.}
%-----------------------------
%durmaitryānyasya            vastu na dadāti/ \E
%bhrātur mitrasya yogyaṃ     vastu na dadāti \P
%durmaitryānyasya            vastu na dadāti/ \B
%bhrātu mitrasya ca yogyaṃ ca vastu na dadāti/ \N1
%bhrātu mitrasya ca yogyaṃ    vastu na dadāti/ \N2
%bhrātṛr mitraṃ ca yogyaṃ    vastu na dadāti \U1
%bhrātur mitrasya yogyaṃ     vastu na dadāti// \U2
%\om                               \L
%\om \D
%-----------------------------
%who does not gives [single] thing, which benefits friend and brother,  
%-----------------------------
\app{\lem[wit={P,U2},alt={bhrātur}]{bhrātu\skp{r-mi}}
  \rdg[wit={N1,N2}]{bhrātu°}
  \rdg[wit={U1}]{bhrātṛr}
  \rdg[wit={B,E}]{dur°}
}\app{\lem[wit={ceteri}]{mitrasya}
  \rdg[wit={U1}]{mitraṃ}
  \rdg[wit={B,E}]{maitryānyasya}}
\app{\lem[wit={N2,U1}]{ca yogyaṃ}
  \rdg[wit={N1}]{ca yogyaṃ ca}
  \rdg[wit={P,U2}]{yogyaṃ}
  \rdg[wit={B,E}]{\om}}
vastu na dadāti/ 
\note[type=philcomm, labelb=420, lem={bhrātur \ldots dadāti}]{The sentence is omitted in L.}
%-----------------------------
%[p.83]
%ya asatyaṃ vadati/  yo yoga-----------------nindāṃ karoti/  \E
%yo 'satyaṃ vadati   yo yogināṃ   manomadhye niṃdāṃ karoti   \P
%so 'satyaṃ vadati/     yoginā    manomadhye niṃdāṃ karoti/  \N1
%so 'satyaṃ vadati/     yoginā    manomadhye niṃdāṃ karoti/  \N2
%so satyaṃ  vadati      yogināṃ   manomadhye ni-----karoti   \U1
%yo 'satyaṃ vadati//    yogināṃ   manomadhye niṃdāṃ karoti// \U2
%\om                                                         \B
%\om                                                         \L
%\om                                                         \D
%-----------------------------
%who does not speak the truth and despises yoga in spirit, 
%-----------------------------
\note[type=source, labelb=421, lem={nindāṃ}]{Ysv\textsuperscript{PT}: stutir nindā na karttavyā sādhunā satyavādinā || yogānadhikāriṇam āha tatraiva ||}
\app{\lem[wit={P,U2}]{yo}
  \rdg[wit={N1,N2,U1}]{so}
  \rdg[wit={E}]{ya}}
\app{\lem[wit={ceteri}]{satyaṃ}
  \rdg[wit={E}]{asatyaṃ}}
vadati/
\app{\lem[wit={E,P}]{yo}
  \rdg[wit={ceteri}]{\om}}
\app{\lem[wit={ceteri}]{yogināṃ}
  \rdg[wit={N1,N2}]{yoginā}
  \rdg[wit={E}]{yoga°}}
\app{\lem[wit={ceteri}]{manomadhye}
  \rdg[wit={E}]{\om}}
\app{\lem[wit={ceteri}]{nindāṃ}
  \rdg[wit={U1}]{ni°}}
karoti/ 
\note[type=philcomm, labelb=422, lem={yo śatyaṃ \ldots karoti}]{The sentence is omitted in B and L.}
%-----------------------------
%yasya manomadhye dayā na bhavati/ \E
%yasya manomadhye dayā na bhavati  \P
%yasya manomadhye dayā na bhavati/ \B
%yasya manomadhye dayā na bhavati/ \L
%yasya manomadhye dayā na bhavati  \N1
%yasya manomadhye dayā na bhavati  \N2
%\om \D
%yasya manomadhye dayā na bhavati  \U1
%yasya manomadhye dayā na bhavati// \U2
%-----------------------------
%in whose mind compassion does not arise.  
%-----------------------------
\note[type=source, labelb=423, lem={dayā na bhavati}]{Ysv\textsuperscript{PT}: manomadhye dayā nāsti sadā yaḥ kalahapriyaḥ |}
yasya manomadhye dayā na bhavati/ 
%-----------------------------
%yaḥ    kalaha--priyo    bhavati/ \E
%yasya  kalaha--priyo    bhavati/ \P
%yasya  kalahaṃ priyo na bhavati/ \B
%yasya  kalahaṃ priyo na bhavati// \L
%yaḥ    kalaha--priyo    bhavati/ \N1 [em. to  yasya kalahapriyo bhavati ||
% \om N2
%\om \D
%yaḥ    kalaha--priyo    bhavati \U1
%yasya  kalahaḥ priyo    bhavati// \U2
%-----------------------------
%who is devoted to dispute,   
%-----------------------------
\note[type=philcomm, labelb=424, lem={yasya śatyaṃ \ldots karoti}]{The sentence is omitted in B and L.}
\app{\lem[wit={E,N1,U1}]{yaḥ}
  \rdg[wit={B,L,P,U2}]{yasya}}
\app{\lem[wit={E,P,N1,U1}]{kalahapriyo}
  \rdg[wit={B,L}]{kalahaṃ priyo}
  \rdg[wit={U2}]{kalahaḥ priyo}}
\app{\lem[wit={ceteri}]{bhavati}
  \rdg[wit={B,L}]{na bhavati}}/ 
%-----------------------------
%svakāryakaraṇe  sāvadhāno bhavati/ \E
%svakāryakaraṇe  sāvadhāno bhavati  \P
%svakāryākaraṇeṃ sāvadhāno bhavati/ \B
%svakāryākaraṇe  sāvadhāno bhavati// \L
%svakāryyākaraṇe sāvadhāno bhavati/ \N1
%svakāryyākaraṇā sāvadhāno bhavati/ \N2
%\om \D
%svakāryakaraṇe  sāvadhāno bhavati \U1
%svakāryakaraṇe  sāvadhāno bhavati// \U2
%-----------------------------
%, attention arises for him only with regard to his selfish intentions, 
%-----------------------------
\app{\lem[wit={E,P,U1,U2}]{svakāryakaraṇe}
  \rdg[wit={L,N1}]{svakāryākaraṇe}
  \rdg[wit={B}]{svakāryākaraṇeṃ}
  \rdg[wit={N2}]{svakāryyākaraṇā}}
sāvadhāno bhavati/
%-----------------------------
%guroḥ  kāryakaraṇe na dattacitto bhavati/ \E
%guroḥ  kāryakaraṇe 'nādṛto??? bhavati        \P
%guro   kārye karaṇe anādarano    bhavati/ \B [em. to anādaro = disrespect] 
%guroḥ  kāryakaraṇe  anādare no    bhavati/ \Ĺ
%guroḥ  kāryakaraṇe  ādaro na     bhavati/ \N1  [em. zu kāryakāraṇa]
%guro   kāryakaraṇe  ādaro na     bhavati/ \N2
%\om \D
%guroḥ  kāryakaraṇe  ādaro na    bhavati  \U1
%guro   kāryakaraṇe  nādṛto      bhavati//  \U2
%-----------------------------------
%disrespect arises towards the intentions of the teacher.
%-----------------------------
\note[type=source, labelb=425, lem={kāryakaraṇe}]{Ysv\textsuperscript{PT}: svakāryalobhane śīlo gurukāryaparāṅmukhaḥ | etasmai ca na dātavyaṃ vaktavyaṃ tasya sannidhau |}
\app{\lem[wit={ceteri}]{guroḥ}
  \rdg[wit={B,N2,U2}]{guro}}
\app{\lem[type=emendation, resp=egoscr]{kāryakāraṇe}
  \rdg[wit={ceteri}]{\korr kāryakaraṇe}
  \rdg[wit={B}]{kārye karaṇe}}
\app{\lem[wit={P,U2}]{'nādṛto}
  \rdg[wit={N1,N2,U1}]{ādaro na}
  \rdg[wit={B}]{anādarano}
  \rdg[wit={L}]{anādare no}
  \rdg[wit={E}]{na dattacitto}}
bhavati/
%-----------------------------
%etādṛśasyāgre   na yogaḥ  kriyate   na paṭhyate// \E
%etādṛśasyāgre   na yogaḥ  kriyate   na paṭhyate    \P
%etādṛśasyāgre   na yogaḥ  kriyate/  na paṭhayate/ \B
%etādṛśasyāgre   na yogaḥ  kriyate   na paṭhayate... \L
%etādṛśasyāgre   na        kriyate/  na padyaṃte//  \N1
%etādṛśasyāgre   na        kriyate/  na padyaṃte//  \N2
%\om \D
%etādṛśasya agre na        kriyate   na paṭhyate \U1
%etādṛśasyāgre   na yogaḥ  kriyate// na paṭhyate// \U2
%-------------------------
%In front of such a one yoga is neither done nor taught. 
%-----------------------------
\app{\lem[wit={ceteri}]{etādṛśasyāgre}
  \rdg[wit={U1}]{etādṛśasya agre}}
na \app{\lem[wit={ceteri}]{yogaḥ}
  \rdg[wit={N1,N2,U1}]{\om}}
kriyate na \app{\lem[wit={E,P,U1,U2}]{paṭhyate}
  \rdg[wit={N1,N2}]{padyaṃte}
  \rdg[wit={B,L}]{paṭhayate}}/
  \end{prose}
\end{ekdosis}
\ekdpb*{}
%%%%%%%%%%%%%%%%%%%%%%%%%%%%%%%%%%%%%%%%%%
%%%%%%%%PAGEBREAK%%%%%%%PAGEBREAK%%%%%%%%%
%%%%%%%%%%%%%%%%%%%%%%%%%%%%%%%%%%%%%%%%%%
%%%%%%%%%%%%%%%%PAGEBREAK%%%%%%%%%%%%%%%%%
%%%%%%%%%%%%%%%%%%%%%%%%%%%%%%%%%%%%%%%%%%
%%%%%%%%PAGEBREAK%%%%%%%PAGEBREAK%%%%%%%%%
%%%%%%%%%%%%%%%%%%%%%%%%%%%%%%%%%%%%%%%%%%
%%%%%%%%%%%%%%%%%%%%%%%%%%%%%%%%%%%%%%%%%%
%%%%%%%%%%%%%%%%%%%%%%%%%%%%%%%%%%%%%%%%%%
%%%%%%%%%%%%%%%%%%%%%%%%%%%%%%%%%%%%%%%%%%
%%%%%%%%PAGEBREAK%%%%%%%PAGEBREAK%%%%%%%%%
%%%%%%%%%%%%%%%%%%%%%%%%%%%%%%%%%%%%%%%%%%
%%%%%%%%%%%%%%%%PAGEBREAK%%%%%%%%%%%%%%%%%
%%%%%%%%%%%%%%%%%%%%%%%%%%%%%%%%%%%%%%%%%%
%%%%%%%%PAGEBREAK%%%%%%%PAGEBREAK%%%%%%%%%
%%%%%%%%%%%%%%%%%%%%%%%%%%%%%%%%%%%%%%%%%%
%%%%%%%%%%%%%%%%%%%%%%%%%%%%%%%%%%%%%%%%%%
%%%%%%%%%%%%%%%%%%%%%%%%%%%%%%%%%%%%%%%%%%
%%%%%%%%%%%%%%%%%%%%%%%%%%%%%%%%%%%%%%%%%%
%%%%%%%%PAGEBREAK%%%%%%%PAGEBREAK%%%%%%%%%
%%%%%%%%%%%%%%%%%%%%%%%%%%%%%%%%%%%%%%%%%%
%%%%%%%%%%%%%%%%PAGEBREAK%%%%%%%%%%%%%%%%%
%%%%%%%%%%%%%%%%%%%%%%%%%%%%%%%%%%%%%%%%%%
%%%%%%%%PAGEBREAK%%%%%%%PAGEBREAK%%%%%%%%%
%%%%%%%%%%%%%%%%%%%%%%%%%%%%%%%%%%%%%%%%%%

\begin{ekdosis}
  \begin{prose}
    \noindent
%-----------------------------
%śrṛṇvan prītādikān  śabdān  paśyan rūpaṃ manoharam/ \E
%śrṛṇvan gītādikān   śabdān  paśyan rūpaṃ manoharaṃ   \P
%śrṛṇvan gītādikān   śabdān  paśyan rūpaṃ manoharaṃ/ \B
%śṛṇvan  gītādikān   śabdān/ paśyan rūpaṃ manoharaṃ/ \N1
%śuśvana gītādikāna  śabdāt/ paśyan rūpaṃ manoharaṃ// \N2
%śṛṇvan  gītādikān   śabdān  paśyan rūpaṃ manoharaṃ// \L
%śṛṇvan  gītādikān   śabdān  paśyat rūpaṃ manoharaṃ \U1
%śrṛṇvan gītādikān// śabdān  paśyan rūpaṃ manoharam// \U2
%\om \D
%-----------------------------
%While hearing the sound of music etc., while seeing heart pounding forms,  
%-----------------------------
\app{\lem[wit={N1,L,U1},alt={śṛṇvan}]{śṛṇva\skp{n-gī}}
  \rdg[wit={N2}]{śuśvana}
  \rdg[wit={ceteri}]{śrṛṇvan}
}\app{\lem[wit={ceteri},alt={gītādikān}]{\skm{n-gī}tādikān}
  \rdg[wit={E}]{prītādikān}}\dd
 \app{\lem[wit={ceteri},alt={śabdān}]{śabdā\skp{n-pa}}
  \rdg[wit={N2}]{śabdāt |}
}\app{\lem[wit={ceteri},alt={paśyan}]{\skm{n-pa}śya\skp{n-rū}}
    \rdg[wit={U1}]{paśyat}
  }\skm{n-rū}paṃ manoharaṃ\dd{}
%-----------------------------
%jāgrat                  sphuran  spṛśan sparśa---mṛdupriyam     svādān manoramān          bhrāmyan    deśān/ manoramān \E
%jighran gaṃdhāṃś ca     surabhin spṛśan sparśaṃ  mṛḍupriyaṃ     svādān manoramān  khādan  bhrāmyan    deśān manoramān \P
%jighran agachan       sphurabhi  spṛśaṃ          mṛdupriyaṃ     svādān manorathān khādavan bhrāman    deśān manoramān \B
%jighran agachan        surabhin  spṛśaṃ          mṛdupriyaṃ     svādān manorathān khādavan bhrāman    deśān manoramān \L
%jighran gaṃdhāṃ       śusurabhīn spṛśyanasya     mṛdupriyaṃ/    svādān manomān svādan      bhrāmye na deśān manoramān \N1
%jighraṃ gaṃdhāṃ       śusurabhīn spṛśyanasyaṃ śarmṛdupriyaṃ     svādān manomān             bhrāmya na deśān manoramān \N2
%\om \D
%jighraṃ nāṃdhaś ca      surabhīn        sparśaṃ  mṛdupriyaṃ     svādān manoramān khādaṃta--bhrāmyan tveṣāṃn manoramān/ \U1 %%%302.jpg
%jighran spṛśan  gaṃdhan surabhīn spṛśan sparśaṃ  mṛḍu// priyaṃ  svādān manoramān           bhrāmyan    deśān manoramān//  \U2
%-----------------------------
%while smelling fragnent scent, while touching lovely and soft touch, while eating food that pleases the mind, while travelling to beautiful places,    
%-----------------------------
\app{\lem[wit={ceteri},alt={jighran}]{jighra\skp{n-ga}}
  \rdg[wit={E}]{jāgrat}
  \rdg[wit={U1}]{jighraṃ}
}\app{\lem[wit={N1,N2},alt={gandhān}]{gandhā\skp{n-su}}
  \rdg[wit={P}]{gaṃdhāṃś ca}
  \rdg[wit={U1}]{nāṃdhaś ca}
  \rdg[wit={B,P}]{agachan}
  \rdg[wit={U2}]{spṛśan gaṃdhan}
  \rdg[wit={E}]{\om}
}\app{\lem[wit={U1,U2},alt={surabhīn}]{\skm{n-su}rabhī\skp{n-spṛ}}
  \rdg[wit={E}]{sphuran}
  \rdg[wit={P,L}]{surabhin}
  \rdg[wit={B}]{sphurabhi}
  \rdg[wit={N1,N2}]{śusurabhīn}
}\app{\lem[wit={B,E,L,P,U2},alt={spṛśan}]{spṛśa\skp{n-spa}}
  \rdg[wit={N1}]{spṛśyanasya}
  \rdg[wit={N2}]{spṛśyanasyaṃ}
  \rdg[wit={U1}]{\om}
}\app{\lem[wit={P,U1,U2},alt={sparśaṃ}]{\skp{n-spa}rśaṃ}
  \rdg[wit={E}]{sparśa°}
  \rdg[wit={ceteri}]{\om}
}\app{\lem[wit={ceteri}]{mṛḍupriyaṃ}
  \rdg[wit={N2}]{śarmṛdupriyaṃ}
  \rdg[wit={U2}]{mṛḍu || priyaṃ}}\dd{}
svādā\skp{n-ma}\app{\lem[wit={ceteri},alt={manoramān}]{\skm{n-ma}noramā\skp{n-khā}}
  \rdg[wit={B,L}]{manorathān}
  \rdg[wit={N1,N2}]{manomān}
}\app{\lem[wit={ceteri},alt={khādan}]{\skm{n-khā}da\skp{n-bhrā}}
  \rdg[wit={B,L}]{khādavan}
  \rdg[wit={U1}]{khādaṃta°}
  \rdg[wit={N1}]{svādan}
  \rdg[wit={E,N2}]{\om}
}\app{\lem[wit={ceteri},alt={bhrāmyan}]{\skm{n-bhrā}mya\skp{n-de}}
  \rdg[wit={B,L}]{bhrāman}
  \rdg[wit={N1}]{bhrāmyena}
  \rdg[wit={N2}]{bhrāmya na}
}\app{\lem[wit={ceteri},alt={deśān}]{\skm{n-de}śā\skp{n-ma}}
  \rdg[wit={U1}]{tveṣāṃn}
}\skm{n-ma}noramān\dd{}
\end{prose}
\end{ekdosis}
\begin{ekdosis}
  \begin{tlg}
%-----------------------------
%bhāṣamāṇaḥ             ramamāṇaḥ svalīlayā/   bhāvābhāvavinirmukto  sarvagrahavivarjitaḥ// 1// %[p.84] \E
%bhāṣamāṇaḥ  sumadhuraṃ ramamāṇaḥ svalīlayā    bhāvābhāvavinirmuktaḥ sarvagrahavivarjjitaḥ   \P  %%%7681.jpg 
%bhakṣamāṇa  samaghura--ramāṇa    svalilayā    bhāvāvinir muktaḥ/    sarvagrāhavivarjitaḥ/            \B
%bhakṣamāṇaḥ samadhura--ramamāṇaṃ svalīlayā    bhāvāvinir muktaḥ     sarvagrāhavivarjitaḥ...      \L  [bhakṣamāṇaḥ= PPA] [rasamānaḥ =PPA] 
%bhāṣamāṇasya madhuraṃ  rasamānaḥ svalīlayā // bhāvābhāvavinirmuktaḥ sarvāgrahavivarjitaḥ// N1
%bhāṣamāṇasya madhuraṃ  rasamānaḥ svalīlayā // bhāvābhāvavinirmuktaḥ sarvāgrahavivarjitaḥ// N2
%\om \D
%bhāṣamāṇasya madhuraṃ  ramamāṇaḥ svalīlayā//  bhāvābhāvavinirmuktaḥ sarvagrāhavivarjitaḥ  \U1
%bhāṣamāṇaḥ sumadhuraṃ  ramamāṇaḥ svalīlayā//  bhāvābhāvavinirmuktaḥ sarvāgrahavivarjjitaḥ// \U2
%-----------------------------
%Während er überaus Süßes verspeist und sich im eigenen Spiel vergnügt, ist er einer, der befreit von sein und nicht sein ist und aller Anhaftung entledigt ist.  
%-----------------------------
\note[type=source, labelb=426, lem={bhāvābhāvavinirmuktaḥ}]{Ysv\textsuperscript{PT}: nañubhayatra sambadhyate na vaktavyamityarthaḥ | yogādhikāriṇo'pi tatraiva || bhāvābhāvavinirmuktaḥ sarvagrahavivarjitaḥ |}
\tl{
  \app{\lem[wit={L}]{bhakṣamāṇaḥ}
    \rdg[wit={B}]{bhakṣamāṇa}
    \rdg[wit={E,P,U2}]{bhāṣamāṇaḥ}
    \rdg[wit={N1,N2,U1}]{bhāṣamāṇasya}}
    \app{\lem[wit={P,U2}]{sumadhuraṃ}
      \rdg[wit={B}]{samaghura°}
      \rdg[wit={L}]{samadhura°}
      \rdg[wit={N1,N2,U1}]{madhuraṃ}
      \rdg[wit={E}]{\om}}
    \app{\lem[wit={E,P,U1,U2}]{ramamāṇaḥ}
      \rdg[wit={N1,N2}]{rasamānaḥ}
      \rdg[wit={L}]{°ramamāṇaṃ}
      \rdg[wit={B}]{°ramāṇa}}
    svalilayā/}\\
  \tl{
    \app{\lem[wit={ceteri}]{bhāvābhāvavinirmuktaḥ}
      \rdg[wit={E}]{bhāvābhāvavinirmukto}
      \rdg[wit={B,L}]{bhāvāvinir muktaḥ}}
    sarvagrāhavivarjitaḥ\dd{}1\hskip-2pt\dd{}}
\end{tlg}
\end{ekdosis}
\begin{ekdosis}
  \begin{tlg}
%-----------------------------
%sadānaṃdamayo yogī sadābhyāsī sadā bhavet/     viruddhaduḥkhade  deśe virūpe   tibhayānake//1// \E
%sadānaṃdamayo yogī sadābhyāsī sadā bhavet      viruddhaduḥkhade  deśe virūpe   tibhayānake      \P
%sadāmayo      yogī sadābhyāsī sadā bhavet/     viruddhe duḥkhe   deśe śovirūpe   bhayānake/ \B
%sadāmayo    yoyogī sadābhyāsī sadā bhavet//    viruddhe duḥkha---deśe śovirūpe   bhayānake... \L
%sadānaṃdamayo yogī sadābhyāsī sadā bhavet/     viruddhe duḥkhade deśe śovirūpe tibhayānake/ \N1
%sadānaṃdamayo yogī sadābhyāsī sadā bhavet//1// viruddhe duḥkhade deśe virūpe   tibhayānake/ \N2 %%%last folio verso!!!!
%\om \D
%sadānaṃdamayo yogī sadābhyāso sadā bhavetd     viruddhe duḥkhade deśe vivarūpe   bhayānake \U1
%\om \U2
%-----------------------------
%The Yogin that is made of permanent bliss, is always engaged in practice, [even] in land which is hostile and uncomfortable, ugly and extremely terrible.  
%-----------------------------
  \note[type=source, labelb=527, lem={sadānandamayo yogī}]{Ysv\textsuperscript{PT}: sadānandamayo yogī sadābhyāsī sadā bhavet | viruddhe duḥkhadeśe ca virūpe 'tibhayānake |}
  \note[type=philcomm, labelb=528, lem={sadānaṃdamayo \ldots bhayānake}]{The verse is omitted in U\textsubscript{2}.}
\tl{
\app{\lem[wit={ceteri}]{sadānaṃdamayo}
  \rdg[wit={B,L}]{sadāmayo}}
\app{\lem[wit={ceteri}]{yogī}
  \rdg[wit={L}]{yoyogī}}
\app{\lem[wit={ceteri}]{sadābhyāsī}
  \rdg[wit={U1}]{sadābhyāso}}
sadā bhavet/}\\
\tl{
\app{\lem[wit={B,L,N1,N2,U1}]{viruddhe}
  \rdg[wit={E,P}]{viruddha°}}
\app{\lem[type=emendation, resp=egoscr]{duḥkhadeśe ca}
  \rdg[wit={E,P,N1,N2,U1}]{\korr duḥkhade deśe}
  \rdg[wit={B}]{duḥkhe deśe}
  \rdg[wit={L}]{duḥkhadeśe}}
\app{\lem[wit={E,P,N2}]{virūpe}
  \rdg[wit={B,L,N1}]{śovirūpe}
  \rdg[wit={U1}]{vivarūpe}}
\app{\lem[wit={E,P,N1,N2}]{'tibhayānake}
  \rdg[wit={B,L,U1}]{bhayānake}}\dd{}2\hskip-2pt\dd{}} 
\end{tlg}
\end{ekdosis}
\begin{ekdosis}
  \begin{tlg}
%-----------------------------
%iṣṭādyaniṣṭasaṃsparśe     rase  ca lavaṇādike/       pratyādāv api gaṃdhe  ca kaṃkoṣṇādivivarjayet//2// \E
%iṣṭādhaniṣṭaṃ saṃsparśe   rase  ca lavaṇādike        pratyādāv api gaṃdhe  ca kaṃṭakoṣyādivivarjjite   \P
%iṣṭādyaniṣṭasaṃsparśe     rase  ca lavaṇādike        pratyādāv api gaṃdhe  ca kaṭakoṣmādivarji/         \B
%iṣṭādyaniṣṭasaṃsparśe     rase  ca lavaṇādike        pūtyādāv  api gaṃdhe  ca kaṃṭakoṣmādivarji//        \L
%iṣṭādyaniṣṭasaṃsparśe/    rase  ca lavaṇādike        pūtyādāv  api gaṃdhe  ca kaṃṭakoṣmādivarjjite/     \N1
%iṣṭādyaniṣṭaṃ saṃsparśe/  rasaṃ ca lavaṇādiko //2//  pūtyādāv  api gaṃdhaṃ ca kaṇṭakeṣmādivarjjite//     \N2
%iṣṭādyaniṣṭasaṃsparśe     rase  ca lavaṇādike        pūjādāv   api gaṃdhe  ca kuṃṭakoṣmādivarjite         \U1
% \om \U2
%\om \D
%-----------------------------
%in desireable and undesireable contact in tastes like salty etc., evil smells, thorns, etc. [and in] renunciation. 
%-----------------------------
\note[type=philcomm, labelb=529, lem={iṣṭādyaniṣṭasaṃsparśe \ldots kaṇṭakoṣmādivarjjite}]{The verse is omitted in U\textsubscript{2}.}
  \tl{
  \app{\lem[wit={ceteri}]{iṣṭādyaniṣṭasaṃsparśe}
  \rdg[wit={P,N2}]{iṣṭādhaniṣṭaṃ saṃsparśe}} 
\app{\lem[wit={ceteri}]{rase}
  \rdg[wit={N2}]{\om}}
\app{\lem[wit={ceteri}]{lavaṇādike}
  \rdg[wit={N2}]{lavaṇādiko}}/}\\
\tl{
  \app{\lem[wit={L,N1,N2},alt={pūtyādāv}]{pūtyādā\skp{v-a}}
    \rdg[wit={B,E,P}]{pratyādāv}
    \rdg[wit={U1}]{pūjādāv}
  }\skm{v-a}pi
  \app{\lem[wit={ceteri}]{gandhe}
    \rdg[wit={N2}]{gaṃdhaṃ}} ca
  \app{\lem[type=emendation, resp=egoscr]{kaṇṭakoṣmādivarjane}
    \rdg[wit={E}]{\korr kaṃkoṣṇādivivarjayet}
    \rdg[wit={P}]{kaṃṭakoṣyādivivarjjite}
    \rdg[wit={B}]{kaṭakoṣmādivarji}
    \rdg[wit={L}]{kaṃṭakoṣmādivarji}
    \rdg[wit={N1}]{kaṃṭakoṣmādivarjjite}
    \rdg[wit={N2}]{kaṇṭakeṣmādivarjjite}
    \rdg[wit={U1}]{kuṃṭakoṣmādivarjite}}\dd{}3\hskip-2pt\dd{}}
\end{tlg}
\end{ekdosis}
\begin{ekdosis}
  \begin{tlg}
%-----------------------------
%sarvadaiva  sadābhyāsaḥ samaḥ syāt  sukhaduḥkhayoḥ/   evaṃ yogasya  karmmāṇi saṃkalparahitāni ca//3// %[p.85] \E
%sarvadaiva  sadābhyāsaḥ samaḥ syāt  sukhaduḥkhayoḥ    evaṃ yogasya  karmāṇi  saṃkalparahitāni ca              \P
%sarvadeva   sadābhyāsāḥ samaḥ syāt  sukhaduḥkhayoḥ/   evaṃ yogasya  karmmāṇi saṃkalparahitāni ca/ \B
%sarvadeva   sadābhyāsāḥ samaḥ sya/t sukhaduḥkhayoḥ//  evaṃ yogasya  karmmāṇi saṃkalparahitāni ca// \L
%sarvadeva   sadābhyāsāḥ sama  syāt  sukhaduḥkhayoḥ/   evaṃ yogasya  karmmāṇi saṃkalparahitāni ca/ \N1
%sarvadaiva  sadābhyāsāḥ mana  syāt  sukhaduḥkhayoḥ/   evaṃ bhūtasya karmāṇi saṃkalparahitāni ca/ \N2
%\om \D
%sarvadeva   sadābhyāsāḥ sama  syā   sukhaduḥkhayoḥ    evaṃ bhūta----karmāṇī saṃkalparahitāni ca        \U1
%sarvadaivaṃ sadābhyāsaḥ samaḥ syāt  sukhaduḥkhayoḥ//  evaṃ yogasya  karmāṇi saṃkalparahitāni ca//   \U2 
%-----------------------------
%He who is always truly in permanent practice shall be equanimous towards happiness and suffering. In this way the actions of the great yogin (\textit{bhūtasya}) are free from desire. 
%-----------------------------
\tl{
\note[type=source, labelb=530, lem={evaṃ bhūtasya}]{Ysv\textsuperscript{PT}: etad aniṣṭasaṃsparśe nyūnādhike balādhike | evam bhūtasya karmāṇi saṅkalparahitāni ca |}
\note[type=source, labelb=531, lem={sukhaduḥkhayoḥ}]{Ysv\textsuperscript{PT}: utpātarahite deśe kaṇṭakādivivarjite | abhyasyate sadā yogaḥ samaḥ syāt sukhaduḥkhayoḥ |}
\app{\lem[wit={E,P,N2}]{sarvadaiva}
  \rdg[wit={B,L,N1}]{sarvadeva}
  \rdg[wit={U2}]{sarvadaivaṃ}}
\app{\lem[wit={E,P,U2}]{sadābhyāsaḥ}
  \rdg[wit={B,L,N1,N2,U1}]{sadābhyāsāḥ}}
\app{\lem[wit={ceteri}]{samaḥ}
  \rdg[wit={N1,U1}]{sama}
  \rdg[wit={N2}]{mana}}
\app{\lem[wit={ceteri},alt={syāt}]{syā\skp{t-su}}
  \rdg[wit={L}]{sya | t}
  \rdg[wit={U1}]{syā}
}\skm{t-su}khaduḥkhayoḥ/}\\
\tl{
evaṃ \app{\lem[wit={N2}]{bhūtasya}
  \rdg[wit={U1}]{bhūta}
  \rdg[wit={ceteri}]{yogasya}}
karmāṇi saṃkalparahitāni ca\dd{}4\hskip-2pt\dd{}}
\end{tlg}
\end{ekdosis}
\ekdpb*{}
%%%%%%%%%%%%%%%%%%%%%%%%%%%%%%%%%%%%%%%%%%
%%%%%%%%PAGEBREAK%%%%%%%PAGEBREAK%%%%%%%%%
%%%%%%%%%%%%%%%%%%%%%%%%%%%%%%%%%%%%%%%%%%
%%%%%%%%%%%%%%%%PAGEBREAK%%%%%%%%%%%%%%%%%
%%%%%%%%%%%%%%%%%%%%%%%%%%%%%%%%%%%%%%%%%%
%%%%%%%%PAGEBREAK%%%%%%%PAGEBREAK%%%%%%%%%
%%%%%%%%%%%%%%%%%%%%%%%%%%%%%%%%%%%%%%%%%%
%%%%%%%%%%%%%%%%%%%%%%%%%%%%%%%%%%%%%%%%%%
%%%%%%%%%%%%%%%%%%%%%%%%%%%%%%%%%%%%%%%%%%
%%%%%%%%%%%%%%%%%%%%%%%%%%%%%%%%%%%%%%%%%%
%%%%%%%%PAGEBREAK%%%%%%%PAGEBREAK%%%%%%%%%
%%%%%%%%%%%%%%%%%%%%%%%%%%%%%%%%%%%%%%%%%%
%%%%%%%%%%%%%%%%PAGEBREAK%%%%%%%%%%%%%%%%%
%%%%%%%%%%%%%%%%%%%%%%%%%%%%%%%%%%%%%%%%%%
%%%%%%%%PAGEBREAK%%%%%%%PAGEBREAK%%%%%%%%%
%%%%%%%%%%%%%%%%%%%%%%%%%%%%%%%%%%%%%%%%%%
%%%%%%%%%%%%%%%%%%%%%%%%%%%%%%%%%%%%%%%%%%
%%%%%%%%%%%%%%%%%%%%%%%%%%%%%%%%%%%%%%%%%%
%%%%%%%%%%%%%%%%%%%%%%%%%%%%%%%%%%%%%%%%%%
%%%%%%%%PAGEBREAK%%%%%%%PAGEBREAK%%%%%%%%%
%%%%%%%%%%%%%%%%%%%%%%%%%%%%%%%%%%%%%%%%%%
%%%%%%%%%%%%%%%%PAGEBREAK%%%%%%%%%%%%%%%%%
%%%%%%%%%%%%%%%%%%%%%%%%%%%%%%%%%%%%%%%%%%
%%%%%%%%PAGEBREAK%%%%%%%PAGEBREAK%%%%%%%%%
%%%%%%%%%%%%%%%%%%%%%%%%%%%%%%%%%%%%%%%%%%
\begin{ekdosis}
  \begin{tlg}
%-----------------------------
%gacchan nṝṇāṃ ca saṃsparśāt tapaḥ kurvan na lipyate/  utpannatattvabodhasya hy udāsīnasya sarvadā//4// \E %%%% Amanaska 2.36 
%gacchan nṝṇāṃ ca saṃsparśāt pāpaḥ kurvan na lipyate   utpannatattvabodhasya udāsīnasya sarvadā        \P
%gacchan nṝṇāṃ ca saṃsparśot pāpaṃ kurvaṃ na lipyate/  utpannatattvabodhasya udāsīnasya sarvadā    \B
%gachan  nṝṇāṃ ca saṃsparśāt pāpaṃ kurvan na lipyate/  utpannatattvabodhasya udāsīnasya sarvadā    \L
%gacchan nṝṇāṃ ca saṃsparśot pāpaṃ kurvan na lipyate/  utpannatattvabodhasya udāsīnasya sarvadā//    \N1
%gacchan nṝṇāṃ ca saṃsparśāt pāpaṃ kurvaṃ na lipyate/  utpannatattvabodhasya udāsīnasya sarvadā//    \N2
%\om \D
%gacha   nṛṇāṃ ca saṃsparśāt pāpaṃ kurvaṃ nna lipyate  utpannatatvabodhasyād udāsīnasya sarvadā    \U1
%gacchan nṝṇāṃ ca saṃsparśāt pāpaṃ kurvan na lipyate// utpannatattvabodhasya udāsīnasya sarvadā// \U2
%-----------------------------
%He does not become tainted by sin by touching men while walking for him who has arisen to the awakening of reality, who is in every way equaminous. 
%-----------------------------
\tl{
\note[type=source, labelb=532, lem={gacchan}]{Ysv\textsuperscript{PT}: evaṃ gacchan svapan paśyan pāpapuṇyairna lipyate | utpannatattvabodhaḥ syāt sadā śīlasya sarvadā |}
\note[type=testium, labelb=533, lem={utpannatattvabodhasya}]{Amanaska 2.36: utpannatattvabodhasya hy udāsīnasya sarvadā | sadābhyāsaratasyaitan naikatrāpy upayujyate ||}
\app{\lem[wit={ceteri},alt={gacchan}]{gaccha\skp{n-nṛ}}
  \rdg[wit={U1}]{gacha}
}\skm{n-nṛ}ṇāṃ ca
\app{\lem[wit={ceteri},alt={saṃsparśāt}]{saṃsparśā\skp{t-pā}}
  \rdg[wit={B,N1}]{saṃsparśot}
}\app{\lem[wit={ceteri},alt={pāpaṃ}]{\skm{t-pā}paṃ}
  \rdg[wit={P}]{pāpaḥ}
  \rdg[wit={E}]{tapaḥ}}
kurvan na lipyate/}\\
\tl{
utpannatattvabodhasya
\app{\lem[wit={ceteri}]{udāsīnasya}
  \rdg[wit={E}]{hy udāsīnasya}}
sarvadā\dd{}5\hskip-2pt\dd{}}
\end{tlg}
\end{ekdosis}
\begin{ekdosis}
  \begin{tlg}
%-----------------------------
%tadā dṛṣṭiviśeṣaḥ syād vividhāny āsanāni ca/      aṃtaḥkaraṇajā bhāvā yogino nopayoginaḥ//5// \E %%%Amanaska 2.37
%tadā dṛṣṭiviśeṣa--syād vidhāny   āsanāni ca       aṃtaḥkaraṇajā bhāvā yogino nopayoginaḥ \P
%tadā dṛṣṭiviśeṣa--syād vidhāny   āsanāni ca/      aṃtaḥkaraṇajā bhāvā yogino nopayoginaḥ/ \B
%tadā dṛṣṭiviśeṣa--syād vidhāny   āsanāni ca//     aṃtaḥkaraṇajā bhāvā yogino nopi yoginaḥ// \L
%tadā dṛṣṭiviśeṣaḥ syād vidhāny   āsanāni ca/      aṃtaḥkaraṇajā bhavā yogino nopayoginaḥ/ \N1
%tadā dṛṣṭiviśeṣaḥ syād vividhāny āsanāni ca//5//  aṃtaḥkaraṇajā bhavā yogino nopayoginaḥ/ \N2
%tadā dṛṣṭir viśeṣasyād vidhāny   āsanāni ca       aṃtaḥkaraṇayo bhāvā yogino nopayoginaḥ \U1
%tadā dṛṣṭiviśeṣaḥ syād vividhāny āsanāni ca//     aṃtaḥkaraṇajā bhāvā yogino nopi yoginaḥ// \U2
%\om \D
%-----------------------------
%Then the different gazing points, the various postures and the states born from the sense-faculties won't be useful to the yogī. 
%-----------------------------
\note[type=source, labelb=534, lem={aṃtaḥkaraṇajā}]{Ysv\textsuperscript{PT}: pare dṛṣṭivilaṃ na syād vividhāni mṛtāni ca | antaḥkaraṇam etasya yogino niṣkriyaṃ [niṣkalā?] tu sa |}
\note[type=testium, labelb=535, lem={tadā dṛṣṭiviśeṣaḥ}]{Amanaska 2.37: tadā dṛṣtiviśeṣāś ca vividhāny āsanāni ca | antaḥkaraṇabhāvaś ca yogino nopayujyate ||} %%%Variants see Birch 57
\tl{
tadā
\app{\lem[wit={E,N1,N2,U2}]{dṛṣṭiviśeṣaḥ}
  \rdg[wit={B,L,P}]{dṛṣṭiviśeṣa}
  \rdg[wit={U1}]{dṛṣṭir viśeṣa}}
syād
\app{\lem[wit={E,N2,U2},alt={vividhāny}]{vividhā\skp{ny-ā}}
  \rdg[wit={ceteri}]{vidhāny}
}\skm{ny-ā}sanāni ca/}\\
\tl{
  \app{\lem[wit={ceteri}]{antaḥkaraṇajā}
    \rdg[wit={U1}]{aṃtaḥkaraṇayo}}
  bhavā yogino
  \app{\lem[wit={ceteri}]{nopayoginaḥ}
    \rdg[wit={L,U2}]{nopi yoginaḥ}}\dd{}6\hskip-2pt\dd{}} 
\end{tlg}
\end{ekdosis}
\begin{ekdosis}
\begin{tlg}
%-----------------------------
%sarva rājapadasthasya  niṣkalādhyātmavedinaḥ/   yadyat prayatnaniḥpāyaṃ   tattatsarvam akāraṇam// 6// \E
%sarvadā sahajasthasya  niṣkalādhyātmavedinaḥ    yadyat prayatnaniḥpārdhaṃ tattatsarvam akāraṇam    \P
%sarvadya-sahajasya     niṣkalādhyātmavedinā/    yadyat prayatnaniḥpādya   tat sarvam   akāraṇāt/ \B
%sarvadya-sahajasthasya niṣkalādhyātmavedinā//   yadyat prayatnaniḥpādya   tat sarvem  ikāraṇāt//2// \L
%sarvadā  sahajasthasya niṣkalādhyātmavedina/    yadyat prayatnaniḥpādyaṃ  tattatsarvam akāraṇaṃ/ \N1
%sarvadā  sahajasthasya niṣkalādhyātmavedina/    yadyat prayatnaniḥpādyaṃ  tattatsarvam na kāraṇaṃ//7//  \N2
%sarvadā  mahajarasya   niṣkalādhyātmavedanā     yadyat aprayatra niṣyayiṃ tat sarvam  akāraṇāṃ  \U1
%sarvadā  sahajasthasya niṣkalādhyātmavedinaḥ//  yadyat prayatnaniḥpādyaṃ  tat sarvaṃ  kāraṇa//  \U2
%\om \D
%-----------------------------
%For the knower of the undivided supreme self, who is always in the natural state, whatever is to be generated with effort, all of that is without motive. 
%-----------------------------
  \note[type=source, labelb=536, lem={sarvadā}]{Ysv\textsuperscript{PT}: sarvadā sahajas tasya niṣkalādhyātmavādinaḥ | yadā prayatnaniṣpādyaṃ grāhyaṃ sarvam akāraṇam |}
\tl{
  \app{\lem[wit={ceteri}]{sarvadā}
    \rdg[wit={B,L}]{sarvadya}
    \rdg[wit={E}]{sarva°}
}\app{\lem[type=emendation, resp=egoscr,alt={sahajas tasya}]{sahajas\skp{-}tasya}
  \rdg[wit={L,P,N1,N2,U2}]{\korr sahajasthasya}
  \rdg[wit={B}]{sahajasya}
  \rdg[wit={U1}]{mahajarasya}
  \rdg[wit={E}]{rājapadasthasya}}
\app{\lem[wit={E,P,U2}]{niṣkalādhyātmavedinaḥ}
  \rdg[wit={B,L,U1}]{niṣkalādhyātmavedinā}
  \rdg[wit={N1,N2}]{niṣkalādhyātmavedina}}/}\\
\tl{
  yatyat \app{\lem[wit={N1,N2,U2}]{prayatnaniṣpādyaṃ}
    \rdg[wit={U1}]{aprayatra niṣyayiṃ}
    \rdg[wit={B,L}]{prayatnaniḥpādya}
    \rdg[wit={P}]{prayatnaniḥpārdhaṃ}
    \rdg[wit={E}]{prayatnaniḥpāyaṃ}}
\app{\lem[wit={E,P,N1,N2},alt={tattatsarvam}]{tattatsarva\skp{m-a}}
  \rdg[wit={B,U1,U2}]{tat sarvam}
  \rdg[wit={L}]{tat sarvem}
}\app{\lem[wit={E,P,U1}]{\skm{m-a}kāraṇāṃ}
  \rdg[wit={B}]{akāraṇāt}
  \rdg[wit={L}]{ikāraṇāt}
  \rdg[wit={N2}]{na kāraṇaṃ}
  \rdg[wit={U2}]{kāraṇa}}\dd{}7\hskip-2pt\dd{}} 
\end{tlg}
\end{ekdosis}
\bigskip
\begin{ekdosis}
  \begin{prose}
%-----------------------------
%vilāsinīnāṃ manohārigānaśravaṇāt/ \E
%vilāsinīnāṃ manohārigānaśravaṇāt  \P
%vilāsinīnāṃ manohārigānaśravaṇāt/ \B
%vilāsinīnāṃ manohārigānaśravaṇāt// \L
%vilāsinīnāṃ manohārigītaśravaṇāt/ \N1
%vilāsinīnāṃ manohārigītaśravaṇāt/ \N2
%vilāsinīnāṃ manohārigītaśravaṇāt \U1
%vilāsinīnāṃ manohārigānaśravaṇāt// \U2
%\om \D
%-----------------------------
%Because of listening to mindblowing musical performance of charming women, ... 
%-----------------------------
vilāsinīnāṃ
\app{\lem[wit={N1,N2,U1}]{manohārigītaśravaṇāt}
  \rdg[wit={B,E,L,P,U2}]{manohārigānaśravaṇāt}}\dd{} 
%-----------------------------
%atisauṃdaryakāminīnāṃ rūpadarśanāt/ \E
%atisuṃdaraṃ kāmināṃ   rūpadarśanāt    \P
%atisauṃdarakāminināṃ  rūpadarśanāt/ \B
%atisuṃdarakāminināṃ   rūpadarśanāt// \L
%atisuṃdarakāminīnāṃ   rūpadarśanāt// \N1
%atisuṃdarakāminīnāṃ   rūpadarśanāt// \N2
%\om \D
%atisuṃdarakāminīnāṃ   rūpadarśanāt  \U1
%atisuṃdarakāminīnāṃ   rūpadarśanāt//  \U2
%-----------------------------
%because of seeing the form of a extremely beautiful women, ... 
%-----------------------------
\app{\lem[wit={N1,N2,U1,U2}]{atisundarakāminīnāṃ}
  \rdg[wit={E}]{atisauṃdaryakāminīnāṃ}
  \rdg[wit={P}]{atisuṃdaraṃ kāmināṃ}
  \rdg[wit={B}]{atisauṃdarakāminināṃ}
  \rdg[wit={L}]{atisauṃdarakāminināṃ}}
rūpadarśanāt\dd{}
%-----------------------------
%kastūrī karpūrayor gaṃdhagrahaṇāt/ \E
%kastūrī karpūrayor gaṃdhagrahaṇāt \P
%kastūrī karpūrayor gaṃdhagrahaṇāt/ \B
%kastūrī karpūra----gaṃdhayor grahaṇāt/ \L  %%%S.15 Anfang!!
%\om \D
%kastūrī karpūra----gaṃdhayār gaṃdhagrahaṇāt/ \N1???
%kasturī karpūra----gandhagrahaṇāt/ \N2
%kastūri karpuro    gaṃdhagrahaṇāt \U1
%kastūrī karpūrayo  gaṃdhagrahaṇāt// \U2
%-----------------------------
%from smelling the fragnance of camphor and musk, ... 
%-----------------------------
\app{\lem[wit={ceteri},alt={kastūrī°}]{kastūrī}
  \rdg[wit={U1}]{kastūri°}
}\app{\lem[wit={L}]{karpūragandhayo\skp{r-gra}}
  \rdg[wit={B,E,P}]{karpūrayor gaṃdha°}
  \rdg[wit={N1}]{karpūragaṃdhayār gaṃdha°}
  \rdg[wit={N2}]{karpūragandha°}
  \rdg[wit={U1}]{karpuro gaṃdha°}
  \rdg[wit={U2}]{karpūrayo gaṃdha°}}
grahaṇāt\dd{}
%-----------------------------
%manaḥ śaityakāri    komala----vastunaḥ  sparśakāraṇāt/ \E
%manaḥ śaityakāri    komala----vastunaḥ  sparśakāraṇāt \P
%manaḥ śaityakāri    komala----vastunaḥ  saṃsparśakāṃ... \B
%manaḥ śaityakāri    komala----vastunaḥ/ saṃsparśakaṃ... \L
%manaḥ śītalakārī atikomalaparavastunaḥ  sparśakaraṇāt// \N1
%manaḥ śītalakārī atikomalaparavastunaḥ  sparśakaraṇāt \N2
%\om \D
%manaḥ sīlakārī   atikomalaparavastunaḥ  sparśakaraṇāt \U1
%manaḥ śaityakāri    komala----vastunaḥ  sparśakāraṇāt// \U2
%-----------------------------
%because of the execution of touching of very soft things  .... the mind that is free from passion 
%-----------------------------
\app{\lem[wit={N1,N2}]{śītalakārī}
  \rdg[wit={U1}]{sīlakārī}
  \rdg[wit={ceteri}]{śaityakāri}}
\app{\lem[wit={N1,N2,U1}]{atikomalaparavastunaḥ}
  \rdg[wit={ceteri}]{komalavastunaḥ}}
\app{\lem[wit={ceteri}]{sparśakāraṇāt}
  \rdg[wit={B}]{saṃsparśakāṃ}
  \rdg[wit={L}]{saṃsparśakaṃ}}\dd{}
%-----------------------------
%atimādhuryaṃ citte karoti/ \E
%atimādhuryaṃ citte karoti \P
%atimādhuryaṃ citte karoti/ \B
%atimādhuryaṃ citte karoti// \L
%atimādhuryaṃ citte karoti/ \N1
%atidhūryaṃ  cittaṃ karoti/ \N2
%\om \D
%atimādhuryaṃ citte karoti \U1
%atimādhuryaṃ cikrī karoti/ \U2
%-----------------------------
%creates excellent friendlyness in the mind/in thinking? citti?!  
%-----------------------------
atimādhuryaṃ \app{\lem[wit={ceteri}]{citte}
  \rdg[wit={N2}]{cittaṃ}
  \rdg[wit={U2}]{cikrī}}
karoti/
%-----------------------------
%tādṛśaḥ svādanāt/ \E
%tādṛśaḥ svādanāt \P
%tādṛśaḥ svādanāt/ \B
%tādṛśaḥ svādanāt// \L
%tādṛśā  svādanāt/ \N1
%tādṛśā  svādanāt/ \N2
%\om \D
%tādṛśā  svādanāt \U1
%tādṛśā  svādanāt// \U2
%-----------------------------
%From such enjoyment 
%-----------------------------
\app{\lem[wit={B,E,L,P}]{tādṛśaḥ}
  \rdg[wit={N1,N2,U1,U2}]{tādṛśā}}
svādanāt\dd{}
%-----------------------------
%anekadeśānāṃ sādhvasādhusthānadarśanāt/ \E
%anekadeśānāṃ sādhvasādhusthānadarśanāt  \P
%anekadeśānāṃ sādhvasādhusthānadarśanāt/ \B
%anekadeśānāṃ sādhvasādhusthānadarśanāt// \L
%anekadeśānāṃ sādhvasādhusthānadarśanāt/ \N1
%anekadeśānāṃ sādhvasādhusthānadarśanāt/ \N2
%\om \D
%anekadeśānāṃ       sādhūsthānadarśanāt \U1
%anekadeśānāṃ sādhvasādhusthānadarśanāt// \U2
%-----------------------------
%From seeing good and bad places of many countries, 
%-----------------------------
anekadeśānāṃ \app{\lem[wit={ceteri}]{sādhvasādhusthānadarśanāt}
  \rdg[wit={N1}]{sādhūsthānadarśanāt}}\dd{} 
%-----------------------------
%mitreṇa  saha komalavacanāt/ \E
%maitreṇa saha komalavacanāt \P
%maitreṇa saha komalavacanāt/ \B
%maitreṇa saha komalavacanāt// \L
%maitreṇa saha komalavacanāt/ \N1
%maitreṇa saha komalavacanāt/ \N2
%\om \D
%maitreṇa saha komalavacanāt \U1
%maitreṇa saha komalavacanāt// \U2
%-----------------------------
%from speaking sweet with friends,  
%-----------------------------
\app{\lem[wit={ceteri}]{maitreṇa}
  \rdg[wit={E}]{mitreṇa}}
saha komalavacanāt\dd{}
%-----------------------------
%śatruṇā  saha kaṭhinavacanāt/ \E
%śatruṇā  saha kaṃvinya vacanāt  \P
%śatruṇā  saha kaṭhinya vacanāt/ \B
%śatruṇā  saha kāṭhinya vacanāt// \L
%śatruṇāṃ saha kaṭhinya vacanān \N1
%śatruṇāṃ saha kavinya vacanād- \N2
%\om \D
%śatṛṇā   saha kāṭhinya vacanāt    \U1
%śatruṇāṃ saha kāṭhinya vacanāt// \U2
%-----------------------------
%from speaking with firmness of character to enemies, 
%-----------------------------
\app{\lem[wit={B,E,L,P,U1}]{śatruṇā}
  \rdg[wit={N1,N2,U2}]{śatruṇāṃ}}
saha \app{\lem[wit={L,U1,U2},alt={kāṭhinya°}]{kāṭhinya}
  \rdg[wit={E}]{kaṭhina°}
  \rdg[wit={P}]{kaṃvinya°}
  \rdg[wit={B}]{kaṭhinya°}
  \rdg[wit={N2}]{kavinya°}}
\app{\lem[wit={ceteri}]{vacanāt}
  \rdg[wit={N1}]{vacanān}
  \rdg[wit={N2}]{vacanād}}\dd{}
\note[type=philcomm, labelb=537, lem={vacanāt}]{Evidence of B stops here. The last folio of the manuscript is missing.}
%-----------------------------
%yasya manasi harṣo vā dveṣo na    bhavati   sa puruṣa  īśvaropadeśiko jñeyaḥ/ \E
%yasya manasi harṣo vā dveṣo na    bhavati   sa puruṣa  īśvaropadeśako jñeyaḥ \P
%yasya mana   harṣo vā dveṣo       bhavati   sa puruṣa  īśvaropade ko  jñeyaḥ/ \L
%yasya manasi harṣo vā dveṣo na    bhavati/  sa puruṣa  īśvaropadeśako jñeyaḥ// \N1
%yasya manasi harṣo vā dveṣo na    bhavati/  sa puruṣa  īśvaropadeśako jñeyaḥ  \N2
%yasya manasī harṣo vā dveṣo vā na bhavati   sa puruṣa  īśvaropadeśako jñeyaḥ \U1
%yasya manasī harṣo vā dveṣo na    bhavati//    pururṣo īśvaropadeśako jñeyaḥ// \U2
%                                     vati// sa puruṣa  īśvaropadeśako jñeyaḥ// \D
% missing last folio \B
%-----------------------------
%love and hatred does not arise in his mind. This person is to be known as a teacher of the supreme god. 
%-----------------------------
yasya
\app{\lem[wit={ceteri}]{manasi}
    \rdg[wit={U1,U2}]{manasī}
    \rdg[wit={L}]{mana}}
harṣo vā dveṣo
\app{\lem[wit={ceteri}]{na}
    \rdg[wit={U1}]{vā na}
    \rdg[wit={L}]{\om}}
bhavati/
\note[type=philcomm, labelb=538, lem={dveṣo na bhavati}]{Evidence of witness D resumes from here.}
\app{\lem[wit={ceteri}]{sa puruṣa}
   \rdg[wit={U2}]{puruṣo}}
\app{\lem[wit={ceteri}]{īśvaropadeśako}
   \rdg[wit={L}]{īśvaropade ko}}
 jñeyaḥ/
   \end{prose}
 \end{ekdosis}
 \end{document}
\ekdpb*{}
\begin{ekdosis}
\begin{prose}
%-----------------------------
%svalīlayā    vadati calati      bhāvābhāvayoś cittam udāsīnaṃ  bhavati   kasyāṃcid vārtāyāṃ  harṣaviṣādaṃ na karoti   yasya manaḥ   sahajānaṃde    magnaṃ  bhavati/ \E   [p.87] 
%svalīlayā    vadati calati va   bhāvābhāvayoś cittam udāsīnaṃ  bhavati   kasyāṃcid vārttāyāṃ       haṭhaṃ na karoti   yasya manaḥ   sahajānaṃde    magnaṃ  bhavati \P
%svalīlayā    vadati calati ca   bhāvābhāvayoś cittam udāsīna   bhavati// kasyāṃcid vārttāyāṃ        haṭaṃ na karoti// yasya manaḥ// sahajānaṃdam    añjaṃ  bhavati/ \L
%svalīyayā    vadati calati ca   bhāvābhāvayoś cittam udāsīnaṃ  bhavati/  kasyāṃcid vārttāyāṃ       haṭhaṃ na karoti   yasya manaḥ   sahajānaṃde    magnaṃ  bhavati/ \N1
%svalīyayā    vadati calati ca/  bhāvābhāvayoś cittam udāsīnaṃ  bhavati/  kasyāṃcid vārttāyāṃ       haṭhaṃ na karoti/  yasya mana    sahajānaṃde    magnaṃ  bhavati \N2
%svalīlayā    vadati calati ca   bhāvābhāvayoś cittam udāsīnaṃ  bhavati// kasyāṃcid vārttāyāṃ       haṭhaṃ na karoti   yasya manaḥ   sahajānaṃde    magnaṃ  bhavati/ \D
%svalīlayā    vadati calati ca   bhāvābhāvayoś cittam udāsīnaṃ  bhavati   kasyāṃcid vārttāyāṃ        haṭaṃ na karoti   yasya manaḥ   sahajānaṃda    saṃjñaṃ bhavati \U1 %%%304.jpg
%svalīlayā    vadati calatī ca// bhāvābhāvayoś cittam udāsīnaṃ  bhavati// kasyāṃcid vārttāyāṃ       haṭhaṃ na karoti// yasya manaḥ   sahajānaṃ daṃde magnaṃ bhavati// \U2
%missing last folio \B  -----------------------------------------------------------vārttāyāṃ= Loc. Singular 
%-----------------------------
%Durch das eigene [göttliche] Spiel redet er und geht er. Der Geist is gleichmütig in Existenz und nicht Existenz. Nicht mal gewaltsam ist es möglich irgendwie in Worte zu fassen [wie es ist, wenn] dessen Geist in innewohnender Glückseligkeit versunken ist. 
%-----------------------------
\app{\lem[wit={ceteri}]{svalīlayā}
  \rdg[wit={N1,N2}]{svalīyayā}}
vadati calati
\app{\lem[wit={ceteri}]{ca}
  \rdg[wit={P}]{va}
  \rdg[wit={E}]{\om}}
bhāvābhāvayoś-cittam-udāsīnaṃ bhavati/
kasyāṃcid-vārttāyāṃ
\app{\lem[wit={ceteri}]{haṭhaṃ}
  \rdg[wit={E}]{harṣaviṣādaṃ}
  \rdg[wit={L,U1}]{haṭaṃ}} na karoti/
yasya
\app{\lem[wit={ceteri}]{manaḥ}
  \rdg[wit={N2}]{mana°}}
\app{\lem[wit={ceteri}]{sahajānande}
  \rdg[wit={L}]{sahajānaṃdam}
  \rdg[wit={U1}]{sahajānaṃda}
  \rdg[wit={U2}]{sahajānaṃ daṃde}}
\app{\lem[wit={ceteri}]{magnaṃ}
  \rdg[wit={L}]{añjaṃ}
  \rdg[wit={U1}]{saṃjñaṃ}}
bhavati/
%-----------------------------
%tena      puruṣeṇa dṛṣṭiḥ sthirā karttavyā/ \E
%tena bhya puruṣeṇa dṛṣṭiḥ sthirā karttavyā \P
%tena      puruṣeṇa dṛṣṭiḥ sthirā karttavyā/ \L
%tena      puruṣeṇa dṛṣṭiḥ sthirā karttavyaṃ// \N1
%tena      puruṣeṇa dṛṣṭiḥ sthirā karttavyaṃ \N2
%tena   svapuruṣeṇa dṛṣṭiḥ sthirā karttavyaṃ// \D
%tena      puruṣeṇa dṛṣṭi--sthirā karttavyaḥ \U1
%tena      puruṣeṇa dṛṣṭiḥ sthirā karttavyā// \U2
% missing last folio \B
%-----------------------------
%Von diesem Menschen soll der Blick fixiert werden, 
%-----------------------------
tena
\app{\lem[wit={ceteri}]{puruṣeṇa}
  \rdg[wit={P}]{bhya puruṣeṇa}
  \rdg[wit={D}]{svapuruṣeṇa}}
\app{\lem[wit={ceteri}]{dṛṣṭiḥ}
  \rdg[wit={U1}]{dṛṣṭi°}}
sthirā
\app{\lem[wit={ceteri}]{karttavyā}
  \rdg[wit={D,N1,N2}]{karttavyaṃ}}/ 
%-----------------------------
%āsanaṃ dṛḍhaṃ karttavyam/ \E
%āsanaṃ dṛḍhaṃ karttavyaṃ  \P
%āsanaṃ dṛḍhaṃ karttavyaṃ/  \L
%āsanaṃ dṛḍhaṃ karttavyaṃ/  \N1
%āsanaṃ dṛḍhaṃ karttavyaṃ/  \N2
%āsanaṃ dṛḍhaṃ karttavyaṃ//  \D
%āsana--dṛḍhaṃ karttavyaṃ  \U1
%āsanaṃ dṛḍhaṃ karttavyam// \U2
%missing last folio \B
%-----------------------------
% soll eine stabile Sitzposotion eingenommen werden, 
% -----------------------------
\app{\lem[wit={ceteri}]{āsanaṃ}
  \rdg[wit={U1}]{āsana°}}
dṛḍhaṃ karttavyaṃ/ 
%-----------------------------
%pavanaḥ sthiraḥ karttavyaḥ/ \E
%pavanaḥ sthiraḥ karttavyaḥ \P
% pavanaḥ sthiraḥ karttavyaḥ \N1
% pavanaḥ sthiraḥ karttavyaṃ// \N2
%pavanaḥ sthiraḥ karttavyaḥ// \D
%pavanaḥ sthiraḥ karttavyaḥ \U1
%pavanaḥ sthiraḥ karttavyaḥ// \U2
% missing last folio \B
% \om \L
%-----------------------------
%soll der Atem stabilisiert werden, 
%-----------------------------
\note[type=philcomm, labelb=539, lem={pavanaḥ \ldots}]{Sentence is omitted in L.}
pavanaḥ sthiraḥ
\app{\lem[wit={ceteri}]{karttavyaḥ}
  \rdg[wit={N2}]{karttavyaṃ}}/  
%-----------------------------
%etādṛśaḥ   kaścin niyamaḥ/  siddhasya noktaḥ \E
%etādṛśaḥ   kaścin niyamaḥ   siddhasya noktaḥ  \P
%etādṛśaḥ// kaścin niyamaḥ// siddhasya noktaḥ//  \L
%etādṛśaḥ   kaścinniyamaḥ/   siddhasya noktaḥ/ \N1
%etādṛśaḥ   kaścinniyamaḥ    siddhasya noktaḥ  \N2
%etādṛśaḥ   kaścin niyamaḥ// siddhasya noktaḥ//   \D
%etādṛśaḥ   kaścinīyamaḥ     siddhasya noktaḥ \U1
%etādṛśaḥ   kaścinnīyamaḥ//  siddhasya noktaḥ \U2
% missing last folio \B
%-----------------------------
%No such rule is stated for the Siddha. 
%-----------------------------
etādṛśaḥ \app{\lem[wit={ceteri},alt={kaścin niyamaḥ}]{kaścin\skp{-}niyamaḥ}
  \rdg[wit={U1}]{kaści nīyamaḥ}
  \rdg[wit={U2}]{kaścin nīyamaḥ}}
siddhasya noktaḥ/ 
%-----------------------------
%manaḥpavanābhyāṃ  yadā sahajānaṃda----svasvarūpeṇa prakāśyate   sa sahajayogaḥ   kathyate/ \E
%manaḥpavanābhyāṃ  yadā sahajānaṃdaḥ   svasvarūpeṇa prakāśyate   sa sahajayogaḥ   kathyate \P
%mana--pavanābhyāṃ yadā sahajānaṃda----svasvarūpeṇa prakāśyate   sa sahajayogaḥ// kathyate... \L
%manaḥpavanābhyāṃ  yadā sahajānaṃdaḥ/  svasvarūpeṇa prakāśyate   sa sahajayoga    kathyate/ \N1
%manaḥpavanābhyāṃ  yadā sahajānaṃdaḥ   svasvarūpeṇa prakāśyate   sa sahajo yogya  kathyate// \N2
%manaḥpavanābhyāṃ  yadā sahajānaṃdaḥ// svasvarūpeṇa prakāśyate   sa sahajayoga    kathyate// \D
%manaḥpavanābhyāṃ  yadā sahajānaṃdaḥ   svasvarūpeṇa prakāśate    sa sahayogaḥ     kathyate \U1
%manaḥpavanābhyāṃ  yadā sahajānaṃdaḥ   svasvarūpeṇa prakāśyate// sa sahajayogaḥ   kathyate// \U2
% missing last folio \B
%-----------------------------
%When by means of mind and breath the natural bliss appears through ones own true nature, it is called Sahajayoga.   
%-----------------------------
\app{\lem[wit={ceteri}]{manaḥpavanābhyāṃ}
  \rdg[wit={L}]{manapavanābhyāṃ}}
yadā
\app{\lem[wit={ceteri}]{sahajānandaḥ}
  \rdg[wit={E,L}]{sahajānaṃda°}}
svasvarūpeṇa
\app{\lem[wit={ceteri}]{prakāśyate}
  \rdg[wit={U1}]{prakāśate}}/
sa
\app{\lem[wit={ceteri}]{sahajayogaḥ}
  \rdg[wit={D,N1}]{sahajayoga}
  \rdg[wit={N2}]{sahajo yogya}
  \rdg[wit={U1}]{sahayogaḥ}}
kathyate/ 
%-----------------------------
%te      madhye     iti cakravarttikathanam// \E
%te      madhye     iti cakravarttikathanam// \P
%rājayogamadhye     iti cakravarti         kathyate  \L
%rājayogamadhye     iti cakravartti  nāma kathanaṃ// \N1
%rājayogamadhye     iti cakravarttī  nāma kathanam // \N2
%rājayogamadhye     iti cakravarttī  nāma kathanaṃ// \D
%rājayogamadhye     iti cakravaktya  nāma madhye iti cakravartye nāma madhye kathanaṃ  \U1
%rājayogasya madhye iti cakravarti         kathyate// \U2
% missing last folio \B
%-----------------------------
%Within Rājayoga the name Cakravarttī is given [to it]. 
%-----------------------------
\app{\lem[wit={ceteri}]{rājayogamadhye}
  \rdg[wit={U2}]{rājayogasya madhye}
  \rdg[wit={E,P}]{te madhye}}
iti
\app{\lem[wit={D,N2}]{cakravartī}
  \rdg[wit={E,P,L,N1,U2}]{cakravarti}
  \rdg[wit={U1}]{cakravaktya}}
\app{\lem[wit={D,N1,N2,U1}]{nāma}
  \rdg[wit={ceteri}]{\om}} 
\app{\lem[wit={ceteri}]{kathanaṃ}
  \rdg[wit={L,U2}]{kathyate}
  \rdg[wit={U1}]{madhye iti cakravartye nāma madhye kathanaṃ}}/\\\\
%-----------------------------
%iti śrīsarvaguṇasampannapaṃḍita-sukhānandamiśrasūrisūnupaṇḍita-jvālāprasādamiśrakṛtabhāṣāṭīkāsahito rājayoge binduyogaḥ samāptaḥ// śubhamastu//śrīrastu// \E
%iti śrīrāmacaṃdraparamahaṃsa viracitas tatvabinduyogasamāptaḥ saṃvat 1867 pauṣakṛṣṇaḥ 12 ravau śubham bhuyāt //??//\P
% missing last folio \B
% iti rājamacaṃdraparahaṃsa viracites tatvabiṃduyogasamāptaṃ// śrī kṛṣṇārpaṇamastu// cha// \L
% iti śrī paramarahasyāṃ śrīrāmacaṃdraviracitāyāṃ tatvayogabiṃdu samāptaḥ// //śrī svasti// //saṃvat 837   \N1  %%%% 1716 n. Chr.
% iti śrī paramarahasye  śrīrāmacaṃdraviracitāyāṃ tatvayogabindu samāptam// //śubham// yadakṣarapadabhraṣṭaṃ mātrāhīnaṃcayaḍ? bhavet// tat sarvaṃ kṣamya tā?m eva prasīdaparameśvara //1 // sūrye turaṅge navacandraghasre jyeṣṭhākhyakṛṣṇe bhṛguvārayuktam || tattvaprayogaḥ ṣaḍaharṣasaṇjñaṃ likhitaṃ suhetoḥ bhavatīha dehi || bhūyāt \N2
% iti paramarahasyāṃ śrīrāmacaṃdraviracitāyāṃ tatvayogabiṃdu samāptaḥ// śubhamastu/ saṃvat 1841// bhādau śudha 15tnīo vesarva śake rārāma rāma cha    \D
% iti śrī pāramahaṃsyāṃ śrī rāmacaṃdraviracitāyāṃ tatvayogaviduḥ samāptaḥ śubhaṃ bhūyāt // // atarlakṣyaṃ bahi dṛḍhir nirmeṣomeṣa varjitaḥ saiṣāśāṃbhavīmudrā sarvata,n treṣugopitā 1 aṃtark ..... %see last 2 folios verses beyond text quote from fourth chapter of HP \U1
% iti śrī rāmacaṃdraparamahaṃsaviracitas tatvabiṃduyogasamāptaḥ// śrī śubhaṃ bhavatu// śrīsītārāmārpaṇamastuḥ// idaṃ pustakaṃ// śake 1805// vikramārka saṃmat// 1140// jayanām asaṃvatsare// udagayaṇe// griṣmaṛtau?// vaiśālhemāse// kṛṣṇapakṣe// tithau 23// bhānuvāsare// prathamayāmye// śrī kṣetra avaṃtikāyāṃ// śrī mahārudramahākālasaṃnidhāne na saṃpūrṇaṃ// lekhanaṃ ānaṃt? suta bābājoo rājadherakareṇa likhyate// yādṛśaṃ pustakaṃ dṛṣtvā tādṛsaṃ likhitaṃ mayā// yadi śuddhaṃ aśuddho cā mama doṣo na dīyate//1// śrīrāma// cha//    \U2
% -----------------------------
\bigskip
iti śrīrāmacandraparamahaṃsaviracitas-tatvayogabinduḥ samāptaḥ/
\end{prose}
\end{ekdosis}
\end{document}
%-----------------------------
%%%%deciphering last folio margin note of %N1!!!  
%\begin{alignment}[
%    texts=edition[class="edition"];
%    translation[class="translation"],
%  ]
%\begin{edition}
% \ekddiv{type=ed}
%\begin{prose}homa\end{prose}
%\end{edition}
%\begin{translation}
%  \ekddiv{type=trans}
%  \begin{tlate}\end{tlate}
%   \end{translation}
% \end{alignment}
%\begin{otherlanguage}{english}
\chapter{Translation of the Yogatattvabindu}    
\ekddiv{type=trans}
\centerline{\textrm{\small{[Introduction]}}}
\bigskip
\begin{tlate}
Homage to Śrī Gaṇeśa. Now the methods of Rājayoga are laid down. This is the result of Rājayoga\footnote{This statement seems unconnected to the definition of rājayoga that follows.}: Rājayoga is that by which longterm durability of the body arises even amongst manifold royal pleasures even amongst the manifold royal entertainments and spectacle. This truly is Rājayoga. These are the varieties of this Rājayoga:
\noindent 1. Kriyāyoga, the Yoga of [mental] action; 2. Jñānayoga, the Yoga of knowledge; 3. Caryāyoga, the Yoga of wandering;\footnote{The first three Yogas allude to the four \textit{pāda}s of the Śaiva \textit{āgama}s; namely \textit{kriyā[pāda], caryā[pāda], yoga[padā]} and \textit{jñāna[pāda]}.\parencite[77]{nishvasa2015}.} 4. Haṭhayoga, the Yoga of force; 5. Karmayoga, the Yoga of deeds; 6. Layayoga, the Yoga of absorption; 7. Dhyānayoga, the Yoga of meditation, 8.Mantrayoga, the Yoga of Mantras; 9. Lakṣyayoga, the Yoga of fixation objects, 10. Vāsanāyoga, Yoga of mental residues; 11. Śivayoga, the Yoga of Śiva, 12. Brahmayoga, the Yoga of Brahman; 13. Advaitayoga, the Yoga of non-duality; 14. Siddhayoga, the Yoga of the Siddhas; 15. Rājayoga, the King of Yogas. These are the fifteen \textit{yoga}s.\footnote{At the current stage of research it is not clear if this list is a later addition by another scribe or, if indeed it originally stems from Rāmacandra. The list suggests a text following the order of yogas according to this list. However, the order of the yogas given in the list is not followed closely in the text.}
\end{tlate}
%%%%%%%%%%%%%
 \begin{tlate}
   \ekddiv{type=trans}
      \centerline{\textrm{\small{[Description of \textit{kriyāyoga}]}}}
      \bigskip
Now the characteristic of Kriyāyoga, the Yoga of [mental] action\footnote{In comparison to the Pātañjalean variant of Kriyāyoga, this variat consists of specific mental actions.} are described.
\paragraph{1.} This Yoga is liberation through [mental] action, it bestows success(\textit{siddhi}) in ones own body. Each wave the mind creates at the beginning of an action, of all those one shall withdraw oneself. Then Kriyāyoga ari
\paragraph{2.} Patience, discrimination, equanimity, peace, modesty, desireless: The Yogī who is endowed with these means is said to be a Kriyāyogī. 
\paragraph{3.} Envy, selfishness, cheating, violence, desire and intoxication, pride, lust, anger, fear, laziness, greed, error and impurity. 
\paragraph{4.} Attachment and aversion, indignation and idleness, impatience and dizzyness: Whoever doesn't experience these is called a Kriyāyogī.\footnote{The source of the four verses on Kriyāyoga is unknown.} \bigskip \bigskip

Patience, discrimination, equanimity, peace, contentment etc. are generated in his mind. He alone is called a Yogī of many actions (\textit{bahukriyāyogī})\footnote{The term \textit{bahukriyāyogī} seems to be unique in yoga literature.} Fraud, illusion, property,violence, craving, envy, ego, anger, anxiety, shame, greed, error, impurity, attachment, aversion, idleness, heterodoxy, false view, affection of the senses, sexual desire: He who diminishes these from day to day in is mind, he alone is called a Yogī of many actions (\textit{bahukriyāyogī}).
   \end{tlate}
   %%%%%%%%%%%%%%%%%%%%
   %%%%%%%%%%%%%%%%%%%%
   %%%%%%%%%%%%%%%%%%%%
   %%%%%%%%%%%%%%%%%%%%
  \ekddiv{type=trans}
        \bigskip
    \centerline{\textrm{\small{[Varieties of \textit{rājayoga}: Siddhakuṇḍalinīyoga and Mantrayoga]}}}
    \bigskip
    \begin{tlate}
      Now varieties of Rājayoga will be described. Which are these? One is Siddhakuṇḍalinīyoga\footnote{On the one hand it suprises that we find the term Siddhakuṇḍalinīyoga instead of Siddhayoga as given in the initial list, on the other hand it is suprising that this type of Yoga, given as the second last item in the Yoga taxnomy is introduced as the second type right after Kriyāyoga, which was the first item in the initial list as well as in the following material.What makes this term even more strange is the fact that \textit{kuṇḍaliṇī} is not mentioned at all in the following description of this type of Yoga.} [and one\footnote{It is not entirely clear if those are two different Yogas or one and the same type of Yoga. Just the pretty late witness U2 gives us a sort of description of Mantrayoga. Judging on the basis of U2 only one could translate ``One is Siddhakuṇḍalinīyoga being Mantrayoga.'' Judging by the contents given by the rest of the witnesses this passage leaves a big queastion mark.}] is Mantrayoga\footnote{It seems odd that Mantrayoga is mentioned in the same breath as Sidhdakuṇḍalinīyoga, even though it is not directly expressed in the following. Just the additional descriptions of witness U2, highlighted in a different colour than the main text, indirectly refers to a certain practice of Mantra which is \textit{japājapa} of the \textit{so 'haṃ} for a certain duration of the practioce of meditation that is presrcibed to be performed on every \textit{cakra}.}. These two Rājayogas are described [in the following]. At the location of the root-bulb exists one major vessel in the form of energy. This single vessel reaches to these openings which are \textit{iḍā}, \textit{piṅgalā} and \textit{suṣumnā}. On the left side is the \textit{iḍā}-channel, being a resemblence of the moon. On the right side exists the \textit{piṅgalā}-channel, being a resemblence of the sun. Within the middle path is a lotuspond being very subtle. [It is] made from a web of light [and it] shines like a thousand lightnings. She \extra{emerges as the central channel, assuming the form of benevolence (\textit{śiva}),} is the bestower of enjoyment and liberation. While abiding in (\textit{satyāṃ}) her (\textit{asyāṃ}) knowledge arises [to the point of which] the person becomes all-knowing.
    \end{tlate}
    %%%%%%%%%%%%%%%%%%%
    %%%%%%%%%%%%%%%%%%%
    %%%%%%%%%%%%%%%%%%%
    %%%%%%%%%%%%%%%%%%%
     \ekddiv{type=trans}
      \bigskip
    \centerline{\textrm{\small{[Description of the first Cakra]}}}
    \bigskip
    \begin{tlate}
      The means for the genesis of knowledge in the central channel will now be described. At the beginning\footnote{Supposedly at the beginning of the central channel.} exists the root \textit{cakra} having four petals. \extra{The first \textit{cakra} of support (\textit{ādhāra}) is at the anus [and] is red-colored. Gaṇeśa is the deity. He is success, intelligence and power. A rat is the mount. The Ṛṣi is Kūrma. The seal is contraction. The vitalwind is \textit{apāna}. The \textit{kalā} is the ``wave of consciousness'' (\textit{urmī}). The concentration is ``she who is powerful'' (\textit{ojasvinī}). In the four petals [of it resides] \textit{rajas}, \textit{sattva}, \textit{tamas} and the mind-faculties (\textit{manāṃsi}), [symbolized by the syllables or \textit{bīja}s] vaṃ śaṃ ṣaṃ and saṃ. A trident is situated in the middle of the triangle\footnote{This passage is odd since a triagle wasn't mentioned before.}.} In the middle is a trident, and \textit{kāmapīṭha}\footnote{Discuss the term \textit{kāmapīṭha}.} in the shape of a triangle. In the middle of this seat (\textit{pīṭha}) exists a single form in the shape of a flame. By meditating on this form the whole literature, all \textit{śāstra}s, all poems, dramas etc., everything [related to] elocution, appears in the mind of the person without learning. \extra{[Assigned to it] is external bliss\footnote{Discuss the four blisses.}, yogic bliss, heroic bliss [and] the bliss of coming to rest.}\footnote{It is noteworthy that only the first \textit{cakra} adds a detailled description of mounts, Ṛṣis, gods, seals and so forth among the current majority of witnesses at hand: E, P, L and U2. All other descriptions of the remaining eight \textit{cakra}s leave this out. The only exception is U2, a relatively late witness that adds similar descriptions for the other \textit{cakra}s as well. Since they are interesting for the history of the text I have added them to the edition's text. To indicate the extra status of those passages I have highlighted them in blue color.} An [over] hundredfold recitation of the non-recited [śataḥ = \ldots hundreds of?];  600 [repetitions for]; 9 \textit{ghaṭi}s [and] 40 \textit{palā}s.\footnote{Instructions for the duration of practice are found in all additions of U2 for each \textit{cakra}. It's not entirely clear if either the duration of meditation on the respective cakra, or the duration for the items in the list being visualised by the practitioner are meant here. However, to it seems to be done for the duration of 600 \textit{ajapājapa}, the ritualized repetition of the \textit{ajapā}, which is the voiceless uttering of the ``natural'' \textit{mantra} of the breath: \textit{so 'haṃ} - \textit{haṃ sa}. I suppose this means the practice is to be done for 600 in- and exhalations. The following part of the entry, namely ``\textit{ghaṭi} 9 \textit{palāni} 40'', probably refers to the exact time in which those 600 \textit{ajapājapa}s shall be performed. One \textit{ghaṭi} equals 1/60 of a day, which is 24 minutes. One \textit{pala} equals 1/60 of a \textit{ghaṭi} which is 24 seconds. This would equal 232 minutes or 3 hours and 52 minutes. Dividing the 600 \textit{ajapājapa}s by 232 minutes, this would result in a very slow frequence of breath of 2,586206897 in- and exhalations per minute.}
  \end{tlate}
    %%%%%%%%%%%%%%%%%%%
    %%%%%%%%%%%%%%%%%%%
    %%%%%%%%%%%%%%%%%%%
    %%%%%%%%%%%%%%%%%%%
  \ekddiv{type=trans}
      \bigskip
    \centerline{\textrm{\small{[Description of the second Cakra]}}}
    \bigskip
    \begin{tlate}
      Now the second, the six-petalled \textit{Svādhiṣṭānacakra} known as the seat of \textit{Uḍḍīyāṇa}\footnote{Discuss the term \textit{uḍḍīyāna}.}. \extra{The gender is the location. The color is yellow. The shine is yellow. \textit{Rajas} is the quality. The deity is Brahmā. The speech is \textit{vaikharī}\footnote{vaikharī f. in Kaśm. Śiv. °the 4. form of appearance of \textit{parā}, the empirical speech sound, Utpala's Ṭīkā to Śivadṛṣṭi 2, 7. [B.]― Schmidt p. 337. Welches Buch???} (\textit{vaikharī vāca}). The power is Sāvitrī. The mount is the goose. The \textit{Rṣi} is Vahaṇa. The appearance (\textit{prabhā} is the fire of love (\textit{kāmāgni}). The body is gross, The state is that of being awake. The Veda is Ṛg. The spiritual guide is the characteristic (\textit{liṅga}). The liberation is residing in the world of Brahma. The principle is pure level (\textit{śuddhabhūmikā}). The sphere is smell. The vitalwind is \textit{apāna}. The internal matrix [is]: vaṃ bhaṃ maṃ yaṃ raṃ laṃ. The external matrix: Kāmā ``she who is desire'', Kāmākhyā ``she who is the \textit{tīrtha} of \textit{Kāmākhyā}''\footnote{The Kāmākhyā is situated in Kāmarūpa on the Nīlakūṭa mountain in present day Assam. It's strange that it appears here, since Kāmarūpa appears already as the \textit{tīrtha} associated with the first \textit{cakra}.}, Tejasvinī ``she who is shining'', Ceṣṭikā ``she who is active'', Alasā ``she who is lazy'' [and] Mithunā ``she who is \textit{mithunā}''. A [more than] thousandfold recitation of the non-recited; 6000 [repetitions for]; 16 \textit{ghaṭi}s [and] 40 \textit{palā}s.\footnote{The practice is supposed to be done for the duration of 6000 \textit{ajapājapa}s divided into \textit{ghaṭi}s and 40 \textit{pala}s, resulting in 2320 minutes or 38,67 hours. Again this would result in a frequence of breath of 2,586206897 in- and exhalations per minute.}} In its middle exists extremely red glow. The adept becomes very handsome through meditation on it. \extra{He becomes one who is desired by young women.} The vital force increases from day to day.
    \end{tlate}
   %%%%%%%%%%%%%%%%%%%
    %%%%%%%%%%%%%%%%%%%
    %%%%%%%%%%%%%%%%%%%
    %%%%%%%%%%%%%%%%%%%
  \ekddiv{type=trans}
    \bigskip
    \centerline{\textrm{\small{[Description of the third Cakra]}}}
    \bigskip
    \begin{tlate}
      The third, a lotus with ten petals exists at the location of the navel. \extra{The colour is red (\textit{kapila}). Viṣṇu is the deity. Lakṣmī is the power. Vāyu is the Rṣi. Samāna is the vitalwind. The mount is Garuḍa. The deity is the suble body\footnote{Why another deity is given here?}. The state is sleep. The speech is the inaudible speech (\textit{madhyamāvāg})\footnote{<Śā, Ling>name of the speech which is inaudible and which is of the type of a thought without any definite presence of words making up the expression. Vkp I.143.<Abhyankar 1986: 300>}. The Veda is the Yajurveda. The [fire is the] southern fire. The liberation is ``proximity'' (\textit{samīpatā}).\footnote{What is this exactly?}. Viṣṇu is the characteristic of the teacher (\textit{guruliṅga}). The principle is water. The sphere is athmosphere (\textit{rajo viṣaya}). There are ten petals [and] ten matrices. [The] inner matrix: \textit{ḍaṃ ṭaṃ ṇaṃ taṃ thaṃ daṃ dhaṃ naṃ paṃ phaṃ}. The external matrix: Śānti ``she who peaceful'', Kṣamā ``she who is patient'', Medhā ``she who is insightful'', Tanayā ``the daughter'', Medhavinī ``she who is a learned teacher'', Puṣkarā ``she who is a lotus'', Haṃsagamanā ``she who moves like a swan'', Lakṣyā ``she who is the object aimed at'', Tanmayā ``she who is absorption'' and Amṛtā ``she who is immortality''. A [more than] thousandfold recitation of the non-recited; 6000 [repetitions for]; 16 \textit{ghaṭi}s [and] 40 \textit{palā}s.\footnote{Here we find the same instruction as in the previous description of the second \textit{cakra}. The practice is supposed to be done for the duration of 6000 \textit{ajapājapa}s divided into \textit{ghaṭi}s and 40 \textit{pala}s, resulting in 2320 minutes or 38,67 hours. Again this would result in a frequence of breath of 2,586206897 in- and exhalations per minute.}} In its middle exists a \textit{cakra} with five angles. In its middle is a single [divine] form. It's not possible to describe her shine with speech. Through the execution of meditation on this [divine] form the body of the person is going to be strong.
 \end{tlate}
   %%%%%%%%%%%%%%%%%%%
    %%%%%%%%%%%%%%%%%%%
    %%%%%%%%%%%%%%%%%%%
    %%%%%%%%%%%%%%%%%%%
         \ekddiv{type=trans}
       \bigskip
    \centerline{\textrm{\small{[Description of the fourth Cakra]}}}%%%%%See Jogpradipikaya Edition Page 163 
       \bigskip
         \begin{tlate}
           The fourth lotus having twelve-petals exists in the middle of the heart. \extra{[The] place of the Anāhatacakra is within the heart\footnote{This is redundant.}. The color is white. The quality is Tamas. The deity is Rudra. The power is Umā. The Ṛṣi is Hiraṇyagarbha. The mount is Nandi. The vitalwind is Prāṇa. The body is the cause of digits of light. The state is deep sleep. The speech is Paśyantī\footnote{Add footnote of entry in \textit{Tāntrikābhidhānakośa}.}. [The Veda is] Sāmaveda. The fire is the fire of the householder\footnote{Add explanation.}. The characteristic is Śiva. The level is the ability to attain everything on earth\footnote{Quote \textit{Tantrikābhidhānakośa}.}. The liberation is uniform [with the deity]. [There are] twelve petals, [and] twelve matrices: kaṃ khaṃ gaṃ ghaṃ ṇaṃ caṃ chaṃ jaṃ jhaṃ yaṃ taṃ [and] thaṃ. The external matrix: Rudrāṇī ``she who is Rudra's wife'', Tejasā ``she who is brilliant''\footnote{To be understood as \textit{tejasvinī}.}, Tāpinī ``she who is glow'', Sukhadā ``she who bestows happiness'', Caitanyā ``she who is consciousness'', Śivadā ``she who bestows grace'', Śānti ``she who is peaceful'', Umā ``she who is glorious'', Gaurī ``she who is beautiful'', Mātarā ``she who is bestowing the mother'', Jvalā ``she who is the flame'' [and] Prajvālinī ``she who is blazing''. A [more than] thousandfold recitation of the non-recited; 6000 [repetitions for]; 16 \textit{ghaṭi}s [and] 40 \textit{palā}s.\footnote{The \textit{ajapājapa} for this \textit{cakra} is to be performed 6000 times for a duration of 96 \textit{ghaṭi}s and 40 \textit{pala}s, resulting in 2320 minutes or 38.67 hours. Again this would result in a frequence of breath of 2,586206897 in- and exhalations per minute.}} Due to being made of [such an] intense light [the fourth lotus] is not in the range of sight. In its middle exists a lotus facing downward having eight petals. \extra{The mind resides in the \textit{cakra}. The mind is the deity. The power is external\footnote{n Muktabodha-Texte sehe ich 3 Belege für bahiśśakti Muktabodha/krīyakramādyotikā.html 2938 suṣirānte bahiśśaktiṃ vinyasedvyomarūpiṇīm | tasyā madhye tu Muktabodha/sakalāgamasārasaṅgraha.html 2186 suṣirāntabahiśśaktiṃ vyāpinīṃ cintayet tataḥ || Muktabodha/kriyakramadyotikavyākhyā.html 1846 tanmadhye ca bahiśśaktiṃ sudhābindu parisrutim}, [its] Ṛṣi is the self. In the middle of the navel exists a lotus. Its stalk measures ten \textit{aṅgula}s. The stalk of it is soft (\textit{komala}), pure [and] facing downwards. In its middle is [something] shining like a banana-flower. The mind has no determination of will, giving a firmer direction to man's thoughts for the moment by means of [conscious] submission. [It is] truly changeable in nature.} \extra{While the mind rests on the eastern petal [which is] white in colour clear intellekt arises, which is [endowed with] \textit{dharma}, fame and knowledge etc. While [the mind rests on] the south-east, [which is] reddish in color a mind that is weak due to sleep, laziness and illusion arises. While [the mind is situated] in the right south, [which is] black in color the generation of anger arises. While [the mind is situated] in the southwest, [which is] blue in color a mind of pride arises. While [the mind is situated] in the west, [which is] brown in color a mind that is longing for play, laughing, and celebration arises. While [the mind is situated] in the northwest, [which is] dark in color a mind which is restless by sorrow arises. While [the mind is situated] in the north, [which is] yellow in color a very happy mind with erotic and enjoyment arises. While [the mind is situated] in north-east [which is] whitish in color a mind of unity through knowledge arises.}
           
It's said that in its middle is the place of the \textit{prāṇa}-vitalwind [and] in the middle [of] the eight-petalled lotus is a pericarp (\textit{karṇikā}) in the form of a \textit{liṅga}. The technical designation of her is \textit{kalikā}. In the middle of this \textit{kalikā} exists a single thumbsized [divine] figurine (\textit{puttalikā}) being similiar to a ruby-gem in color. Her technical designation is embodied soul (\textit{jīva}). Not even with a thousand tongues it is possible to talk about her nature and her power. Here it is said [that]: ``Because of the exercise of meditation on this form the inhabitants of the universe [which are] Humans, Gandharvas, Kinnaras, Guhyakas, Vidyādharas and [their] females, in the heavenly world, underworld and open space are obedient to the will of the practicing person.''.
  \end{tlate}
    %%%%%%%%%%%%%%%%%%%
    %%%%%%%%%%%%%%%%%%%
    %%%%%%%%%%%%%%%%%%%
    %%%%%%%%%%%%%%%%%%%
  \ekddiv{type=trans}
      \bigskip
    \centerline{\textrm{\small{[Description of the fifth Cakra]}}}
    \bigskip
    \begin{tlate}
Now the fifth lotus having sixteen petals existing at the location of the throat. \extra{The colour is grey. The deity is the embodied soul (\textit{jīva}). The power is ignorance (\textit{avidyā}). The Ṛṣi is Virāṭ\footnote{Who is this?}. The mount is the wind (\textit{vāyu}). The vitalwind is \textit{udāna}. The digit (\textit{kalā}) is the flame. The binding (\textit{bandha}) is Jālandhara. The body is the primordial cause (\textit{mahākāraṇa}). The state is the fourth state (\textit{tūrya}). The speech is Parā\footnote{Im Kaśm. Śiv. °das ewige Wort, in welchem potentiell alle Begriffe und Worte ruhen; vgl. das śabdabrahma des Vyākaraṇa. [B.]― Schmidt S. 246}. [The Veda is the] Atharvaṇa Veda. The movable characteristic (\textit{jaṅgamaṃ liṅgaṃ}). The earth is Jīvaprāptā\footnote{What is this?}. The liberation is union with the deity (\textit{sāyujyatā}). [There are] sixteen petals [and] sixteen matrices. The internal matrix: aṃ āṃ iṃ īṃ u ūṃ ṛṃ ṝṃ ḷṃ ḹṃ eṃ aiṃ oṃ auṃ aṃ aṃḥ. The external matrix: Vidyā ``she who is knowledge'', Avidyā ``she who is ignorance'', Icchā ``she who is desire'', Śakti ``she who is power'', Jñānaśakti ``she who is the power of knowledge'', Śatalā ``she who is manifold'', Mahāvidyā ``she who is great knowledge'', Mahāmayā ``she who is great illusion'', Buddhi ``she who is intellect'', Tāmasī ``she who is darkness'', Maitrā ``she who is love'', Kumārī ``she who is a young girl'', Maitrāyaṇī ``she who is onb the path of benevolence'', Rudrā ``she who is howling'', Puṣṭā ``she who is abundance'', Siṃhanī ``she who is a lioness''. A thousandfold recitation of the non-recited; 1000 [repetitions for]; 2 \textit{ghaṭi}s, 46 \textit{palā}s. and 40 \textit{akṣara}s.\footnote{It is not entirely clear what kind of measure an \textit{akṣara} is. Maybe see Amanaska 1. Chapter second half in thesis of Jason to clear things up.}} In its middle exists a single person which shines like a thousand moons. Because of the exercise of meditation on this person all diseases which are (otherwise) not possible to be controlled vanish. The person lives up to 1001 years.
    \end{tlate}
   %%%%%%%%%%%%%%%%%%%
    %%%%%%%%%%%%%%%%%%%
    %%%%%%%%%%%%%%%%%%%
    %%%%%%%%%%%%%%%%%%%
\ekddiv{type=trans}
    \bigskip
    \centerline{\textrm{\small{[Description of the sixth Cakra]}}}
    \bigskip
    \begin{tlate}
      Now it exists a sixth \textit{cakra} named Ājñā. \extra{The deity is fire (\textit{agni}). The power is the godess of the centre (\textit{suṣumṇā}). The Ṛṣi is ``the violent'' (\textit{hiṃsa}). The mount is consciousness (\textit{caitanya}). The body is knowledge. The state is understanding. The speech is the ``incomparable'' (\textit{anupama}). The [Veda] is Sāmaveda. The \textit{liṅgaṃ} is intoxication (\textit{pramāda}). The half-matrix: the principle of ether. The gander is the living soul. The origin is the play of conciousness. Twofold matrix: haṃ kṣam is the inner matrix. The external matrix: Sthiti ``she who maintains'' [and] Prabhā ``she who is splendour''. A thousandfold recitation of the non-recited; 1000 [repetitions for]; 2 \textit{ghaṭi}s, 46 \textit{palā}s, and 40 \textit{akṣara}s.\footnote{It's not entirely clear what kind of measure is an \textit{akṣara}.}} This \textit{cakra} is located in the middle of the eyebrows and is two-petalled. In its middle exists a certain object being a form of blazing fire without parts, not being female not being male. Because of the exercise of meditation on it the body of the person becomes non-aging and immortal.
    \end{tlate}
   %%%%%%%%%%%%%%%%%%%
    %%%%%%%%%%%%%%%%%%%
    %%%%%%%%%%%%%%%%%%%
    %%%%%%%%%%%%%%%%%%%
 \ekddiv{type=trans}
    \bigskip
    \centerline{\textrm{\small{[Description of the seventh Cakra]}}}
    \bigskip
    \begin{tlate}
Now the seventh cakra having 64 petals and being full of nectar exists in the middle of the palate. \extra{The forehead is the Maṇḍala. The moon is the deity. The power is the nectar of immortality. The Rṣi is the supreme self. The seventeenth digit is the resident with the nectar of immortality. The wavy stream of nectar is great space. The uvula is the mother. The ornament/rhythm? (\textit{tālikā}) is a small bell. The own form of the body is the unspeakable Gāyatrī, [which has] the face of a crow, the eye of a human, the horn of a cow, a forehead that is Brahmapaṭhā?, a neck like a horse, the face of a peacock [and] limbs like a goose. [This is] the specific nature of the unspeakable Gayatrī.}    
  It is endowed with superabundant beauty. [It is] very bright. In its middle, red in color [is that which is] known as "uvula" (\textit{ghāṃṭikā}). [It] exists as a single pericarp. In its middle is a [certain] site. In the middle of it exists a hidden digit of the moon, being a stream of nectar like a river (\textit{amṛtādhārāsravantī}). Because of the exercise of meditation on this digit death does not come near him. Due to uninterrupted meditation, the stream (\textit{dhārā}) of nectar flows. Then the appearances of emaciation (\textit{kṣayaroga}), fever due to disordered bile (\textit{pittajvara}), heartburn (\textit{hṛdayadāha}), head-disease (\textit{śiroroga}) and tongue insensibility (\textit{jihvājaḍa}) vanish. Also eaten venom doesn't trouble him. If the mind is here, [it] becomes stable.     
  \end{tlate}
  %%%%%%%%%%%%%%%%%%%
    %%%%%%%%%%%%%%%%%%%
    %%%%%%%%%%%%%%%%%%%
    %%%%%%%%%%%%%%%%%%%
 \ekddiv{type=trans}
    \bigskip
    \centerline{\textrm{\small{[Description of the eighth Cakra]}}}
    \bigskip
          \begin{tlate}
Now exists the eigth \textit{cakra} having one hundred petals located at the aperture of Brahman. \extra{The teacher is the deity. Consciousness is the power. Virāṭ is the Ṛṣi, the witness above everything. Made of consciousness is that which is associated with (\textit{bhūta°}) the state beyond the fourth state. It has all colours. It has all matrices. It has all petals. The body is Virāṭ. The state is the standing still. The speech is wisdom.  The "I am that"-[expression] (\textit{sohaṃ}) is the Veda. The place is unsurpassed. A thousandfold recitation of the non-recited; 1000 [repetitions for]; 2 \textit{ghaṭi}s, 46 \textit{palā}s. and 40 \textit{akṣara}s.\footnote{It's not entirely clear what kind of measure is an \textit{akṣara}.} The count is all silent mutterings, [being] 21600. In this way it carries on day and night. He who knows the breath is a learned person. With the sound "sa" he exhales, with the sound "ha" he inhales again: "I'm he, he's I". Because of that the embodied soul constantly utters the Mantra.\footnote{Add intertextual evidence.}} ``The (divine) seat of  Jālaṃdhara'' is the designation of its lotus.\footnote{Find parallels where Jālandhara is situated on top of the head.} [It is] the place of the accomplished person. In its middle looking like a streak [and] having the form of smoke and fire, exists such a single [divine] form of the person (\textit{puruṣa}). Of her exists no end, nor a beginning. Due to the exercise of meditation on this [divine] form both coming and going of the person in space occurs. Affliction from the earth-element doesn't arise [anymore] even if one is situated in the middle of the earth. He constantly sees everything in front of his eyes and he becomes separated [from the material world]. The force of life increases eminently.    
     \end{tlate}
    %%%%%%%%%%%%%%%%%%%
    %%%%%%%%%%%%%%%%%%%
    %%%%%%%%%%%%%%%%%%%
    %%%%%%%%%%%%%%%%%%%
\ekddiv{type=trans}
       \bigskip
    \centerline{\textrm{\small{[Description of the ninth Cakra]}}}
    \bigskip
    \begin{tlate}
Now the divisions/differentiations of the ninth cakra are explained. The designation of it is ``the \textit{cakra} of the great void''. Above that there is no other. Therefore it is declared to be the \textit{cakra} of the great perfection. [Another] such name of it is ``(divine) seat of Pūrṇagiri''. In the middle of the \textit{mahāśūnyacakra} exists one lotus facing upward, very red in colour, with a thousand petals - an abode of brilliance and wholeness, whose fragrance is not in range of mind and speech. In the middle of this lotus exists one pericarp having the shape of a triangle. In the middle of the pericarp exists one seventeenth digit in the shape of a immaculé form. A light of the part exists shining like a thousand suns. [But] excessive heat is not arising. Shining like a thousand moons, excess of cold is not arising. \extra{Here at this location the ``I''(\textit{aham}) is the deity. The ``he is I'' (\textit{so 'ham}) is the power. This self is the Ṛṣi. The path is liberation. Brahma is the I above. ``I'm a circle''. In fire-area is the letter "sa". [There?] life arises, the living soul ascends and decends. The place is the hidden place of being. The colour is yellow. The light is the shine of ten million suns. The shine is always and visible. Śiva is the deity. The power is primordial illusion. The state is the dissolution of the self into Hara\footnote{Epiphet of Śiva.}. The transcendental sound has the nature of a sound with stable resonance. The seal is the ``fearless''. The illusion is the root. The body is the original matter. The range is speech and mind. Without delusion, without doubt, the unaffected and undefiled goal is dissolution, meditation [and] final absorption.} Above that is the place of infinite supreme bliss. There above is power (\textit{śakti}). Being designated as such she is one single digit. Due to the exercise of meditation on this part, the person manifests whatever he wishes for. He is furnished with royal pleasure and enjoyment. [Even] amusing oneself amongst women, and watching musical pleasures, the \textit{kāla} of the person grows daily like the \textit{kalā} of the moon in the bright half of the month. His body is not affected by merit and sin. Due to uninterrupted meditation the power of the light of the innate nature arises. He sees remotely located objects as if they'd be near.
\end{tlate}
    %%%%%%%%%%%%%%%%%%%
    %%%%%%%%%%%%%%%%%%%
    %%%%%%%%%%%%%%%%%%%
    %%%%%%%%%%%%%%%%%%%
  \ekddiv{type=trans}
     \bigskip
    \centerline{\textrm{\small{[Lakṣyayoga, the yoga of fixation]}}}
    \bigskip
 \begin{tlate}
   Now the yoga of fixation (\textit{lakṣyayoga}), which is easily accomplished is explained. Of this yoga of fixation there are five subdivisions:
   1. The upward directed fixation (\textit{ūrdhvalakṣya}),
   2. the downward directed fixation (\textit{adholakṣya}),
   3. the outer fixation (\textit{baḥyalakṣya}),
   4. the central fixation (\textit{madhyalakṣya}),
   5. the inner fixation (\textit{antaralakṣya}).
 \end{tlate}
   %%%%%%%%%%%%%%%%%%%
    %%%%%%%%%%%%%%%%%%%
    %%%%%%%%%%%%%%%%%%%
    %%%%%%%%%%%%%%%%%%%
  \ekddiv{type=trans}
     \bigskip
    \centerline{\textrm{\small{[1. Ūrdhvalakṣya - The upward directed fixation]}}}
    \bigskip    
  \begin{tlate}
At first the upward directed fixation (\textit{ūrdhvalakṣya}) is explained. The gaze (\textit{dṛṣṭi}) [should be] in the middle of the sky. And then having caused the mind to be directed upwards, it is caused to be fixed there. Due to the exercise of stabilizing of this fixation (\textit{lakṣya}) arises unity of the gazing point (\textit{dṛṣṭi}) with the light of the highest lord (\textit{parameśvara}). And then an indefinable invisible object arises in the middle of the sky. It arises in the range of sight of the practitioner. This is truly the upward directed fixation (\textit{ūrdhvalakṣya}).
  \end{tlate}
   %%%%%%%%%%%%%%%%%%%
    %%%%%%%%%%%%%%%%%%%
    %%%%%%%%%%%%%%%%%%%
    %%%%%%%%%%%%%%%%%%%
\ekddiv{type=trans}
   \bigskip
    \centerline{\textrm{\small{[2. Adholakṣya - The downward directed fixation]}}}
    \bigskip
  \begin{tlate}
    Now the downward directed fixation object (\textit{adholakṣya}). One should stabilize the gaze within the circumference (\textit{paryanta}) of twelve \textit{aṅgula}s beyond the nose. Or one should stabilize the gaze onto the tip of the nose. The fixation becomes stable due to firm exercise [on one] of the twofold aims [of fixation]. The breath becomes stable. Vitality increases.
\end{tlate}
 %%%%%%%%%%%%%%%%%%%
    %%%%%%%%%%%%%%%%%%%
    %%%%%%%%%%%%%%%%%%%
    %%%%%%%%%%%%%%%%%%%
\ekddiv{type=trans}
   \bigskip
    \centerline{\textrm{\small{[3. Bāhyalakṣya - The external fixation]}}}
    \bigskip
  \begin{tlate}
    Just as this [aim] is twofold, also the external fixation is said to be [like this]. Internally or externally the aim of fixation is to be done onto the heavenly emptiness. The fear of dying doesn't arise due to the exercise of meditation on the void at all places during ones life - while eating, moving and waking.\footnote{Note that the description of the five types of Lakṣyayoga stops here and the new topic about the body of the Rājayogin is introduced. However, the subject is resumed later on in the text. Even though all witnesses follow this specific and suprising order. Maybe a copist in the early stages of transmission of the text copied the text without noticing the folios of his template to be in the wrong order.}
  \end{tlate}
   %%%%%%%%%%%%%%%%%%%
    %%%%%%%%%%%%%%%%%%%
    %%%%%%%%%%%%%%%%%%%
    %%%%%%%%%%%%%%%%%%%
  \ekddiv{type=trans}
    \bigskip
    \centerline{\textrm{\small{[Description of the Rājayogin's Body]}}}
    \bigskip
      \begin{tlate}
Now it is said that this is the characteristic of the embodied person who is endowed with the royal yoga: Abundance arises at all times. No distance exists on earth. He dwells on earth having pervaded [it]. Birth and death both don't exist. Happiness does'nt exist. Suffering does'nt exist. Impediment does'nt exist. Habit doesn't exist. Place does'nt exist. The manifestation of permanent perception of the connection with god arises in the middle of the mind of this accomplished one. And he is shining - not cold, and not hot, not white [and] not yellow. Neither is there birth of him, nor does he have any attributes. And he is without parts, immacule and uncharacterized. His desire etc. doesn't arise in [situations of] lust [and] is not located within the duality of the result. He attains expanded enjoyment. However, his mind does not suffer attachment in this very state.     
    \end{tlate}
    %%%%%%%%%%%%%%%%%%%%%
    %%%%%%%%%%%%%%%%%%%
    %%%%%%%%%%%%%%%%%%%
    %%%%%%%%%%%%%%%%%%%
    %%%%%%%%%%%%%%%%%%%
 \ekddiv{type=trans}
    \bigskip
    \centerline{\textrm{\small{[Other Attributes]}}}
    \bigskip
  \begin{tlate}
    Another attribute of Rājayoga is described. Even ``of one who is in gain of a kingdom etc.'' [it is said that] perception of success does'nt arise. Even due to loss suffering does'nt arise in the mind. And then desire doesn't arise. And then with regards to an object that has been obtained for whatever reason towards ones object aversion does'nt arise. With regard to this object affection of the mind does'nt arise. Just this is said to be Rājayoga. And then his mind which knows the sacred speech is equal towards a person, friend and enemy. And a neutral view arises. In the mind of one who is entirely situated in the middle of the earth, the pride of authorship does't arise, because of death and rebirth, and because of happiness and enjoyment. Wile wandering the world he doesn't whish to know authorship. This is also said to be Rājayoga. New durable clothes made of silk, or however, old, worn [clothes] with holes smeared with sandalwood and musk, or smeared with mud. In whose mind joy and sorrow are not situated, just he is [in the state of] Rājayoga. Just he is in the state of Rājayoga for whom the mind is neither in abundance nor in lack, being located in a city, a forest, an uninhabited village or a village full of people.    
  \end{tlate}
    %%%%%%%%%%%%%%%%%%%%%
    %%%%%%%%%%%%%%%%%%%
    %%%%%%%%%%%%%%%%%%%
    %%%%%%%%%%%%%%%%%%%
    %%%%%%%%%%%%%%%%%%%
\ekddiv{type=trans}
      \bigskip
    \centerline{\textrm{\small{[Description of Caryāyoga]}}}
      \bigskip
     \begin{tlate}
        Now \textit{caryāyogaḥ}, the Yoga of wandering is explained. Shapeless, unchangeable, permanent [and] unsplitable. Such is the self. It is seen as such by the one whose mind abides in the self without moving. His self is not touched by sin and merit. Just as the leave of the lotus situated in the amidst water doesn't touch the water; likewise the self [is not touched by sin and merit]. Just as the wind wanders according to its own will in space, likewise the mind of one who is absorbed into the universal spirit [wanders according to its own will in space]. This is \textit{caryāyoga}.
      \end{tlate}
    %%%%%%%%%%%%%%%%%%%%%
    %%%%%%%%%%%%%%%%%%%
    %%%%%%%%%%%%%%%%%%%
    %%%%%%%%%%%%%%%%%%%
    %%%%%%%%%%%%%%%%%%%
  \ekddiv{type=trans}
       \bigskip
    \centerline{\textrm{\small{[Description of Haṭhayoga]}}}
      \bigskip
      \begin{tlate}
        Now \textit{haṭhayoga}, the forceful Yoga is explained. The practice of breath shall be done in this manner: "Exhalation, Inhalation [and] Retention etc. And then due to the six practices (\textit{ṣaṭkarma}), like \textit{dhauti} etc. the purification of the body arises. When the full breath abides in the middle of the sun-channel. Then the mind is unmovable. The form of bliss immediately shines through the motionless mind. Due to the execution of Haṭhayoga the mind becomes absorbed into emptiness. The time of death does not approach. Now, the second division of Haṭhayoga is explained. The shine of ten million suns in one's own body beginning from the feet to the top of head is contemplated in any color equal to white, yellow [or] red. Due to the execution of meditation in the entire body disease does'nt arise, fever doesn't arise and vitality grows.
      \end{tlate}
  %%%%%%%%%%%%%%%%%%%%%
    %%%%%%%%%%%%%%%%%%%
    %%%%%%%%%%%%%%%%%%%
    %%%%%%%%%%%%%%%%%%%
    %%%%%%%%%%%%%%%%%%%
  \ekddiv{type=trans}
    \bigskip
        \centerline{\textrm{\small{[Description of \textit{Jñānayoga}]}}}
          \bigskip
    \begin{tlate}
      Now the characteristic of \textit{jñānayoga} is explained.      
  \paragraph{1.} He shall see the world truly as being one, shining in all selves. By applying indistinctness he shall accomplish \textit{Jñānayoga}.
  \paragraph{2.} Wherever the world is established or made of omniscience, who knows thus by means of insight, he is a like an expert of knowledge.
  \paragraph{3.} He always attains the reality of \textit{śāmbhavī} - the goal of eternal non-duality. Just as the seed of the Nyagrodha\footnote{In rituals, the nyagrodha (Ficus indica or India fig or banyan tree) danda, or staff, is assigned to the kshatriya class, along with a mantra, intended to impart physical vitality or 'ojas'.27. Brian K. Smith. Reflections on Resemblance, Ritual, and Religion, Motilal Banarsidass Publishe, 1998} scattered onto the soil [always] becomes a tree.
  \paragraph{4.} The absolute unity (\textit{ekāntaṃ}), is seen as multibel (namely) made up of ten parts by oneself. The rolled up shoots of the branches are the sprouting stalks of the root shoot.
  \paragraph{5.} By virtue of its inherent nature, this branch with its branches, which is the fruit of the flower of love, is in the seed. Certainly, that is pure, eternal, unchanging and immaculate. 
  \paragraph{6.} One, not one and self-existing, existing in manifold ways through its own rule and work, [as] five principles (\textit{tattva}) which are: thinking mind (\textit{manas}), intellect (\textit{buddhi}), illusion (\textit{māya}), individuation (\textit{ahaṃkāra}) and modifications (\textit{vikriyā}).
  \paragraph{7.}In this way, the ten variations fully permeate the world and the non-world. Only one thing is and not something else: Whoever knows this is a connoisseur of reality.
  
   Transmigration is the appearance of the plant world, mountains, trees, earth etc. Transmigration is the appearance of living beings beginning with birds, horses, elephants and humans. And then whoever is one who is a [sense] object of sight is said to be visible. He who is not seen by sight is said to be invisible. In this way the view of separation of one's own self which is subjected to transmigration is to be removed by means of [applying the view of] unity. Only this is Jñānayoga. Because of the execution of it, time does'nt destroy the body.
\end{tlate}
  %%%%%%%%%%%%%%%%%%%%%
    %%%%%%%%%%%%%%%%%%%
    %%%%%%%%%%%%%%%%%%%
    %%%%%%%%%%%%%%%%%%%
    %%%%%%%%%%%%%%%%%%%
\ekddiv{type=trans}
      \bigskip
        \centerline{\textrm{\small{[The Division of the Inherent Nature]}}}
          \bigskip
\begin{tlate}    
  Now the division of the inherent nature is described.\footnote{This refers to the mention of \textit{svabhāva} in verse 5 of the description of Jñānayoga.} Just as the seed of the banyan tree ripens into the shape of the banyan tree, and by its own inherent nature attains such a tenfold division. [Namely]: "Root, shoot, bark, branch, twig, bud, the unfolding flower, flower, fruit and nectar." The division reaches [those] ten parts. In this way, the pure, unchanging, unblemished, attains such [division] precisely because of the inherent nature of the self. [Namely] the division "Earth, Water, Fire, Wind, Space, Mind, Intellect, Illusion, Transformations and Form". Because of the power of Jñānayoga, there arises the certainty that "The Self is verily one." As some particular soil (\textit{ekaika}) sometimes appears soft, sometimes beautiful, sometimes fragrant, sometimes unscented, sometimes golden, sometimes silver, is sometimes made of precious stone, sometimes appearing white, sometimes black, sometimes copper, sometimes yellow, sometimes mottled, sometimes like various fruit, sometimes like flowers, sometimes like the nectar of immortality, [and that only] because of its inherent nature. In this way, the self also takes the form of a human, a bird, a gazelle, an elephant, a vidyādhara, a gandharva, a centaur, great scholar or a great fool, a sick or healthy, an angry or or peaceful person, by virtue of its inherent nature. Because of Jñānayoga, transformation is recognized as formless, Just as the place of origin of the fruit is only one. But the transformation of the fruit is seen as manifold.
  
  One fruit falls onto the ground. It is getting bright. A bee drinks the flower juice of a fruit. The lover [bee] places itself on the flower wreath above the protuberant circular pistil. A bee drinks the juice of a fruit. The lover (bee) places herself on the flower wreath above the upstanding circular pistil. ne fruit throws the nectar over the flower. This is the inherent nature of the matter. In the same way also the one self enjoys the eight pleasures because of its own being.  
\\ \\
What are the eight enjoyments?  \hfill \break
A beautiful dwelling, good clothing, a good bed, a well-trained horse?, a nice place, food and drink.\footnote{The verse only gives 7 enjoyments!} Those are the eight enjoyments of the wise.
\\ \\
  1. Clothes made from silk; \hfill \break
  2. A site of the palace in which there are mainsions endowned with five or seven rooms.\hfill \break
  3. A huge, very soft and lovely bed; \hfill \break
  4. [on which] there is seated a lotus-like youthful, charming and virtuous wife;\hfill \break 
  5. An excellent throne;\hfill \break
  6. An exceptional valuable horse; \hfill \break
  7. Food that pleases the senses; \hfill \break
  8. Various drinks. \hfill \break 

Like the rays of the sun, the butter of milk, the burning of fire, the stupor of poison, the sesame oil from the sesame seed, the shade from the tree, the sweet odor from a fruit, the fire from a scabbard, the sweet sap of Śārkara\footnote{A liquor prepared from Dhātakī with sugar.} and so on, the cold of piles of snow, and so on is the inherent essence of things. In the same way, the course of the world is also in the center of the highest God's own form. And the Most High God is indivisible and all-filling.
\end{tlate}
  %%%%%%%%%%%%%%%%%%%%%
    %%%%%%%%%%%%%%%%%%%
    %%%%%%%%%%%%%%%%%%%
    %%%%%%%%%%%%%%%%%%%
    %%%%%%%%%%%%%%%%%%%
\begin{tlate}
  \ekddiv{type=trans}
   \bigskip
        \centerline{\textrm{\small{[Continuation of \textit{Lakṣyayoga} - Bāhyalakṣya]}}}
          \bigskip
Now the external fixation is taught. Beginning with a four finger wide distance from the tip of the nose, the space[-element] full of light whose appearance is blue shall be made the object of fixation. Or, a six finger wide distance from the tip of the nose, the wind-element whose appearance is greyish shall be made the object of fixation. Or, an eight finger wide distance from the tip of the nose, the very red fire[-element] shall be made the object of fixation. Or, a ten finger wide distance from the tip of the nose, the white water[-element] being fickle shall be made the object of fixation. Or, a twelve finger wide distance from the tip of the nose, the yellow-colored earth-element shall be made the object of fixation. Or beginning at the tip of the nose\footnote{Given the clear instructions of the respective distance of the exercise in the previous sentences, it is surprising that this instruction is lacking the mention of the distance.} the space-element full of fire shining like ten million suns shall be made the object of fixation. After having fixed the gaze on the space[-element?] or above the space[-element?], due to the execution of meditation he sees the sun without the group of thousand rays related to the sun. Or the mass of light situated seventeen fingers wide distance above the head shall be made the fixation object. Diseases of the limbs are removed without medical herbs. All enemies become friends while sleeping. The lifespan increases up to 1000 years. 
\end{tlate}
  %%%%%%%%%%%%%%%%%%%%%
    %%%%%%%%%%%%%%%%%%%
    %%%%%%%%%%%%%%%%%%%
    %%%%%%%%%%%%%%%%%%%
    %%%%%%%%%%%%%%%%%%%
\begin{tlate}
  \ekddiv{type=trans}
     \bigskip
        \centerline{\textrm{\small{[Continuation of \textit{Lakṣyayoga} - Antaralakṣya]}}}
          \bigskip
Now the inner fixation objects are taught. At the location of the root bulp rising from the staff of Brahma up to the aperture of Brahma exists the one white coloured Brahma channel. The interior of the Brahma channel, which equals a pale-red string shining like 10 million suns, goes upwards. A particular manifestation exists as such. Due to the execution of meditation on this manifestation, the eight great supernatural powers of humans beginning with \textit{aṇima} etc.\footnote{Write something about siddhis.} become established after one has entered into [the manufestation's] imminence. Or from the execution of meditation onto the bright light at the centre within the space at the forehead diseases related to the body beginning with leprosy vanish. Lifeforce increases. Or because of executing meditation on the middle of the eyebrows [of which there is] a very subtle and red colored light, he is one who is beloved among all royal people. Having seen this person, everybody's gaze is fixed onto him.
\end{tlate}
  %%%%%%%%%%%%%%%%%%%%%
    %%%%%%%%%%%%%%%%%%%
    %%%%%%%%%%%%%%%%%%%
    %%%%%%%%%%%%%%%%%%%
    %%%%%%%%%%%%%%%%%%%
\begin{tlate}
  \ekddiv{type=trans}
 \bigskip
 \centerline{\textrm{\small{[The Ten Main Bodily Channels]}}}
 \bigskip
 Now the divisions of channels within the body are explained. There are ten primary channels. Among them exists the pair of channels designated Idā and Piṅgalā at the entrance of the nose. The central channel leads from the palate to the door of Brahma. The Sarasvatī[-channel] exists at the centre of the face. The two rivers Gāṃdhārī and Hastjihvā exist within the centre of the two ears. The two rivers Pūṣā and Ālaṃbuṣā are situated at the center of the two eyes. The Śaṃkhinī channel strechtes from the the beginning of the opening of the penis through the Iḍā-channel. In such a way the channels are situated at the 10 openings. The other channels measured as 72000 are situated with a subtle form at the roots of the hairs.
\end{tlate}
  %%%%%%%%%%%%%%%%%%%%%
    %%%%%%%%%%%%%%%%%%%
    %%%%%%%%%%%%%%%%%%%
    %%%%%%%%%%%%%%%%%%%
    %%%%%%%%%%%%%%%%%%%
 \begin{tlate}
  \ekddiv{type=trans}
\bigskip
 \centerline{\textrm{\small{[The Ten Vitalwinds]}}}
 \bigskip
 Now [there are] ten vitalwinds are situated within the body. The Prāṇa vitalwind is located in the middle of the heart and causes inhalation and exhalation. The wish for eating an drinking exists. At the center of the anus the Apāna-Vitalwind exists. He does contraction and checking. At the center of the navel the Samāna[-vitalwind] exists. He causes to dry up all the channels. In this way the channels are caused to thrive, beauty is caused to be generated and the fire is caused to light up. Within the throat the Udāna-vitalwind is situated. This wind swallows food, [and] it drinks water. The Nāga-vitalwind exists in the entire body. Through the vitalwind the body is caused to move. The Kūrma-vitalwind exists within the eyes. It causes [the] opening and closing [of the eyes]. From the Kṛkala-vitalwind gagging arises. From the Devadatta-vitalwind jawning arises. From the Dhanaṃjaya-vitalwind speech arises.
\end{tlate}
  %%%%%%%%%%%%%%%%%%%%%
    %%%%%%%%%%%%%%%%%%%
    %%%%%%%%%%%%%%%%%%%
    %%%%%%%%%%%%%%%%%%%
    %%%%%%%%%%%%%%%%%%%
\begin{tlate}
  \ekddiv{type=trans}
\bigskip
 \centerline{\textrm{\small{[Continuation of \textit{Lakṣyayoga} - Madhyalakṣya]}}}
 \bigskip
Now the central fixation is taught. White-colored, or yellow-colored or red-coloured or smoke-coloured or blue-coloured, like the flame of fire, equal to a lightning, like the orb of the sun, like a half-moon, appearing like flaming space, measured according to ones own body, the fixation shall be directed onto the center of the glowing mind. While abiding in this fixation the burning of the impurity in the center of the mind arises. The Sattva-quality of the mind becomes revealed. After this has happend, the person abides supreme bliss.  
\end{tlate}
  %%%%%%%%%%%%%%%%%%%%%
    %%%%%%%%%%%%%%%%%%%
    %%%%%%%%%%%%%%%%%%%
    %%%%%%%%%%%%%%%%%%%
    %%%%%%%%%%%%%%%%%%%
\begin{tlate}
  \ekddiv{type=trans}
 \bigskip
 \centerline{\textrm{\small{[The Divisions of Space]}}}
 \bigskip
 Now the divisions of space are taught.The fixations of them are taught: Space, beyond space, great space, space of reality, the space of the sun. The fixation onto the pure and formless space \textit{akāśa} shall be done internally as well as externally. Moreover, the fixation of the beyond-space \textit{parākāśa} which is equal to dense darkness shall be done internally and externally. Moreover, the fixation of the great space (\textit{mahākāśa}) which is the plethora of the burning fire of the time of dissolution shall be done internally and externally. Moreover, for whom internally and externally the brightness of millions of blazing lights arises, he shall execute the fixation [directed onto] the reality-space (\textit{tattvakāśa}). After that the fixation of the sun-space (\textit{sūryakāśa}) which is associated with sundisk's appearance of light shall be done internally and externally. From the execution of these fixations contact of diseases does not arise within the body. Thus wrinkles and grey hair, sin or merit do not arise. The nine cakras, the sixteen Adhāras, the three lakṣyas and die five spaces. Who does not know [them?] within ones own body, he is only a Yogin by name.
\end{tlate}
  %%%%%%%%%%%%%%%%%%%%%
    %%%%%%%%%%%%%%%%%%%
    %%%%%%%%%%%%%%%%%%%
    %%%%%%%%%%%%%%%%%%%
    %%%%%%%%%%%%%%%%%%%
\begin{tlate}
  \ekddiv{type=trans}
 \bigskip
 \centerline{\textrm{\small{[The order of Cakras]}}}
 \bigskip
Now the practice of the cakras is explained. At the pelvic floor there is the Brahmacakra. Above the pelvic floor at the root of the gender is the Svadiṣṭhānacakra. At the navel there is the Maṇipūrakacakra. At the heart the Anāhatacakra. Situated within the throat is the Viśuddhicakra. The sixth is the cakra of the palate. In the center of the eyebrows is the Ājñācakra. At the opening of Brahma is the Kālacakra. The ninth is the Ākāśacakra. It is supreme emptiness.
\end{tlate}
  %%%%%%%%%%%%%%%%%%%%%
    %%%%%%%%%%%%%%%%%%%
    %%%%%%%%%%%%%%%%%%%
    %%%%%%%%%%%%%%%%%%%
    %%%%%%%%%%%%%%%%%%%
\begin{tlate}
  \ekddiv{type=trans}
 \bigskip
 \centerline{\textrm{\small{[The sixteen Container]}}}
 \bigskip
 Now the  divisions of the container-\textit{cakra}s are taught.
 From the execution of the fixation onto the light at the big toes of the feet stability of the gaze arises.
 The root-container is the second [one]. The heel of the backfoot is caused to be placed at the root of the big toe. As a result the fire is strengthened.
 One heel is caused to be placed at the Root-container. The heel of the other foot is caused to be placed at the root of the big toe of this foot.
 The fire of it is caused to be kindled.
 The third is the place of the anus-container. From the execution of expansion and contraction a stable vitalwind arises. Additionally death of the person does not arise. Additionally the person does not die.
 The fourth is the penis-container. Due to the execution of repeated practice of contracting the penis in the midst of therof, the adamantine channel appears in the middle of the staff of the back. From the repeated practice again [and again] the transition of both breath and mind into its center arises. Caused by the transition of them both into the center the trinity of knots breaks. There, from the breaking of that, the vitalwind after having filled up (the central channel?) resides in the center of the Brahma-lotus. Then virility and strength arises. The person becomes youthful forever.
 The fifth is Udyāna. From performing \textit{bandha} there, urine and faeces disappear.
 The sixth is the navel-container. From repeated practice of \textit{praṇava}, the unstruck sound arises by itself.
 The seventh is the container of the heart-form.
 The throat-support is the eighth. There the contraction of Jālaṃdhara is produced. While abiding therein the vitalwind in the Iḍā-and Piṅgalā-channels becomes stable.
 The ninth is the container of the uvula. There, the tip of the tongue becomes attached [to the uvula]. Then the nectar of immortality flows from the immortality-digit. From drinking the nectar of immortality diseases do not spread in the body.
 The tenth is the container of the palate. After the moving and milking has been done therein while abiding at the door of the uvula, the tongue resides inserted within the palate.
 The eleventh is the tongue-container at the surface of the tongue. Within it the tip of the tongue has to be churned. While doing [that] a sweet drink flows out. And in that manner the knowledge of areas like poetry, singing, metric and dance is generated.
 On top thereof is the twelfth, being the teeth-support, which is situated inbetween the teeth. At this place the tip of the tongue is to be positioned with force for the duration of one and a half \textit{ghāṭī}s (24+12 = 36 minutes). Abiding therein the diseases of the practitioner will entirely disappear!
 The thirteenth is the nose-container. While making it into the fixation object the mind becomes stable.
 The fourteenth is the container of breath at the root of the nose. From the execution of stabilizing of the gaze onto this, the light of one's own becomes perceptible within 60 months.
 He breaks the mundane prison through the perceptibility of the light.
 The fifteenth container is situated in the middle of the eyebrows. Due to stabilizing the gaze therein 10 million rays of light sparkle.
 The sixteenth is the eye-container. Without wavering, the gaze [ayam] is to be held at the tip of the finger without wavering. From practicing this on earth any energy exists [for him]. The light of everything that is arises as the object of sight. From that sight the person becomes omniscient. 
\end{tlate}
  %%%%%%%%%%%%%%%%%%%%%
    %%%%%%%%%%%%%%%%%%%
    %%%%%%%%%%%%%%%%%%%
    %%%%%%%%%%%%%%%%%%%
    %%%%%%%%%%%%%%%%%%%
\end{otherlanguage} 
%\section{Bibliography}
% \label{sec:bibli}
%\printshorthands[keyword=critEd]
%\printbibliography[title=Consulted Manuskcipts, keyword=codex]
%\printbibliography[title=Printed Editions, keyword=printsource]
%\printbibliography[title=Secondary Literature, keyword=seclit]
%\end{document}

