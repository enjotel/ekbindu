\begin{ekdosis}
\ekddiv{type=ed}
    \centerline{\textrm{\small{[Introduction]}}}
    \bigskip
    \begin{prose}
%--------------------------
% śrī gaṇeśāya namaḥ /                                                     rājayogāntargataḥ //  binduyogaḥ   \E 
% śrī gaṇeśāya namaḥ /                                                     atha tattvabiṃduyogaprāraṃbhaḥ     \L
% śrī ṇe ya maḥ /                                                          atha rājayoga         liṣyate      \P
% \om                                                                                                         \B      
% śrī gaṇeśāya namaḥ // śrī gurave namaḥ //                                atha rājayogaprakāro  likhyate //  \N1
% śrī gaṇeśāya namaḥ //                                                //  atha rājayogaprakāro  likhyate //  \N2
% śrī gaṇeśāya namaḥ // śrī sarasvatyai namaḥ // śrī nirañjanāya namaḥ //  atha rājayogaprakāro  likhyate //  \D
% \om                                                                                                         \D2
% śrī gaṇeśāya namaḥ /  oṃ śrī niraṃjanāya //                              atha rājayogaprakāra  likhyate //  \U1
% śrī gaṇeśāya namaḥ /                                                     atha rājayoga         likhyate //  \U2
%--------------------------
%Homage to Śrī Gaṇeśa. Now the methods of rājayoga are laid down.
%--------------------------          
\noindent \app{\lem[wit={ceteri}]{śrī gaṇeśāya namaḥ}
        \rdg[wit={P}]{śrī ṇe ya maḥ}
        \rdg[wit={N1}]{śrī gaṇeśāya namaḥ || śrī gurave namaḥ ||}
        \rdg[wit={D}]{śrī gaṇeśāya namaḥ || śrī sarasvatyai namaḥ || śrī nirañjanāya namaḥ ||}
        \rdg[wit={U1}]{śrī gaṇeśāya namaḥ || oṃ śrī niraṃjanāya ||}}\dd{}
\app{\lem[wit={N1,N2,D}]{atha rājayogaprakāro likhyate}
        \rdg[wit={U1}]{atha rājayogaprakāra likhyate}
        \rdg[wit={E}]{rājayogāntargataḥ || binduyogaḥ}
        \rdg[wit={L}]{atha tattvabiṃduyogaprāraṃbhaḥ}
        \rdg[wit={P}]{atha rājayoga liṣyate}
        \rdg[wit={U2}]{atha rājayoga likhyate}}\dd{}
%-------------------------- 
% \om                        \E
% \om                        \L
% \om                        \B
% rājayogasyedaṃ phalaṃ      \P
% rājayogasya idaṃ phalaṃ    \N1
% rājayogasya idaṃ phalaṃ    \N2
% rājayogasya idaṃ phalaṃ // \D
% \om                        \D2
% rājayogasya idaṃ phalaṃ    \U1
% rājayogasyedaṃ phalaṃ /    \U2
%--------------------------    
\app{\lem[wit={P,U2}]{rājayogasyedaṃ phalaṃ}
  \rdg[wit={N1,N2,D}]{rājayogasya idaṃ phalaṃ}
  \rdg[wit={E,L}]{\om}}/
%--------------------------
%This is the result of \textit{rājayoga}:
%--------------------------
% \om                                                                                                                                                                         \E
% \om                                                                                                                                                                         \L
% \om                                                                                                                                                                         \B
% yena rājayogenāneka---rājyabhogasamaya   eva   anekapārthivavinodaprekṣaṇasamaya  eva   bahutarakālaṃ  śarīrasthitir  bhavati    sa eva  rājayogaḥ tasyaite     bhedāḥ      \P
% yena rājayogenāneka---rājyabhogasamaya   eva/  anekapārthivavinodaprekṣaṇasamaya  eva/  bahutarakālaṃ  śarīrasthitir  bhavati    sa eva  rājayogaḥ /  tasya ete bhedāḥ /    \N1
% yena rājayogena  anekarājyabhogasamaya   eva// anekapārthivavinodaprekṣaṇasamaya  eva   bahuttarakālaṃ śarīrasthitir  bhavati    sa eva  rājayogaḥ /  tasya ete bhedāḥ /    \N2
% yena rājayogena  anekarājyabhogasamaya   eva// anekapārthivavinodaprekṣaṇasamaya  eva// bahutarakālaṃ  śarīrasthitir  bhavati//  sa eva  rājayogaḥ // tasya ete bhedāḥ /    \D
% \om                                                                                                                                                                         \D2
% yena rājayogena  anekarājyabhogasamaya   eva// anekapārthivavinodaprekṣaṇasamaya  eva// bahutarakālaṃ  śarīrasthitir  bhavati    sa evaṃ rājayogaḥ    tasya ete bhedāḥ //   \U1 
% yena rājayogena  anekarājyabhogasamaya   eva// anekapārthivavinodaprekṣyaṇasamaya eva// bahutarakālaṃ  śarīrasthitir  bhavati//  sa eva  rājayogastaisyaite     bhedāḥ //   \U2
% --------------------------
%\textit{Rājayoga} is that by which longterm durability of the body arises even amongst manifold royal pleasures even amongst the manifold royal entertainments and spectacle. This truly is \textit{rājayoga}. Of this [\textit{rājayoga}] these are the varieties: \end{tlate}
%--------------------------
yena rāja\app{\lem[wit={P,N1}, alt={°yogenāneka°}]{yogenāneka}
  \rdg[wit={N2,D,U1,U2}]{°yogena aneka°}
}rājyabhogasamaya eva/ anekapārthivavinoda
      \app{\lem[wit={ceteri}]{prekṣaṇasamaya}
        \rdg[wit={U2}]{prekṣyaṇasamaya}}
      eva/ bahutarakālaṃ śarīrasthitir-bhavati/ sa
      \app{\lem[wit={ceteri}]{eva}
        \rdg[wit={U2}]{evaṃ}}
      \app{\lem[wit={ceteri}]{rājayogaḥ}
        \rdg[wit={U2}]{rājayogas}}/ 
      \app{\lem[wit={P,U2}]{tasyaite}
        \rdg[wit={ceteri}]{tasya ete}} bhedāḥ/
%-------------------------
%
% \om                                                                                                                                                                \E
% \om                                                                                                                                                                \L
% \om                                                                                                                                                                \B
% kriyāyogaḥ 1 jñānayogaḥ 2 caryāyogaḥ 3 haṭhayogaḥ 4 karmayogaḥ 5 layayogaḥ 6 dhyānayogaḥ 7 maṃtrayogaḥ 8 lakṣyayogaḥ 9 vāsanāyogaḥ 10 śivayogaḥ 11 brahmayogaḥ 12 advaitayogaḥ 13 siddhayogaḥ 14 rājayogaḥ 15 ete paṃcadaśayogāḥ \P
%
% kriyāyogaḥ / jñānayogaḥ / caryāyogaḥ / haṭhayogaḥ / karmayogaḥ / layayogaḥ / dhyānayogaḥ / maṃtrayogaḥ / lakṣyayogaḥ / vāsanāyogaḥ / śivayogaḥ / brahmayogaḥ / advaitayogaḥ / rājayogaḥ / siddhayogaḥ / ete paṃcadaśayogāḥ // \N1
%
% kriyāyogaḥ jñānayogaḥ caryāyogaḥ haṭhayogaḥ karmayogaḥ layayogaḥ dhyānayogaḥ maṃtrayogaḥ lakṣayogaḥ vāsanāyogaḥ śivayogaḥ brahmayogaḥ advaitayogaḥ rājayogaḥ siddhayogaḥ // ete paṃcadaśayogāḥ // \N2      
%      
% kriyāyogaḥ // jñānayogaḥ // caryāyogaḥ // haṭhayogaḥ // karmayogaḥ // layayogaḥ // dhyānayogaḥ // maṃtrayogaḥ // lakṣyayogaḥ // vāsanāyogaḥ // śivayogaḥ // brahmayogaḥ // advaitayogaḥ // rājayogaḥ // siddhayogaḥ // ete paṃcadaśayogāḥ // \D
% \om                                                                                                                                                                         \D2      
%
% kriyāyogaḥ // jñānayogaḥ // tvaryāyogaḥ // haṭhayogaḥ // karmayogaḥ // layayogaḥ // dhyānayogaḥ maṃtrayogaḥ  lakṣayogaḥ  vāsanāyogaḥ  śivayogaḥ  brahmayogaḥ  advaitayogaḥ  rājayogaḥ  siddhayogaḥ ete paṃcadaśayogāḥ  \U1
%
% kriyāyogaḥ // jñānayogaḥ // caryāyogaḥ // haṭhayogaḥ // karmayogaḥ // nayayogaḥ // dhyānayogaḥ // maṃtrayogaḥ // lakṣyayogaḥ // vāsanāyogaḥ // śivayogaḥ // brahmayogaḥ // advaitayogaḥ // siddhayogaḥ // rājayogaḥ // evaṃ paṃcadaśāyogā bhavaṃti // \U2
%-------------------------
         kriyāyogaḥ 1\dd{}
         jñānayogaḥ 2\dd{}
         \app{\lem[wit={ceteri}]{caryāyogaḥ}
          \rdg[wit={U1}]{tvaryāyogaḥ}} 3\dd{}
        haṭhayogaḥ 4\dd{}
        karmayogaḥ 5\dd{}
        \app{\lem[wit={ceteri}]{layayogaḥ}
          \rdg[wit={U2}]{nayayogaḥ}} 6\dd{}
        dhyānayogaḥ 7\dd{}
        mantrayogaḥ 8\dd{}
        \app{\lem[wit={ceteri}]{lakṣyayogaḥ}
          \rdg[wit={U1}]{lakṣayogaḥ}} 9\dd{}
        vāsanāyogaḥ 10\dd{}
        śivayogaḥ 11\dd{} 
        brahmayogaḥ 12\dd{}
        advaitayogaḥ 13\dd{} 
        \app{\lem[wit={P,U2}]{siddhayogaḥ}
          \rdg[wit={N1,N2,D,U1}]{rājayogaḥ}} 14\dd{}
        \app{\lem[wit={P,U2}]{rājayogaḥ}
          \rdg[wit={ceteri}]{siddhayogaḥ}} 15\dd{}     
        \note[type=philcomm, labelb=1, lem={rājayoga}]{The initial codification of 15 \textit{yoga}s appears in N\textsubscript{1}, N\textsubscript{2}, P, D, U\textsubscript{1} and U\textsubscript{2}. It is ommitted in E and L. B can't be determined due to missing folios. It is also missing in the Ysg.}
        \note[type=source, labelb=2, lem={pañcadaśāyogā}]{Ysv (PT): pañcadaśaprakāro 'yaṃ rājayogaḥ || kriyāyogo jñānayogaḥ karmayogo haṭhas tathā | dhyānayogo mantrayoga urayogaś ca vāsanā |  rājaty etad brahmavaśīva ebhiś ca pañcadaśadhā | idānīṃ lakṣaṇañ caiṣāṃ kathayāmi śṛṇu priye |}
        \note[type=testium, labelb=3, lem={pañcadaśāyogā}]{YSC: ity ādinā 'mnātāni | tatra nididhyāsanaṃ pradhānam | tat sahakṛtād eva manaso 'laukikā 'bādhitātmagocara pramāsambhavāt sarvavijñānādirūpaphalasaṃvādāc ca | nididhyāsanañcaika tānatādirūpo rājayogāparaparyāyaḥ samādhiḥ | tatsādhanaṃ tu kriyāyogaḥ, caryāyogaḥ, karmayogo, haṭhayogo, mantrayogo, jñānayogaḥ, advaitayogo, lakṣyayogo, brahmayogaḥ, śivayogaḥ, siddhiyogo, vāsanāyogo, layayogo, dhyānayogaḥ, premabhaktiyogaś ca |}
        \app{\lem[wit={P,N1,D,U1}]{ete pañcadaśayogāḥ}
          \rdg[wit={U2}]{evaṃ paṃcadaśāyogā bhavaṃti}}\dd{}
      \end{prose}
    \end{ekdosis}
    %%%%%%%%%%%%
    %%%%%%%%%%%%
    %%%%%%%%%%%
    %%%%%%%%%%%%%
    %%%%%%%%%%%%
\begin{ekdosis}
  \ekddiv{type=ed}
        \bigskip
        \centerline{\textrm{\small{[Kriyāyoga]}}}
        \bigskip
%--------------------------        
% \om                                      \E
% \om                                      \L
% \om                                      \B
% idānīṃ kriyāyogasya lakṣaṇaṃ kathyate/   \P
% idānīṃ kriyāyogasya lakṣaṇaṃ kathyate/   \N1
% idānī  kriyāyogasya lakṣaṇaṃ kathyate//  \N2
% idānīṃ kriyāyogasya lakṣaṇaṃ kathayate/  \D
% \om                                      \D2
% idānīṃ kriyāyogasya lakṣaṇaṃ kathyate/   \U1
% atha   kriyāyogas   lakṣaṇaṃ          // \U2
%--------------------------
%Now the characteristic of the Yoga of [mental] action (\textit{kriyāyoga}) described.
%--------------------------
 \begin{prose}
        \app{\lem[wit={ceteri}]{idānīṃ}
            \rdg[wit={N2}]{idānī}
            \rdg[wit={U2}]{atha}}
          \app{\lem[wit={ceteri}]{kriyāyogasya}
            \rdg[wit={U2}]{kriyāyogas}} lakṣaṇaṃ
          \app{\lem[wit={ceteri}]{kathyate}
            \rdg[wit={D}]{kathayate}
            \rdg[wit={U2}]{\om}}/
\end{prose}
 \ekddiv{type=ed}
 \begin{tlg}
%--------------------------   
% \om                                                    \E
% \om                                                    \L
% \om                                                    \B
% kriyāmuktir    ayaṃ yogaḥ    svapiṇḍe siddhidāyakaḥ    \P
% kriyāmuktir    ayaṃ yogaḥ /  svapiṇḍe siddhidāyakaḥ /  \N1
% kriyāmukti    layaṃ yogaḥ    svapiṇḍe siddhidāyakaḥ /  \N2
% kriyāmuktir    ayaṃ yogaḥ    svapiṇḍe siddhidāyakaḥ /  \D
% \om                                                    \D2
% kriyāyuktir    ayaṃ yogaḥ /  svapiṇḍe siddhidāyakaḥ /  \U1
% kriyāmuktiḥ // ayaṃ yogaḥ    svapiṇḍe siddhidāyakaṃ // \U2 
%--------------------------
%This Yoga is liberation through [mental] action, it bestows success(\textit{siddhi}) in ones own body.
%-------------------------- 
   \tl{\note[type=source, labelb=4, lem=kriyāmuktir]{Ysv (PT): kriyāmuktimayo [kriyāmuktir ayaṃ (YK)] yogaḥ sapiṇḍisiddhidāyakaḥ [sapiṇḍe (YK)] | yatkāromīti saṅkalpaṃ kāryārambhe manaḥ sadā ||}
     \app{\lem[wit={ceteri}, alt={kriyāmuktir}]{kriyāmukti\skp{r-a}}
    \rdg[wit={N2}]{kriyāmukti}
    \rdg[wit={U2}]{kriyāmuktiḥ ||}
}\app{\lem[wit={ceteri}, alt={ayaṃ}]{\skm{r-a}yaṃ}
  \rdg[wit={N2}]{layaṃ}}
\app{\lem[wit={ceteri}]{yogaḥ}
  \rdg[wit={N1,U1}]{yogaḥ |}} svapiṇḍe
\app{\lem[wit={ceteri}]{siddhidāyakaḥ}
  \rdg[wit={U2}]{siddhidāyakaṃ}}/}\\
%-------------------------
% \om                                                    \E
% \om                                                    \L
% \om                                                    \B
% yaṃ yaṃ karoti kallolaṃ kāryāraṃbhe manaḥ sadā         \P
% yaṃ yaṃ karoti kallolaṃ kāryāraṃbhe manaḥ sadā/        \N1
% yaṃ yaṃ karoti kallolaṃ kāryāraṃbhe manaḥ sadā//1//    \N2
% yaṃ yaṃ karoti kallolaṃ kāryāraṃbhe manaḥ sadā/        \D
% \om                                                    \D2
% yaṃ yaṃ karoti kallolaṃ kāryāraṃbhe manaḥ sadā/ 1      \U1
% yaṃ yaṃ karoti kallolaṃ kāryāraṃbhe manaḥ sadā/        \U2
%--------------------------
%Each wave the mind creates at the beginning of an action,
%-------------------------- 
\tl{yaṃ yaṃ karoti kallolaṃ kāryāraṃbhe manaḥ sadā/}\\
%--------------------------
% \om                                                        \E
% \om                                                        \L
% \om                                                        \B
% tattataḥ   kuñcanaṃ kurvan kriyāyogas tato bhavet            \P
% tattataḥ   kuñcanaṃ kurvan kriyāyogas ato bhava     //       \N1
% tattataḥ   kūrcanaṃ kurvan kriyāyogas ato bhava     //       \N2
% tattataḥ   kuñcanaṃ kurvan kriyāyogas ato bhava     //       \D
% \om                                                          \D2
% taṃkṛ taṃ  kuñcanaṃ kurvan kriyāyogas ato ?va      //1//    \U1
% tatastataḥ kuṃcanaṃ kurvan kriyāyogas tato bhavet //1//     \U2
%--------------------------
%of all those one shall withdraw oneself. Then \textit{kriyāyoga} arises.
%--------------------------
\tl{\note[type=source, labelb=5, lem=tattataḥ]{Ysv (PT=YK): tatsāṅgācaraṇaṃ kurvan kriyāyogarato bhavet |}
  \app{\lem[wit={ceteri}]{tattataḥ}
    \rdg[wit={U2}]{tatas tataḥ}
    \rdg[wit={U1}]{taṃkṛ taṃ}}
  \app{\lem[wit={ceteri}]{kuñcanaṃ}
    \rdg[wit={N2}]{kūrcanaṃ}}
  kurvan-kriyāyoga\skp{s-ta}\app{\lem[wit={P,U2}, alt={tato bhavet}]{\skm{s-t}ato bhavet}
    \rdg[wit={N1,N2,D}]{ato bhava}
    \rdg[wit={U1}]{ato va}}\dd{}1\hskip-2pt\dd{}}
\end{tlg}
\end{ekdosis}
\ekdpb*{}
%%%%%%%%%%%%%%%%%%%%%%%%%%%%%%%%%%%%%%%%%%
%%%%%%%%%%%%%%%%%%%%%%%%%%%%%%%%%%%%%%%%%%
%%%%%%%%PAGEBREAK%%%%%%%PAGEBREAK%%%%%%%%%
%%%%%%%%%%%%%%%%%%%%%%%%%%%%%%%%%%%%%%%%%%
%%%%%%%%%%%%%%%%PAGEBREAK%%%%%%%%%%%%%%%%%
%%%%%%%%%%%%%%%%%%%%%%%%%%%%%%%%%%%%%%%%%%
%%%%%%%%PAGEBREAK%%%%%%%PAGEBREAK%%%%%%%%%
%%%%%%%%%%%%%%%%%%%%%%%%%%%%%%%%%%%%%%%%%%
%%%%%%%%%%%%%%%%%%%%%%%%%%%%%%%%%%%%%%%%%%
%%%%%%%%%%%%%%%%%%%%%%%%%%%%%%%%%%%%%%%%%%
%%%%%%%%%%%%%%%%%%%%%%%%%%%%%%%%%%%%%%%%%%
%%%%%%%%PAGEBREAK%%%%%%%PAGEBREAK%%%%%%%%%
%%%%%%%%%%%%%%%%%%%%%%%%%%%%%%%%%%%%%%%%%%
%%%%%%%%%%%%%%%%PAGEBREAK%%%%%%%%%%%%%%%%%
%%%%%%%%%%%%%%%%%%%%%%%%%%%%%%%%%%%%%%%%%%
%%%%%%%%PAGEBREAK%%%%%%%PAGEBREAK%%%%%%%%%
%%%%%%%%%%%%%%%%%%%%%%%%%%%%%%%%%%%%%%%%%%
%%%%%%%%%%%%%%%%%%%%%%%%%%%%%%%%%%%%%%%%%%
%%%%%%%%%%%%%%%%%%%%%%%%%%%%%%%%%%%%%%%%%%
%%%%%%%%%%%%%%%%%%%%%%%%%%%%%%%%%%%%%%%%%%
%%%%%%%%PAGEBREAK%%%%%%%PAGEBREAK%%%%%%%%%
%%%%%%%%%%%%%%%%%%%%%%%%%%%%%%%%%%%%%%%%%%
%%%%%%%%%%%%%%%%PAGEBREAK%%%%%%%%%%%%%%%%%
%%%%%%%%%%%%%%%%%%%%%%%%%%%%%%%%%%%%%%%%%%
%%%%%%%%PAGEBREAK%%%%%%%PAGEBREAK%%%%%%%%%
%%%%%%%%%%%%%%%%%%%%%%%%%%%%%%%%%%%%%%%%%%
%%%%%%%%%%%%%%%%%%%%%%%%%%%%%%%%%%%%%%%%%%
\begin{ekdosis}
  \ekddiv{type=ed}
    \begin{tlg}
%--------------------------      
% \om                                                                                                 \B
% \om                                                                                                 \L
% kṣamā vivekaṃ vairāgyaṃ śāntiḥ santoṣaniṣpṛhā       etadyuktiyuto  yogī   kriyāyogī nigadyate       \E
% kṣamāvivekavairāgyaṃ    śāntiḥ santoṣanispṛhāḥ      etadyuktiyuto  yogī   kriyāyogī nigadyate       \P
% kṣamāvivekavairāgyaṃ    śāntiḥ santoṣanispṛhā       etat yuktiyuto yogī   kriyāyogī nigadyate       \N1
% kṣamāvivekavairāgyaṃ    śāntiḥ santoṣanispṛhā //2// etat yuktiyuto yo sau kriyāyogī nigadyate//     \N2
% kṣamāvivekavairāgyaṃ    śāntiḥ santoṣanispṛhaḥ      etat yuktiyuto yogī   kriyāyogī nigadyate       \D
% \om                                                                                            \D2
% kṣamāvivekavairāgya---- śāntisantoṣaniḥspṛhī        etad yuktiyuto  yo sau kriyāyogī nigadyate       \U1 
% kṣamā vivekaṃ vairāgyaṃ śāntisaṃtoṣaniṣpṛhāḥ //     etat muktiyuto yogī   kriyāyogī nigadyate //2// \U2
%--------------------------
%Patience, discrimination, equanimity, peace, modesty, desireless: The \textit{yogī} who is endowed with these means is said to be a \textit{kriyāyogī}.
%--------------------------
% The text of the Printed Edition starts here ---> 
%--------------------------
      \tl{\note[type=source, labelb=6, lem=kṣamā°]{Ysv (PT): kṣamāvivekavairāgyaśāntisantoṣanispṛhāḥ | etan muktiyuto yo'sau kriyāyogo nigadyate |}
        \note[type=source, labelb=7, lem={kṣamā°}]{Ysv (YK): kṣamāvivekavairāgyaśāntisantoṣanispṛhāḥ | etan muktiyutaś cāsau kriyāyogī nigadyate || 211 ||}
kṣamā\app{\lem[wit={ceteri}, alt={°viveka°}]{viveka}\rdg[wit={E,U2}]{vivekaṃ}}vairāgyaṃ\note[type=philcomm, labelb=8, lem={°kṣamā°}]{The printed edition E starts here.}śāntisantoṣa\app{\lem[wit={P},alt={°nispṛhāḥ}]{nispṛhāḥ}
          \rdg[wit={U2}]{°niṣpṛhāḥ ||}
          \rdg[wit={E,N1}]{°nispṛhā}
          \rdg[wit={N2}]{°niṣpṛhā ||2||}
          \rdg[wit={D}]{°nispṛhaḥ}
          \rdg[wit={U1}]{°niṣpṛhī}}/}\\
      \tl{\app{\lem[wit={E,P,U1},alt={etad}]{eta\skp{d-yu}}
          \rdg[wit={N1,N2,D,U2}]{etat}
}\app{\lem[wit={ceteri}, alt={yuktiyuto}]{\skm{d-yu}ktiyuto}  %%%SANDHI
    \rdg[wit={U2}]{muktiyuto}}
  \app{\lem[wit={E,P,N1,D,U2}]{yogī}    
    \rdg[wit={N2,U1}]{yo sau}}
kriyāyogī nigadyate\dd{}2\hskip-2pt\dd{}}
\end{tlg}
       \ekddiv{type=ed}
     \begin{tlg}
%-----------------------
% \om                                                \B
% \om                                                \L
% mātsaryaṃ mamatā māyā hiṃsā ca   madagarvitā /     \E
% mātsarya  mamatā māyā hiṃsāśā    madagarvitāḥ      \P
% mātsarya  mamatā māyā hiṃsāḥ //  madagarvatā /     \N1    -> the hiṃsā---''ḥ//'' in \nepal looks like a śā -> indicator that the others copied from \nepal? 
% mātsarya  mamatā māyā hiṃsāśā    madagārvatā //3// \N2
% mātsarya  mamatā māyā hiṃsāśā    madagarvatā /     \D
% \om                                                \D2
% mātsaryaṃ mamatā māyā hiṃsāśā    madagarvatā /     \U1
% mātsaryaṃ mamatā māyā hiṃsāśā    madagarvatā /     \U2
%-----------------------
%Envy, selfishness, cheating, violence, desire and intoxication, pride,
%-----------------------
       \tl{\note[type=source, labelb=9, lem=mātsaryaṃ]{Ysv (PT): mātsaryaṃ mamatā māyā hiṃsā ca madagarvitā | kāmaḥ krodho bhayaṃ lajjā lobho mohas tathā 'śuciḥ [śuciḥ (YK)] ||}
         \app{\lem[wit={E,U1,U2}]{mātsaryaṃ}
           \rdg[wit={P,N1,D}]{mātsarya}}
         mamatā māyā
         \app{\lem[wit={E}]{hiṃsā ca}
           \rdg[wit={ceteri}]{hiṃsāśā}
           \rdg[wit={E}]{hiṃsā ca}
           \rdg[wit={N1}]{hiṃsāḥ ||}}
         madagarvatā/}\\
%-----------------------
% \om                                                   \B
% \om                                                   \L
% kāmakrodhabhayaṃ   lajjā lobhamohau tathā śuciḥ //    \E
% kāmakrodhabhayaṃ   lajjā lobhamohau tathā 'śuciḥ      \P
% kāmakrodhabhayaṃ   lajjā lobhamohau tathā 'śuciḥ /    \N1    -> the hiṃsā---''ḥ//'' in \nepal looks like a śā -> indicator that the others copied from \nepal? 
% kāmakrodho bhayaṃ  lajjā lobhamohau tathā śuciḥ //    \N2
% kāmakrodho bhayaṃ  lajjā lobhamohau tathā 'śuciḥ //   \D
% \om                                                   \D2
% kāmakrodhau bhayaṃ lajjā lobhamohau tathā 'śuciḥ      \U1
% kāmakrodhau bhayaṃ lajjā lobhamohau tathā śuciḥ //3// \U2
% -----------------------
% lust, anger, fear, laziness, greed, error and impurity.
%-----------------------
       \tl{kāma\app{\lem[wit={U1,U2}, alt={°krodhau}]{krodhau}
           \rdg[wit={E,P,N1}]{krodha°}
           \rdg[wit={D}]{°krodho}}
         bhayaṃ lajjā lobhamohau tathā
         \app{\lem[wit={ceteri}]{'śuciḥ}
           \rdg[wit={E,N2,U2}]{śuciḥ}}\dd{}3\hskip-2pt\dd{}}    %%%AVAGRAHA
\end{tlg}
        \ekddiv{type=ed}
      \begin{tlg}
%-----------------------
%  \om                                                           \B
%  atha dveṣo ghṛṇālasyaṃ bhrāṃtir   daṃbho kṣamā bhramaḥ //     \L
%  rāgadveṣau ghṛṇālasyaṃ bhrāntitvaṃ     mokṣamā bhramaḥ /      \E
%  rāgadveṣau ghṛṇālasyaṃ bhrāṃtir   ddaṃbhokaṣmā bhramaḥ        \P
%  rāgadveṣau ghṛṇālasyaṃ bhrāṃtir   daṃbho kṣamā bhramaḥ //4//  \N1
%  rāgadveṣau ghṛnālasyaṃ bhrāṃtir   daṃbho kṣamā bhramaḥ //4    \N2
%  rāgadveṣau ghṛṇālasyaṃ bhrāṃtir   debho  kṣamā bhramaḥ //     \D
% \om                                                            \D2
%  rāgadoṣau  ghṛṇālasyaṃ bhrāṃti    daṃbha kṣamī bhramaḥ 4      \U1
%  rāgadveṣau ghṛṇālasyaṃ bhrāṃtir   daṃbho kṣamā bhramaḥ //     \U2
%-----------------------
%Attachment and aversion, indignation and idleness, impatience and dizzyness
%-----------------------
        \tl{\note[type=source, labelb=10, lem=rāgadveṣau]{Ysv (PT): rāgadveṣau ghṛṇālasyaśrāntidambhakṣamābhramāḥ [ghṛṇālasyaṃ bhrāntir dambho 'kṣamā bhramaḥ (YK)] | yasyaitāni na vidyante kriyāyogī sa ucyate ||}
          \app{\lem[wit={ceteri}]{rāgadveṣau}
            \rdg[wit={U1}]{rāgadoṣau}
            \rdg[wit={L}]{athadveṣo}}\note[type=philcomm, labelb=11, lem={rāga°}]{L starts here.}
          \app{\lem[wit={ceteri},alt={ghṛṇā°}]{ghṛṇā}
            \rdg[wit={N2}]{ghṛnā°}}lasyaṃ 
          \app{\lem[wit={ceteri}, alt={bhraṃtir daṃbho}]{bhrantir-daṃbho}
            \rdg[wit={D}]{bhrāṃtir debho}
            \rdg[wit={E}]{bhrāntitvaṃ}
            \rdg[wit={U1}]{bhrāṃti daṃbha°}}
          \app{\lem[wit={ceteri}]{kṣamā bhramaḥ}
            \rdg[wit={E}]{mokṣam ābhramaḥ}
            \rdg[wit={U1}]{kṣamī bhramaḥ}}/}\\
%-----------------------
%  \om                                               \B
%  yasyai tāni na vidyaṃte kriyāyogī sa ucyate //    \L
%  yasyai tāni ca vidyante kriyāyogī sa ucyate 3     \E
%  yasyai tāni na vidyaṃte kriyāyogī sa ucyate       \P
%  yasyai tāni na vidyaṃte kriyāyogī sa ucyate //    \N1
%  yasyai tāni na vidyaṃte kriyāyogī sa ucyate //    \N2
%  yasyai tāni na vidyaṃte kriyāyogī sa ucyate //    \D
%  yasyai tāni na vidyaṃte kriyāyogī sa ucyate       \U1
%  yasyai tāni na vidyaṃte kriyāyogī sa ucyate //4// \U2 
%  -----------------------
% Whoever doesn't experience these is called a \textit{kriyāyogī}. 
%  -----------------------        
        \tl{
yasyai tāni \app{\lem[wit={ceteri}]{na}\rdg[wit={E}]{ca}} vidyante kriyāyogī sa ucyate\dd{}4\hskip-2pt\dd{}}\\
      \end{tlg}
     \ekddiv{type=ed}
      \begin{prose}
%-----------------------
%  \om                                                                                          \B
%  yasyāntaḥkaraṇe kṣamāvivekavairāgyaśāntisantoṣādīny                         utpadyante //     \E
%  yasyāṃtaḥkaraṇe kṣamāvivekavairāgyaśāṃtisaṃtoṣa         ityādīny            utpādyaṃte        \P
%  tasyāṃtaḥkaraṇe kṣamāvivekavairāgyaśāṃtisaṃtoṣa         ityādīnotpādyaṃte                    \L
%  yasyāṃtaḥkaraṇe kṣamāḥ vivekavairāgya /    śāṃtisaṃtoṣa ityādīni            utpādyaṃte        \N1
%  yasyāṃtaḥkaraṇe kṣamā' vivekavairāgyā      śāṃtisaṃtoṣa ityādīni            utpādyaṃte /      \N2 %see Mss p3 recto vierte Zeile von unten  
%  yasyāṃtaḥkaraṇe kṣamā // vivekavairāgya // śāṃtisaṃtoṣa ityādīni            utpādyaṃte //     \D
%  yasyāṃtaḥkaraṇe kṣamāvivekavairāgyaśāṃtisaṃtoṣa         ityādīna niraṃtaram utyaṃte        \U1
%  yasyāṃtaḥkaraṇe kṣamāvivekavairāgyaśāṃtisaṃtoṣa         ityādayo niraṃtaraṃ utpādyaṃte       \U2
%  -----------------------
%  Patience, discrimination, equanimity, peace, contentment etc. are generated in his mind.
%  -----------------------        
        yasyāntaḥkaraṇe
        \app{\lem[wit={ceteri},alt={kṣamā°}]{kṣamā}
          \rdg[wit={N1}]{kṣamāḥ}
          \rdg[wit={N2}]{kṣamā'}
        }\app{\lem[wit={ceteri}]{vivekavairāgyaśānti}
          \rdg[wit={N1}]{kṣamāḥ vivekavairāgya | śāṃti°}
          \rdg[wit={N2}]{°vairāgyāśānti°}
          \rdg[wit={D}]{kṣamā || vivekavairāgya || śāṃti°}
        }\app{\lem[wit={ceteri}, alt={°santoṣa ityādīny}]{santoṣa ityādī\skp{ny-u}} %the°-problem
          \rdg[wit={E}]{°santoṣādīny}
          \rdg[wit={L}]{°santoṣa ity ādīno°}
          \rdg[wit={U1}]{°santoṣa ity ādīna niraṃtaram}
          \rdg[wit={U2}]{°santoṣa ity ādayo niraṃtaraṃ}
        }\app{\lem[wit={ceteri}]{\skm{ny-u}tpādyante}
          \rdg[wit={E}]{utpadyante}
          \rdg[wit={L}]{°tpādyaṃte}
          \rdg[wit={U1}]{utyaṃte}}/
%-----------------------
% \om \oxford
%  sa eva bahukriyāyogī kathyate /      \E
%  sa eva bahukriyāyogī kathyate        \P
%  sa eva bahukriyāyogī kathyate //     \L
%  sa eva bahukriyāyogī kathyate /      \N1
%  sa eva bahukriyāyogī sa kathyate /   \N2
%  sa eva bahukriyāyogā sa kathyate //  \D
%  sa eva bahukriyāyogī kathyate /      \U1
%  sa eva bahukriyāyogī tkacyate /      \U2
%-----------------------
% He alone is called a \textit{yogī} of many actions (\textit{bahukriyāyogī}).
%-----------------------
        sa eva
        \app{\lem[wit={ceteri}]{bahukriyāyogī}
          \rdg[wit={D}]{bahukriyāyogā}}
        \app{\lem[wit={ceteri}]{kathyate}
          \rdg[wit={D,N2}]{sa kathyate}
          \rdg[wit={U2}]{tkacyate}}/\\
%-----------------------
% \om \B
%               kāpaṭyaṃ      vittaṃ   hiṃsā    tṛṣṇā    mātsaryam    ahaṃkāraḥ    roṣaḥ kṣayaṃ    lajjā lobhamohā      aśucitvaṃ                       pākhaṃḍatvaṃ       bhrāntiḥ indriyavikāraḥ kāmaḥ          ete yasya manasi pratidinaṃ vyunā bhavanti /    \E
%               kāpaṭyaṃ      vittaṃ   hiṃsā    tṛṣṇā    mātsaryaṃ    ahaṃkāraḥ    roṣo bhayaṃ     lajjā lobhaḥ mohaḥ   aśucitvaṃ rāgaḥ dveṣaḥ   ālasyaṃ pākhaṃḍitvaṃ       bhrāṃtiḥ indriyaṃ vikāraḥ kāmaḥ        ete yasya manasi pratidinaṃ nyunā bhavanti     \P
%               kāpayaṃ     //vitaṃ // hiṃsā // tṛṣṇā // mātsaryaṃ // ahaṃkāraḥ // roṣo bhayaṃ //  lajjā lobhaḥ // moha aśucitvaṃ // rājadveṣa  alasyaṃ // pākhaṃḍitvaṃ // bhrāṃtiḥ // itivikāraḥ // kāmaḥ        eta yasya manasi pratidinaṃ nyunā bhavaṃti//    \L
%yasyāṃtakaraṇe kapatyaṃ māyā vitvaṃ   hiṃsā    tṛṣṇā    mātsaryaṃ    ahaṃkāraḥ    roṣo bhayaṃ     lajjā // lobhamohā   asucitvaṃ rāgadveṣaḥ // alasyaṃ pāṣaṃḍitvaṃ      bhraṃtiḥ / iṃdriyaivikāraḥ / kāmaḥ       ete yasya manasi pratidinaṃ nyunā bhavaīti/     \N1
%               kāpaṭyaṃ māyā vitvaṃ   hiṃsā    tṛṣṇā    mātsaryaṃ    ahaṃkāraḥ    e?ṣo bhayaṃ     lajjā/ lobhamoha     asūcitvaṃ rāgadveṣaḥ    ālasyaṃ pārṣaḍitvaṃ        bhrāṃtiḥ iṃdriyavikāraḥ // kāma         ete yasya manasi pratidinaṃ nyunā bhavaṃti //  \N2      
%               kāpaṭyaṃ māya vitvaṃ   hiṃsā    tṛṣṇā    mātsarya     ahaṃkāraḥ    roṣo bhayaṃ     lajjā // lobhamohā   asucitvaṃ rāgadveṣaḥ // ālasyaṃ pāṣaṃḍitvaṃ        bhraṃtiḥ // iṃdriyavikāraḥ // kāmaḥ // ete yasya manasi pratidinaṃ nyunā bhavaṃti //   \D
%               kāpachaṃ yāya vitvaṃ   hiṃsā    tṛṣṇā    mātsarya     ahaṃkāraḥ    roṣaḥ bhayaṃ    lajā     lobhamohā   aśucitvaṃ rāgadveṣaḥ    ālasyaṃ pākhaṃḍitvaṃ       bhraṃtiḥ iṃdriyavīkāraḥ    kāmaḥ       rāte yasya manasi pratidinaṃ nyunā bhavaṃti //  \U1
%               kāpaṭyaṃ pāpā titaṃ    hiṃsā    tṛṣṇā    mātsaryaṃ // ahaṃkāraḥ    roṣo bhayaṃ     lajjā ----mohā       aśucitvaṃ rāgadveṣaḥ    ālasyaṃ pākhaṃḍitvaṃ //    bhraṃtiḥ iṃdriyavikāraḥ //-----        etate yasya manasi pratidinaṃ nyunā bhavaṃti // \U2
%-----------------------
%Fraud, illusion, property, violence, craving, envy, ego, anger, anxiety, shame, greed, error, impurity, attachment, aversion, idleness, heterodoxy, false view, affection of the senses, sexual desire: He who diminishes these from day to day in is mind,
%-----------------------              
\note[type=testium, labelb=12, lem={lobhaḥ}]{Ysg: lobhamohau aśucitvaṃ rāgadveṣau ālasyaṃ pāṣaṃḍitvaṃ bhrāṃtiḥ iṃdryiavikāraḥ kāmaḥ ete yasya pratidinaṃ nyunā bhavaṃti}
        \app{\lem[wit={ceteri}]{kāpaṭyaṃ}
        \rdg[wit={N1}]{yasyāntaḥkaraṇe kapatyaṃ}
        \rdg[wit={L}]{kāpayaṃ}
        \rdg[wit={U1}]{kāpachaṃ}}\dd{}
      \app{\lem[wit={N1,N2}]{māyā}
        \rdg[wit={D}]{māya}
        \rdg[wit={U1}]{yāya}
        \rdg[wit={U2}]{pāpa}
        \rdg[wit={E,P,L}]{\om}}\dd{}
        %\rdg[wit={E,P,L}]{\textbf{omitted in}}}
      \app{\lem[wit={E,P}]{vittaṃ}
        \rdg[wit={L}]{vitaṃ}
        \rdg[wit={N1,N2,D,U1}]{vitvaṃ}
        \rdg[wit={U2}]{titaṃ}}\dd{}
      hiṃsā\dd{}
      tṛṣṇā\dd{}
      \app{\lem[wit={ceteri}]{mātsaryaṃ}
        \rdg[wit={E}]{mātsaryam}
        \rdg[wit={D,U1}]{mātsarya}}\dd{}
      ahaṃkāraḥ\dd{}
      \app{\lem[wit={E,U1}]{roṣaḥ}
        \rdg[wit={ceteri}]{roṣo}
        \rdg[wit={N2}]{eṣo}}\dd{}
      \app{\lem[wit={ceteri}]{bhayaṃ}
        \rdg[wit={E}]{kṣayaṃ}}\dd{}
      \app{\lem[wit={ceteri}]{lajjā}
        \rdg[wit={U1}]{lajā}}\dd{}
      \app{\lem[wit={P,L}]{lobhaḥ}
        \rdg[wit={ceteri}]{lobha°}
        \rdg[wit={U2}]{\om}}\dd{}
      \app{\lem[wit={P}]{mohaḥ}
        \rdg[wit={L,N2}]{moha}
        \rdg[wit={ceteri}]{mohā}}\dd{}        
      \app{\lem[wit={ceteri}]{aśucitvaṃ}  %%%Frage: vor daṇḍa wird m zu ṃ??? 
        \rdg[wit={N1,D}]{aśucitvaṃ}
        \rdg[wit={N2}]{aśūcitvaṃ}}\dd{}
      \app{\lem[wit={P}]{rāgaḥ}
        \rdg[wit={ceteri}]{rāga°}
        \rdg[wit={L}]{rāja°}
        \rdg[wit={E}]{\om}}\dd{}
      \app{\lem[wit={ceteri}]{dveṣaḥ}
        \rdg[wit={L}]{dveṣa}
        \rdg[wit={E}]{\om}}\dd{}
      \app{\lem[wit={ceteri}]{ālasyaṃ}
        \rdg[wit={E}]{\om}}\dd{}
      \app{\lem[wit={ceteri}]{pākhaṃḍitvaṃ}
        \rdg[wit={D,N1}]{pāṣaṃḍitvaṃ}
        \rdg[wit={E}]{pākhaṃḍatvaṃ}
        \rdg[wit={N2}]{pārṣaḍitvaṃ}}\dd{}
     bhrāntiḥ\dd{}
     \app{\lem[wit={ceteri}]{indriyavikāraḥ}
        \rdg[wit={U1}]{iṃdriyavīkāraḥ}
        \rdg[wit={P}]{iṃdriyaṃ vīkāraḥ}
        \rdg[wit={L}]{itivikāraḥ}}\dd{}
      \app{\lem[wit={ceteri}]{kāmaḥ}
        \rdg[wit={N2}]{kāma}
        \rdg[wit={U2}]{\om}}\dd{}
      \app{\lem[wit={ceteri}]{ete}
        \rdg[wit={L}]{eta}
        \rdg[wit={U1}]{rāte}
        \rdg[wit={U2}]{etate}}
      yasya manasi pradidinaṃ nyūna
      \app{\lem[wit={ceteri}]{bhavanti}
        \rdg[wit={N1}]{bhavaīti}}/ 
%-----------------------       
%sa eva bahukriyāyogī kathyate// \E
%sa eva bahukriyāyogī kathyate// \P
%sa eva bahukriyāyogī kathyate// \L
%sa eva bahukriyāyogī kathyate// \N1
%sa eva bahukriyāyogī kathyate// \N2
%sa eva bahukiyāyogī  kathyate//  \D
%sa eva bahukiyāyogī  kathyaṃte// \U1
%sa eva bahukiyāyogī  kathyaṃte// \U2
%-----------------------
%he alone is called a yogī of many actions (\textit{bahukriyāyogī})
%-----------------------
      \note[type=testium, labelb=13, lem={bahukriyāyogī}]{Ysg: sa eva kriyāyogī kathyate ||}
                    \note[type=philcomm, labelb=14, lem={bahukriyāyogī}]{The term \textit{bahukriyāyogī} currently seems to be unique in Sanskrit literature. The elaborations of Rāmacandra on Kriyāyoga after the quotes of the Ysv are either taken from an unknown source or his own creation.}
sa eva \app{\lem[wit={ceteri}]{bahukriyāyogī}
  \rdg[wit={D,U1,U2}]{bahukiyāyogī}}
      \app{\lem[wit={ceteri}]{kathyate}
        \rdg[wit={U1,U2}]{kathyaṃte}}\dd{}
    \end{prose}
  \end{ekdosis}
\ekdpb*{}
%%%%%%%%%%%%%%%%%%%%%%%%%%%%%%%%%%%%%%%%%%
%%%%%%%%%%%%%%%%%%%%%%%%%%%%%%%%%%%%%%%%%%
%%%%%%%%PAGEBREAK%%%%%%%PAGEBREAK%%%%%%%%%
%%%%%%%%%%%%%%%%%%%%%%%%%%%%%%%%%%%%%%%%%%
%%%%%%%%%%%%%%%%PAGEBREAK%%%%%%%%%%%%%%%%%
%%%%%%%%%%%%%%%%%%%%%%%%%%%%%%%%%%%%%%%%%%
%%%%%%%%PAGEBREAK%%%%%%%PAGEBREAK%%%%%%%%%
%%%%%%%%%%%%%%%%%%%%%%%%%%%%%%%%%%%%%%%%%%
%%%%%%%%%%%%%%%%%%%%%%%%%%%%%%%%%%%%%%%%%%
%%%%%%%%%%%%%%%%%%%%%%%%%%%%%%%%%%%%%%%%%%
%%%%%%%%%%%%%%%%%%%%%%%%%%%%%%%%%%%%%%%%%%
%%%%%%%%PAGEBREAK%%%%%%%PAGEBREAK%%%%%%%%%
%%%%%%%%%%%%%%%%%%%%%%%%%%%%%%%%%%%%%%%%%%
%%%%%%%%%%%%%%%%PAGEBREAK%%%%%%%%%%%%%%%%%
%%%%%%%%%%%%%%%%%%%%%%%%%%%%%%%%%%%%%%%%%%
%%%%%%%%PAGEBREAK%%%%%%%PAGEBREAK%%%%%%%%%
%%%%%%%%%%%%%%%%%%%%%%%%%%%%%%%%%%%%%%%%%%
%%%%%%%%%%%%%%%%%%%%%%%%%%%%%%%%%%%%%%%%%%
%%%%%%%%%%%%%%%%%%%%%%%%%%%%%%%%%%%%%%%%%%
%%%%%%%%%%%%%%%%%%%%%%%%%%%%%%%%%%%%%%%%%%
%%%%%%%%PAGEBREAK%%%%%%%PAGEBREAK%%%%%%%%%
%%%%%%%%%%%%%%%%%%%%%%%%%%%%%%%%%%%%%%%%%%
%%%%%%%%%%%%%%%%PAGEBREAK%%%%%%%%%%%%%%%%%
%%%%%%%%%%%%%%%%%%%%%%%%%%%%%%%%%%%%%%%%%%
%%%%%%%%PAGEBREAK%%%%%%%PAGEBREAK%%%%%%%%%
%%%%%%%%%%%%%%%%%%%%%%%%%%%%%%%%%%%%%%%%%%
%%%%%%%%%%%%%%%%%%%%%%%%%%%%%%%%%%%%%%%%%%
\begin{ekdosis}
    \ekddiv{type=ed}
    \bigskip
    \centerline{\textrm{\small{[Siddhakuṇḍalinīyoga and Mantrayoga]}}}
    \bigskip
    \begin{prose}
%-----------------------   
% \om                                   \B
%idānīṃ rājayogasya bhedāḥ kathyante // \E
%idānīṃ rājayogasya bhedāḥ kathyaṃte    \P
%idānīṃ rājayogasya bhedāḥ              \L
%idānīṃ rājayogasya bhedāḥ kathyaṃte    \N1
%idānīṃ rājayogasya bhedā  kathyate//    \N2
%idānīṃ rājayogasya bhedāḥ kathyaṃte // \D
% \om                                   \U1
%idānīṃ rājayogasya bhedāḥ kathyaṃte // \U2
%-----------------------
%Now varieties of \textit{rājayoga} will be described.
%-----------------------
      \noindent idānīṃ rājayogasya
      \note[type=testium, labelb=15, lem={rājayogasya}]{Ysg: atha rājayogasya bhedau kathyete ||}
       \app{\lem[wit={ceteri}]{bhedāḥ}
         \rdg[wit={N2}]{bhedā}}
       \app{\lem[wit={ceteri}]{kathyante}
         \rdg[wit={N2}]{kathyate}
         \rdg[wit={L}]{\om}}/\note[type=philcomm, labelb=16, lem={kathyante}]{The whole sentence is \om in U\textsubscript{1}.}     
%-----------------------
%te ke     \E
%te ke     \P
%te ke     \L
%ke te //  \D
%ke te /   \N1
%kriyate// \N2       
%ke te     \U1
%te ke     \U2
%-----------------------
%Which are these?
%-----------------------       
\app{\lem[wit={D,N1,U1}]{ke te}
         \rdg[wit={ceteri}]{te ke}
         \rdg[wit={N2}]{kriyate}}/ 
%-----------------------
%\om                                       \B
%ekaḥ siddhakuṇḍalinīyogaḥ / mantrayogaḥ / \E
%ekaḥ siddhakuṃḍaṃliṃ yogaḥ maṃtrayogaḥ    \P
%ekaḥ siddhakuṇḍalanīyoga /                \L 
%ekaḥ siddhakuṇḍalinīyogaḥ maṃtrayogaḥ /   \N1
%ekaḥ siddhakuṇḍalanīyogaḥ maṃtrayogaḥ //  \N2
%ekaḥ siddhakuṃḍalanīyogaḥ mantrayogaḥ //  \D
%ekaḥ siddhakuṇḍaliniyogaḥ mantrayogaḥ     \U1
%ekaḥ siddhakuṇḍalinīyoga // mantrayogaḥ   \U2
%-----------------------
%One is \textit{siddhakuṇḍalinīyoga} [and one] is \textit{mantrayoga}.       
%-----------------------
\note[type=testium, labelb=17, lem={siddhakuṇḍalinīyogaḥ}]{Ysg: siddhakuṃḍaliyogaḥ mantrayogaś ceti}
ekaḥ
\app{\lem[wit={E,N1}]{siddhakuṇḍalinīyogaḥ |}
   \rdg[wit={U1}]{siddhakuṇḍalinīyogaḥ}
   \rdg[wit={U2}]{siddhakuṇḍalinīyoga ||}
   \rdg[wit={L}]{siddhakuṇḍalanīyoga |}
   \rdg[wit={N2,D}]{siddhakuṃḍalanīyogaḥ}
   \rdg[wit={P}]{siddhakuṃḍaṃliṃ yogaḥ}}
\app{\lem[wit={ceteri}]{mantrayogaḥ}
   \rdg[wit={L}]{\om}}/ \note[type=philcomm, labelb=18, lem={siddhakuṇḍalinīyogaḥ mantrayogaḥ}]{The sudden appearance of the term \textit{mantrayoga} here seems odd: This section that follows doesn't mention the practice of \textit{mantra} at all. It might simply be an early scribal mistake that has been copied by most of the manuscripts. However, all witnesses preserve this reading except L. The sentence that follows confirms the reading of Mantrayoga by the usage of dual forms. Although the YTB follows the Ysv very closely in structure and content, the yoga introduced in the Ysv at this point is \textit{jñānayoga}. The subject of \textit{jñāna} in this context, however, is picked up by the YTB. It is also well in the range of realistic possibilities that already in the text's early transmission folios got lost and confused. This szenario is supported by the diffuse arrangement of the the five types of Lakṣyayoga and the Yogas missing from the list. Currently it seems not possible to fix this issue conclusively.}
       \note[type=source, labelb=19, lem={siddhakuṇḍalinīyogaḥ mantrayogaḥ}]{Ysv (PT): jñānayogaṃ pravakṣyāmi tajjñānī śivatāṃ vrajet | paṭhanāt smaraṇād vyānānmaṇḍanāt brahmasādhakaḥ | tad bhedasyaikasandhānamaṣṭaiśvaryamayo bhavet | tritīrthaṃ yatra nāḍī ca tripuṇyaṃ parameśvari | \ldots eṣo 'sya viśvarūpasya rājayogo mato budhaiḥ | viśeṣaṃ kathayiṣyāmi śṛṇu caikamanāḥ sati |}
%-----------------------
% \om                         \B
%astu rājayogaḥ kathyate /    \E
%amū rājayogau kathyete       \P
%amū rājayogau kathyate //    \L
%amū rājayogau kathyate       \N1
%amū rājayogau kathyate//     \N2  %%%p3verso
%amū rājayogau kathyate //    \D
%amū rājayogau kathyate       \U1
%amū rājayogau kathyaṃte //   \U2
%-----------------------
%These two rājayogas are described [in the following].
%-----------------------
       \app{\lem[wit={ceteri}]{amū}
         \rdg[wit={E}]{astu}}
       \app{\lem[wit={ceteri}]{rājayogau}
         \rdg[wit={E}]{rājayogaḥ}}
       \app{\lem[wit={P}]{kathyete}
         \rdg[wit={ceteri}]{kathyate}
         \rdg[wit={U2}]{kathyaṃte}}/
%-----------------------
% \om                                                              \B
%mūlakandasthāne    ekā tejorūpā    mahānāḍī varttate /            \E
%mūlaṃ kaṃdasthāne  ekā tejorūpā    mahānāḍī varttate              \P
%mūlakaṃdasthāne    ekā tejorūpā    mahānāḍī vartate               \L
%mūlakaṃdasthāne    eka tejorūpā    mahānāḍī varttate /            \N1
%mūlakaṃdasthāne    eka tejorūpā    mahānāḍī varttate /            \N2
%mūlakaṃdasthāne    ekā tejorūpā    mahānāḍī varttate //           \D
%mūlakaṃdasthāne    ekā tejorūpā    mahānāḍī vartate /             \U1
%mūlakaṃdasthāne // ekā tejorūpā // mahānāḍī pravarttate /         \U2
%-----------------------
%At the location of the root-bulb exists one major vessel in the form of energy.
%-----------------------       
\note[type=testium, labelb=20, lem={mūlakanda°}]{Ysg: mūlakandasthāne ekā tejomayā mahānāḍī vartate |}
       \app{\lem[wit={ceteri}]{mūlakandasthāne}
         \rdg[wit={U2}]{mūlakaṃdasthāne ||}
         \rdg[wit={P}]{mūlaṃ kaṃdasthāne}}
       \app{\lem[wit={ceteri}]{ekā}
         \rdg[wit={N1,N2}]{eka}}
       \app{\lem[wit={ceteri}]{tejorūpā}
         \rdg[wit={U2}]{tejorūpā ||}}
       mahānāḍī
       \app{\lem[wit={ceteri}]{vartate}
         \rdg[wit={U2}]{pravartate}}/
       \note[type=source, labelb=21, lem={mūlakanda°}]{Ysv (PT): mūlakande sthale caikā nāḍī tejasvatī parā | gudorddhe sā tribhāgābhūdiḍā nāma śaśiprabhā | śaktirūpāmahānāḍī dhyānāt sarvārthadāyinī | dakṣiṇe 'pi kulākhyeti puṃrūpā sūryavigrahā | madhyabhāge suṣumnākhyā brahmaviṣṇuśivātmikā | śuddhacittena sā vijñā vidyutkoṭisamaprabhā |}
\note[type=source, labelb=22, lem={mūlakanda°}]{Ysv (YK): mūlakandasthale caikā nāḍī tejasvitāparā || 246 || gudordhve sā tridhā bhūyādiḍāvāme śaśiprabhā | śaktirūpā mahānāḍī dhyānātsarvārthadāyinī || 247 || dakṣiṇe piṅgalākhyeti puṃrūpā sūryavigrahā | madhyabhāge suṣumnākhyā brahmaviṣṇuśivātmikā || 248 || śuddhacittena sā vijñā vidyutkoṭisamaprabhā |}
%-----------------------
% \om                                                            \B
%iyam ekanāḍī /  iḍāpiṃgalāsuṣumṇā      etān bhedān prāpnoti /    \E
%iyaṃ ekanāḍī    iḍāpiṃgalāsuṣumṇā      etān bhedān prāpnoti      \P
%trayaṃ kā nāḍī  iḍāpiṃgalāsuṣumnā //   etān bhedān prāpnoti      \L
%iyaṃ ekā nāḍī   iḍāpiṃgalāsuṣumnān /   ete  bhedān prāpnoti      \N1
%iyaṃ ekā nāḍī   iḍāpiṃgalāsuṣumnān//   ete  bhedān prāpnoti/     \N2
%iyaṃ ekā nāḍī   iḍāpiṃgalasuṣumnān //  ete  bhedān prāpnoti      \D    
%iyaṃ ekā nāḍī   iḍāpiṃgalāsuṣumnā      etān bhedān prāpnoti      \U1
%iyaṃ eka nāḍī   iḍāpiṃgalāsuṣumṇā      etān bhegān prāpnoti      \U2
%-----------------------
%This single vessel reaches to these openings which are \textit{iḍā}, \textit{piṅgalā} and \textit{suṣumnā}.
%-----------------------       
\note[type=testium, labelb=23, lem={ekā nāḍī}]{Ysg: iyaṃ iḍāpiṃgalasuṣumnā bhedā tridhā |}
\app{\lem[wit={E},alt={iyam}]{iya\skm{m-e}}
         \rdg[wit={ceteri}]{iyaṃ}
         \rdg[wit={L}]{trayaṃ}}
\app{\lem[wit={ceteri}, alt={ekā}]{\skp{m-e}kā}
         \rdg[wit={E}]{eka |}
         \rdg[wit={P}]{eka}
         \rdg[wit={L}]{kā}}
nāḍī iḍāpiṅgalā\app{\lem[wit={N1,N2,D},alt={°suṣumṇān ||}]{suṣumṇān ||}
         \rdg[wit={L}]{suṣumnā |}
         \rdg[wit={ceteri}]{°suṣumṇā}}
       \app{\lem[wit={ceteri}]{etān}
         \rdg[wit={N1,N2,D}]{ete}}
bhedān prāpnoti/
%-----------------------
%\om                                           \B
%vāmabhāge candrarūpā iḍā nāḍī varttate /      \E
%vāmabhāge caṃdrarūpā iḍā nāḍī varttate        \P
%vāmabhāge caṃdrarūpā iḍā nāḍī varttate //     \L
%vāmabhāge caṃdrarūpā iḍā nāḍī varttate /      \N1
%vāmabhāge caṃdrarūpā iḍā nāḍī varttate //     \N2
%vāmabhāge caṃdrarūpā iḍā nāḍī varttate /      \D
%vāmabhāge caṃdrarūpā iḍā nāḍī vartate         \U1
%vāmabhāge caṃdrarūpā     nāḍī pravarttate //  \U2
%-----------------------
%On the left side is the \textit{iḍā}-channel, being a resemblence of the moon.
%-----------------------        
\note[type=testium, labelb=24, lem={vāma°}]{Ysg: vāmabhāge caṃdrarūpā iḍā}
vāmabhāge candrarūpā
        \app{\lem[wit={ceteri}]{iḍā}
          \rdg[wit={U2}]{\om}}
        \app{\lem[wit={ceteri}]{nāḍī}}
        \app{\lem[wit={ceteri}]{vartate}
          \rdg[wit={U2}]{pravarttate}}/
%-----------------------
% \om                                                \B
%dakṣiṇabhāge  sūryarūpā piṅgalā  nāḍī    varttate /  \E
%dakṣiṇabhāge  sūryarūpā piṃgalā  nāḍī    varttate    \P
%dakṣiṇabhāge  sūryarūpā piṃgalā  nāḍī    varttate // \L
%dakṣiṇabhāge  sūryarūpā piṃgalā  nāḍī    varttate // \N1
%dakṣiṇabhāge  sūryarūpā piṃgalā  nāḍī    varttate/   \N2
%dakṣiṇabhāge  sūryarūpā piṃgalā  nāḍī    varttate // \D       
%dakṣiṇe bhāge sūryarūpā piṃgalā  nāḍī    vartate     \U1
%dakṣiṇabhāge  sūryarūpā piṃgalā  nāḍī pravartate //  \U2
%-----------------------
%On the right side exists the \textit{piṅgalā}-channel, being a resemblence of the sun.        
%-----------------------
\note[type=testium, labelb=25, lem={dakṣiṇa°}]{Ysg: dakṣiṇabhāge sūryarūpā piṃgalā |}
        \app{\lem[wit={ceteri}]{dakṣiṇabhāge}
          \rdg[wit={U1}]{dakṣiṇe bhāge}}
        sūryarūpā piṅgalā nāḍī
        \app{\lem[wit={ceteri}]{vartate}
          \rdg[wit={U2}]{pravarttate}}/
%-----------------------
% \om                                                                   \B
%madhyamārge `tisūkṣmā padminī taṃtusamākārā  koṭividyutsamaprabhā      \E
%madhyamārge `tisūkṣmā padmanī taṃtusamākāra  koṭividyutsamaprabhā      \P
%madhyamārge `tisūkṣmā padmanī taṃtusamākārā  koṭividyutsamaprabhā      \L
%madhyamārge atisūkṣmā padmanī taṃtusamākārā  koṭividyutsamaprabhā //   \N1
%madhyamārge atisūkṣmā padmanī taṃtusamākārā  koṭividyutsamaprabhā //   \N2
%madhyarge   atisūkṣmā padminī taṃtusamākārā  koṭividyutsamaprabhā //   \D
%madhyamārge atisūkṣmā padminī taṃtusamākārā  koṭividyutsamaprabaḥ      \U1
%madhyamārge  tisūkṣmā padminī taṃtusamākārā  koṭividyutsamaprabhā //   \U2
%-----------------------
%Within the middle path is a lotuspond being very subtle. [It is] made from a web of light [and it] shines like a thousand lightnings.
%----------------------- 
\note[type=testium, labelb=26, lem={madhyamārge}]{Ysg: madhyamārge atisūkṣmā visa? taṃtusamākārā koṭividyutprabhā}
        \app{\lem[wit={ceteri}]{madhyamārge}
          \rdg[wit={D}]{madhyarge}}
        \app{\lem[wit={E,P,L,U2}]{'tisūkṣmā}
          \rdg[wit={D,N1,N2,U1}]{atisūkṣmā}}
        \app{\lem[wit={ceteri}]{padminī}
          \rdg[wit={P,L,N1,N2}]{padmanī}}/
        \app{\lem[wit={ceteri}]{tantusamākārā}
          \rdg[wit={P}]{taṃtusamākāra°}}
      koṭividyutsama\app{\lem[wit={ceteri},alt={°prabhā}]{prabhā}
        \rdg[wit={U1}]{°prabhaḥ}}/
      \note[type=testium, labelb=27, lem={madhyamārge}]{SSP 2.26: mūlakandād daṇḍalagnāṃ brahmanāḍīṃ śvetavarṇāṃ brahmarandhraparyantaṃ gatāṃ saṃsmaret | tanmadhye kamalatantunibhāṃ vidyutkoṭiprabhām ūrdhvagāminīṃ tāṃ mūrtiṃ manasā lakṣayet | sarvasiddhipradā bhavati |}
    \end{prose}
  \end{ekdosis}
\ekdpb*{}
%%%%%%%%%%%%%%%%%%%%%%%%%%%%%%%%%%%%%%%%%%
%%%%%%%%%%%%%%%%%%%%%%%%%%%%%%%%%%%%%%%%%%
%%%%%%%%PAGEBREAK%%%%%%%PAGEBREAK%%%%%%%%%
%%%%%%%%%%%%%%%%%%%%%%%%%%%%%%%%%%%%%%%%%%
%%%%%%%%%%%%%%%%PAGEBREAK%%%%%%%%%%%%%%%%%
%%%%%%%%%%%%%%%%%%%%%%%%%%%%%%%%%%%%%%%%%%
%%%%%%%%PAGEBREAK%%%%%%%PAGEBREAK%%%%%%%%%
%%%%%%%%%%%%%%%%%%%%%%%%%%%%%%%%%%%%%%%%%%
%%%%%%%%%%%%%%%%%%%%%%%%%%%%%%%%%%%%%%%%%%
%%%%%%%%%%%%%%%%%%%%%%%%%%%%%%%%%%%%%%%%%%
%%%%%%%%%%%%%%%%%%%%%%%%%%%%%%%%%%%%%%%%%%
%%%%%%%%PAGEBREAK%%%%%%%PAGEBREAK%%%%%%%%%
%%%%%%%%%%%%%%%%%%%%%%%%%%%%%%%%%%%%%%%%%%
%%%%%%%%%%%%%%%%PAGEBREAK%%%%%%%%%%%%%%%%%
%%%%%%%%%%%%%%%%%%%%%%%%%%%%%%%%%%%%%%%%%%
%%%%%%%%PAGEBREAK%%%%%%%PAGEBREAK%%%%%%%%%
%%%%%%%%%%%%%%%%%%%%%%%%%%%%%%%%%%%%%%%%%%
%%%%%%%%%%%%%%%%%%%%%%%%%%%%%%%%%%%%%%%%%%
%%%%%%%%%%%%%%%%%%%%%%%%%%%%%%%%%%%%%%%%%%
%%%%%%%%%%%%%%%%%%%%%%%%%%%%%%%%%%%%%%%%%%
%%%%%%%%PAGEBREAK%%%%%%%PAGEBREAK%%%%%%%%%
%%%%%%%%%%%%%%%%%%%%%%%%%%%%%%%%%%%%%%%%%%
%%%%%%%%%%%%%%%%PAGEBREAK%%%%%%%%%%%%%%%%%
%%%%%%%%%%%%%%%%%%%%%%%%%%%%%%%%%%%%%%%%%%
%%%%%%%%PAGEBREAK%%%%%%%PAGEBREAK%%%%%%%%%
%%%%%%%%%%%%%%%%%%%%%%%%%%%%%%%%%%%%%%%%%%
%%%%%%%%%%%%%%%%%%%%%%%%%%%%%%%%%%%%%%%%%%
\begin{ekdosis}
     \ekddiv{type=ed}
    \centerline{\textrm{\small{[First Cakra]}}}
    \bigskip
    \begin{prose}
%-----------------------
%\om                                                    \B
%idānīṃ suṣumṇāyāṃ jñānotpattāv---upāyāḥ  kathyante      \E
%idānīṃ suṣumṇāyā  jñānotpattau   upāyāḥ  kathyaṃte      \P
%idānīṃ suṣumnā    jñānotpattau   upāyaḥ  kathyate //    \L
%idānīṃ suṣumnāyāḥ jñanotpanno    'pāyāḥ  kathyaṃte //   \N1
%idānīṃ suṣumnāyāḥ jñanotpanno    upāyāḥ  kathyaṃte //   \N2
%idānīṃ suṣumnāyāḥ jñanotpattau   upāyāḥ  kathyaṃte //   \D
%idānīṃ suṣumnāya--jñanotpattau    upāyāḥ kathyaṃte //   \U1
%idānīṃ suṣumṇāyā  jñānotpattau   upāyā   kathyaṃte //   \U2
%-----------------------
\noindent 
      \note[type=testium, labelb=28, lem={upāyāḥ}]{Ysg: atas taj jñotpattāv upāyā ucyaṃte ||}
\note[type=source, labelb=29, lem={mūlacakraṃ}]{Ysv (PT): mūlādhāraṃ catuṣpatraṃ gudorddhe [gudordhve (YK)] varttate mahat | tanmadhye svarṇapīṭhe tu trikoṇaṃ maṇḍalaṃ [trikoṇamaṇḍalaṃ (YK)] param | tatra vahniśikhākārā mūrttiḥ sarvatra siddhidā | asyā dhyānaṃ manomadhye vinā pīṭhena [pāṭhena (YK)] vāṅmayam | sarvaśāstrāṇi saṅkarṣaṃ [saṃkarṣa (PT)] sadā sphurati yogavit |}
idānīṃ  
    \app{\lem[wit={E}]{suṣumṇāyāṃ}
      \rdg[wit={P,U2}]{suṣumṇāyā}
      \rdg[wit={U1}]{suṣumnāya°}
      \rdg[wit={N1,N2,D}]{suṣumṇāyāḥ}
      \rdg[wit={L}]{suṣumnā°}}
    \app{\lem[wit={E}, alt={jñānotpattāv upāyāḥ}]{jñānotpattāv-upāyāḥ}
      \rdg[wit={ceteri}]{jñānotpattau upāyāḥ}
      \rdg[wit={U2}]{jñānotpattau upāyā}
      \rdg[wit={N1,N2}]{jñānotpanno 'pāyāḥ}}
    \app{\lem[wit={E,P,N1,N2,D,U1,U2}]{kathyante}
      \rdg[wit={L}]{kathyate}}\dd{}
%-----------------------
%\om                                            \B
%ādau caturdalaṃ mūlaṃ cakraṃ varttate /        \E
%ādau caturddalaṃ mūlaṃ cakraṃ varttate /       \P
%ādau caturdalamūlacakraṃ varttate //           \L
%ādau caturdalaṃ mūlacakraṃ varttate            \N1
%ādau prathamacaturdalamūlacakraṃ pravarttate// \N2      
%ādau caturdalaṃ mūlacakraṃ varttate            \D
%ādau caturdalaṃ mūlaṃ cakraṃ vartate           \U1
%ādau caturdalaṃ mūlacakraṃ pravarttate //      \U2
%-----------------------
%At the beginning\footnote{Supposedly at the beginning of the central channel.} exists the root-cakra having four petals.     
%-----------------------      
\note[type=testium, labelb=30, lem={mūlacakraṃ}]{Ysg: gudamūlacakraṃ caturdalaṃ |}
ādau
   \app{\lem[wit={N1,D,U2}]{caturdalaṃ mūlacakraṃ}
        \rdg[wit={E,P,U1}]{caturdalaṃ mūlaṃ cakraṃ}
        \rdg[wit={L}]{caturdalamūlacakraṃ}
        \rdg[wit={N2}]{prathamacaturdalamūlacakraṃ}}
      \app{\lem[wit={ceteri}]{vartate}
        \rdg[wit={U2}]{pravartate}}/
%-----------------------
%
%\om                                       \B
%prathamādhāracakraṃ varttate / gudāsthānaṃ    raktavarṇaṃ    gaṇeśadaivataṃ    siddhibuddhiśaktimuṣakavāhanam       kurmaṛṣiḥ /  ākuṃcamudrā /    apānavāyuḥ                                   caturdaleṣu     rajaḥsattvatamomanāṃsi /  vaṃ śaṃ ṣaṃ saṃ    madhyatrikoṇe triśikhāt    tanmadhye trikoṇākāraṃ kāmapīthaṃ varttate//    \E
%prathamaṃ ādhāracakraṃ         gudāsthānaṃ    raktavarṇaṃ    gaṇeśāṃ daivataṃ  siddhibuddhiśaktir mukhako vāhanam   kurmaṛṣiḥ    ākuṃcanamudrā    apānavāyuś-----------------------------------caturddaleṣu    rajaḥsattvatamomanāṃsi    vaṃ śaṃ ṣaṃ saṃ    madhyatrikoṇe triśikhā     tanmadhye trikoṇākāraṃ kāmapīthaṃ varttate //   \P
%prathamaṃ ādhāracakraṃ         gudāsthānaṃ    raktavarṇaṃ    gaṇeśadaivataṃ    siddhibuddhiśaktimuṣako vāhanaṃ //   kurmaṛṣiḥ    ākuṃcanamudrā    apānavāyuḥ                                   caturddaleṣu    rajaḥsattvatamomanāṃsi // vaṃ śaṃ ṣaṃ saṃ    madhyatrikoṇe triśikhā     tanmadhyatrikoṇākāraṃ kāmapīthaṃ vartate        \L
%---------------------------------------------------------------------------------------------------------------------------------------------------------------------------------------------------------------------------------------------------------------------------------------tanmadhyatrikoṇākāraṃ kāmapiṭhaṃ varttate /     \N1
%---------------------------------------------------------------------------------------------------------------------------------------------------------------------------------------------------------------------------------------------------------------------------------------tanmadhye trikoṇākāraṃ kāmapiṭhaṃ varttate /    \N2
%---------------------------------------------------------------------------------------------------------------------------------------------------------------------------------------------------------------------------------------------------------------------------------------tanmadhye trikoṇākāraṃ kāmapiṭhaṃ varttate /    \D
%---------------------------------------------------------------------------------------------------------------------------------------------------------------------------------------------------------------------------------------------------------------------------------------tanmadhye trikoṇākāraṃ kāmapiṭhaṃ varttate /    \U1
%prathamaṃ ādhāracakraṃ         gudāsthānaṃ // raktavarṇaṃ // gaṇeśadaivataṃ // siddhibuddhiśaktiḥ muṣako vāhanaṃ // kurmaṛṣiḥ // ākuṃcanamudrā // apānavāyu // urmīkalā // ojasvinīdhāraṇā // caturddaleṣu // rajaḥsattvatamomanāṃsi //  vaṃ śaṃ ṣaṃ saṃ // madhyatrikoṇe trirekhā //  tanmadhye trikoṇākāraṃ kāmapīthaṃ varttate //   \U2
%-----------------------
%The first cakra of support (\textit{ādhāra}) is at the anus [and] is red-colored. Gaṇeśa is the deity. He is success, intelligence and power. A rat is the mount. The Ṛṣi is Kūrma. The seal is contraction. The vitalwind is \textit{apāna}. The \textit{kalā} is the wave of consciousness (\textit{urmī}). The concentration is ``she who is powerful'' (\textit{ojasvinī})}. In the four petals [of it resides] \textit{rajas}, \textit{sattva}, \textit{tamas} and the mind-faculties (\textit{manāṃsi}), [symbolized by the syllables or \textit{bīja}s] vaṃ śaṃ ṣaṃ and saṃ. A trident is situated in the middle of the triangle\footnote{This passage is odd since a triagle wasn't mentioned before.}
%-----------------------
\note[type=testium, labelb=31, lem={kāmapiṭhaṃ}]{Ysg: tanmadhye trikoṇākāraṃ kāmapiṭhaṃ |}
      \extra{
          \app{\lem[wit={P,L,U2}]{prathamaṃ ādhāracakraṃ}
            \rdg[wit={E}]{prathamādhāracakraṃ vartate |}}/
                 gudā sthānaṃ\dd{}
                 \app{\lem[type=emendation, resp=egoscr]{raktaṃ}
                   \rdg[wit={E,P,L,U2}]{\korr rakta°}}varṇaṃ\dd{}
            \app{\lem[type=emendation, resp=egoscr]{gaṇeśaṃ daivataṃ}
                 \rdg[wit={E,L,U2}]{\korr gaṇeśadaivataṃ}
                 \rdg[wit={P}]{gaṇeśāṃ daivataṃ}}\dd{}
            \app{\lem[type=emendation, resp=egoscr]{siddhibuddhiśaktiṃ muṣako vāhanaṃ} %Emendation!!!
                 \rdg[wit={E}]{\korr °śaktimuṣakavāhanam}
                 \rdg[wit={P}]{°śaktir mukhako vāhanam}
                 \rdg[wit={L}]{°śaktimuṣako vāhanaṃ}
                 \rdg[wit={U2}]{°śaktiḥ muṣako vāhanaṃ}}\dd{}
            \app{\lem[type=emendation, resp=egoscr]{kūrma} %%sandhi aḥ vor ṛ wird zu a + ṛ 
                 \rdg[wit={U2}]{\korr kurma}}ṛṣiḥ\dd{}
            \app{\lem[type=emendation, resp=egoscr]{ākuñcanaṃ mudrā}
                 \rdg[wit={P,L,U2}]{\korr ākuñcana°}
                 \rdg[wit={E}]{ākuṃca°}}mudrā\dd{}
            \app{\lem[type=emendation, resp=egoscr]{apānaḥ vāyuḥ}
                 \rdg[wit={E,L}]{\korr apānavāyuḥ}
                 \rdg[wit={P}]{°vāyuś}
                 \rdg[wit={U2}]{°vāyu}}\dd{}
               \extra{
                 \app{\lem[type=emendation, resp=egoscr]{ūrmī}
                   \rdg[wit={U2}]{\korr urmī}} kalā\dd{}
                 ojasvinī dhāraṇā\dd{}}
                 caturdaleṣu rajaḥsattvatamomanāṃsi\dd{}
                 vaṃ śaṃ ṣaṃ saṃ\dd{} madhyatrikoṇe
            \app{\lem[wit={P,L}]{triśikhā}
                 \rdg[wit={E}]{triśikhāt}
                 \rdg[wit={U2}]{trirekhā}}\dd{}}
        %%%%%%%%%%%%%%%%%
        %%%%%%%%%%%%%%%%%
        %%%%%%%%%%%%%%%%%
        %%%%%%%%%%%%%%%%%
        %%%%%%%%%%%%%%%%%          
            \app{\lem[wit={ceteri}]{tanmadhye}
                 \rdg[wit={L,N1}]{tanmadhya}}
               trikoṇākāraṃ kāmapiṭhaṃ vartate/
\note[type=philcomm, labelb=32, lem={prathamaṃ \ldots triśikhā}]{The whole section from \textit{prathamaṃ} to \textit{triśikhā} is missing in D, N\textsubscript{1}, N\textsubscript{2} and U\textsubscript{1}. Equally detailled passages for the other \textit{cakra}s which include assigments to various categories like \textit{daivata}, \textit{bīja}s etc. occur only in witness \textit{U2}. It is currently not possible to decide if a) these passages were lost in transmission in all other witnesses and were preserved in U\textsubscript{2} only or b), if the extensive descriptions for the first \textit{cakra} occurred randomly and the additions found in U\textsubscript{2} are not authorial. At least we might assume that it was not the the scribe of U\textsubscript{2} himself who wrote the additions. He explicitly states that he copied his template faithfully in this colophon: \begin{quote} yādṛśaṃ pustakaṃ dṛṣtvā tādṛsaṃ likhitaṃ mayā || \\yadi śuddhaṃ aśuddho cā mama doṣo na dīyate ||1||\end{quote}}
%-----------------------
%\om                                                      \B
%tatpīṭhamadhye 'gniśikhākāraikā    mūrtir varttate /        \E
%tatpīṭhamadhye magniśikhākārā ekā  mūrtir varttate /      \P
%tatpīṭhamadhye   jniśikhāka!rāṇakā mūrti varttate //     \L
%tatpīṭhamadhye  agniśikhākārā ekā  mūrttir varttate //    \N1
%tatpīṭhamadhye  agniśikhākārā ekā  mūrttir varttate /     \N2
%tatpīṭhamadhye  agniśikhākārā ekā  mūrttir varttate //    \D
%tatpīṭhamadhye  agniśikhākārā ekā  mūrttir varttate //    \U1
%tatpīṭhamadhye  agniśikhākārā ekā  mūrttir asmi      //    \U2
%-----------------------
%In the middle of this seat (\textit{pīṭha}) exists a single form having the shape of a flame.             
%-----------------------
\note[type=testium, labelb=33, lem={'gniśikhākāraikā}]{Ysg: tatpīṭhamadhye agniśikhākārā gaṇeśamūrttir varttate ||}
tatpīṭhamadhye
\app{\lem[wit={E}]{'gniśikhākāraikā}
  \rdg[wit={ceteri}]{agniśikhākārā ekā}
  \rdg[wit={P}]{magniśikhākārā ekā}
  \rdg[wit={L}]{jñiśikhākarāṇakā}}
murti\skp{r-va}\app{\lem[wit={E,P,L,N1,N2,D,U1}, alt={vartate}]{\skm{r-va}rtate}
  \rdg[wit={U2}]{asmi}}/
%-----------------------%
%\om                                       \B
%tasyāḥ mūrtirdhyānakāraṇāt   sakalaśāstrakāvya-nāṭakādi-sakalavāṅmayaṃ vinābhyāsena puruṣasya manomadhye sphurati,     \E
%tasyā mūrter dhyānakaraṇāt   sakalaśāstrakāvya-nāṭakādi-sakalavāṅmayaṃ vinābhyāsena puruṣasya manomadhye sphurati      \P
%tasyā mūrtir dhyānakāraṇāt   sakalaśāstrakāvya-nāṭakādi //----vāṅmayaṃ vinābhyāsena puruṣasya manomadhye sphuraṃti!    \L
%tasyāḥ mūrter dhyānakaraṇāt  sakalaśāstrakāvya-nāṭakādi-sakalavāgmayaṃ vinābhyāsena puruṣasya manomadhye sphurati      \N1
%tasyā mūrtter dhyānakaraṇāt  sakalaśāstrakāvya-nāṭakādi-sakavāgmayaṃ   vinābhyāsena puruṣasya manomadhye sphurati//    \N2
%tasyāḥ mūrter dhyānakaraṇāt  sakalaśāstrakāvya-nāṭakādi-sakalavāgmayaṃ vinābhyāsena puruṣasya manomadhye sphurati      \D
%tasyā  mūrtair dhyānakaraṇāt sakalaśāstrakāvya-nāṭakādi-sakalavāgmayaṃ vinābhyāsena puruṣasya manomadhye sphurati      \U1
%tasyā          dhyānakaraṇāt sakalaśāstrakāvya-nāṭakādi-sakalavāṅmayaṃ vinābhyāsena puruṣasya manomadhye sphurati // asya bahir mānaṃdā // yogānaṃdā virānaṃdā // uparamānaṃdā // ajapājapa śat // 600 // ghaṭi 9 palāni 40 // \U2 %
%-----------------------
%Trough the practice of meditation on this form the whole literature, all \textit{śāstra}s, all poems, dramas etc., everything [related to] elocution, appears in the mind of the person without [prior] learning. \extra{[Assigned to it] is external bliss, yogic bliss, heroic bliss [and] the bliss of coming to rest.}
%-----------------------
\note[type=testium, labelb=34, lem={sakalaśāstra°}]{Ysg: tasyā mūrter dhyānakaraṇāt sakalakāvyanāṭakādisakalavāṅmayaṃ vinābhyāsena puruṣasya manomadhye sphurati}
\app{\lem[wit={ceteri}]{tasyā}
    \rdg[wit={E,N1,D}]{tasyāḥ}}
\app{\lem[wit={ceteri}, alt={mūrter}]{mūrte\skp{r-dhyā}}
    \rdg[wit={E,L}]{mūrtir}
    \rdg[wit={U1}]{mūrtair}
    \rdg[wit={U2}]{\om}
}\skm{r-dhyā}nakaraṇāt-śāstrakāvya\app{\lem[wit={ceteri}, alt={°nāṭakādi°}]{nāṭakādi}
    \rdg[wit={L}]{°nāṭakādi ||}}\app{\lem[wit={ceteri}, alt={°sakala}]{sakala}
    \rdg[wit={L}]{\om}
    \rdg[wit={N2}]{saka°}}\app{\lem[wit={E,P,L,U2}]{vāṅmayaṃ}
    \rdg[wit={N1,N2,D,U1}]{vāgmayaṃ}} vinābhyāsena puruṣasya manomadhye
\app{\lem[wit={ceteri}]{sphurati}
  \rdg[wit={L}]{sphuraṃti}}/
      \extra{asya
        \app{\lem[type=emendation, resp=egoscr, alt={bahir ānandā}]{bahir\skp{-}ānandā}
          \rdg[wit={U2}]{\korr bahir mānandā}}\dd{}
        yogānandā\dd{}
        \app{\lem[type=emendation, resp=egoscr]{vīrānandā}
          \rdg[wit={U2}]{\korr virānandā}}\dd{}
        uparamānandā\dd{}
        \app{\lem[type=emendation, resp=egoscr]{ajapājapaḥ śataḥ}
          \rdg[wit={U2}]{\korr ajapājapaśat}}\dd{} 600\dd{} ghaṭi 9 palāni 40\dd{}} 
     \end{prose}
  \end{ekdosis}
\ekdpb*{}
%%%%%%%%%%%%%%%%%%%%%%%%%%%%%%%%%%%%%%%%%%
%%%%%%%%%%%%%%%%%%%%%%%%%%%%%%%%%%%%%%%%%%
%%%%%%%%PAGEBREAK%%%%%%%PAGEBREAK%%%%%%%%%
%%%%%%%%%%%%%%%%%%%%%%%%%%%%%%%%%%%%%%%%%%
%%%%%%%%%%%%%%%%PAGEBREAK%%%%%%%%%%%%%%%%%
%%%%%%%%%%%%%%%%%%%%%%%%%%%%%%%%%%%%%%%%%%
%%%%%%%%PAGEBREAK%%%%%%%PAGEBREAK%%%%%%%%%
%%%%%%%%%%%%%%%%%%%%%%%%%%%%%%%%%%%%%%%%%%
%%%%%%%%%%%%%%%%%%%%%%%%%%%%%%%%%%%%%%%%%%
%%%%%%%%%%%%%%%%%%%%%%%%%%%%%%%%%%%%%%%%%%
%%%%%%%%%%%%%%%%%%%%%%%%%%%%%%%%%%%%%%%%%%
%%%%%%%%PAGEBREAK%%%%%%%PAGEBREAK%%%%%%%%%
%%%%%%%%%%%%%%%%%%%%%%%%%%%%%%%%%%%%%%%%%%
%%%%%%%%%%%%%%%%PAGEBREAK%%%%%%%%%%%%%%%%%
%%%%%%%%%%%%%%%%%%%%%%%%%%%%%%%%%%%%%%%%%%
%%%%%%%%PAGEBREAK%%%%%%%PAGEBREAK%%%%%%%%%
%%%%%%%%%%%%%%%%%%%%%%%%%%%%%%%%%%%%%%%%%%
%%%%%%%%%%%%%%%%%%%%%%%%%%%%%%%%%%%%%%%%%%
%%%%%%%%%%%%%%%%%%%%%%%%%%%%%%%%%%%%%%%%%%
%%%%%%%%%%%%%%%%%%%%%%%%%%%%%%%%%%%%%%%%%%
%%%%%%%%PAGEBREAK%%%%%%%PAGEBREAK%%%%%%%%%
%%%%%%%%%%%%%%%%%%%%%%%%%%%%%%%%%%%%%%%%%%
%%%%%%%%%%%%%%%%PAGEBREAK%%%%%%%%%%%%%%%%%
%%%%%%%%%%%%%%%%%%%%%%%%%%%%%%%%%%%%%%%%%%
%%%%%%%%PAGEBREAK%%%%%%%PAGEBREAK%%%%%%%%%
%%%%%%%%%%%%%%%%%%%%%%%%%%%%%%%%%%%%%%%%%%
%%%%%%%%%%%%%%%%%%%%%%%%%%%%%%%%%%%%%%%%%%
   \begin{ekdosis}    
     \ekddiv{type=ed}
    \centerline{\textrm{\small{[Second Cakra]}}}
    \bigskip
    \begin{prose}
%-----------------------
% \om                                       \oxford
%idānīṃ dvitīyaṃ svādhiṣṭānacakraṃ   ṣaḍdalaṃ upāyanapīṭhasaṃjñakaṃ bhavati //  \E
%idānīṃ dvitīyaṃ svādhiṣṭānacakraṃ   ṣaṭdalaṃ uḍḍīyānapīṭhaṃ saṃjñakaṃ bhavati  \P
%idānīṃ dvitīyaṃ svādhiṣṭānacakraṃ   ṣaṭdalaṃ uḍḍīyān pīṭhaṃ saṃjñakaṃ bhavati  \L
%idānīṃ dvitīyaṃ svādhiṣṭānacakraṃ   ṣaṭdalaṃ uḍyānapīṭhasaṃjñakaṃ bhavati /    \N1
%idānī  dvitīyaṃ svādhinacakraṃ      ṣaḍḍalaṃ uḍyānapīṭhasaṃjñakaṃ bhavati      \N2
%idānīṃ dvitīyaṃ svādhiṣṭānacakraṃ   ṣaṭdalaṃ uḍyāṇāpīṭhasaṃjñikaṃ bhavati //   \D
%idānīṃ dvitīyaṃ svādhiṣṭhānacakraṃ  ṣaṭdalaṃ uḍāganapīṭasaṃjñakaṃ bhavati      \U1
%idānīṃ dvitīye svādhiṣṭānacakraṃ // ṣaṭdalaṃ // uḍḍīyāṇapīṭhasaṃjñakaṃ bhavati // liṃgasthānaṃ // pītavarṇaṃ // pītaprabhā // rajoguṇa // brahmādevatā // vaikharīvāca // sāvitrīśaktiḥ // haṃsavāhanaṃ // vahaṇaṛṣiḥ // kāmāgniprabhā //sthūladehā // jāgradavasthā // ṛgveda // ācāryaliṃgaṃ // braṃhmasalokatāmokṣaḥ // śuddhabhumikātatvaṃ // gaṃdho viṣayaḥ // apānavāyuḥ // aṃtarmātṛkā // vaṃ bhaṃ maṃ yaṃ raṃ laṃ // bahir mātrā // kāmā // kāmākhyā // tejasī // ceṣṭṛikā // alasā // mithunā // ajapājapaḥ sahasra // 6000 //gha 0 96 pa 0 40// \U2
%-----------------------
%Now the second, the six-petalled \textit{Svādhiṣṭānacakra} known as the seat of \textit{uḍḍīyāṇa}\footnote{Discuss the term \textit{uḍḍīyāna}.}. \extra{The gender is the location. The color is yellow. The shine is yellow. \textit{Rajas} is the quality. The deity is Brahmā. The speech is \textit{vaikharī}\footnote{vaikharī f. in Kaśm. Śiv. °the 4. form of appearacne of \textit{parā}, the empirical speech sound, Utpala's Ṭīkā to Śivadṛṣṭi 2, 7. [B.]― Schmidt p. 337. Welches Buch???} (\textit{vaikharīvāca}). The power is Sāvitrī. The mount is the goose. The \textit{Rṣi} is Vahaṇa. The appearance (\textit{prabhā} is the fire of love (\textit{kāmāgni}). The body is gross, The state is that of being awake. [The Veda associated with it is] the Ṛgveda. The spiritual guide is the \textit{liṅga}. The liberation is residing in the world of Brahma. The level is the pure earth (\textit{śuddhabhumikā}). The sphere is smell. The vitalwind is \textit{apāna}. The internal alphabet [is]: vaṃ bhaṃ maṃ yaṃ raṃ laṃ. The outer alphabet?: desire, the Tīrtha of \textit{Kāmākhyā}\footnote{The Kāmākhyā is situated in Kāmarūpa on the Nīlakūṭa mountain in present day Assam. It's strange that it appears here, since Kāmarūpa appears already as the Tīrtha associated with the first \textit{cakra}.}, beauty of both\footnote{Why dual here?}, \textit{ceṣṭṛikā} (what is that?), lazy [and] copulation.}
%-----------------------      
\noindent
\note[type=testium, labelb=35, lem={svādhiṣṭānacakraṃ}]{Ysg: liṃgo dvitīyaṃ ṣaṭdalaṃ svādhiṣṭānasaṃjñakaṃ kamalaṃ udyānapīṭhasaṃjñakaṃ vartate ||}
\note[type=source, labelb=36, lem={svādhiṣṭhāna°}]{Ysv (PT): liṅgamūle tu pīṭhābhaṃ [raktābhaṃ (YK)] svādhiṣṭhānantu ṣaḍdalam | tanmadhye bālasūryābhaṃ mahajjyotiḥ susiddhidam | dhyānāc ca varddhate āyuḥ kandarpasamatāṃ vrajet |}
\app{\lem[wit={ceteri}]{idānīṃ}
          \rdg[wit={N2}]{idānī}}
        \app{\lem[wit={ceteri}]{dvitīyaṃ}
            \rdg[wit={U2}]{dvitīye}}
        \app{\lem[wit={U1}]{svādhiṣṭhānacakraṃ}
            \rdg[wit={E,P,L,N1,D,U2}]{svādhiṣṭānacakraṃ}
            \rdg[wit={N2}]{svādhinacakraṃ}}
        \app{\lem[wit={ceteri}]{ṣaṭdalaṃ}
            \rdg[wit={E}]{ṣaḍdalaṃ}
            \rdg[wit={N2}]{ṣaḍḍalaṃ}}
        \app{\lem[wit={U2},alt={uḍḍīyāṇapīṭha°}]{uḍḍīyāṇapīṭha}
            \rdg[wit={E}]{upāyanapīṭha°}
            \rdg[wit={L}]{uḍḍīyān pīṭhaṃ}
            \rdg[wit={N1,N2}]{uḍyānapīṭha°}
            \rdg[wit={D}]{uḍyāṇāpīṭha°}
            \rdg[wit={U1}]{uḍāganapīṭa°}}saṃjñakaṃ
bhavati/         
      %%%%%%%%%%%%%%%%
      %%%%%%%%%%%%%%%
      %%%%%%%%%%%%%%%%
      %%%%%%%%%%%%%%%
      %%%%%%%%%%%%%%%    
      \extra{\app{\lem[type=emendation, resp=egoscr]{liṅgaṃ}
          \rdg[wit={U2}]{\korr liṅga°}} sthānaṃ\dd{}
        \app{\lem[type=emendation, resp=egoscr]{pītaṃ}
          \rdg[wit={U2}]{\korr pīta°}} varṇaṃ\dd{}
        \app{\lem[type=emendation, resp=egoscr]{pītā}
          \rdg[wit={U2}]{\korr pīta°}} prabhā\dd{}
        rajo \app{\lem[type=emendation, resp=egoscr]{guṇaḥ}
          \rdg[wit={U2}]{\korr guṇa}}\dd{}
        brahmā devatā\dd{}
        vaikharī \app{\lem[type=emendation, resp=egoscr]{vāk}
          \rdg[wit={U2}]{\korr vāca}}\dd{}
        sāvitrī śaktiḥ\dd{}
        \app{\lem[type=emendation, resp=egoscr]{haṃso}
          \rdg[wit={U2}]{\korr haṃsa°}} vāhanaṃ\dd{}
        \app{\lem[type=emendation, resp=egoscr]{vahaṇo}
          \rdg[wit={U2}]{\korr vahaṇa}} ṛṣiḥ\dd{}
        \app{\lem[type=emendation, resp=egoscr, alt={kāmāgnir}]{kāmāgni\skp{r-pra}}
          \rdg[wit={U2}]{\korr kāmāgni°}}\skm{r-pra}bhā\dd{}
        \app{\lem[type=emendation, resp=egoscr]{sthūlo dehaḥ}
          \rdg[wit={U2}]{\korr sthūladehā}}\dd{}
        jāgrad-avasthā\dd{}
        \app{\lem[type=emendation, resp=egoscr]{ṛg vedaḥ}
          \rdg[wit={U2}]{\korr ṛg veda}}\dd{}
        \app{\lem[type=emendation, resp=egoscr]{ācāryaḥ}
          \rdg[wit={U2}]{\korr ācārya°}} liṅgaṃ\dd{}
        brahmasalokatā mokṣaḥ\dd{}
        \app{\lem[type=emendation, resp=egoscr]{śuddhabhumikā}
          \rdg[wit={U2}]{\korr śuddhabhumikā}} tattvaṃ\dd{}
        gaṃdho viṣayaḥ\dd{}
        \app{\lem[type=emendation, resp=egoscr]{apānaḥ}
          \rdg[wit={U2}]{apāna°}} vāyuḥ\dd{}
        aṃtar\skp{-}mātṛkā\dd{}
        vaṃ bhaṃ maṃ yaṃ raṃ laṃ\dd{}
        bahir-mātrā\dd{}
        kāmā\dd{}
        kāmākhyā\dd{}
        \app{\lem[type=emendation, resp=egoscr]{tejasvinī}
          \rdg[wit={U2}]{\korr tejasī}}\dd{}
        ceṣṭikā\dd{}
        alasā\dd{}
        mithunā\dd{}
        ajapājapaḥ \app{\lem[type=emendation, resp=egoscr]{sahasraḥ}
          \rdg[wit={U2}]{\korr sahasra}}\dd{} 6000 \dd{} gha. 16 pa. 40\dd{}}
%-----------------------
%
% \om                                        \B
%tanmadhye atiraktavarṇaṃ tejo varttate /    \E
%tanmadhye 'tiraktavarṇaṃ tejo varttate      \P
%tanmadhye  tiraktavarṇaṃ tejo varttate //   \L
%tanmadhye  atiraktavarṇaṃ tejo varttate     \N1
%tanmadhye  atiraktavarṇatejo varttate      \N2
%tanmadhye  atiraktavarṇaṃ tejo varttate     \D
%tanmadhye  atiraktavarṇatejo varttate       \U1
%tanmadhye 'tiraktavarṇaṃ tejo vartate //    \U2
%-----------------------
%In its middle exists extremely red glow. The adept becomes very handsome by meditation on it.       
%-----------------------          
\note[type=testium, labelb=37, lem={atirakta°}]{Ysg: tatra atiraktaṃ \sic{yahbhā} saṃjñakaṃ tejaḥ |}
tanmadhye         
        \app{\lem[wit={P,U2}]{'tiraktavarṇaṃ}
            \rdg[wit={ceteri}]{atiraktavarṇaṃ}
            \rdg[wit={U1,N2}]{atiraktavarṇa°}}
tejo vartate/
%-----------------------
% \om                                          \B
%tasya dhyānāt sādhako 'tisundaro bhavati /    \E
%tasya dhyānāt sādhako   tisuṃdaro bhavati      \P
%tasya dhyānāt sādhako   tisuṃdaro bhavati //   \L
%tasya dhyānāt sādhakaḥ  atisuṃdaro bhavati // \N1
%tasya dhyānāt sādhakaḥ  atisuṃdaro bhavati/   \N2
%tasya dhyānāt sādhakaḥ  atisuṃdaro bhavati // \D
%tasyā     nāt sādhakaḥ  atisuṃdarāṃgasan  // \D2
%tasya dhyānāt sādhakaḥ  atisuṃdaro bhavati    \U1
%tasya dhyānāt sādhako  'tisundaro bhavati //   \U2
%-----------------------
%The adept becomes very handsome through meditation on it.
%-----------------------       
\note[type=testium, labelb=38, lem={tasya dhyānāt}]{Ysg: tasyā nāt sādhakaḥ atisuṃdarāṃgasan ||}
tasya dhyānāt
\app{\lem[wit={E,P,L,U2}]{sādhako}
  \rdg[wit={ceteri}]{sādhakaḥ}}
\app{\lem[wit={E,P,L,U2}]{'tisundaro}
  \rdg[wit={D,N1,N2,U1}]{atisuṃdaro}}
bhavati/ 
%-----------------------
% \om                                  \B
%                                pratidinam-āyur vardhate /             \E
%                                pratidinam-āyur vardhate               \P
%                                pratidinam-āyur vardhate //2//         \L
%                                dinaṃ dinaṃ prati āyurvarddhate // //  \N1
%yuvatīnāṃ ativallabho? bhavati dinadinaṃ prati āyur varddhate//        \N2  %%%3verso
%                                dinaṃ prati āyurvarddhate //2//        \D
%                                dinaṃ dinaṃ prati āyurvarddhate        \U1
%                                pratidinaṃ āyur varddhate //          \U2
%-----------------------
%\extra{He becomes one who is very desired by virgins.} The vital force increases from day to day. \end{tlate}
%-----------------------
\note[type=testium, labelb=39, lem={yuvatīnām}]{Ysg: yuvatīnām ativallabhaḥ san pratidinam āyuṣyābhivṛddhimān bhavati | cha |} % \D2 %%%S.2 Z. 11}
\extra{yuvatīnāṃ ativallabho bhavati/\note[type=philcomm, labelb=40, lem={yuvatīnāṃ\ldots bhavati}]{This additional sentence occurs in N\textsubscript{2} and the Ysg only.}}
\app{\lem[wit={ceteri}, alt={pratidinam}]{pratidina\skp{m-ā}}
  \rdg[wit={N1,U1}]{dinaṃ dinaṃ prati}
  \rdg[wit={N2}]{dinadinaṃ prati}
  \rdg[wit={D}]{dinaṃ prati}}
\skp{m-ā}yur-vardhate\dd{}
    \end{prose}
  \end{ekdosis}
    %%%%%%%%%%%%%%
    %%%%%%%%%%%%%%%
    %%%%%%%%%%%%%%
    %%%%%%%%%%%%%%
  %%%%%%%%%%%%%%
\begin{ekdosis}
 \ekddiv{type=ed}
  \bigskip
    \centerline{\textrm{\small{[Third Cakra]}}}
    \bigskip
    \begin{prose}
      \note[type=source, labelb=50, lem={tṛtīyaṃ}]{Ysv (PT): tṛtīyaṃ nābhideśe tu digdalaṃ paramādbhutam | mahāmeghaprabhaṃ tattu koṭividyutsamanvitam | kalpāntāgnisamaṃ [kalpānto 'gni° (YK)] jyotis tanmadhye saṃsthitaṃ svayam | tasya [asya (YK)] dhyānāc cirāyuḥ syād arogo [arogī (YK)] jagatāṃ varaḥ [jagatāmvaraḥ (YK)] | sarvapāpavinirmukto jagatkṣobhakaro [jaganmokṣakaro (YK)] mahān |}
%-----------------------
% \om                                                 \B
%tṛtīye                      nābhisthāne    daśadalaṃ padmaṃ vartate      \E
%tṛtīyaṃ                     nābhisthāne    daśadalaṃ padmaṃ vartate      \P
%tṛtīyaṃ                     nābhisthāne // daśadalapadme vartate         \L
%tṛtīyaṃ                     nābhisthāne    daśadalaṃ padma varttate //   \N1
%tṛtīyacakraṃ                nābhisthāne    daśadalaṃ padma varttate /    \N2
%tṛtīyaṃ                     nābhisthāne    daśadalaṃ padma varttate //   \D
%tṛtīyaṃ                     nābhisthāne    daśadalakaṃ padmaṃ varttate   \U1
%atha tṛtīyaṃ maṇipūracakraṃ nābhisthāne // kapilavarṇaṃ // viṣṇudevatā // lakṣmīśaktiḥ // vāyuṛṣiḥ // samānavāyuḥ // garuḍavāhanaṃ // sūkṣmaliṃgadevatāha // svapnāvasthā // madhyamāvāk // yajurvedaḥ // dakṣināgniḥ // samipatāmokṣaḥ // guruliṃgaviṣṇuḥ // āpastatvaṃ // rajoviṣayaḥ daśadalāni // daśamātrāḥ // aṃtarmātrā // ḍaṃ ṭaṃ ṇaṃ taṃ thaṃ daṃ dhaṃ naṃ paṃ phaṃ // bahirmātrāḥ // śāṃtiḥ // kṣamā // medhā // tanyā // medhāvinī // puṣkarā // ahaṃsagamanā // lakṣyā //tanmayā // amṛtā // ajapājapa // 6000 gha 016 pa 040 //    \U2
%-----------------------
%\extra{The colour is red (\textit{kapila}). Viṣṇu is the deity. Lakṣmī is the power. Vāyu is the Rṣi. Samāna is the vitalwind. The mount is Garuḍa. The deity is the suble body\footnote{Why another deity is given here?}. The state is sleep. The speech is the inaudible speech (\textit{madhyamāvāg})\footnote{<Śā, Ling>name of the speech which is inaudible and which is of the type of a thought without any definite presence of words making up the expression. Vkp I.143.<Abhyankar 1986: 300>}. The Veda is the Yajurveda. The [fire is the] southern fire. The liberation is ``proximity'' (\textit{samīpatā}).\footnote{What is this exactly?}. Viṣṇu is the characteristic of the teacher (\textit{guruliṅga}). The principle is water. The sphere is athmosphere (\textit{rajo viṣaya}). There are ten petals [and] ten matrices. [The] inner matrix: \textit{ḍaṃ ṭaṃ ṇaṃ taṃ thaṃ daṃ dhaṃ naṃ paṃ phaṃ}. The external matrix : peace, patience, insight, the ``daughter''\textit{tanayā}, the ``learned teacher'', the ``lotus'', \textit{haṃsagamanā}, the ``fixation object'', absorption and immortality.} 
%-----------------------
\note[type=testium, labelb=60, lem={daśadalaṃ}]{Ysg: nābhistnāne daśadalaṃ cakraṃ |}
      \app{\lem[wit={ceteri}]{tṛtīyaṃ}
      \rdg[wit={E}]{tṛtīye}
      \rdg[wit={U2}]{atha tṛtīyaṃ maṇipūracakraṃ}
      \rdg[wit={N2}]{tṛtīyacakraṃ}}
    nābhisthāne
    \app{\lem[wit={ceteri}]{daśadalaṃ}
      \rdg[wit={L}]{daśadala°}
      \rdg[wit={U1}]{daśadalakaṃ}
      \rdg[wit={U2}]{\om}}
    \app{\lem[wit={E,P,U1}]{padmaṃ}
      \rdg[wit={L}]{°padme}
      \rdg[wit={N1,N2,D}]{padma}
      \rdg[wit={U2}]{\om}}
    \app{\lem[wit={ceteri}]{vartate}
      \rdg[wit={U2}]{\om}}/
      %%%%%%%%%%%%
      %%%%%%%%%%%%%%%
      %%%%%%%%%%%%%%
      %%%%%%%%%%%%%
      %%%%%%%%%%%% 
      \extra{
        \app{\lem[type=emendation, resp=egoscr]{kapilaṃ}
          \rdg[wit={U2}]{\korr kapila°}} varṇaṃ\dd{}
        \app{\lem[type=emendation, resp=egoscr]{viṣṇur}
          \rdg[wit={U2}]{\korr viṣṇu}} devatā\dd{}
        lakṣmī śaktiḥ\dd{}
        \app{\lem[type=emendation, resp=egoscr, alt={vāyur}]{vāyu\skp{r-ṛ}}
          \rdg[wit={U2}]{\korr vayu°}}\skm{r-ṛ}ṣiḥ\dd{}
        \app{\lem[type=emendation, resp=egoscr]{samāno}
          \rdg[wit={U2}]{\korr samāna°}} vāyuḥ\dd{}
        \app{\lem[type=emendation, resp=egoscr]{garuḍo}
          \rdg[wit={U2}]{\korr garuḍa°}} vāhanaṃ\dd{}
      \app{\lem[type=emendation, resp=egoscr]{sūkṣmaliṅgaṃ devatā}
        \rdg[wit={U2}]{\korr sūkṣmaliṅgadevatāha}}\dd{}
      \app{\lem[type=emendation, resp=egoscr,alt={svapnā avasthā}]{svapnā-avasthā}
        \rdg[wit={U2}]{\korr svapnāvasthā}}\dd{}
      madhyamā vāk\dd{}
      yajur-vedaḥ\dd{}}
     \end{prose}
  \end{ekdosis}
\ekdpb*{}
%%%%%%%%%%%%%%%%%%%%%%%%%%%%%%%%%%%%%%%%%%
%%%%%%%%%%%%%%%%%%%%%%%%%%%%%%%%%%%%%%%%%%
%%%%%%%%PAGEBREAK%%%%%%%PAGEBREAK%%%%%%%%%
%%%%%%%%%%%%%%%%%%%%%%%%%%%%%%%%%%%%%%%%%%
%%%%%%%%%%%%%%%%PAGEBREAK%%%%%%%%%%%%%%%%%
%%%%%%%%%%%%%%%%%%%%%%%%%%%%%%%%%%%%%%%%%%
%%%%%%%%PAGEBREAK%%%%%%%PAGEBREAK%%%%%%%%%
%%%%%%%%%%%%%%%%%%%%%%%%%%%%%%%%%%%%%%%%%%
%%%%%%%%%%%%%%%%%%%%%%%%%%%%%%%%%%%%%%%%%%
%%%%%%%%%%%%%%%%%%%%%%%%%%%%%%%%%%%%%%%%%%
%%%%%%%%%%%%%%%%%%%%%%%%%%%%%%%%%%%%%%%%%%
%%%%%%%%PAGEBREAK%%%%%%%PAGEBREAK%%%%%%%%%
%%%%%%%%%%%%%%%%%%%%%%%%%%%%%%%%%%%%%%%%%%
%%%%%%%%%%%%%%%%PAGEBREAK%%%%%%%%%%%%%%%%%
%%%%%%%%%%%%%%%%%%%%%%%%%%%%%%%%%%%%%%%%%%
%%%%%%%%PAGEBREAK%%%%%%%PAGEBREAK%%%%%%%%%
%%%%%%%%%%%%%%%%%%%%%%%%%%%%%%%%%%%%%%%%%%
%%%%%%%%%%%%%%%%%%%%%%%%%%%%%%%%%%%%%%%%%%
%%%%%%%%%%%%%%%%%%%%%%%%%%%%%%%%%%%%%%%%%%
%%%%%%%%%%%%%%%%%%%%%%%%%%%%%%%%%%%%%%%%%%
%%%%%%%%PAGEBREAK%%%%%%%PAGEBREAK%%%%%%%%%
%%%%%%%%%%%%%%%%%%%%%%%%%%%%%%%%%%%%%%%%%%
%%%%%%%%%%%%%%%%PAGEBREAK%%%%%%%%%%%%%%%%%
%%%%%%%%%%%%%%%%%%%%%%%%%%%%%%%%%%%%%%%%%%
%%%%%%%%PAGEBREAK%%%%%%%PAGEBREAK%%%%%%%%%
%%%%%%%%%%%%%%%%%%%%%%%%%%%%%%%%%%%%%%%%%%
%%%%%%%%%%%%%%%%%%%%%%%%%%%%%%%%%%%%%%%%%%
\begin{ekdosis}
    \ekddiv{type=ed}
  \begin{prose}
\noindent
\extra{\app{\lem[type=emendation, resp=egoscr]{dakṣiṇo 'gniḥ}
        \rdg[wit={U2}]{\korr dakṣināgniḥ}}\dd{}
      \app{\lem[type=emendation, resp=egoscr]{samīpatā}
        \rdg[wit={U2}]{samipatā}} mokṣaḥ\dd{}
      \app{\lem[type=emendation, resp=egoscr]{guruliṅgo}
        \rdg[wit={U2}]{\korr guruliṅga°}} viṣṇuḥ\dd{}
      āpas-tattvaṃ\dd{}
      rajo viṣayaḥ\dd{}
      daśadalāni\dd{}
      daśamātrāḥ\dd{}
      antar-mātrā\dd{}
      ḍaṃ ṭaṃ ṇaṃ taṃ thaṃ daṃ dhaṃ naṃ paṃ phaṃ\dd{}
      bahir-mātrāḥ\dd{}
      śāṃtiḥ\dd{}
      kṣamā\dd{}
      medhā\dd{}
      tanayā\dd{}
      medhāvinī\dd{}
      puṣkarā\dd{}
      \app{\lem[type=emendation, resp=egoscr]{haṃsagamanā}
        \rdg[wit={U2}]{\korr ahaṃsagamanā}}\dd{}
      lakṣyā\dd{}
      tanmayā\dd{}
      amṛtā\dd{}
      ajapājapaḥ \app{\lem[type=emendation, resp=egoscr]{sahasraḥ}
        \rdg[wit={U2}]{\korr sahasra}}\dd{} 6000\dd{} gha. 16 pa. 40\dd{}}   
%-----------------------
% \om                                       \B
%tanmadhye paṃcakoṇaṃ cakraṃ varttate//    \E
%tanmadhye paṃcakoṇaṃ cakraṃ varttate       \P
% \om  \L
%tanmadhye paṃcakoṇaṃ cakraṃ varttate//    \N1
%tanmadhye paṃcakoṇaṃ cakraṃ varttate/    \N2
%tanmadhye paṃcakoṇaṃ cakraṃ varttate//    \D
%tanmadhye paṃcakoṇaṃ cakraṃ varttate       \U1
%tanmadhye paṃcakoṇaṃ cakraṃ vartate//     \U2
%-----------------------
% In its middle exists a \textit{cakra} with five angles.
%-----------------------
\note[type=testium, labelb=61, lem={paṃcakoṇaṃ}]{Ysg: tanmadhye paṃcakoṇaṃ pīṭhe lakṣmīnāparvatī saṃjñakaṃ sahitā śiva saṃjñakaṃ rāmaṇaṃ rūpā}
tanmadhye pancakoṇaṃ cakraṃ vartate/ \note[type=philcomm, labelb=62, lem={tanmadhye ... cakraṃ vartate}]{This sentence is entirely \om L.}
%-----------------------
% \om                                  \B
%tanmadhye ekā mūrtir vartate/         \E
%tanmadhye ekā mūrtir vartate          \P
%\om                                   \L
%tanmadhye ekā mūrttir varttate //     \N1
%tanmadhye ekā mūrttir varttate/       \N2
%tanmadhye ekā mūrttir varttate//      \D
%tanmadhye ekā mūrtir vartate          \U1
%tanmadhye ekā mūrtir asmi//           \U2
%-----------------------
%In its middle is a single (divine) form. 
%-----------------------
\app{\lem[wit={ceteri}]{tanmadhye}
  \rdg[wit={L}]{\om}}
\app{\lem[wit={ceteri}]{ekā}
  \rdg[wit={L}]{\om}}
\app{\lem[wit={ceteri}]{mūrti\skp{r-va}}
  \rdg[wit={L}]{\om}}\app{\lem[wit={ceteri}, alt={vartate}]{\skm{r-va}rtate}
  \rdg[wit={U2}]{asmi}}/
%-----------------------
% \om                                           \B
%tasyās tejo jihvayā kathayituṃ na śakyate /    \E
%tasyās tejo jihvayā kathayituṃ na śakyate      \P
%tasyās tejo jihvayā kathyituṃ  na śakyate       \L
%tasyā  tejo jihvayā kathayituṃ  na śakyate //    \N1
%tasyā  tejo jihvayā kathayituṃ  na śakyate/      \N2
%tasyā  tejo jihvayā kathayituṃ  na śakyate //    \D
%tasyās tejo jihvayā kathatuṃ   na śakyate        \U1
%tasyās tejo jihvayā vaktuṃ     na śakyate //       \U2
%-----------------------
%It's not possible to describe her shine with speech (lit. with the tongue).
%-----------------------
\note[type=testium, labelb=63, lem={tasyās tejo}]{Ysg: yasyās tejo jihvayā kathituṃ na śakyate}
\app{\lem[wit={ceteri}, alt={tasyās}]{tasyā\skp{s-te}}
   \rdg[wit={N1,N2,D}]{tasyā}}\skm{s-te}jo jihvayā
 \app{\lem[wit={ceteri}]{kathayituṃ}
    \rdg[wit={L}]{kathyituṃ}
    \rdg[wit={U1}]{kathatuṃ}
    \rdg[wit={U2}]{vaktuṃ}}
  na śakyate/
%-----------------------
% \om                                                                    \B
%tasyāḥ mūrter dhyānakāraṇāt    puruṣasya śarīraṃ sthiraṃ bhavati //     \E
%tasyā  mūrter dhyānakaraṇāt    -------------------------------------    \P
%tasyā  mūrtir dhyānakaraṇāt // puruṣasya śarīraṃ sthiram bhavati //     \L
%tasyāḥ mūrter dhyānakaraṇāt    puruṣasya śarīraṃ sthiraṃ bhavati /      \N1
%tasyāḥ mūrter dhyānakaraṇāt    puruṣasya śarīraṃ sthiraṃ bhavati//      \N2
%tasyāḥ mūrter dhyānakaraṇāt    puruṣasya śarīraṃ sthiraṃ bhavati /      \D
%tasā          dhyānakaraṇāt    sādhakasya śarīraṃ sthiraṃ bhavati /cha/ \D2
%tasyāḥ mūrter dhyānakaraṇāt    puruṣasya śarīraṃ sthiraṃ bhavati vā     \U1
%tasyāḥ        dhyānakaraṇāt    puruṣasya śarīraṃ sthiraṃ bhavati //     \U2
%-----------------------
%Through the execution of meditation on this (divine) form the body of the person is going to be strong.   
%-----------------------
\note[type=testium, labelb=64, lem={tasyāḥ mūrter}]{Ysg: tasā dhyānakaraṇāt sādhakasya śarīraṃ sthiraṃ bhavati |cha|}
  \app{\lem[wit={ceteri}]{tasyāḥ}
  \rdg[wit={P,L}]{tasyā}}
  \app{\lem[wit={ceteri}, alt={mūrter}]{mūrte\skp{r-dhyā}}
      \rdg[wit={L}]{mūrtir}
      \rdg[wit={U2}]{\om}}\skm{r-dhyā}na\app{\lem[wit={ceteri}, alt={°karaṇāt}]{karaṇāt}
      \rdg[wit={L}]{karaṇāt ||}
      \rdg[wit={E}]{°kāraṇāt}}
\app{\lem[wit={ceteri}]{puruṣasya}
  \rdg[wit={P}]{\om}}
\app{\lem[wit={ceteri}]{śarīraṃ}
  \rdg[wit={P}]{\om}}
\app{\lem[wit={ceteri}]{sthiraṃ}
  \rdg[wit={P}]{\om}}    
  \app{\lem[wit={ceteri}]{bhavati}
    \rdg[wit={U1}]{bhavati vā}
    \rdg[wit={P}]{\om}}\dd{}
 \end{prose}
\end{ekdosis}
%%%%%%%%%%%%%%%%
%%%%%%%%%%%%%%%
%%%%%%%%%%%%%%%
%%%%%%%%%%%%%%
%%%%%%%%%%%%%%%
\begin{ekdosis}
    \ekddiv{type=ed}
  \ekddiv{type=ed}
   \bigskip
    \centerline{\textrm{\small{[Fourth Cakra]}}}
    \bigskip
    \begin{prose}
\note[type=source, labelb=65, lem={caturthaṃ}]{Ysv (PT): anāhatam aṣṭapīṭhaṃ [mahāpīṭhaṃ (YK)] caturthakamalaṃ hṛdi | sūryapatraṃ mahājyotir mahāsūkṣman tu cākṣuṣam | sūryapatraṃ dvādaśadalam [sentence \om in YK] | tanmadhye 'ṣṭadalaṃ padmamūrddhavaktraṃ mahāprabham |}
%-----------------------
% \om                                                   \B
%caturthaṃ hṛdayamadhye dvādaśadalaṃ kamalaṃ vartate/   \E
%caturthaṃ hṛdayamadhye dvadaśadalaṃ kamalaṃ varttate/  \P
%caturthaṃ hṛdayamadhye dvadaśadalaṃ kamalaṃ varttate/  \L
%caturthaṃ hṛdayamadhye dvadaśadalaṃ kamalaṃ varttate/  \N1
%caturthacakrakamalaṃ hṛdayamadhye dvadaśadalaṃ bhavati \N2    
%caturthaṃ hṛdayamadhye dvadaśadalaṃ kamalaṃ varttate   \D
%caturthaṃ hṛdayamadhye dvadaśadalaṃ kamalaṃ varttate/  \U1   
%caturthaṃ hṛdayamadhye dvadaśadalaṃ kamalam asti/      \U2
%
% anāhatacakraṃ hṛdayasthānaṃ // śvetavarṇaṃ tamoguṇaḥ // rudrodevatā // umāśaktiḥ // hiraṇyagarbhaṛṣiḥ // naṃdivāhanaṃ // prāṇavāyuḥ // jyotiḥ kalākāraṇaṃ dehe // suṣuptir avasthā // paśyaṃtivācā // sāmavedaḥ // gārhasyatyogniḥ? // śivaliṇgaṃ // prāptibhūmikā // sarūpatāmuktiḥ // dvādaśādalāni //dvādaśamātrā // kaṃ khaṃ gaṃ ghaṃ ṇaṃ caṃ chaṃ jaṃ jhaṃ yaṃ taṃ thaṃ // bahirmātrā // rudrāṇī // tejasā // tāpinī // sukhadā // caitanyā // śivadā // śānti // umā // gaurī // mātara // jvālā // prajvālinī // ajapājapasahasra // 6000 gha. 96 pa. 40 // U2
%-----------------------
%The fourth lotus having twelve-petals exists in the middle at the heart. \extra{[The] Anāhatacakras place is within the heart\footnote{This seems to be redundant.}. The color is white. The quality is \textit{tamas}. The deity is Rudra. The power is Umā. The Ṛṣi is Hiraṇyagarbha. The mount is Nandi. The vitalwind is Prāṇa. In the body it is the light that causes parts (\textit{kalākaraṇa})\footnote{What is this?!}. The state is deep sleep. The speech is \textit{Paśyantī}\footnote{Add footnote of entry in \textit{Tāntrikābhidhānakośa}.}.The [Veda] is Sāmaveda. The fire is Gārhapatya\footnote{Add explanation.}. The Liṅgam is Śivaliṅga. The ability to attain everything on the earth [and] the uniform liberation [are attributed to this \textit{cakra}]. [There are] twelve petals, [and] twelve measures: kaṃ khaṃ gaṃ ghaṃ ṇaṃ caṃ chaṃ jaṃ jhaṃ yaṃ taṃ [and] thaṃ. The external measure: Rudra's wife, light (\textit{tejasā?}), glow, \textit{sphakadā}?, consciousness (\textit{caitanyā}), bestower of grace, peace, Umā, Gaurī, Mātara, the flame [and] Prajvālinī.}
%-----------------------
\note[type=testium, labelb=66, lem={caturthaṃ}]{Ysg: hṛdayamadhye dvadaśadalaṃ}
\app{\lem[wit={ceteri}]{caturthaṃ}
      \rdg[wit={N2}]{caturthacakrakamalaṃ}} hṛdayamadhye dvādaśadalaṃ
    \app{\lem[wit={ceteri}]{kamalaṃ}
       \rdg[wit={N2}]{\om}} 
    \app{\lem[wit={ceteri}]{vartate}
       \rdg[wit={U2}]{asti}
       \rdg[wit={N2}]{bhavati}}/
       %%%%%%%%%%%%%%%%%
       %%%%%%%%%%%%%%%%
       %%%%%%%%%%%%%%%%%%
       %%%%%%%%%%%%%%%%%
       %%%%%%%%%%%%%%%%
      \extra{anāhatacakraṃ hṛdayasthānaṃ\dd{}
        \app{\lem[type=emendation, resp=egoscr]{śvetaṃ}
          \rdg[wit={U2}]{\korr śveta°}} varṇaṃ\dd{}
        tamo guṇaḥ\dd{}
        rudro devatā\dd{}
        umā śaktiḥ\dd{}
        hiraṇyagarbha ṛṣiḥ\dd{}
        nandi vāhanaṃ\dd{}
        \app{\lem[type=emendation, resp=egoscr]{prāṇo}
          \rdg[wit={U2}]{\korr prāṇa°}} vāyuḥ\dd{}
        \app{\lem[type=emendation, resp=egoscr]{jyotiskalākāraṇaṃ deham}
          \rdg[wit={U2}]{\korr jyotiḥ kalākāraṇaṃ dehe}}\dd{}
        suṣuptir-avasthā\dd{}
        \app{\lem[type=emendation, resp=egoscr]{paśyantī}
          \rdg[wit={U2}]{\korr paśyaṃti}} vācā\dd{}
        sāmavedaḥ\dd{}
        \app{\lem[type=emendation, resp=egoscr]{gārhapatyo 'gniḥ}
          \rdg[wit={U2}]{\korr gārhasyatyo gniḥ}}\dd{}
        \app{\lem[type=emendation, resp=egoscr]{śivo}
          \rdg[wit={U2}]{\korr śiva°}} liṅgaṃ\dd{}
        \app{\lem[type=emendation, resp=egoscr]{prāptiḥ}
          \rdg[wit={U2}]{\korr prāpti°}} bhūmikā\dd{}
        sarūpatā muktiḥ\dd{}
        dvādaśādalāni\dd{}
        dvādaśamātrā\dd{}
        kaṃ khaṃ gaṃ ghaṃ ṇaṃ caṃ chaṃ jaṃ jhaṃ yaṃ taṃ thaṃ\dd{}
        bahir-mātrā\dd{}
        rudrāṇī\dd{}
        tejasā\dd{}
        tāpinī\dd{}
        sukhadā\dd{}
        caitanyā\dd{}
        śivadā\dd{}
        \app{\lem[type=emendation, resp=egoscr]{śāntiḥ}
          \rdg[wit={U2}]{\korr śānti}}\dd{}
        umā\dd{}
        gaurī\dd{}
        \app{\lem[type=emendation, resp=egoscr]{mātarā} %%%?????
          \rdg[wit={U2}]{\korr mātara}}\dd{}
        jvālā\dd{}
        prajvālinī\dd{}
        \app{\lem[type=emendation, resp=egoscr]{ajapājapaḥ}
          \rdg[wit={U2}]{\korr ajapājapaḥ}} \app{\lem[type=emendation, resp=egoscr]{sahasraḥ}
          \rdg[wit={U2}]{\korr sahasra}}\dd{} 6000\dd{} gha. 96 pa. 40\dd{}}
  %%%%%%%%%%%%%
  %%%%%%%%%%%%%%
  %%%%%%%%%%%%%
  %%%%%%%%%%%%%%
  %%%%%%%%%%%%%%%
%-----------------------
% \om                                        \B
%atitejomayatvād   dṛṣṭigocaraṃ na bhavati   \E  
%atitejomayatvāt   dṛṣṭigocaraṃ na bhavati   \P
%atitejomayatvād   dṛṣṭigocaraṃ na bhavati// \L
%atitejomayatvāt / dṛṣṭigocaraṃ na bhavati/ \N1
%atitejomayatvāt   dṛṣṭigocaraṃ na bhavati/ \N2
%atitejomayatvāt / dṛṣṭigocaraṃ na bhavati/ \D
%atitejomayatvāt / dṛṣṭigocaraṃ na bhavati/ \U1
%atitejomayatvād   dṛṣṭigocaratāṃ na yāti// \U2 
%-----------------------
%Due to being made of [such an] intense light [the fourth lotus] is not in the range of sight.
%-----------------------
\note[type=testium, labelb=67, lem={dṛṣṭigocaraṃ}]{Ysg: tejomayatvāt | dṛṣṭigocaraṃ na bhavaty etādṛśaṃ vartate}
    atitejomayatvād-dṛṣṭi\app{\lem[wit={ceteri}, alt={°gocaraṃ}]{gocaraṃ}  %SANDHI einbauen?! 
       \rdg[wit={U2}]{gocaratāṃ}}
na
\app{\lem[wit={ceteri}]{bhavati}
  \rdg[wit={U2}]{yāti}}/
%-----------------------
% \om                                               \B
%tanmadhye 'ṣṭadalam adhomukhaṃ kamalaṃ varttate // \E  
%tanmadhye 'ṣṭadale  mukhaṃ kamalaṃ varttate //     \P
%tanmadhye ṣṭadalaṃ  adhomukha--kamalaṃ vartate //  \L
%tanmadhye aṣṭadalaṃ adhomukhaṃ kamalaṃ vartate //  \N1
%tanmadhye aṣṭadalaṃ adhomukhaṃ kamalaṃ varttate//  \N2
%tanmadhye aṣṭadalaṃ adhomukhaṃ kamalaṃ vartate //  \D
%tanmadhye aṣṭadalaṃ adhomukhaṃ kamalaṃ vartate /   \U1
%tanmadhye 'ṣṭadalaṃ adhomukhaṃ kamalaṃ asti / manaś-cakre// manodevatā// bahiśaktiḥ// ātmaṛṣih// nābhimadhye sthitaṃ padmaṃ nālaṃ tasya daśāgulaṃ/ komalaṃ tasya tan nālaṃ nirmalaṃ cāpy adhomukhaṃ/ kadalīpuṣpasaṃkāśaṃ tanmadhye ca pratiṣṭhitaṃ/ mana unnaty-asaṃkalpa/ vikalpātmakam-eva ca/ pūrvadale svetavarṇe yadā viśrāmate manaḥ// dharmakīrtividyādi sadbuddhir-bhavati/ agnikoṇe āraktavarṇe nidrā ālasyamāyāmandamatir-bhavati/ dakṣiṇe kṛṣṇavarṇeti tadā krodhotpattir bhavati/ naiṛtye nīlavarṇe mamatāmatir bhavati/ paścime kapilavarṇe/ krīḍāhāsotsavotsāhamatir bhavati/ vāyav ye śāmavarṇe cintodvegamatir bhavati/ uttare pītavarṇe bhogaśṛṇgāramahodayamatir bhavati/ īśāne gauravarṇe jñānasaṃdhāne matir bhavati/} \U2
%-----------------------
\note[type=testium, labelb=68, lem={'ṣṭadalaṃ}]{Ysg: tanmadhye 'ṣṭadalaṃ adhomukhaṃ kamalaṃ ||}
    tanmadhye \app{\lem[wit={E,U2},alt={'ṣṭadalam}]{'ṣṭadala\skp{m-a}}
      \rdg[wit={P}]{'ṣṭadale}
      \rdg[wit={L}]{ṣṭadalaṃ}
      \rdg[wit={N1,N2,D,U1}]{aṣṭadalaṃ}}\app{\lem[wit={ceteri},alt={adhomukhaṃ kamalaṃ}]{\skp{m-a}dhomukhaṃ kamalaṃ}
        \rdg[wit={L}]{adhomukhakamalaṃ}
        \rdg[wit={P}]{mukhaṃ kamalaṃ}}
      \app{\lem[wit={ceteri}]{vartate}
        \rdg[wit={U2}]{asti}}/    
%%%%%%%%%%%%%%%%
%%%%%%%%%%%%%%%
%%%%%%%%%%%%%%%
%%%%%%%%%%%%%%
%%%%%%%%%%%%%%%
  \extra{manaś-cakre\dd{}
    mano devatā\dd{}
        \app{\lem[type=conjecture, resp=egoscr]{bahiśśaktiḥ}
          \rdg[wit={U2}]{\conj bahiśaktiḥ}}\dd{}   
        \app{\lem[type=emendation, resp=egoscr]{ātmā}
          \rdg[wit={U2}]{\korr ātma°}} ṛṣiḥ\dd{}
        nābhimadhye}
\end{prose}
\end{ekdosis}
\ekdpb*{}
%%%%%%%%%%%%%%%%%%%%%%%%%%%%%%%%%%%%%%%%%%
%%%%%%%%%%%%%%%%%%%%%%%%%%%%%%%%%%%%%%%%%%
%%%%%%%%PAGEBREAK%%%%%%%PAGEBREAK%%%%%%%%%
%%%%%%%%%%%%%%%%%%%%%%%%%%%%%%%%%%%%%%%%%%
%%%%%%%%%%%%%%%%PAGEBREAK%%%%%%%%%%%%%%%%%
%%%%%%%%%%%%%%%%%%%%%%%%%%%%%%%%%%%%%%%%%%
%%%%%%%%PAGEBREAK%%%%%%%PAGEBREAK%%%%%%%%%
%%%%%%%%%%%%%%%%%%%%%%%%%%%%%%%%%%%%%%%%%%
%%%%%%%%%%%%%%%%%%%%%%%%%%%%%%%%%%%%%%%%%%
%%%%%%%%%%%%%%%%%%%%%%%%%%%%%%%%%%%%%%%%%%
%%%%%%%%%%%%%%%%%%%%%%%%%%%%%%%%%%%%%%%%%%
%%%%%%%%PAGEBREAK%%%%%%%PAGEBREAK%%%%%%%%%
%%%%%%%%%%%%%%%%%%%%%%%%%%%%%%%%%%%%%%%%%%
%%%%%%%%%%%%%%%%PAGEBREAK%%%%%%%%%%%%%%%%%
%%%%%%%%%%%%%%%%%%%%%%%%%%%%%%%%%%%%%%%%%%
%%%%%%%%PAGEBREAK%%%%%%%PAGEBREAK%%%%%%%%%
%%%%%%%%%%%%%%%%%%%%%%%%%%%%%%%%%%%%%%%%%%
%%%%%%%%%%%%%%%%%%%%%%%%%%%%%%%%%%%%%%%%%%
%%%%%%%%%%%%%%%%%%%%%%%%%%%%%%%%%%%%%%%%%%
%%%%%%%%%%%%%%%%%%%%%%%%%%%%%%%%%%%%%%%%%%
%%%%%%%%PAGEBREAK%%%%%%%PAGEBREAK%%%%%%%%%
%%%%%%%%%%%%%%%%%%%%%%%%%%%%%%%%%%%%%%%%%%
%%%%%%%%%%%%%%%%PAGEBREAK%%%%%%%%%%%%%%%%%
%%%%%%%%%%%%%%%%%%%%%%%%%%%%%%%%%%%%%%%%%%
%%%%%%%%PAGEBREAK%%%%%%%PAGEBREAK%%%%%%%%%
%%%%%%%%%%%%%%%%%%%%%%%%%%%%%%%%%%%%%%%%%%
%%%%%%%%%%%%%%%%%%%%%%%%%%%%%%%%%%%%%%%%%%
\begin{ekdosis}
    \ekddiv{type=ed}
  \begin{prose}
    \noindent
\extra{sthitaṃ padmaṃ nālaṃ tasya
        \app{\lem[type=emendation, resp=egoscr]{daśāṅgulaṃ}
          \rdg[wit={U2}]{\korr daśāgulaṃ}}/ %In the middle of the navel [exists] a place, being a lotus, its tube measures ten \textit{aṅgula}s,
        komalaṃ tasya tan-nālaṃ nirmalaṃ cāpy-adhomukhaṃ/ %The fluid (\textit{komala}) of the tube is pure facing upwards.
        kadalīpuṣpasaṃkāśaṃ tanmadhye ca pratiṣṭhitaṃ/ % In its middle is a place shining like a banana-flower.
        mana \app{\lem[type=conjecture, resp=egoscr, alt={ānati}]{āna\skp{ty-a}}
          \rdg[wit={U2}]{\conj unnaty}}
      \app{\lem[type=emendation, resp=egoscr,alt={asaṃkalpam}]{\skm{ty-a}saṃkalpam}
          \rdg[wit={U2}]{\korr asaṃkalpa}}/  
        vikalpātmakam-eva ca/} %The mind isn't willing to rise up and is of changing nature.     
%%%%%%%%%%
%%%%%%%%%%
%%%%%%%%%%%%
%%%%%%%%%%%%%%
%%%%%%%%%%%%%%%
        \extra{
          pūrvadale \app{\lem[type=emendation, resp=egoscr, alt={°śveta}]{śveta}
            \rdg[wit={U2}]{\korr sveta°}}varṇe yadā \app{\lem[type=emendation, resp=egoscr]{viśramate}
            \rdg[wit={U2}]{\korr viśrāmate}} manaḥ\dd{}
        dharmakīrtividyādisadbuddhir-bhavati/ %While the mind rests on the eastern petal [which is] white in colour clear intellekt arises, which is [endowed with]  \textit{dharma}, fame and knowledge etc. 
        %%%%%
        agnikoṇe āraktavarṇe \app{\lem[type=emendation, resp=egoscr, alt={nidrālasya}]{nidrālasya}
          \rdg[wit={U2}]{\korr nidrā ālasya°}}māyāmandamatir-bhavati/  %While [the mind rests on] the south-east, [which is] reddish in color a mind that is weak due to sleep, laziness and illusion arises.
        %%%%
        dakṣiṇe kṛṣṇavarṇeti tadā krodhotpattir-bhavati/ %While [the mind is situated] in the right south, [which is] black in color the generation of anger arises.
        %%%
        \app{\lem[type=emendation, resp=egoscr]{nairṛtye}
          \rdg[wit={U2}]{\korr naiṛtye}} nīlavarṇe mamatāmatir-bhavati/ %While [the mind is situated] in the southwest, [which is] blue in color a mind of pride arises.
        %%%
        paścime kapilavarṇe krīḍāhāsotsavotsāhamatir-bhavati/ %While [the mind is situated] in the west, [which is] brown in color a mind that is longing for play, laughing, and celebration arises.
        %%%
        vāyavye \app{\lem[type=emendation, resp=egoscr, alt={°śyāma}]{śyāma}
          \rdg[wit={U2}]{\korr śāma}}varṇe cintodvegamatir-bhavati/ %While [the mind is situated] in the northwest, [which is] dark in color a mind which is restless by sorrow arises.
        %%%
        uttare pītavarṇe bhogaśṛṅgāramahodayamatir-bhavati/ %While [the mind is situated] in the north, [which is] yellow in color a very happy mind with erotic and enjoyment arises.
        īśāne gauravarṇe
        \app{\lem[type=emendation, resp=egoscr, alt={jñānasaṃdhāna°}]{jñānasaṃdhāna}
          \rdg[wit={U2}]{\korr jñānasaṃdhāne}}
        matir-bhavati/} \\%While [the mind is situated] in north-east [which is] whitish in color a mind of unity arises through knowledge arises.
  %%%%%%%%%%%%
  %%%%%%%%%%%%
  %%%%%%%%%%%%
  %%%%%%%%%%%%
  %%%%%%%%%%%%
%-----------------------
% \om                                                     \B      
%tanmadhye prāṇavāyoḥ sthānam    aṣṭadalakamalamadhye liṃgākārā karṇikā  kathyate/  \E 
%tanmadhye prāṇavāyoḥ sthānam    aṣṭadalakamalamadhye liṃgākārā karṇikā  kathyate/  \P
%tanmadhye prāṇavāyoḥ sthānam    aṣṭadalakamalamadhye liṃgākārā karṇikā  kathyate// \L
%tanmadhye prāṇavāyoḥ sthānam    aṣṭadalakamalamadhye liṃgākārā karṇikā  kathyate// \N1
%tanmadhye prāṇavāyoḥ sthānam/   aṣṭadalakamalamadhye liṃgākārā karṇikā  kathyate// \N2
%tanmadhye prāṇavāyoḥ sthānam // aṣṭadalakamalamadhye liṃgākārā karṇi    kathyate// \D
%ta ca     prāṇavāyoḥ sthānam /  aṣṭadalakamalamadhye liṃgākārā karṇikā              \D2      
%tanmadhye prāṇavāyo  sthānam    aṣṭadalakamalamadhye liṃgākārā karṇikā  kathyate    \U1
%tanmadhye prāṇavāyo  sthānam // aṣṭadalakamalamadhye liṃgākārā karṇikā  kathyate    \U2
%-----------------------
%It's said that in its middle is the place of the \textit{prāṇa}-vitalwind [and] in the middle [of] the eight-petalled lotus is a pericarp (\textit{karṇikā}) in the form of a \textit{liṅga}.
%-----------------------
      \note[type=testium, labelb=69, lem={prāṇavāyoḥ }]{Ysg: ta ca prāṇavāyoḥ sthānam | aṣṭadalakamalamadhye liṃgākārā karṇikā}
      \note[type=source, labelb=70, lem={prāṇavāyoḥ}]{Ysv (PT): prāṇavāyoḥ sthalañcāsya liṅgākāran tu karṇikā | kālikākhyā karṇikeyaṃ asyā madhye tu kuṇḍalī |}
tanmadhye prāṇa\app{\lem[wit={ceteri},alt={°vāyoḥ}]{vāyoḥ}
       \rdg[wit={U1,U2}]{°vāyo}} sthānam-aṣṭadalakamalamadhye liṃgākārā
        \app{\lem[wit={ceteri}]{karṇikā}
          \rdg[wit={U2}]{karṇi}}
kathyate/   
%-----------------------
% \om                                                     \B
%tasyāḥ karṇiketi saṃjñā tatkarṇikāmadhye    padmarāgasamānavarṇāṃ-----------guṣṭhapramāṇaikā     puttalikā varttate //  \E  
%tasyāḥ kaliketi saṃjñā tatkalikāmadhye      padmarāgaratnasamānavarṇāṃ    aṃguṣṭhapramāṇā    ekā puttalikā varttate     \P
%tasyāḥ kalikeli                 madhye      padma    ratnasamānavarṇā //  aṃguṣṭhapramāṇā // ekā puttalikā varttate //  \L
%tasyāḥ kaliketi saṃjñā tatkalikāmadhye      padmarāgaratnasamānavarṇāṃ    aṃguṣṭhapramāṇā    ekā puttalikā varttate     \N1
%tasyāḥ kaliketi saṃjñā/tataḥ kalikāmadhye   padmarāgaratnasamānavarṇa     aṃguṣṭhapramāṇā    ekā putalikā  varttate/    \N2 %%%p4recto
%tasyāḥ kaliketi saṃjñā tatkalikāmadhye      padmarāgaratnasamānavarṇā     aṃguṣṭhapramāṇāt   ekā puttalikā varttate /   \D
%tasyāḥ kaliketi saṃjñā tatkalikāmadhye      padmarāgaratnasamānavarṇā     aṃguṣṭhapramāṇāt   ekā puttalikā varttate /   \U1
%tasyāḥ kaliketi saṃjñā tatkalikāmadhye      padmarāgaratnasamānavarṇā  // aṃguṣṭhapramāṇā    ekā puttalikā varttate /   \U2
%-----------------------
%The technical designation of her is kalikā. In the middle of this kalikā exists a single thumbsized (divine) figurine (puttalikā) being similiar to a ruby-gem in color.
%-----------------------        
\note[type=testium, labelb=71, lem={kaliketi}]{Ysg: kaliketi saṃjñikāsti tanmadhye padmarāgaratnasamānavarṇā aṃguṣṭhapramāṇā ekā puttalikā}
\note[type=source, labelb=72, lem={padma°}]{Ysv (PT): padmavatyāḥ [padmāvatyāḥ (YK)] prabhāṅguṣṭhapramāṇā [°prāmāṇa° (YK)] ratnasannibhā | tasyāsaṅgī [tasya saṅgī (YK)] jīva iti ananto balarūpataḥ | asya dhyānaṃ [dhyānād (YK)] jagadvaśyaṃ khecarīsarvago bhavet | bhavanti vaśyā devādyāś cintākartturna [citta° (YK)] cānyathā | iṣṭāniṣṭo [iṣṭāniṣṭa (YK)] bhaved vaśyaḥ [vaśyaṃ (YK)] satyaṃ satyaṃ na saṃśayaḥ | iṣṭasiddhir bhavet tasya sarvajñādiguṇodayaḥ |}
tasyāḥ
\app{\lem[wit={ceteri}]{kaliketi}
  \rdg[wit={L}]{kalikeli}
  \rdg[wit={E}]{karṇiketi}}
\app{\lem[wit={ceteri}]{saṃjñā}
  \rdg[wit={L}]{\om}}
\app{\lem[wit={ceteri}]{tatkalikāmadhye}
  \rdg[wit={N2}]{tataḥ}
  \rdg[wit={L}]{\om}}
\app{\lem[type=emendation, resp=egoscr]{padmarāgaratnasamānavarṇāṅguṣṭhapramāṇaikā}
  \rdg[wit={E}]{\korr padmarāgasamānavarṇāṃguṣṭhapramāṇaikā}
  \rdg[wit={P,N1}]{padmarāgaratnasamānavarṇāṃ || aṃguṣṭhapramāṇā || ekā}
  \rdg[wit={N2}]{padmarāgaratnasamānavarṇa aṃguṣṭhapramāṇā ekā}
  \rdg[wit={L}]{padmaratnasamānavarṇā aṃguṣṭhapramāṇā ekā}
  \rdg[wit={D,U1}]{padmarāgaratnasamānavarṇā aṃguṣṭhapramāṇāt ekā}} puttalikā
vartate/
%-----------------------
%
%tasyā  jīvasaṃjñā          tasyā  balamadhyasvarūpaṃ        koṭijihvābhir  vaktuṃ naiva śakyate // \E
%tasyā  jīvasaṃjñā          tasyā  balam atha svarūpaṃ       koṭijihvābhir  vaktuṃ naiva śakyate // \P 
%tasya                             bala sappa svarūpaṃ       koṭijihvāyābhi vaktuṃ na    śakyate // \L 
%tasyāḥ jīveti saṃjñāḥ      tasyāḥ balaṃ atha ca svarūpaṃ    koṭijihvābhir  vaktuṃ na    śakyate // \N1
%tasyāḥ jīveti saṃjñaḥ//    tasyā  balaṃ atha ca svarūpaṃ    koṭijihvābhir  vaktuṃ na    śakyate // \N2
%tasyāḥ jīveti saṃjña/      tasyāḥ balaṃ atha ca svarūpaṃ    koṭijihvābhir  vaktuṃ na    śakyate // \D
%       jīveti saṃjñikāsti/ tasyāḥ balaṃ         svarūpaṃ ca koṭijihvābhir  vaktuṃ na    śakyaṃ  //  D2
%tasyāḥ jīveti saṃjñā       tasyāḥ balaṃ atha ca svarūpaṃ    koṭijihvābhir  vaktuṃ na    śakyate // \U1
%tasyā  jīvasaṃjñā//        tasya  balaṃ tasya atha svarūpaṃ koṭijihvābhir  vaktuṃ na    śakyate // \U2
%-----------------------  
%Her technical designation is embodied soul. Not even with a thousand tongues it is possible to talk about her nature and her power.
%-----------------------
\note[type=testium, labelb=73 lem={jīveti}]{Ysg: jīveti saṃjñikāsti | tasyāḥ balaṃ svarūpaṃ ca koṭijihvābhir vaktuṃ na śakyaṃ ||}
\app{\lem[wit={E,P}]{tasyā}
     \rdg[wit={N1,N2,D,U1}]{tasyāḥ}
     \rdg[wit={L}]{tasya}}
\app{\lem[wit={U2}]{jīveti saṃjñā}
     \rdg[wit={N1}]{jīveti saṃjñāḥ}
     \rdg[wit={N2}]{jīveti saṃjñaḥ ||}
     \rdg[wit={D}]{jīveti saṃjña |}
     \rdg[wit={E,P,U2}]{jīvasaṃjñā ||}
     \rdg[wit={L}]{\om}}
\app{\lem[wit={E,N2,P}]{tasyā}
     \rdg[wit={N1,D,U1}]{tasyāḥ}
     \rdg[wit={U2}]{tasya}}
\app{\lem[wit={ceteri}]{balaṃ atha ca svarūpaṃ}
     \rdg[wit={P}]{balam atha svarūpaṃ}
     \rdg[wit={U2}]{balaṃ tasya atha svarūpaṃ}
     \rdg[wit={L}]{bala sappa svarūpaṃ}
     \rdg[wit={E}]{balamadhyasvarūpaṃ}}
\app{\lem[wit={ceteri}, alt={koṭijihvābhir}]{koṭijihvābhi\skp{r-va}}
    \rdg[wit={L}]{koṭijihvāyābhi}}\skp{r-va}ktuṃ
\app{\lem[wit={ceteri}]{na}
    \rdg[wit={E,P}]{naiva}}
śakyate/
%-----------------------
% \om \B
%asyā  mūrter   dhyānakāraṇāt           svarga-pātāl--ākaśamanuṣyagandharvakinnaraguhyakavidyādharalokasambandhinyaḥ strīyo 'pi--------------------       vaśyā bhavanti / \E
%asyā  mūrter   dhyānakaraṇāt           svarga-pātāl--ākāśamanuṣyagandharvakiṃnaraguhyakavidyādharalokasaṃbaṃdhinyaḥ strīyo 'pi--------------------       vaśyā bhavanti / \P
%asyā  mūrtir   dhyānāt                 svarga-pātāl--ākāśamanuṣyagaṃdharvakinnaraguhyakavidyādharalokasambandhinyaḥ strīyo 'pi--------------------       vaśyā bhavanti /L
%asyāḥ mūrter  dhyānakaraṇāt            svarga-pātāla ākāśamanuṣyagaṃdharvakinnaraguhyakavidyādharalokasaṃbaṃdhinyaḥ strīyaḥ sādhakasya puruṣasya         vaśyā bhavanti // \N1
%asyā  mūrttir dhyānakaraṇāt/           svarga-pātāla ākāśamanuṣya/ gaṃdharvakinnara/ guhyaka/vidyādhara/lokasaṃbaṃdhinyaḥ strīyaḥ sādhakasya puruṣasya   vaśyo bhavati/ \N2
%asyāḥ mūrter  dhyānakaraṇāt            svarga-pātāla ākāśamanuṣyagaṃdharvakiṃnaraguhyakavidyādharalokasaṃbaṃdhinyaḥ strīyaḥ sādhakasya puruṣasya         vaśyā bhavanti // \D
%asyāḥ mūrter  dhyānakaraṇāt            svarga-pātāla ākāśamanuṣyagaṃdharvakiṃnaraguhyakavidyādharalokasaṃbaṃdhinyaḥ strīyaḥ sādhakasya puruṣasya         vaśyā bhavanti // \U1
%tasyāḥ mūrter dhyānaṃ karaṇāt //       svarga-pātāl--ākāśamanuṣyagandharvakinnaraguhyakavidyādharalokasaṃbadhinya---striyo  pi---------------------------vaśyā bhavaṃti // \U2
%-----------------------
%“Because of the exercise of meditation on this form the inhabitants of the universe (which are) Humans, Gandharvas, Kinnaras, Guhyakas, Vidyādharas and (their) females, in the heavenly world, underworld and open space are obedient to the will of the practicing person.”, is what is said here.  
%-----------------------
\note[type=testium, labelb=74, lem={svarga°}]{Ysg: :asyā mūrtter dhyānakaraṇāt sādhakasya svargapātāla ākāśagaṃdharvakiṃnaraguhyakavidyādharastrīyo vaśā bhavati |}
\app{\lem[wit={ceteri}]{asyā}
    \rdg[wit={N1,D,U1}]{asyāḥ}
    \rdg[wit={U2}]{tasyāḥ}}
 \app{\lem[wit={ceteri}, alt={mūrter}]{mūrte\skp{r-dhyā}}
    \rdg[wit={L,N2}]{mūrtir}}\app{\lem[wit={ceteri}, alt={dhyānakāraṇāt}]{\skm{r-dhyā}nakāraṇā\skp{t-sva}}
    \rdg[wit={U2}]{dhyānaṃ karaṇāt ||}
    \rdg[wit={L}]{dhyānāt}
  }\skm{t-sva}rga\app{\lem[wit={E,P,L,U2},alt={°pātālākaśa°}]{pātālākaśa}
    \rdg[wit={D,N1,N2,U1}]{°pātāla ākāśa°}}\app{\lem[wit={ceteri},alt={°manuṣyagandharvakinnaraguhyakavidyādharaloka°}]{manuṣyagandharvakinnaraguhyakavidyādharaloka}
    \rdg[wit={N2}]{°manuṣya| gaṃdharvakinnara| guhyaka| vidyādhara| loka°}
  }\app{\lem[wit={ceteri}]{saṃbandhinyaḥ}
    \rdg[wit={U2}]{saṃdadhinya}}
 \app{\lem[wit={ceteri}]{strīyaḥ sādhakasya puruṣasya}
    \rdg[wit={E,P,L}]{strīyo 'pi}
    \rdg[wit={U2}]{striyo pi}}
 \app{\lem[wit={ceteri}]{vaśyā bhavanti}
   \rdg[wit={N2}]{vaśyo bhavati}}/
\note[type=philcomm, labelb=75, lem={vaśyā bhavanti}]{D\textsubscript{2} adds: pṛthvī loke manuṣyādi striṇāṃ kākathā cha |}
%tanmadhye koṭicaṃdrasamaprabhaḥ ekaḥ puruṣo varttate  \N1bhavanti/\note[type=philcomm, labelb=s16, lem={bhavanti}]{\getsiglum{U1} adds a flawed phrase hereafter: \textit{pṛtvī lokasaṃbaṃdhanyo pi striyaḥ vaśyā bhavaṃti/}. I refrained to include it in the apparatus due to its redundance.}
%-----------------------
% \om \B 
%ityatra kathyate// /E
%ityatra kathyate// \P
%ityatra kathyate// \L
%ityatra kiṃ kathyate // \N1
%ityatra kiṃ kathyate// \N2
%ityaṃtra kiṃ kathyate // \D
%ityatra kiṃ kathyate vā \U1
%ityatra kathyate // \U2
%-----------------------
%is what is said here.  
%-----------------------  
ity-atra  
\app{\lem[wit={ceteri}]{kiṃ}
  \rdg[wit={E,P,L,U2}]{\om}}
\app{\lem[wit={ceteri}]{kathyate}
  \rdg[wit={U1}]{kathyate vā}}\dd{}
\end{prose}
\end{ekdosis}
\ekdpb*{}
%%%%%%%%%%%%%%%%%%%%%%%%%%%%%%%%%%%%%%%%%%
%%%%%%%%%%%%%%%%%%%%%%%%%%%%%%%%%%%%%%%%%%
%%%%%%%%PAGEBREAK%%%%%%%PAGEBREAK%%%%%%%%%
%%%%%%%%%%%%%%%%%%%%%%%%%%%%%%%%%%%%%%%%%%
%%%%%%%%%%%%%%%%PAGEBREAK%%%%%%%%%%%%%%%%%
%%%%%%%%%%%%%%%%%%%%%%%%%%%%%%%%%%%%%%%%%%
%%%%%%%%PAGEBREAK%%%%%%%PAGEBREAK%%%%%%%%%
%%%%%%%%%%%%%%%%%%%%%%%%%%%%%%%%%%%%%%%%%%
%%%%%%%%%%%%%%%%%%%%%%%%%%%%%%%%%%%%%%%%%%
%%%%%%%%%%%%%%%%%%%%%%%%%%%%%%%%%%%%%%%%%%
%%%%%%%%%%%%%%%%%%%%%%%%%%%%%%%%%%%%%%%%%%
%%%%%%%%PAGEBREAK%%%%%%%PAGEBREAK%%%%%%%%%
%%%%%%%%%%%%%%%%%%%%%%%%%%%%%%%%%%%%%%%%%%
%%%%%%%%%%%%%%%%PAGEBREAK%%%%%%%%%%%%%%%%%
%%%%%%%%%%%%%%%%%%%%%%%%%%%%%%%%%%%%%%%%%%
%%%%%%%%PAGEBREAK%%%%%%%PAGEBREAK%%%%%%%%%
%%%%%%%%%%%%%%%%%%%%%%%%%%%%%%%%%%%%%%%%%%
%%%%%%%%%%%%%%%%%%%%%%%%%%%%%%%%%%%%%%%%%%
%%%%%%%%%%%%%%%%%%%%%%%%%%%%%%%%%%%%%%%%%%
%%%%%%%%%%%%%%%%%%%%%%%%%%%%%%%%%%%%%%%%%%
%%%%%%%%PAGEBREAK%%%%%%%PAGEBREAK%%%%%%%%%
%%%%%%%%%%%%%%%%%%%%%%%%%%%%%%%%%%%%%%%%%%
%%%%%%%%%%%%%%%%PAGEBREAK%%%%%%%%%%%%%%%%%
%%%%%%%%%%%%%%%%%%%%%%%%%%%%%%%%%%%%%%%%%%
%%%%%%%%PAGEBREAK%%%%%%%PAGEBREAK%%%%%%%%%
%%%%%%%%%%%%%%%%%%%%%%%%%%%%%%%%%%%%%%%%%%
%%%%%%%%%%%%%%%%%%%%%%%%%%%%%%%%%%%%%%%%%%
\begin{ekdosis}
  \ekddiv{type=ed}
    \centerline{\textrm{\small{[Fifth Cakra]}}}
    \bigskip  
    \begin{prose}
\noindent
%-----------------------      
%-------------pañcamaṃ kaṇṭhasthāne ṣoḍaśadalaṃ kamalaṃ                         vartate //  \E
%-------------paṃcamaṃ kaṃṭhasthāne ṣoḍaśadalaṃ kamalaṃ                         vartate     \P
%-------------paṃcamaṃ kaṃṭhasthāne ṣoḍaśadalaṃ kamalaṃ                         vartate     \L
%idānīṃ       paṃcamaṃ kamalaṃ      ṣodaśadalaṃ                   kaṃṭhasthāne varttate // \N1
%idānīṃ       paṃcamaṃ kamalaṣodaśadalaṃ                          kaṃṭhasthāne varttate // \N2
%idānīṃ       paṃcamaṃ kamalaṃ      ṣodaśadalaṃ                   kaṃṭhasthāne  varttate // \D --------> Was in diesem Falle machen?
%idānīṃ       paṃcamaṃ kamalaṃ      ṣodaśadalaṃ                   kaṃṭhasthāne        varttate // \U1
%-------------paṃcamaṃ                          viśuddhacakraṃ    kaṃṭhastāne              \U2     
%-----------------------
%Now (follows the description of) the fifth lotus having sixteen petals (which) exists at the location of the throat.
%-----------------------
%U2 continues: dhūmra?varṇe jīvodevatā// avidyāśaktiḥ// virāṭrṣiḥ// vāyurvāhanaṃ// udānavāyuḥ// jvālākalā jālaṃdharobaṃdhaḥ mahākāraṇadeha// tūryāvasthā// parāvācā// atharvaṇavedaḥ// jaṃgamaliṅgaṃ jīvaprāptābhūmikā// sāyujyatāmokṣaḥ// ṣoḍaśadalāni// ṣoḍaśamātrāḥ// atarmātrār-carāḥ// aṃ āṃ iṃ īṃ u ūṃ ṛṃ ṝṃ ḷṃ ḹṃ eṃ aiṃ oṃ auṃ aṃ aṃḥ// bahirmātrā vidyā// avidyā// ichā// śakti// jñānaśaktiḥ// śatalā// mahāvidyā// mahāmāyā// buddhiḥ// tamasī// maitrā?// kumārī// maitrāyaṇī// rudrā// puṣṭa// siṃhanī// ajapājapasahasra/ 1000 gha. 2 pa. 46 akṣara 40//
%-----------------------     
%The colour is smoke-colour. The deity is the embodied soul (\textit{jīva}). The power is ignorance (\textit{avidyā}). The Ṛṣi is Virāṭ\footnote{Who is this?}. The mount is the vitalwind (\textit{vāyu}). The vitalwind is \textit{udāna}. Its Kalā is the flame. The \textit{bandha} is Jālandhara. The body supra-causal (\textit{mahākāraṇa}). The state is the fourth state (\textit{tūrya}). The speech is Parā\footnote{Im Kaśm. Śiv. °das ewige Wort, in welchem potentiell alle Begriffe und Worte ruhen; vgl. das śabdabrahma des Vyākaraṇa. [B.]― Schmidt S. 246}. The [Veda is] Atharvaṇa Veda. The \textit{liṅga} is the living. The level is Jīvaprāptā\footnote{What is this?}. The liberation is absorption into the divine essence (\textit{sāyujyatā}). [There are] sixteen petals [and] sixteen matrices. The internal matrix: aṃ āṃ iṃ īṃ u ūṃ ṛṃ ṝṃ ḷṃ ḹṃ eṃ aiṃ oṃ auṃ aṃ aṃḥ. The external matrix: Vidyā ``she who is knowledge'', Avidyā ``she who is ignorance'', Icchā ``she who is desire'', Śakti ``she who is power'', Jñānaśakti ``she who is the power of knowledge'', Śatalā ``she who is manifold'', Mahāvidyā ``she who is great knowledge'', Mahāmayā ``she who is great illusion'', Buddhi ``she who is intellect'', Tamasī ``she who is darkness'', Maitrā ``she who is love'', Kumārī ``she who is a young girl'', Maitrāyaṇī ``she who is???'', Rudrā ``she who is howling'', Puṣṭā ``she who is abundance'', Siṃhanī ``she who is a lioness''. A thousandfold recitation of the non-recited; 1000 [repetitions for]; 2 \textit{ghaṭi}s, 46 \textit{palā}s. and 40 \textit{akṣara}s.
%-----------------------  
\note[type=testium, labelb=76, lem={paṃcamaṃ}]{Ysg: kaṃṭhasthāne paṃcamaṃ ṣodaśadalaṃ viśudhhasaṃjñakaṃ cakraṃ varttate ||}
\note[type=source, labelb=77, lem={paṃcamaṃ}]{Ysv (PT=YK): iṣṭasiddhir bhavet tasya sarvajñādiguṇodayaḥ | kalāpatraṃ pañcaman tu viśuddhaṃ kaṇṭhadeśataḥ | asya madhye pumān ekaḥ koṭicandrasamaprabhaḥ | naśyantya sādhyarogā hi sahasrāyuś ca cintanāt |}
\app{\lem[wit={N1,N2,D,U1}]{idānīṃ}
        \rdg[wit={ceteri}]{\om}}
pañcamaṃ
\app{\lem[wit={N1,D,U1}]{kamalaṃ ṣodaśadalaṃ kaṇṭhasthāne}
  \rdg[wit={N2}]{kamalaṣodaśadalaṃ kaṇṭhasthāne}
  \rdg[wit={E,P,L}]{kaṇṭhasthāne ṣoḍaśadalaṃ kamalaṃ}
  \rdg[wit={U2}]{viśuddhacakraṃ kaṃṭhastāne}}
vartate/
  %%%%%%%%%%%
  %%%%%%%%%%%
  %%%%%%%%%%%
  %%%%%%%%%%%
  %%%%%%%%%%%
      \extra{\app{\lem[type=emendation, resp=egoscr]{dhūmraṃ varṇaṃ}
          \rdg[wit={U2}]{\korr dhūmravarṇe}}\dd{}
        jīvo devatā\dd{}
        avidyā śaktiḥ\dd{}
        \app{\lem[type=emendation, resp=egoscr]{virāṭ}
          \rdg[wit={U2}]{\korr virāṭha}} ṛṣiḥ\dd{}
        vāyur-vāhanaṃ\dd{}
        \app{\lem[type=emendation, resp=egoscr]{udāno}
          \rdg[wit={U2}]{\korr udāna°}} vāyuḥ\dd{}
        jvālā kalā\dd{}
        jālaṃdharo bandhaḥ\dd{}
        \app{\lem[type=emendation, resp=egoscr]{mahākāraṇaḥ dehaḥ}
          \rdg[wit={U2}]{\korr mahākāraṇadeha}}\dd{}
        \app{\lem[type=emendation, resp=egoscr]{tūrya āvasthā}
          \rdg[wit={U2}]{\korr tūryāvasthā}}\dd{}
        parā vācā\dd{}
        \app{\lem[type=emendation, resp=egoscr]{atharvaṇo}
          \rdg[wit={U2}]{\korr atharvaṇa}} vedaḥ\dd{}
        \app{\lem[type=emendation, resp=egoscr]{jaṅgamaṃ}
          \rdg[wit={U2}]{\korr jaṃgama°}} liṅgaṃ\dd{}
        jīvaprāptā bhūmikā\dd{}
        sāyujyatā mokṣaḥ\dd{}
        ṣoḍaśadalāni\dd{}
        ṣoḍaśamātrāḥ\dd{}
        \app{\lem[type=emendation, resp=egoscr]{antarmātrā}
          \rdg[wit={U2}]{\korr antarmātrār carāḥ}}\dd{}  %%%what does carā here mean? I emend to the formulation found for the U2 additions in the previous cakra 
        aṃ āṃ iṃ īṃ u ūṃ ṛṃ ṝṃ ḷṃ ḹṃ eṃ aiṃ oṃ auṃ aṃ aṃḥ\dd{}
        bahir-mātrā\dd{}
        vidyā\dd{}
        avidyā\dd{}
        \app{\lem[type=emendation, resp=egoscr]{icchā}
          \rdg[wit={U2}]{\korr ichā}}\dd{}
        \app{\lem[type=emendation, resp=egoscr]{śaktiḥ}
          \rdg[wit={U2}]{\korr śakti}}\dd{}
        jñānaśaktiḥ\dd{}
        śatalā\dd{}
        mahāvidyā\dd{}
        mahāmāyā\dd{}
        buddhiḥ\dd{}
        \app{\lem[type=emendation, resp=egoscr]{tāmasī}
          \rdg[wit={U2}]{\korr tamasī}}\dd{} %%%She who is darkness????
        maitrā\dd{}
        kumārī\dd{}
        maitrāyaṇī\dd{} %%%what's this??? 
        rudrā\dd{}
        \app{\lem[type=emendation, resp=egoscr]{puṣṭā}
          \rdg[wit={U2}]{\korr puṣṭa°}}\dd{}
        siṃhanī\dd{}
        \app{\lem[type=emendation, resp=egoscr]{ajapājapaḥ sahasraḥ}
          \rdg[wit={U2}]{\korr ajapājapasahasra}}\dd{} 1000\dd{} gha. 2 pa. 46 akṣara 40\dd{}}%%%%%Kolloquium besprechen! Was is akṣara? 
 %%%%%%%%%%%
 %%%%%%%%%%%
 %%%%%%%%%%%
 %%%%%%%%%%%
 %%%%%%%%%%%
%----------------------- 
%tanmadhye koṭisūryasamāna       ekaḥ puruṣo vartate / \E
%tanmadhye koṭicaṃdrasamaprabhaḥ ekaḥ puruṣo vartate   \P
%tanmadhye koṭicaṃdrasamaprabhā  ekaḥ puruṣo vartate   \L
%tanmadhye koṭicaṃdrasamaprabhaḥ ekaḥ puruṣo varttate  \N1
%tanmadhye koṭicaṃdrasamaprabhaḥ ekaḥ puruṣo varttate  \N2
%tanmadhye koṭicaṃdrasamaprabhā  eka--puruṣo varttate  \D
%tatra     koṭicaṃdraprabha      ekaḥ puruṣo sti       \D2
%tanmadhye koṭicaṃdrasamaprabhaḥ ekaḥ puruṣo varttate  \U1
%tanmadhye koṭicaṃdrasamaprabhaḥ // eka pumān varttate // \U2
%----------------------- 
%In its  middle exists a single person which shines like a thousand moons.
%----------------------- 
\note[type=testium, labelb=78, lem={koṭicaṃdra°}]{Ysg: tatra koṭicaṃdraprabha ekaḥ puruṣo sti}
 tanmadhye
koṭi\app{\lem[wit={ceteri}, alt={°candrasamaprabhaḥ}]{candrasamaprabhaḥ}
  \rdg[wit={U2}]{°caṃdrasamaprabhaḥ ||}
  \rdg[wit={L,D}]{°caṃdrasamaprabhā}
  \rdg[wit={E}]{°caṃdrasūryasamāna}}
\app{\lem[wit={ceteri}]{ekaḥ puruṣo}
  \rdg[wit=D]{ekapuruṣo}
  \rdg[wit={U2}]{eka pumān}}
vartate/
%----------------------- 
%tasya puruṣasya dhyānakāraṇād--- asādhyarogā naśyanti // \E
%tasya puruṣasya dhyānakāraṇād--- asādhyarogā naśyanti // \L
%tasya puruṣasya dhyānakāraṇād--- asādhyarogā naśyaṃti // \P
%tasya puruṣasya dhyānakaraṇāt--  asādhyarogā naśyaṃti // \N1
%tasya puruṣasya dhyānakaraṇāt    asādhyarogā naśyaṃti    \N2
%tasya puruṣasya dhyānakaraṇāt /  asādhyarogā naśyaṃti // \D
%tasya puruṣasya dhyānakaraṇāt /  asādhyarogā naśyaṃti    \U1
%tasya puṃsaḥ    dhyānakaraṇāt // asādhyarogā naśyaṃti // \U2
%----------------------- 
%Because of the exercise of meditation on this person all diseases which are (otherwise) not possible to be controlled vanish.
%----------------------- 
\note[type=testium, labelb=79, lem={asādhyarogā}]{Ysg: tasya puruṣasya dhyānakaraṇād asādhyarogā naśyaṃti ||}
tasya
\app{\lem[wit={ceteri}]{puruṣasya}
  \rdg[wit={U2}]{puṃsaḥ}}
\app{\lem[wit={ceteri}, alt={dhyānakāraṇād}]{dhyānakaraṇā\skp{d-a}}
  \rdg[wit={N1,N2}]{dhyānakaraṇāt}
  \rdg[wit={D,U1,U2}]{dhyānakaraṇāt |}}\skm{d-a}sādhyarogā naśyanti/
%----------------------- 
%ekasahasravarṣaparyaṃtaṃ sa puruṣo jīvatīdānīṃ     \E
%ekasahasravarṣaparyaṃtaṃ sa puruṣo jīvati          \P
%ekasahasravarṣa             puruṣo jīvati //       \L
%ekasahasravarṣaparyaṃtaṃ    puruṣo jīvati /        \N1
%ekasahasravarṣaparyaṃta     puruṣo jīvati /        \N2
%ekasahasravarṣaparyaṃtaṃ    puruṣo jīvati /        \D
%ekasahasravarṣaparyaṃtaṃ    puruṣo jīvati / cha    \U1
%ekasahasravarṣaparyaṃtaṃ    puruṣo jīvati //       \U2
%----------------------- 
%The person lives up to 1001 years.
%----------------------- 
\note[type=testium, labelb=80, lem={°varṣa°}]{Ysg: sahasravarṣaṃ jīvati |}
ekasahasravarṣa\app{\lem[wit={ceteri},alt={°paryantaṃ}]{paryantaṃ}
  \rdg[wit={N2}]{°paryaṃta}
  \rdg[wit={L}]{\om}}
\app{\lem[wit={ceteri}]{puruṣo}
  \rdg[wit={E,P}]{sa puruṣo}}
  \app{\lem[wit={ceteri}]{jīvati}
    \rdg[wit={U1}]{jīvati |cha|}
    \rdg[wit={E}]{jīvatīdānīṃ}}\dd{}
    \end{prose}
  \end{ekdosis}
%%%%%%%%%%%%%%%%
%%%%%%%%%%%%%%%%
%%%%%%%%%%%%%%%%
%%%%%%%%%%%%%%%
%%%%%%%%%%%%%%%%
\begin{ekdosis}
  \ekddiv{type=ed}
   \bigskip
    \centerline{\textrm{\small{[Sixth Cakra]}}}
    \bigskip
    \begin{prose}
%----------------------- 
%īdānīṃ ṣaṣṭhaṃ bhrūmadhye ājñācakraṃ                vartate//   \E
%īdānīṃ ṣaṣṭhaṃ bhrūmadhye ājñācakraṃ                vartate//   \P
%īdānīṃ ṣaṣṭhaḥ bhrūmadhye ājñācakraṃ                vartate//   \L
%idānīṃ ṣaṣṭhacakraṃ       ajñānāmakaṃ               varttate // \N1
%idānīṃ ṣaṣṭhacakraṃ       ajñānāmaka                varttate    \N2
%idānīṃ ṣaṣṭhacakraṃ       ajñānāmakaṃ               varttate // \D
%idānīṃ ṣaṣṭhacakraṃ       ājñānāmakaṃ               vartate     \U1
%idānīṃ ṣaṣṭa   bhrūmadhye ājñācakraṃ raktavarṇaṃ //             \U2
%-----------------------
%āgnirdevatā suṣumṇāśaktiḥ// hiṃsaṛṣiḥ// caitanyavāhanaṃ// jñānadehī// vijñānāvathā// anupamavācā// sāmadevaḥ// pramādaliṃgaṃ// ardhamātrā// ākāśātatvaṃ// jīvahiṃsa// caitanyalīlraṃbhaḥ// dvemātrā// hiṃkṣaṃ// aṃtarmātrā// bahirmātrā//sthiti//prabhā?// ajapājapasahasra// 1000 gha. 2 pa. 46 akṣara 40// \U2
%-----------------------
%The deity is fire. The power is the godess of the centre (\textit{suṣumṇā}). The Ṛṣi is ``the violent'' (\textit{hiṃsa}). The mount is consciousness (\textit{caitanya}. The body is knowledge. The state is understanding. The speech is the ``incomparable'' (\textit{anupama}). The [Veda] is Sāmaveda.The \textit{liṅgaṃ} is intoxication (\textit{pramāda}). The half-measure: the reality of ether, ``the violence of living'' (\textit{jīvahiṃsa}) [and] the origin of the play of Conciousness. Two measures: haṃ kṣam. The inner measure is external measure: maintenance of life (\textit{sthiti}) [and] splendour (\textit{prabhā}).
%-----------------------
\note[type=source, labelb=81, lem={ajñā°}]{Ysv (PT): ājñākhyaṃ ṣaṣṭhakaṃ [ṣaṭkaṃ (YK)] cakraṃ bhruvor madhye dvipatrakam | agnijvālānibhaṃ jyotiḥ puṃsaḥ strīto [pūṃsastrīto (YK)] vivarjitam | dhyānāc cāsya sarvasiddhirajarāmaratāṃ vrajet |}
\note[type=testium, labelb=82, lem={ajñā°}]{Ysg: bhrūvor madhye dvidalaṃ ājñācakraṃ ṣaṣṭhaṃ |}
idānīṃ
    \app{\lem[wit={ceteri}]{ṣaṣṭhacakraṃ}
       \rdg[wit={E,P}]{ṣaṣṭhaṃ bhrūmadhye}
       \rdg[wit={L}]{ṣaṣṭhaḥ bhrūmadhye}
       \rdg[wit={U2}]{ṣaṣṭa bhrūmadhye}}
     \app{\lem[wit={ceteri}]{ājñā}
      \rdg[wit={N1,N2,D}]{ajñā}
    }\app{\lem[wit={U1,D,N1}]{nāmakaṃ}
       \rdg[wit={E,P,L}]{cakraṃ}
       \rdg[wit={U2}]{cakraṃ raktavarṇaṃ}
       \rdg[wit={N2}]{nāmaka}}
vartate/
  %%%%%%%%%%%%%%%
  %%%%%%%%%%%%%%
  %%%%%%%%%%%%%%
  %%%%%%%%%%%%%%
  %%%%%%%%%%%%%%
     \extra{\app{\lem[type=emendation, resp=egoscr, alt={agnir}]{agni\skp{r-de}}
         \rdg[wit={U2}]{\korr āgnir}
       }\skm{r-de}vatā\dd{}
       suṣumṇā śaktiḥ\dd{}
       \app{\lem[type=emendation, resp=egoscr]{hiṃso}
         \rdg[wit={U2}]{\korr hiṃsa°}} ṛṣiḥ\dd{}
       \app{\lem[type=emendation, resp=egoscr]{caitanyaṃ}
         \rdg[wit={U2}]{\korr caitanya°}} vāhanaṃ\dd{}
       \app{\lem[type=emendation, resp=egoscr]{jñāno dehaḥ}
         \rdg[wit={U2}]{\korr jñānadehī}}\dd{}
       vijñānāvasthā\dd{}
       \app{\lem[type=emendation, resp=egoscr]{anupamā}
         \rdg[wit={U2}]{\korr anupama°}} vācā\dd{}
       sāmavedaḥ\dd{}
       \app{\lem[type=emendation, resp=egoscr]{pramādaḥ}
         \rdg[wit={U2}]{\korr pramāda°}} liṃgaṃ\dd{}
       \app{\lem[type=emendation, resp=egoscr]{ardhā mātrā}
         \rdg[wit={U2}]{\korr ardhamātrā}}\dd{}}
\end{prose}
\end{ekdosis}
\ekdpb*{}
%%%%%%%%%%%%%%%%%%
%%%%%%%%%%%%%%%%%%
%%%%PAGEBREAK%%%%
%%%%%%%%%%%%%%%%%%
%%%%%%%%%%%%%%%%%%
\begin{ekdosis}
    \ekddiv{type=ed}
  \begin{prose}
\noindent
    \extra{\app{\lem[type=emendation, resp=egoscr]{ākāśaṃ}
         \rdg[wit={U2}]{\korr ākāśā}}tattvaṃ\dd{}
       \app{\lem[type=emendation, resp=egoscr]{jīvo haṃsaḥ}
         \rdg[wit={U2}]{\korr jīvahiṃsa}}\dd{}
       caitanya\app{\lem[type=emendation, resp=egoscr, alt={°līlā}]{līlā āraṃbhaḥ}
         \rdg[wit={U2}]{\korr °līlāraṃbhaḥ}}\dd{}
       dve mātrā\dd{}
       haṃ kṣaṃ\dd{}
       aṃtar-mātrā\dd{}
       bahir-mātrā\dd{}
       \app{\lem[type=emendation, resp=egoscr]{sthitiḥ}
         \rdg[wit={U2}]{\korr sthiti}}\dd{}
       prabhā\dd{}
       \app{\lem[type=emendation, resp=egoscr]{ajapājapaḥ sahasraḥ}
         \rdg[wit={U2}]{\korr ajapājapasahasra}}\dd{} 1000\dd{} gha. 2 pa. 46 akṣara 40\dd{}}     
%----------------------- 
                                       %dvidalaṃ tanmadhye  'gnijvālākārakamalaṃ     kiṃcid vastu vartate/    \E
                                       %dvidalaṃ tanmadhye  agnijvālākārakamalaṃ     kiṃcid vastu vartate/    \P
                                       %dvidalaṃ tanmadhye  agnijvālākārakamalaṃ     kiṃcid vastu vartate/    \L
%                                                           agnijvālākārakamalaṃ     kiṃcid vastu vartate/    \B
%tac cakraṃ bhruvor madhye dvidalakaṃ sthitaṃ // tanmadhye  agnijvālākāraṃ akalaṃ    kiṃcid vastu varttate/   \N1
%tac-cakraṃ bhruvor-madhye dvidalakaṃ sthitaṃ /  tanmadhye  agnijvālākāraṃ akalaṃ    kiṃcid-vastu vartate/    \N2
%tac cakraṃ bhruvor madhye dvidalakaṃ sthitaṃ // tanmadhye  agnijvālākāraṃ akalaṃ    kiṃcid vastu varttate/   \D
%tac-cakraṃ bhruvor-madhye dvidalakaṃ sthitaṃ    tanmadhye  agnijvālākāraṃ akala     kiṃcit vastu vartate/    \U1  
%                                                tanmadhye  agnijvālākārakamalaṃ //  kiṃcid-vastu varttate/   \U2   
%-----------------------    
\note[type=testium, labelb=83, lem={agnijvālā°}]{Ysg: agnijvālākāraṃ paramātmasaṃjñakaṃ vastvāsti |}
     \app{\lem[wit={ceteri}, alt={tac cakraṃ bhruvor madhye dvidalakaṃ sthitaṃ}]{tac\skp{-}cakraṃ bhruvor-madhye dvidalakaṃ sthitaṃ}
     \rdg[wit={E,P,L}]{dvidalaṃ}
     \rdg[wit={U2}]{\om}}
tanmadhye
   \app{\lem[wit={N1,N2,D}]{'gnijvālākāraṃ akalaṃ}
     \rdg[wit={ceteri}]{agnijvālākāraṃ akalaṃ}
     \rdg[wit={U1}]{agnijvālākāraṃ akala}}
   \note[type=philcomm, labelb=84, lem={agnijvālākāra°}]{Witness B starts here.}
   kiṃcid-vastu vartate/
%-----------------------  
%na strī pumān     / tasya dhyānakāraṇāt  puruṣasya  śarīraṃ  ajarāmaraṃ bhavati /     \E
%na strī pumān    // tasyā dhyānakaraṇāt  puruṣasya  śarīraṃ  ajarāmaro  bhavati /     \B
%na strī pumān    // tasyā dhyānakaraṇāt  puruṣasya  śarīraṃ  ajarāmaro  bhavati /     \L
%na strī na pumān // tasyā dhyānakaraṇāt  puruṣasya  śarīraṃ  ajarāmaro  bhavati /     \P
%na strī na pumān /  tasya dhyānakaraṇāt  puruṣasya  śarīraṃ  ajarāmaraṃ bhavati      \N1
%na strī na pumān /  tasya dhyānakaraṇāt  puruṣasya  śarīraṃ  ajarāmaraṃ bhavati //   \N2
%na strī na pumān /  tasya dhyānakaraṇāt  puruṣasya  śarīraṃ  ajarāmaraṃ bhavati      \D
%na strī na pumān    tasya dhyānakaraṇāt  puruṣasya  śarīraṃ  ajarāmaraṃ bhavati vā   \U1
%na strī na pumān /  tasya dhyānakāraṇāt/ puruṣasya--śarīram--ajarāmaraṃ bhavati /    \U2   
%-----------------------
\note[type=testium, labelb=85, lem={na strī}]{Ysg: tac ca na strīpumān | tasya dhyānakaraṇād ajarāmaraḥ sādhako bhavati |cha|}
\app{\lem[wit={ceteri}]{na strī na pumān}
     \rdg[wit={E,B,L}]{na strī pumān}}/
  tasya dhyāna\app{\lem[wit={ceteri},alt={°karaṇāt}]{karaṇāt}
    \rdg[wit={U2}]{°karaṇāt |}}
puruṣasya
  \app{\lem[wit={U2}, alt={śarīram ajarāmaraṃ}]{śarīram\skp{-}ajarāmaraṃ}
    \rdg[wit={E,N1,N2,D,U1}]{śarīraṃ ajarāmaraṃ}
    \rdg[wit={B,L,P}]{śarīraṃ ajarāmaro}}
   \app{\lem[wit={ceteri}]{bhavati}
     \rdg[wit={U2}]{bhavati vā}}\dd{}   
 \end{prose}
\end{ekdosis}
%%%%%%%%%%%%%%%
%%%%%%%%%%%%%%%
%%%%%%%%%%%%%%%
%%%%%%%%%%%%%%%
%%%%%%%%%%%%%%%
\begin{ekdosis}
  \ekddiv{type=ed}
   \bigskip
    \centerline{\textrm{\small{[Seventh Cakra]}}}
    \bigskip
    \begin{prose}
%-----------------------
% idānīṃ saptamaṃ  tālumadhye catuḥṣaṣṭidalaṃ              amṛtapūrṇaṃ vartate / \E
% idānīṃ saptamaṃ  tālumadhye catuḥṣaṣṭhidalaṃ             amṛtapūrṇaṃ vartate / \P
% idānīṃ saptamaṃ  // tāludeśe madhye catuḥṣaṣṭhidala      amṛtapūrṇaṃ vartate / \L
% idānīṃ saptamaṃ  // tāludeśe madhye catuḥṣaṣṭhidala      amṛtapūrṇaṃ vartate / \B
% idānīṃ saptamaṃ  cakraṃ     catuḥṣaṣṭhidalaṃ tālumadhye  amṛtapūrṇaṃ varttate // \N1
% idānīṃ saptamaṃ  cakraṃ     catuṣaṣṭhidalaṃ tālumadhye   amṛtapūrṇa  varttate // \N2      
% idānīṃ saptamaṃ  cakraṃ     catuḥṣaṣṭhidalaṃ tālumadhye  amṛtapūrṇaṃ varttate // \D
% idānīṃ saptamaṃ  cakraṃ     catuḥṣaṣṭhidalaṃ tālumadhye  amṛtapūrṇaṃ varttate // \U1
% idānīṃ saptamaṃ  tālumadhye catuḥṣaṣṭidalaṃ //           amṛtapūrṇaṃ vartate / \U2      
%-----------------------
% Now the seventh cakra having 64 petals and being full of nectar exists in the middle of the palate.
%-----------------------
%U2: \extra{lalāṭa maṃḍalaṃ// caṃdro devatā// amṛtā śaktiḥ// paramātmā ṛṣiḥ// amṛtavāsinīkalāsaptadaśī amṛtakallolanadī// mahākāśa// aṃbikā// laṃbikā// ghaṃṭikā// tālikā// ajapāgāyatrīdehasvarūpaṃ// kākamukhī// naranetrāgośṛṃgālalāṭabrahmapaṭhāhayagrīvā// mayūramukhā// haṃsavadaṃgāni// ajapāgāyatrīsvarūpaṃ// 
%-----------------------
      \note[type=testium, labelb=86, lem={tālu°}]{Ysg: tālumadhye catuḥṣaṣṭhidalaṃ amṛtapūrṇaṃ}
      \note[type=source, labelb=87, lem={tālu°}]{Ysv (PT): catuḥṣaṣṭidalaṃ tālumadhye cakran tu madhyamam | pīyūṣapūrṇaṃ [pīyūṣapūrṇa° (YK)] koṭīndusannibhaṃ [°sannibha° (YK)] cāmṛtasthalī | tanmadhye ghaṭikāsaṃjñā karṇikā raktasannibhā | saha cendukalā tatrāmṛtadhārāṃ [tābdrā° (YK)] sravaty asau | etad dhyātvāmṛtaiḥ snātvā sadā yogāt pramucyate | unmādajvarapittādidāhaśūlādivedanāḥ [°śūnyā° (YK)] | naśyanti ca śiroduḥkhaṃ jāḍyabhāvo 'pi naśyati | sadyodhyānādbhuktaviśvaṃ jihvājāḍyañ ca naśyati [last sentence \om in YK] |}
idānīṃ saptamaṃ
      \app{\lem[wit={N1,D,U1}]{cakraṃ catuḥṣaṣṭhidalaṃ tālumadhye}
        \rdg[wit={N2}]{cakraṃ catuṣaṣṭhidalaṃ tālumadhye}
        \rdg[wit={E,P,U2}]{tālumadhye catuḥṣaṣṭidalaṃ}
        \rdg[wit={L,B}]{tāludeśe madhye catuḥṣaṣṭhidala}}
      \app{\lem[type=emendation, resp=egoscr]{'mṛtapūrṇaṃ}
        \rdg[wit={ceteri}]{\korr amṛtapūrṇaṃ}
        \rdg[wit={N2}]{amṛtapūrṇa}}
vartate/
  %%%%%%%%%%%%%%
  %%%%%%%%%%%%%%
  %%%%%%%%%%%%%%
  %%%%%%%%%%%%%%
      \extra{\app{\lem[type=emendation, resp=egoscr]{lalāṭaṃ}
          \rdg[wit={U2}]{\korr lalāṭa°}} maṇḍalaṃ\dd{}
        caṃdro devatā\dd{}
        amṛtā śaktiḥ\dd{}
        paramātmā ṛṣiḥ\dd{}
        amṛtavāsinī kalāsaptadaśī\dd{}
        amṛtakallolanadī \app{\lem[type=emendation, resp=egoscr]{mahākāśā}
          \rdg[wit={U2}]{\korr mahākāśa}}\dd{}
        aṃbikā laṃbikā\dd{}
        ghaṃṭikā tālikā\dd{}
        ajapāgāyatrī dehasvarūpaṃ\dd{}
        kākamukhī\dd{}
        naranetrā\dd{}
        gośṛṃgā\dd{}
        lalāṭabrahmapaṭhā\dd{}
        hayagrīvā\dd{}
        mayūramukhā\dd{}
        haṃsavad-aṃgāni\dd{}  
        ajapāgāyatrī svarūpaṃ\dd{}\note[type=philcomm, labelb=88, lem={lalāṭaṃ maṇḍalaṃ}]{This additional passage is found in U2 only. Suprisingly after the additions to this \textit{cakra}, the scribe/author of these additions does'nt add instructions for the duration of practice as before.}}
        %%%%%%%%%%%%%%%
        %%%%%%%%%%%%%%%
        %%%%%%%%%%%%%%%
        %%%%%%%%%%%%%%%
        %%%%%%%%%%%%%%%
%-----------------------
%adhikaśobhāyuktam-----atiśvetaṃ          tanmadhye       raktavarṇaṃ ghāṃṭikāsaṃjñaikā      karṇikā varttate / \E 
%adhikataraśobhayuktaṃ atiśvetaṃ          tanmadhye       raktavarṇaṃ ghaṭikāsaṃjñā ekā      karṇikā varttate / \P
%adhikataraśobhayuktaṃ // atiśvetaṃ //    tanmadhye       raktavarṇaṃ ghaṇikāsaṃjñā ekā ekā  karṇikā varttate / \L
%adhikataraśobhayuktaṃ // atiśvetaṃ //    tanmadhye       raktavarṇaṃ ghaṃṭikāsaṃjñā ekā ekā karṇikā varttate / \B
%adhikataraśobhayuktaṃ atiśvetaṃ          tanmadhye       raktavarṇaṃ ghaṃṭikāsaṃjñā ekā     karṇikā varttate / \N1
%adhikataraśobhāyuktaṃ  atiśvetaṃ         tanmadhye       raktavarṇa--ghaṇṭikāsaṃjñā ekā     karṇikā vartate /  \N2
%adhikataraśobhayuktaṃ atiśvetaṃ          tanmadhye       raktavarṇaṃ ghaṃṭikāsaṃjñā ekā     karṇikā varttate / \D
%adhikataraśobhayuktaṃ atiśvetaṃ          tanmadhye       raktavarṇaṃ ghaṃṭikāsaṃjñā ekā     karṇikā varttate / \U1      
%adhikataraprabhāmuktaṃ // atiśvetaṃ //   tanmadhye       raktavarṇaṃ ghaṃṭikāsaṃjñā// ekā   karṇikā varttate / \U2   
%-----------------------
%[It is] endowed with superabundant beauty. [It is] very bright. In its middle, red in color [is that] known as "uvula" (\textit{ghāṃṭikā}). [It] exists as a single pericarp.  
%-----------------------      
\note[type=testium, labelb=89, lem={adhikatara°}]{Ysg: adhikataraśobhayuktaṃ atiśvetaṃ cakraṃ | tanmadhye raktavarṇaghaṃṭikāsaṃjñā varttate |}
      adhi\app{\lem[wit={ceteri},alt={°kataraśobhayuktaṃ}]{kataraśobhayuktaṃ}
        \rdg[wit={N2}]{°kataraśobhāyuktaṃ}
        \rdg[wit={E}]{°kaśobhāyuktam}
        \rdg[wit={U2}]{°kataraprabhāmuktaṃ}}\dd{}
      \app{\lem[wit={ceteri}]{atiśvetaṃ}
        \rdg[wit={L,B,U2}]{||atiśvetaṃ||}}\dd{}
        tanmadhye
        \app{\lem[wit={ceteri}]{raktavarṇaṃ}
          \rdg[wit={N2}]{raktavarṇa°}}
        \app{\lem[wit={ceteri},alt={ghaṇṭikā°}]{ghaṇṭikā}
          \rdg[wit={E}]{ghāṃṭikā°}
          \rdg[wit={P}]{ghaṭikā°}
          \rdg[wit={L}]{ghaṇikā°}}saṃjñā/
        \app{\lem[wit={ceteri}]{ekā}
          \rdg[wit={L,B}]{ekā ekā}}
          karṇikā vartate/
%-----------------------          
%tanmadhye bhūmiḥ / \E
%tanmadhye bhūmiḥ / \P
%tanmadhye bhūmiḥ / \L
%tanmadhye bhūmiḥ / \B
%tanmadhye bhūmiḥ / \N1
%tanmadhye bhūmiḥ / \N2
%tanmadhye bhūmiḥ / \D
%tanmadhye bhūmis- / \U1
%tanmadhye bhūmi   / \U2         
%-----------------------
%In its middle is a place. 
%-----------------------        
tanmadhye
\app{\lem[wit={ceteri}]{bhūmiḥ}
     \rdg[wit={U1}]{bhūmis°}
     \rdg[wit={U2}]{bhūmi}}/
%-----------------------  
%tanmadhye prakaṭacandrakalā 'mṛtādhārā bhavati         / \E
%tanmadhye prakaṭacandrakalā 'mṛtādhārā sravati         / \P
%tanmadhye prakaṭacandrakalā 'mṛtādhārā sravaṃti        / \L
%tanmadhye prakaṭacandrakalā 'mṛtādhārā sravaṃti        / \B
%tanmadhye prakaṭacandrakalā amṛtādhārāsravaṃtī varttate/ \N1
%tanmadhye prakaṭacaṃdrakalā amṛtādhārāsravaṃtī varttate/ \N2
%tanmadhye prakaṭacandrakalā 'mṛtādhārāsravaṃtī varttate/ \D %sravantī f. Fluss Nom Sg
%tanmadhye pragaṭacaṃdrakalā amṛtadhārāsravaṃtī varttate  \U1
%tanmadhye-ṃdrakaṭaṃ caṃdrakalā amṛtadhārā sravati       /\U2       
%-----------------------
%In its middle exists a hidden digit of the moon, being a stream of nectar like a river (\textit{amṛtādhārāsravantī}. 
%-----------------------
\note[type=testium, labelb=90, lem={prakaṭa°}]{Ysg: tanmadhye prakaṭacandrakalā amṛtādhārāsravaṃtī varttate |}
tanmadhye
       \app{\lem[wit={ceteri},alt={prakaṭa°}]{'prakaṭa}
         \rdg[wit={U1}]{pragaṭa}
         \rdg[wit={U2}]{°ṃdrakaṭaṃ}}candrakalā
       \app{\lem[wit={ceteri}]{amṛtadhārāsravantī}
         \rdg[wit={L,B}]{'mṛtādhārā sravaṃti}
         \rdg[wit={P,U2}]{'mṛtādhārā sravati}
         \rdg[wit={E}]{'mṛtādhārā bhavati}}
       \app{\lem[wit={N1,N2,D,U1}]{vartate}
         \rdg[wit={ceteri}]{\om}}/
\end{prose}
\end{ekdosis}
\ekdpb*{}
%%%%%%%%%%%%%%%%%%
%%%%%%%%%%%%%%%%%%
%%%%PAGEBREAK%%%%
%%%%%%%%%%%%%%%%%%
%%%%%%%%%%%%%%%%%%
\begin{ekdosis}
 \ekddiv{type=ed}
 \begin{prose}
   \noindent
%-----------------------
%tasyāḥ kalāyā     dhyānakāraṇāt tasya samīpe maraṇaṃ nāyāti/     \E -> does not come near to death -> na-ā-yāti
%tasyāḥ kalāyā     dhyānakaraṇāt tasya samīpe maraṇaṃ nāyāti/     \P
%tasyāḥ karṇikāyā  dhyānakaraṇāt tasya samīpe maraṇaṃ na yāti     \L
%tasyāḥ karṇikāyā  dhyānakaraṇāt tasya samīpe maraṇaṃ na yāti     \B
%tasyāḥ kalāyāḥ    dhyānakaraṇāt tasya samīpe maraṇaṃ nāyāti      \N1
%tasyāḥ kalāyāḥ    dhyānakaraṇāt tasya samīpe maraṇaṃ nāyāti/     \N2       
%tasyāḥ kalāyāḥ    dhyānakaraṇāt tasya samīpe maraṇaṃ nāyāti      \D
%tasyāḥ kalāyā     dhyānakaraṇāt tasya samīpe maraṇaṃ nāyāti/     \U1
%tasyāḥ kalāyā     dhyānakāraṇāt// tasya samīpe maraṇaṃ na yāti/  \U2
%-----------------------
%Because of the exercise of meditation on this digit death does not come near him. 
%-----------------------
\note[type=testium, labelb=91, lem={maraṇaṃ}]{Ysg: tasyāḥ kalāyā nirantaraṃ dhyānakartum maraṇaṃ}
       tasyāḥ
       \app{\lem[wit={ceteri}]{kalāyā}
         \rdg[wit={N1,N2,U1}]{kalāyāḥ} %Sandhi-mistake in apparatus in this case?
         \rdg[wit={L,B}]{karṇikāyā}}
dhyānakaraṇāt tasya samīpe maraṇaṃ
       \app{\lem[wit={ceteri}]{nāyāti}
         \rdg[wit={L,B,U2}]{na yāti}}/
%-----------------------
%nirantaradhyānād        -amṛtadhārāyāḥ sajīvo bhavati /    \E
%niraṃtaradhyānāt---------amṛtadhārā plāvanaṃ   bhavati /   \P
%niraṃtaradhyānakaraṇād   amṛtadhārā           sravati /    \L
%niraṃtaradhyānakaraṇād   amṛtadhārā           sravati /    \B
%niraṃtaradhyānakaraṇāt / amṛtadhārā           sravaṃti /   \N1
%niraṃtaradhyānakaraṇāt   amṛtadhārā            sravaṃti    \N2
%niraṃtaradhyānakaraṇāt / amṛtadhārā           sravaṃti /   \D   
%niraṃtaradhyānakaraṇāt   amṛtadhārā             sravati /  \U1
%niraṃtaradhyānakaraṇāt / amṛtadhārā plavanaṃ  bhavati /    \U2
%-----------------------
%Due to uninterrupted meditation the stream (\textit{dhārā}) of nectar flows. 
%-----------------------
nirantara\app{\lem[wit={ceteri},alt={°dhyānakaraṇād}]{dhyānakaraṇā\skm{d-a}}
        \rdg[wit={E,P}]{°dhyānād}}
      \app{\lem[wit={ceteri}, alt={amṛtadhārā}]{\skp{d-a}mṛtadhārā}
         \rdg[wit={E}]{amṛtadhārāyāḥ sajīvo}
         \rdg[wit={P}]{amṛtadhārā plāvanaṃ}
         \rdg[wit={U2}]{amṛtadhārā plavanaṃ}}
       \app{\lem[wit={L,B,U1}]{sravati}
         \rdg[wit={N1,N2,D}]{sravaṃti}
         \rdg[wit={E,P,U2}]{bhavati}}/       
%-----------------------
%tadā  yakṣam-aroga----pittajvarahṛdayadāha-śiroroga-jihvā--jaḍa-bhāvā           naśyanti / \E
%tadā     kṣayaroga----pittajvarahṛdayadāha-śiroroga-jihvā--jaḍa-bhāvān          naśyanti / \P
%tadā     kṣayaroga----pittajvarahṛdayadāha-----roga-jihvāyājaḍa-bhāvān          naśyanti / \L
%tadā     kṣayaroga----pittajvarahṛdayadāha-----roga-jihvāyājaḍa-vān             naśyanti / \B
%         kṣayarogaṃ   pittajvarahṛdayadāha-śiroroga-jihvāyājaḍa-bhāvā           naśyanti / \N1 %besser kṣayarogaṃ emendieren zu vollem Kompositum?
%         kṣayarogaṃ   pittajvarahṛdayadāha-śiroroga-jihvāyājaḍa-bhāvātā         naśyanti / \N2
%         kṣayaṃ rogaṃ pittajvarahṛdayadāha-śiroroga-jihvāyājaḍa-bhāvā           naśyanti / \D
%         kṣayaroga----pittajvarahṛdayadāha-śiroroga-jihvāyājaḍa-bhāvā           naśyanti / \U1  
%tadā     kṣayarogo----ptatti// jvara hṛdayadāha// śiroroga// jihvājaḍatā// dayo naśyanti /cha/ \U2       
%-----------------------
%Then the appearances of emaciation (\textit{kṣayaroga}), fever due to disordered bile (\textit{pittajvara), heartburn (\textit{hṛdayadāha}), head-disease (\textit{śiroroga}) and tongue insensibility (\textit{jihvājaḍa}) vanish. %!!!Krankheiten in Ayurvedabuch checken! medizinische Identifikationen!
%-----------------------
\note[type=testium, labelb=92, lem={kṣaya°}]{Ysg: kṣayarogaḥ pettajvarahṛdayadāhaśiro..jihvāyājaḍyaṃ ca naśyati |}
       \app{\lem[wit={E,P,L,B,U2}]{tadā}
         \rdg[wit={ceteri}]{\om}}
       \app{\lem[type=emendation, resp=egoscr]{kṣayarogapittajvarahṛdayadāhaśirorogajihvājaḍabhāvā}
         \rdg[wit={E}]{\korr yakṣamarogapittajvarahṛdayadāhaśirorogajihvājaḍabhāvā}
         \rdg[wit={P}]{kṣayarogapittajvarahṛdayadāhaśirorogajihvājaḍabhāvān}
         \rdg[wit={L}]{kṣayarogapittajvarahṛdayadāharogajihvāyājaḍabhāvān}
         \rdg[wit={B}]{kṣayarogapittajvarahṛdayadāharogajihvāyājaḍavān}
         \rdg[wit={N1}]{kṣayarogaṃ pittajvarahṛdayadāhaśirorogajihvāyājaḍabhāvā}
         \rdg[wit={N2}]{kṣayarogaṃ pittajvarahṛdayadāhaśirorogajihvāyājaḍabhāvātā}
         \rdg[wit={D}]{kṣayaṃ rogaṃ pittajvarahṛdayadāhaśirorogajihvāyājaḍabhāvā}
         \rdg[wit={U1}]{kṣayarogapittajvarahṛdayadāhaśirorogajihvāyājaḍabhāvā}
         \rdg[wit={U2}]{kṣayarogoptatti || jvara hṛdayadāha || śiroroga || jihvājaḍatā || dayo}}
naśyanti/
%-----------------------       
%bhakṣitam--api   viṣan    na bādhate  / \E
%bhakṣitam--api   viṃṣa    na bādhate  / \P
%bhākṣitam--api   viṣaṃ    na bādhyate / \L
%bhākṣitamār pi   viṣaṃ    na bādhyate / \B
%bhakṣitam        viṣamapi na bādhyate / \N1
%bhakṣitaṃ        viṣamapi na bādhate  / \N2
%bhakṣitāṃ        viṣamapi na bādhyate / \D
%bhakṣitaṃ        viṣamapi na bādhyate   \U1
%bhakṣitam--api   viṣaṃ    na bādhyate / \U2       
%-----------------------       
%Also eaten venom doesn't trouble him. 
%-----------------------
         \app{\lem[wit={N2,U1}]{bhakṣitaṃ}
           \rdg[wit={N1}]{bhakṣitam}
           \rdg[wit={D}]{bhakṣitāṃ}
           \rdg[wit={E,P,L,U2}]{bhakṣitam api}
           \rdg[wit={B}]{bhākṣitamār pi}}
         \app{\lem[wit={N1,N2,D,U1}, alt={viṣam api}]{viṣam-api}
           \rdg[wit={L,B,U2}]{viṣaṃ}
           \rdg[wit={E}]{viṣan}
           \rdg[wit={P}]{viṃṣa}}
na \app{\lem[wit={E,P,N2}]{bādhate}
           \rdg[wit={ceteri}]{bādhyate}}/     
%-----------------------
%yady-atra manaḥ sthiraṃ   bhavati /  \E
%yady-atra manaḥ sthiraṃ   bhavati /  \P
%yady-atramapi manasthiraṃ bhavati /  \L              %VARIANTE UNSICHER!!!WAS MEINT JÜRGEn??
%yady-atramapi manasthiraṃ bhavati /  \B
%yady-atra     manasthiraṃ bhavati /  \N1
%yadyanna      manasthiraṃ bhavati // \N2
%yadyanna      manasthiraṃ bhavati /  \D
%yadyatra      manasthiraṃ bhavati    \U1
%yadyatra      manasthiraṃ bhavati//  \U2       
%-----------------------
%If here the mind becomes stable.       
%-----------------------
         \app{\lem[wit={ceteri}]{yadyatra}
           \rdg[wit={L,B}]{yadyatram api}
           \rdg[wit={N1,D}]{yadyanna}}
         \app{\lem[wit={E,P}]{manaḥ sthiraṃ}
           \rdg[wit={ceteri}]{manasthiraṃ}}
         \app{\lem[wit={ceteri}]{bhavati}}\dd{}         
    \end{prose}
  \end{ekdosis}
%%%%%%%%%%%%%%% 
%%%%%%%%%%%%%%%
%%%%%%%%%%%%%%%
%%%%%%%%%%%%%%%
%%%%%%%%%%%%%%%
\begin{ekdosis}
  \ekddiv{type=ed}
   \bigskip
    \centerline{\textrm{\small{[Eighth Cakra]}}}
    \bigskip
    \begin{prose}
%-----------------------
%idānīṃ brahmarandhrasthāne 'ṣṭamaṃ śatadalaṃ cakraṃ varttate / \E
%idānīṃ brahmaraṃdhrasthāne 'ṣṭamaṃ śatadalaṃ cakraṃ vartate / \P
%idānīṃ brahmaraṃdhrasthāne aṣṭamaṃ śatadalaṃ cakraṃ vartate / \L
%idānīṃ brahmaraṃdhrasthāne aṣṭamaṃ śatadalaṃ cakraṃ vartate / \B
%idānīṃ aṣṭamacakraṃ brahmaraṃdhrasthāne śatadalaṃ   vartate / \N1
%idānīṃ aṣṭamacakraṃ brahmaraṃdhrasthāne śatadalaṃ   vartate  \N2
%idānīṃ aṣṭamacakraṃ brahmaraṃdhrasthāne śatadalaṃ   vartate / \D
%idānīṃ aṣṭamaṃ cakraṃ brahmaraṃdhrasthāne śatadalaṃ   vartate . \U1
%idānīṃ brahmaraṃdhrasthāne 'ṣṭamaṃ śatadalaṃ cakraṃ varttate // \U2
%-----------------------
%guru devatā// caitanya śaktiḥ// virāṭ ṛṣiḥ// sarvotkṛṣṭasākṣiḥ// bhūtaturyātītacaitanyātmakaṃ// sarvavarṇāḥ// sarvamātrāḥ// sarvadalāni virāṭdeha sthitāvasthā prajñāvācā sohaṃ veda anupamasthānaṃ// ajapājapasahasra/ 1000 gha 02 pa 046 akṣara 40// sarvajapasaṃkhyā// 21600// ekaviṃśatisahasrāṇiṣaṭśatāni// tathaivaca niśāhevahate// prāṇaḥ yojānātisapaṃḍitaḥ// sakāreṇa bahiryātihakāreṇaviśotpunaḥ// haṃsaḥ sohaṃ// tato maṃtraṃ jīvojapati sarvadā//    
%-----------------------
%Now exists the eigth \textit{cakra} having one hundred petals located at the aperture of Brahman.
%-----------------------
      \note[type=testium, labelb=93, lem={śatadala}]{Ysg: brahmaraṃdhre śatadalaṃ}
      \note[type=source, labelb=94, lem={śatadala}]{Ysv (PT): brahmarandhre 'ṣṭamaṃ cakraṃ śatapatraṃ mahāprabham | jālandharaṃ nāma pīṭhaṃ etat tu parikīrttitam | siddhapuṃsaḥ [°puṃsa° (YK)] sthalaṃ jñātvā agnidhūmanibhā śikhā | ādimadhyāntahīnā strīpuṃmūrtti [°mūrtir (YK)] varttate parā | antajñānī [antaryāmī (YK)] bhaved dhyānād ākāśe 'pi samāgamaḥ | nirantaraṃ sarvavettā ity ūccāno mahān bhavet | jaganmadhye sthito jantur jagadbādhāvivarjitaḥ |}    
idānīṃ
\app{\lem[wit={N1,N2,D}]{aṣṭamacakraṃ brahmaraṃdhrasthāne śatadalaṃ}
    \rdg[wit={E,P,U2}]{brahmarandhrasthāne 'ṣṭamaṃ śatadalaṃ cakraṃ}
    \rdg[wit={L,B}]{brahmaraṃdhrasthāne aṣṭamaṃ śatadalaṃ cakraṃ}
    \rdg[wit={U1}]{cakraṃ brahmaraṃdhrasthāne śatadalaṃ}}
  vartate/
  %%%%%%%%%%
  %%%%%%%%%%%
  %%%%%%%%%%
\extra{\app{\lem[type=emendation, resp=egoscr, alt={gurur}]{guru\skp{r-de}}
          \rdg[wit={U2}]{\korr guru°}}\skm{r-de}vatā\dd{}
        \app{\lem[type=emendation, resp=egoscr]{caitanyaḥ}
          \rdg[wit={U2}]{\korr caitanya°}} śaktiḥ\dd{}
        virāṭ ṛṣiḥ sarvotkṛṣṭasākṣiḥ\dd{}
        \app{\lem[type=emendation, resp=egoscr]{bhūtaturyātītaṃ}
          \rdg[wit={U2}]{\korr bhūtaturyātīta°}} caitanyātmakaṃ\dd{}
        sarvavarṇāḥ\dd{}
        sarvamātrāḥ\dd{}
        sarvadalāni\dd{}
        virāṭ \app{\lem[type=emendation, resp=egoscr]{dehaḥ}
          \rdg[wit={U2}]{\korr deha°}}
        sthitāvasthā\dd{} 
        prajñā vācā\dd{}
        sohaṃ \app{\lem[type=emendation, resp=egoscr]{vedaḥ}
          \rdg[wit={U2}]{\korr veda}}\dd{}
        \app{\lem[type=emendation, resp=egoscr]{anupamaṃ}
          \rdg[wit={U2}]{\korr anupama°}} sthānaṃ\dd{}
         \app{\lem[type=emendation, resp=egoscr]{ajapājapaḥ sahasraḥ}
          \rdg[wit={U2}]{\korr ajapājapasahasra}}\dd{} 1000 ghaṭi 2 palā 46 akṣara 40\dd{}
        \app{\lem[type=emendation, resp=egoscr]{sarvajapaḥ}
          \rdg[wit={U2}]{\korr sarvajapa°}} saṃkhyā\dd{}
        21600\dd{}
        ekaviṃśatisahasrāṇiṣaṭśatāni\dd{}
        tathaiva ca niśāhe vahate\dd{}
        prāṇaḥ yo jānāti sa paṃḍitaḥ\dd{} %%prāṇaḥ = m nom pl
        sakāreṇa bahir-yāti hakāreṇa viśet punaḥ\dd{}  
        haṃsaḥ sohaṃ\dd{}
        tato mantraṃ jīvo japati sarvadā\dd{}}
%The teacher is the deity. Consciousness is the power. Virāṭ is the Ṛṣi, the witness above everything. Made of consciousness is that which is associated with (\textit{bhūta°) the state beyond the fourth state. It has all colours. It has all matrices. It has all petals. The body is Virāṭ. The state is the standing still. The speech is wisdom.  The "I am that"-[expression] (\textit{sohaṃ}) is the Veda. The place is unsurpassed. A thousandfold recitation of the non-recited; 1000 [repetitions for]; 2 \textit{ghaṭi}s, 46 \textit{palā}s. and 40 \textit{akṣara}s.\footnote{It's not entirely clear what kind of measure is an \textit{akṣara}.} The count is all silent mutterings, [being] 21600. Day and night in this way it carries on. He who knows the breath is a learned person. With the sound of "sa" he exhales, with the sound of "ha" he inhales again: "I'm he, he's I". Because of that the embodied soul constantly utters the Mantra.\footnote{Add intertextual evidence.}
  %%%%%%%%%%%%%%%
  %%%%%%%%%%%%%%%
  %%%%%%%%%%%%%%%
  %%%%%%%%%%%%%%
  %%%%%%%%%%%%%%%
%----------------------
%tasya kamala----jātyadharaṇīpīṭha iti saṃjñā / \E
%tasya kamalasya jālaṃdharapīṭha iti saṃjñā / \P
%tasya kamalasya jālaṃdharapīṭha iti saṃjñā ...  \L
%tasya kamalasya jālaṃdharapīṭhasaṃjñā ...  \B
%tasya kamalasya jālaṃdharapīṭha iti saṃjñā ...  \N1
%tasya kamalasya jālaṃdharapīṭha iti saṃjñā ...  \N2
%tasya kamalasya jālaṃdharapīṭha iti saṃjñā ...  \D
%tasya kamalasya jālaṃdharapīṭha iti saṃjñā ...  \U1      
%tasya kamalasya jālaṃdharapīṭha iti saṃjñā //   \U2
%----------------------
%``The (divine) seat of  Jālaṃdhara'' is the designation of the lotus of it. 
%----------------------      
\note[type=testium, labelb=95, lem={jālaṃdhara°}]{Ysg: jālaṃdharapīṭhasaṃjñakaṃ}
tasya
\app{\lem[wit={ceteri}]{kamalasya}
  \rdg[wit={E}]{kamala°}}
      \app{\lem[wit={ceteri}]{jālandharapīṭha}
        \rdg[wit={B}]{jālandharapīṭha°}
        \rdg[wit={E}]{jātyadharaṇīpīṭha}}
      \app{\lem[wit={ceteri}]{iti}
        \rdg[wit={B}]{\om}}
      \app{\lem[wit={ceteri}]{saṃjñā}
        \rdg[wit={B}]{°saṃjñā}}/
%---------------------- 
%siddhapuruṣasya sthānam / \E
%siddhapuruṣasya sthānam / \P
%siddhapuruṣasya sthānam mūrti vartate // \L                         %%% schwerer Satz -> wie soll ich hier entscheiden?! 
%siddhapuruṣasya sthānam mūrti vartate // \B %Zeilensprung
%siddhapuruṣasya sthānam // \N1
%siddhapuruṣasya sthānam // \N2
%siddhapuruṣasya sthānam // \D  
%siddhapuruṣasya sthānam    \U1
%siddhapuruṣasya sthānaṃ    \U2
%----------------------      
%[It is] the place of the accomplished person.
%----------------------
\note[type=testium, labelb=96, lem={siddha°}]{Ysg: siddhapuruṣasyānacakraṃ}
      siddha\app{\lem[wit={ceteri},alt={°puruṣasya sthānam}]{puruṣasya\skp{-}sthānaṃ}
        \rdg[wit={L,B}]{sthānam mūrti vartate}}/
     \end{prose}
  \end{ekdosis}
\ekdpb*{}
%%%%%%%%%%%%%%%%%%%%%%%%%%%%%%%%%%%%%%%%%%
%%%%%%%%%%%%%%%%%%%%%%%%%%%%%%%%%%%%%%%%%%
%%%%%%%%PAGEBREAK%%%%%%%PAGEBREAK%%%%%%%%%
%%%%%%%%%%%%%%%%%%%%%%%%%%%%%%%%%%%%%%%%%%
%%%%%%%%%%%%%%%%PAGEBREAK%%%%%%%%%%%%%%%%%
%%%%%%%%%%%%%%%%%%%%%%%%%%%%%%%%%%%%%%%%%%
%%%%%%%%PAGEBREAK%%%%%%%PAGEBREAK%%%%%%%%%
%%%%%%%%%%%%%%%%%%%%%%%%%%%%%%%%%%%%%%%%%%
%%%%%%%%%%%%%%%%%%%%%%%%%%%%%%%%%%%%%%%%%%
%%%%%%%%%%%%%%%%%%%%%%%%%%%%%%%%%%%%%%%%%%
%%%%%%%%%%%%%%%%%%%%%%%%%%%%%%%%%%%%%%%%%%
%%%%%%%%PAGEBREAK%%%%%%%PAGEBREAK%%%%%%%%%
%%%%%%%%%%%%%%%%%%%%%%%%%%%%%%%%%%%%%%%%%%
%%%%%%%%%%%%%%%%PAGEBREAK%%%%%%%%%%%%%%%%%
%%%%%%%%%%%%%%%%%%%%%%%%%%%%%%%%%%%%%%%%%%
%%%%%%%%PAGEBREAK%%%%%%%PAGEBREAK%%%%%%%%%
%%%%%%%%%%%%%%%%%%%%%%%%%%%%%%%%%%%%%%%%%%
%%%%%%%%%%%%%%%%%%%%%%%%%%%%%%%%%%%%%%%%%%
%%%%%%%%%%%%%%%%%%%%%%%%%%%%%%%%%%%%%%%%%%
%%%%%%%%%%%%%%%%%%%%%%%%%%%%%%%%%%%%%%%%%%
%%%%%%%%PAGEBREAK%%%%%%%PAGEBREAK%%%%%%%%%
%%%%%%%%%%%%%%%%%%%%%%%%%%%%%%%%%%%%%%%%%%
%%%%%%%%%%%%%%%%PAGEBREAK%%%%%%%%%%%%%%%%%
%%%%%%%%%%%%%%%%%%%%%%%%%%%%%%%%%%%%%%%%%%
%%%%%%%%PAGEBREAK%%%%%%%PAGEBREAK%%%%%%%%%
%%%%%%%%%%%%%%%%%%%%%%%%%%%%%%%%%%%%%%%%%%
%%%%%%%%%%%%%%%%%%%%%%%%%%%%%%%%%%%%%%%%%%
\begin{ekdosis}
  \begin{prose}
%%%%%%%%%%%%%%% 
%%%%%%%%%%%%%%%
%%%%%%%%%%%%%%%
%%%%%%%%%%%%%%
%%%%%%%%%%%%%%%
%----------------------
%tanmadhye    'gnidhūmākārarekhā     yādṛśy    ādṛśy ekā  puruṣasya mūrttir varttate /  \E
%tanmadhye    'gnidhūmākārarekhā     yādṛśī   tādṛśy ekā  puruṣasya mūrttir varttate /  \P
%tanmadhye    'gnidhūmākārārekhā     yādṛśī   tādṛśy ekā  puruṣasya mūrttir varttate /  \L               
%tanmadhye    'gnidhūmākārārekhā     yādṛśī   tādṛśy ekā  puruṣasya mūrttir varttate /  \B     
%tanmadhye    'gnidhūmākārāreṣā      yādṛśī   tādṛśī ekā  puruṣasya mūrttir varttate /  \N1
%tanmadhye    agnidhūmrākārarekhā    yādṛśī / tādṛśī ekā  puruṣasya mūrttir varttate /  \N2
%tanmadhye    agnidhūmākārāreṣā      yādṛśī   tādṛśī ekā  puruṣasya mūrttir varttate /  \D
%tanmadhye    agnidhūmrākārārekhā    yādṛśī   tādṛśī ekā  puruṣasya mūrtir  vartate     \U1
%tanmadhye    'gnidhūmrākārārekhāyāḥ  etādṛśī         ekā  puruṣasya mūrtir  vartate // \U2
%----------------------      
%In its middle [is] something like a streak having the form of smoke and fire. Such a single [divine] form of the person (\textit{puruṣa}) exists [there].        
%---------------------      
\noindent
    \note[type=testium, labelb=97, lem={Ysg: 'gnidhūmrā°}]{tanmadhye gnidhūmrāreṣākārā ādimadhyaṃtarahitā puruṣasya mūrttir asti |}
tanmadhye \app{\lem[wit={E,P,L,B}]{'gnidhūmākārarekhā}
        \rdg[wit={N1,D}]{'gnidhūmākārāreṣā}
        \rdg[wit={N2,U1}]{agnidhūmrākārarekhā}
        \rdg[wit={U2}]{'gnidhūmrākārārekhāyāḥ}}
      \app{\lem[wit={ceteri}]{yādṛśī}
        \rdg[wit={E}]{yādṛśy°}
        \rdg[wit={U2}]{etādṛśī}}/
      \app{\lem[wit={P,L,B}]{yādṛśy}
        \rdg[wit={E}]{ādṛsy}
        \rdg[wit={N1,N2,D,U1}]{yādṛśī}
        \rdg[wit={U2}]{\om}}
      ekā puruṣasya mūrtir-vartate/
%---------------------
%tasyā  nādir nāṃto 'sti / \E
%tasyā  nādināṃ 'to sti / \P
%tasyā  nādir nāṃto sti / \L -> vor dem bei allen anderen vorigen Satz!?!?!?! 
%tasyā  nādir nāṃto sti / \B -> vor dem bei allen anderen vorigen Satz!?!?!?! 
%tasyāḥ nāsty aṃtaḥ ādir-api nāsti / \N1????
%tasyāḥ nāsty aṃtaḥ ādir-api nāsti / \N2
%tasyāḥ nāsty aṃtaḥ ādir api nāsti / \D
%tasyāḥ nāsty aṃtaḥ ādir-api nāsti    \U1
%tasyā  nādir naṃto sti              \U2
%---------------------
% Of her exists no end, nor a beginning.
%---------------------      
\app{\lem[wit={E,P,L,B,U2}]{tasyā}
  \rdg[wit={D,N1,N2,U1}]{tasyāḥ}}
\app{\lem[alt={nādir nānto 'sti}, wit={ceteri}]{nādir-nānto 'sti}
        \rdg[wit={N1,N2,D,U1}]{nāsty aṃtaḥ ādir api nāsti}
        \rdg[wit={P}]{nādināṃ 'to sti}}/
%---------------------    
%tasyā  mūrtter dhyānakāraṇāt pratyakṣaṃ niraṃtaraṃ  puruṣasyākāśe   gamāgamau   bhavataḥ / \E
%tasyā  mūrtter dhyānakaraṇāt pratyakṣaniraṃtaraṃ    puruṣasyākāśe   gamāgamau   bhavataḥ / \P
%tasyā  mūrtir  dhyānakaraṇāt pratyakṣaniraṃtaraṃ    puruṣasyākāśe   gamāgamau   bhavataḥ / \L         
%tasyā  mūrtir  dhyānakaraṇāt pratyakṣaṃ niraṃtaraṃ  puruṣasyākāśe   gamāgamau   bhavataḥ / \B
%tasyāḥ mūrttir dhyānakaraṇāt pratyakṣaniraṃtaraṃ    puruṣasya ākāśe gamāgamau   bhavataḥ / \N1
%tasyāḥ mūrttir dhyānakaraṇāt pratyakṣaniraṃtaraṃ    puruṣa ākāśe    gamāgame    bhavataḥ / \N2
%tasyāḥ mūrtir  dhyānakaraṇāt pratyakṣaniraṃtaraṃ    puruṣasya ākāśe gamāgamau   bhavataḥ / \D
%tasyāḥ mūrter  dhyānakaraṇāt/ pratyakṣaniraṃtaraṃ   puruṣasya ākāśi gamāmamo   bhavataḥ   \U1
%tasyāḥ mūrter  dhyānakaraṇāt pratyakṣaniraṃtaraṃ    puruṣasyākāśa---gamāgamau bhavata //      \U2
%---------------------    
%BEDEUTUNG DES SATZES BIS JETZT UNKLAR! Idee: Zeilensprung aus übernächstem Satz! Streiche pratyakṣaṃ niraṃtaraṃ und der Satz ergibt Sinn!  
%gamāgamau nom.  dual = coming and going ; bhavataḥ = 3p du ind pres von bhū
%Due to the exercise of meditation on this (divine) form both coming and going of the person in space occurs. 
%Kolloquium: Meinung zu Kompositum pratyakṣaniraṃtaraṃ = macht wenig Sinn oder?
%{\englishnote{\small Even though every single witness at hand transmits the latter reading right after \textit{°karaṇāt}, several considerations make it reasonable to conject that the original sentence is corrupted and was written without it. The main consideration to assume the corruption is that \textit{pratyakṣaṃ nirantaraṃ} is ungrammatical. The second is that the sentence is way more meaningful without it. The third that two sentences later we get the phrase in a meaningful context. Due to the last consideration my best guess is an interlace at an early stage of transmission.}}
%---------------------
\note[type=testium, labelb=98, lem={dhyānakaraṇāt}]{Ysg: tasyāḥ dhyānakartuḥ}
      \app{\lem[wit={B,E,L,P}]{tasyā}
  \rdg[wit={ceteri}]{tasyāḥ}}
\app{\lem[alt={mūrter},wit={E,P,U1,U2}]{mūrte\skp{r-dhyā}}
  \rdg[wit={B,D,L,N1,N2}]{mūrtir}}
\app{\lem[alt={dhyānakaraṇāt},type=conjecture, resp=egoscr]{\skm{r-dhyā}nakaraṇāt}
        \rdg[wit={E,B}]{\conj dhyānakāraṇāt pratyakṣaṃ niraṃtaraṃ}
        \rdg[wit={ceteri}]{dhyānakaraṇāt pratyakṣaniraṃtaraṃ}}
         \note[type=philcomm, labelb=99, lem={°kāraṇāt pratyakṣaṃ niraṃtaraṃ}]{Even though every single witness at hand transmits the latter reading right after °\textit{karaṇāt}, several considerations make it reasonable to conject that the original sentence is corrupted and was written without it. The main consideration to assume the corruption is that the syntactical unit \textit{pratyakṣaṃ nirantaraṃ} is ungrammatical in this construction. The second is that the sentence is way more meaningful without it. The third that two sentences later we get the phrase in a meaningful context. Due to the last consideration my best guess is an interlace at an early stage of transmission.}
      \app{\lem[wit={ceteri}]{puruṣasyākāśe}
        \rdg[wit={N2}]{puruṣa ākāśe}
        \rdg[wit={U2}]{puruṣasyākāśa°}
        \rdg[wit={U1}]{puruṣasya ākāśi}}
      gamā\app{\lem[wit={ceteri},alt={°gamau}]{gamau}
        \rdg[wit={U1}]{°gamo}
        \rdg[wit={N2}]{°game}}
        \app{\lem[wit={ceteri}]{bhavataḥ}
          \rdg[wit={U2}]{bhavata}}/
%---------------------     
%pṛthvīmadhye  sthitasyāpi    pṛthvī-------bādho   na bhavati / \E
%pṛthvīmadhye  sthitasyāpi    pṛthaka                 bhavati   \P %Zeilenspringer führt zu Verlust von Zeile in Pune
%pṛthvīmadhye  sthitasyāpi    pṛthvī-------bādho   na bhavati / \L
%pṛthivīmadhye sthitasyāpi // pṛtvī--------bādho   na bhavati // \B
%pṛthvīmadhye  sthitāv-api    pṛthvī kṣato bādho   na bhavati // \N1
%pṛthvīmadhye  sthitāv-api    pṛthvī kṣato bādho   na bhavati // \N2      
%pṛthvīmadhye  sthitāv-api    pṛthvī kṣato bādho   na bhavati // \D
%pṛthvīmadhye  sthitāv-api    pṛthvī kṣato bādho   na bhavati     \U1
%pṛthīvīmadhye sthitasyāpi    pṛthvī       bādhoko na bhati     \U2
%---------------------
%Affliction from the earth-element does not arise [anymore] even if one is situated in the middle of the earth.        
%---------------------
\note[type=testium, labelb=100 lem={pṛthvīmadhye}]{Ysg: pṛthivyāṃ sthitāv api pṛthvī kṛtabādho na bhavati ||}
        \app{\lem[wit={ceteri}]{pṛthvīmadhye}
          \rdg[wit={B,U2}]{pṛtivīmadhye}}
        \app{\lem[wit={ceteri}]{sthitasyāpi}     
          \rdg[wit={D,N1,N2,U1},alt={sthitāv api}]{sthitāv\skp{-}api}}
        \app{\lem[wit={E,L}]{pṛthvībādho}
          \rdg[wit={B}]{pṛtvībādho}
          \rdg[wit={N1,N2,D,U1}]{pṛthvī kṣato bādho}
          \rdg[wit={P}]{pṛthaka}
          \rdg[wit={U2}]{pṛthvī bādhoko}}
        \app{\lem[wit={ceteri}]{na bhavati}
          \rdg[wit={P}]{bhavati}}/
%---------------------
%sakalān pratyakṣaṃ niraṃtaraṃ paśyati ca pṛthagbhavati / \E
% \om                                                       \P      
%sakalāḥ pratyakṣaṃ niraṃtara paśyatī  ca pṛthak bhavati // \B
%sakalāḥ pratyakṣaṃ niraṃtara paśyatī  ca pṛthak bhavati / \L
%sakalāpratyakṣaniraṃtaraṃ    paśyati  ca pṛthak ca bhavati // \N1
%sakalapratyakṣaniraṃtaraṃ    paśyati  ca pṛthak ca bhavati    \N2      
%sakalāpratyakṣaniraṃtaraṃ    paśyati  ca pṛthak pṛthak bhavati \D      
%sakalāpratyakṣaniraṃtaraṃ    paśyati  ca/ pṛthak ca bhavati // \U1
%\om                                                     \U2
%---------------------
%He constantly sees everything in front of his eyes and he becomes separated (from the material world).
%---------------------
        \app{\lem[type=emendation, resp=egoscr]{sakalaṃ pratyakṣaṃ nirantaraṃ}
          \rdg[wit={N1,N2,D,U1}]{\korr sakalāpratyakṣaṃ nirantaraṃ}
          \rdg[wit={B,L}]{sakalāḥ pratyakṣaṃ niraṃtara}
          \rdg[wit={E}]{sakalān pratyakṣaṃ niraṃtaraṃ}
          \rdg[wit={P,U2}]{\om}}
        \app{\lem[wit={ceteri}]{paśyati}
          \rdg[wit={L,B}]{paśyatī}
          \rdg[wit={P,U2}]{\om}}
        \app{\lem[wit={E}]{pṛthagbhavati}
          \rdg[wit={B,L}]{ca pṛthak bhavati}
          \rdg[wit={N1,N2,U1}]{ca pṛthak ca bhavati}
          \rdg[wit={P,U2}]{\om}}/  
%---------------------
%atiśayenāyur vardhate /   \E
%atiśayenāyur vardhate     \P      
%atīśayanāyur vardhayate / \B
%atīśayanāyur vardhayate // \L
%atiśayena āyur varddhate // \N1
%atiśayena āyur varddhate // \N2     
%atiśayena āyur varddhate // \D
%atiśayena āyur varddhate // \U1
%\om                         \U2
%---------------------
% The force of life increases eminently. 
%---------------------
        \app{\lem[alt={atiśayenāyur},wit={E,P}]{atiśayenāyu\skp{r-va}}
          \rdg[wit={B,L}]{atīśayanāyur}
          \rdg[wit={N1,N2,D,U1}]{atiśayena āyur}
          \rdg[wit={U2}]{\om}}\app{\lem[alt={vardhate},wit={ceteri}]{\skm{r-va}rdhate}
          \rdg[wit={B,L}]{vardhayate}}\dd{}        
    \end{prose}
  \end{ekdosis}
%%%%%%%%%%%%%%%
%%%%%%%%%%%%%%
%%%%%%%%%%%%%
%%%%%%%%%%%%%
%%%%%%%%%%%%%
\begin{ekdosis}
 \ekddiv{type=ed}
   \bigskip
    \centerline{\textrm{\small{[Ninth Cakra]}}}
    \bigskip
    \begin{prose}
%---------------------
%idānīṃ navamacakrasya   bhedāḥ kathyante /  \E
%idānīṃ navamacakrasya   bhedāḥ kathyante /  \P
%idānīṃ navamacakrasya   bhedāḥ kathyate     \L
%idānīṃ navamaṃ cakrasya bhedāḥ kathyate //  \B
%idānīṃ navamacakrasya   bhedāḥ kathyaṃte // \N1
%idānīṃ navamacakrasya   bheda  kathyate  // \N2
%idānīṃ navamacakrasya   bhedāḥ kathyaṃte // \D
%idānīṃ navamaś cakrasya bhedāḥ kathyaṃte    \U1   
%idānīṃ navamacakrasya   bhedaḥ kathyate /   \U2
%---------------------
%Now the divisions/differentiations of the ninth cakra are explained.
%---------------------
\note[type=testium, labelb=101, lem={mahāśūnyacakram}]{Ysg: brahmaraṃdhre eva śatadalacakropari mahāśūnyacakraṃ mahāsiddhacakraṃ pūrṇagiricakraṃ iti saṃjñakaṃ sahasradalaṃ cakraṃ asti | tad upari kiṃcin nāsti | tac cakraṃ atiraktaṃ ūrdhvamukhaṃ sakalaśobhāspadaṃ anekakalyāṇapūrṇaṃ mano vācā ma gocara parimalo petaṃ | tat kamalamadhye trikoṇākarṇikā | tasyāṃ karṇikāyāṃ saptadaśī niraṃjanarūpā koṭisūryaprabhā satī uṣṇabhava hīnā koṭicandrasama sītalaikākal nāsti | tasyāṃ anaṃta paramānaṃta paramānaṃdānāṃ sthānaṃ tasyāḥ kalāyā dhyānakaraṇāt sādako yadyādi śati tatra bhavati |}
\note[type=source, labelb=102, lem={mahāśūnyacakram}]{Ysv (PT): navaman tu mahāśūnyaṃ cakran tu tatparātparam | tad upari paraṃ kiñcin nāsti kiñcin mahāparam | mahācakraṃ siddhacakraṃ pūrṇagauryādisaṃjñakam | tanmadhye varttate padmaṃ sahasradalamadbhutam | ūrddhvavaktraṃ mahāvaktre [mahāvaktraṃ (YK)] varṇaśobhāpadaṃ mahat | sarvakalyāṇasampūrṇamasya tulyaṃ na vidyate | parimāṇaṃ vaktam asya [vaktum (YK)] manasā vacasā na hi | trikoṇakarṇikā tatra [°tantraṃ (YK)] varttate jagad īśvari |}
idānīṃ
\app{\lem[wit={ceteri},alt={°navama}]{navama}
  \rdg[wit={B}]{navamaṃ}
  \rdg[wit={U1}]{navamaś°}}cakrasya
\app{\lem[wit={ceteri}]{bhedāḥ}
  \rdg[wit={N2}]{bheda}}
\app{\lem[wit={ceteri}]{kathyante}
  \rdg[wit={L,B,N2,U2}]{kathyate}}/
%------------------------------
%tasya mahāśūnyacakram    iti  saṃjñā /  \E
%tasya mahāśūnyacakram    iti  saṃjñā /  \P
%tasya mahāśūnye cakram   iti  saṃjñā    \L
%tasye mahāśūnye cakram   iti  saṃjñā    \B
%tasya mahāśūnye cakreti       saṃjñā // \N1
%tasya mahāśūnyacakreti        saṃjñā // \N2
%tasya mahāśūnyacakreti        saṃjñā // \D
%tasya mahāśūnyacakreti        saṃjñā /  \U1
%\om /                                   \U2
%---------------------
%The designation of it is ``the \textit{cakra} of the great void (\textit{mahāśūnyacakra})''.
%------------------------------
tasya \app{\lem[wit={ceteri}, alt={mahāśūnya°}]{mahāśūnya}
  \rdg[wit={L,B,N1}]{mahāśūnye}
  \rdg[wit={U2}]{\om}
}\app{\lem[wit={ceteri},alt={°cakreti}]{cakreti}
  \rdg[wit={E,P}]{°cakram iti}
  \rdg[wit={L,B}]{cakram iti}
  \rdg[wit={U2}]{\om}}
\app{\lem[wit={ceteri}]{saṃjñā}
  \rdg[wit={U2}]{\om}}/
%------------------------------
%tadupary aparaṃ kimapi nāsti / \E
%tadupary aparaṃ kimapi nāsti \P
%tadupary        kimapi nāsti \B ??-> auch mögliche Lesart
%tadupari        kimapi nāsti \L
%tadupari aparaṃ kiṃapi nāsti / \N1
%tadupari aparaṃ kiṃapi nāsti / \N2
%tadupari aparaṃ kiṃapi nāsti / \D
%tadupari aparaṃ kiṃapi nāsti   \U1
% \om                           \U2
%---------------------
%kim api: somewhat, to a considerable extent, rather, much more, still, further. Śa
%---------------------
%Above that there is no other. 
%---------------------
\app{\lem[wit={E,P,B},alt={tad upary}]{tad\skp{-}upar\skm{y-a}}
  \rdg[wit={ceteri}]{tad upari}
  \rdg[wit={U2}]{\om}}\app{\lem[wit={ceteri}, alt={aparaṃ}]{\skp{y-a}paraṃ}
  \rdg[wit={B,L,U2}]{\om}}
\app{\lem[wit={ceteri}]{kimapi}
  \rdg[wit={N1,N2,D,U1}]{kiṃ api}
  \rdg[wit={U2}]{\om}} nāsti/
     \end{prose}
  \end{ekdosis}
\ekdpb*{}
%%%%%%%%%%%%%%%%%%%%%%%%%%%%%%%%%%%%%%%%%%
%%%%%%%%%%%%%%%%%%%%%%%%%%%%%%%%%%%%%%%%%%
%%%%%%%%PAGEBREAK%%%%%%%PAGEBREAK%%%%%%%%%
%%%%%%%%%%%%%%%%%%%%%%%%%%%%%%%%%%%%%%%%%%
%%%%%%%%%%%%%%%%PAGEBREAK%%%%%%%%%%%%%%%%%
%%%%%%%%%%%%%%%%%%%%%%%%%%%%%%%%%%%%%%%%%%
%%%%%%%%PAGEBREAK%%%%%%%PAGEBREAK%%%%%%%%%
%%%%%%%%%%%%%%%%%%%%%%%%%%%%%%%%%%%%%%%%%%
%%%%%%%%%%%%%%%%%%%%%%%%%%%%%%%%%%%%%%%%%%
%%%%%%%%%%%%%%%%%%%%%%%%%%%%%%%%%%%%%%%%%%
%%%%%%%%%%%%%%%%%%%%%%%%%%%%%%%%%%%%%%%%%%
%%%%%%%%PAGEBREAK%%%%%%%PAGEBREAK%%%%%%%%%
%%%%%%%%%%%%%%%%%%%%%%%%%%%%%%%%%%%%%%%%%%
%%%%%%%%%%%%%%%%PAGEBREAK%%%%%%%%%%%%%%%%%
%%%%%%%%%%%%%%%%%%%%%%%%%%%%%%%%%%%%%%%%%%
%%%%%%%%PAGEBREAK%%%%%%%PAGEBREAK%%%%%%%%%
%%%%%%%%%%%%%%%%%%%%%%%%%%%%%%%%%%%%%%%%%%
%%%%%%%%%%%%%%%%%%%%%%%%%%%%%%%%%%%%%%%%%%
%%%%%%%%%%%%%%%%%%%%%%%%%%%%%%%%%%%%%%%%%%
%%%%%%%%%%%%%%%%%%%%%%%%%%%%%%%%%%%%%%%%%%
%%%%%%%%PAGEBREAK%%%%%%%PAGEBREAK%%%%%%%%%
%%%%%%%%%%%%%%%%%%%%%%%%%%%%%%%%%%%%%%%%%%
%%%%%%%%%%%%%%%%PAGEBREAK%%%%%%%%%%%%%%%%%
%%%%%%%%%%%%%%%%%%%%%%%%%%%%%%%%%%%%%%%%%%
%%%%%%%%PAGEBREAK%%%%%%%PAGEBREAK%%%%%%%%%
%%%%%%%%%%%%%%%%%%%%%%%%%%%%%%%%%%%%%%%%%%
%%%%%%%%%%%%%%%%%%%%%%%%%%%%%%%%%%%%%%%%%%
\begin{ekdosis}
  \begin{prose}
\noindent
%------------------------------
%tadeva-mahāsiddhacakraṃ kathyate // \E
%tadeva-mahāsiddhacakraṃ kathyate    \P 
%tadeva-mahāsiddhacakraṃ kathyate // \B
%tadeva-mahāsiddhacakraṃ kathyate // \L
%tadeva-mahāsiddhacakraṃ kathyate // \N1
%tadeva-mahāsiddhacakraṃ kathyate // \N2
%tadeva-mahāsiddhacakraṃ kathyate // \D
%tadeva-mahāsiddhacakraṃ kathyate /  \U1
% \om                                \U2
%---------------------
%Therefore it is declared to be the \textit{cakra} of the great perfection (\textit{mahāsiddhacakra}).
%---------------------
tad-eva mahāsiddhacakraṃ kathyate/
%------------------------------
%       tasya           pūrṇagiripīṭha               etadṛśaṃ nāma /  \E 
%       tasya           pūrṇagiripīṭham-iti          etādṛśaṃ nāma    \P
%       tasya           pūrṇagiripīṭham-iti saṃjñā   etādṛsaṃ nāma    \B ->!!! 
%       tasya           pūrṇagiripīṭham-iti saṃjñā   etādṛsaṃ nāma    \L
%       tasya cakrasya  pūrṇagiri                    etādṛśaṃ nāma /  \N1
%       tasya cakrasya  pūrṇagiri                    etādṛśaṃ nāma /  \N2
%       tasya cakrasya  pūrṇagiri                    etādṛśaṃ nāma /  \D
%       tasya cakrasya  pūrṇagire                    etādṛśaṃ nāmaḥ   \U1
%madhye tasya           pūrṇagiripīṭham-iti          ekādaśaṃ nāma // \U2   
%-----------------------------
%Such a name of it is ``(divine) seat of Pūrṇagiri''.   
%------------------------------
\app{\lem[wit={ceteri}]{tasya}
  \rdg[wit={N1,N2,D,U1}]{tasya cakrasya}
  \rdg[wit={U2}]{madhye tasya}}
pūrṇagiri\app{\lem[wit={P,B,L,U2}, alt={°pīṭham}]{pīṭha\skm{m-i}}
  \rdg[wit={E}]{pīṭha}
  \rdg[wit={ceteri}]{\om}
}\app{\lem[wit={P,U2},alt={iti}]{\skp{m-i}ti}
  \rdg[wit={B,L}]{iti saṃjñā}
  \rdg[wit={ceteri}]{\om}}
\app{\lem[wit={ceteri}]{etādṛśaṃ}
  \rdg[wit={E}]{etadṛśaṃ}
  \rdg[wit={U2}]{ekādaśaṃ}}
\app{\lem[wit={ceteri}]{nāma}
  \rdg[wit={U1}]{nāmaḥ}}/
%------------------------------
%tasya mahāśūnyacakrasya madhye ūrdhvamukham iti raktavarṇaṃ sakalaśobhāspadam    \E
%tasya mahāśūnyacakrasya madhye ūrdhvamukham iti raktavarṇa--sakalaśobhāspadaṃ     \P
%tasya mahāśūnyacakrasya madhye ūrdhvamukhem iti raktavarṇaṃ sakalaśobhāspadaṃ // \B    
%tasya mahāśūnyacakrasya madhye ūrdhvamukham iti raktavarṇaṃ sakalaśobhāspadaṃ // \L
%tasya mahāśūnyacakramadhye     ūrdhvamukhaṃ atiraktavarṇaṃ  sakalaśobhāspadaṃ /   \N1 ->!!!
%tasya mahāśūnyacakramadhye     ūrdhvamukhaṃ atiraktavarṇaṃ  sakalaśobhāspadaṃ     \N2
%tasya mahāśūnyacakramadhye     ūrdhvamukhaṃ atiraktavarṇaṃ  sakalaśobhāspadaṃ /   \D
%tasya mahāśūnyacakramadhye     ūrdhvamukhaṃ atiraktavarṇaṃ  sakalaśobhāspadaṃ     \U1
%tasya mahāśūnyacakrasya        urdhvamukham-ativarṇaṃ       sakalaśobhanāsyadaṃ / \U2                                             
%------------------------------
%anekakalyāṇapūrṇaṃ sahasradalan      ekaṃ kamalaṃ  varttate / \E
%anekakalyāṇapūrṇaṃ sahasradalaṃ      ekaṃ kamalaṃ  vartate    \P
%anekakalyāṇapūrṇa--sahasradalaṃ      ekaṃ kamalaṃ  vartato    \B
%anekakalyāṇapūrṇaṃ sahasradalaṃ      ekaṃ kamalaṃ  vartate    \L
%anekakalyāṇapūrṇaṃ sahasradalaṃ      eka--kamalaṃ  varttate   \D
%anekakalyāṇapūrṇaṃ sahasradalaṃ      ekaṃ kamalaṃ  vartate    \N1
%anekakalyāṇapūrṇa--sahasradalaṃ      ekaṃ kamalaṃ  varttate    \N2
%anekakalyāṇapūrṇaṃ sahasradalaṃ           kamalaṃ  vartate /   \U1
%anekakalyāṇapūrṇaṃ // sahasradalaṃ   ekaṃ kamalaṃ  vartate / \U2
%Fragezeichen in |nepal ... schreiber Einfügung? 
%------------------------------
%In the middle of the \textit{mahāśūnyacakra} exists one lotus facing upward, very red in color with a thousand petals - an abode of brilliance and wholeness.
%------------------------------
tasya mahāśūnya\app{\lem[wit={ceteri},alt={°cakramadhye}]{cakramadhye}
  \rdg[wit={E,P,B,L}]{°cakrasya madhye}
  \rdg[wit={U2}]{°cakrasya}}
\app{\lem[wit={ceteri},alt={ūrdhvamukham}]{ūrdhvamukha\skp{m-a}}
  \rdg[wit={E,P,L}]{ūrdhmukham}
  \rdg[wit={U2}]{urdhvamukham}
  \rdg[wit={B}]{ūrdhvamukhem}}
\app{\lem[wit={ceteri}]{\skm{m-a}tiraktavarṇaṃ}
  \rdg[wit={E,L,B}]{iti raktavarṇaṃ}
  \rdg[wit={P}]{iti raktavarṇa°}
  \rdg[wit={U2}]{ativarṇaṃ}}
sakala\app{\lem[wit={ceteri},alt={°śobhāspadaṃ}]{śobhāspadaṃ}
  \rdg[wit={E}]{°śobhāspadam}
  \rdg[wit={U2}]{°śobhanāsyadaṃ}}
\app{\lem[wit={ceteri}]{anekakalyāṇapūrṇaṃ}
  \rdg[wit={B,N2}]{°pūrṇa°}}
sahasradalaṃ
\app{\lem[wit={ceteri}]{ekaṃ}
  \rdg[wit={D}]{eka°}
  \rdg[wit={U1}]{\om}}
kamalaṃ
\app{\lem[wit={ceteri}]{vartate}
  \rdg[wit={B}]{vartato}}/
%---------------------
%yasya           parimalo manaso vacaso na gocaraḥ // \E
%yasya           parimalo manasā vacasā na gocaraḥ /  \P
%yasya           parimalo manasā vacasā    gocaraḥ /  \L
%yasya           parimalo manasā vacasā na gocaraḥ /  \B
%yasya           parimalo manasā vacasā na gocaraḥ /  \N1
%yasya           parimalo manasā vacasā na gocara /   \N2
%yasya           parimalo manasā vacasā na gocaraḥ /  \D
%yasya           parimalo vacasā manasā na gocaraḥ    \U1
%yasya kamalasya parimalo manasā vācā   na gocara ..  \U2
%---------------------
%Whose fragrance is not in range by mind and speech. 
%Dessen Duft ist nicht in Reichweite von Geist und Sprache. 
%---------------------
\app{\lem[wit={ceteri}]{yasya}
  \rdg[wit={U2}]{yasya kamalasya}}
parimalo
\app{\lem[wit={E}]{manaso vacaso}
  \rdg[wit={P,L,B,N1,N2,D}]{manasā vacasā}
  \rdg[wit={U1}]{vacasā manasā}
  \rdg[wit={U2}]{manasā vācā}
}
\note[type=philcomm, labelb=103, lem={°manaso vacaso}]{All manuscripts at hand share this usage of the instrumentals. Only the printed edition conjectures the forms into the exspected genitiv. I adopted the variant of the printed edition to arrive at a grammatically correct text.}
\app{\lem[wit={ceteri}]{na}
  \rdg[wit={L}]{\om}
}
\app{\lem[wit={ceteri}]{gocaraḥ}
  \rdg[wit={N2,U2}]{gocara}}/
%---------------------
%tasya kamalasya madhye trikoṇarūpa-ikā karṇikā varttate /    \E
%tasya kamala----madhye trikoṇārūpā ekā karṇikā varttate/ \P
%tasya kamalasya madhye trikoṇarūpā ekā karṇikā varttate/     \L
%tasya kamalasya madhye trikoṇarūpā ekā karṇikā varttate/     \B
%tasya kamalasya madhye trikoṇarūpā eka karṇikā varttate/     \N1
%tasya kamalasya madhye trikoṇarūpā eka karṇikā varttate/     \N2
%tasya kamalasya madhye trikoṇarūpā ekā karṇikā varttate/     \D
%tasya kamalasya madhye trikoṇarūpā ekā karṇikā vartate       \U1
%tasya kamalasya madhye trikoṇarūpā ekā karṇikā vartate //    \U2
%---------------------
%In the middle of this lotus exists one pericarp having the shape of a triangle. 
%------------------------------
tasya
\app{\lem[wit={ceteri}]{kamalasya}
  \rdg[wit={P}]{kamala°}}
madhye
\app{\lem[wit={E}]{trikoṇarūpaikā}
  \rdg[wit={ceteri}]{trikoṇārūpā ekā}
  \rdg[wit={N1,N2}]{trikoṇārūpā eka}}
karṇikā vartate\dd{}
%------------------------------
%tatkarṇikāmadhye saptadaśī         niraṃjanarūpā kalā varttate/ \E
%tatkarṇikāmadhye saptadaśireṇa ekā niraṃjanarūpā kalā vartate// \L
%tatkarṇikāmadhye saptadaśireṇa ekā niraṃjanarūpā kalā vartate// \B
%tatkarṇikāmadhye saptadaśī     ekā niraṃjanarūpā kalā vartate// \P
%tatkarṇikāmadhye saptadaśī     ekā niraṃjanarūpā kalā vartate// \N1
%tatkarṇikāmadhye saptadaśī     ekā niraṃjanarūpā kalā vartate/  \N2
%tatkarṇikāmadhye saptadaśī     ekā niraṃjanarūpā kalā vartate// \D
%tatkarṇikāmadhye saptadaśī     ekā niraṃjanarūpā kalā vartate  \U1
%tatkarṇikāmadhye saptadaśī     eka niraṃjanarūpā kalā varttate/ \U2
%---------------------
%In the middle of the pericarp exists one seventeenth digit in the shape of a immaculé form.
%---------------------
\note[type=source, labelb=104, lem={saptadaśī}]{Ysv (PT): kalā saptadaśī tatra varttate parameśvari | nirañjanakalā sā tu koṭisūryasamaprabhā | koṭicandraprabhā caiva śītoṣṇādivivarjitā | asya dhyānāt sādhakasya manoduḥkhaṃ bhaven na hi | anantaparamānandasthānaṃ jñeyaṃ tadūrddhvataḥ [tadarddhataḥ (YK)] | ūrddhvagatakalā tatra tasya dhyānād bhaved iti | iti siddhirājayogaṃ strīṇāṃ bhogaṃ mahāsukham | gītavādyavinodādi saśivaṃ varddhate kṣitau | dhyānaṃ nirantarañ cāsya puṇyapāpe sthire [sthirau (YK)] na hi | nijarūpasya dṛṣṭiḥ syād dūrasyārthañ ca paśyati ||}
tatkarṇikāmadhye
\app{\lem[wit={ceteri}]{saptadaśī}
  \rdg[wit={L,B}]{saptadaśireṇa}}\note[type=philcomm, labelb=105, lem={saptadaśī}]{A \textit{saptadaśī kalā} appears frequently in Śaiva literature. References need to be added here.}
\app{\lem[wit={ceteri}]{ekā}
  \rdg[wit={E}]{\om}}
nirañjanarūpā kalā varttate/
%---------------------
%koṭisūryasamaprabhaṃ kalāyās tejo vartate /    \E
%koṭisūryasamaprabhā kalāyās tejo vartate /     \L
%koṭisūryasamaprabhā kalāyās tejo vartate /     \B
%koṭisūryasamaprabha kalāyās tejo vartate /     \P
%koṭisūryasamaprabhaṃ kalāyās tejo vartate /    \N1
%koṭisūryasamaprabhaṃ kalāyā  tejo varttate //  \N2
%koṭisūryasamaprabhaṃ kalāyās tejo vartate /    \D
%koṭisūryasadṛṣaprabhaṃ kalāyās tejo vartate /  \U1
%koṭisūryasamaprabhā // kalāyās tejo varttate / \U2
%---------------------
%A light of the part exists shining like a thousand suns. 
%------------------------------
koṭisūrya\app{\lem[alt={°samaprabhaṃ}, wit={ceteri}]{samaprabhaṃ}
  \rdg[wit={L,B,U2}]{samaprabhā}
  \rdg[wit={P}]{samaprabha}
  \rdg[wit={U1}]{sadṛṣaprabhaṃ}}
kalāyās-tejo vartate/
%------------------------------
%param udbhavo nāsti /     \E
%parim uṣṇabhavo nāsti /   \P
%parim uṣṇabhavo nāsti /   \L
%parim uṣṇabhavo nāsti /   \B
%parim uṣṇabhāvo nāsti /   \N1
%para  uṣṇabhāvo nāsti     \N2
%parim auṣṇabhāvo nāsti /  \D
%paraṃ uṣṇabhāvo nāsti     \U1
%param uṣṇabhāvo nāsti /   \U2
%---------------------
%[But] excessive heat is not arising. 
%------------------------------
\app{\lem[alt={param},wit={E,U1,U2}]{para\skp{m-u}}
  \rdg[wit={U1}]{paraṃ}
  \rdg[wit={N2}]{para}
  \rdg[wit={ceteri}]{parim}
}\app{\lem[wit={ceteri}, alt={uṣṇabhāvo}]{\skm{m-u}ṣṇabhāvo}
  \rdg[wit={P,L,B}]{uṣṇabhavo}
  \rdg[wit={D}]{auṣṇabhāvo}
  \rdg[wit={E}]{udbhavo}
}
nāsti/
%------------------------------
%koṭicandrasamaprabhā    śītalaṃ paraṃ   śītabhāvo   nāsti / \E
%koṭicandrasamaprabhā    śītalaṃ paraṃ   śītabhavo   nāsti / \P
%\om /                                                      \L
%koṭicandrasamaprabhā    śītalaṃ paraṃ   śītabhavo   nāsti / \B
%koṭicandrasamaprabhaṃ   śītalaparaṃ         bhavo   nāsti / \N1
%koṭicandrasamaprabhaṃ   śītalapara----------bhavo   nāsti // \N2
%koṭicaṃdrasamaprabhaṃ   śītalaparaṃ         bhavo   nāsti / \D
%koṭicaṃdrasamaṃ prabhaṃ śītalaṃ paraṃ       bhavo   nāsti / \U1
%koṭicaṃdrasamaprabhā    śītalaṃ paraṃ śītalabhāvo   nāsti / \U2
%---------------------
%Shining like a thousand moons, excess of cold is not arising.
%---------------------
koṭicandra\app{\lem[alt={°samaprabhaṃ},wit={N1,N2,D}]{samaprabhaṃ}
  \rdg[wit={E,P,B,U2}]{°samaprabhā}
  \rdg[wit={U1}]{°samaṃ prabhaṃ}
  \rdg[wit={L}]{\om}}
\app{\lem[wit={N1,D}]{śītalaparaṃ}
  \rdg[wit={ceteri}]{śītalaṃ paraṃ}
  \rdg[wit={N2}]{śītalapara}
  \rdg[wit={L}]{\om}}
\app{\lem[wit={ceteri}]{bhāvo} 
  \rdg[wit={E,P,B}]{śītabhāvo}
  \rdg[wit={U2}]{śītalabhāvo}
  \rdg[wit={L}]{\om}}
nāsti/
%------------------------------
%asyāḥ kalāyā   dhyānayogāt    sādhakasya manasi duḥkhaṃ na bhavati / \E
%asyāḥ kalādhyānayogāt         sādhakasya manasi duḥkhaṃ na bhavati / \P
%asyāḥ kalāyāḥ  dhyānakaraṇāt  sādhakasya manasi duḥkhaṃ na bhavati / N1
%asyā kalāyā    dhyānakaraṇāt  sādhaka----manasi duḥkhaṃ na bhavati / N2
%asyāḥ kalāyāḥ  dhyānakaraṇāt  sādhakasya manasi duḥkhaṃ na bhavati / D
%
%asyāḥ kalāyā   dhyānayogāt    sādhakasya manasi duḥkhaṃ bhavati /B
%asyāḥ kalāyā   dhyānayogāt    sādhakasya manasi duḥkhaṃ bhavati /L
%asyāḥ kalāyā   dhyānakaraṇāt/ sādhakasya manasi duḥkhaṃ na bhavati / U1
%asyā  kalāyāḥ  dhyānayogāt//  sādhakasya manasi duḥkhaṃ na bhavati // \U2
%atrastāne 'haṃ devatā// sohaṃ śaktiḥ// ātmāṛṣiḥ// mokṣamārhaḥ// haṃbhrahmordhaṃ// haṃcakra iti// agnicakre sakaro bhavatī// prāṇīrūḍho bhave jjīva ārohaty avarohati bhavaguhāsthānaṃ pitavarṇaṃ// koṭisūryapratikāśaṃ tejaḥ sadoditaprabhā śīvodevatā// mūlamāyā śaktiḥ// hara ātmālayāvsthā dhvanisthirānādātmako khaṃḍa 'dhvani// adhorāmudrā// mūlamāyā// prakṛtidehaḥ// vāṅmanogocaraḥ// niḥprapaṃcaḥ// niḥsaṃśayaḥ// nistaraṃganirlepalakṣaṃ laya// dhyānasamādhi 
%---------------------
%asyāḥ kalāyā dhyānakaraṇāt\varc{\emend kalāyāḥ dhyānakaraṇāt \nepal \dehlia}{kalāyā dhyānayogāt \nepal \dehlia kalādhyānayogāt \pune} sādhakasya manasi duḥkhaṃ na\varc{na \edprint \pune \nepal \dehlia}{\om \oxford \lalchand} bhavati /
%Due to the exercise of meditation upon the digit suffering does not arise in the mind of the practitioner (anymore). 
%------------------------------
\app{\lem[wit={ceteri}]{asyāḥ}
  \rdg[wit={N2,U2}]{asyā}}
kalā\app{\lem[wit={E,B,L,N2,U1}, alt={°yā}]{yā}
  \rdg[wit={N1,D}]{°yāḥ}
  \rdg[wit={E,B,L}]{°yā}
  \rdg[wit={U2}]{°yāḥ}
  \rdg[wit={P}]{\om}}
dhyāna\app{\lem[wit={N1,N2,D,U1}, alt={°karaṇāt}]{karaṇāt}
  \rdg[wit={ceteri}]{°yogāt}}
\app{\lem[wit={ceteri}]{sādhakasya}
  \rdg[wit={N2}]{sādhaka°}}
duḥkhaṃ
\app{\lem[wit={ceteri}]{na}
  \rdg[wit={B,L}]{\om}}
bhavati/
%%%%%%%%%%%%%
%%%%%%%%%%%%
%%%%%%%%%%%%
%%%%%%%%%%%%
%%%%%%%%%%%%
\extra{atra
   \app{\lem[type=emendation, resp=egoscr]{sthāne}
    \rdg[wit={U2}]{\korr stāne}} 'haṃ devatā\dd{}
  sohaṃ śaktiḥ\dd{}
  ātmāṛṣiḥ\dd{}
  \app{\lem[type=emendation, resp=egoscr]{mokṣo}
    \rdg[wit={U2}]{\korr mokṣa°}} mārgaḥ\dd{}
   \app{\lem[type=emendation, resp=egoscr]{ahaṃ brahmordhvaṃ}
    \rdg[wit={U2}]{\korr haṃ brahmordhaṃ}}\dd{}
   \app{\lem[type=emendation, resp=egoscr]{ahaṃ cakra iti}
     \rdg[wit={U2}]{\korr haṃcakra iti}}\dd{}
   agnicakre
   \app{\lem[type=emendation, resp=egoscr]{sakāro}
     \rdg[wit={U2}]{\korr sakaro}}
   \app{\lem[type=emendation, resp=egoscr]{bhavati}
     \rdg[wit={U2}]{\korr bhavatī}}\dd{}
   prāṇī rūḍho bhavej-jīva ārohaty-avarohati\note[type=philcomm, labelb=106, lem={prāṇī}]{Find parallels of hemistich.}\dd{}
bhavaguhā sthānaṃ\dd{}
   \app{\lem[type=emendation, resp=egoscr]{pitaṃ}
     \rdg[wit={U2}]{\korr pita°}} varṇaṃ\dd{}
   koṭisūryapratikāśaṃ tejaḥ\dd{}
   \app{\lem[type=emendation, resp=egoscr]{sadoditā}
     \rdg[wit={U2}]{\korr sadodita°}} prabhā\dd{}
   \app{\lem[type=emendation, resp=egoscr]{śivo}
     \rdg[wit={U2}]{\korr śīvo}} 
   devatā\dd{}
   mūlamāyā śaktiḥ\dd{}
   \app{\lem[type=emendation, resp=egoscr]{harātmālayāvasthā}
     \rdg[wit={U2}]{\korr hara ātmālayāvasthā}}\dd{}
   dhvanisthirānādātmako \app{\lem[type=emendation, resp=egoscr]{'khaṇḍadvaniḥ}
     \rdg[wit={U2}]{\korr khaṃḍadhvani}}\dd{} 
   aghorā mudrā\dd{}
   \app{\lem[type=emendation, resp=egoscr]{mūlā} %macht diese emdendation wirklich Sinn? 
     \rdg[wit={U2}]{\korr mūla°}} māyā\dd{}
   \app{\lem[type=emendation, resp=egoscr,alt={prakṛtir}]{prakṛti\skp{r-de}}
     \rdg[wit={U2}]{\korr prakṛti°}}\skm{r-de}haḥ\dd{}
   vāṅmano 'gocaraḥ\dd{} %%
   niḥprapañcaḥ\dd{}
   niḥsaṃśayaḥ\dd{}
   nistaraṃganirlepalakṣaṃ %%%see pw Vol. 3, S. 229 for nistaranga
  \app{\lem[type=emendation, resp=egoscr]{layo}
     \rdg[wit={U2}]{\korr laya}}
   \app{\lem[type=emendation, resp=egoscr]{dhyānaḥ samādhiḥ}
     \rdg[wit={U2}]{\korr dhyānasamādhi}}\dd{}}
%\extra{Here at this location the ``I''(\textit{aham}) is the deity. The ``he is I'' (\textit{so 'ham}) is the power. This self is the Ṛṣi. The path is liberation. Brahma is the I above. ``I'm a circle''. In the circle of fire is the letter "sa". [There?] life arises, the living soul ascends and decends. The place is the hidden place of being. The colour is yellow. The light is the shine of ten million suns. The shine is always and visible. Śiva is the deity. The power is primordial illusion. The state is the dissolution of the self into Hara\footnote{Epiphet of Śiva.}. The transcendental sound has the nature of a sound with stable resonance. The seal is the ``fearless''. The illusion is the root. The body is the original matter. It is not within reach of speech and mind. It is without delusion. It is without doubt. The unaffected and undefiled goal is dissolution, meditation [and] final absorption.}
%---------------------
%tadupari anaṃtaparamānandasya sthānam / \E
%tadupari anaṃtaparamānandasya sthānaṃ   \P
%tadupari anantaparamānaṃdasya sthānam / \N1
%tadupari anantaparamānaṃdasya sthānam / \N2
%tadupari anantaparamānaṃdasya sthānaṃ / \D
%tadupari anantaparamānaṃdasya sthānam vartate/ \B
%tadupari anaṃtaparamānaṃdasya sthānam vartate/ \L
%tadupari alakṣaparamānaṃdasya sthānam   \U1
%tadupari anaṃtaparamānaṃdasya sthānaṃ// U2
%---------------------
%Above that is the place of infinite supreme bliss.
%---------------------
tadupari
\app{\lem[wit={ceteri}, alt={ananta°}]{ananta}
  \rdg[wit={U1}]{alakṣa°}}paramānaṃdasya
\app{\lem[wit={ceteri}]{sthānam}
  \rdg[wit={D,U2}]{stānaṃ}
  \rdg[wit={B,L}]{sthānam vartate}}/
 \end{prose}
\end{ekdosis}
\ekdpb*{}
%%%%%%%%%%%%%%%%%%%%%%%%%%%%%%%%%%%%%%%%%%
%%%%%%%%%%%%%%%%%%%%%%%%%%%%%%%%%%%%%%%%%%
%%%%%%%%PAGEBREAK%%%%%%%PAGEBREAK%%%%%%%%%
%%%%%%%%%%%%%%%%%%%%%%%%%%%%%%%%%%%%%%%%%%
%%%%%%%%%%%%%%%%PAGEBREAK%%%%%%%%%%%%%%%%%
%%%%%%%%%%%%%%%%%%%%%%%%%%%%%%%%%%%%%%%%%%
%%%%%%%%PAGEBREAK%%%%%%%PAGEBREAK%%%%%%%%%
%%%%%%%%%%%%%%%%%%%%%%%%%%%%%%%%%%%%%%%%%%
%%%%%%%%%%%%%%%%%%%%%%%%%%%%%%%%%%%%%%%%%%
%%%%%%%%%%%%%%%%%%%%%%%%%%%%%%%%%%%%%%%%%%
%%%%%%%%%%%%%%%%%%%%%%%%%%%%%%%%%%%%%%%%%%
%%%%%%%%PAGEBREAK%%%%%%%PAGEBREAK%%%%%%%%%
%%%%%%%%%%%%%%%%%%%%%%%%%%%%%%%%%%%%%%%%%%
%%%%%%%%%%%%%%%%PAGEBREAK%%%%%%%%%%%%%%%%%
%%%%%%%%%%%%%%%%%%%%%%%%%%%%%%%%%%%%%%%%%%
%%%%%%%%PAGEBREAK%%%%%%%PAGEBREAK%%%%%%%%%
%%%%%%%%%%%%%%%%%%%%%%%%%%%%%%%%%%%%%%%%%%
%%%%%%%%%%%%%%%%%%%%%%%%%%%%%%%%%%%%%%%%%%
%%%%%%%%%%%%%%%%%%%%%%%%%%%%%%%%%%%%%%%%%%
%%%%%%%%%%%%%%%%%%%%%%%%%%%%%%%%%%%%%%%%%%
%%%%%%%%PAGEBREAK%%%%%%%PAGEBREAK%%%%%%%%%
%%%%%%%%%%%%%%%%%%%%%%%%%%%%%%%%%%%%%%%%%%
%%%%%%%%%%%%%%%%PAGEBREAK%%%%%%%%%%%%%%%%%
%%%%%%%%%%%%%%%%%%%%%%%%%%%%%%%%%%%%%%%%%%
%%%%%%%%PAGEBREAK%%%%%%%PAGEBREAK%%%%%%%%%
%%%%%%%%%%%%%%%%%%%%%%%%%%%%%%%%%%%%%%%%%%
%%%%%%%%%%%%%%%%%%%%%%%%%%%%%%%%%%%%%%%%%%
\begin{ekdosis}
  \begin{prose}
    \noindent
%---------------------
%tatrordhvaśaktiḥ / \E
%tatordhvaśaktiḥ \P
%rdhaśakti ardhaśakti \B
%rdhaśakti ardhaśakti \L
%tatrordhvaśaktiḥ / \N1
%tatra ūrdhva śaktiḥ / \D
%tatra ūrdhva śakti / \N2
%urdhvaśaktir         \U1
%tatrordhvaśaktiḥ// \U2
%---------------------
%There above is \textit{śakti},
%------------------------------
\app{\lem[wit={E,N1,U2}]{tatrordhvaśaktiḥ}
  \rdg[wit={P}]{tatordhvaśaktiḥ}
  \rdg[wit={U1}]{urdhvaśaktir}
  \rdg[wit={D}]{tatra ūrdhva śaktiḥ}
  \rdg[wit={N2}]{tatra ūrdhva śakti}
  \rdg[wit={B,L}]{rdhaśakti ardhaśakti}}/
%------------------------------
%etādṛśī  saṃjñā   ekā kalā vartate / \E
%ekādaśā  saṃjñā   ekā kalā vartate   \P
%etādṛśī  saṃjñā   ekā kalā vartate /  \N1
%etādṛśī  saṃjñā   ekā kalā varttate / \N2
%etādṛsaṃ saṃjñā   ekā kalā vartate / \D
%ekādaśā  saṃjñā   ekā kalā vartate / \B
%ekādaśā  saṃjñā   ekā kalā vartate / \L
%etādṛśī  saṃjñakā ekā kalā vartate /  \U1
%etādṛśā  saṃjñā   ekā kalā vartate/ \U2 
%---------------------
%Being designated as such she is one single digit. 
%------------------------------
\app{\lem[wit={ceteri}]{etādṛśī}
  \rdg[wit={U2}]{etādṛśā}
  \rdg[wit={D}]{etādṛsaṃ}
  \rdg[wit={P,B,L}]{ekādaśā}}
\app{\lem[wit={ceteri}]{saṃjñā}
  \rdg[wit={U1}]{saṃjñakā}}
ekā kalā vartate/ 
%------------------------------
%asyāḥ  kalāyā   dhyānakāraṇāt     puruṣo yadicchati / \E
%asyāḥ  kalāyā   dhyānakāraṇāt     puruṣo yadicchati ?Zeichen? \P
%asyāḥ  kalāyā   dhyānakāraṇāt     puruṣo yadicchati  tad bhavati \N1
%tasyāḥ kalāyāḥ  dhyānakāraṇāt     puruṣo yadicchati  tad bhavati \N2
%asyāḥ  kalāyā   dhyānakāraṇā      puruṣo yadicchati  tad bhavati \D
%asyāḥ  kalāyā   dhyānakāraṇāt /   puruṣo yadicchati / \B
%asyāḥ  kalāyā   dhyānakāraṇāt /   puruṣo yadicchati / \L
%asyā   kalāyā   dhyānakāraṇāt     puruṣo yadicchati tad bhavati vā \U1
%asyāḥ  kalāyāḥ  dhyānakāraṇāt //  puruṣo yadicchati // \U2
%---------------------
%Due to the exercise of meditation on this part the person manifests whatever he wishes for.
%------------------------------
\app{\lem[wit={ceteri}]{asyāḥ}
  \rdg[wit={U1}]{asyā}
  \rdg[wit={N2}]{tasyāḥ}}
\app{\lem[wit={ceteri}]{kalāyā}
  \rdg[wit={N2,U2}]{kalāyāḥ}}
\app{\lem[wit={ceteri}]{dhyānakāraṇāt}
  \rdg[wit={D}]{dhyānakāraṇā}}
puruṣo yad-icchati
\app{\lem[wit={N1,N2,D}, alt={tad bhavati}]{tad-bhavati}
  \rdg[wit={U1}]{tad bhavati vā}
  \rdg[wit={ceteri}]{\om}}/ 
%------------------------------
%tasya sukhabhogavataḥ / \E
%tasya sukhabhogavataḥ \P
%rājya-sukhabhogavataḥ \N1
%rājya-sukhabhogavataḥ \N2
%rājya-sukhabhogavṛtaḥ \D !!!
%tasya-khaṃ bhogavataṃ / \B
%tasya-sukhaṃ bhogavaṃtaṃ / \L
%rājya-sukhabhogavataḥ \U1
%tasya-sukhabhogavataḥ / \U2
%---------------------
%He is furnished with royal pleasure and enjoyment. 
%------------------------------
\note[type=testium, labelb=107, lem={rājyasukhabhoga°}]{Ysg: rājyasukhabhogavatah̤ strī vilāsavataḥ saṃgītavinoda prekṣāvato pi sādhakasya śuklapakṣacaṃdravat pratidinaṃ tejaso vapuṣaś ca vṛddiḥ puṇyapāpasya śārbhāvaḥ nijasva rūpaprakāśasāmarthaṃ dūrasthapy arthasya samīpastham iva darśanaṃ ca bhavati | cha | tad uktaṃ tattvajñānapradīpikāyāṃ ||}
\note[type=philcomm, labelb=108, lem={rājyasukhabhoga°}]{Here ends the testimonia of the \textit{Yogasaṃgraha}.}
\app{\lem[wit={D}]{rājyasukhabhogavṛtaḥ}
  \rdg[wit={N1,N2,U1}]{rājyasukhabhogavataḥ}
  \rdg[wit={E,P,U2}]{tasya sukhabhogavataḥ}
  \rdg[wit={B}]{tasya khaṃ bhogavataṃ}
  \rdg[wit={L}]{tasya sukhaṃ bhogavaṃtaṃ}}/
%------------------------------
%strīmadhye     vilāsavataḥ    saṃgītavilāsavataḥ vinodaprekṣāvataḥ        puruṣasya pratidinaṃ śuklapakṣe candrakalāvat   kalā     vardhate/   \E
%strīmadhye     vilāsavataḥ    saṃgītavinodaprekṣāvataḥ              eva   puruṣasya pratidinaṃ śuklapakṣe candrakalāvat   kalā     vardhate /  \P
%strīmadhye     vilāsavaṃtaṃ   saṃgītaṃ prekṣāvatāḥ //               evaṃ  puruṣasya pratidinaṃ śuklapakṣe caṃdrakalāvat / kalā     vartate /   \L
%strīmadhye     vilāsavaṃtaṃ   saṃgītaṃ vinodavaṃtaṃ prekṣāvaṃtāḥ // eva   puruṣasya pratidinaṃ śuklapakṣe caṃdrakalāvat / kalā     vartate /   \B
%strīmadhye     vilāsavataḥ    saṃgītavinodaprekṣyāvataḥ             evaṃ  puruṣasya pratidinaṃ śuklapakṣe candrakalā vṛddhivato?   vardhate / \N1
%śrī strīmadhye vilāsavataḥ    saṃgītavinodaprekṣāvataḥ              evaṃ  puruṣasya pratidinaṃ śuklapakṣa candrakalā vṛddhi vaṃto  varttate /  \N2
%strīmadhye     vilāsavataḥ // saṃgītavinodaprekṣyāvataḥ //          evaṃ  puruṣasya pratidinaṃ śuklapakṣe candrakalā vṛddhivato    vardhate / \D
%strīmadhye     vilāśavataḥ    saṃgītavinodaprekṣyāvataḥ             eka   puruṣasya pratidinaṃ śuklapakṣe caṃdrakalā vṛddhir       varddhate / \U1
%strīmadhye     vilāsavata     saṃgītavinodaprekṣāvata//             evaṃ  puruṣasya pratidinaṃ śuklapakṣe candrakalāvat   kalā     varttate/   \U2
%---------------------
%(Selbst) bei einem Menschen, der sich inmitten von Frauen vergnügt, (und) ein Musikvergnügen
%ansieht, wächst täglich die Kraft (kalā = śakti?) wie die "kalā" (Phase) des Mondes in der hellen Monatshälfte.
%The \textit{kalā} of a person grows daily, like the \textit{kalā} of the moon in the bright half of the month, even amusing oneself amongst women and watching a musical pleasure.
%(Even) amusing oneself amongst women, and watching musical pleasures, the \textit{kāla} of the person grows daily like the \textit{kalā} of the moon in the bright half of the month. 
%------------------------------
\app{\lem[wit={ceteri}]{strīmadhye}
  \rdg[wit={N2}]{śrī strīmadhye}}
\app{\lem[wit={ceteri}]{vilāsavataḥ}
  \rdg[wit={U2}]{vilāsavata°}
  \rdg[wit={L,B}]{vilāsavaṃtaṃ}} 
saṃgīta\app{\lem[wit={N1,D,U1},alt={°vinodaprekṣyāvataḥ}]{vinodaprekṣyāvataḥ}
  \rdg[wit={P,N2}]{°vinodaprekṣāvataḥ}
  \rdg[wit={U2}]{°vinodaprekṣāvata}
  \rdg[wit={B}]{°ṃ vinodavaṃtaṃ prekṣāvaṃtāḥ}
  \rdg[wit={E}]{°vilāsavataḥ vinodaprekṣāvataḥ}
  \rdg[wit={L}]{°ṃ prekṣāvatāḥ}}
 \app{\lem[wit={P,B}]{eva}
  \rdg[wit={ceteri}]{evaṃ}
  \rdg[wit={U1}]{eka}}
puruṣasya pratidinaṃ śuklapakṣe
candrakalā\app{\lem[wit={E,P,L,B,U2},alt={°vat kalā}]{vat kalā}
  \rdg[wit={N1,D}]{vṛddhivato}
  \rdg[wit={N2}]{vṛddhi vaṃto}
  \rdg[wit={U1}]{vṛddhir}}
\app{\lem[wit={E,P,N1,D,U1}]{vardhate}
  \rdg[wit={ceteri}]{vartate}}/
%------------------------------
%puṇyapāpe  'sya śarīraṃ   na spṛśataḥ /    \E
%\om                                     \P
%puṇyapāpe  asya śarīrena     spṛśataḥ /      \N1
%puṇyapāpe  asya śarīrena     spṛśataḥ /      \N2
%puṇyapāpe  asya śarīrena     spṛśataḥ /      \D
%puṇyapāpe  asya śarīrasya na spṛśataḥ // \B
%puṇyapāpe  asya śarīrasya na spṛśataḥ // \L
%puṇyapāpau asya śarīrena     spṛśāt         \U1
%puṇyapāpe  asya śarīraṃ   na spṛśataḥ // \U2
%---------------------
%puṇyapāpe\varc{puṇyapāpe \edprint \lalchand \oxford \nepal \dehlia}{\om \pune} 'sya\varc{'sya \edprint}{asya \nepal \dehlia \oxford \lalchand \om \pune} śarīrasya\varc{śarīrasya \lalchand \oxford}{śarīraṃ \edprint śarīrena \nepal \dehlia \om \pune} na\varc{na \edprint \oxford \lalchand}{\om \nepal \dehlia \pune} spṛśataḥ\varc{spṛśataḥ \edprint \lalchand \oxford \nepal \dehlia}{\om \pune} /
%---------------------
%His body is not affected by merit and sin. 
%------------------------------
\app{\lem[wit={ceteri}]{puṇyapāpe}
  \rdg[wit={U1}]{puṇyapāpau}
\rdg[wit={P}]{\om}}
\app{\lem[wit={E}]{'sya}
  \rdg[wit={P}]{\om}
  \rdg[wit={ceteri}]{asya}}  
śarīr\app{\lem[wit={B,L}, alt={°asya}]{asya}
  \rdg[wit={N1,N2,D,U1}]{°ena}
  \rdg[wit={E,U2}]{°aṃ}
  \rdg[wit={P}]{\om}}
\app{\lem[wit={E,B,L,U2}]{na}
  \rdg[wit={N1,N2,D,U1,P}]{\om}}
spṛ\app{\lem[wit={ceteri},alt={°śataḥ}]{śataḥ}
  \rdg[wit={U1}]{°śāt}}/
%------------------------------
%                          nirantaradhyānakaraṇāt     nijasvarūpaṃ prakāśanasāmarthyaṃ bhavati / \E
%                          \om until .....            nijasvarūpaprakāśasāmarthyaṃ     bhavati / \P
%                          niraṃtaraṃ dhyānakaraṇāt   nijasvarūpaprakāśasāmarthyaṃ     bhavati / \B
%                          niraṃtaraṃ dhyānakaraṇāt// nijasvarūpaprakāśasāmarthyaṃ     bhavati / \L
%                          nirantaradhyānakaraṇāt /   nijasvarūpaprakāśasāmarthyaṃ     bhavati / \N1 <-----
%                          niraṃtaradhyānakaraṇāt /   nijasvarūpaprakāśasāmarthyaṃ     bhavati // \N2
%                          nirantaradhyānakaraṇāt /   nijasvarūpaprakāśasāmarthyaṃ     bhavati / \D
%                          nirantaradhyānakaraṇāt /   nijasvarūpaprakāśasāmarthyaṃ     bhavati    \U1
%evaṃ puruṣasya pratidinaṃ niraṃtaraṃ dhyānakaraṇāt   nijasvarūpaṃ prakāśanasāmarthyaṃ bhavati// \U2 
%---------------------
%Due to uninterrupted meditation the power of the light of the innate nature arises. 
%------------------------------
\app{\lem[wit={ceteri}]{nirantaradhyānakaraṇāt}
  \rdg[wit={B,L}]{niraṃtaraṃ dhyānakaraṇāt}
  \rdg[wit={U2}]{evaṃ puruṣasya pratidinaṃ niraṃtaraṃ dhyānakaraṇāt}
  \rdg[wit={P}]{\om}}
nijasvarūpa\app{\lem[wit={ceteri},alt={°prakāśa°}]{prakāśa}
  \rdg[wit={E,U2}]{°ṃ prakāśana°}
}sāmarthyaṃ bhavati/
%------------------------------
%dūrasthopi ca dūrasthavastu                   samīpa iva   paśyati // \E
%dūrasthamapi                                  samīpam iva  paśyati // \N1
%dūrasthamapi                                  samīpaṃ iva  paśyati // \N2
%dūrasthamapy-arthaṃ                           samīpa iva   paśyati // \D
%dūrasthamapi padārthaṃ                        samīpa iva   paśyati // \B
%dūrasthamapi parārthaṃ                        samīpa iva   paśyati // \L
%dūrasthamapi padārthaṃ                        samīpa iva   paśyati // \P
%dūrasthamapy-arthaṃ                           samīpam eva  paśyati // \U1
%dūrasthamapi bhavati //dūrasthamapi padārthaṃ samīpa iva   paśyati // \U2
%------------------------------
%dūrasthamapyarthaṃ\varc{dūrasthamapyarthaṃ \dehlia}{dūrasthamapi padārthaṃ \oxford \pune durasthamapi parārthaṃ \lalchand sūrastamapi \nepal ca dūrasthavastu \edprint} samīpa\varc{samīpa \dehlia \edprint \lalchand \oxford \pune}{samīpam \nepal} iva paśyati //
%------------------------------
%He sees remotely located objects as if they'd be near.
%------------------------------
dūra\app{\lem[wit={D,U1},alt={°stham apy arthaṃ}]{stham-apy-arthaṃ}
  \rdg[wit={B,P}]{°stham api padārthaṃ}
  \rdg[wit={L}]{°stham api parārthaṃ}
  \rdg[wit={E}]{°sthopi ca dūrasthavastu}
  \rdg[wit={N1,N2}]{°stham api}
  \rdg[wit={U2}]{°stham api bhavati || dūrastham api padārthaṃ}}
\app{\lem[wit={ceteri}]{samīpa}
  \rdg[wit={N1}]{samīpam}
  \rdg[wit={N2}]{samīpaṃ}
  \rdg[wit={U1}]{samīpam}}
\app{\lem[wit={ceteri}]{iva}
  \rdg[wit={U1}]{eva}} 
paśyati\dd{}
\end{prose}
\end{ekdosis}
%%%%%%%%%%%%%%%%
%%%%%%%%%%%%%%%%
%%%%%%%%%%%%%%%%
%%%%%%%%%%%%%%%%
%%%%%%%%%%%%%%%
\begin{ekdosis}
 \ekddiv{type=ed}
   \bigskip
    \centerline{\textrm{\small{[Lakṣyayoga]}}}
    \bigskip
    \begin{prose}
%------------------------------
%idānīṃ sukhasādhyo lakṣyayogaḥ kathyate / \E
%idānīṃ sukhasādho  lakṣyayogaḥ kathyate / \P
%idānīṃ sukhasādho  lakṣayogaḥ  kathyate / \B
%idānīṃ sukhasādhe  lakṣayogaḥ  kathyate // \L
%idānīṃ sukhasādhyo lakṣyayogaḥ kathyate / \N1
%idānīṃ sukhasādhya lakṣanayogaḥ kathyate / \N2
%idānīṃ sukhasādhyo lakṣyayogaḥ kathyate / \D
%idānīṃ sukhasādhyopalakṣayogaḥ kathyate / \U1
%idānīṃ sukhasādhyo lakṣyayogaḥ kathyate / \U2
%------------------------------
%Now the yoga of fixation{\textit{lakṣyayoga}}, which is easily accomplished is explained. 
%------------------------------
      \note[type=source, labelb=109, lem={lakṣyayogaḥ}]{Ysv (YK): sukhasādhyaṃ lakṣayogam idānīṃ śrṛṇu pārvati | pañcadhā lakṣayogaś ca ūrdhvalakṣādibhedataḥ [ūrddha (PT)] ||1||}  
      idānīṃ
      \app{\lem[wit={ceteri}]{sukhasādhyo}
        \rdg[wit={N2}]{°sādhya}
        \rdg[wit={P,B}]{°sādho}
        \rdg[wit={L}]{°sādhe}
        \rdg[wit={U1}]{°sādhyopa°}}
    \app{\lem[wit={ceteri}]{lakṣyayogaḥ}
        \rdg[wit={B,L}]{lakṣayogaḥ}
        \rdg[wit={U1}]{°lakṣayogaḥ}
        \rdg[wit={N2}]{lakṣanayogaḥ}}
      kathyate/
%------------------------------      
%asya lakṣyayogasya  paṃcabhedā     bhavanti   ūrdhvalakṣyam / adholakṣyam / lakṣyam /      bāhyalakṣyam /  aṃtaralakṣyam /  \E
%asya lakṣyayogasya  paṃcabhedā     bhavanti   ūrdhvalakṣyam   adholakṣyam / madhyalakṣyam  bāhyalakṣyam    aṃtaralakṣyam /  \P
%asya lakṣayogasya   paṃce bhedāḥ   bhavaṃtī   ūrdhvalakṣam//  adholakṣam// bāhyakṣam//                     aṃtaralakṣam //  \B
%asya lakṣayogasya   paṃcabhedāḥ    bhavaṃti   ūrdhvalakṣam    adholakṣam// madhyalakṣam//  bāhyakṣam//     aṃtaralakṣam //  \L
%     lakṣyayogasya  paṃcabhedā     bhavaṃti// urdhvalakṣya    adholakṣya   bāhyalakṣya     madhyalakṣya    antaralakṣya //  \N1
%     lakṣanayogasya paṃcabhedā     bhavati//  urdhvalakṣa     adholakṣa    bāhyalakṣa      madhyalakṣa     antaralakṣa //   \N2
%     lakṣyayogasya  paṃcabhedā     bhavaṃti// urdhvalakṣya    adholakṣya   bāhyalakṣya     madhyalakṣya    antaralakṣya //  \D
%a----lakṣayogasya   paṃcabhedā     bhavati    urdhvalakṣa                  bāhyalakya      madhyalakṣa     aṃtaralakṣya     \U1
%asya lakṣayogasya   paṃcabhedā     bhavaṃti// ūrdhvalakṣam//  adholakṣam/  bāhyalakṣyam /  madhyalakṣaṃ/   sarvalakṣyam /   \U2
%------------------------------
%Of this yoga of fixation (\textit{lakṣyayoga}) there are five subdivisions: 1. The upward directed fixation {\textit{ūrdhvalakṣya}), 2. the downward directed fixation (\textit{adholakṣya}),3. the central fixation (\textit{madhyalakṣya}) 4. the outer fixation (\textit{baḥyalakṣya}), 5. the inner fixation (\textit{antaralakṣya}).
%------------------------------
      \note[type=source, labelb=110, lem={ūrdhvalakṣyam}]{Ysv (YK): ūrdhvalakṣam [ūrddha° (PT)] adholakṣaṃ [°lakṣo (PT)] vāhyalakṣaṃ [bāhyalakṣas (PT)] tathaiva ca | madhyalakṣaṃ [°lakṣas (PT)] tathā jñeyam [jñeyo (PT)] antarlakṣaṃ [°lakṣas (PT)] tathaiva ca ||2||}
      \app{\lem[wit={E,P,B,L,U2}]{asya}
        \rdg[wit={ceteri}]{\om}}
      \app{\lem[wit={ceteri},alt={lakṣya°}]{lakṣya}
        \rdg[wit={B,L,U2}]{lakṣa°}
        \rdg[wit={U1}]{alakṣa°}
        \rdg[wit={N2}]{lakṣana°}}yogasya
      \note[type=philcomm, labelb=111, lem={lakṣyayogasya}]{The designation of this type of yoga is transmitted in various variants. Given the list of the 15 yogas at the beginning of the text it is very likely that the correct name of the yoga is \textit{lakṣyayoga} and not \textit{lakṣayoga} or \textit{lakṣanayoga}.}
      \app{\lem[wit={ceteri}]{pañcabhedā}
        \rdg[wit={L}]{paṃcabhedāḥ}
        \rdg[wit={B}]{paṃce bhedāḥ}}
     \app{\lem[wit={ceteri}]{bhavanti}
       \rdg[wit={B}]{bhavaṃtī}
       \rdg[wit={N2,U1}]{bhavati}}/
    1 \app{\lem[wit={E,P}]{ūrdhvalakṣyam}
       \rdg[wit={L,B,N2}]{ūrdhvalakṣam}
       \rdg[wit={N1,D}]{urdhvalakṣya}
       \rdg[wit={N2,U1}]{urdhvalakṣa}}/
    2 adho\app{\lem[wit={E,P}, alt={°lakṣyam}]{lakṣyam}
       \rdg[wit={B,L,U2}]{°lakṣam}
       \rdg[wit={N1,D}]{°lakṣya}
       \rdg[wit={N2}]{°lakṣa}
       \rdg[wit={U1}]{\om}}/
    3 \app{\lem[wit={U2}]{bāhyalakṣyam}
       \rdg[wit={N1,D}]{bāhyalakṣya}
       \rdg[wit={N2}]{bāhyalakṣa}
       \rdg[wit={U1}]{bāhyalakya}
       \rdg[wit={B}]{bāhyakṣam}
       \rdg[wit={E}]{lakṣyam}
       \rdg[wit={P}]{madhyalakṣyam}
       \rdg[wit={L}]{madhyalakṣam}}/
\end{prose}
   \end{ekdosis}
   \ekdpb*{}
%%%%%%%%%%%%%%%%%%%%%%%%%%%%%%%%%%%%%%%%%%
%%%%%%%%%%%%%%%%%%%%%%%%%%%%%%%%%%%%%%%%%%
%%%%%%%%PAGEBREAK%%%%%%%PAGEBREAK%%%%%%%%%
%%%%%%%%%%%%%%%%%%%%%%%%%%%%%%%%%%%%%%%%%%
%%%%%%%%%%%%%%%%PAGEBREAK%%%%%%%%%%%%%%%%%
%%%%%%%%%%%%%%%%%%%%%%%%%%%%%%%%%%%%%%%%%%
%%%%%%%%PAGEBREAK%%%%%%%PAGEBREAK%%%%%%%%%
%%%%%%%%%%%%%%%%%%%%%%%%%%%%%%%%%%%%%%%%%%
%%%%%%%%%%%%%%%%%%%%%%%%%%%%%%%%%%%%%%%%%%
%%%%%%%%%%%%%%%%%%%%%%%%%%%%%%%%%%%%%%%%%%
%%%%%%%%%%%%%%%%%%%%%%%%%%%%%%%%%%%%%%%%%%
%%%%%%%%PAGEBREAK%%%%%%%PAGEBREAK%%%%%%%%%
%%%%%%%%%%%%%%%%%%%%%%%%%%%%%%%%%%%%%%%%%%
%%%%%%%%%%%%%%%%PAGEBREAK%%%%%%%%%%%%%%%%%
%%%%%%%%%%%%%%%%%%%%%%%%%%%%%%%%%%%%%%%%%%
%%%%%%%%PAGEBREAK%%%%%%%PAGEBREAK%%%%%%%%%
%%%%%%%%%%%%%%%%%%%%%%%%%%%%%%%%%%%%%%%%%%
%%%%%%%%%%%%%%%%%%%%%%%%%%%%%%%%%%%%%%%%%%
%%%%%%%%%%%%%%%%%%%%%%%%%%%%%%%%%%%%%%%%%%
%%%%%%%%%%%%%%%%%%%%%%%%%%%%%%%%%%%%%%%%%%
%%%%%%%%PAGEBREAK%%%%%%%PAGEBREAK%%%%%%%%%
%%%%%%%%%%%%%%%%%%%%%%%%%%%%%%%%%%%%%%%%%%
%%%%%%%%%%%%%%%%PAGEBREAK%%%%%%%%%%%%%%%%%
%%%%%%%%%%%%%%%%%%%%%%%%%%%%%%%%%%%%%%%%%%
%%%%%%%%PAGEBREAK%%%%%%%PAGEBREAK%%%%%%%%%
%%%%%%%%%%%%%%%%%%%%%%%%%%%%%%%%%%%%%%%%%%
%%%%%%%%%%%%%%%%%%%%%%%%%%%%%%%%%%%%%%%%%%
   \noindent
\begin{ekdosis}
     \ekddiv{type=ed}
  \begin{prose}
    4 \app{\lem[type={emendation}, resp={egoscr}]{madhyalakṣyam}
       \rdg[wit={N1,D}]{\korr madhyalakṣya}
       \rdg[wit={N2,U1}]{madhyalakṣa}
       \rdg[wit={U2}]{madhyalakṣaṃ}
       \rdg[wit={E,P}]{bāhyalakṣyam}
       \rdg[wit={L}]{bāhyakṣam}
       \rdg[wit={B}]{\om}}/
    5 \app{\lem[wit={E,P}]{antaralakṣyam}
       \rdg[wit={N1,D,U1}]{antaralakṣya}
       \rdg[wit={B,L}]{aṃtaralakṣam}
       \rdg[wit={N2}]{antaralakṣa}
       \rdg[wit={U2}]{sarvalakṣyam}}/
   \end{prose}
   \end{ekdosis}
%%%%%%%%%%%%%
%%%%%%%%%%%%%
%%%%%%%%%%%%%
%%%%%%%%%%%%%
%%%%%%%%%%%%%
 \begin{ekdosis}
   \ekddiv{type=ed}
   \bigskip
     \centerline{\textrm{\small{[1. Ūrdhvalakṣya]}}}
     \bigskip
\begin{prose}    
%------------------------------      
%prathamam ūrdhvalakṣyaṃ kathyate/  \E
%prathamam ūrdhvalakṣyaḥ kathyate/  \P
%atha      ūrdhvalakṣaṃ          // \L
%athama    urdhalakṣaṃ           // \B
%prathamaṃ urdhvalakṣaḥ  kathyate/  \N1
%prathamaṃ urdhvalakṣaḥ  kathyate/  \N2
%prathamaṃ urdhvalakṣaḥ  kathyate/  \D
%prathamaṃ urdhvalakṣya/ kathyate/  \U1
%prathamaṃ urdhvalakṣaṃ  kathyate/  \U2
%------------------------------
%At first the upward directed fixation{\textit{adholakṣya} is explained. 
%------------------------------
  \note[type=source, labelb=112, lem={ūrdhvalakṣyaṃ}]{Ysv (YK): lakṣaṇaṃ śrṛṇu caiṣāṃ hi phalaṃ jñātvā maheśvari | ākāśe dṛṣṭim āsthāya mana ūrdhvan [ūrddhan (PT)] tu kārayet ||3||}
  \app{\lem[wit={E,P},alt={prathamam}]{prathama\skp{m-ū}}
       \rdg[wit={N1,N2,D,U1,U2}]{prathamaṃ}
       \rdg[wit={L}]{atha}
       \rdg[wit={B}]{athama}}\app{\lem[wit={E},alt={ūrdhvalakṣyaṃ}]{\skm{m-ū}rdhvalakṣyaṃ}
       \rdg[wit={P}]{ūrdhvalakṣyaḥ}
       \rdg[wit={U1}]{urdhvalakṣya}
       \rdg[wit={L}]{ūrdhvalakṣaṃ}
       \rdg[wit={U2}]{urdhvalakṣaṃ}
       \rdg[wit={N1,N2,D}]{urdhvalakṣaḥ}
       \rdg[wit={B}]{urdhalakṣaṃ}}
     \app{\lem[wit={ceteri}]{kathyate}
       \rdg[wit={L,B}]{\om}}/
%------------------------------     
%ākāśamadhye dṛṣṭiḥ / \E
% \om                 \P
%ākāśamadhye dṛṣṭiḥ / \L
%ākāśamadhye dṛṣṭi    \B
%ākāśamadhye dṛṣṭiḥ / \N1
%ākāśamadhye dṛṣṭiḥ / \N2
%ākāśamadhye dṛṣṭiḥ / \D
%ākāśamadhye dṛṣṭiḥ / \U1
%ākāśamadhye dṛṣṭiḥ / \U2
%------------------------------
%The gaze (\textit{dṛṣṭi)) [should be] in the middle of the sky. 
%------------------------------
  \app{\lem[wit={ceteri}]{ākāśamadhye}
    \rdg[wit={P}]{\om}}
  \app{\lem[wit={ceteri}]{dṛṣṭiḥ}
    \rdg[wit={B}]{dṛṣṭi}
    \rdg[wit={P}]{\om}}/
%------------------------------     
%kadā ca    mana    ūrdhvaṃ      kṛtvā sthāpayati /     \E x
%atha ca    mana    ūrdhvaṃ      kṛtvā sthāpyate /      \P x
%atha vā            ūrdhvaṃ mana kṛtvā sthāpyate        \L
%atha vā            ūrdhvamana   kṛtvā sthāpyate        \B
%atha ca // mana    urdhvaṃ      kṛtvā sthāpyate /      \N1 x
%atha ca mana       ūrdhvaṃ      kṛtvā sthāpyate /      \N2 x
%atha vā mana       ūrdhaṃ       kṛtvā sthāpyate        \D x
%atha ca maner------ddhvaṃ       kṛtvā sthāpyate        \U1
%atha    mana       urdhvaṃ      kṛtvā sthāpyate//      \U2 x
%------------------------------
%And then having caused the mind to be directed upwards, it is caused to be fixed there. 
%------------------------------
  \app{\lem[wit={P,N1,N2,U1}]{atha ca}
    \rdg[wit={L,B,D}]{atha vā}
    \rdg[wit={U2}]{atha}
    \rdg[wit={E}]{kadā ca}}
  \app{\lem[wit={E,P,N2}]{mana ūrdhvaṃ}
    \rdg[wit={N1,U2}]{mana urdhvaṃ}
    \rdg[wit={D}]{mana ūrdhaṃ}
    \rdg[wit={U1}]{manerddhvaṃ}
    \rdg[wit={L}]{ūrdhvaṃ mana}
    \rdg[wit={B}]{ūrdhvamana}}
  kṛtvā
  \app{\lem[wit={ceteri}]{sthāpyate}
    \rdg[wit={E}]{sthāpayati}}/
%------------------------------ 
%etasya lakṣyasya  dṛḍhakaraṇāt   parameśvarasya tejasā saha dṛṣṭer-aikyaṃ  bhavati /  \E
%etasya lakṣyasya  dṛḍhakaraṇāt   parameśvarasya tejasā saha dṛṣṭer-aikyaṃ  bhavati /  \P
%etasya lakṣasya   dṛḍhīkṛtvā//   parameśvarasya teja---saha dṛṣṭair-aikā   bhavati //  \L
%etasya lakṣasya   dṛḍhīkṛtvā//   parameśvarasya teja---saha dṛṣṭair-aikā   bhavati //  \B
%etasya lakṣyasya  dṛḍhīkaraṇāt / parameśvarasya tejasā saha dṛṣteḥ aikyaṃ  bhavati /  \N1
%etasya lakṣaṇasya dṛḍhīkaraṇāt   parameśvarasya tejasā saha dṛṣteḥ ekaṃ    bhavati //  \N2
%etasya lakṣasya   dṛḍhīkaraṇāt// parameśvarasya tejasā saha dṛṣṭeḥ aikyaṃ  bhavati // \D
%etasya lakṣasya   dṛḍhīkaraṇāt/  parameśvarasya tejasā saha dṛṣṭer-aikyaṃ  bhavati/ \U1
%etasya lakṣasya   dṛḍhīkaraṇāt   parameśvarasya tenasā saha dṛṣṭer-aikyaṃ  bhavati // \U2
%------------------------------
%Due to the exercise of stabilizing of this fixation (\textit{lakṣya}) arises unity of the gazing point (\textit{dṛṣṭi}) with the light of the highest lord (\textit{parameśvara}). 
%------------------------------
\note[type=testium, labelb=113, lem={parameśvarasya}]{Ysv (YK): ūrdhvalakṣaṃ [ūrdha° (PT)] bhaved eṣā parameśasya caikatā |}
  etasya
  \app{\lem[wit={E,P,N1}]{lakṣyasya}
    \rdg[wit={ceteri}]{lakṣasya}
    \rdg[wit={N2}]{lakṣaṇasya}}
  \app{\lem[wit={ceteri}]{dṛḍhīkaraṇāt}
    \rdg[wit={E,P}]{dṛḍhakaraṇāt}
    \rdg[wit={L,B}]{dṛḍhīkṛtvā}}
  parameśvarasya
\app{\lem[wit={ceteri}]{tejasā}
  \rdg[wit={U2}]{tenasā}
  \rdg[wit={L,B}]{teja°}}
saha
\app{\lem[wit={E,P,U1,U2}]{dṛṣṭer-aikyaṃ}
  \rdg[wit={N1,D}]{dṛṣṭeḥ aikyaṃ}
  \rdg[wit={N2}]{dṛṣteḥ ekaṃ}
  \rdg[wit={L,B}]{ dṛṣṭair aikā}}
bhavati/  
%------------------------------
%atha cākāśa----madhye    yaḥ kaścidadṛṣṭaḥ   padārtho bhavati /  \E x
%atha cākāśa----madhye    yaḥ kaścidadṛṣṭaḥ   padārtho bhavati /  \P  x
%atha vākāśa----madhye    yaḥ kacciddṛṣṭaḥ    padārtho bhavati    \L  x
%athā cākāśa----madhye    yaḥ kaccit dṛṣṭaḥ   padārtho bhavati    \B   x
%atha ca ākāśa--madhye    yaḥ kaścitadṛṣtaḥ   padārthe bhavati /  \N1   x
%atha// ākāśa---madhye    yaḥ kaścita adṛṣtaḥ padārtha bhavati /  \N2  x 
%atha ca ākāśa--madhye    yaḥ kaścitadṛsṭaḥ   padārtho bhavati /  \D    x
%atha ca/ ākāśa-madhye    yaḥ kaścidadṛsṭaḥ   padārtho bhavati    \U1    x
%atha cākāśa----madhye    yaḥ kaściddṛsṭa-----padārtho bhavati /  \U2
%------------------------------
%And then an indefinable invisible object arises in the middle of the sky.
%------------------------------
\app{\lem[wit={ceteri}]{atha}
  \rdg[wit={B}]{athā}}
\app{\lem[wit={E,P,B,U2},alt={cākāśa°}]{cākāśa}
  \rdg[wit={N1,D,U1}]{ca ākāśa°}
  \rdg[wit={L}]{vākāśa°}
  \rdg[wit={N2}]{ākāśa°}}madhye
yaḥ
\app{\lem[wit={ceteri},alt={kaścid adṛṣṭaḥ}]{kaścid\skp{-}adṛṣṭaḥ}
  \rdg[wit={L}]{kaccid dṛṣṭaḥ}
  \rdg[wit={B}]{kaccit dṛṣṭaḥ}
  \rdg[wit={N2}]{kaścita adṛṣtaḥ}
  \rdg[wit={U2}]{kaścid dṛsṭa°}}
\app{\lem[wit={ceteri}]{padārtho}
  \rdg[wit={N1}]{padārthe}
  \rdg[wit={N2}]{padārtha}}
bhavati/ 
%------------------------------
%sa sādhakasya dṛṣṭigocaro bhavati//  \E
%sa sādhakasya dṛṣṭigocaro bhavati//  \P
%   sādhakasya dṛṣṭigocaro bhavati//  \L
%   sādhakasya dṛṣṭigocaro bhavatī    \B
%sa sādhakasya dṛṣṭigocare bhavati // \D  saḥ-Sonderregel -> ḥ fällt aus vor allen Konsonanten
%sa sādhakasya dṛṣṭigocare bhavati // \N1
%   sādhakasya dṛṣṭigocarā bhavati // \N2
%sa sādhakasya dṛṣṭigocaro bhavati    \U1
%   sādhakasya dṛṣṭigocare bhavati // \U2
%------------------------------
%It arises in the range of sight of the practitioner.  
%------------------------------
\app{\lem[wit={ceteri}]{sa}
  \rdg[wit={L,B,N2,U2}]{\om}}
sādhakasya
\app{\lem[wit={D,N1,U2}]{dṛṣṭigocare}
  \rdg[wit={ceteri}]{dṛṣṭigocaro}
  \rdg[wit={N2}]{dṛṣṭigocarā}}
\app{\lem[wit={ceteri}]{bhavati}
  \rdg[wit={B}]{bhavatī}}/
%------------------------------
%ayam evordhvalakṣyaḥ      \E
%ayam evordhvalakṣyaḥ      \P
%ayam evordhvalakṣaḥ  //   \L
%ayam evordhalakṣaḥ  //    \B
%ayam evordhvalakṣya  //   \N1
%ayam eva vodhalakṣaṇam // \N2
%ayam evordhvalakṣyaḥ //   \D
%ayam evordhvalakṣyaḥ      \U1
%ayam evordhvalakṣya //    \U2
%------------------------------
%This is truly the upward directed fixation (\textit{ūrdhvalakṣya}).
%------------------------------
aya\skp{m-e}\app{\lem[wit={E,P,D,U1},alt={evordhvalakṣyaḥ}]{\skm{m-e}vordhvalakṣayaḥ}
  \rdg[wit={L}]{°lakṣaḥ}
  \rdg[wit={B}]{evordhalakṣaḥ}
  \rdg[wit={N1,U2}]{°lakṣya}
  \rdg[wit={N2}]{eva vodhalakṣaṇam}}/
\end{prose}
\end{ekdosis}
%%%%%%%%%%%%%%%%
%%%%%%%%%%%%%%%
%%%%%%%%%%%%%%%%
%%%%%%%%%%%%%%%%
%%%%%%%%%%%%%%%%%
\begin{ekdosis}
  \ekddiv{type=ed}
   \bigskip
    \centerline{\textrm{\small{[2. Adholakṣya]}}}
    \bigskip
 \begin{prose}
%------------------------------
%                            nāsikāyāḥ  upari     dvādaśāṃgulamūlaparyantaṃ dṛṣṭiḥ sthirā karttavyā /   \E
%       athādholakṣaḥ        nāsikāyā   upari     dvādaśāṃgulaparyantaṃ     dṛṣṭiḥ sthirā karttavyā /   \P
%       athādholakṣaḥ //     nāsikāyā   upari     dvādaśāṃgulaparyaṃtaṃ     dṛṣṭiḥ sthirā karttavyā     \L
%       athādholakṣa //      nāsikāyā   upari     dvādaśāṃgulaparyaṃtaṃ     dṛṣṭiḥ sthirā karttavyā     \B
%       atha adholakṣyaḥ //  nāsikāyā   upari     dvādaśaṃgulaparyaṃtaṃ     dṛṣṭiḥ sthirā karttavyā //  \N1
%       atha adholakṣanaḥ // nāsikāyā   upari     dvādaśāṃgulaparyaṃtaṃ     dṛṣṭiḥ sthirā karttavyā //  \N2
%       atha adholakṣaḥ //   nāsikāyā   upari     dvādaśaṃgulaparyaṃtaṃ     dṛṣṭiḥ sthirā karttavyā //  \D
%       atha adholakṣa       nāsikāyā   upari     dvādaśaṃgulaparyaṃtaṃ     dṛṣṭi--sthirā karttavyā     \U1
%                            nāsikāyāḥ  upariṣṭāt    daśāṃgulaparyaṃtaṃ     dṛṣṭiḥ sthirā karttavyā //  \U2
%------------------------------
%Now the downward directed fixation object (\textit{adholakṣya}). One should stabilize the gaze within the circumference (\textit{paryanta}) of twelve \textit{aṅgula}s beyond the nose.
%------------------------------
   \note[type=source, labelb=114, lem={athādholakṣyaḥ}]{Ysv (YK): nāsikopari deveśi dvādaśāṅgulamānataḥ ||4|| dṛṣṭisthiran [dṛṣṭiḥ sthirā (PT)] tu karttavyam [karttavyā (PT)] adholakṣam idaṃ bhajet [bhaja (PT)] | tathā ca [athavā (PT)] nāsikāgre tu sthirā dṛṣṭir iyaṃ śṛṇu [bhavet (PT)] ||5|| yasya bhavet sthirā dṛṣṭiś cirāyuḥ [sthirā dṛṣṭiś cirāyuḥ syāt tathāsau (PT)] sthiradṛṣṭimān|}   
\app{\lem[type=emendation, resp=egoscr]{athādholakṣyaḥ}
  \rdg[wit={N1}]{\korr atha adholakṣyaḥ}
  \rdg[wit={P,L}]{athādholakṣaḥ}
  \rdg[wit={B}]{athādholakṣa}
  \rdg[wit={N2}]{atha adholakṣanaḥ}
  \rdg[wit={D}]{atha adholakṣaḥ}
  \rdg[wit={U1}]{atha adholakṣa}
  \rdg[wit={E,U2}]{\om}}/
\app{\lem[wit={ceteri}]{nāsikāyā}
  \rdg[wit={E,U2}]{nāsikāyāḥ}}
\app{\lem[wit={ceteri}]{upari}
  \rdg[wit={U2}]{upariṣṭāt}}
\app{\lem[wit={ceteri}]{dvādaśāṃgulaparyantaṃ}
  \rdg[wit={E}]{dvādaśāṃgulamūlaparyantaṃ}
  \rdg[wit={U2}]{daśāṃgulaparyaṃtaṃ}}
\app{\lem[wit={ceteri}]{dṛṣṭiḥ}
  \rdg[wit={U1}]{dṛṣṭi°}}
sthirā karttavyā/
%------------------------------
%atha vā nāsikāyā agre dṛṣṭiḥ sthirā karttavyā / \E
%atha vā nāsikāyā agre dṛṣṭiḥ sthirā karttavyā / \P
%\om / \L
%\om / \B
%atha vā nāsikāyā  agre dṛṣṭiḥ sthirā karttavyā // \N1
%atha vā nāsikā    agre dṛṣṭi-sthirā karttavyā      \N2
%atha vā nāsikāyā  agre dṛṣṭiḥ sthirā karttavyā // \D
%atha vā nāśikāyāḥ/ agre dṛṣṭiḥ/ sthirā karttavyā / \U1
%atha vā nāsikāyā  agre dṛṣṭiḥ sthirā karttavyā // \U2
%------------------------------
%Or one should stabilize the gaze onto the tip of the nose.
%------------------------------
\app{\lem[wit={ceteri}]{atha vā}
  \rdg[wit={L,B}]{\om}}
\app{\lem[wit={ceteri}]{nāsikāyā}
  \rdg[wit={U1}]{nāsikāyāḥ}
  \rdg[wit={N2}]{nāsika}}
\app{\lem[wit={ceteri}]{agre}
  \rdg[wit={L,B}]{\om}}
\app{\lem[wit={ceteri}]{dṛṣṭiḥ}
  \rdg[wit={N2}]{dṛṣṭi°}}
\app{\lem[wit={ceteri}]{sthirā}
  \rdg[wit={L,B}]{\om}}
\app{\lem[wit={ceteri}]{karttavyā}
  \rdg[wit={L,B}]{\om}}/ 
%------------------------------
%lakṣadūyasya  dṛḍhīkaraṇāt / dṛṣṭiḥ sthirā bhavati / \E
%lakṣadvayasya dṛṣṭīkaraṇāt / dṛṣṭiḥ sthirā bhavati / \P
%lakṣadvayasya dṛḍhīkaraṇāt   dṛṣṭi--sthiro bhavati / \L
%lakṣadvayasya dṛḍhīkaraṇān---dṛṣṭiḥ sthiro bhavatī   \B
%lakṣadvayasya dṛdhīkaraṇāt   dṛṣṭiḥ sthirā bhavati / \N1
%lakṣadvayasya dṛḍhīkaraṇād---dṛṣṭi--sthirā bhavati / \N2
%lakṣadvayasya dṛḍhīkaraṇāt   dṛṣṭiḥ sthirā bhavati / \D
%lakṣadvayasya dṛḍhīkaraṇāt   dṛṣṭiḥ sthirā bhavati / \U1
%lakṣadvayasya dṛḍhīkaraṇāt   dṛṣṭi--sthirā bhavati // \U2
%------------------------------
%The fixation becomes stable due to firm exercise [on one] of the twofold aims [of fixation]. 
%------------------------------
\app{\lem[wit={ceteri}]{lakṣadvayasya}   %emend to lakṣyadvayasya??? 
  \rdg[wit={E}]{lakṣadūyasya}} 
\app{\lem[wit={N2}, alt={dṛḍhīkaraṇād}]{dṛḍhīkaraṇā\skm{d-ṛ}}
  \rdg[wit={E,L,N1,D,U1,U2}]{dṛḍhīkaraṇāt}
  \rdg[wit={P}]{dṛṣṭīkaraṇāt}
  \rdg[wit={B}]{dṛḍhīkaraṇān}
}\app{\lem[wit={ceteri}, alt={dṛṣṭiḥ}]{\skp{d-ṛ}ṣṭiḥ}
  \rdg[wit={L,N2,U2}]{dṛṣṭi°}}
\app{\lem[wit={ceteri}]{sthirā}  
  \rdg[wit={B}]{sthiro}
  \rdg[wit={L}]{°sthiro}}
\app{\lem[wit={ceteri}]{bhavati}
  \rdg[wit={B}]{bhavatī}}/
%------------------------------
%pavanaḥ sthiro bhavati / \E
%pavanaḥ sthiro bhavati / \P
%\om                    / \L
%\om                    / \B
%pavanaḥ sthiro bhavati / \N1
%pavana--sthiro bhavati /   \N2
%pavanaḥ sthiro bhavati / \D
%pavana--sthiro bhavati  / \U1
%pavana--sthiro bhavati  / \U2
%------------------------------
%The breath becomes stable. 
%------------------------------
\app{\lem[wit={E,P,N1,D}]{pavanaḥ}
  \rdg[wit={N2,U1,U2}]{pavana°}
  \rdg[wit={L,B}]{\om}}
\app{\lem[wit={ceteri}]{sthiro}
  \rdg[wit={L,B}]{\om}}
\app{\lem[wit={ceteri}]{bhavati}
  \rdg[wit={L,B}]{\om}}/
%------------------------------
%āyurvarddhate / \E
%āyurvarddhate / \P
%āyurvarddhate / \L
%āyurvardhate /  \B
%āyurvardhate /  \N1
%āyurvardhate /  \N2
%āyurvardhate /  \D
%āyurvarddhate   \U1
%āyurvarddhate //  \U2
%------------------------------
%Vitality increases. 
%------------------------------
āyur-varddhate/
%%%%%%%%%%%%
%%%%%%%%%%%%
%%%%%%%%%%%%
%%%%%%%%%%%%
%%%%%%%%%%%%
\end{prose}
\end{ekdosis}
\ekdpb*{}
%%%%%%%%%%%%%%%%%%%%%%%%%%%%%%%%%%%%%%%%%%
%%%%%%%%%%%%%%%%%%%%%%%%%%%%%%%%%%%%%%%%%%
%%%%%%%%PAGEBREAK%%%%%%%PAGEBREAK%%%%%%%%%
%%%%%%%%%%%%%%%%%%%%%%%%%%%%%%%%%%%%%%%%%%
%%%%%%%%%%%%%%%%PAGEBREAK%%%%%%%%%%%%%%%%%
%%%%%%%%%%%%%%%%%%%%%%%%%%%%%%%%%%%%%%%%%%
%%%%%%%%PAGEBREAK%%%%%%%PAGEBREAK%%%%%%%%%
%%%%%%%%%%%%%%%%%%%%%%%%%%%%%%%%%%%%%%%%%%
%%%%%%%%%%%%%%%%%%%%%%%%%%%%%%%%%%%%%%%%%%
%%%%%%%%%%%%%%%%%%%%%%%%%%%%%%%%%%%%%%%%%%
%%%%%%%%%%%%%%%%%%%%%%%%%%%%%%%%%%%%%%%%%%
%%%%%%%%PAGEBREAK%%%%%%%PAGEBREAK%%%%%%%%%
%%%%%%%%%%%%%%%%%%%%%%%%%%%%%%%%%%%%%%%%%%
%%%%%%%%%%%%%%%%PAGEBREAK%%%%%%%%%%%%%%%%%
%%%%%%%%%%%%%%%%%%%%%%%%%%%%%%%%%%%%%%%%%%
%%%%%%%%PAGEBREAK%%%%%%%PAGEBREAK%%%%%%%%%
%%%%%%%%%%%%%%%%%%%%%%%%%%%%%%%%%%%%%%%%%%
%%%%%%%%%%%%%%%%%%%%%%%%%%%%%%%%%%%%%%%%%%
%%%%%%%%%%%%%%%%%%%%%%%%%%%%%%%%%%%%%%%%%%
%%%%%%%%%%%%%%%%%%%%%%%%%%%%%%%%%%%%%%%%%%
%%%%%%%%PAGEBREAK%%%%%%%PAGEBREAK%%%%%%%%%
%%%%%%%%%%%%%%%%%%%%%%%%%%%%%%%%%%%%%%%%%%
%%%%%%%%%%%%%%%%PAGEBREAK%%%%%%%%%%%%%%%%%
%%%%%%%%%%%%%%%%%%%%%%%%%%%%%%%%%%%%%%%%%%
%%%%%%%%PAGEBREAK%%%%%%%PAGEBREAK%%%%%%%%%
%%%%%%%%%%%%%%%%%%%%%%%%%%%%%%%%%%%%%%%%%%
%%%%%%%%%%%%%%%%%%%%%%%%%%%%%%%%%%%%%%%%%%
\begin{ekdosis}
  \ekddiv{type=ed}
    \centerline{\textrm{\small{[3. Bāhyalakṣya]}}}
    \bigskip
 \begin{prose}
\noindent
%------------------------------
%etad dūyam       api bāhyalakṣyam eva  bhavati      bāhyāṃtara       ākāśe         śūnyalakṣyaṃ    karttavyaḥ / \E
%etad dvayam      api bāhyalakṣyam eva  bhavati      bāhyābhyaṃtare   ākāśe cet     śūnyalakṣyaṃ    karttavyaḥ / \P
%etad dvayam      api bāhyalakṣam  eva  bhavati//    bāhyābhyaṃtare   ākāśacen      śūnyaṃ lakṣaṃ   karttavyā // \L
%etad dvayadvayam api bāhyalakṣam  eva  bhavatī//    bāhyābhyaṃtare   ākāśacvat     śūnyaṃ lakṣaṃ   karttavyā // \B
%etat advayam     eva bāhyalakṣam  api  kathyate //  bāhyo bhyaṃtaraṃ ākāśavat------śūnyalakṣyaḥ    karttavyaḥ / \N1
%etad dvayam      eva bāhyalakṣam  api  kathyate //  bāhyābhyaṃtaram--ākāśavat------śūnyalakṣaḥ     karttavyaḥ   \N2
%etat advayam     eva bāhyalakṣam  api  kathyate //  bāhyo bhyaṃtaraṃ ākāśavat //   śūnyalakṣyaḥ    karttavyaḥ / \D
%etat dvayam      eva bāhyalakṣyam api  kathyate/    bāhyābhyaṃtare   ākāśavat------śūnyalakṣyaḥ    karttavyaḥ  \U1
%etat dvayam      api bāhyalakṣyam eva  bhavati//    bāhyābhyaṃtare   ākāśe cet     śūnyalakṣyaṃ    karttavyaḥ / \U2
%------------------------------
%Just as this [aim] is twofold, also the external fixation is said to be [like this]. Internally or externally the aim of fixation is to be done onto the heavenly void.  
%------------------------------
\note[type=source, labelb=115, lem={}]{Ysv (YK): bāhyalakṣaṃ [vāhya° (PT)] svayaṃ jñeyaṃ yāti tattvanirāsinām [nivāsinām (PT)] ||6|| kāmināṃ tu bahir dṛṣṭiś cintādiṣu susiddhidā | etad bāhyamadhyalakṣaṃ dṛṣṭicintānirākulaḥ [iṣṭacintā nirākulam (PT)] ||7||}
\app{\lem[wit={P,L,N2},alt={etad dvayam}]{etad-dvaya\skp{m-e}}
  \rdg[wit={E}]{etad dūyam}
  \rdg[wit={B}]{etad dvayadvaya}
  \rdg[wit={N2,D}]{etat advayam}
  \rdg[wit={U1,U2}]{etat dvayam}}\app{\lem[wit={N1,N2,D,U1}, alt={eva}]{\skm{m-e}va}
  \rdg[wit={ceteri}]{api}} 
\app{\lem[wit={E,P,U1,U2},alt={bāhyalakṣyam}]{bāhyalakṣya\skp{m-a}}
  \rdg[wit={ceteri}]{°lakṣam}}\app{\lem[wit={N1,N2,D,U1},alt={api}]{\skm{m-a}pi}
  \rdg[wit={E,P,L,B,U2}]{eva}}
\app{\lem[wit={N1,N2,D,U1}]{kathyate}
  \rdg[wit={E,P,L,U2}]{bhavati}
  \rdg[wit={B}]{bhavatī}}/
\app{\lem[wit={N2},alt={bāhyābhyantaram}]{bāhyābhyantara\skm{m-ā}}                %Übersetzung nochmal überdenken! 
  \rdg[wit={N1,D}]{bāhyo bhyaṃtaraṃ}
  \rdg[wit={P,L,B,U1,U2}]{bāhyābhyaṃtare}
  \rdg[wit={E}]{bāhyāṃtara}}\app{\lem[wit={N1,N2,D,U1},alt={ākāśavat}]{\skp{m-ā}kāśavat}
  \rdg[wit={B}]{ākāśacvat}
  \rdg[wit={L}]{ākāśacen}
  \rdg[wit={P,U2}]{ākāśe cet}
  \rdg[wit={E}]{ākāśe}}
\app{\lem[wit={N1,D,U1}]{śūnyalakṣyaḥ}
  \rdg[wit={E,P,U2}]{śūnyalakṣyaṃ}
  \rdg[wit={N2}]{śūnyalakṣaḥ}
  \rdg[wit={L,B}]{śūnyaṃ lakṣaṃ}}
\app{\lem[wit={ceteri}]{karttavyaḥ}
  \rdg[wit={L,B}]{karttavyā}}/
%------------------------------
%jāgraddaśāyāṃ    calanadaśāyāṃ   bhojanadaśāyāṃ   sthitikāle sarvasthāne   śūnyasya dhyānakāraṇāt //                              \E
%jāgraddaśāyāṃ    calanadaśāyāṃ   bhojanaṃ daśāyāṃ sthitikāle sarvasthāne   śūnyasya dhyānakāraṇāt //                              \P
%jāgradādidaśāyāṃ calanadaśāyāṃ// bhojanadaśāyāṃ   sthitikāle sarvasthāneṣu śūnyasya dhyānakāraṇāt //                              \L
%jāgradādidaśāyāṃ calanadaśāyāṃ// bhojanadaśāyāṃ   sthitikāle sarvasthāneṣu śūnyasya dhyānakaraṇāt //                              \B
%jāgraddaśāyāṃ    cakabadaśāyāṃ   bhojanadaśāyāṃ   sthitikāle sarvvasthāne  śūnyasya dhyānakaraṇāt  maraṇatrāso na bhavati//       \N1
%jāyadaśāyāṃ      calanadaśāyāṃ/  bhojanadaśāyāṃ   sthitikāle sarvasthāne   śūnyasya dhyānakaraṇāt  maraṇatrāśo na bhavati//       \N2
%jāgraddaśāyāṃ    calanadaśāyāṃ   bhojanadaśāyāṃ   sthitikāle sarvvasthāne  śūnyasya dhyānakaraṇāt  maraṇatrāso na bhavati// śūnya \D
%jāgraddaśāyāṃ    calanadaśāyāṃ                    sthitikāle sarvasthāne   śūnyasya dhyānakaraṇāt/ maraṇasautrāṃ na bhavati vā    \U1
%jāgṛaddaśāyāṃ    calanadaśāyāṃ   bhojanadaśāyāṃ   sthitikāle sarvasthāne   śūnyasya dhyānakaraṇāt//                               \U2
%------------------------------
%The fear of dying does not arise due to the exercise of meditation on the void at all places during ones life - while eating, moving and waking. 
%------------------------------
\app{\lem[wit={ceteri}]{jāgraddaśāyāṃ}
    \rdg[wit={N2}]{jāgṛaddaśāyāṃ}
    \rdg[wit={N2}]{jāyadaśāyāṃ}
    \rdg[wit={L,B}]{jāgradādidaśāyāṃ}}
\app{\lem[wit={ceteri}]{calanadaśāyāṃ}
    \rdg[wit={N1}]{cakabadaśāyāṃ}}
\app{\lem[wit={ceteri}]{bhojanadaśāyāṃ}
    \rdg[wit={P}]{bhojanaṃ daśāyāṃ}
    \rdg[wit={U1}]{\om}}
  sthitikāle
\app{\lem[wit={ceteri}]{sarvasthāne}
    \rdg[wit={L,B}]{sarvasthāneṣu}}
  śūnyasya dhyānakāraṇāt
\app{\lem[wit={N1,D}]{maraṇatrāso}
    \rdg[wit={N2}]{maraṇatrāśo}
    \rdg[wit={U1}]{maraṇasautrāṃ}
    \rdg[wit={E,P,L,B,U2}]{\om}}
\app{\lem[wit={ceteri}]{na}
    \rdg[wit={E,P,B,U2}]{\om}}
\app{\lem[wit={N1,N2}]{bhavati}
    \rdg[wit={D}]{bhavati || śūnya}
    \rdg[wit={U1}]{bhavati vā}
    \rdg[wit={ceteri}]{\om}}\dd{}
 \end{prose}
\end{ekdosis}
%%%%%%%%%%%%%%%%%
%%%%%%%%%%%%%%%
%%%%%%%%%%%%%%%%
%%%%%%%%%%%%%%%%
%%%%%%%%%%%%%%%%%
\begin{ekdosis}
  \ekddiv{type=ed}
       \bigskip
    \centerline{\textrm{\small{[The Rājayogin's Body]}}}
    \bigskip
    \begin{prose}
%------------------------------  
%idānīṃ rājayogayuktasya           śarīre yaccihnaṃ  tat    kathyate / \E
%idānīṃ rājayogayuktasya puruṣasya yaccharīracihnaṃ         kathyate / \P
%idānīṃ rājayogayuktasya puruṣasya          cinhnaṃ         kathyate / \L
%idānīṃ rājayogayuktasya puruṣasya          cinhnaṃ         kathyate // \B
%idānīṃ rājayogayuktasya puruṣasya yaccarīracihnaṃ   tat    kathyate / \N1
%idānīṃ rājayogayuktasya puruṣasya yaccharīracihūṃ   tat    kathyate// \N2
%idānīṃ rājayogayuktasya puruṣasya yaccarīracihnaṃ   tat    kathyate / \D
%idānīṃ rājayogayuktasya puruṣasya yaccharīre cinhaṃ tata   kathyate \U1
%idānīṃ rājayogayuktasya puruṣasya yat śarīracinhaṃ         kathyate / \U2
%------------------------------
%Now it is said that this is the characteristic of the embodied person who is endowed with the royal yoga:
%------------------------------
\note[type=source, labelb=116, lem={rājayoga°}]{Ysv (PT): idānīṃ kathayiṣyāmi rājayogasya lakṣaṇam | rājayoge kṛte puṃbhiḥ siddhicihnaṃ bhavediti |}
      idānīṃ rājayogayuktasya
  \app{\lem[wit={ceteri}]{puruṣasya}
    \rdg[wit={E}]{\om}}
  \app{\lem[wit={N1,D,P},alt={yac carīracihnaṃ}]{yac-carīracihnaṃ}
    \rdg[wit={U2}]{yat śarīracinhaṃ}
    \rdg[wit={E}]{śarīre yac cihnaṃ}
    \rdg[wit={U1}]{yac charīre cinhaṃ}
    \rdg[wit={N2}]{yac charīracihūṃ}
    \rdg[wit={L,B}]{cinhnaṃ}}
  \app{\lem[wit={E,N1,N2,D}]{tat}
    \rdg[wit={U1}]{tata}
    \rdg[wit={ceteri}]{\om}}
  kathyate/
%------------------------------  
%tatsarvatra pūrṇo bhavati / \E
%tatsarvatra pūrṇā bhavati / \P
%tatsarvatra pūrṇo bhavati / \L
%tatsarvatra pūrṇo bhavatī / \B
%  sarvvatra pūrṇo bhavati / \N1
%  sarvvatra pūrṇā bhavati  \N2
%  sarvvatra pūrṇo bhavati  \D
%  sarvvatra pūrṇo bhavati   \U1
%tatsarvatra pūrṇo bhavati// \U2
%------------------------------
%Abundance arises at all times. %Alternative=permanent Abundance arises because of that.   
%------------------------------
\note[type=source, labelb=117, lem={pūrṇo}]{Ysv (PT): paripūrṇaṃ bhavec cittaṃ jagatstho 'pi jagadbahiḥ |}
  \app{\lem[wit={N1,N2,D,U1},alt={sarvatra°}]{sarvatra}
  \rdg[wit={ceteri}]{tatsarvatra°}}
\app{\lem[wit={ceteri}, alt={°pūrṇo}]{pūrṇo}
  \rdg[wit={P,N2}]{pūrṇā}}
\app{\lem[wit={ceteri}]{bhavati}
  \rdg[wit={B}]{bhavatī}}/
%------------------------------  
%pṛthivyāḥ dūre tiṣṭhati / \E
%pṛthivyāḥ hare tiṣṭhati / \P
%\om                      \L
%\om                      \B
%pṛthivyāḥ dūre  tiṣṭhati / \N1
%pṛthivyāḥ dūra  tiṣṭhati / \N2
%pṛthivyāḥ dūre  tiṣṭhati / \D
%pṛthivyāḥ ddūre tiṣṭhati / \U1 %emend to na tiṣṭhati? 
%pṛthivyā dūraṃ  tiṣṭhati // \U2 !!dūraṃ
%------------------------------
%No distances exist on earth.
%------------------------------
\app{\lem[type=conjecture, resp=egoscr]{pṛthivyāṃ}
  \rdg[wit={ceteri}]{\conj pṛthivyāḥ}
  \rdg[wit={U2}]{pṛthivyā}
  \rdg[wit={L,B}]{\om}} 
\app{\lem[wit={U2}]{dūraṃ}
  \rdg[wit={E,N1,D}]{dūre}
  \rdg[wit={U1}]{ddūre}
  \rdg[wit={N2}]{dūra}
  \rdg[wit={L,B}]{\om}}
\app{\lem[type=conjecture, resp=egoscr]{na tiṣṭhati}
  \rdg[wit={ceteri}]{\conj tiṣṭhati}
  \rdg[wit={L,B}]{\om}}/
%------------------------------
%pṛthivyāṃ vyāpya tiṣṭhati / \E
%pṛthi-----vyāpya tiṣṭhati / \P
%\om                         \L
%\om                         \B
%pṛthvāṃ vyāpya   tiṣṭhati /   \N1
%pṛthvīṃ vyāpya   tiṣṭhati /   \N2
%pṛthvīṃ vyāpya   tiṣṭhati /   \D  %geht auch pṛthu für Erde? 
%\om   \U1
%pṛthivyā vyāti   tiṣṭhati     \U2
%------------------------------
%He dwells on earth having pervaded [it]. 
%------------------------------
\app{\lem[type=emendation, resp=egoscr]{pṛthivīṃ}
  \rdg[wit={E}]{pṛthivyāṃ}
  \rdg[wit={P}]{pṛthi°}
  \rdg[wit={N1}]{pṛthvāṃ}
  \rdg[wit={N2,D}]{pṛthvīṃ}
  \rdg[wit={U2}]{pṛthivyā}
  \rdg[wit={L,B,U2}]{\om}}
\app{\lem[wit={ceteri}]{vyāpya}
  \rdg[wit={U2}]{vyāti}
  \rdg[wit={L,B,U1}]{\om}} 
\app{\lem[wit={ceteri}]{tiṣṭhati}
  \rdg[wit={L,B,U2}]{\om}}/
%------------------------------
% yasya janmamaraṇe  na staḥ sukhaṃ na bhavati /  \E
% yasya janmamaraṇe  na staḥ sukhaṃ na bhavati /  \P
% \om                                            \L
% \om                                            \B
% yasya janmamaraṇe  na staḥ sukhaṃ na bhavati /  \N1
% yasya janmamaraṇe  na staḥ sukhaṃ na bhavati /  \N2
% yasya janmamaraṇe  na staḥ sukhaṃ na bhavati /  \D
% \om                                            \U1
% yasya jananamaraṇe na staḥ sukhaṃ na bhavati /  \U2 maraṇe nom/acc dual! staḥ von as 3. dual 
%------------------------------
% Birth and death both do not exist. Happiness does not exist. 
% ------------------------------
\note[type=source, labelb=118, lem={janma°}]{Ysv (PT): na kṣobho janma mṛtyuś ca na duḥkhaṃ na sukhaṃ tathā |}
\app{\lem[wit={ceteri}]{yasya}
  \rdg[wit={L,B,U1}]{\om}}
\app{\lem[wit={ceteri}]{janmamaraṇe}
  \rdg[wit={U2}]{jananamaraṇe}
  \rdg[wit={L,B,U1}]{\om}}
\app{\lem[wit={ceteri}]{na}
  \rdg[wit={L,B,U1}]{\om}}
\app{\lem[wit={ceteri}]{staḥ}
  \rdg[wit={L,B,U1}]{\om}}/
\app{\lem[wit={ceteri}]{sukhaṃ}
  \rdg[wit={L,B,U1}]{\om}}
\app{\lem[wit={ceteri}]{na}
  \rdg[wit={L,B,U1}]{\om}}
\app{\lem[wit={ceteri}]{bhavati}
  \rdg[wit={L,B,U1}]{\om}}/
% ------------------------------
% \om                 \E
% \om                 \P
% \om                 \L
% \om                  \B
% duḥkhaṃ na bhavati / \N1
% duḥkhaṃ na bhavati / \N2
% duḥkham na bhavati / \D
% \om                  \U1
% \om                  \U2
% ------------------------------
%Suffering does not exist. 
%------------------------------
\app{\lem[wit={N1,N2,D}]{duḥkhaṃ}
  \rdg[wit={ceteri}]{\om}} 
\app{\lem[wit={N1,N2,D}]{na}
  \rdg[wit={ceteri}]{\om}} 
\app{\lem[wit={N1,N2,D}]{bhavati}
  \rdg[wit={ceteri}]{\om}}/
%------------------------------
% \om               \E
% kalaṃ na bhavati  \L
% kulaṃ na bhavatī// \B
% kūlaṃ na bhavati / \P
% kūlaṃ na bhavati / \N1
% kūlaṃ na bhavati / \N2
% kūlaṃ na bhavati / \D
% \om               \U1
% kulaṃ na bhavatī// \U2
%------------------------------
%Impediment does not exist.
%------------------------------
\note[type=source, labelb=119, lem={kūlaṃ}]{bhedābhedau manaḥsthau na jñānaṃ śīlaṃ kulaṃ tathā |}
\app{\lem[wit={P,N1,N2,D}]{kūlaṃ}
  \rdg[wit={B,U2}]{kulaṃ}
  \rdg[wit={L}]{kalaṃ}
  \rdg[wit={E,U1}]{\om}}
\app{\lem[wit={ceteri}]{na}
  \rdg[wit={E,U1}]{\om}}
\app{\lem[wit={ceteri}]{bhavati}
  \rdg[wit={B,U2}]{bhavatī}
  \rdg[wit={E,U1}]{\om}}/
%------------------------------
% \om                  \E
% śītalaṃ na bhavati / \P
% \om                  \L
% \om                  \B
% śīlaṃ na bhavati /   \N1
% śīlaṃ na bhavati /   \N2
% śīlaṃ na bhavati /   \D
% śīlaṃ na bhavati /   \U1
% śīlaṃ na bhavati /   \U2
%------------------------------
% Habit doesn't exist. 
% ------------------------------
\app{\lem[wit={ceteri}]{śīlaṃ}
  \rdg[wit={P}]{śītalaṃ}
  \rdg[wit={E,L,B}]{\om}}
\app{\lem[wit={ceteri}]{na}
  \rdg[wit={E,L,B}]{\om}}
\app{\lem[wit={ceteri}]{bhavati}
  \rdg[wit={E,L,B}]{\om}}/
%------------------------------
% \om                 \E
% sthānaṃ na bhavati / \P
% \om                  \L
% \om                  \B
% sthānaṃ na bhavati / \N1
% sthānaṃ na bhavati / \N2
% sthānaṃ na bhavati / \D
% sthānaṃ na bhavati / \U1
% sthānaṃ na bhavati / \U2
%------------------------------
% Place does not exist. 
%------------------------------
\app{\lem[wit={ceteri}]{sthānaṃ}
  \rdg[wit={E,L,B}]{\om}}
\app{\lem[wit={ceteri}]{na}
  \rdg[wit={E,L,B}]{\om}}
\app{\lem[wit={ceteri}]{bhavati}
  \rdg[wit={E,L,B}]{\om}}/
%------------------------------
% \om                                                                             \E
%asya siddhasya manomadhye īśvarasaṃbaṃdhī prakāśo niraṃtaraṃ     pratyakṣo bhavati  \P
%asya siddhasya manomadhye īśvarasaṃbaṃdhi prakāśo  niraṃtaraṃ    pratyakṣo bhavati  \L
%asya siddhasya manomadhye īśvaraṃ saṃbaṃdhī prakāśo  niraṃtaraṃ  pratyakṣo bhavatī//  \B
%asya siddhasya manomadhye īśvarasaṃbaṃdhī prakāśaḥ niraṃtaraṃ    pratyakṣa bhavati  \N1
%asya siddhasya manomadhye īśvarasaṃbaṃdhī prakāśaḥ niraṃtaraṃ    pratyakṣa bhavati/  \N2
%asya siddhasya manomadhye īśvarasaṃbaṃdhi prakāśaḥ niraṃtaraṃ    pratyakṣo bhavati  \D
%asya siddhasyaṃ pṛthivī vyāpya tiṣṭhati yasya yanma maraṇai na saḥ sukhaṃ na bhati kulaṃ na bhavati śīlaṃ na bhavati sthānaṃ na bhavati ..... asya siddhasya manomadhye īśvarasaṃbaṃdhī prakāśaḥ niraṃtaraṃ pratyakṣo bhavati  \U1
%asya siddhasya manomadhye īśvarasaṃbaṃdhī prakāśo nirattaraṃ  pratyakṣo bhavati//  \U2
%------------------------------
%The manifestation of permanent perception of the connection with god arises in the middle of the mind of this accomplished one. 
%------------------------------
\note[type=source, labelb=120, lem={prakāśo}]{Ysv (PT): prakāśakuśasambandhiprasaṅgo 'yaṃ nirantaram | sarvaprakāśako'sau tu naṣṭabhedādir eva ca |}
\app{\lem[wit={ceteri}]{asya}
  \rdg[wit={E}]{\om}}
\app{\lem[wit={ceteri}]{siddhasya}
  \rdg[wit={U1}]{siddhasyaṃ pṛthivī vyāpya tiṣṭhati yasya yanma maraṇai na saḥ sukhaṃ na bhati kulaṃ na bhavati śīlaṃ na bhavati sthānaṃ na bhavati asya siddhasya}
  \rdg[wit={E}]{\om}}
%\note[type=philcomm, labelb=s34.z3, lem={asya siddhasyaṃ}]{U\textsubscript{1} repeats the whole section from \textit{pṛthivī} to \ldots \textit{sthānaṃ na bhavati} due to an eyeskip in the process of copying.}
\app{\lem[wit={ceteri}]{manomadhye}
  \rdg[wit={E}]{\om}}
\app{\lem[wit={ceteri}]{īśvarasaṃbandhī}
  \rdg[wit={B}]{īśvaraṃ saṃbaṃdhī}
  \rdg[wit={E}]{\om}}
\app{\lem[wit={ceteri}]{prakāśo}
  \rdg[wit={N1,N2,D,U1}]{prakāśaḥ}
  \rdg[wit={E}]{\om}}
\app{\lem[wit={ceteri}]{nirantaraṃ}
  \rdg[wit={U2}]{nirattaraṃ}
  \rdg[wit={E}]{\om}}
\app{\lem[wit={ceteri}]{pratyakṣo}
  \rdg[wit={N1}]{prakyakṣa}
  \rdg[wit={E}]{\om}}
\app{\lem[wit={ceteri}]{bhavati}
  \rdg[wit={B}]{bhavatī}
  \rdg[wit={E}]{\om}}/
\end{prose}
\end{ekdosis}
\ekdpb*{}
%%%%%%%%%%%%%%%%%%%%%%%%%%%%%%%%%%%%%%%%%%
%%%%%%%%%%%%%%%%%%%%%%%%%%%%%%%%%%%%%%%%%%
%%%%%%%%PAGEBREAK%%%%%%%PAGEBREAK%%%%%%%%%
%%%%%%%%%%%%%%%%%%%%%%%%%%%%%%%%%%%%%%%%%%
%%%%%%%%%%%%%%%%PAGEBREAK%%%%%%%%%%%%%%%%%
%%%%%%%%%%%%%%%%%%%%%%%%%%%%%%%%%%%%%%%%%%
%%%%%%%%PAGEBREAK%%%%%%%PAGEBREAK%%%%%%%%%
%%%%%%%%%%%%%%%%%%%%%%%%%%%%%%%%%%%%%%%%%%
%%%%%%%%%%%%%%%%%%%%%%%%%%%%%%%%%%%%%%%%%%
%%%%%%%%%%%%%%%%%%%%%%%%%%%%%%%%%%%%%%%%%%
%%%%%%%%%%%%%%%%%%%%%%%%%%%%%%%%%%%%%%%%%%
%%%%%%%%PAGEBREAK%%%%%%%PAGEBREAK%%%%%%%%%
%%%%%%%%%%%%%%%%%%%%%%%%%%%%%%%%%%%%%%%%%%
%%%%%%%%%%%%%%%%PAGEBREAK%%%%%%%%%%%%%%%%%
%%%%%%%%%%%%%%%%%%%%%%%%%%%%%%%%%%%%%%%%%%
%%%%%%%%PAGEBREAK%%%%%%%PAGEBREAK%%%%%%%%%
%%%%%%%%%%%%%%%%%%%%%%%%%%%%%%%%%%%%%%%%%%
%%%%%%%%%%%%%%%%%%%%%%%%%%%%%%%%%%%%%%%%%%
%%%%%%%%%%%%%%%%%%%%%%%%%%%%%%%%%%%%%%%%%%
%%%%%%%%%%%%%%%%%%%%%%%%%%%%%%%%%%%%%%%%%%
%%%%%%%%PAGEBREAK%%%%%%%PAGEBREAK%%%%%%%%%
%%%%%%%%%%%%%%%%%%%%%%%%%%%%%%%%%%%%%%%%%%
%%%%%%%%%%%%%%%%PAGEBREAK%%%%%%%%%%%%%%%%%
%%%%%%%%%%%%%%%%%%%%%%%%%%%%%%%%%%%%%%%%%%
%%%%%%%%PAGEBREAK%%%%%%%PAGEBREAK%%%%%%%%%
%%%%%%%%%%%%%%%%%%%%%%%%%%%%%%%%%%%%%%%%%%
%%%%%%%%%%%%%%%%%%%%%%%%%%%%%%%%%%%%%%%%%%
\begin{ekdosis}
  \begin{prose}
    \noindent
%------------------------------
%sa ca prakāśo na śīto na coṣṇo na śveto na pīto bhavati/ \E
%sa ca prakāśo na śīto na coṣṇo na śveto na pīto bhavati/ \P
%sa ca prakāśo na śīto na coṣṇo na śveto na pīto bhavatī// \L
%sa ca prakāśo na śīto na coṣṇo na śveto na pīto bhavatī// \B
%sa ca prakāśo na śīto na coṣṇo na śveto na pīto bhavati/ \N1
%sa ca prakāśo na śīto na coṣṇo na śveto na pīto bhavati    \D
%sa ca prakāśo na śīto na coṣṇo na kheto na pīto bhavati/ \N2
%sa ca prakāśo na śīto na ?hbho?na kheto na pīto bhavati // \U1
%sa ca prakāśo// na śīto na coṣṇo na śveto pīto na bhavati // \U2
%------------------------------
%And he is shining - not cold, and not hot, not white [and] not yellow. 
%------------------------------
sa ca prakāśo na śīto na
\app{\lem[wit={ceteri}]{coṣṇo}
  \rdg[wit={U1}]{...o}}
na
\app{\lem[wit={ceteri}]{śveto}
  \rdg[wit={N2,U1}]{kheto}}
\app{\lem[wit={ceteri}]{na pīto}
  \rdg[wit={U2}]{pīto na}}
\app{\lem[wit={ceteri}]{bhavati}
  \rdg[wit={L,B}]{bhavatī}}/
%------------------------------
%tasya na jātir na kiñciccihnam  \E
%tasya na jātir na kiñciccihnaṃ  \P
%tasya na jātir na kiṃciccinhaṃ  \L
%tasya na jātir na kiṃciccinhaṃ  \B
%tasya na jātir na kiṃciccihūṃ  \N1
%tasya na jāti na kiṃciccihūṃ//  \D
%tasya na jāti na  kiṃciccihūṃ  \N2
%tasya na jātir na kiṃcit khecha cinhaṃ  \U1
%tasya na jānāti na kiṃcit cinhaṃ //  \U2
%------------------------------
%Neither is there birth of him, nor does he have any attributes.
%------------------------------
\note[type=source, labelb=121, lem={jātir}]{asya jāterna cihnañ ca niṣkalo 'yaṃ nirañjanaḥ | ananto 'yaṃ mahājyotir vāñchāṃ bhogaṃ dadāti ca |}
tasya na
\app{\lem[wit={ceteri}, alt={jātir}]{jāti\skp{r-na}}
  \rdg[wit={D,N2}]{jāti}
  \rdg[wit={U2}]{jānāti}
}\skm{r-na}
\app{\lem[wit={ceteri}, alt={kiñcic cihnaṃ}]{kiñcic\skp{-}cihnaṃ}
  \rdg[wit={E}]{°cihnam}
  \rdg[wit={D,N1,N2}]{°cihūṃ}
  \rdg[wit={U1}]{kiṃcit khecha cinhaṃ}
  \rdg[wit={U2}]{na kiṃcit cinhaṃ}}/
%------------------------------
%ayaṃ   ca niṣkalo   niraṃjanaḥ   alakṣyaś ca bhavati \E
%ayaṃ   ca niṣkalo   niraṃjanaḥ   alakṣyaś ca bhavati \P
%vyayaṃ ca niṣkalo   niraṃjanaṃ// alakṣaś  ca bhavati// \L
%vyayaṃ ca nīṣkalo   niraṃjanaṃ// alakṣaś  ca bhavatī// \B
%ayaṃ   ca niṣkalo   niraṃjanaḥ// alakṣyaś ca bhavati// \D
%ayaṃ   ca nīṣkalo   niraṃjanaḥ   alakṣaś  ca bhavati// \N1
%ayaṃ   ca niṣkalo   niraṃjanaḥ   alakṣaś  ca bhavati// \N2
%ayaṃ   ca niḥkalo   niraṃjanaḥ   alakṣyaḥ    bhavati/ \U1
%ayaṃ   ca nīṣkalo   niraṃjanaḥ// alakṣyaḥ    bhavati// \U2
%------------------------------
%And he is without parts, immacule and uncharacterized.  
%------------------------------
\app{\lem[wit={ceteri}]{ayaṃ}
  \rdg[wit={L,B}]{vyayaṃ}}
ca
\app{\lem[wit={ceteri}]{niṣkalo}
  \rdg[wit={B,U2}]{nīṣkalo}
  \rdg[wit={U1}]{niḥkalo}}
nirañjanaḥ/
\app{\lem[wit={ceteri}, alt={alakṣyaś}]{alakṣya\skp{ś-ca}}
  \rdg[wit={U1,U2}]{alakṣyaḥ}
  \rdg[wit={L,B,N1,N2}]{alakṣaś}
}\app{\lem[wit={ceteri}, alt={ca}]{\skm{ś-ca}}
  \rdg[wit={U1,U2}]{\om}}
\app{\lem[wit={ceteri}]{bhavati}
  \rdg[wit={B}]{bhavati}}/
%------------------------------
%atha ca phaladvaṃde  na         kāminy āder   yasyecchā         na bhavati // \E
%atha ca phalacaṃda   na         kāminy āder   yasyochā          na bhavati    \P
%atha ca phalavaṃda   na         kāminy ādir   yasya             na bhavati    \L
%atha ca phalaṃ jaṃda na         kāminy ādar   yasye             na bhavatī    \B
%atha ca phalacaṃdra  na         kāminy āder   yasya  yasyeccha     bhavati/   \N1
%atha ca phalacaṃda   na         kāminy āde    yasya  yasyechā      bhavati//  \D
%atha ca phalaṃ/caṃdra           kāminy āder   yasya  yasyeccha     bhavati/   \N2
%atha ca phalaṃ caṃda na         kāminy āder   yasya  yaṃ           bhavati    \U1
%atha ca phalacaṃda   na         kāminy āder   yasyechā             bhavati//  \U2
%------------------------------
%And his desire etc. doesn't arise in [situations of] lust [and] is not located within the duality of the result.  
%------------------------------
\note[type=source, labelb=122, lem={yasyecchā}]{Ysv (PT): asya citte nānurāgo virāgo na bhaved iti | rājya prāpte 'pi no harṣo hānau duḥkhaṃ bhaven nahi | kvacid vastuni deśasya niḥsvane keṣu kutracit |}
aatha ca
\app{\lem[wit={E}]{phaladvande}
     \rdg[wit={P,D,U2}]{phalacaṃda}
     \rdg[wit={U1}]{phalaṃ caṃda}
     \rdg[wit={L}]{phalavaṃda}
     \rdg[wit={B}]{phalaṃ jaṃda}
     \rdg[wit={N1}]{phalacaṃdra}
     \rdg[wit={N2}]{phalaṃ/ caṃdra}}
\app{\lem[wit={ceteri}]{na}
     \rdg[wit={N2}]{\om}}
kāmi\skp{ny-ā}\app{\lem[wit={ceteri}, alt={āder}]{\skm{ny-ā}de\skp{r-ya}}
     \rdg[wit={D}]{āde}
     \rdg[wit={B}]{ādar}
     \rdg[wit={L}]{ādir}
}\app{\lem[wit={E},alt={yasyecchā}]{\skm{r-ya}syecchā}
     \rdg[wit={P}]{yasyochā}
     \rdg[wit={L}]{yasya}
     \rdg[wit={B}]{yasye}
     \rdg[wit={N1,N2}]{yasya yasyeccha}
     \rdg[wit={D}]{yasya yasyechā}
     \rdg[wit={U1}]{yasya yaṃ}
     \rdg[wit={U2}]{yasye chā}}
\app{\lem[wit={E,P,L,B}]{na}
     \rdg[wit={ceteri}]{\om}}
\app{\lem[wit={ceteri}]{bhavati}
  \rdg[wit={B}]{bhavatī}}/
%------------------------------
% \om                      \E
% \om                      \P
% \om                      \L
% \om                      \B
%taṃ taṃ bhogaṃ prāpnoti   \D
%taṃ taṃ bhogaṃ prāpnoti   \N1
%taṃ taṃ bhogaṃ prāpnoti// \N2
%tataṃ bhogaṃ prāpnoti     \U1
% \om                      \U2
%------------------------------
%He attains widespread enjoyment. 
%------------------------------
\app{\lem[wit={D,N1,N2}]{taṃ taṃ}
  \rdg[wit={U1}]{tataṃ}
\rdg[wit={ceteri}]{\om}}
\app{\lem[wit={D,N1,N2,U1}]{bhogaṃ prāpnoti}
  \rdg[wit={ceteri}]{\om}}/ 
%------------------------------
% \om                      \P
% \om                      \L
% \om                      \B
%atha vā yasya    mana    eva   sthāne 'nurāgaṃ        na prāpnoti// \D
%atha vāsya/vātya mana   eva   sthāne 'nurāgaṃ        na prāpnoti/ \N1
%atha vā syamana         eva   sthāne 'nurāgaṃ        na prāpnoti/ \N2
%atha vā svāmana         etata sthāne  nurāgaṃ         na prāpnoti/ \U1
% \om                      \U2
%------------------------------
%However, his mind does not suffer attachment in this very state.  
%------------------------------
\app{\lem[wit={D,N1,N2,U1}]{atha}
  \rdg[wit={ceteri}]{\om}} 
\app{\lem[wit={D}]{vā yasya}
  \rdg[wit={N1}]{vāsya}
  \rdg[wit={N2}]{vā syamana}
  \rdg[wit={U1}]{vā svāmana}
  \rdg[wit={ceteri}]{\om}}
\app{\lem[wit={D,N1,N2,U1}]{mana}
  \rdg[wit={ceteri}]{\om}}
\app{\lem[wit={D,N1,N2,U1}]{eva}
  \rdg[wit={U1}]{etata}
  \rdg[wit={ceteri}]{\om}}
\app{\lem[wit={D,N1,N2,U1}]{sthāne}
  \rdg[wit={ceteri}]{\om}}
\app{\lem[wit={D,N1,N2}]{'nurāgaṃ}
  \rdg[wit={U1}]{nurāgaṃ}
  \rdg[wit={ceteri}]{om}}
\app{\lem[wit={D,N1,N2,U1}]{na prāpnoti}
  \rdg[wit={ceteri}]{\om}}/ 
\end{prose}
\end{ekdosis}
%%%%%%%%%%%%%%%%
%%%%%%%%%%%%%%%
%%%%%%%%%%%%%%%
%%%%%%%%%%%%%%%%
%%%%%%%%%%%%%%%
  \begin{ekdosis}
 \ekddiv{type=ed}
   \bigskip
    \centerline{\textrm{\small{[Other Attributes]}}}
    \bigskip
    \begin{prose}
%------------------------------
%anyad  rājayogasya cihnaṃ kathyate   \E
% \om                                 \P
%anyata rājayogasya cinhaṃ kathyate// \L
%anyata rājayogasya cinhaṃ kathyate// \B
%anyat  rājayogasya cinhaṃ kathyate// \N1 yasyecchā bhavati??? taṃ taṃ bhogaṃ prāpnoti/ atha vāsya mana eva sthāne 'nu rāgaṃ na prāpnoti/ anyat rājayogasya cinhaṃ kathate//
%anyat  rājayogasya cihuṃ  kathyate// \D
%anyad  rājayogasya ciṃhuṃ kathyate// \N2
%anyat  rājayogacinhaṃ     kathyate/  \U1
%anyat  rājayogasya cinhaṃ kathyate// \U2
%------------------------------
%[Now] another attribute of Rājayoga is described. 
%------------------------------
\app{\lem[wit={E,N2},alt={anyad}]{anya\skm{d-rā}}
  \rdg[wit={N1,D,U1,U2}]{anyat}
  \rdg[wit={L,B}]{anyate}
  \rdg[wit={P}]{\om}
}\app{\lem[wit={ceteri},alt={rājayogasya}]{\skp{d-rā}jayogasya}
  \rdg[wit={U1}]{rājayoga°}
  \rdg[wit={P}]{\om}}
\app{\lem[wit={E}]{cihnaṃ}
  \rdg[wit={L,B,N1,U2}]{cinhaṃ} %????
  \rdg[wit={N2}]{ciṃhuṃ}
  \rdg[wit={D}]{cihuṃ}
  \rdg[wit={P}]{\om}}
\app{\lem[wit={ceteri}]{kathyate}
  \rdg[wit={P}]{\om}}/
%------------------------------
%yasya rājyādilābhe 'pi    phalalābho na bhavati/ \E
% \om                                            \P
%yasya rājādilābhe   ty     aphalalābho       na bhavatī \L
%yasya rājādilābhe   ty     aphalalābho       na bhavatī \B
%yasya rājyādilābhe  pi     phalalābho       ba bhavati/ \N1
%yasya rājyādilābhe  pi     phalalābho       na bhavati// \D
%yasya rājyādilobhe  pi ca  phalalābho       na bhavati// \N2
%yasya rājyādilābe  'pi ca  palalābho        na bhavati/ \U1
%yasya rājyādilābho                          na bhavati/ \U2
%------------------------------
%Even ``of one who is in gain of a kingdom etc.'' [it is said that] perception of success does'nt arise.
%------------------------------
\app{\lem[wit={ceteri}]{yasya}
  \rdg[wit={P}]{\om}} 
\app{\lem[wit={E,N1,D}]{rājyādilābhe}
  \rdg[wit={L,B}]{rājā°}
  \rdg[wit={N2}]{°lobhe}
  \rdg[wit={U1}]{°lābe}
  \rdg[wit={U2}]{°lābho}
  \rdg[wit={P}]{\om}}
\app{\lem[wit={E,N1,D}]{'pi}
  \rdg[wit={N2,U1}]{'pi ca}
  \rdg[wit={L,B}]{ty}
  \rdg[wit={P,U2}]{\om}}
\app{\lem[wit={E,N1,D,N2}]{phalalābho}
  \rdg[wit={U1}]{pala°}
  \rdg[wit={L,B}]{aphala°}
  \rdg[wit={P,U2}]{\om}}
\app{\lem[wit={E,D,N2,U1,U2}]{na bhavati}
  \rdg[wit={L,B}]{na bhavatī}
  \rdg[wit={N1}]{ba bhavati}
  \rdg[wit={P}]{\om}}/
%------------------------------
%hānāv api manomadhye duḥkhaṃ na bhavati/ \E
% \om                                      \P
%hananād pi mānomadhye duḥkahṃ na bhavatī/ \L
%hananād pi mānomadhye duḥkahṃ na bhavatī/ \B
%hānāv api manomadhye duḥkhaṃ na bhavati/ \N1 %emend to hānau loc. sg. of hāni -> abandonment
%hānāv api manomadhye duḥkhaṃ na bhavati// \D
%hānāv  pi manomadhye duḥkhaṃ na bhavati// \N2
%hānāv api manomadhye duḥkhaṃ na bhavati/  \U1
%hānād api manomadhye duḥkhaṃ na bhavati// \U2
%------------------------------
%Even due to loss suffering does'nt arise in the mind.  
%------------------------------
\app{\lem[wit={ceteri},alt={hānāv}]{hānā\skp{v-a}}
  \rdg[wit={U2}]{hānād}
  \rdg[wit={P,L}]{nahanād}
  \rdg[wit={P}]{\om}
}\app{\lem[wit={ceteri},alt={api}]{\skm{v-a}pi}
  \rdg[wit={L,B,N2}]{pi}
  \rdg[wit={P}]{\om}}
manomadhye duḥkhaṃ na
\app{\lem[wit={ceteri}]{bhavati}
  \rdg[wit={L,B}]{bhavatī}}/
%------------------------------
%atha ca tṛṣṇā na bhavati/ \E
% \om                      \P
%atha ca tṛṣṇā na bhavati/ \L
%atha ca tṛṣṇā na bhavatī/ \B
%atha ca tṛṣṇā na bhavati/ \N1
%atha ca tṛṣṇā na bhavati  \D
%atha ca tṛṣṇā na bhavati/ \N2
%atha ca tṛṣṇā na bhavati/ \U1
%atha ca tṛṣṇā na bhavati/ \U2
%------------------------------
%And then desire doesn't arise. 
%------------------------------
\app{\lem[wit={ceteri}]{atha ca}
  \rdg[wit={P}]{\om}}
\app{\lem[wit={ceteri}]{tṛṣṇā}
  \rdg[wit={P}]{\om}}
\app{\lem[wit={ceteri}]{na}
  \rdg[wit={P}]{\om}}
\app{\lem[wit={ceteri}]{bhavati}
  \rdg[wit={B}]{bhavatī}
  \rdg[wit={P}]{\om}}/
%------------------------------
%atha ca kasmin                                  padārthasyopary   anicchā na bhavati/ \E
% \om                                                                                       \P
%atha ca kasmin na    padārtho   prāpte kasyāpi  padārthasyopari   ānīcha  na  bhavati//    \L
%atha ca kasmin na    padārthau  prāpte kasyāpi  padārthāsyopari   ānīchā  ni  bhavati//    \B
%atha ca kasminn pi   padārthe   prāpta kasyāpi  padārthasya upari anusthā na  bhavaṃti//   \N1 
%atha ca kasminn api  padārthe   prāpte kasyāpi  padārthasya upari anichā      bhavaṃti     \D
%atha ca kasminn pi   padārthe   prāpte kasyāpi  padārthasya upari anisthā na  bhavati//    \N2
%atha ca kasminn api  padārthe   prātpe kasyāpi  padārthasya upari aniṣṭā  na  bhavati      \U1
%atha ca kasmin   adhipadārtha   prāpte kābhyādi padārthopari      anicha  na  bhavati//    \U2 %%%407.jpg
%------------------------------
%And then with regards to some object that has been obtained for whatever reason towards ones object aversion does'nt arise.   
%------------------------------
\app{\lem[wit={ceteri}]{atha ca}
  \rdg[wit={P}]{\om}}
kasmi\skp{n-na}\skm{-a}\app{\lem[wit={D,U1},alt={api}]{\skm{n-na}pi}
  \rdg[wit={L,B}]{na}
  \rdg[wit={N1,N2}]{pi}
  \rdg[wit={U2}]{adhi}
  \rdg[wit={E,P}]{\om}}
\app{\lem[wit={ceteri}]{padārthe}
  \rdg[wit={L}]{padārtho}
  \rdg[wit={B}]{padārthau}
  \rdg[wit={U2}]{padārtha°}
  \rdg[wit={E,P}]{\om}}
\app{\lem[wit={ceteri}]{prāpte}
  \rdg[wit={N1}]{prāpta}
  \rdg[wit={E,P}]{\om}}
\app{\lem[wit={ceteri}]{kasyāpi}
  \rdg[wit={U2}]{kābhyādi}
  \rdg[wit={E,P}]{\om}}
\app{\lem[wit={E},alt={padārthasyopary}]{padārthasyopa\skp{ry-a}}
  \rdg[wit={L,B}]{padārthasyopari}
  \rdg[wit={U2}]{padārthopari}
  \rdg[wit={ceteri}]{padārthasya upari}
  \rdg[wit={P}]{\om}}\app{\lem[wit={E},alt={anicchā}]{\skm{r-ya}nicchā}
  \rdg[wit={L}]{ānīcha}
  \rdg[wit={B}]{ānīchā}
  \rdg[wit={N1}]{anusthā}
  \rdg[wit={D}]{anichā}
  \rdg[wit={N2}]{anisthā}
  \rdg[wit={U1}]{aniṣṭā}
  \rdg[wit={U2}]{anicha}}
\app{\lem[wit={ceteri}]{na}
  \rdg[wit={B}]{ni}
  \rdg[wit={P,D}]{\om}}
\app{\lem[wit={ceteri}]{bhavati}
  \rdg[wit={N1,D}]{bhavaṃti}
  \rdg[wit={P}]{\om}}/
%------------------------------
%kasmin    padārthe manaso   nurāgo na bhavati/    \E
%asminnapi padārthe manaso   nurāgo na bhavati... ayam api padārthe manasonurāgo na bhavati... \P
%asminn    padārthe manaso   nurāgo na bhavatī/    \L
%asminn    padārthe manaso   nurāgo na bhavatī/    \B
%asminnapi padārthe manasaḥ anurāgo    bhavati/    \N1
%asminnapi padārthe manasaḥ anurāgo    bhavati//   \D
%asminnapi padārthe manasaḥ anurāgo    bhavati/    \N2
%asminnapi padārthe manasa  anurāgo    bhavati     \U1 
%kasminnpi padārthe         anurāgo na bhavati// ayam api padārthe anurāgo na bhavati//  \U2
%------------------------------
%With regard to this object also affection of the mind does'nt arise. 
%------------------------------
\app{\lem[wit={ceteri},alt={asminn}]{asmi\skp{n-a}}
  \rdg[wit={E,U2}]{kasmin}
}\app{\lem[wit={ceteri},alt={api}]{\skm{n-a}pi}
  \rdg[wit={E,L,B}]{\om}} 
padārthe
\app{\lem[wit={E,P,L,B}]{manaso}
  \rdg[wit={N1,D,N2,U1}]{manasaḥ}
  \rdg[wit={U1}]{manasa}
  \rdg[wit={U2}]{\om}}
\app{\lem[wit={E,P,L,B}]{'nurāgo}
  \rdg[wit={ceteri}]{anurāgo}}
\app{\lem[wit={E,P,U2}]{na bhavati}
  \rdg[wit={L,B}]{na bhavatī}
  \rdg[wit={ceteri}]{bhavati}}/\note[type=philcomm, labelb=123, lem={na bhavati}]{P and U2 add \textit{ayam api padārthe anurāgo na bhavati ||} after this sentence, which is clearly a corruption.}
%------------------------------
%ayam  api rājayogaḥ kathyate/  \E
%atham api rājayogaḥ kathyate   \P
%atha  samarājayogaḥ kathyate/  \L
%ayam  api rājayogaḥ kathyate/  \B
%ayam  api rājayogaḥ kathyate/  \N1
%ayam  api rājayogaḥ kathyate// \D
%ayam  api rājayoga  kathyate// \N2
%ayam  api rājayogaḥ kathyate/  \U1
%ayam  api rājayoga  kathyate// \U2
%------------------------------
%Just this is said to be Rājayoga. 
%------------------------------
\app{\lem[wit={ceteri},alt={ayam}]{aya\skp{m-a}}
  \rdg[wit={P}]{atham}
  \rdg[wit={L}]{atha}
}\app{\lem[wit={ceteri},alt={api}]{\skm{m-a}pi}
  \rdg[wit={L}]{sama}}
\app{\lem[wit={ceteri}]{rājayogaḥ}
  \rdg[wit={N2,U2}]{rājayoga}}
kathyate/
%------------------------------ %%%%split in stemma?! maitre mitre!!!
%atha caḥ yasya manaḥ   munividvat  puruṣeṣu maitre        ca samaṃ bhavati/ \E
%atha ca  yasya manaḥ   śunividvat  puruṣe   maitre śatrau ca samaṃ bhavati \P
%atha ca  yasya manaḥ   bhunividvat puruṣe   maitre śatrau ca samaṃ bhavati/ \L
%atha ca  yasya manaḥ   śrunividvat puruṣe   maitre śatro  ca samaṃ bhavatī/ \B
%atha ca  yasya manaḥ/  śrutividyut puruṣe   mitre  śatrau ca samaṃ bhavati/ \N1
%atha ca  yamanaḥ       śrutividyut puruṣe   mitre  śatrau ca samaṃ bhavati// \D
%atha ca  yasya manaḥ   śrutividyut puruṣe   mitre  śatrau ca samaṃ bhavati/ \N2
%atha ca  yasya mana    śrunividvat puruṣe   mitre  śatrau ca samaṃ bhavati \U1
%atha ca  yasya manaḥ   śuciviśuddhapuruṣe   mitre  śatrau ca samaṃ bhavati// \U2
%------------------------------
%And then his mind which knows the sacred speech is equal towards a person, friend and enemy.  
%------------------------------
\note[type=source, labelb=124, lem={mitre  śatrau}]{Ysv (PT): vidyāvidyāmitraśatrau samā dṛṣṭiś ca sarvaśaḥ | bhogāsaktādikarttṛtvena mano no bhavet khavat |}
atha
\app{\lem[wit={ceteri}]{ca}
  \rdg[wit={E}]{caḥ}}
\app{\lem[wit={ceteri}]{yasya}
  \rdg[wit={D}]{ya}}
manaḥ
\app{\lem[resp=egoscr, type=emendation]{śrutividvat}
  \rdg[wit={E}]{munividvat}
  \rdg[wit={P}]{śunividvat}
  \rdg[wit={L}]{bhunividvat}
  \rdg[wit={B,U1}]{śrunividvat}
  \rdg[wit={N1,N2,D}]{śrutividyut}
  \rdg[wit={U2}]{śuciviśuddha°}
}\app{\lem[wit={ceteri}]{puruṣe}
  \rdg[wit={E}]{puruṣeṣu}}
\app{\lem[wit={ceteri}]{mitre}
  \rdg[wit={E,P,L,B}]{maitre}}
 \app{\lem[wit={ceteri}]{śatrau}
   \rdg[wit={B}]{śatro}
   \rdg[wit={E}]{\om}}
 ca samaṃ bhavati/
   \end{prose}
\end{ekdosis}
\ekdpb*{}
%%%%%%%%%%%%%%%%%%%%%%%%%%%%%%%%%%%%%%%%%%
%%%%%%%%%%%%%%%%%%%%%%%%%%%%%%%%%%%%%%%%%%
%%%%%%%%PAGEBREAK%%%%%%%PAGEBREAK%%%%%%%%%
%%%%%%%%%%%%%%%%%%%%%%%%%%%%%%%%%%%%%%%%%%
%%%%%%%%%%%%%%%%PAGEBREAK%%%%%%%%%%%%%%%%%
%%%%%%%%%%%%%%%%%%%%%%%%%%%%%%%%%%%%%%%%%%
%%%%%%%%PAGEBREAK%%%%%%%PAGEBREAK%%%%%%%%%
%%%%%%%%%%%%%%%%%%%%%%%%%%%%%%%%%%%%%%%%%%
%%%%%%%%%%%%%%%%%%%%%%%%%%%%%%%%%%%%%%%%%%
%%%%%%%%%%%%%%%%%%%%%%%%%%%%%%%%%%%%%%%%%%
%%%%%%%%%%%%%%%%%%%%%%%%%%%%%%%%%%%%%%%%%%
%%%%%%%%PAGEBREAK%%%%%%%PAGEBREAK%%%%%%%%%
%%%%%%%%%%%%%%%%%%%%%%%%%%%%%%%%%%%%%%%%%%
%%%%%%%%%%%%%%%%PAGEBREAK%%%%%%%%%%%%%%%%%
%%%%%%%%%%%%%%%%%%%%%%%%%%%%%%%%%%%%%%%%%%
%%%%%%%%PAGEBREAK%%%%%%%PAGEBREAK%%%%%%%%%
%%%%%%%%%%%%%%%%%%%%%%%%%%%%%%%%%%%%%%%%%%
%%%%%%%%%%%%%%%%%%%%%%%%%%%%%%%%%%%%%%%%%%
%%%%%%%%%%%%%%%%%%%%%%%%%%%%%%%%%%%%%%%%%%
%%%%%%%%%%%%%%%%%%%%%%%%%%%%%%%%%%%%%%%%%%
%%%%%%%%PAGEBREAK%%%%%%%PAGEBREAK%%%%%%%%%
%%%%%%%%%%%%%%%%%%%%%%%%%%%%%%%%%%%%%%%%%%
%%%%%%%%%%%%%%%%PAGEBREAK%%%%%%%%%%%%%%%%%
%%%%%%%%%%%%%%%%%%%%%%%%%%%%%%%%%%%%%%%%%%
%%%%%%%%PAGEBREAK%%%%%%%PAGEBREAK%%%%%%%%%
%%%%%%%%%%%%%%%%%%%%%%%%%%%%%%%%%%%%%%%%%%
%%%%%%%%%%%%%%%%%%%%%%%%%%%%%%%%%%%%%%%%%%
\begin{ekdosis}
  \begin{prose}
    \noindent
%------------------------------
%dṛṣṭiś ca samā bhavati/   \E
%dṛṣṭiś ca namnā bhavati   \P
% \om                      \L
% \om                      \B
%dṛṣṭiś ca samā bhavati//  \N1
%dṛṣṭiś ca samā bhavati//  \D
%dṛṣṭiś ca samā bhavati//  \N2
%dṛṣṭiś ca samā bhavati/   \U1
%dṛṣṭiś ca samā bhavati/   \U2
%------------------------------
%And a neutral view arises. 
%------------------------------
\app{\lem[wit={ceteri},alt={dṛṣṭiś}]{dṛṣṭi\skm{ś-ca}}
  \rdg[wit={L,B}]{\om}
}\app{\lem[wit={ceteri}, alt={ca}]{\skp{ś-ca}}
  \rdg[wit={L,B}]{\om}}
\app{\lem[wit={ceteri}]{samā}
  \rdg[wit={P}]{namnā}
  \rdg[wit={L,B}]{\om}}
\app{\lem[wit={ceteri}]{bhavati}
  \rdg[wit={L,B}]{\om}}/
%------------------------------
%sakalapṛthvīmadhye gamanavataḥ       sukhabhogavataḥ      yasya manasi karttṛtvābhimāno   nāsti/ \E
%sakalapṛthvīmadhye gamanāgamanavataḥ sukhabhogavataḥ      yasya manasi kartṛtvābhimāno    nāsti/ \P
%sakalapṛtvīmadhye  gamanāgamanataḥ   sukhabogho bhavataḥ  yasya manasi kartu tvābhimano   nāsti/ \L
%sakalapṛthvīmadhye gamanāgamanataḥ   sukhabogho bhavataḥ  yasya manasi kartutvābhimano    nāsti// \B
%sakalapṛthvīmadhye gamanavataḥ//     sukhabhogavataḥ/     yasya manasi kartṛtvādyabhimāno nāsti/  \N1
%sakalapṛthvīmadhye gamanaṃvataḥ//    sukhabhogavataḥ      yasya manasi kartṛtvādyabhimāno nāsti// \D
%sakalapṛthvīmadhye gamavataḥ         sukhabhogavataḥ      yasya manasi kartṛtvādyabhimāno nāsti// \N2
%sakalapṛthvīmadhye gamanavataḥ       sukho bhogavataḥ     yasya manasi kartṛtvābhimāno    nāsti   \U1
%sakalapṛthvīmadhye gamanāgamanavat// sukhabhogavat        yasya manasi kartṛtvābhimāno    nāsti// \U2
%------------------------------
%In the mind of one who is situated in the centre of the entire earth, the pride of authorship does't arise, because of death and rebirth, and because of happiness and enjoyment.  %%%check translation think about the Sanskrit 
%------------------------------
\app{\lem[wit={ceteri}]{sakalapṛthvīmadhye}
  \rdg[wit={L}]{°pṛtvī°}}
\app{\lem[wit={P}]{gamanāgamanavataḥ}
  \rdg[wit={U2}]{gamanāgamanavat}
  \rdg[wit={L,B}]{gamanāgamanataḥ}
  \rdg[wit={E,N1,U1}]{gamanavataḥ}
  \rdg[wit={D}]{gamanaṃvataḥ}
  \rdg[wit={U1}]{gamavataḥ}}
\app{\lem[wit={ceteri}]{sukhabhogavataḥ}
  \rdg[wit={L,B}]{sukhabogho bhavataḥ}
  \rdg[wit={U1}]{sukho bhogavataḥ}
  \rdg[wit={U2}]{sukhabhogavat}}
yasya manasi
\app{\lem[wit={E,P,U1,U2}]{kartṛtvābhimāno}
  \rdg[wit={B}]{kartutvābhimano}
  \rdg[wit={L}]{kartu tvābhimano}
  \rdg[wit={N1,N2,D}]{kartṛtvādyabhimāno}}
nāsti/
%------------------------------
%atha ca lokamadhye gamanavataḥ sukhabhogavataḥ yasya manasi karttṛtvābhimāno nāsti/....
%atha ca lokamadhye kartṛtvaṃ na jñāpayati/ \E
%anucalokamadhye    kartṛtvaṃ na jñāpayati/ \P
%anucaralokamadhya  kartṛtvābhimano nāsti \L
%anucaralokamadhya--kartṛtvābhimano nāsti// \B
%anucalokamadhye    kartṛtvaṃ    jñāpayati// \N1
%anucalokamadhye    kartṛtvaṃ na jñātvā payati/ \D
%anucalokamadhye    kartṛtvaṃ na jñāpayati/ \N2
%anucalokamadhye    kartṛtvaṃ    jñātva payati \U1
%anucalokamadhye    kartṛtvaṃ na jñāpayati \U2
%------------------------------
%while wandering the world he doesn't whish to know authorship. 
%------------------------------
\note[type=source, labelb=125, lem={lokamadhye°}]{Ysv (PT): lokamadhye bhavet karttā manomadhye 'pi niṣkriyaḥ |}
\app{\lem[wit={L,B}]{anucara}
  \rdg[wit={N1,N2,D,U1,U2,P}]{anuca°}
  \rdg[wit={L,B}]{anucara°}
  \rdg[wit={E}]{atha ca}
}\app{\lem[wit={ceteri}]{lokamadhye}
  \rdg[wit={L,B}]{°madhya}}
\app{\lem[wit={E,P,D,N2,U2}]{kartṛtvaṃ na}
  \rdg[wit={L,B}]{kartṛtvābhimano}
  \rdg[wit={N1,U1}]{kartṛtvaṃ}}
\app{\lem[wit={E,P,N1,N2,U2}]{jñāpayati}
  \rdg[wit={D,U1}]{jñātva payati}
  \rdg[wit={L,B}]{nāsti}}/
%------------------------------
%so  pi  rājayogaḥ kathyate// \E
%so  pi  rājayogaḥ kathyate   \P
%so  pi  rājayoga  kathyate/   \L
%so  pi  rājayoga  kathyate/   \B
%so  pi  rājayogaḥ kathyate//  \N1
%so  pi  rājayoga  kathyate//  \D
%so 'pi  rājayoga  kathyate// \N2
%so  pi  rājayoga  kathyate/   \U1
%so  pi  rājayoga  kathyate    \U2
%------------------------------
%This is also said to be Rājayoga. 
%------------------------------
\note[type=source, labelb=126, lem={so 'pi \ldots}]{eṣo 'pi rājayogīti sukhe duḥkhe samas tathā |}
so 'pi
\app{\lem[wit={E,P,N1}]{rājayogaḥ}
  \rdg[wit={ceteri}]{rājayoga}}
kathyate/
%------------------------------
%navīnāni         paṭṭasūtramaya     dhṛtāni vastrāṇi   \E
%navīnāni         paṭasūtramayāni    dhṛtāni vastrāṇi   \P
%navinīnīśpī      paṭṭasūtramayāni   dhṛtāni vastrāṇi// \L
%navinīnīr api    paṭṭasūtramayāni   dhṛtāni vastrāṇi// \B
%navīnāni         paṭasūtramayāni    dhṛtāni vastrāṇi/  \N1
%navīnāni         paṭasūtramayāni    dhṛtāni vastrāṇi// \D
%navīnāni         paṭasūtramayāni    dhṛtāni vastrāṇi/  \N2
%navīnāni         padasūtramayāni       tāni vastrāṇi   \U1
%navīnāni      paṭ(h)asūtramayāni    dhṛtāni            \U2
%------------------------------
%New durable clothes made of silk,  
%------------------------------
\app{\lem[wit={ceteri}]{navīnāni}
  \rdg[wit={L}]{navīnīnīś pī}
  \rdg[wit={B}]{navinīnīr api}}
\app{\lem[wit={E,L,B}, alt={paṭṭa°}]{paṭṭa}
  \rdg[wit={P,N1,D,N2,U2}]{paṭa°}
  \rdg[wit={U1}]{pada}
}sūtra\app{\lem[wit={ceteri},alt={°mayāni}]{mayāni}
  \rdg[wit={E}]{maya}}
\app{\lem[wit={ceteri}]{dhṛtāni}
  \rdg[wit={U1}]{tāni}}
\app{\lem[wit={ceteri}]{vastrāṇi}
  \rdg[wit={U2}]{\om}}
%------------------------------ %%%%KOLLOQUIUM: was hier tun? kastūrī/kasturikā = gleichwertig 
%atha vā jīrṇāni chidrāṇi    dhṛtāni    kastūrīcandanalepair   vā  kardamalepena   yasya manasi harṣaśokau  na staḥ/ \E
%atha vā jīrṇāni sachadrāṇi  dhūtāni    kastūrīcaṃdanalepo     vā  karddamalepo vā yasya manasi harṣaśokau na staḥ/ \P
%atha vā jīrṇāni svachidrāṇi dhṛtāni    kasturīcaṃdanalepo     cā  kardamalepo  vā yasya manasi harṣaśokau na sthaḥ// \L
%atha vā jīrṇāni svachidrāṇi dhṛtāni    kastūrīcaṃdanalepo     vā  kardamalepo  vā yasya manasi harṣaśokau na sthaḥ// \B
%atha vā jīrṇāni sacchidrāṇi dhṛtāni/   kasturikā caṃdanalepo vā/ kardamalepo  vā yasya manasi harṣaśoko  na sthaḥ  \N1
%atha vā jīrṇāni sacchidrāṇi dhṛtāni//  kasturikā caṃdanalepo vā/ kardamalepo  vā yasya manasi harṣaśoko  na sthaḥ  \D
%atha vā jīrṇāni sacchidrāṇi dhṛtāni // kasturikā caṃdanalepo vā/ kardamalepo  vā yasya manasi harṣaśoka  na sthāḥ \N2
%atha vā jīrṇāni sachidrāṇi  dhvatāni   kasturikā caṃdanalepo vā  kardamalepo  vā yasya manasi harṣaśokau na sthāḥ \U1 %%272.jg
%                                       kastūrīcaṃdanalepo     vā                  yasya manasi harṣaśoko  na sta// \U2
%------------------------------
%or however, old, worn (clothes) with holes smeared with sandalwood and musk, or smeared with mud. In whose mind joy and sorrow are not situated,
%------------------------------
atha vā jīrṇāni
\app{\lem[wit={N1,N2,D}]{sacchidrāṇi}
  \rdg[wit={U2}]{sachidrāṇi}
  \rdg[wit={P}]{sachadrāṇi}
  \rdg[wit={L,B}]{svachidrāṇi}
  \rdg[wit={E}]{chidrāṇi}}
\app{\lem[wit={ceteri}]{dhṛtāni}
  \rdg[wit={U2}]{dhvātāni}
  \rdg[wit={P}]{dhūtāni}}
\app{\lem[wit={E,P,B,U2}]{kastūrī}
  \rdg[wit={L}]{kasturī}
  \rdg[wit={N1,N2,D,U1}]{kasturikā}
}\app{\lem[wit={E},alt={candana°}]{candana}
  \rdg[wit={ceteri}]{caṃdana°}
}\app{\lem[wit={E},alt={lepair}]{lepai\skp{r-vā}}
  \rdg[wit={ceteri}]{lepo}} 
\app{\lem[wit={ceteri},alt={vā}]{\skm{r-vā}}
  \rdg[wit={L}]{cā}}
\app{\lem[wit={E}]{kardamalepena}
  \rdg[wit={ceteri}]{kardamalepo}}
\app{\lem[wit={ceteri}]{vā}
  \rdg[wit={E}]{\om}}
yasya manasi
harṣa\app{\lem[wit={ceteri},alt={°śokau}]{śokau}
  \rdg[wit={N1,D,U2}]{°śoko}
  \rdg[wit={N2}]{°śoka}}
na
\app{\lem[type=emendation, resp=egoscr]{sthau}
  \rdg[wit={ceteri}]{\korr sthaḥ}
  \rdg[wit={N2,U1}]{sthā}
  \rdg[wit={U2}]{sta}}
%------------------------------
%sa evātra tiṣṭhati/         \E
%sa eva rājayogaḥ            \P
%sa eva rājayogaḥ// idānīṃ// \L
%sa eva rājayogaḥ// idānīṃ// \B
%sa eva rājayogaḥ//          \N1
%sa eva rājayogaḥ//          \D
%sa eva rājayogaḥ//          \N2
%sa eva rājayogaḥ            \U1
%sa eva rājayoga             \U2
%------------------------------
%just he is in the state of Rājayoga. 
%------------------------------
%yasya janmamaraṇe na staḥ sukhaṃ na bhavati/ kulaṃ na bhavati śīlaṃ na bhavati/ sthānaṃ na bhavati/ \E
%\om \P
%\om \L
%\om \B
%\om \N1
%\om \D
%\om \N2
%\om \U1
%\om \U2
%------------------------------
%One who is not situated in birth and death has no happiness, has no family, and cold does not arise, place does not arise.?!?!!?
%----------------------------
\app{\lem[wit={ceteri}]{sa eva}
  \rdg[wit={E}]{sa evātra}}
\app{\lem[wit={ceteri}]{rājayogaḥ}
  \rdg[wit={U2}]{rājayoga}
  \rdg[wit={L,B}]{rājayogaḥ || idānīṃ ||}
  \rdg[wit={E}]{tiṣṭhati}}/\note[type=philcomm, labelb=127, lem={°tiṣṭhati}]{E adds \textit{yasya janmamaraṇe na staḥ sukhaṃ na bhavati | kulaṃ na bhavati śīlaṃ na bhavati | sthānaṃ na bhavati |} here, which seems to be a dittography of previous sentences.}
%----------------------------
%rājayogaḥ naramadhye      atha ca vanamadhye             yuddhe saṃgrāmamadhye                        vā yasya manaḥ        bhayapūrṇaṃ vā  na bhavati/  so pi rājayogaḥ kathyate// \E
%          nagaramadhye    'tha ca vanamadhye                  utasaṃgrāmamadhye                       vā yasya mana      ūnaṃ    pūrṇaṃ vāṃ na bhavati   so pi rājayogaḥ            \P
%          nagaramadhye     tha ca vanamadhye                 udvastagrāmamadhye                       vā yasya manaḥ     unaṃ    pūrṇaṃ vā  na bhavati   so pi rājayogaḥ//          \L
%          nagaramadhye  (')tha ca vanamadhye                udvastagrāmaṃmadhye                       vā yasya manaḥ     unaṃ    pūrṇaṃ vā  na bhavatī   so pi rājayogaḥ//          \B
%          nagaramadhye    atha ca vanamadhye/                 udvesūgrāmamadhye .. ..pūrṇagrāmamadhye vā yasya manaḥ     ūnaṃ na pūrṇaṃ vā  na bhavati// so pi rājayogaḥ//          \N1
%          ṣagaramadhye    atha ca vanamadhye//                udvesūgrāmamadhye svetapūrṇagrāmamadhye vā yasya manaḥ     ūnan na pūrṇaṃ vā  na bhavati/  so pi rājayogaḥ//          \D
%          nagaramadhye    atha ca vanamadhye//                udvesūgrāmamadhye svetapūrṇagrāmamadhye vā yasya manaḥ     ūnan na pūrṇaṃ vā  na bhavati/  so pi rājayogaḥ//          \N2
%       vā nagaramadhye    atha ca vanamadhye                 udassaṃgrāmamadhye  lokapūrṇagrāmamadhye vā yasya manaḥ     unaṃ    pūrṇaṃ     na bhavati   so pi rājayogaḥ            \U1
%          nagaramadhye    'tha vā vanamadhye                  udvasagrāmamadhye                       vā yasya mana      ūnaṃ    pūrṇaṃ vāṃ na bhavati   so pi rājayogaḥ            \U2
%------------------------------
%Just he is in the state of Rājayoga for whom the mind is neither in abundance nor in lack, being located in a city, a forest, an uninhabited village or a village full of people. 
%----------------------------
\app{\lem[wit={ceteri}]{nagaramadhye}
  \rdg[wit={E}]{rājayogaḥ nagaramadhye}
  \rdg[wit={D}]{ṣagaramadhye}
  \rdg[wit={U1}]{vā nagaramadhye}}
\app{\lem[wit={P,L,B,U2}]{'tha ca}
  \rdg[wit={E,N1,N2,D,U1}]{atha ca}}
vanamadhye
\app{\lem[wit={U2},alt={udvasa°}]{udvasa}
  \rdg[wit={E}]{yuddhe saṃ°}
  \rdg[wit={P}]{utasaṃ°}
  \rdg[wit={L,B}]{udvasta°}
  \rdg[wit={N1,N2,D}]{udvesū°}
  \rdg[wit={U1}]{udassaṃ°}
}\app{\lem[wit={ceteri}]{grāmamadhye}
  \rdg[wit={B}]{grāmaṃ madhye}}
\app{\lem[wit={U1}]{lokapūrṇagrāmamadhye}
  \rdg[wit={N1}]{....pūrṇagrāmamadhye}
  \rdg[wit={D,N2}]{svetapūrṇagrāmamadhye}}
vā yasya
\app{\lem[wit={P,U2}]{mana}
  \rdg[wit={ceteri}]{manaḥ}}
\app{\lem[wit={P,N1,N2,U2}]{ūnaṃ}
  \rdg[wit={D,N2}]{ūnan}
  \rdg[wit={L,B,U1}]{unaṃ}
  \rdg[wit={E}]{bhaya°}}
\app{\lem[wit={N1,N2,D}]{na}
  \rdg[wit={ceteri}]{\om}}
pūrṇaṃ
\app{\lem[wit={ceteri}]{vā}
  \rdg[wit={P,U2}]{vāṃ}
  \rdg[wit={U1}]{\om}}
na bhavati/ so
\app{\lem[type=emendation, resp=egoscr]{'pi}
  \rdg[wit={ceteri}]{\korr pi}}
\app{\lem[wit={ceteri}]{rājayogaḥ}
  \rdg[wit={E}]{rājayogaḥ kathyate}}\dd{}
\end{prose}
\end{ekdosis}
%%%%%%%%%%%%%%
%%%%%%%%%%%%%%
%%%%%%%%%%%%%%
%%%%%%%%%%%%%
%%%%%%%%%%%%%%%
\begin{ekdosis}
  \ekddiv{type=ed}
  \bigskip
   \centerline{\textrm{\small{[Caryāyoga]}}}
      \bigskip
      \begin{prose}
%----------------------------
%idānīṃ      yogaḥ  kathyate/ \E
%idānīṃ caryāyogaḥ  kathyate   \P
%idānīṃ caryāyogaḥ  kathyate// \L
%idānīṃ caryāyogaḥ  kathyate// \B
%idānīṃ caryāyoga   kathyate// \N1
%idānīṃ caryāyogaḥ  kathyate// \D [S.7, Z.7]
%idānīṃ caryāyoga   kathyate// \N2
%idānīṃ tvaryāyogaḥ kathyate \U1
%idānīṃ caryāyoga   kathyate// \U2
%------------------------------
%Now \textit{caryāyogaḥ}, the Yoga of wandering is explained.
%----------------------------
idānīṃ
\app{\lem[wit={ceteri}]{caryāyogaḥ}
     \rdg[wit={U1}]{tvaryāyogaḥ}
     \rdg[wit={E}]{yogaḥ}} kathyate/
%----------------------------
%nirākāro         nityo 'bhedyaḥ    sa etādṛśaḥ ātmani                  mano   yasya  niścalaṃ tiṣṭhati/  \E
%nirākāro  'calo  nityo  bhedhyaḥ   sa etādṛa   ātmā    etādṛśo  ātmani mano   yasya  niścala  tiṣṭhati   \P %%%7639.jpg
%nirākāro  calo   nityo  bhedhyaḥ   sa etādṛa   ātmā sa etādṛśe  ātmani               niścala  tiṣṭhati/  \L     %daṇḍa nach ātmā besser -> emend? oder in weiteren Hss?
%nirākāro  calo   nityo  bhedhyaḥ   sa etādṛa   ātmā sa etādṛśye ātmani               niścalaṃ tiṣṭhati/  \B
%nirākālo  nityo   calo 'bhedhyaḥ/  sa etādṛśaḥ ātmā    etādṛśe  ātmani manaḥ  yasya  niścalaṃ tiṣṭhati   \N1
%nirākālo  nityo   calo 'bhedhyaḥ// sa etādṛśaḥ ātmā    etādṛśe  ātmani manaḥ  yasya  niścalaṃ tiṣṭhati   \D
%nirākālo  nityo   calo 'bhedhyaḥ   sa etādṛśaḥ ātmā    etādṛśa  ātmani manaḥ  yasya  niścala  tiṣṭhati/  \N2
%nirākāro  nityo   calo abhedhyaḥ   sa etādṛśaḥ ātmā    etādṛśo  ātmani mano   yasya  niścalaṃ bhavati    \U1
%nirvikāro  'calo nityo 'bhedhya    sa etādṛśā  ātmani                  mano   yasya  niścalaṃ tiṣṭhati// \U2
%------------------------------
%Shapeless, unchangeable, permanent [and] unsplitable. Such is the self. It is seen as such by the one whose mind abides in the self without moving. 
%------------------------------
   \note[type=source, labelb=128, lem={caryāyogaḥ}]{harṣaśokau na jātveṣāṃ nodvego lokasaṅgame | nityollāse nirākāre nirāsane nirātmani | manasā niścalo bhūtvā sadā tiṣṭhet samo 'pi ca |}
   \note[type=philcomm, labelb=129, lem={caryāyogaḥ}]{Notwithstanding that \textit{cāryayoga} ist not mentioned in Ysv, Rāmacandra decides to utilizes this passage to construe another type of yoga from his list.}
\app{\lem[wit={E,P,L,B,U1}]{nirākāro}
  \rdg[wit={N1,N2,D}]{nirākālo}
  \rdg[wit={U2}]{nirvikāro}}
\app{\lem[wit={P,U2}]{'calo}
  \rdg[wit={L,B}]{calo}
  \rdg[wit={N1,N2,D,U1}]{nityo}
  \rdg[wit={E}]{\om}}
\app{\lem[wit={E,P,L,B,U2}]{nityo}
  \rdg[wit={ceteri}]{calo}}
\app{\lem[wit={E,N1,N2,D}]{'bhedyaḥ}
  \rdg[wit={P,L,B}]{bhedhyaḥ}
  \rdg[wit={U1}]{abhedhyaḥ}
  \rdg[wit={U2}]{'bhedyha}}
   sa
\app{\lem[wit={P,L,B}]{etādṛśa}
  \rdg[wit={E,N1,N2,D,U1}]{etādṛśaḥ}
  \rdg[wit={U2}]{etādṛśā}}
\app{\lem[wit={ceteri}]{ātmā}
  \rdg[wit={E,U2}]{ātmani}}
\app{\lem[wit={L,B}]{sa}
  \rdg[wit={ceteri}]{\om}}
\app{\lem[wit={N2}]{etādṛśa}
  \rdg[wit={P,U1}]{etādṛśo}
  \rdg[wit={L,N1,D}]{etādṛśe}
  \rdg[wit={B}]{etādṛśye}
  \rdg[wit={E,U2}]{\om}}
\app{\lem[wit={ceteri}]{ātmani}
  \rdg[wit={E,U2}]{\om}}
\app{\lem[wit={E,P,U1,U2}]{mano}
  \rdg[wit={N1,N2,D}]{manaḥ}
  \rdg[wit={L,B}]{\om}}
\app{\lem[wit={ceteri}]{yasya}
  \rdg[wit={L,B}]{\om}}
\app{\lem[wit={ceteri}]{niścalaṃ}
  \rdg[wit={P,L,N2}]{niścala}}
\app{\lem[wit={ceteri}]{tiṣṭhati}
  \rdg[wit={U1}]{bhavati}}/
%------------------------------
%tasyātmanaḥ puṇyapāpasparśo na bhavati/ \E
%tasyātmanaḥ puṇyapāpasparśo na bhavati  \P
%tasyātmanaḥ puṇyapāpasparśo na bhavati/ \L
%tasyātmanaḥ puṇyapāpasparśo na bhavatī/ \B
%tasyātmanaḥ punyapāpasparśo na bhavati/  \N1
%tasyātmanaḥ punyapāpasparśo na bhavati// \D
%tasyātmanaḥ puṇyapāpasparśo na bhavati/ \N2
%tasya ātmanaḥ puṇyapāsya sparśo na bhavati  \U1
%tasya ātmanaḥ puṇyapāsya sparśo na bhavati//  \U2
%------------------------------
%His self is not touched by sin and merit. 
%------------------------------
\app{\lem[wit={ceteri}]{tasyātmanaḥ}
  \rdg[wit={U1,U2}]{tasya ātmanaḥ}}
\app{\lem[wit={ceteri}]{puṇyapāpasparśo}
  \rdg[wit={U1,U2}]{puṇyapāsya sparśo}}
na bhavati/
%------------------------------
%udakamadhye sthitasya padmapatre       yathodakasya sparśo    bhavati/  tathaivātmani   \E
%udakamadhye sthitasya padmanī patrasya yathodakasya sparśo na bhavati   tathaivātmani   \P
%udakamadhye sthitasya padmanī patrasya yathodakasya sparśo na bhavati/  tathaivātmani   \L
%udakamadhye sthitasya padmanī patrasya yathodakasya sparśā na bhavatī/  tathaivātmani   \B
%udakamadhye sthitasya padminī patrasya yathā/ udakasparśo  na bhavati/  tathaivātmani   \N1
%udakamadhye sthitasya padminī patrasya yathā  udakasparśo  na bhavati// tathaivātmani   \D
%udakamadhye sthitasya padminī patrasya yathā  udakasparśo  na bhavati/  tathaivātmani   \N2
%udakamadhye sthitasya padminī patrasya yathā  udakasparśo  na bhavati   tathaivātmani   \U1
%udakamadhye sthitasya padminī patrasya yathodakasparśo     na bhavati// tathaivātmani   \U2
%------------------------------
%Just as the leave of the lotus situated in the amidst water doesn't touch the water; likewise the self [is not touched by sin and merit].
%------------------------------
udakamadhye sthitasya
\app{\lem[wit={ceteri}]{padminī patrasya}
  \rdg[wit={P,L,B}]{padmanī patrasya}
  \rdg[wit={E}]{padmapatre}}
\end{prose}
\end{ekdosis}
\ekdpb*{}
%%%%%%%%%%%%%%%%%%%%%%%%%%%%%%%%%%%%%%%%%%
%%%%%%%%%%%%%%%%%%%%%%%%%%%%%%%%%%%%%%%%%%
%%%%%%%%PAGEBREAK%%%%%%%PAGEBREAK%%%%%%%%%
%%%%%%%%%%%%%%%%%%%%%%%%%%%%%%%%%%%%%%%%%%
%%%%%%%%%%%%%%%%PAGEBREAK%%%%%%%%%%%%%%%%%
%%%%%%%%%%%%%%%%%%%%%%%%%%%%%%%%%%%%%%%%%%
%%%%%%%%PAGEBREAK%%%%%%%PAGEBREAK%%%%%%%%%
%%%%%%%%%%%%%%%%%%%%%%%%%%%%%%%%%%%%%%%%%%
%%%%%%%%%%%%%%%%%%%%%%%%%%%%%%%%%%%%%%%%%%
%%%%%%%%%%%%%%%%%%%%%%%%%%%%%%%%%%%%%%%%%%
%%%%%%%%%%%%%%%%%%%%%%%%%%%%%%%%%%%%%%%%%%
%%%%%%%%PAGEBREAK%%%%%%%PAGEBREAK%%%%%%%%%
%%%%%%%%%%%%%%%%%%%%%%%%%%%%%%%%%%%%%%%%%%
%%%%%%%%%%%%%%%%PAGEBREAK%%%%%%%%%%%%%%%%%
%%%%%%%%%%%%%%%%%%%%%%%%%%%%%%%%%%%%%%%%%%
%%%%%%%%PAGEBREAK%%%%%%%PAGEBREAK%%%%%%%%%
%%%%%%%%%%%%%%%%%%%%%%%%%%%%%%%%%%%%%%%%%%
%%%%%%%%%%%%%%%%%%%%%%%%%%%%%%%%%%%%%%%%%%
%%%%%%%%%%%%%%%%%%%%%%%%%%%%%%%%%%%%%%%%%%
%%%%%%%%%%%%%%%%%%%%%%%%%%%%%%%%%%%%%%%%%%
%%%%%%%%PAGEBREAK%%%%%%%PAGEBREAK%%%%%%%%%
%%%%%%%%%%%%%%%%%%%%%%%%%%%%%%%%%%%%%%%%%%
%%%%%%%%%%%%%%%%PAGEBREAK%%%%%%%%%%%%%%%%%
%%%%%%%%%%%%%%%%%%%%%%%%%%%%%%%%%%%%%%%%%%
%%%%%%%%PAGEBREAK%%%%%%%PAGEBREAK%%%%%%%%%
%%%%%%%%%%%%%%%%%%%%%%%%%%%%%%%%%%%%%%%%%%
%%%%%%%%%%%%%%%%%%%%%%%%%%%%%%%%%%%%%%%%%%
\begin{ekdosis}
  \begin{prose}
    \noindent
\app{\lem[wit={E,P,L}]{yathodakasya sparśo}
  \rdg[wit={B}]{yathodakasya sparśā}
  \rdg[wit={N1,N2,D,U1}]{yathā udakasparśo}
  \rdg[wit={U2}]{yathodakasparśo}}
na
\app{\lem[wit={ceteri}]{bhavati}
  \rdg[wit={B}]{bhavatī}}
tathaivātmani/
%------------------------------
%yathākāśamadhye   pavanaḥ svecchayā bhramati/ \E
%yathākāśamadhye   pavanaḥ svechayā  bhramati \P
%yathā ākāśamadhye pavanaḥ svechayā  bhramati/ \L
%yathā ākāśamadhye pavanaḥ svechayā  bhramatī/ \B
%yathā ākāśamadhye pavanasvachayā    bhramati/ \N1
%yathā ākāśamadhye pavanasvachayā    bhramati \D
%yathā ākāśamadhye pavanasvachayā    bhramati/ \N2
%yathā ākāśamadhye pavanaḥ svechayā  bhramayati \U1
%yathā 'kāśamadhye pavanaḥ svechayā  bhramati// \U2
%------------------------------
%Just as the wind wanders according to its own will in space,...  
%------------------------------
\note[type=source, labelb=130, lem={pavanaḥ}]{Ysv (PT): yathākāśe bhraman vāyur ākāśaṃ vrajate svayam | tathākāśe mano līnaṃ rājayogakriyā matā | jagatsaṃsarganirlepaṃ padmapatrajalaṃ yathā ||}
yathā\app{\lem[wit={E,P}]{kāśamadhye}
  \rdg[wit={U2}]{'kāśamadhye}
  \rdg[wit={ceteri}]{ākāśamadhye}}
\app{\lem[wit={ceteri}]{pavanaḥ svechayā}
  \rdg[wit={N1,N2,D}]{pavanasvachayā}}
\app{\lem[wit={ceteri}]{bhramati}
  \rdg[wit={U1}]{brahmayati}}
%------------------------------
%tathā yasya manaḥ nirākāramadhye līnaṃ bhavati/  sa eva caryāyogaḥ// \E
%tathā yasya manaḥ nirākāramadhye līnaṃ bhavati   sa eva caryāyogaḥ   \P
%tathā yasya manaḥ nirākāramadhye līnaṃ bhavati   sa eva caryāyogaḥ// \L
%tathā yasya manaḥ nirākāramadhye līnaṃ bhavatī   sa eva caryāyogaḥ// \B
%tathā yamanaḥ     nirākāramadhye līnaṃ bhavati/  sa eva kriyāyogaḥ// \N1
%tathā yasya manaḥ nirākāramadhye līnaṃ bhavati/  sa eva kriyāyogaḥ// \D !!!!!Stemma point!!!!!!
%tathā       pavananirākāramadhye līnaṃ bhavati/  sa eva kriyāyogaḥ// \N2
%tathā yasya manaḥ nirākāramadhye līnaṃ bhavati   sa eva kriyāyogaḥ   \U1 
%tathā yasya manaḥ nirākāramadhye līnaṃ bhavati// sa eva caryāyogaḥ// \U2
%------------------------------
%Likewise is the mind of whom is absorbed into the universal spirit [wanders according to its own will in space]. This is \textit{\caryāyoga}.  
%------------------------------
tathā
\app{\lem[wit={ceteri}]{yasya manaḥ}
  \rdg[wit={D}]{yamanaḥ}
  \rdg[wit={N2}]{pavana°}}
nirākāramadhye līnaṃ
\app{\lem[wit={ceteri}]{bhavati}
  \rdg[wit={B}]{bhavatī}}/
sa eva
\app{\lem[wit={ceteri}]{caryāyogaḥ}
  \rdg[wit={N1,N2,D,U1}]{kriyāyogaḥ}}\dd{}
\end{prose}
\end{ekdosis}
 %%%%%%%%%%%%%%%%%%%%%%%%%%%%%
 %%%%%%%%%%%%%%%%%%%%%%%%%%%%%
 %%%%%%%%%%%%%%%%%%%%%%%%%%%%%
 %%%%%%%%%%%%%%%%%%%%%%%%%%%%%
 %%%%%%%%%%%%%%%%%%%%%%%%%%%%%
\begin{ekdosis}
  \ekddiv{type=ed}
  \bigskip
  \centerline{\textrm{\small{[Haṭhayoga]}}}
    \bigskip
      \begin{prose}       
%------------------------------
%idānīṃ grahayogaḥ kathyate/  \E %[p.23]
%idānīṃ haṭhayogaḥ kathyate   \P
%idānīṃ haṭhayogaḥ kathyate/  \L
%idānīṃ haṭayoga   kathyate/  \B
%idānīṃ haṭhayogaḥ kathyate//  \N1
%idānīṃ haṭhayogaḥ kathyate/  \D
%idānīṃ haṭhayoga  kathyate// \N2
%idānīṃ haṭhayogaḥ kathyate   \U1
%idānīṃ haṭhayoga  kathyate   \U2
%------------------------------
%Now \textit{haṭhayoga} is explained. 
%------------------------------
\app{\lem[wit={P,L,N1,D,U1}]{haṭhayogaḥ}
          \rdg[wit={U2}]{haṭhayoga}
          \rdg[wit={B}]{haṭayoga}
          \rdg[wit={E}]{grahayogaḥ}} kathyate/\note[type=source, labelb=131, lem={haṭhayogaḥ}]{Ysv (PT): idānīṃ haṭhayogas tu kathyate haṭhasiddhidaḥ | kṛtvāsanaṃ pavanāśaṃ śarīre rogahārakam | pūrakaṃ kumbhakañcaiva recakaṃ vāyunā bhajet | itthaṃ kramotkramaṃ jñātvā pavanaṃ sādhayet sadā | dhauty ādikarmaṣaṭkañ ca prakuryādd haṭhasādhakaḥ | etan nāḍyān tu deveśi vāyupūrṇaṃ pratiṣṭhitam | tato mano niścalaṃ syāt tata ānanda eva hi | haṭhayogān na kālaḥ syān manonāśo bhaved yadi |}
idānīṃ
%------------------------------
%recakapūrakakumbhaka  ityādiprakāreṇa   pavanasādhanaṃ     kartavyam/ \E
%recakapūrakakuṃbhaka  ityādiprakāreṇa   pavanasādhanaṃ     karttavyaṃ \P
%recakapūrakakumbhaka  ityādiprakāreṇa   pavanasya sādhanaṃ kartavyam// \L
%recakapūrakakuṃbhaka  ityādiprakāreṇa// pavanasya sādhanaṃ kartavyam \B
%recakapūrakakuṃbhaka/ ityādiprakāreṇa   pavanasya sādhanaṃ kartavyaṃ/ \N1
%recakapūrakakuṃbhaka  ityādiprakāreṇa   pavanasya sādhanaṃ kartavyaṃ// \D
%recakapūrakakuṃbhaka  ityādhiprakāreṇa  pavanasya sādhanaṃ kartavyaṃ// \N2
%recakapūrakakuṃbhaka  ityādiprakāreṇa   pavanasya sādhanaṃ kartavyaṃ \U1
%recakapūrakakuṃbhaka  ityādiprakāreṇa   pavanasya sādhanaṃ kartavyaṃ// \U2
%------------------------------
%The practice of breath shall be done in this manner: "Exhalation, Inhalation [and] Retention etc.
%------------------------------        
        recakapūrakakuṃbhaka
        \app{\lem[wit={ceteri}, alt={ityādi}]{ityādi}
          \rdg[wit={N2}]{ityādhi°}
        }prakāreṇa
        \app{\lem[wit={ceteri}]{pavanasya sādhanaṃ}
          \rdg[wit={E,P}]{pavanasādhanaṃ}}
 \app{\lem[wit={E,L,B}]{kartavyam}
   \rdg[wit={ceteri}]{kartavyaṃ}}/
%------------------------------
%atha ca dhautyādiṣaṭkarmakāraṇāt   śarīrasya śuddhir bhavati/ \E
%atha ca dhautyādiṣaṭkarmakāraṇāt   śarīrasya śuddhir bhavati \P
%atha ca dhautyādiṣaṭkarmakāraṇāt// śarīrasya śuddhir bhavati \L
%atha ca  dhotyādiṣaṭkarmakaraṇāt// śarīrasya śuddhir bhavatī \B
%atha ca dhautyādiṣaṭkarmakaraṇāt/  śarīrasya śuddhir bhavati/ \N1
%atha ca dhautyādiṣaṭkarmakaraṇāt   śarīrasya śuddhir bhavati// \D
%atha ca dhautyādiṣaṭkarmakaraṇāt// śarīrasya śuddhir bhavati// \N2
%atha   vidhotyādiṣaṭkarmakaraṇāt   śarīrasya śuddhir bhavati/ \U1
%atha ca dhautyādiṣaṭkarmakaraṇāt// śarīrasya śuddhir bhavati// \U2 %%%408.jpg 
%------------------------------
%And then due to the six practices(\textit{ṣaṭkarma}), like \textit{dhauti} etc. the purification of the body arises. 
%------------------------------        
 atha
 \app{\lem[wit={ceteri}]{ca}
   \rdg[wit={U1}]{\om}}
 \app{\lem[wit={ceteri}]{dhautyādi}
   \rdg[wit={B}]{dhotyādi}
   \rdg[wit={U1}]{vidhotyādi}
 }ṣaṭkarmakāraṇāt śarīrasya śuddhir\skp{-}bhavati/
 %------------------------------
%sūryanāḍīmadhye       pavanaḥ pūrṇo yadā tiṣṭati/   \E %!
%sūryanāḍīmadhye       pavanaḥ pūrṇo yadā tiṣṭati    \P
%sūryanāḍīmadhye       pavanapūrṇo   yadāti/         \L
%sarvasūryanāḍīmadhye  pavanapūrṇo   yadāti/         \B
%sūryanāḍīmadhye       pavanaḥ pūrṇo yadā tiṣṭhati/  \N1
%sūryanāḍīmadhye       pavanaḥ pūrṇo yadā tiṣṭhati   \D
%sūryanāḍīmadhye       pvanaḥ  pūrṇo yadā tiṣṭhati/  \N2
%sūryanāḍīmadhye       pavanaḥ pūrṇo yadā tiṣṭhati/  \U1
%sūryanāḍīmadhye       pavanaḥ sūryo yadā tiṣṭhati// \U2
%------------------------------
%When the full breath abides in the middle of the sun-channel, ... 
%------------------------------
 \app{\lem[wit={ceteri}]{sūryanāḍīmadhye}
   \rdg[wit={B}]{sarvasūryanāḍīmadhye}}
 \app{\lem[wit={ceteri}]{pavanaḥ pūrṇo}
   \rdg[wit={L,B}]{pavanapūrṇo}
   \rdg[wit={N2}]{pvanaḥ pūrṇo}}
 \app{\lem[wit={ceteri}]{yadā tiṣṭhati}
   \rdg[wit={L,B}]{yadāti}}/
%------------------------------
%tadā mano  niścalaṃ bhavati/  \E
%tadā mano  niścalo  bhavati   \P
%tadā mano  niścalo  bhavati/  \L
%tadā mano  niścalo  bhavatī// \B
%tadā manaḥ niścalaṃ bhavati/  \N1
%tadā manaḥ niścalaṃ bhavati   \D
%tadā manaḥ niścalaṃ bhavati   \N2
%tadā manaḥ niścalaṃ bhavati   \U1
%tadā mano  niścalaṃ bhavati// \U2
%------------------------------
%Then the mind is unmovable. 
%------------------------------
 tadā
 \app{\lem[wit={ceteri}]{mano}
   \rdg[wit={N1,N2,D,U1}]{manaḥ}}
\app{\lem[wit={ceteri}]{niścalaṃ}
  \rdg[wit={P,L,B}]{niścalo}}
bhavati/
%------------------------------
%manaso  niścalatvena ānandarūpaṃ      pratyakṣaṃ bhāsate/  \E
%manaso  niścalatve   ānandaṃ svarūpa--pratyakṣaṃ bhāsate   \P %%%%7640.jpg
%manaso  niścalatve   ānandaṃ svarūpaṃ pratyakṣaṃ bhāsate/  \L
%manaso  niścalatve   ānaṃdaṃ svarūpaṃ pratyakṣaṃ bhāsate// \B
%manasaḥ niścalatve   ānaṃdasvarūpaṃ   pratyakṣaṃ bhāsate/  \N1
%manasaḥ niścalatve   ānaṃdasvarūpaṃ   pratyakṣaṃ bhāsate/  \D
%manasaḥ niścalatve   ānaṃdasvarūpaṃ   pratyakṣaṃ bhāṣate/  \N2
%manasaḥ niścalatve   ānaṃdasvarūpaṃ   pratyakṣaṃ bhāṣate/  \U1 %%%273.jpg
%manaso  niścalatve   ānaṃdasvarūpaṃ   pratyakṣaṃ bhāsate// \U2
%------------------------------
%The form of bliss immediately shines through the motionless mind.  
%------------------------------
\app{\lem[wit={ceteri}]{manaso}
  \rdg[wit={N1,N1,D,U1}]{manasaḥ}}
\app{\lem[wit={ceteri}]{niścalatve}
  \rdg[wit={E}]{niścalatvena}}
\app{\lem[wit={ceteri}]{ānandasvarūpaṃ}
  \rdg[wit={L,B}]{ānaṃdaṃ svarūpaṃ}
  \rdg[wit={P}]{ānandaṃ svarūpa°}
  \rdg[wit={E}]{ānandarūpaṃ}}
pratyakṣaṃ
\app{\lem[wit={ceteri}]{bhāsate}
  \rdg[wit={N2,U1}]{bhāṣate}}/
%------------------------------
%haṭhayogakāraṇāt  manaḥ   śūnyamadhye līnaṃ   bhavati/  kālaḥ samīpe   nāgacchati/  \E
%haṭhayogakāraṇāt  manaḥ   śūnyamadhye līnaṃ   bhavati   kālaḥ samīpe   nāgacchati   \P %%%%7640.jpg
%haṭhayogakāraṇāt  manaḥ   śūnyamadhye līnaṃ   bhavati/  kālaḥ samīpe   nāgacchati// \L
%haṭayogākāraṇāt   manaḥ// śūnyamadhye līnaṃ   bhavatī/  kālāsamīpe nāma gacchati//  \B
%haṭhayogakaraṇāt  manaḥ   śūnyamadhye līnaṃ   bhavati/  kālaḥ samīpe   nāgachati//  \N1
%haṭhayogakaraṇāt  manaḥ   śūnyamadhye līnaṃ   bhavati// kālaḥ samīpe   nāgachaṃti// \D
%haṭhayogakaraṇāt  mana----śūnyamadhye līnaṃ   bhavati/  kālasamīpe     nāgachati//  \N2
%haṭhayogakaraṇāt/ manaḥ   śūnyamadhye līnaṃ   bhavati/  kālasamīpe ti  nāgachati    \U1 %%%273.jpg
%haṭhayogakaraṇāt  manaḥ   śūnyamadhye sthānaṃ bhavati// kāsaḥ samīpe   nāgachati//  \U2
%------------------------------
%Due to the execution of haṭhayoga the mind becomes absorbed into emptiness. The time of death does not approach.
%------------------------------
\app{\lem[wit={ceteri}, alt={haṭha°}]{haṭha}
  \rdg[wit={B}]{haṭa}
}\app{\lem[wit={ceteri},alt={yoga°}]{yoga}
  \rdg[wit={B}]{yogā°}
}\app{\lem[wit={ceteri}]{karaṇāt}
  \rdg[wit={E,P,L,B}]{kāraṇāt}}
\app{\lem[wit={ceteri}]{manaḥ}
  \rdg[wit={N2}]{mana}}
śūnyamadhye
\app{\lem[wit={ceteri}]{līnaṃ}
  \rdg[wit={U2}]{sthānaṃ}}
bhavati/
\app{\lem[wit={ceteri}]{kālaḥ}
  \rdg[wit={B}]{kālā°}
  \rdg[wit={N2,U1}]{kāla°}
  \rdg[wit={U2}]{kāsaḥ}}
samīpe
\app{\lem[wit={ceteri}]{nāgacchati}
  \rdg[wit={B}]{nāma gacchati}
  \rdg[wit={D}]{nāgachaṃti}
  \rdg[wit={U1}]{ti nāgachati}}\dd{}
%------------------------------
%idānīṃ haṭhayogasya dvitīyo  bhedaḥ kathyate/   \E
%idānīṃ haṭhayoga----dvitīya--bhedaḥ kathyate    \P
%idānīṃ haṭhayogasya dvitīya--bhedāḥ kathyante/  \L
%idānīṃ haṭayogasya  dvitīyaṃ bhedāḥ kathyaṃte// \B
%idānīṃ haṭhayogasya dvitīyo  bhedaḥ kathyate//  \N1
%idānīṃ haṭhayogasya dvitīya--bhedaḥ kathyate    \D
%idānīṃ haṭayogasya  dvitīyo  bhedaḥ kathyate    \U1
%idānīṃ haṭhayogasya dvitīyo  bhedaḥ kathyate//  \U2 
%------------------------------
%Now, the second division of haṭhayoga is explained.
%------------------------------
\app{\lem[wit={ceteri}]{haṭhayogasya}
  \rdg[wit={B,U1}]{haṭayogasya}
  \rdg[wit={P}]{haṭhayoga°}}
\app{\lem[wit={ceteri}]{dvitīyo}
  \rdg[wit={P,L,D}]{dvitīya°}
  \rdg[wit={B}]{dvitīyaṃ}}
\app{\lem[wit={ceteri}]{bhedaḥ}
  \rdg[wit={L,B}]{bhedāḥ}}
\app{\lem[wit={ceteri}]{kathyate}
  \rdg[wit={L,B}]{kathyante}}/ \note[type=source, labelb=132, lem={dvitīyo bhedaḥ}]{Ysv (PT): idānīṃ haṭhayogasya dvitīyaṃ bhedam acchṛṇu | ākāśe nāsikāgre tu sūryakoṭisamaṃ smaret | śvetaṃ raktaṃ tathā pītaṃ kṛṣṇamityādirūpataḥ | evaṃ dhyātvā cirāyuḥ syād aṅgājananavarjitam | śivatulyo mahātmāsau haṭhayogaprasādataḥ | haṭhāj jyotir mayo bhūtvā hyantareṇa śivo bhavet | ato 'yaṃ haṭhayogaḥ syāt siddhidaḥ siddhasevitaḥ |}
idānīṃ
%------------------------------
%pādādārabhya śiraḥ paryaṃtaṃ    svaśarīre  koṭisūryatejaḥ   samānaṃ śvetaṃ pītaṃ       raktaṃ kiṃcidvarṇaṃ ciṃtyate/  \E
%pādādārabhya śiraḥ paryaṃtaṃ    svaśarīre  koṭisūryatejaḥ   samānaṃ śvetaṃ pītaṃ nīlaṃ raktaṃ kiṃdrupaṃ    cityate    \P
%pādādārabhya śira--paryaṃtaṃ    svaśarīre  koṭisūryatejaḥ   samānaśvetaṃ nīlaṃ         raktaṃ tiṃdrupaṃ    ciṃtate/   \L
%pādādārabhya śira--paryaṃtaṃ    svaśarīre  koṭisūryatejaḥ// samānaśvetanīlaṃ           raktaṃ kiṃdrupaṃ    ciṃtate//  \B
%pādādārabhyā śiraḥ paryentaṃ    svaśarīre  koṭisūryatejaḥ   samānaṃ śvetaṃ pītaṃ nīlaṃ laktaṃ kiṃcidrūpaṃ  ciṃtyate   \N1 
%pādādārabhyā śiraḥ paryaṃtaṃ    svaśarīre  koṭisūryatejaḥ   samānaṃ śvetaṃ pītaṃ nīlaṃ raktaṃ kiṃcidrūpaṃ  ciṃtyate   \D
%pādādārabhya śiraḥ pariyataṃ    svaśarīraṃ koṭisūryatejaḥ   samānaṃ śvetaṃ pītaṃ nīlaṃ raktaṃ ciṃrūpaṃ     ciṃtyate   \U1
%pādādārabhya śiro  paryaṃtaṃ    svaśarīre  koṭisūryye tejaḥ samānaṃ śvetaṃ pītaṃ nīlaṃ raktaṃ kiṃcidrūpaṃ  ciṃtyate// \U2
%------------------------------
%The shine of ten million suns in one's own body beginning from the feet to the top of head is contemplated in any color equal to white, yellow [or] red.
%------------------------------
\app{\lem[wit={ceteri}]{pādādārabhya}
  \rdg[wit={N1,D}]{pādādārabhyā}}
\app{\lem[wit={ceteri}]{śiraḥ}
  \rdg[wit={L,B}]{śira°}
  \rdg[wit={U2}]{śiro}}
\app{\lem[wit={ceteri}]{paryantaṃ}
  \rdg[wit={N1}]{paryentaṃ}
  \rdg[wit={U1}]{pariyataṃ}}
\app{\lem[wit={ceteri}]{svaśarīre}
  \rdg[wit={U1}]{svaśarīraṃ}}
\app{\lem[wit={ceteri}]{koṭisūryatejaḥ}
  \rdg[wit={U2}]{koṭisūryye tejaḥ}}
\app{\lem[wit={ceteri}]{samānaṃ}
  \rdg[wit={L,B}]{samāna°}
  \rdg[wit={ceteri}]{śvetaṃ}
  \rdg[wit={B}]{śveta°}}
\app{\lem[wit={ceteri}]{pītaṃ}
  \rdg[wit={L,B}]{\om}}
nīlaṃ
\app{\lem[wit={ceteri}]{raktaṃ}
  \rdg[wit={N1}]{laktaṃ}}
\app{\lem[wit={N1,D,U2}]{kiṃcidrūpaṃ}
  \rdg[wit={P,B}]{kiṃdrupaṃ}
  \rdg[wit={L}]{tiṃdrupaṃ}
  \rdg[wit={U1}]{ciṃrūpaṃ}
  \rdg[wit={E}]{kiṃcidvarṇaṃ}}
\app{\lem[wit={ceteri}]{cintyate}
  \rdg[wit={P}]{cityate}
  \rdg[wit={L,B}]{ciṃtate}}/
%------------------------------
%ttad  dhyānakāraṇāt     sakalaṃ   rogajvalanaṃ     bhavati/                      āyur          vardhate/          \E
%tad   dhyānakāraṇāt     sakalāṃge rogajvalanaṃ  na bhavati                       āyur vṛddhir  bhavati   \P
%tad   dhyānakāraṇāt     sakalaṃge rogajvalanaṃ  na bhavati/                      āyur          vardhate/          \L
%tat   dhyānakāraṇāt     sakalaṃge rogajvalanaṃ  na bhavati/                      āyur vṛddhir  bhavatī/  \B
%na    dhyānaṃ kāraṇāt/  sakalāṃge roga          na bhavati/  jvalanaṃ na bhavati āyur vṛddhir  bhavati/  \N1
%ta    dhyānaṃ karaṇāt// sakalāṃge rogajvalanaṃ  na bhavati//                                             \D
%tad---dhyānaṃ karaṇāt / sakalāṃge roga          na bhavati   jvaranaṃ na bhavati āyu--vṛddhir  bhavati// \N2
%ta    dhyānaṃ karaṇāt   sakalāṃge roga kṣataṃ?  na bhavati                       āyur vṛddhir  bhavati   \U1
%tat   dhyānakāraṇāt     sakalāṃge rogajvalanaṃ     bhavati//                     āyur vṛddhir  bhavati// \U2
%------------------------------
%aDue to the execution of meditation in the entire body disease does'nt arise, fever doesn't arise and vitality grows.  
%------------------------------
\app{\lem[wit={E,P,L,N2},alt={tad}]{ta\skp{d-dhyā}}
  \rdg[wit={B,U2}]{tat}
  \rdg[wit={D,U1}]{ta}
  \rdg[wit={N1}]{na}
}\app{\lem[wit={ceteri},alt={dhyānakāraṇāt}]{\skm{d-dhyā}nakāraṇāt}
  \rdg[wit={N1,N2,D,U1}]{dhyānaṃ karaṇāt}}
\app{\lem[wit={P,N1,D,N2,U1,U2}]{sakalāṅge}
  \rdg[wit={L,B}]{sakalaṃge}
  \rdg[wit={E}]{sakalaṃ}}
\app{\lem[type=emendation, resp=egoscr]{rogaḥ}
\rdg[wit={N1,N2}]{\korr roga}
\rdg[wit={E,P,L,B,D,U2}]{rogajvalanaṃ}
\rdg[wit={U1}]{roga kṣataṃ}}
\app{\lem[wit={ceteri}]{na}
  \rdg[wit={E,U2}]{\om}}
bhavati/
\end{prose}
\end{ekdosis}
\ekdpb*{}
%%%%%%%%%%%%%%%%%%%%%%%%%%%%%%%%%%%%%%%%%%
%%%%%%%%%%%%%%%%%%%%%%%%%%%%%%%%%%%%%%%%%%
%%%%%%%%PAGEBREAK%%%%%%%PAGEBREAK%%%%%%%%%
%%%%%%%%%%%%%%%%%%%%%%%%%%%%%%%%%%%%%%%%%%
%%%%%%%%%%%%%%%%PAGEBREAK%%%%%%%%%%%%%%%%%
%%%%%%%%%%%%%%%%%%%%%%%%%%%%%%%%%%%%%%%%%%
%%%%%%%%PAGEBREAK%%%%%%%PAGEBREAK%%%%%%%%%
%%%%%%%%%%%%%%%%%%%%%%%%%%%%%%%%%%%%%%%%%%
%%%%%%%%%%%%%%%%%%%%%%%%%%%%%%%%%%%%%%%%%%
%%%%%%%%%%%%%%%%%%%%%%%%%%%%%%%%%%%%%%%%%%
%%%%%%%%%%%%%%%%%%%%%%%%%%%%%%%%%%%%%%%%%%
%%%%%%%%PAGEBREAK%%%%%%%PAGEBREAK%%%%%%%%%
%%%%%%%%%%%%%%%%%%%%%%%%%%%%%%%%%%%%%%%%%%
%%%%%%%%%%%%%%%%PAGEBREAK%%%%%%%%%%%%%%%%%
%%%%%%%%%%%%%%%%%%%%%%%%%%%%%%%%%%%%%%%%%%
%%%%%%%%PAGEBREAK%%%%%%%PAGEBREAK%%%%%%%%%
%%%%%%%%%%%%%%%%%%%%%%%%%%%%%%%%%%%%%%%%%%
%%%%%%%%%%%%%%%%%%%%%%%%%%%%%%%%%%%%%%%%%%
%%%%%%%%%%%%%%%%%%%%%%%%%%%%%%%%%%%%%%%%%%
%%%%%%%%%%%%%%%%%%%%%%%%%%%%%%%%%%%%%%%%%%
%%%%%%%%PAGEBREAK%%%%%%%PAGEBREAK%%%%%%%%%
%%%%%%%%%%%%%%%%%%%%%%%%%%%%%%%%%%%%%%%%%%
%%%%%%%%%%%%%%%%PAGEBREAK%%%%%%%%%%%%%%%%%
%%%%%%%%%%%%%%%%%%%%%%%%%%%%%%%%%%%%%%%%%%
%%%%%%%%PAGEBREAK%%%%%%%PAGEBREAK%%%%%%%%%
%%%%%%%%%%%%%%%%%%%%%%%%%%%%%%%%%%%%%%%%%%
%%%%%%%%%%%%%%%%%%%%%%%%%%%%%%%%%%%%%%%%%%
\begin{ekdosis}
  \ekddiv{type=ed}
  \begin{prose}
\app{\lem[wit={N2}]{jvaranaṃ na bhavati}
  \rdg[wit={N1}]{jvalanaṃ na bhavati}
  \rdg[wit={ceteri}]{\om}}/
\app{\lem[wit={ceteri}, alt={āyur}]{āyu\skp{r-vṛ}}
  \rdg[wit={N2}]{āyu°}
  \rdg[wit={D}]{\om}
}\app{\lem[wit={ceteri},alt={vṛddhir}]{\skm{r-vṛ}ddhi\skp{r-bha}}
  \rdg[wit={E,L,D}]{\om}
}\app{\lem[wit={ceteri},alt={bhavati}]{\skm{r-bha}vati}
  \rdg[wit={B}]{bhavatī}
  \rdg[wit={E,L}]{vardhate}
  \rdg[wit={D}]{\om}}\dd{}
\end{prose}
\end{ekdosis}
 %%%%%%%%%%%%%%%%%%%%%%%%%%%%%
 %%%%%%%%%%%%%%%%%%%%%%%%%%%%%
 %%%%%%%%%%%%%%%%%%%%%%%%%%%%%
 %%%%%%%%%%%%%%%%%%%%%%%%%%%%%
 %%%%%%%%%%%%%%%%%%%%%%%%%%%%%
\begin{ekdosis}
  \ekddiv{type=ed}
          \bigskip
    \centerline{\textrm{\small{[Jñānayoga]}}}
          \bigskip
          \begin{prose}
%------------------------------
%idānīṃ jñānayogasya lakṣaṇaṃ kathyate/ \E
%idānīṃ jñānayogasya lakṣaṇaṃ kathyate \P
%idānīṃ jñānayogasya lakṣaṇaṃ// \L 5976_0011.jpg 
%idānīṃ jñānayogasya lakṣaṇaṃ// \B
%idānīṃ jñānayogasya lakṣaṇaṃ// \N1 %%%%p.6 verso 
%idānīṃ jñānayogasya lakṣaṇaṃ// \D
%idānīṃ jñānayogasya lakṣaṇaṃ kathyate// \N2
%idānī  jñānayogasya lakṣaṇaṃ kathyate   \U1
%idānīṃ jñānayogasya lakṣaṇaṃ kathyate// \U2
%------------------------------
%Now the characteristic of jñānayoga is explained. 
%-----------------------------
\note[type=source, labelb=133, lem={jñānayogasya}]{Ysv (PT): idānīṃ jñānayogasya lakṣaṇaṃ kathyate śive | yaj jñātvā jñānasampūrṇaḥ śivaḥ syān na punarbhavaḥ |}
\app{\lem[wit={ceteri}]{idānīṃ}
  \rdg[wit={U1}]{idānī}}
jñānayogasya lakṣaṇaṃ
\app{\lem[wit={E,P,N2,U1,U2}]{kathyate}
  \rdg[wit={L,B,N1,D}]{\om}}/
\end{prose}
%--------------------------------------
%ekam eva jagat paśyed viśvāvasu vibhāsvaram/
%avikalpatayā yuktyā jñānayogaṃ samācaret//1// \E
%
%ekam eva cayat paśyed viśvātmāsu vibhāsvaram       
%avikalpatayā yuktyā jñānayogaṃ samācaret 1 \P
%
%ekam evā jagat paśyed viśvātmāsu vibhāsvaraṃ//
%avikalpatayā yuktā jñānayogaṃ samācaret// \L
%
%ekam evā jagat paśyad visvātmāsu vibhāsvaraṃ//
%avikalpatayā yuktā jñānayogaṃ samācaret// \B
%
%ekam eva jagat paśyed viśvātmā viśvabhāvanaḥ/
%iti kṛtvā tu vai yukto jñānayogaṃ samācaret// SVARODAYA
%
%ekam eva jagat paśyed dviśvātmāsu vibhāsvaraṃ/
%avikalpatayā yuktyā jñānayogaṃ samācaret//1// \N1
%
%ekam eva jagat paśyed dviśvātmāsu vibhāsvaraṃ//
%avikalpatayā yuktyā jñānayogaṃ samācaret//1// \D
%
%ekam eva jagat paśyed dviśvātmāsu vibhāsvaraṃ//
%avikalpatayā yuktyā jñānayogaṃ samācaret//1// \N2
%
%ekam eva jagataḥ paśyed dviśvātmāsu vibhāsvaraṃ
%āvikalpatayā yuktyā jñānayogaṃ samācaret//1// \U1
%
%ekam eva jagataḥ paśyed dviśvātmāsu vibhāsvaraṃ
%āvikalpatayā yuktyā jñānayogaṃ samācaret// \U2
%------------------------------
%He shall see the world truly as being one, shining in all selves. 
%By applying indistinctness he shall accomplish \textit{jñānayoga}.   
%------------------------------
\begin{tlg}
  \tl{\note[type=source, labelb=134, lem={ekam eva}]{Ysv (PT): ekam eva jagat paśyed viśvātmā viśvabhāvanaḥ | iti kṛtvā tu vai yukto jñānayogaṃ samācaret ||}
eka\skp{m-e}\app{\lem[wit={ceteri}, alt={eva}]{\skm{m-e}va}
  \rdg[wit={L,B}]{evā}}
\app{\lem[wit={ceteri},alt={jagat}]{jaga\skp{t-pa}}
  \rdg[wit={P}]{cayat}
}\app{\lem[wit={ceteri},alt={paśyed}]{\skm{t-pa}śye\skp{d-vi}}
  \rdg[wit={B}]{paśyad}
}\app{\lem[wit={P,L,B},alt={viśvātmāsu}]{\skm{d-vi}śvātmāsu}
  \rdg[wit={E}]{viśvāvasu}
  \rdg[wit={N1,D,N2,U1,U2}]{dviśvātmāsu}}
vibhāsvaraṃ/}\\
\tl{\app{\lem[wit={ceteri}]{avikalpatayā}
  \rdg[wit={U1,U2}]{āvikalpatayā}}
\app{\lem[wit={ceteri}]{yuktyā}
  \rdg[wit={L,B}]{yuktā}} 
jñānayogaṃ samācaret\dd{}1\hskip-2pt\dd{}}
\end{tlg}
%------------------------------
%yatra yatra sthito vāpi sarvajñānamayaṃ jagat/ 
%sa evaṃ vetti bodhena so pi jñānādhikāraṇāt//2// \E 
%
%yatra yatra sthito vāpi sarvajñānamayaṃ jagat  
%ya evaṃ vetti bodhena so pi jñānādhikāravān \P
%
%yatra yatra sthito vāpi sarvajñānamayaṃ jagat//  
%ya evaṃ vetti bodhena so pi jñānādhikāravān// \L
%
%yatra yatra sthito vāpi sarvajñānamayaṃ jagat//  
%ya evaṃ ve bodhena so pi jñānādhikāravān// \B
%
%yatra tatra sthito vāpi sarvajñānamayaṃ jagat/
%ya evam asti bodhena so'pi jñānādhikāravān/ \SVARODAYA
%
%yatra yatra sthito vāpi sarvajñānamayaṃ jagat/
%ya evaṃ vetti bodhena so pi jñānādhikāravān//2//\N1
%
%yatra yatra sthito vāpi sarvajñānamayaṃ jagat//
%ya evaṃ vetti bodhena so pi jñānādhikāravān//2//\D
%
%yatra yatra sthito vāpi sarvajñānamayaṃ jagat//
%ya evaṃ vetti bodhena so pi jñānādhikāravān//2//\N2
%
%yatra yatra sthito vāpi sarvajñānamayaṃ jagat  %%%273.jpg
%evaṃ vette na bodhena so pi jñānādhikāravān 2    \U1
%
%yatra yatra sthito hiṃsa sarvajñānamayaṃ jagat//  
%evaṃ vetti bodhena so pi jñānādhikāravān// 2    \U2
%------------------------------
%Wherever the world is established or made of omniscience,
%who knows thus by means of insight, he is a like an expert of knowledge.      
%------------------------------
\begin{tlg}
  \tl{\note[type=testium, labelb=135, lem={yatra yatra}]{Ysv (PT): yatra tatra sthito vāpi sarvajñānamayaṃ jagat | ya evam asti bodhena so'pi jñānādhikāravān ||}
    yatra tatra sthito \app{\lem[wit={ceteri}]{vāpi}
      \rdg[wit={U2}]{hiṃsa°}} sarvajñānamayaṃ jagat/}\\
  \tl{\app{\lem[wit={ceteri}]{ya evaṃ}
      \rdg[wit={U1,U2}]{evaṃ}}
    \app{\lem[wit={ceteri}]{vetti}
      \rdg[wit={U1}]{vette na}
      \rdg[wit={B}]{ve}} bodhena so pi
    \app{\lem[wit={ceteri}]{jñānādhikāravān}
      \rdg[wit={E}]{jñānādhikāraṇāt}}\dd{}2\hskip-2pt\dd{}}
\end{tlg}
%------------------------------
%
%\om!!!!!                                                                                                        \E
%
%prāpnoti śāmbhavīmantrān  sadā nityaparāyaṇaḥ/   yathā nyagrodhavījaṃ hi kṣitau   vaptur drumāyate/               \SVARODAYA  
%prāpnoti śāmbhavīṃ sattāṃ sadāṃdvaitaparāyaṇaḥ   yathā nyagrodhabījaṃ hi kṣitāv   uptaṃ drumāyate likāṃ pa..vāḥ 4 \P  7640.jpg last line check word!!!
%prāpnoti śāmbhavīṃ sattān sadādvaitaparāyaṇaḥ//  yathā nyagrodhavīja  hi kṣitāv   utpadyate yathā//               \L
%prāpnoti śāmbhaviṃ sattāṃ sadādvaitaparāyaṇaḥ//  yathā nyagrodhabījāṃ hi kṣitī    utpadyate//                      \B
%prāpnoti sāṃbhavīṃ satta  sadādvaitaparāyaṇaḥ//  yathā nyagrodhavījaṃ hi kṣitāv   uptaṃ drumāyate 3//              \N1
%prāpnoti sāṃbhavīsattāṃ   sadādvaitaparāyaṇaḥ//  yathā nyagrodhavījaṃ hi kṣitāv   uptaṃ drumāyate//                \D
%prāpnoti sāṃbhavīsattā    sadādvaitaparāyaṇaḥ//  yathā nyagrodhavījaṃ hi kṣitāv   uptaṃ drumāyate//                \N2 %drumaayate=denom. wie ein beim  sein 
%prāpnoti sāṃbhavīsattāṃ   sadādvaitaparāyaṇaḥ    yathā nyagrodhabījaṃ hi kṣitāptā ukta drumāyate 3              \U1
%prāpnoti sāṃbhavīsattāṃ   yadādvaitaparāyaṇaḥ//  yathā nyagrodhabījaṃ hi kṣitāv   uptaṃ drumāyate//               \U2
%------------------------------
%He always attains the reality of śāmbhavī - the goal of eternal non-duality.  
%Just as the seed of the Nyagrodha scattered onto the soil [always] becomes a tree.
%------------------------------
\begin{tlg}
  \tl{\note[type=source, labelb=136, lem={prāpnoti}]{Ysv (PT): prāpnoti śāmbhavīmantrān sadā nityaparāyaṇaḥ | yathā nyagrodhavījaṃ hi kṣitau vaptur drumāyate ||}
    \app{\lem[wit={ceteri}]{prāpnoti}
      \rdg[wit={E}]{\om}}
    \app{\lem[wit={D,U1,U2}]{sāṃbhavīsattāṃ}
      \rdg[wit={P,B}]{śāmbhavīṃ sattāṃ}
      \rdg[wit={L}]{śāmbhavīṃ sattān}
      \rdg[wit={N1}]{sāṃbhavīṃ satta}
      \rdg[wit={N2}]{sāṃbhavīsattā}
      \rdg[wit={E}]{\om}}
    \app{\lem[wit={ceteri}]{sadādvaitaparāyaṇaḥ}
      \rdg[wit={U1}]{sadāṃdvaita°}
      \rdg[wit={E}]{\om}}/}\\
  \tl{\app{\lem[wit={ceteri}]{yathā}
      \rdg[wit={E}]{\om}}
    \app{\lem[wit={ceteri}]{nyagrodhabījaṃ}
      \rdg[wit={N1,N2,D}]{°vījaṃ}
      \rdg[wit={L}]{°vīja}
      \rdg[wit={E}]{\om}}
    \app{\lem[wit={ceteri}]{hi}
      \rdg[wit={E}]{\om}}
    \app{\lem[wit={ceteri},alt={kṣitāv}]{kṣitā\skp{v-u}}
      \rdg[wit={B}]{kṣitī}
      \rdg[wit={U1}]{kṣitāptā}
      \rdg[wit={E}]{\om}
 }\app{\lem[wit={ceteri},alt={uptaṃ drumāyate}]{\skm{v-u}ptaṃ drumāyate}
      \rdg[wit={P}]{uptaṃ drumāyate likāṃ pa..vāḥ}
      \rdg[wit={L}]{utpadyate yathā}
      \rdg[wit={B}]{utpadyate}
      \rdg[wit={U1}]{ukta drumāyate}
      \rdg[wit={E}]{\om}}\dd{}3\hskip-2pt\dd{}}
\end{tlg}
%------------------------------
%ekāntaṃ  naikadā  svena   dṛśyate  daśadhā  kṛtaḥ/  mūlāṅkurasya  coddaṇḍāḥ śākhākuṇḍalapallavāḥ//3//   \E    cod?von cud? Wurzel in guṇa + daṇḍa? !!! em. zu śaśvadhā = immer wieder, jederzeit 
% \om                                                                                                    \P
%ekāṃte   nekadhā  svena   dṛśyaṃte daśadhāt kṛp?tā/ mūlāṃkurutva kudaṃḍaḥ  śākhākilekālapallavā        \B
%ekāṃte   nekadhā  svena   dṛśyaṃte daśadhāt kṛtaḥ/  mūlāṃkurutva kudaṃḍa   śākhākalikālapallavā        \L
%ekāṃtaṃ  naikadhā śveta   dṛśyate  daśadhā  kṛtā//  mūlāṃkurutva codaṃḍaḥ  śāvārakumbhalapallavaḥ//4// \N1   
%ekāṃtaṃ  naikadhā śvetana dṛśyate  daśadhā  kṛtā//  mūlāṃkurutva codarāṭaḥ śālavākumapadṛtravā//4//    \D
%ekāṃtaṃ  naikadhā śvetana dṛśyet   śadhā    kṛtā//  mūlāṃkurutva codarāṭaḥ śākhākumbhalapallavā//4//   \N2
%yekāṃtaṃ naikadhā svena   dṛśyate  śadhā    kṛtā    mūlāṃkurutva codaṃḍa   śākhākumbhalapallavaḥ       \U1
%ekāṃtaṃ  naikadhā svetana dṛśyate  daśadhā  kṛtiḥ// mūlāṃkurutva codaṃḍaḥ  śākhākusumapallavāḥ//       \U2
%------------------------------
%Nur eines, nicht zusammen mit dem Ich wird das zehnfach gemachte gesehen. 
%Die aufgerollten Sprossen der Äste, welche die austreibendem Stöcke sind vom Spross der Wurzel. 
%------------------------------
%Die absoluten Einheit (ekāntaṃ), wird als multibel (nämlich) aus zehn Teilen gemacht von einen selbst gesehen. !!!!!
%Die aufgerollen Sprösslinge der Zweige sind austreibende Stengel des Wurzeltriebes.  
%------------------------------
%The absolute unity (ekāntaṃ), is seen as a multiple (namely) made up of ten parts by oneself.
%The rolled up shoots of the branches are the sprouting stalks of the root shoot.  
%------------------------------
%The absolute unity (ekāntaṃ), is seen as manifoldly created again and again by oneself.
%The rolled up shoots of the branches are the sprouting stalks of the root shoot.  
%------------------------------
\note[type=source, labelb=137, lem={naikadhā}]{Ysv (PT): ādāv ekas tato 'nekaḥ svabhāvāc chādanādibhiḥ | varddhate'harniśaṃ vṛkṣaḥ patrapallavavistṛtaḥ|}
\begin{tlg}
  \tl{\app{\lem[wit={ceteri}]{ekāntaṃ}
  \rdg[wit={B,L}]{ekānte}
  \rdg[wit={U1}]{yekāṃtaṃ}
  \rdg[wit={P}]{\om}}
\app{\lem[wit={ceteri}]{naikadhā}
  \rdg[wit={E}]{naikadā}
  \rdg[wit={B,L}]{nekadhā}
  \rdg[wit={P}]{\om}}
\app{\lem[wit={ceteri}]{svena}
  \rdg[wit={N1}]{śveta}
  \rdg[wit={D,N1}]{śvetana}
  \rdg[wit={P}]{\om}}
\app{\lem[wit={ceteri}]{dṛśyate}
  \rdg[wit={B,L}]{dṛśyaṃte}
  \rdg[wit={N2}]{dṛśyet}
  \rdg[wit={P}]{\om}}
\app{\lem[wit={E,N1,N2}]{daśadhā}    %%[type=conjecture, resp=egoscr]{śaśvadhā}????
  \rdg[wit={B,L}]{daśadhāt}
  \rdg[wit={N2,U1}]{śadhā}
  \rdg[wit={P}]{\om}}
\app{\lem[type=emendation, resp=egoscr]{kṛtaṃ}
  \rdg[wit={E,L}]{\korr kṛtaḥ}
  \rdg[wit={N1,N2,D,U1}]{kṛtā}
  \rdg[wit={B}]{kṛptā}
  \rdg[wit={U2}]{kṛtiḥ}
  \rdg[wit={P}]{\om}}/}\\
 \tl{\app{\lem[wit={E}]{mūlāṅkurasya}
  \rdg[wit={ceteri}]{mūlāṃkurutva}
  \rdg[wit={P}]{\om}}
\app{\lem[wit={E,N1,U2}]{coddaṇḍāḥ}
  \rdg[wit={D,N2}]{codarāṭaḥ}
  \rdg[wit={B}]{kudaṃjaḥ}
  \rdg[wit={L}]{kudaṃḍa}
  \rdg[wit={P}]{\om}}
\app{\lem[wit={E}]{śākhākuṇḍalapallavāḥ}
  \rdg[wit={B,L}]{śākhākilekālapallavā}
  \rdg[wit={N1,U1}]{śāvārakumbhalapallavaḥ}
  \rdg[wit={N2}]{śākhākumbhalapallavā}
  \rdg[wit={D}]{śālavākumapadṛtravā}
  \rdg[wit={U2}]{śākhākusumapallavāḥ}
  \rdg[wit={P}]{\om}}\dd{}4\hskip-2pt\dd{}}
\end{tlg}
\end{ekdosis}
\ekdpb*{}
%%%%%%%%%%%%%%%%%%%%%%%%%%%%%%%%%%%%%%%%%%
%%%%%%%%%%%%%%%%%%%%%%%%%%%%%%%%%%%%%%%%%%
%%%%%%%%PAGEBREAK%%%%%%%PAGEBREAK%%%%%%%%%
%%%%%%%%%%%%%%%%%%%%%%%%%%%%%%%%%%%%%%%%%%
%%%%%%%%%%%%%%%%PAGEBREAK%%%%%%%%%%%%%%%%%
%%%%%%%%%%%%%%%%%%%%%%%%%%%%%%%%%%%%%%%%%%
%%%%%%%%PAGEBREAK%%%%%%%PAGEBREAK%%%%%%%%%
%%%%%%%%%%%%%%%%%%%%%%%%%%%%%%%%%%%%%%%%%%
%%%%%%%%%%%%%%%%%%%%%%%%%%%%%%%%%%%%%%%%%%
%%%%%%%%%%%%%%%%%%%%%%%%%%%%%%%%%%%%%%%%%%
%%%%%%%%%%%%%%%%%%%%%%%%%%%%%%%%%%%%%%%%%%
%%%%%%%%PAGEBREAK%%%%%%%PAGEBREAK%%%%%%%%%
%%%%%%%%%%%%%%%%%%%%%%%%%%%%%%%%%%%%%%%%%%
%%%%%%%%%%%%%%%%PAGEBREAK%%%%%%%%%%%%%%%%%
%%%%%%%%%%%%%%%%%%%%%%%%%%%%%%%%%%%%%%%%%%
%%%%%%%%PAGEBREAK%%%%%%%PAGEBREAK%%%%%%%%%
%%%%%%%%%%%%%%%%%%%%%%%%%%%%%%%%%%%%%%%%%%
%%%%%%%%%%%%%%%%%%%%%%%%%%%%%%%%%%%%%%%%%%
%%%%%%%%%%%%%%%%%%%%%%%%%%%%%%%%%%%%%%%%%%
%%%%%%%%%%%%%%%%%%%%%%%%%%%%%%%%%%%%%%%%%%
%%%%%%%%PAGEBREAK%%%%%%%PAGEBREAK%%%%%%%%%
%%%%%%%%%%%%%%%%%%%%%%%%%%%%%%%%%%%%%%%%%%
%%%%%%%%%%%%%%%%PAGEBREAK%%%%%%%%%%%%%%%%%
%%%%%%%%%%%%%%%%%%%%%%%%%%%%%%%%%%%%%%%%%%
%%%%%%%%PAGEBREAK%%%%%%%PAGEBREAK%%%%%%%%%
%%%%%%%%%%%%%%%%%%%%%%%%%%%%%%%%%%%%%%%%%%
%%%%%%%%%%%%%%%%%%%%%%%%%%%%%%%%%%%%%%%%%%
\begin{ekdosis}
  \ekddiv{type=ed}
  \noindent
%------------------------------
%srehapuṇyaphalaṃ   bīje vistaro yaṃ svabhāvataḥ/  tathāsau   nirmalo  nityo nirvikāro niraṃjanaḥ//4// \E
%snehapuṣpaphalaṃ   bīje vistāro yaṃ svabhāvataḥ   tāthāpasau nirmalau nityo nirvikāro niraṃjanaḥ     \P   %%7641.jpg Z.1
%snehe puṣpaphala---bīja-vistāro ya  svabhāvatāḥ   yāthāsau   nirmalo  nityo nirvikāro niraṃjanaḥ//    \B
%snehe puṣpaphala---bīja-vistāro ya  svabhāvatāḥ// tāthāsau   nirmalo  nityo nirvikāro niraṃjanaḥ//    \L
%snehapuṣpaphalaṃ   bīje vistārā yaṃ svabhāvataḥ/  tathāsau   nirmalo  nityo nirvikāro niraṃjanaḥ//5// \N1
%snehapuṣpaphalaṃ   bīje vistārā yasya  bhāvataḥ// tathāsau   nirmalo  nityo nirvikāro niraṃjanaḥ//5// \D
%snehapuṣpaphalaṃ   vīje vistāro yaṃ svabhāvataḥ// tathāsau   nirmalo  nityo nirvikāro niraṃjanaḥ//5// \N2
%snehapuṣpaṃ phalaṃ bīje vistāro yaḥ svabhāvataḥ   tathāsau   nirmalo  nityo nirvikāro niraṃjanaḥ 5    \U1  %%%%274.jpg
%snehapuṣpaphalaṃ   bīje vistāro yaṃ svabhāvataḥ// tathāsau   nirmalo  nityo nirvikāro niraṃjanaḥ// 5  \U2 %%%first Śloka in this series that is numbered in U2 
%------------------------------
%Aufgrund seines inhärenten Wesens ist dieser Ast mit seinen Zweigen, welcher die Frucht der Blüte der Liebe ist, im Samen.
%Gewiss, ist jenes rein, ewig, unveränderlich und makellos. 
%------------------------------
%By virtue of its inherent nature, this branch with its branches, which is the fruit of the flower of love, is in the seed.
%Certainly, that is pure, eternal, unchanging and immaculate.
%------------------------------
\note[type=source, labelb=138, lem={sneha°}]{Ysv (PT): snehapuṣpaphalair vījair vistāro 'yaṃ svabhāvataḥ | tathāsau nirmalo nityo nirvikāro nirañjanaḥ |}
\begin{tlg}
  \tl{
\app{\lem[wit={P,N1,N2,D,U2}]{snehapuṣpaphalaṃ}
  \rdg[wit={B,L}]{snehe puṣpaphala°}
  \rdg[wit={U1}]{snehapuṣpaṃ phala}
  \rdg[wit={E}]{srehapuṇyaphalaṃ}}
\app{\lem[wit={ceteri}]{bīje}
  \rdg[wit={B,L}]{bīja}
  \rdg[wit={N2}]{vīje}}
\app{\lem[wit={ceteri}]{vistāro}
  \rdg[wit={N1,D}]{vistārā}}
\app{\lem[wit={E,P,N1,N2,U2}]{'yaṃ}
  \rdg[wit={B,L}]{ya}
  \rdg[wit={U1}]{yaḥ}
  \rdg[wit={D}]{yasya}}
\app{\lem[wit={ceteri}]{svabhāvataḥ}
  \rdg[wit={B,L}]{svabhāvatāḥ}
  \rdg[wit={D}]{bhāvataḥ}}/}\\
\tl{\app{\lem[wit={ceteri}]{tathāsau}
    \rdg[wit={B}]{yathāsau}
    \rdg[wit={P}]{tathāpasau}}
  \app{\lem[wit={ceteri}]{nirmalo}
    \rdg[wit={P}]{nirmalau}}
nityo nirvikāro niraṃjanaḥ\dd{}5\hskip-2pt\dd{}}
\end{tlg}
%------------------------------
%eko  nekaḥ  svayaṃbhūś ca dhāmnā ca    bahudhā sthitaḥ/   paṃcatattvamanobuddhi-māyāhaṃkāravikriyāḥ //5//   \E
%eko  nekaḥ  svayaṃbhūś ca svadhāmnā    bahudhā sthitāḥ    paṃcatatvamanobuddhir māyāhaṃkāravikriyāḥ   6     \P
%eko  neka   svayaṃbhūś ca dhāmnāya     bahudhā sthitaḥ//  paṃcatatvamanobuddhi--māyāhaṃkāravikriyā  //      \B
%eko  nekaḥ  svayaṃbhūś ca svadhābhāva  bahudhā sthitāḥ//  paṃcatatvamanobuddhi--māyāhaṃkāravikriyā  //      \L
%eko  nekaḥ  svayaṃbhuś ca svayāṃmnā    bahudhā sthitaḥ/   paṃcatatvamanobuddhir māyāhaṃkāravikriyā  //6//   \N1
%eko  nekaḥ  svayaṃbhaś ca svadhā...ṣ   bahudhā sthitāḥ//  paṃcatatvamanobuddhir māyāhaṃkāravikriyā  //6//   \D
%eko  neka   svayaṃbhūś ca svadhāmnāva  bahudhā sthitaḥ//  paṃcatatvamanobuddhir māyāhaṃkāravikriyā  //6//   \N2
%yeko naika/ svayaṃbhūtyā  svabhāvā     bahudhā sthitaḥ    paṃcatatvamanobuddhir māyāhaṃkāravikriyāḥ   6     \U1
%eko  naiko  svayaṃbhūś ca svadhāmnā    bahudhā sthitaḥ//  paṃcatatvamanobuddhir māyāhaṃkāravikriyā  //6//   \U2
%------------------------------
%Eins, nicht eins und aus sich selbst heraus seiend durch das eigene Walten und Wirken mannigfach existierend,
%[als] fünf Prinzipien (\textit{tattva}), welche da sind: denkender Verstand (\textit{manas}), Intellekt (\textit{buddhi}), Illusion (\textit{māya}), Individuation (\textit{ahaṃkāra}) und Modifikationen (\textit{vikriyā}). 
%------------------------------
%One, not one and self-existing, existing in manifold ways through its own rule and work,
%[as] five principles (\textit{tattva}) which are: thinking mind (\textit{manas}), intellect (\textit{buddhi}), illusion (\textit{māya}), individuation (\textit{ahaṃkāra}) and modifications ( \textit{vikriya}).
%------------------------------
\note[type=source, labelb=139, lem={eko}]{Ysv (PT): eko 'nekaḥ khayaṃ bhūyān sādhanād bahudhā sthitaḥ | pañcatattvamayo buddhimāyāhaṅkāravikriyaḥ |}
\begin{tlg}
  \tl{
\app{\lem[wit={ceteri}]{eko}
  \rdg[wit={U1}]{yeko}}
\app{\lem[type=emendation, resp=egoscr]{naikaḥ}
  \rdg[wit={U1}]{\korr naika}
  \rdg[wit={U2}]{naiko}
  \rdg[wit={ceteri}]{nekaḥ}
  \rdg[wit={B,N2}]{neka}}
\app{\lem[wit={ceteri}]{svayaṃbhūś-ca}
  \rdg[wit={U1}]{svayaṃbhūtyā}}
\app{\lem[wit={P,U2}]{svadhāmnā}
  \rdg[wit={E}]{dhāmnā ca}
  \rdg[wit={B}]{dhāmnāya}
  \rdg[wit={L}]{svadhābhāva}
  \rdg[wit={N1}]{svayāṃmnā}
  \rdg[wit={D}]{svadhā..ṣa}
  \rdg[wit={N2}]{svadhāmnāva}
  \rdg[wit={U1}]{svabhāvā}}
bahudhā
\app{\lem[wit={P,L,D}]{sthitāḥ}
  \rdg[wit={ceteri}]{sthitaḥ}}/}\\
\tl{paṃcatattvamano\app{\lem[wit={E,P,L},alt={°buddhi°}]{buddhi}
    \rdg[wit={ceteri}]{°buddhir}
}māyāhaṃkāra\app{\lem[wit={ceteri},alt={°vikriyā}]{vikriyā}
  \rdg[wit={E,P,U1}]{°vikriyāḥ}}\dd{}6\hskip-2pt\dd{}}
\end{tlg}
%------------------------------ 
%evaṃ daśavidhaṃ viśvaṃ lokālokasavistaram/   eka  eva na cānyo sti yo jānāti sa tattvavit//6// \E
%evaṃ daśavidhaṃ viśvaṃ lokālokasavistaraṃ    eka  eva na cānyo sti yo jānāti sa tatvavit 6 \P
%evaṃ daśavidhā  viśvaṃ lokālokasavistaraṃ//  eka  eva na cānyā sti yo jānāti sa tatvavit// \B
%evaṃ daśavidhā  viśvaṃ lokālokasavistaraṃ//  eka  eva na cānyo sti yo jānāti sa tatvavit// \L
%evaṃ daśavidhaṃ viśvaṃ lokālokasavistarāṃ/   eka  eva na cānyo sti yo nānāti sa tatvavit//7// \N1
%evaṃ daśavidhaṃ viśvaṃ lokālokasavistaraṃ//  eka  eva na cānyo sti yo jānāti sa tatvavit//7// \D
%evaṃ daśavidhā  viśvaṃ lokālokasavistaraṃ//  eka  eva na cānyo sti yo jānāti sa tatvavit//7// \N2
%evaṃ daśavidha--viśvaṃ lokālokasavistaraṃ    eka yeva na cānyo sti yo jānāti sa tatvavit 7 \U1
%evaṃ daśavidhaṃ viśvaṃ lokāloke savistaraṃ// ekam eva na cānyo sti yo jānāti sa tatvavit//7// \U2 %%%409.jpg 
%------------------------------
%Auf diese Weise durchdringen die zehn Variationen die Welt und die Nicht-Welt im vollen Umfang.  
%Nur das Eine ist und nicht etwas anderes: Wer das weiß ist ein Kenner der Realität.  
%------------------------------
%In this way, the ten variations fully permeate the world and the non-world.
%Only one thing is and not something else: Whoever knows this is a connoisseur of reality.
%------------------------------
\note[type=source, labelb=140, lem={daśavidhā}]{Ysv (PT): evaṃ bahuvidhaṃ viśvaṃ lokālokasuvistaram | ekam eva na cānvo 'sti yo jānāti sa tattvavit |}
\begin{tlg}
   \tl{
     evaṃ
     \app{\lem[wit={B,L,N2}]{daśavidhā viśvaṃ}
       \rdg[wit={E,P,N1,D,U2}]{daśavidhaṃ viśvaṃ}
       \rdg[wit={U1}]{daśavidhaviśvaṃ}}
     \app{\lem[wit={ceteri}]{lokālokasavistaram}  
       \rdg[wit={N1}]{°savistarāṃ}
       \rdg[wit={U2}]{°loke savistaraṃ}}/}\\
   \tl{\app{\lem[wit={ceteri}]{eka}
       \rdg[wit={U2}]{ekam}}
       \app{\lem[wit={ceteri}]{eva}
         \rdg[wit={U1}]{yeva}}
       na cānyo 'sti yo jānāti sa tattvavit \dd{}7\hskip-2pt\dd{}}\\
   \end{tlg}
    \begin{prose}     
%------------------------------
%pṛthvīvanaspatiparvatādisthārarūpaḥ         saṃsāra---manuṣyahastyaśvapakṣītyādiko    jaṃgamarūpaḥ   saṃsāraḥ// \E
%pṛthvīvanaśpatiparvatādisthāvararūpaḥ       saṃsāraḥ  manuṣyahastyaś ca pakṣītyādiko  jaṃgamarūpaḥ   saṃsāraḥ \P
%pṛthvīvanaspatīparvatādisthāvararūpā        saṃsāraḥ/ manuṣyahasteśvapakṣītyādiko     jaṃgamarūpaḥ   saṃsāraḥ// \B
%pṛthvīvanaspatiparvatādisthāvararūpā        saṃsāraḥ  manuṣyahasteśvapakṣītyādiko     jaṃgamarūpā    saṃsāraḥ// \L
%pṛthvīvanaspatīparvvate tyādisthāvararūpaḥ  saṃsāraḥ  manuṣyahastīaśvapakṣītyādiko    jaṃgamarūpaḥ   saṃsāraḥ// \N1
%pṛthvīvanaspatīparvato tyādisthāṃvararūpaḥ  saṃsāraḥ  manuṣyahastīaśvapakṣītyādiko    jaṃgamaḥ rūpaḥ saṃsāraḥ// \D
%pṛthvīvanaspatiparvate 'thyādisthāvararūpa  saṃsāraḥ  manuṣyahastipakṣītyādiko        jaṃgamarūpaḥ   saṃsāraḥ// \N2
%pṛthivīvanaspatīparvate iyādisthāvararūpaḥ  saṃsāra---manuṣyahastiasvapakṣītyādiko    jagadrūpaḥ     saṃsāro \U1
%pṛthvīvanaspatiparvatādisthāvararūpaḥ       saṃsāraḥ//manuṣyahasttyaś ca pakṣītyādiko jaṃgamarūpaḥ   saṃsāraḥ//8// \U2
%------------------------------
%Der Geburtenkreislauf ist die Erscheinung der Pflanzenwelt, der Berge, der Bäume, der Erde etc. Der Geburtenkreislauf ist die Erscheinung der Lebewesen beginnend mit Vögeln, Pferden, Elefanten und Menschen. 
%------------------------------
%The cycle of birth is the appearance of the plant world, mountains, trees, earth etc. The cycle of birth is the appearance of living beings beginning with birds, horses, elephants and humans. 
%------------------------------
\note[type=source, labelb=141, lem={saṃsāraḥ}]{Ysv (PT): sthāvarāḥ parvatādyā hi jaṅgamāḥ khecarādayaḥ | jaṅgamasthāvarākāraḥ saṃsāraḥ syāt sa īśvaraḥ |}
\app{\lem[wit={ceteri},alt={pṛthvī°}]{pṛthvī}
        \rdg[wit={U1}]{pṛthivī°}
      }\app{\lem[wit={E,N2,U2},alt={°vanaspati°}]{vanaspati}
        \rdg[wit={P}]{vanaś°}
        \rdg[wit={B,L,N1,D,U1}]{°patī°}
      }\app{\lem[wit={P,B,L,U2}, alt={°parvatādisthāra°}]{parvatādisthāvara}
        \rdg[wit={E}]{°parvatādisthāra°}
        \rdg[wit={N1}]{°parvvate tyādisthāvara°}
        \rdg[wit={N2}]{°parvate 'thyādisthāvara°}
        \rdg[wit={D}]{°parvato tyādisthāṃvara°}
        \rdg[wit={N2}]{°parvate 'thyādisthāvara°}
        \rdg[wit={U1}]{°parvate iyādisthāvara°}
      }\app{\lem[wit={ceteri}]{rūpaḥ} \rdg[wit={L,B}]{rūpā}
        \rdg[wit={N2}]{rūpa}} \app{\lem[wit={ceteri}]{saṃsāraḥ}
        \rdg[wit={E,U1}]{saṃsāra°}}/
      manuṣya\app{\lem[wit={B,L},alt={°hasteśvapakṣīty ādiko}]{hasteśvapakṣīty\skp{-}ādiko}
          \rdg[wit={E}]{°hasty aśvapakṣīty ādiko}
          \rdg[wit={N1,D}]{°hastīaśvapakṣīty ādiko}
          \rdg[wit={N2}]{°hastipakṣīty ādiko}
          \rdg[wit={U1}]{°hastiasvapakṣīty ādiko}
          \rdg[wit={U2}]{°hasttyaś ca pakṣīty ādiko}}
        \app{\lem[wit={ceteri}]{jaṃgamarūpaḥ}
          \rdg[wit={L}]{°rūpā}
          \rdg[wit={D}]{jaṃgamaḥ rūpaḥ}
          \rdg[wit={U1}]{jagad°}}
        \app{\lem[wit={ceteri}]{saṃsāraḥ}
          \rdg[wit={U1}]{saṃsāro}}/
%------------
%atha ca   yo  dṛṣṭiviṣayaḥ  sa dṛśya  ucyate/  yo dṛṣṭyā na vīkṣyate sa adṛśya ity  ucyate/ \E
%atha ca   yo  dṛṣṭiviṣayaḥ  sa dṛśya  ucyate   yo dṛṣṭyā na vīkṣyate sa adṛśya ity  ucyate  %%%7641.jog
%atha ca// yo  daṣṭiviṣayaḥ  sa dṛśya  ucyate// yo dṛṣṭyā na vīkṣyate sa adṛśya ty   ucyate// \B
%atha ca   yo ddṛṣṭiviṣayaḥ  sa dṛśya  ucyate// yo dṛṣṭyā na vīkṣyate sa adṛśye ty   ucyate... \L
%atha ca   ya ddṛṣṭiviṣayaḥ  sa dṛśyad ucyate   yo dṛṣṭyā na vīkṣyate sa adṛśya ity  ucyate// \N1
%atha vā   ya dārṣṭiviṣayaḥ  sa dṛśya  ucyate/  yo dṛṣṭyā na vīkṣyate sa adṛśya ity  ucyate// \D
%atha ca   ya  drṣṭiviṣayaḥ  sa dṛśya  ucyate/  yo dyā    na vīkṣyate sa adṛśya śaty ucyate/ \N2
%atha ca   yaḥ drṣṭiviṣayaḥ  sa dṛśy---ucyate   yo dṛṣṭvā na vīkṣyate sa adṛśya ity  ucyate \U1
%atha ca   yo  dṛṣṭiviṣayaḥ  sa dṛśya  ucyate// yo dṛṣṭyā na vīkṣyate sa adṛśya ity  ucyate// \U2
%------------------------------
%Und dann, wer einer ist, der ein [Sinnes]objekt des Sehens ist, der wird gesagt, ist sichtbar. Wer nicht durch das Sehen gesehen wird, der wird gesagt ist unsichtbar. 
%------------------------------
\note[type=source, labelb=142, lem={drṣṭiviṣayaḥ}]{Ysv (PT): svabhāvalīlayā bhāti śūnye'sau śūnyabuddhitaḥ | yad dṛṣṭaṃ viṣayaṃ vastu tad dṛśyam iti kathyate | yo dṛṣṭātītaḥ so'dṛśyas tadā dṛṣṭaṃ hi manyate | svatanūbhedam evan tu saṃsāraṃ duḥkhasaṅkulam |}
atha
      \app{\lem[wit={ceteri}]{ca}
        \rdg[wit={D}]{vā}}
      \app{\lem[wit={ceteri}]{yo}
        \rdg[wit={U1}]{yaḥ}
        \rdg[wit={N1,N2,D}]{ya}}
      \app{\lem[wit={ceteri}]{dṛṣṭi}
        \rdg[wit={L,N1}]{ddṛṣṭi}
        \rdg[wit={B}]{daṣṭi}
        \rdg[wit={D}]{dārṣṭi}
}viṣayaḥ sa
\app{\lem[wit={ceteri}]{dṛśya}
  \rdg[wit={N1}]{dṛśyad}
  \rdg[wit={U1}]{dṛṣy°}}
ucyate/
yo
\app{\lem[wit={ceteri}]{dṛṣṭyā}
  \rdg[wit={N2}]{dyā}}
na vīkṣyate sa adṛṣya
\app{\lem[wit={ceteri},alt={ity}]{i\skp{ty-u}}
  \rdg[wit={L,B}]{ty}
  \rdg[wit={N2}]{śaty}
}\skm{ty-u}cyate/
%------------------------------
%evaṃ saṃsārasya svātmano  bhedaṃ dūrīkṛty---aikam eva darśanaṃ sa eva jñānayogaḥ/   \E
%evaṃ saṃsāra----svātmano  bhedaṃ dūrīkṛtya  aikyena   darśanaṃ        jñānayogaḥ    \P
%evaṃ saṃsārasya svātmano  bheda--dūrīkṛtya  aikyona   darśanaṃ        jñānayogaḥ/   \B
%evaṃ saṃsāra----svātmano  bhedaṃ dūrīkṛtya  aikyona   darśanaṃ        jñānayogaḥ... \L
%evaṃ saṃsārasya svātmanaḥ bhedāṃ dūrīkṛtya  ekyena    darśanaṃ        jñānayogaḥ//  \N1
%evaṃ saṃsārasya svātmanaḥ bhedāṃ dūrīkṛtya  ekyena    darśanaṃ        jñānayogaḥ/   \D
%evaṃ saṃsārasya svātmanaḥ bhedaṃ dūrīkṛtya  ekena     darśanaṃ        jñānayogaḥ/   \N2
%evaṃ saṃsārasya svātmanaḥ bhedaṃ dūrīkṛtya  ekānta    darśanaṃ        jñānayogaḥ    \U1
%evaṃ saṃsāra....svātmanoḥ bhedaṃ dūrīkṛtyaṃ ekye?     darśanaṃ        jñānayoga     \U2
%------------------------------
%In this way the view of separation of one's own self which is subjected to transmigration is to be removed by means of [applying the view of] unity. Only this is Jñānayoga.  
%------------------------------
evaṃ
       \app{\lem[wit={ceteri}]{saṃsārasya}
         \rdg[wit={P,L,U2}]{saṃsāra°}}
       \app{\lem[wit={E,P,B,L}]{svātmano}
         \rdg[wit={N1,D,N2,U1}]{svātmanaḥ}
         \rdg[wit={U2}]{svātmanoḥ}}
       \app{\lem[wit={ceteri}]{bhedaṃ}
         \rdg[wit={B}]{bheda}
         \rdg[wit={D,N1}]{bhedāṃ}}
\app{\lem[wit={U2}]{dūrīkṛtyaṃ}
  \rdg[wit={ceteri}]{°kṛtya}
  \rdg[wit={E}]{°kṛty}}
\app{\lem[wit={P}]{aikyena}
  \rdg[wit={E}]{aikam eva}
  \rdg[wit={P,B,L}]{aikyona}
  \rdg[wit={N1,D}]{ekyena}
  \rdg[wit={N2}]{ekena}
  \rdg[wit={U1}]{ekānta}
  \rdg[wit={U2}]{ekye}}
darśanaṃ
\app{\lem[wit={E}]{sa eva}
  \rdg[wit={ceteri}]{\om}}
\app{\lem[wit={ceteri}]{jñānayogaḥ}
  \rdg[wit={U2}]{jñānayoga}}/ 
%------------------------------
%tasya         kāraṇāt kālaḥ śarīranāśaṃ na karoti/ \E
%tasya         kāraṇāt kālaḥ śarīranāśaṃ na karoti/ \P
%tasya         karaṇāt kālaḥ śarīranāśaṃ na karoti// \B
%tasya         karaṇāt kālaḥ śarīranāśaṃ na karoti... \L
%tasya         karaṇāt kālaḥ śarīranāśaṃ na karoti// \N1
%tasya         karaṇāt kālaḥ śarīranāśaṃ na karoti// \D
%tasya         karaṇāt kālaḥ śarīranāśaṃ    karoti/ \N2
%gatasya dhyānakaraṇāt kālaḥ śarīranāśaṃ na karoti 8 \U1
%tasya         karaṇāt kālaśarīranāśanaṃ    karoti// \U2
%------------------------------
%Because of the execution of this time does not destroy the body. 
%------------------------------
\app{\lem[wit={ceteri}]{tasya}
  \rdg[wit={U1}]{gatasya}}
\app{\lem[wit={ceteri}]{kāraṇāt}
  \rdg[wit={U1}]{dhyānakaraṇāt}}
\app{\lem[wit={ceteri}]{kālaḥ}
  \rdg[wit={U1}]{kāla°}}
śarīranāśaṃ
\app{\lem[wit={ceteri}]{na}
  \rdg[wit={N2,U2}]{\om}}
karoti\dd{}
    \end{prose}
  \end{ekdosis}
  \ekdpb*{}
%%%%%%%%%%%%%%%%%%%%%%%%%%%%%%%%%%%%%%%%%%
%%%%%%%%%%%%%%%%%%%%%%%%%%%%%%%%%%%%%%%%%%
%%%%%%%%PAGEBREAK%%%%%%%PAGEBREAK%%%%%%%%%
%%%%%%%%%%%%%%%%%%%%%%%%%%%%%%%%%%%%%%%%%%
%%%%%%%%%%%%%%%%PAGEBREAK%%%%%%%%%%%%%%%%%
%%%%%%%%%%%%%%%%%%%%%%%%%%%%%%%%%%%%%%%%%%
%%%%%%%%PAGEBREAK%%%%%%%PAGEBREAK%%%%%%%%%
%%%%%%%%%%%%%%%%%%%%%%%%%%%%%%%%%%%%%%%%%%
%%%%%%%%%%%%%%%%%%%%%%%%%%%%%%%%%%%%%%%%%%
%%%%%%%%%%%%%%%%%%%%%%%%%%%%%%%%%%%%%%%%%%
%%%%%%%%%%%%%%%%%%%%%%%%%%%%%%%%%%%%%%%%%%
%%%%%%%%PAGEBREAK%%%%%%%PAGEBREAK%%%%%%%%%
%%%%%%%%%%%%%%%%%%%%%%%%%%%%%%%%%%%%%%%%%%
%%%%%%%%%%%%%%%%PAGEBREAK%%%%%%%%%%%%%%%%%
%%%%%%%%%%%%%%%%%%%%%%%%%%%%%%%%%%%%%%%%%%
%%%%%%%%PAGEBREAK%%%%%%%PAGEBREAK%%%%%%%%%
%%%%%%%%%%%%%%%%%%%%%%%%%%%%%%%%%%%%%%%%%%
%%%%%%%%%%%%%%%%%%%%%%%%%%%%%%%%%%%%%%%%%%
%%%%%%%%%%%%%%%%%%%%%%%%%%%%%%%%%%%%%%%%%%
%%%%%%%%%%%%%%%%%%%%%%%%%%%%%%%%%%%%%%%%%%
%%%%%%%%PAGEBREAK%%%%%%%PAGEBREAK%%%%%%%%%
%%%%%%%%%%%%%%%%%%%%%%%%%%%%%%%%%%%%%%%%%%
%%%%%%%%%%%%%%%%PAGEBREAK%%%%%%%%%%%%%%%%%
%%%%%%%%%%%%%%%%%%%%%%%%%%%%%%%%%%%%%%%%%%
%%%%%%%%PAGEBREAK%%%%%%%PAGEBREAK%%%%%%%%%
%%%%%%%%%%%%%%%%%%%%%%%%%%%%%%%%%%%%%%%%%%
%%%%%%%%%%%%%%%%%%%%%%%%%%%%%%%%%%%%%%%%%%
\begin{ekdosis}
    \ekddiv{type=ed}
          \bigskip
        \centerline{\textrm{\small{[Division of the Inherent Nature]}}}
          \bigskip
          \begin{prose}
            \noindent
%------------------------------
%idānīṃ tasya---bhedaḥ    kathyate/   \E
%idānīṃ svabhāvabhedaḥ kathyate    \P
%idānī  svābhāvabhedaḥ kathyate//  \B
%idānīṃ svābhāvabhedaḥ kathyate//  \L
%idānīṃ svabhāvabhedaṃ kathyate//  \N1
%idānīṃ svabhāvabhedaṃ kathyate//  \D
%idānīṃ svabhāvabheda  kathyate//  \N2
%idānīṃ svabhāvabhedāḥ kathyate    \U1
%idānīṃ svabhāvabhedaḥ kathyate//  \U2
%------------------------------
%Now the division of the inherent nature is described. 
%------------------------------  
\note[type=source, labelb=143, lem={}]{Ysv (PT): svabhāvabhedam etat śṛṇu devi prayatnataḥ |}
\app{\lem[wit={ceteri}]{idānīṃ}
  \rdg[wit={B}]{idānī}}
\app{\lem[wit={ceteri},alt={svabhāva°}]{svabhāva}
  \rdg[wit={B,L}]{svābhāva°}
  \rdg[wit={E}]{tasya}
}\app{\lem[wit={D,N1},alt={°bhedaṃ}]{bhedaṃ}
  \rdg[wit={N2}]{°bheda}
  \rdg[wit={ceteri}]{°bhedaḥ}}
kathyate/
%------------------------------  
%yathā vaṭabījam/ vaṭarūpeṇa pariṇataṃ    sat    daśadhā    bhedaṃ svabhāvata eva prāpnoti/  \E %%%[P.27]
%yathā vaṭabījaṃ  vaṭarūpeṇa pariṇāte     sat    dṛśadhā    bhedaṃ svabhāvata eva prāpnoti   \P
%yathā vaṭabījena rūpeṇa     pariṇamate/  śata   daśadhā    bhedaṃ svābhāva   eva prāpnotī// \B
%yathā vaṭabījena rūpeṇa     pariṇamate   śata   daśadhā    bhedaṃ svābhāva   eva prāpnotī// \L
%yathā vaṭabījaṃ  vaṭarūpeṇa pariṇataṃ//  satṛ   daśadhā    bhedaṃ svabhāvata eva prāpnoti/  \N1
%yathā vaṭabījaṃ  vaṭarūpeṇa pariṇataṃ/   sa     daśadhā    bhedaṃ svabhāvata eva prāpnoti// \D
%yathā vathabījaṃ vaṭarūpeṇa pariṇataṃ/   sa tu  daśadhā    bhedaṃ svabhāvata eva prāpnoti/  \N2
%yathā vaṭabījaṃ  vaṭarūpeṇa pariṇataṃ    sa tat daśadhā    bhedaṃ svabhāvata eva prāpnotī   \U1
%yathā vaṭabīja---vaṭarūpeṇa pariṇamate// sa     dasat                            prāpnoti// \U2
%------------------------------
%Wie der Samen des Banyan-Baumes zur Gestalt des Banyan-Baumes heranreift und er sich aufgrund seiner eigenen inhärenten Natur so eine zehnfachen Auftheilung erreicht. [Nämlich]: 
%------------------------------
%Just as the seed of the banyan tree ripens into the shape of the banyan tree, and by its own inherent nature attains such a tenfold division. [Namely]:
%------------------------------
\note[type=philcomm, labelb=144, lem={daśadhā}]{Remarkably, the tenfold division of \textit{svabhāva} is missing in the Ysv and SSP.}
yathā
\app{\lem[wit={ceteri},alt={vaṭa°}]{vaṭa}
  \rdg[wit={N2}]{vatha°}
}\app{\lem[wit={D,P,N1,N2,U1},alt={°bījaṃ}]{bījaṃ}
        \rdg[wit={E}]{°bījam}
        \rdg[wit={U2}]{°bīja°}
        \rdg[wit={B,L}]{°bījena}}
      \app{\lem[wit={ceteri}]{vaṭarūpeṇa}
        \rdg[wit={L,B}]{rūpeṇa}}
      \app{\lem[wit={B,L,U2}]{pariṇamate}
        \rdg[wit={P}]{pariṇāte}
        \rdg[wit={ceteri}]{pariṇataṃ}}
      \app{\lem[wit={U1}]{sa tat}
        \rdg[wit={N2}]{sa tu}
        \rdg[wit={N1}]{satṛ}
        \rdg[wit={E,P}]{sat}
        \rdg[wit={B,L}]{śata}
        \rdg[wit={D,U2}]{sa}}
      \app{\lem[wit={ceteri}]{daśadhā}
        \rdg[wit={P}]{dṛśadhā}
        \rdg[wit={U2}]{dasat}}
      \app{\lem[wit={ceteri}]{bhedaṃ}
        \rdg[wit={U2}]{\om}}
      \app{\lem[wit={ceteri}]{svabhāvata}
        \rdg[wit={B,L}]{svabhāva}
        \rdg[wit={U2}]{\om}}
      \app{\lem[wit={ceteri}]{eva}
        \rdg[wit={U2}]{\om}}
      \app{\lem[wit={ceteri}]{prāpnoti}
        \rdg[wit={B,L,U1}]{prāpnotī}}/
%------------------------------ %%%%STEMMA POINT!!!!
%mūlāṃkura---tvagdaṇḍaśākhā--kalikāpallavapuṣpaphalasnehā                  iti daśabhedān    prāpnoti// \E
%mūla aṃkura-tvakdaṃdaśākhā----kilpikāpallavā puṣpaphalasneha              iti daśabhedān    prāpnotīti \P  %%%7642.jpg
%mūlaṃ aṃkuratvakdaṃdaśākhā----kilakālapallavā// vistāroyaṃ svābhāvataḥ    iti daśabhedān    prāpnoti// \B DSCN7160 Z. 4
%mūlaṃ aṃkuratvakdaṃdaśākhā----kilāpallavā// vistāroyaṃ svābhāvataḥ//      iti daśabhedān    prāpnoti... \L
%mūlāṃ aṃkuratvakdaṃḍaśākhāṃ kalikāpallavapuṣpaphalasneha//                iti bhedo daśadhā prāpnoti// \N1
%mūlāṃkura---tvakdaṇdaśākhāṃ kalikāpallavapuṣpaphalasnehaṃ                 iti bhedo daśadhā prāpnoti// \D
%mūlāṃkura---tvakdaṇdaśākhāṃ kalikāpallavapuṣpaphalasneha/                 iti bhedo daśadhā prāpnoti// \N2
%mūlāṃaṃkura-tvakdaṇdaśākhā--kalikāpallavapuṣpaphalasneha                  iti bhedo daśadhā prāpnoti \U1
%\om                                                                                \U2
%------------------------------
%"Wurzel, Spross, Rinde, Ast, Zweig, Knospe, die sich entfaltende Blüte, Blüte, Frucht und Nektar." Die Auftheilung erreicht [diese] zehn Teile. 
%------------------------------
%"Root, shoot, bark, branch, twig, bud, the unfolding flower, flower, fruit and nectar." The division reaches [those] ten parts.
%------------------------------
\app{\lem[wit={E}]{mūlāṃkuratvagdaṇḍaśākhākalikāpallavapuṣpaphalasnehā}
          \rdg[wit={P}]{mūla aṃkuratvakdaṃdaśākhākilpikāpallavā puṣpaphalasneha}
          \rdg[wit={B}]{mūlaṃ aṃkuratvakdaṃdaśākhākilakālapallavā || vistāroyaṃ svābhāvataḥ}
          \rdg[wit={L}]{mūlaṃ aṃkuratvakdaṃdaśākhākilāpallavā || vistāroyaṃ svābhāvataḥ ||}
          \rdg[wit={N1}]{mūlāṃ aṃkuratvakdaṃḍaśākhāṃ kalikāpallavapuṣpaphalasneha ||}
          \rdg[wit={N2}]{mūlāṃkuratvakdaṇdaśākhāṃ kalikāpallavapuṣpaphalasneha|}
          \rdg[wit={D}]{mūlāṃkuratvakdaṇdaśākhāṃ kalikāpallavapuṣpaphalasnehaṃ}
          \rdg[wit={U1}]{mūlāṃaṃkuratvakdaṇdaśākhākalikāpallavapuṣpaphalasneha}
          \rdg[wit={U2}]{\om}}
        \app{\lem[wit={ceteri}]{iti}
          \rdg[wit={U2}]{\om}}
        \app{\lem[wit={N1,D,N2,U1}]{bhedo daśadhā}
          \rdg[wit={E,P,L,B}]{daśabhedān}
          \rdg[wit={U2}]{\om}}
        \app{\lem[wit={ceteri}]{prāpnoti}
          \rdg[wit={P}]{prāpnotīti}
          \rdg[wit={U2}]{\om}}/
%------------------------------
%yathā nirmalo  nirvikāraḥ niraṃjana   eka  etādṛśa  ātmā svabhāvād eva/ pṛthivyaptejovāyvākāśamanobuddhimāyāvikārarūpabhedān    prāpnoti/ \E
%tathā nirmalaḥ nirvikāraḥ niraṃjanaḥ  eka  etādṛśa  ātmasvabhāvād eva   pṛthvyetetejo vādvyākāśamanobuddhimāyāvikārarūpabhedāt  prāpnoti \P
%tathā nirmalo  nirvikāraḥ niraṃjanaḥ  eka  etādṛśa  ātmasvabhāvād eva   pṛthvyāpatejovādvyākāśamanobuddhimāyāvikārarūpabhedāna  prāpnoti// \B
%tathā nirmalo  nirvikāraḥ niraṃjanaḥ/ eka  etādṛśa  ātmasvabhāvād eva   pṛthvyāpatejovāybākāśamanobuddhimāyāvikārarūpābhedāna   prāpnoti  \L
%tathā nirmalaḥ nirvikāraḥ niraṃjanaḥ  ekaḥ etādṛśaḥ ātmasvabhāvād eva   pṛthvyāpatejovāybākāśamanobuddhimāyāvikārarūpābhedān    prāpnoti/ \N1
%tathā nirmalaḥ nirvikāraḥ niraṃjanaḥ  eka  etādṛśaḥ ātmasvabhāvād eva   pṛthvīpate/ jīvīkāśamanobuddhir māyāvikārarūpabhedāt    prāpnoti \D
%tathā nirmalaḥ nirvikāraḥ niraṃjanaḥ  ekaḥ etādṛśaḥ ātmasvabhāvād eva   pṛthvīpate/ jīvīkāśamanobuddhir māyāvikārarūpabhedāt    prāpnoti/ \N2
%tathā nirmalaḥ nirvikāraḥ niraṃjanaḥ  ekaḥ etādṛśaḥ ātmascabhāvād eva   pṛthakte jīvāyuvākāśamanobuddhir māyāyāvikārarūpabhedāt prāpnoti \U1 %%%275.jpg
%yathā nirmalaḥ nirvikāraḥ niraṃjanaḥ  eka  etādṛśa  ātmasvabhāvād eva// pṛthvyaptejovāyyākāśa// manobuddhimayāvikārarūpabhedān  prāpnoti/ \U2
%------------------------------
%In dieser Weise erreicht auch das reine, unveränderliche, makellose, eine solche [Auftheilung] eben aufgrund der inhärenten Natur des Selbst. [Nämlich] die Aufteilung "Erde, Wasser, Feuer, Wind, Raum, Geist, Intellektekt, Illusion, Umwandlungen und Gestalt".
%------------------------------
%In this way, the pure, unchanging, unblemished, attains such [division] precisely because of the inherent nature of the self. [Namely] the division "Earth, Water, Fire, Wind, Space, Mind, Intellect, Illusion, Transformations and Form".
%------------------------------
        \app{\lem[wit={ceteri}]{tathā}
            \rdg[wit={E,U2}]{yathā}}
          \app{\lem[wit={E,B,L}]{nirmalo}
            \rdg[wit={ceteri}]{nirmalaḥ}}
          nirvikāraḥ
          \app{\lem[wit={E}]{niraṃjana}
            \rdg[wit={ceteri}]{niraṃjanaḥ}}
          \app{\lem[wit={ceteri}]{eka}
            \rdg[wit={N1,N2,U1}]{ekaḥ}}
          \app{\lem[wit={E}]{etādṛśa}
            \rdg[wit={N1,N2,U1}]{etādṛśaḥ}}
          \app{\lem[wit={ceteri}]{ātmasvabhāvād}
            \rdg[wit={E}]{ātmā°}}
          eva
          \app{\lem[wit={N1}]{pṛthvyāpatejovāybākāśamanobuddhimāyāvikārarūpābhedān}
            \rdg[wit={E}]{pṛthivyap°}
            \rdg[wit={B,L}]{°bhedāna}
            \rdg[wit={P}]{pṛthvyetetejovādvyākāśa°}
            \rdg[wit={D,N2}]{pṛthvīpate | jīvīkāśamanobuddhir māyāvikārarūpabhedāt}
            \rdg[wit={U1}]{pṛthakte jīvāyuvākāśamanobuddhir māyāyāvikārarūpabhedāt}
            \rdg[wit={U2}]{pṛthvyaptejovāyyākāśa || manobuddhimayāvikārarūpabhedā}}
          prāpnoti/
%------------------------------
%jñānayogaprabhāvād     eka eva  ātmā iti niścayo bhavati// \E
%jñānayogaḥ prabhāvād   eka eka  ātmā iti niścayo bhavati \P
%jñānayogaḥ// prabhāvād eka eka  ātmā iti niścayā bhavatī// \B
%jñānayogaḥ// prabhāvād eka eka  ātmā iti niścayo bhavati// \L
%jñānayogaprabhāvāt     eka eva  ātmā iti niścayo bhavati// \N1
%jñānayogaprabhāvāt     eka eva  ātmā iti niścayo bhavati// \D
%jñānayogaprabhāvāt     eka eva  ātmā iti niścayo bhavati// \N2
%jñānayogaprabhāvāt tu  eka yeva ātmā iti niścayo bhavati \U1
%jñānayogaprabhāvād     eka eva  ātmā iti niścayo bhavati// \U2
%------------------------------
%Aufgrund der Macht von Jñānayoga entsteht so die Gewissheit "Das Selbst ist wahrlich eins".
%------------------------------
%Because of the power of jñānayoga, there arises the certainty that "The Self is verily one."    
%------------------------------
\app{\lem[wit={E,U2}, alt={jñānayogaprabhāvād}]{jñānayogaprabhāvā\skp{d-e}}
  \rdg[wit={N1,D,N2,U1}]{°bhavāt}
  \rdg[wit={L,B}]{jñānayogaḥ || prabhāvād°}
  \rdg[wit={P}]{jñānayogaḥ prabhāvād}
}\skm{d-e}ka
\app{\lem[wit={ceteri}]{eva}
  \rdg[wit={P,B,L}]{eka}
  \rdg[wit={U1}]{yeva}}
ātmā iti niścayo bhavati/
%------------------------------
%yathaikaiva   pṛthvī  kvacit komalarūpā                                                   kvacit parimalarūparahitā kvacit suvarṇarūpā   kvacid raupyarūpā    \E %%%p.28 
%yathā ekaika  pṛthvī  kvacit komalarūpā                                                                                                                       \P   
%yathā ekaika  pṛthvī  kvacit komalarūpā// kvacit manohararūpā//  kvacit parimalarūpayuktā// kvacit parimalarohitā// kvacit suvarṇarūpa                        \B
%yathā ekaika  pṛthvī  kvacit komalarūpā   kvacit manohararūpāḥ// kvacit parimalarūpayuktā// kvacit parimalarahitā// kvacit suvarṇarūpā                        \L
%yathā ekaiva  pṛthivī kvacit komalarūpa/  kvacit manoharā/       kvacit parimalarūpāyuktā// kvacit parimalarahitā/  kvacit suvarṇarūpā/  kvacit rūpyarūpā/    \N1
%yathā ekaiva  pṛthivī kvacit komalarūpa   kvacit manoharā//      kvacit parimalarūpāyuktā/  kvacit parimalarohitā   kvacit suvarṇarūpa// kvacit rūpyarūpa//   \D
%yathā ekaṃ ca pṛthivī kvacit komalarūpa   kvacit manoha?rā       kvacit parimalarūpāyuktaḥ/ kvacit parimalarohitā   kvacit suvarṇarūpā   kvacit rūpyarūpa     \N2
%yathā ekai ca pṛthivī kvacit                                                                                              khavarṇakupā   kvacit rūpyarūpā     \U1
%yathā ekaika  pṛthvī  kvacit komalarūpā// kvacit manohararūpa//  kvacit parimalarūpāyuktā/  kvacit parimalarohitā// kvacit suvarṇarūpā// kvacit rajatarūpā//  \U2
%------------------------------
%Wie irgendein bestimmter Erdboden (\textit{ekaika}) manchmal weich erscheint, manchmal schön erscheint, manchmal mit Wohlgeruch versehen ist, manchmal ohne Wohlgeruch ist, manchmal golden erscheint, manchmal silbern erscheint, ...
%
%As some particular soil (\textit{ekaika}) sometimes appears soft, sometimes appears beautiful, sometimes fragrant, sometimes unscented, sometimes golden, sometimes silver,... 
%------------------------------
\note[type=source, labelb=145, lem={pṛthivī}]{Ysv (PT): ātmano vā pṛthivyādyāḥ svabhāvaḥ kiñcid ucyate | ātmaiva pṛthivī dhātrī komalā ca kvacid dṛḍhā | kvacin manoharā sā ca vimalā ca malāmalā | durgandhā ca sugandhā ca nirgandhā gandhamohinī | svarṇarūpā dhāturūpā citrā ratnamayī parā | kvacit śvetā kvacid raktā kvacit pītā ca kṛṣṇalā | ūrvarā ūrvarā sā tu viṣāmṛtamayī sadā |}
\app{\lem[type=emendation, resp=egoscr]{yathaikaikaḥ}
  \rdg[wit={E}]{\korr yathaikaiva}
  \rdg[wit={P,B,L,U2}]{yathā ekaika}
  \rdg[wit={N1,D}]{yathā ekaiva}
  \rdg[wit={N2}]{yathā ekaṃ ca}
  \rdg[wit={U1}]{yathā ekai ca}}
\app{\lem[wit={E,P,B,L,U2}]{pṛthvī}
  \rdg[wit={ceteri}]{pṛthivī}}
kvacit
komala\app{\lem[wit={E,P,B,L,U2},alt={°rūpā}]{rūpā}
    \rdg[wit={ceteri}]{°rūpa}}\dd{}
\app{\lem[wit={ceteri}]{kvacit}
  \rdg[wit={E,P,U1}]{\om}}
\app{\lem[wit={B}]{manohararūpā}
  \rdg[wit={L}]{°rūpāḥ}
  \rdg[wit={U2}]{°rūpa}
  \rdg[wit={N1,N2,D}]{manoharā}
  \rdg[wit={E,P,U1}]{\om}}\dd{}
\app{\lem[wit={ceteri}]{kvacit}
  \rdg[wit={E,P,U1}]{\om}}
\app{\lem[wit={ceteri},alt={°parimala}]{parimala}
  \rdg[wit={E,P,U1}]{\om}
}\app{\lem[wit={B,L},alt={°rūpayuktā}]{rūpayuktā}
  \rdg[wit={N1,D}]{°rūpā°}
  \rdg[wit={N2}]{°rūpāyuktaḥ}
  \rdg[wit={E,U1}]{\om}}\dd{}
\app{\lem[wit={ceteri}]{kvacit}
  \rdg[wit={P,U1}]{\om}}
\app{\lem[wit={ceteri},alt={°parimala}]{parimala}
  \rdg[wit={E}]{°parimalarūpa°}
  \rdg[wit={P,U1}]{\om}
}\app{\lem[wit={E,L,N1},alt={°rahitā}]{rahitā}
  \rdg[wit={B,N2,U2}]{°rohitā}
  \rdg[wit={ceteri}]{\om}}\dd{}
\app{\lem[wit={ceteri}]{kvacit}
  \rdg[wit={P,U1}]{\om}}
\app{\lem[wit={E,L,N2,U2}]{suvarṇarūpā}
  \rdg[wit={B,D}]{°rūpa}
  \rdg[wit={U1}]{khavarṇakupā}
  \rdg[wit={P}]{\om}}\dd{}
\app{\lem[wit={ceteri}]{kvacit}
  \rdg[wit={P,B,L}]{\om}}
\app{\lem[wit={E}]{raupyarūpā}
  \rdg[wit={N1,U1}]{rūpyarūpā}
  \rdg[wit={D,N2}]{rūpyarūpa}
  \rdg[wit={U2}]{rajatarūpā}
  \rdg[wit={P,B,L}]{\om}}\dd{}
%------------------------------
%kvacid ratnamayī   kvacic ca śvetā                                kvacidraktā   kvacitpītā    \E %%%p.28 
%                                                                                             \P   
%kvacid ratnamaī//  kvacit śverūpā// kvacitkṛṣṇā//                 kvacidraktā/  kvacitpītā//  \B
%kvacid ratnamaī//  kvacit śvetarūpā kvacitkṛṣṇā//                 kvacidraktā// kvacitpītā//  \L
%kvacid ratnamayī/  kvacit śveta/    kvacitkṛṣṇa??/                kvacidrakta/  kvacitpītā/   \N1
%kvacid ratnamayī// kvacit śvetā//   kvacitkṛṣṇā [S8., Z.7]        kvacidrakta   kvacitpītā//  \D
%kvacid ratnamayī   kvacit śveta     kvacitkṛṣṇā// [S6. verso]     kvacidrakta   kvacitpītā    \N2
%kvacid ratnamayī   kvacit śveta     kvacitkṛṣṇā                   kvacidrakta   kvacitpītā    \U1
%kvacid ratnamayī// kvacit śvetā//   kvacitkṛṣṇā//                 kvacidraktā// kvacitpītā//  \U2
%------------------------------
% ... manchmal aus Edelstein gemacht ist, manchmal weiß erscheint, manchmal schwarz, manchmal kupfern, manchmal gelb,
%
%... is sometimes made of precious stone, sometimes appearing white, sometimes black, sometimes copper, sometimes yellow, 
%------------------------------
\app{\lem[wit={ceteri},alt={°kvacid}]{kvaci\skp{d-ra}}
  \rdg[wit={P}]{\om}
}\app{\lem[wit={ceteri},alt={ratnamayī}]{\skm{d-ra}tnamayī}
  \rdg[wit={B,L}]{°maī}
  \rdg[wit={P}]{\om}}\dd{}
\app{\lem[wit={ceteri}]{kvacit}
  \rdg[wit={E}]{kvacic ca}
  \rdg[wit={P}]{\om}}
\app{\lem[wit={E,D,U2}]{śvetā}
  \rdg[wit={N1,N2,U1}]{śveta}
  \rdg[wit={L}]{śvetarūpā}
  \rdg[wit={B}]{śverūpā}
  \rdg[wit={P}]{\om}}\dd{}
\app{\lem[wit={ceteri}]{kvacit kṛṣṇā}
  \rdg[wit={N1}]{kṛṣṇa}
  \rdg[wit={E,P}]{\om}}\dd{}
\app{\lem[wit={ceteri},alt={°kvacid}]{kvaci\skp{d-ra}}
  \rdg[wit={P}]{\om}
}\app{\lem[wit={E,B,L,U2},alt={raktā}]{\skm{d-ra}ktā}
  \rdg[wit={ceteri}]{°rakta}}\dd{}
kvacit pītā\dd{}
\end{prose}
\end{ekdosis}
\ekdpb*{}
%%%%%%%%%%%%%%%%%%%%%%%%%%%%%%%%%%%%%%%%%%
%%%%%%%%%%%%%%%%%%%%%%%%%%%%%%%%%%%%%%%%%%
%%%%%%%%PAGEBREAK%%%%%%%PAGEBREAK%%%%%%%%%
%%%%%%%%%%%%%%%%%%%%%%%%%%%%%%%%%%%%%%%%%%
%%%%%%%%%%%%%%%%PAGEBREAK%%%%%%%%%%%%%%%%%
%%%%%%%%%%%%%%%%%%%%%%%%%%%%%%%%%%%%%%%%%%
%%%%%%%%PAGEBREAK%%%%%%%PAGEBREAK%%%%%%%%%
%%%%%%%%%%%%%%%%%%%%%%%%%%%%%%%%%%%%%%%%%%
%%%%%%%%%%%%%%%%%%%%%%%%%%%%%%%%%%%%%%%%%%
%%%%%%%%%%%%%%%%%%%%%%%%%%%%%%%%%%%%%%%%%%
%%%%%%%%%%%%%%%%%%%%%%%%%%%%%%%%%%%%%%%%%%
%%%%%%%%PAGEBREAK%%%%%%%PAGEBREAK%%%%%%%%%
%%%%%%%%%%%%%%%%%%%%%%%%%%%%%%%%%%%%%%%%%%
%%%%%%%%%%%%%%%%PAGEBREAK%%%%%%%%%%%%%%%%%
%%%%%%%%%%%%%%%%%%%%%%%%%%%%%%%%%%%%%%%%%%
%%%%%%%%PAGEBREAK%%%%%%%PAGEBREAK%%%%%%%%%
%%%%%%%%%%%%%%%%%%%%%%%%%%%%%%%%%%%%%%%%%%
%%%%%%%%%%%%%%%%%%%%%%%%%%%%%%%%%%%%%%%%%%
%%%%%%%%%%%%%%%%%%%%%%%%%%%%%%%%%%%%%%%%%%
%%%%%%%%%%%%%%%%%%%%%%%%%%%%%%%%%%%%%%%%%%
%%%%%%%%PAGEBREAK%%%%%%%PAGEBREAK%%%%%%%%%
%%%%%%%%%%%%%%%%%%%%%%%%%%%%%%%%%%%%%%%%%%
%%%%%%%%%%%%%%%%PAGEBREAK%%%%%%%%%%%%%%%%%
%%%%%%%%%%%%%%%%%%%%%%%%%%%%%%%%%%%%%%%%%%
%%%%%%%%PAGEBREAK%%%%%%%PAGEBREAK%%%%%%%%%
%%%%%%%%%%%%%%%%%%%%%%%%%%%%%%%%%%%%%%%%%%
%%%%%%%%%%%%%%%%%%%%%%%%%%%%%%%%%%%%%%%%%%
\begin{ekdosis}
  \begin{prose}
    \noindent
%------------------------------
%kvacitkarburā   kvacin nānāvidharūpā        kvacid viṣarūpā    kvacid amṛtarūpamayī svabhāvata eva bhavati//  \E  %%%p.28
%                                                               kvacid amṛtamayī     svabhāvata eva bhavati    \P  %%%rest is \om
%kvacitkarburā// kvacin nānāvidhaphalarūpā   kvacit viṣarūpā//  kvacid amṛtamaī/     svabhāvata eva bhavataḥ// \B
%kvacitkarburā// kvacin nānāvidhāphalarūpā   kvacit viṣarūpā//  kvacid amṛtamaī//    svabhāvata eva bhavataḥ// \L
%kvacitkarburā,  kvacin nānāvidhaphalarūpā/  kvacid puṣparūpā,  kvacid amṛtamayī     svabhāvata eva bhavati/   \N1
%kvacitkarburā   kvacin nānāvidhaphalarūpā// kvacid puṣparūpā// kvacid amṛtamayī/    svabhāvata eva bhavati//  \D
%kvacitkarburā   kvacin nānāvidhaphalarūpā                      kvacid amṛtamayī/    svabhāvata eva bhavati//  \N2
%kvacitkarpurā   kvacin nānāvidhophalarūpā   kvacid ....[rest omitted]                                         \U1
%kvacitkarburā// kvacit nānāvidhaphalarūpā// kvacir viśarūpā//  kvacit amṛtamayī//   svabhāvata eva bhavati//  \U2
%------------------------------
%machmal gesprenkelt, machmal wie verschiedenartige Frucht erscheint, manchmal wie Blumen erscheint, machmal wie der Nektar der Unsterblichkeit erscheint, [und das nur] nur aufgrund seiner inhärenten Natur.
%
%sometimes mottled, sometimes appearing like various fruit, sometimes appearing like flowers, sometimes appearing like the nectar of immortality, [and that only] only because of its inherent nature. 
%------------------------------
kvavit
\app{\lem[wit={ceteri}]{karburā}
  \rdg[wit={U1}]{karpurā}}\dd{}
\app{\lem[wit={ceteri}]{kvaci\skp{n-nā}}
  \rdg[wit={U2}]{kvacit}
  \rdg[wit={P}]{\om}
}\app{\lem[wit={ceteri},alt={nānāvidhaphalarūpā}]{\skm{n-nā}nāvidhaphalarūpā}
  \rdg[wit={U1}]{nānāvidhophalarūpā}
  \rdg[wit={E}]{nānāvidharūpā}
  \rdg[wit={P}]{\om}}\dd{}
\app{\lem[wit={ceteri},alt={kvacid}]{kvaci\skp{d-pu}}
  \rdg[wit={B,L}]{kvacit}
  \rdg[wit={U2}]{kvacir}
  \rdg[wit={P,N2}]{\om}
}\app{\lem[wit={N1,D},alt={puṣparūpā}]{\skm{d-pu}ṣparūpā}
\rdg[wit={E,B,L}]{viṣarūpā}
\rdg[wit={U2}]{vśarūpā}
\rdg[wit={U1}]{\om}}\dd{}
\app{\lem[wit={ceteri}, alt={kvacid}]{kvaci\skp{d-a}}
  \rdg[wit={U2}]{kvacit}
  \rdg[wit={U1}]{\om}
}\app{\lem[wit={ceteri},alt={amṛtamayī}]{\skm{d-a}mṛtamayī}
  \rdg[wit={E}]{amṛtarūpamayī}
  \rdg[wit={B,L}]{amṛtamaī}
  \rdg[wit={U1}]{\om}}\dd{}
\app{\lem[wit={ceteri}]{svabhāvata}
  \rdg[wit={U1}]{\om}}
\app{\lem[wit={ceteri}]{eva}
  \rdg[wit={U1}]{\om}}
\app{\lem[wit={ceteri}]{bhavati}
  \rdg[wit={B,L}]{bhavataḥ}
  \rdg[wit={U1}]{\om}}\dd{}
%------------------------------
%tathaivātmā   manuṣyapakṣihariṇahastividyādharagandharvakinnaramahāpaṃḍitamahāmūrkharogyarogikrodhiśāṃtarūpaḥ           svabhāvād eva bhavati/ \E
%tathaivātmā   manuṣyapakṣihariṇāhastividyādharagaṃdharvakinnaramahāṃpiṃḍitamahārmūkharogī    krodhiśāṃtarūpāḥ           svabhāvād eva bhavati \P
%tathaivātmā// manuṣyapakṣihariṇahastividyādharagaṃdharvakinnaramahāpiṃḍatamahāmūrkharogī     krodhadhiśāṃtarūpaḥ        svabhāvād eva bhavatī/ \B
%tathaivātmā   manuṣyapakṣihariṇahastividyādharagaṃdharvakinnaramahāpaṃḍitamahāmūrkharogī     krodhadhīśāṃtarūpāḥ        svabhāvād eva bhavatī/ \L
%tathātmā//    manuṣya,pakṣi,hariṇa,hastī,vidyādhara,gandharvakiṃnara/mahāpaṃḍitamahāmūrva,rogī, arogī/krodhī,śāntarūpa,svabhāvād eva bhati/ \N1 %%%%%%%CRAZY SWITCH BETWEEN DAṆḌA AND COMMA
%tathātmā//    manuṣyapakṣi// hariṇahastīvidyādharagandharvakinnaramahāpaṃḍitamahāmūrvarogī arogīkrodhīśāṃtarūpa---------svabhāvād eva bhavati/ \D
%tathātmā//    manuṣyapakṣihariṇahastividyādharagandharvakinnaramahāpaṇḍitamahāmūrkharogīarogīkrodhīśāṃtarūpa------------svabhāvād eva bhavati/ \N2
%                                     vidyādharagaṃdharvakinnaramahāpaṇḍitamahāmūrṣarogīarogīkrodhīśāṃtarūpa        evaṃ svabhāvaṃ dharati  \U1
%tathaivātmā   manuṣyapakṣihariṇahastividyādharagaṃdharvakinnaramahāpaṃḍitamahāmūrkharogī arogī krodhiśāṃtarūpaḥ          svabhāvād eva bhavati// \U2 %%%410.jpg
%------------------------------
%Auf diese Weise nimmt auch das Selbst aufgrund seiner inhärenten Natur die Form eines Menschen, Vogels, einer Gazelle, eines Elefants, eines Vidyādharas, eines Gandharvas, Zentauren, eines großen Gelehrten oder großen Dummkopfes, eines Kranken oder Gesunden, eines Zornigen oder Friedlichen an. 
%------------------------------
%------------------------------
%tathaivātmā   manuṣyapakṣihariṇahastividyādharagandharvakinnaramahāpaṃḍitamahāmūrkha  rogyarogikrodhi---śāṃtarūpaḥ      svabhāvād eva bhavati/ \E
%tathaivātmā   manuṣyapakṣihariṇāhastividyādharagaṃdharvakinnaramahāpiṃḍitamahārmūkha  rogī-----krodhi---śāṃtarūpāḥ      svabhāvād eva bhavati \P
%tathaivātmā// manuṣyapakṣihariṇahastividyādharagaṃdharvakinnaramahāpiṃḍatamahāmūrkha  rogī-----krodhadhiśāṃtarūpaḥ      svabhāvād eva bhavatī/ \B
%tathaivātmā   manuṣyapakṣihariṇahastividyādharagaṃdharvakinnaramahāpaṃḍitamahāmūrkha  rogī-----krodhadhīśāṃtarūpāḥ      svabhāvād eva bhavatī/ \L
%tathātmā//    manuṣyapakṣihariṇahastīvidyādharagandharvakiṃnaramahāpaṃḍitamahāmūrva   rogīarogīkrodhī---śāntarūpa-------svabhāvād eva bhati/ \N1 %%%%%%%CRAZY SWITCH BETWEEN DAṆḌA AND COMMA
%tathātmā//    manuṣyapakṣihariṇahastīvidyādharagandharvakinnaramahāpaṃḍitamahāmūrva   rogīarogīkrodhī---śāṃtarūpa-------svabhāvād eva bhavati/ \D
%tathātmā//    manuṣyapakṣihariṇahastividyādharagandharvakinnaramahāpaṇḍitamahāmūrkha  rogīarogīkrodhī---śāṃtarūpa-------svabhāvād eva bhavati/ \N2
%                                     vidyādharagaṃdharvakinnaramahāpaṇḍitamahāmūrṣa   rogīarogīkrodhī---śāṃtarūpa       evaṃ svabhāvaṃ dharati  \U1
%tathaivātmā   manuṣyapakṣihariṇahastividyādharagaṃdharvakinnaramahāpaṃḍitamahāmūrkha  rogīarogīkrodhi---śāṃtarūpaḥ      svabhāvād eva bhavati// \U2 %%%410.jpg
%------------------------------
%Auf diese Weise nimmt auch das Selbst aufgrund seiner inhärenten Natur die Form eines Menschen, Vogels, einer Gazelle, eines Elefants, eines Vidyādharas, eines Gandharvas, Zentauren, eines großen Gelehrten oder großen Dummkopfes, eines Kranken oder Gesunden, eines Zornigen oder Friedlichen an.
%
%In this way, the self also takes the form of a human, a bird, a gazelle, an elephant, a vidyādhara, a gandharva, a centaur, great scholar or a great fool, a sick or healthy, an angry or or peaceful person, by virtue of its inherent nature.       
%------------------------------      
\note[type=source, labelb=146, lem={tathaivātmā}]{Ysv (PT): tathā ca devagandharvakinnarādyāḥ khagādayaḥ | sukhasampiṇḍito rogī tathaiva krodhaśāntadhīḥ |aśeṣarūpabalito nānābuddhirataḥ svayam | devatattvaṃ bhūtaśaktyā jīvasaṃjñā bhramātmikā | jñānayogī nirvikāro nistāpa eka īśvaraḥ | ātmaikamūrttimān bhūtvā nirvikalpo nirañjanaḥ | sukhī duḥkhī mohayukto 'nantacetāḥ svabhāvataḥ |}
\app{\lem[wit={E,P,B,L,U2}]{tathaivātmā}
  \rdg[wit={ceteri}]{tathātmā}}
\app{\lem[wit={ceteri},alt={manuṣya°}]{manuṣya}
  \rdg[wit={U1}]{\om}
}\app{\lem[wit={ceteri},alt={°pakṣi°}]{pakṣi}
  \rdg[wit={U1}]{\om}
}\app{\lem[wit={ceteri},alt={°hariṇa°}]{hariṇa}
  \rdg[wit={P}]{°hariṇā°}
  \rdg[wit={U1}]{\om}
}\app{\lem[wit={N1,D},alt={°hastī°}]{hastī}
  \rdg[wit={ceteri}]{hasti}
  \rdg[wit={U1}]{\om}
}vidyādharagaṃdharvakinnaramahā\app{\lem[wit={ceteri},alt={°paṇḍita°}]{paṇḍita}
  \rdg[wit={B}]{piṃḍata}
}mahā\app{\lem[wit={ceteri},alt={°mūrkha°}]{mūrkha}
  \rdg[wit={P}]{°rmūkha°}
  \rdg[wit={N1,D}]{°mūrva°}
  \rdg[wit={U1}]{°mūrṣa°}
}\app{\lem[type=emendation, resp=egoscr]{rogyarogī}
  \rdg[wit={E}]{\korr °rogyarogi}
  \rdg[wit={N1,N2,D,U1,U2}]{°rogī arogī}
  \rdg[wit={P,B,L}]{°rogī}
}\app{\lem[wit={ceteri},alt={°krodhī°}]{krodhī}
  \rdg[wit={E,P}]{°krodhi°}
  \rdg[wit={B,L}]{°krodha°}
}\app{\lem[wit={ceteri},alt={°śānta°}]{śānta}
  \rdg[wit={B,L}]{°dhiśānta°}
}\app{\lem[wit={ceteri},alt={°rūpaḥ}]{rūpaḥ}
  \rdg[wit={P,L}]{°rūpāḥ}
  \rdg[wit={N1,N2,D,U1}]{°rūpa}}
\app{\lem[wit={ceteri},alt={svabhāvād eva}]{svabhāvād-eva}
  \rdg[wit={U1}]{evaṃ svabhāvaṃ}}
\app{\lem[wit={ceteri}]{bhavati}
  \rdg[wit={B,L}]{bhavatī}
  \rdg[wit={N1}]{bhati}
  \rdg[wit={D}]{dharati}}\dd{}
%------------------------------      
%jñānayogādhikārarūparahito  jñāyate/  yathā plakṣasyotpattiḥ/ sthānam eva bhavati// \E
%jñānayogādhikārarūparahito  jñāyate   yathā phalasyotpattisthānam ekam eva bhavati \P  %%%7643.jpg                          
%jñānayogādhikārarūparahito  jñāyate// yathā phalasyotpattisthānam ekam eva bhavatī// \B
%jñānayogādhikārarūparahito  jñāyate// yathā phalasyotpattisthānam ekam eva bhavati// \L
%jñānayogād vikārarūparahito jñāyate/  yathā phalasyotpattisthānam ekam eva bhavati/ \N1
%jñānayogādhikārarūparahito  jñāyate// yathā phalasyotpattisthānam ekaseva  bhavati// \D
%jñānayogadhikārarūparahito  jñāyate// yathā phalasyotpattisthānam eva kameva bhavati// \N2
%jñānayogāt vikārarūparahito jñāyate   yathā phalasyotpattisthāna  ekam eva ti \U1
%jñānayogādhikārarūparahito  jāyate//  yathā phalasyotpattisthānam ekam eva bhavati// \U2
%------------------------------
%em. zu jñānayogādhikāriṇā? em. zu vikāraṃ rūparahitaṃ -> By the man of Jñānayoga the modifications are known as formless.?! Just as the place of origin of the fruit is only one.
%Aufgrund von Jñānayoga wird die Umwandlung als Formlos erkannt.
%Because of jñānayoga, transformation is recognized as formless. 
%------------------------------
\app{\lem[wit={N1,U1}, alt={jñānayogād vikāra}]{jñānayogād-vikāra}
  \rdg[wit={ceteri}]{jñānayogadhikāra}
}rūparahito
\app{\lem[wit={ceteri}]{jñāyate}
  \rdg[wit={U2}]{jāyate}}\dd{}
yathā
\app{\lem[wit={ceteri}]{phalasyotpatti}
  \rdg[wit={E}]{plakṣasyotpattiḥ}
}\app{\lem[wit={ceteri},alt={°sthānam}]{sthāna\skp{m-e}}
  \rdg[wit={E}]{sthānam}
  \rdg[wit={U1}]{°sthāna}
}\app{\lem[wit={ceteri},alt={ekam}]{\skm{m-e}ka\skp{m-e}}
  \rdg[wit={D}]{ekas}
  \rdg[wit={N2}]{eva}
  \rdg[wit={E}]{\om}
}\app{\lem[wit={ceteri},alt={eva}]{\skm{m-e}va}
  \rdg[wit={N2}]{kam eva}}
\app{\lem[wit={ceteri}]{bhavati}
  \rdg[wit={B}]{bhavatī}
  \rdg[wit={U1}]{ti}}/
%------------------------------
%atha ca phalasya gatir bahudhā dṛśyate/ \E
%atha ca phalasya gati  bahudhā dṛśyate    \P
%atha ca phalasya gatir bahudhā dṛśyate// \B
%atha ca phalasya gatir bahudhā dṛśyate// \L
%atha ca phalasya gatir bahudhā dṛśyate/ \N1
%atha ca phalasya gatir bahudhā dṛśyate// \D
%atha ca phalasya gati  bahudhā dṛśyate/ \N2
%atra ca phalasya gati  bahudhā dṛśyate \U1
%atha ca phalasya gatir bahudhā dṛśyate// \U2
%------------------------------
%But the path of the fruit is seen manifold. 
%------------------------------
atha ca phalasya \app{\lem[wit={ceteri},alt={gatir}]{gati\skp{r-ba}}
  \rdg[wit={P,N2,U1}]{gati}
}\skm{r-ba}hudhā dṛśyate\dd{}
%------------------------------ %%%STEMMAPOINT śuklaṃ//śuṣkaṃ
% ekaṃ phalaṃ pṛthvīmadhye  patati/  śuklaṃ bhavati/   \E
% ekaṃ phalaṃ pṛthvīmadhye  patati   śuklaṃ bhavati    \P
% ekaṃ phalaṃ pṛthvīmadhye  patiśuklaṃ      bhavatī//  \B
% ekaṃ phalaṃ pṛthvīmadhye  patati   śuṣkaṃ bhavatī    \L
% ekaṃ phala--pṛthvīmadhye  patati/  śuklaṃ bhavati/   \N1 %%%p.7 recto letzte Zeile 
% ekaṃ phala--pṛthvīmadhye  patati// śuklaṃ bhavati//  \D
% eva  phala--pṛthvīmadhye  patati   śuklaṃ bhavati//  \N2
% ekaṃ phalaṃ pṛthivīmadhye patati   śuṣkaṃ bhavati    \U1
% ekaphalaṃ   pṛthvīmadhye  patati// śuṣkaṃ bhavati//  \U2
%------------------------------
%One fruit falls onto the ground. It is getting bright.  Dürre entsteht// Licht entsteht// Es wird hell!!!. 
%------------------------------
\app{\lem[wit={ceteri}]{ekaṃ}
  \rdg[wit={U2}]{eka°}
  \rdg[wit={N2}]{eva}}
\app{\lem[wit={ceteri}]{phalaṃ}
  \rdg[wit={N1,N2,D}]{phala°}}
\app{\lem[wit={ceteri},alt={pṛthvī°}]{pṛthvī}
  \rdg[wit={U1}]{pṛthivī°}
}madhye patati/
\app{\lem[wit={ceteri}]{śuklaṃ}
  \rdg[wit={L,U1,U2}]{śuṣkaṃ}}
\app{\lem[wit={ceteri}]{bhavati}
  \rdg[wit={B}]{bhavatī}}/
%------------------------------
% ekasya phalasya makaraṃdaṃ bhramaraḥ  pibati/  \E
% ekasya phalasya makaraṃdaṃ bhramaraḥ  pibaṃti  \P
% ekasya            karaṃdaṃ bhramaraṃ  pibatī/  \B
% ekasya          makaraṃdaṃ bhramaraṃ  pibati   \L
% ekasya phalasya makaraṃdabhramaraḥ    pibati/  \N1 %%%p.7 recto letzte Zeile 
% ekasya phalasya makaraṃdabhramaraḥ    pibati/  \D
% ekasya phalasya makaraṃdaṃ bhramara   pibati/  \N2
% ekasya phalasya makaraṃdaṃ bhramanaḥ  pibati   \U1
% ekasya phalasya makaraṃdaṃ bhramaraḥ  pibati// \U2
% ------------------------------
% Eine Biene trinkt den Blumensaft der einen Frucht.
% A bee drinks the flower juice of a fruit.     
%------------------------------
ekasya
\app{\lem[wit={ceteri}]{phalasya}
  \rdg[wit={P,L}]{\om}} 
\app{\lem[wit={E,P,L,N2,U1,U2}]{makaraṃdaṃ}
  \rdg[wit={L,N1}]{makaraṃda°}
  \rdg[wit={B}]{karaṃdaṃ}}
\app{\lem[wit={ceteri}]{bhramaraḥ}
  \rdg[wit={B,L}]{bhramaraṃ}
  \rdg[wit={N2}]{bhramara}}
\app{\lem[wit={ceteri}]{pibati}
  \rdg[wit={P}]{pibaṃti}
  \rdg[wit={B}]{pibatī}}/
%------------------------------
% ekasya phalasya  mālāṃ kāminī tuṃgakucamaṃḍalopari dadhāti/ \E
% ekasya phalasya  mālāṃ kāminī tuṃgakucamaṃḍalopari dadhāti \P
% ekasya phalasya  mālāṃ kāminī tuṃgakucamaṃḍalopari dadhātī// \B
% ekasya phalasya  mālāṃ kāminī tuṃgakucamaṃḍalopari dadhāti// \L
% ekasya phalasya  mālāṃ kāminī tuṃgakucamaṃḍalopari dadhāvati/ \N1 %%%p.7 recto letzte Zeile 
% ekasya phalasya  mālāṃ kāmibī tuṃgakucamaṇḍalopari dadhāti// \D
% ekasya phalasyaṃ mālākāminī   tuṃgakucamaṇḍalopari dadhovati// \N2
% ekasya phalasya  mālāṃ kāmini tuṃ  kucamaṃḍalopari dadhāti \U1
% ekasya phalasya  mālāṃ kāminī tuṃgakucamaṃḍalopari dadhāti// \U2
%------------------------------
% of the one fruit Blütenkranz/Girlande die Verliebte (biene) führt ein unmittelbar über dem Kreis des Blütenstempels der wie eine Brust ist ein.  %tu.mga = hervorstehend 
%Die Verliebte (Biene) platziert sich auf dem Blütenkranz über dem emportstehenden Kreisförmigen Blütenstempel.    
%------------------------------
ekasya
\app{\lem[wit={ceteri}]{phalasya}
  \rdg[wit={N2}]{phalasyaṃ}}
\app{\lem[wit={ceteri}]{mālāṃ}
  \rdg[wit={N2}]{mālā°}}
\app{\lem[wit={ceteri}]{kāminī}
  \rdg[wit={D}]{kāmibī}}
\app{\lem[wit={ceteri},alt={tuṅga°}]{tuṅga}
  \rdg[wit={U1}]{tuṃ°}
}kucamaṃḍalopari
\app{\lem[wit={ceteri}]{dadhāti}
  \rdg[wit={N1}]{dadhāvati}
  \rdg[wit={N2}]{dadhovati}}/
%------------------------------ 
%ekaṃ phalaṃ mṛtamanuṣyopari   kṣipyate/  ayaṃ vastunaḥ svabhāvaḥ/  tathā eka evātmā   svīyabhāvād evāṣṭau    bhogān  bhunakti/ \E
%ekaṃ phalaṃ mṛtamanuṣyopari   kṣipyate   ayaṃ vastunaḥ svabhāvaḥ   tathā eka evātmā   svīyabhāvād evāṣṭau    bhogān  bhunakti \P
%ekaṃ phalaṃ mṛtamanuṣyopari   kṣapyate// ayaṃ vastunaḥ svabhāvaḥ/  tathā eka evātmā   svabhāvād   evāṣṭau    bhogān  bhunakte// \B
%ekaṃ phalaṃ mṛtamanuṣyopari  kṣipyate// ayaṃ vastunaḥ svabhāvaḥ   tathā eka evātmā   svabhāvād   evāṣṭau    bhogān  bhunakte// \L
%ekaphalaṃ   mṛtamanuṣyopari   kṣipyate// ayaṃ vastunaḥ svabhāvaḥ/  tathā eka evātmā   svīyabhāvād evāṣṭau  bhogānā bhunakti/ \N1
%ekaphalaṃ   mṛtamanuṣyopari  kṣipyate// ayaṃ vastunaḥ svabhāvaḥ// tathā eka evātmā   svīyabhāvād evāṣṭau  bhogān  bhunakti// \D
%ekaphalaṃ   mṛtamanuṣyopari   kṣipyate/  ayaṃ castunaḥ svabhāvaḥ/  tathā ekaevātmā    svīyabhāvād evāstau  bhogāt  bhunakti/ \N2
%ekaphalaṃ   mṛtamanuṣyopari   kṣipyate/  ayaṃ castunaḥ svabhāvaḥ/  tathā ekaevātmā    svīyabhāvād evāstau bhogāt  bhunakti/ \U1 %%%276.jpg
%ekaṃ phalaṃ mṛtamanuṣyopari   kṣipyate// ayaṃ vastunaḥ svabhāvaḥ// tathā ekameva ātmā svīyabhāvād evāṣṭabhogān    bhunakti// \U2
%------------------------------
%Die eine Frucht schleudert den Nektar über die Blüte. (em zu anu.s.a = Blüthe?) Dies ist die inhärente Natur der Sache. So genießt auch das eine Selbst genießt aufgrund des eigenen Seins die acht Genüsse. 
%------------------------------
\note[type=testium, labelb=147, lem={svīyabhāvād}]{strīpuṃrūpī mahān so hi parasparavimohitaḥ | amanaskaḥ svīyabhāvāt jñānayogī nirākulaḥ | srakcandanādivāmāsu svabhāvād bhogam icchukaḥ |}
\app{\lem[type=emendation, resp=egoscr,alt={ekaṃ phalam}]{ekaṃ phala\skp{m-a}}
  \rdg[wit={E,P,B,L}]{\korr ekaṃ phalaṃ}
  \rdg[wit={N1,N2,D,U1}]{eka°}}
\app{\lem[type=emendation, resp=egoscr, alt={amṛtam}]{\skm{m-a}mṛta\skp{m-a}}
  \rdg[wit={ceteri}]{\korr mṛta°}
}\app{\lem[type=emendation, resp=egoscr,alt={anuṣṇopari}]{\skp{m-a}nuṣṇopari}
  \rdg[wit={ceteri}]{\korr manuṣyopari}}
\app{\lem[wit={ceteri}]{kṣipyate}
  \rdg[wit={B}]{kṣapyate}}/
%------------------------------
%ke te ṣṭau  bhogāḥ – suvāsaś ca   suvastrañ ca  suśayyā    sunitaṃbinī/       susthānañ cānnapānāni    aṣṭau bhogāś ca dhīmatām/       padṛsūtramayāni vasrāṇi// \E
%ke te ṣṭau  bhogauḥ  suvāsaś ca   suvāsaś   ca  suyyā      sunitāṃbinīḥ//     susthānāś cānpanānp------aṣṭau bhogāś ca dhīmatāṃ 1      padasūtramayāni vastrāṇi?? \P %%%7643.jpg
%      aṣṭau bhogāḥ   suvāsac ca   suvasaś   ca  suśayyāḥ   sūnitaṃbinī/       susthānaś vānnapānāny----aṣṭau bhogāś cā sudhīmatām//1// paṭasūtrāmayāni vasrāṇi// \B
%      aṣṭau bhogāḥ   suvāsaś ca   suvāsaś   ca  suśayyāḥ   sūnitaṃbinī//      susthānāś cānnapānāny----aṣṭau bhogāś cā sudhīmatāṃ//1// paṭasūtrāmayāni vastrāṇi// \L
%ke te ṣṭau  bhogāḥ// suvāyaś ca/                suśayyā    sunitaṃbinī/       susthātāś cātmapanasyā----ṣṭau bhogāḥ    sudhipaṇa----------padṛsūtrayāni   vasrāṇi \N1
%ke te ṣṭau  bhogāḥ// suvāyaś ca//               suśayyā    sunittaṃbinī//     susthātāś cānmanasyā------ṣṭau bhogāḥ    sudhiṣaṇa----------padṛsūtrayāni   vasrāṇi   \D
%ke te ṣṭau  bhogāḥ   suvāyaś ca                 suśayya    sunitaṃbinī/       susthānāś cānmanasyā------ṣṭau bhogāḥ    sudhiyane          padṛsūtrayāni   vasrāṇi \N2
%ke te ṣṭe   bhogā –  suvāsaś ca                 suśayyā ca sunītavinīta       susthātāś cānnapānaḥ syādaṣṭau bhogāḥ    sudhiṣaṇāṃ         padṛsūtramayāni vasrāṇi \U1
%ke te aṣṭau bhogā // suvāsaś ca// suvaṃśaś ca// suśayyā//  sunitaṃbinī//      sudehaṃ// sukhasaṃtānaṃ// abhayādicāṣṭakaṃ//              paṭasūtramayāni vasrāṇi \U2
%------------------------------
%What are the eight enjoyments? A beautiful dwelling, good clothing, a good bed, a well-trained horse?, a nice place, food & drink. Tose are the eight enjoyments of the wise. Clothes made from silk.
%------------------------------
\app{\lem[wit={ceteri}]{ke te}
  \rdg[wit={L,B}]{\om}}
\app{\lem[wit={ceteri}]{'ṣṭau}
  \rdg[wit={L,B}]{aṣṭau}
  \rdg[wit={U1}]{ṣṭe}}
\app{\lem[wit={ceteri}]{bhogāḥ}
  \rdg[wit={P}]{bhobauḥ}
  \rdg[wit={U1,U2}]{bhogā}}
\end{prose}
\end{ekdosis}
%%%%%%%%%%%
%%%%%%%%%%
%%%%%%%%%%
%%%%%%%%%%%
\begin{ekdosis}
  \ekddiv{type=ed}
  \begin{tlg}
\tl{\app{\lem[wit={ceteri},alt={suvāsaś ca}]{suvāsaś-ca}
  \rdg[wit={B}]{suvāsac ca}}
\app{\lem[wit={E},alt={suvastrañ ca}]{suvastrañ-ca}
  \rdg[wit={U2}]{suvaṃśaś ca}}
\app{\lem[wit={ceteri}]{suśayyā}
  \rdg[wit={U1}]{suśayyā ca}
  \rdg[wit={L,B}]{suśayyāḥ}
  \rdg[wit={P}]{suyyā}}
\app{\lem[wit={ceteri}]{sunitaṃbinī}
  \rdg[wit={P}]{sunitāṃbinīḥ}
  \rdg[wit={U1}]{sunītavinīta}}/}\\
\tl{\app{\lem[wit={E},alt={susthānañ}]{susthāna\skp{ñ-cā}}
  \rdg[wit={P,L,N2}]{susthānāś}
  \rdg[wit={N1,D,U1}]{susthātāś}
  \rdg[wit={U2}]{sudehaṃ}
}\app{\lem[wit={L},alt={°ānnapānāny}]{\skm{ñ-cā}nnapānā\skp{ny-a}}
  \rdg[wit={B}]{°vānna°}
  \rdg[wit={E}]{°pānāni}
  \rdg[wit={P}]{cānpanānp°}
  \rdg[wit={N1}]{cātmapanasyā°}
  \rdg[wit={N2,D}]{cānmanasyā°}
  \rdg[wit={U1}]{cānnapānaḥ syād°}
  \rdg[wit={U2}]{sukhasaṃtānaṃ}
}\app{\lem[wit={E,P},alt={aṣṭau bhogāś ca dhīmatām}]{\skm{ny-a}ṣṭau bhogāś-ca dhīmatām}
  \rdg[wit={B,L}]{aṣṭau bhogāś cā sudhīmatām}
  \rdg[wit={N1}]{ṣṭau bhogāḥ sudhipaṇa°}
  \rdg[wit={D}]{ṣṭau bhogāḥ sudhiṣaṇa°}
  \rdg[wit={U1}]{aṣṭau bhogāḥ sudhiṣaṇāṃ}
  \rdg[wit={U2}]{abhayādicāṣṭakaṃ}
  \rdg[wit={N1,N2,D,U1}]{aṣṭau bhogāḥ}
  \rdg[wit={U2}]{abhayādicāṣṭakaṃ}}\dd{}1\hskip-2pt\dd{}}
\end{tlg}
\end{ekdosis}
\ekdpb*{}
%%%%%%%%%%%%%%%%%%%%%%%%%%%%%%%%%%%%%%%%%%
%%%%%%%%%%%%%%%%%%%%%%%%%%%%%%%%%%%%%%%%%%
%%%%%%%%PAGEBREAK%%%%%%%PAGEBREAK%%%%%%%%%
%%%%%%%%%%%%%%%%%%%%%%%%%%%%%%%%%%%%%%%%%%
%%%%%%%%%%%%%%%%PAGEBREAK%%%%%%%%%%%%%%%%%
%%%%%%%%%%%%%%%%%%%%%%%%%%%%%%%%%%%%%%%%%%
%%%%%%%%PAGEBREAK%%%%%%%PAGEBREAK%%%%%%%%%
%%%%%%%%%%%%%%%%%%%%%%%%%%%%%%%%%%%%%%%%%%
%%%%%%%%%%%%%%%%%%%%%%%%%%%%%%%%%%%%%%%%%%
%%%%%%%%%%%%%%%%%%%%%%%%%%%%%%%%%%%%%%%%%%
%%%%%%%%%%%%%%%%%%%%%%%%%%%%%%%%%%%%%%%%%%
%%%%%%%%PAGEBREAK%%%%%%%PAGEBREAK%%%%%%%%%
%%%%%%%%%%%%%%%%%%%%%%%%%%%%%%%%%%%%%%%%%%
%%%%%%%%%%%%%%%%PAGEBREAK%%%%%%%%%%%%%%%%%
%%%%%%%%%%%%%%%%%%%%%%%%%%%%%%%%%%%%%%%%%%
%%%%%%%%PAGEBREAK%%%%%%%PAGEBREAK%%%%%%%%%
%%%%%%%%%%%%%%%%%%%%%%%%%%%%%%%%%%%%%%%%%%
%%%%%%%%%%%%%%%%%%%%%%%%%%%%%%%%%%%%%%%%%%
%%%%%%%%%%%%%%%%%%%%%%%%%%%%%%%%%%%%%%%%%%
%%%%%%%%%%%%%%%%%%%%%%%%%%%%%%%%%%%%%%%%%%
%%%%%%%%PAGEBREAK%%%%%%%PAGEBREAK%%%%%%%%%
%%%%%%%%%%%%%%%%%%%%%%%%%%%%%%%%%%%%%%%%%%
%%%%%%%%%%%%%%%%PAGEBREAK%%%%%%%%%%%%%%%%%
%%%%%%%%%%%%%%%%%%%%%%%%%%%%%%%%%%%%%%%%%%
%%%%%%%%PAGEBREAK%%%%%%%PAGEBREAK%%%%%%%%%
%%%%%%%%%%%%%%%%%%%%%%%%%%%%%%%%%%%%%%%%%%
%%%%%%%%%%%%%%%%%%%%%%%%%%%%%%%%%%%%%%%%%%
%------------------------------
%padṛsūtramayāni vasrāṇi//  \E
%padasūtramayāni vastrāṇi?? \P %%%7643.jpg
%paṭasūtrāmayāni vasrāṇi//  \B
%paṭasūtrāmayāni vastrāṇi// \L
%padṛsūtrayāni   vasrāṇi    \N1
%padṛsūtrayāni   vasrāṇi    \D
%padṛsūtrayāni   vasrāṇi    \N2
%padṛsūtramayāni vasrāṇi    \U1
%paṭasūtramayāni vasrāṇi    \U2
%------------------------------
%Clothes made from silk,...
%------------------------------
%paṃcasaptā dṛālikā         yuktāni harmyāṇi teṣu vāsaḥ    ativipulā  mṛdutarasukhāsuśayyā/     \E
%paṃcasaptā dadhikā         yuktāni harmyāṇi teṣu cāsaḥ 2  ativipulā  mṛduttarachadavatīśayyā 2  \P
%paṃcasatyā dātikā          yuktāni harmyāṇi teṣu vāstu    ativipulā  mṛdutaralāśayyā//2//        \B
%paṃcasatyā dātikā          yuktāni harmyāṇi teṣu vāstu    ativipulā  mṛdutaralāśayyā//3//        \L
%paṃca vā sapta vā dṛālikā  yuktāni harmyāṇi/              ativapulā  mṛdu/uttaracchaṃdavatīśayyā/  \N1
%paṃca vā sapta vā dṛāṃlikā yuktāni harmyāṇi               ativapulā  mṛduuttarachaṃdavatīśayyā/     \D
%paṃca vā sapta vā tālikā---yuktāni harmyāni               ativipulā  mṛduuttarachaṃdavatīśayyā    \N2
%paṃca vā sapta vā dālikā---yuktāni harmyāṇi               ativipulāṃ mṛduuttarachadavatiśaiyyā     \U1
%--------------------------saudhāni harmyāṇi vāsāya kecit// aṣṭau bhogān āha// sugrahaṃ// suvastraṃ// suśayā sustrī//  \U2
%--------------------------------------------
%,a site of the palace in which there are mainsions endowned with five or seven rooms. A huge, very soft and lovely bed.  
%-------------------------------------------
\begin{ekdosis}
  \ekddiv{type=ed}
  \begin{prose}
\noindent
    \app{\lem[type=emendation, resp=egoscr,alt={paṭṭa°}]{paṭṭa}
  \rdg[wit={E,N1,D,N2,U1}]{\korr padṛ°}
  \rdg[wit={P}]{pada°}
  \rdg[wit={B,L,U2}]{paṭa°}
}\app{\lem[wit={ceteri},alt={sūtra°}]{sūtra}
  \rdg[wit={B,L}]{sūtrā}
}\app{\lem[wit={ceteri}]{mayāni}
  \rdg[wit={N1,N2,D}]{yāni}}
\app{\lem[wit={P,L}]{vastrāṇi}
  \rdg[wit={ceteri}]{vasrāṇi}} 1\dd{}
\app{\lem[wit={N1,N2,D,U1}]{paṃca vā sapta vā}
  \rdg[wit={E,P}]{paṃcasaptā}
  \rdg[wit={L,B}]{paṃcasatyā}}
\app{\lem[type=emendation, resp=egoscr]{śālikā}
  \rdg[wit={E,N1}]{\korr dṛālikā}
  \rdg[wit={D}]{dṛāṃlikā}
  \rdg[wit={P}]{dadhikā}
  \rdg[wit={B,L}]{dātikā}
  \rdg[wit={N2}]{tālikā}
  \rdg[wit={U1}]{dālikā}
}\app{\lem[wit={ceteri}]{yuktāni}
  \rdg[wit={U2}]{saudhāni}}
harmyāṇi
\app{\lem[wit={L,B}]{teṣu vāstu}
  \rdg[wit={E}]{teṣu vāsaḥ}
  \rdg[wit={P}]{teṣu cāsaḥ}
  \rdg[wit={U2}]{vāsāya kecit}
  \rdg[wit={ceteri}]{\om}} 2\dd{}
\app{\lem[wit={ceteri}]{ativipulā}
  \rdg[wit={N1,D}]{ativapulā}
  \rdg[wit={U1}]{ativipulāṃ}
  \rdg[wit={U2}]{aṣṭau bhogān āha ||}}
\app{\lem[type=emendation, resp=egoscr,alt={mṛdūttara}]{mṛdūttara}
  \rdg[wit={E,P,L,B}]{\korr mṛdutara°}
  \rdg[wit={N1,N2,D,U1}]{mṛdu | uttara°}
  \rdg[wit={U2}]{sugrahaṃ ||}
}\app{\lem[wit={N1,N2,D},alt={°chandavatī°}]{chandavatī}
  \rdg[wit={P}]{°chadavatī°}
  \rdg[wit={U1}]{°chadavati°}
  \rdg[wit={U2}]{suvastraṃ ||}
}\app{\lem[wit={ceteri}]{śayyā}
  \rdg[wit={U2}]{suśayā sustrī}} 3\dd{}
%------------------------------
%padminī tārūṇyavatī  manoharā guṇavatī  tatropaviṣṭā kāṃtā/      \E
%padminī tārūṇyavatī  manoharā guṇavatī  tatopaviṣṭā  kāṃtā 4     \P
%padminī tārūnyavatī  manoharā guṇavatī//tatrāpavistā kāṃtā 4     \B
%padminī tārūnyavatī  manoharā guṇavatī//tatropavistā kāṃtā// 4// \L
%padmanī tārūṇyavatī  manoharā guṇavatī  tatropavistā//           \N1
%padminī tārūrāyavatī manoharā guṇavatī  tatropavistā//           \D
%padminī tārūnyavatī  manoharā guṇavatī  tatropavistā             \N2
%padminī tārūnyavati  manoharā guṇavati  tatropavistā             \U1
%                                                                 \U2
%--------------------------------------------
%[On which] there is situated [tatropaviṣṭā] a lotus-like [em. zu tāruṇyavatī] youthful, charming and virtuous wife. 
%-------------------------------------------
\app{\lem[wit={ceteri}]{padminī}
  \rdg[wit={N1}]{padmanī}
  \rdg[wit={U2}]{\om}}
\app{\lem[type=emendation, resp=egoscr]{tāruṇyavatī}
  \rdg[wit={ceteri}]{\korr tārūṇyavatī}
  \rdg[wit={N2}]{tārūrāyavatī}
  \rdg[wit={U2}]{\om}}
\app{\lem[wit={ceteri}]{manoharā guṇavatī}
   \rdg[wit={ceteri}]{tatropavistā}
  \rdg[wit={P}]{tato°}
  \rdg[wit={B}]{tatrā°}
  \rdg[wit={U2}]{\om}}
\app{\lem[wit={E,P,B,L}]{kāntā}
  \rdg[wit={ceteri}]{\om}} 4\dd{}
%------------------------------
%sādhu āśanam/      atimūlyañ ca/         manoramam annaṃ।       tathā vidhaṃ pānam/   \E
%sādhu āsanaṃ 5     atimūlo 'śvaḥ 6       manoramam annaṃ    7   tathā vidhaṃ pānaṃ 8  \P
%sādhu āsanaṃ 5     atimūlyo asvaṃ//6     manoramyam attaṃ //7   tathā vidhapānaṃ//8   \B
%sādhu āsanaṃ// 5// atimūlyo aśvaṃ//6//   manoramyam annaṃ //7// tathā vidhapānaṃ//8// \L
%sādhyāsanaṃ//      amūlyo svaś ca//      manoramam attaṃ        tathā vidhaṃ pānaṃ/   \N1
%sādhyāsanaṃ//      amūlyo svaś ca//      manoramam attaṃ        tathā vidhaṃ pānaṃ//  \D
%sādhyāsanaṃ        amūlyo svaś ca        manotamam annaṃ        tathā vidhapānaṃ//    \N2
%sādhyāsanaṃ        amolyo svaś ca        manoramam annaṃ        tathā vidhaṃ pānaṃ    \U1
%sādhu āsanaṃ//           suśvaḥ//        suṣṭu annaṃ//          tathā vidhayānaṃ//    \U2
%--------------------------------------------
%good throne/seat; atimūlyo (überaus wertvolles) 'śvaṃ (Pferd), manorama ( die Sinne erfreuendes) Essen, verschiedenes Trinken. 
%-------------------------------------------
\app{\lem[type=emendation, resp=egoscr]{sādhvāsanaṃ}
  \rdg[wit={E}]{\korr sādhu āśanam}
  \rdg[wit={P,B,L,U2}]{sādhu āsanaṃ}
  \rdg[wit={N1,N2,D}]{sādhyāsanaṃ}} 5\dd{}
\app{\lem[type=emendation, resp=egoscr]{atimūlyo 'śvaḥ}
  \rdg[wit={E}]{\korr atimūlyañ ca}
  \rdg[wit={P}]{atimūlo 'śvaḥ}
  \rdg[wit={L,B}]{atimūlyo asvaṃ}
  \rdg[wit={N1,N2,D,U1}]{amūlyo svaś ca}
  \rdg[wit={U2}]{suśvaḥ}} 6\dd{}
\app{\lem[wit={ceteri},alt={manoramam annaṃ}]{manoramam-annaṃ}
  \rdg[wit={B}]{manoramyam attaṃ}
  \rdg[wit={L}]{manoramyam annaṃ}
  \rdg[wit={N1,D}]{manoramam attaṃ}
  \rdg[wit={U2}]{suṣṭu annaṃ}} 7\dd{}
tathā
\app{\lem[wit={ceteri}]{vidhaṃ pānaṃ}
  \rdg[wit={L,B,N2}]{vidhapānaṃ}
  \rdg[wit={U2}]{vidhayānaṃ}} 8\dd{}
%------------------------------
%ete   ṣṭau bhogāḥ   kathitāḥ/   eke  duḥkhaṃ   bhajante/  bhikṣāṃ  yācante// kiñca \E
%ete   ṣṭau bhogāḥ   kathitā 9   eke  duḥkha    bhajaṃte   bhikṣāṃ  yāṃcaṃte ca  \P
%ete   ṣṭau bhogāḥ//             eka  duḥkhā    bhajaṃte/  bhikṣā   yāṃcate ca//  \B
%ete   ṣṭau bhogāḥ//             eka  duḥkhā    bhajaṃte// bhikṣā   yāṃcate ca//  \L
%ete  aṣṭau bhogā    kathyate/   eke  duḥkhaṃ   bhajaṃte/  bhikṣyāṃ yācate ca/   \N1
%ete  aṣṭau bhogāḥ   kathyaṃte// ete  duḥkhaṃ   bhajaṃte/  bhikṣyāṃ yācaṃte ca// \D
%ete  aṣṭau ghogā    kathyate//  ete  duḥkhataṃ bhajate    bhikṣāṃ  yācate ca//  \N2
%rāte aṣṭau bhogāḥ   kathyate    ete  duḥkhaṃ   bhajate    bhikṣāṃ  pācate ca    \U1
%ete  ṣṭau  bhogāḥ// kathitāḥ//  ekaṃ duḥkhaṃ   bhajaṃte// bhikṣā   yācaṃte ca// \U2
%------------------------------
%die acht genüsse wurden erzählt. Sie bringen Leid und die Bet. 
%------------------------------
%\note[type=philcomm, labelb=148, lem={'ṣṭau bhogāḥ}]{The eight enjoyments are not attested in any of the sources.}
\app{\lem[wit={ceteri}]{ete}
  \rdg[wit={U1}]{rāte}}
\app{\lem[wit={ceteri}]{'ṣṭau}
  \rdg[wit={N1,N2,D,U1}]{aṣṭau}}
\app{\lem[wit={ceteri}]{bhogāḥ}
  \rdg[wit={N1,N2}]{bhogā}
  \rdg[wit={U1}]{ghogā}}
\app{\lem[wit={E,U2}]{kathitāḥ}
  \rdg[wit={P}]{kathitā}
  \rdg[wit={N1,N2,U1}]{kathyate}
  \rdg[wit={D}]{kathyaṃte}
  \rdg[wit={L,B}]{\om}}\dd{}
%------------------------------
%      yathā sūryasya tejaḥ   dugdhasya    ghṛtam   agner jvalanaṃ viṣān mūrchā   tilāttailam/    vṛkṣāc-chāyā/  phalāt parimalaḥ       kāṣṭhād agniḥ    arkarādibhyo   madhuro rasaḥ/ \E
%      yathā sūryasys tejaḥ   dugdhasya    ghṛtaḥ   agne dvāhaḥ    viṣān mūrchāti tilāttailaṃ     vṛkṣāt-chāyā   phalāsarimalaḥ         kāṣṭād  agniḥ    śarkvarādibhyo madhuro rasaḥ  \P
%      yathā sūryasye tejāḥ   dugdha-------ghṛtaḥ   agne dvāhaḥ//  viṣān mūrchā   tilāttailaṃ//   vṛkṣā--chāyā   phalāsarimalaḥ         kaṣṭād  agniḥ    śarkadībhyo    madhuro  \B
%      yathā sūryasya tejāḥ   dugdha-------ghṛtaḥ   agne dvāhaḥ//  viṣān mūrchā   tilātailaṃ//    vṛkṣā--chāyā   phalāt parimalaḥ       kaṣṭād  agniḥ    śarkadībhyo    madhuro  \L
%      yathā sūryasya tejaḥ/  dugdhasya    ghṛtaṃ/  agne dahiḥ??   viṣān mūrchā   tilāttailaṃ,    vṛkṣāc-chāyā/  phalāt parimalaḥ/      kāṣṭhād āgniḥ/   śarkkarādibhyo madhuro rasaḥ/ \N1
%      yathā sūryasya tejaḥ// dugdhasya    ghṛtaṃ// agne dadhiḥ    viṣān mūrchā   tilāttailaṃ//   vṛkṣā--chāyā// phalāt palātparimalaḥ//kāṣṭhād āgniḥ//  śarkarādibhyo  madhuro rasaḥ/ \D
%      yathā sūryasya tejaḥ   dusya        ghṛtaṃ   agne dadhi     viṣān mūrchā   tilatailaṃ      vṛkṣā--chāyā   phalāt parimalaḥ       kāṣṭhād āgniḥ    śarkarādibhyo  madhuro rasaḥ/ \N2
%      yathā sūryaśca tejaḥ   dugdhasy     ghṛttaṃ  agne dārhaṃ    viṣāt mūrchā   tilātailaṃ      vrakṣā-chāyā   phalāt parimalaḥ       kāṣṭhād āgniḥ    śarkarādibhyo  madhuro rasaḥ \U1
%      yathā sūryasya tejaḥ// dugdhasya    ghṛtaṃ// agne dāhiḥ//   viṣān mūrchā   tilātailaṃ//    vṛkṣā--chāyā// phalāt parimalaḥ//     kāṣṭād  agniḥ    śarkarādibhyo  madhuro rasaḥ// \U2
%------------------------------
%Gleichwie die Strahlen der Sonne, die Butter der Milch, das Brennen des Feuers, die Betäubung aufgrund von Gift, das Sesamöl aus dem Sesamkorn, der Schatten vom Baum, der Wohlgeruch von einer Frucht, das Feuer von einem Holzscheid, der Süße Saft [em. zu śārkara] a liquor prepared from Dhātakī with sugar] und so weiter,   
%------------------------------
%Like the rays of the sun, the butter of milk, the burning of fire, the stupor of poison, the sesame oil from the sesame seed, the shade from the tree, the sweet odor from a fruit, the fire from a scabbard, the sweet sap [em . to śārkara] a liquor prepared from Dhātakī with sugar] and so on,
%------------------------------
\note[type=source, labelb=149, lem={sūryasya}]{Ysv (PT): ravī tejo ghṛtaṃ dugdhe tile tailaṃ svabhāvataḥ | śaśam indau kule śākhaṃ kṣāre ca lavaṇaṃ yathā | tathā brahmaṇi saṃsāro hyakhaṇḍaparipūrvake ||}
yathā
\app{\lem[wit={ceteri}]{sūryasya}
  \rdg[wit={U1}]{sūryaś ca}}
\app{\lem[wit={ceteri}]{tejaḥ}
  \rdg[wit={L,B}]{tejāḥ}}\dd{}
\app{\lem[wit={E,P,N1,D,U2}]{dugdhasya}
  \rdg[wit={L,B}]{dugdha°}
  \rdg[wit={N2}]{dusya}
  \rdg[wit={U1}]{dugdhasy}}
\app{\lem[wit={ceteri}]{ghṛtaṃ}
  \rdg[wit={P,L,B}]{ghṛtaḥ}}\dd{}
\app{\lem[wit={E}, alt={agner}]{agne\skp{r-dā}}
  \rdg[wit={ceteri}]{agne}
}\app{\lem[type=emendation, resp=egoscr, alt={dāhaḥ}]{\skm{r-dā}haḥ}
  \rdg[wit={P,L,B}]{\korr dvāhaḥ}
  \rdg[wit={N1}]{dahiḥ}
  \rdg[wit={N2}]{dadhi}
  \rdg[wit={D}]{dadhiḥ}
  \rdg[wit={U1}]{dārhaṃ}
  \rdg[wit={U2}]{dāhiḥ}
  \rdg[wit={E}]{jvalanaṃ}}\dd{}
\app{\lem[wit={ceteri},alt={viṣān}]{viṣā\skp{n-mū}}
  \rdg[wit={U1}]{viṣāt}
}\skm{n-mū}rchā\dd{}
\app{\lem[wit={ceteri},alt={tilāt}]{tilā\skp{t-tai}}
  \rdg[wit={P}]{titilāt}
  \rdg[wit={N2}]{tila}
  \rdg[wit={U1}]{tilā}
}\skm{t-tai}laṃ\dd{}
\app{\lem[wit={E,N1}, alt={vṛkṣāt}]{vṛkṣā\skp{c-chā}}
  \rdg[wit={P}]{vṛkṣāt}
  \rdg[wit={L,B,N2,D,U2}]{vṛkṣā}
  \rdg[wit={U1}]{vrakṣā}
}\skm{c-chā}yā\dd{}
\app{\lem[wit={ceteri},alt={phalāt}]{phalā\skp{t-pa}}
  \rdg[wit={L,B}]{phalā}
}\app{\lem[wit={ceteri},alt={parimalaḥ}]{\skm{t-pa}rimalaḥ}
  \rdg[wit={L,B}]{sarimalaḥ}
  \rdg[wit={D}]{palāt parimalaḥ}}\dd{}i%\note[type=philcomm, labelb=150, lem={parimalaḥ}]{Clarification: Witness \getsiglum{D} reads \textit{phalāt palāt parimala}.}
\app{\lem[wit={ceteri}, alt={kāṣṭhād}]{kāṣṭhā\skp{d-a}}
  \rdg[wit={P,U2}]{kāṣṭād}
  \rdg[wit={B,L}]{kaṣṭād}
}\app{\lem[wit={ceteri}, alt={agniḥ}]{\skm{d-a}gniḥ}
  \rdg[wit={N1,N2,D,U1}]{āgniḥ}}\dd{}
\app{\lem[type=emendation, resp=egoscr]{śārkarādibhyo}
  \rdg[wit={E}]{\korr arkarādibhyo}
  \rdg[wit={P}]{śarkvarādibhyo}
  \rdg[wit={L,B}]{śarkadībhyo}}
madhuro
\app{\lem[wit={ceteri}]{rasaḥ}
  \rdg[wit={L,B}]{\om}}\dd{}
%------------------------------
%himānībhyaḥ   śītam      ityādipadārthānāṃ svabhāvaḥ         tathā    saṃsāro'pi parameśvarasvarūpamadhye      tiṣṭhati/ \E
%himānībhyaḥ   śītaṃ      ityādipadārthasvabhāva        eva   tathā    saṃsāro'pi parameśvarasvarūpamadhye      tiṣṭhati \P
%sahīmānībhyaḥ śītaḥ/     ityādipadārthāsvabhāvataḥ// eva     tathā    saṃsāro pi paremesvara svarūpasya madhye tiṣṭhatī/ \B
%sahimānibhyaḥ śītaḥ//    ityādiphadārthāḥ svabhāvataḥ// eva  tathā    saṃsāro pi paremesvara svarūpasya madhye tiṣṭhati// \L
%himānibhyaḥ   śaityāṃ    ityādipadārthasvabhāva evā/         tathā    saṃsāro pi parameśvarasvarūpamadhye      tiṣṭhati// \N1
%himānibhyaḥ   śaityaṃ // ityādipadārthasvabhāva eva//        tathā    saṃsāro pi parameśvarasvarūpamadhye      tiṣṭhati// \D 
%himānitpa     śaityāś    atyādipadārtharthasvabhāva eva//    tathā    saṃsāro pi parameśvarasvarūpamadhye      tiṣṭhati \N2
%himānībhyaḥ   śaityaṃ    ityādipadārthasvabhāvaḥ ravaḥ?      tathā vā saṃsāro pi parameśvararūpamadhye         tiṣṭhati/ \U1
%himānībhyaḥ   śītyaṃ//   ityādipadārthāsvabhāva eva//        tathā    saṃsāro pi parameśvarasvarūpamadhye      tiṣṭhaṃti// \U2
%------------------------------
%die Kälte von Schneehaufen, und so weiter ist das inhärente Wesen der Dinge. IN gleicher Weise befindet sich auch der Weltengang im Zentrum der eigenen Gestalt von höchsten Gott.
%the cold of piles of snow, and so on is the inherent essence of things. In the same way, the course of the world is also in the center of the highest God's own form. 
%------------------------------
\app{\lem[wit={ceteri}]{himānībhyaḥ}
  \rdg[wit={L,B}]{sahimānibhyaḥ}
  \rdg[wit={N2}]{himānitpa}}
\app{\lem[wit={D,U1}]{śaityaṃ}
  \rdg[wit={N1}]{śaityāṃ}
  \rdg[wit={U2}]{śītyaṃ}
  \rdg[wit={N2}]{śaityāś}
  \rdg[wit={E,P}]{śītaṃ}
  \rdg[wit={L,B}]{śītaḥ}}\dd{}
\app{\lem[wit={N1,D,P}]{ityādipadārthasvabhāva}
  \rdg[wit={U2}]{°padārthā°}
  \rdg[wit={B}]{ityādipadārthāsvabhāvataḥ}
  \rdg[wit={N2}]{atyādipadārtharthasvabhāva}
  \rdg[wit={U1}]{°svabhāvaḥ}
  \rdg[wit={L}]{ityādiphadārthāḥ svabhāvataḥ}
  \rdg[wit={E}]{ityādipadārthānāṃ svabhāvaḥ}}
\app{\lem[wit={ceteri}]{eva}
  \rdg[wit={N1}]{evā}
  \rdg[wit={U1}]{ravaḥ}
  \rdg[wit={E}]{\om}}\dd{}
\app{\lem[wit={ceteri}]{tathā}
  \rdg[wit={U1}]{tathā vā}}
saṃsāro 'pi
\app{\lem[wit={ceteri}]{parameśvarasvarūpamadhye}
  \rdg[wit={L,B}]{paremesvara svarūpasya madhye}
  \rdg[wit={U1}]{parameśvararūpamadhye}}
\app{\lem[wit={ceteri}]{tiṣṭhati}
  \rdg[wit={B}]{tiṣṭhatī}
  \rdg[wit={U2}]{tiṣṭhaṃti}}\dd{}
%------------------------------
%parameśvaro 'khaṇḍa--paripūrṇaḥ/  \E
%parameśvaro khaṃḍa---paripūrṇaś ca    \P
%parameśvaro khaṃḍa---paripūrṇaś ca// \B
%parameśvaro khaṃḍa---paripūrṇaś ca//  \L
%parameśvaro 'ṣaṃḍa---paripūrṇaś ca//  \N1
%parameśvaro  ṣaṃḍa---paripūrṇaś ca//  \D %%%S.9 verso
%parameśvaro yarāṇḍa--paripūrṇaś ca//  \N2
%parameśvaro khaṃḍaḥ  paripūrṇaś ca   \U1 %%%277.jpg
%parameśvaro 'khaṃḍa--paripūrṇaś ca//   \U2
%------------------------------
%Und der höchste Gott ist unteilbar und das All erfüllend.
%And the Most High God is indivisible and all-filling.
%------------------------------
parameśvaro
\app{\lem[wit={ceteri}, alt={'khaṇḍa°}]{'khaṇda}
  \rdg[wit={N1,D}]{'ṣaṃḍa°}
  \rdg[wit={N2}]{yarānda°}
  \rdg[wit={U1}]{khaṃḍaḥ}
}\app{\lem[wit={ceteri},alt={°paripūrṇaś ca}]{paripūrṇaś\skp{-}ca}
  \rdg[wit={E}]{paripūrṇaḥ}}\dd{}
\end{prose}
\end{ekdosis}
%------------------------------
%idānīṃ lakṣyaṃ       kathyate/ \E
%idānīṃ bāhyalakṣyaṃ  kathyate \P
%idānīṃ ṣāhyalakṣa    kathyate// \B
%idānīṃ bāhyalakṣa    kathyate// \L
%idānīṃ bāhyalakṣaṃ   kathyate// \N1
%idānīṃ bāhyalakṣaṇa  kathyate// \D %%%S.9 verso
%idānīṃ bāhyalakṣaṇa  kathyate/ \N2
%idānīṃ bāhyalakṣyaḥ  kathyate \U1 %%%277.jpg
%idānīṃ bāhyalakṣaṇaṃ kathyate// \U2
%------------------------------
%Now the external fixation is taught.
%------------------------------
\begin{ekdosis}
  \ekddiv{type=ed}
      \bigskip
        \centerline{\textrm{\small{[Bāhyalakṣya]}}}
          \bigskip
          \begin{prose}
                \noindent
\note[type=source, labelb=151, lem={bāhyalakṣyaṃ}]{Ysv (PT): idānīṃ vāhyalakṣāṇi siddhidāni śṛṇu priye | dhāraṇākhyā tu caitāni jñātavyāni viśeṣataḥ |}
idānīṃ
\app{\lem[wit={P}]{bāhyalakṣyaṃ}
  \rdg[wit={E}]{lakṣyaṃ}
  \rdg[wit={B}]{ṣāhyalakṣa}
  \rdg[wit={L}]{bāhyalakṣa}
  \rdg[wit={N1}]{°lakṣaṃ}
  \rdg[wit={D,N2}]{°lakṣaṇa}
  \rdg[wit={U1}]{°lakṣyaḥ}
  \rdg[wit={U2}]{lakṣaṇaṃ}}
kathyate/
%------------------------------
%nāsāgrād ārabhyāṃgulacatuṣṭaya--pramāṇaṃ nīlākāraṃ tejaḥ   pūrṇam ākāśaṃ  lakṣyaṃ  karttavyam/ \E
%nāsāgrād ārabhyāṃgulacatuṣṭaya--pramāṇaṃ nilākāraṃ tejaḥ   pūrṇam ākāśaṃ lakṣyaṃ  kartavyaṃ  \P
%nāsāgrād ārabhyāṃgulacatuṣṭayaṃ pramāṇaṃ nilākāraṃ   jaḥ   pūrṇam ākāśa--lakṣaṃ   kartavyaṃ//    \B
%nāsāgrād ārabhyāṃgulacatuṣṭayaṃ pramāṇaṃ nilākāraṃ tejaḥ// pūrṇam ākāśaṃ lakṣaṃ   kartavyaṃ// \L
%nāsāgrād ārabhyāṃgulacatuṣṭaya--pramāṇaṃ nīlākāraṃ teja----pūrṇam ākāśa--lakṣaṃ   karttavyaṃ \N1
%nāsāgrād ārabhyāṃgulacatuṣṭaya--pramāṇaṃ nīlākāraṃ teja----pūrṇam ākāśa---lakṣaṃ   karttavyaṃ \D
%nāsāgrād ārabhyāṃgulacatuṣṭaya--pramāṇaṃ nirākāraṃ teja----pūrṇam ākāśa---lakṣaṇaṃ karttavyaṃ// \N2
%nāsāgrād ārabhyāṃgulacatuṣṭaya--pramāṇaṃ nīlākāraṃ tejaḥ   pūrṇam ākāśaṃ lakṣyaṃ  karttavyam \U1
%nāsāgrād ārabhyāṃgulacatuṣṭaya--pramāṇaṃ nīlākāraṃ tejaḥ   pūrṇakām ākāśa-lakṣyaṃ  karttavyaṃ \U2 %%%411.jpg
%------------------------------
%Beginning with a four finger wide distance from the tip of the nose, the space[-element?] full of light whose appearance is blue shall be made the object of fixation.
%------------------------------
nāsāgrād-ārabhyāṃgula\app{\lem[wit={ceteri}, alt={catuṣṭaya°}]{catuṣṭaya\skp{-}}
  \rdg[wit={B,L}]{catuṣṭayaṃ}
}pramāṇaṃ
\app{\lem[wit={ceteri}]{nīlākāraṃ}
  \rdg[wit={P,B,L}]{nilākaraṃ}
  \rdg[wit={N2}]{nirākāraṃ}}
\app{\lem[wit={N1,N2,D},alt={teja°}]{teja}
  \rdg[wit={ceteri}]{tejaḥ}
  \rdg[wit={B}]{jaḥ}
}\app{\lem[wit={ceteri}, alt={pūrṇam}]{pūrṇa\skp{m-ā}}
  \rdg[wit={U2}]{pūrṇakām}
}\app{\lem[wit={ceteri},alt={ākāśa°}]{\skm{m-ā}kāśa}
    \rdg[wit={E,P,L,U1}]{ākāśaṃ}
}\app{\lem[wit={E,P,U1,U2}]{lakṣyaṃ}
  \rdg[wit={B,L,N1,D}]{lakṣaṃ}
  \rdg[wit={N2}]{lakṣaṇaṃ}}
kartavyaṃ/
\end{prose}
\end{ekdosis}
\ekdpb*{}
%%%%%%%%%%%%%%%%%%%%%%%%%%%%%%%%%%%%%%%%%%
%%%%%%%%%%%%%%%%%%%%%%%%%%%%%%%%%%%%%%%%%%
%%%%%%%%PAGEBREAK%%%%%%%PAGEBREAK%%%%%%%%%
%%%%%%%%%%%%%%%%%%%%%%%%%%%%%%%%%%%%%%%%%%
%%%%%%%%%%%%%%%%PAGEBREAK%%%%%%%%%%%%%%%%%
%%%%%%%%%%%%%%%%%%%%%%%%%%%%%%%%%%%%%%%%%%
%%%%%%%%PAGEBREAK%%%%%%%PAGEBREAK%%%%%%%%%
%%%%%%%%%%%%%%%%%%%%%%%%%%%%%%%%%%%%%%%%%%
%%%%%%%%%%%%%%%%%%%%%%%%%%%%%%%%%%%%%%%%%%
%%%%%%%%%%%%%%%%%%%%%%%%%%%%%%%%%%%%%%%%%%
%%%%%%%%%%%%%%%%%%%%%%%%%%%%%%%%%%%%%%%%%%
%%%%%%%%PAGEBREAK%%%%%%%PAGEBREAK%%%%%%%%%
%%%%%%%%%%%%%%%%%%%%%%%%%%%%%%%%%%%%%%%%%%
%%%%%%%%%%%%%%%%PAGEBREAK%%%%%%%%%%%%%%%%%
%%%%%%%%%%%%%%%%%%%%%%%%%%%%%%%%%%%%%%%%%%
%%%%%%%%PAGEBREAK%%%%%%%PAGEBREAK%%%%%%%%%
%%%%%%%%%%%%%%%%%%%%%%%%%%%%%%%%%%%%%%%%%%
%%%%%%%%%%%%%%%%%%%%%%%%%%%%%%%%%%%%%%%%%%
%%%%%%%%%%%%%%%%%%%%%%%%%%%%%%%%%%%%%%%%%%
%%%%%%%%%%%%%%%%%%%%%%%%%%%%%%%%%%%%%%%%%%
%%%%%%%%PAGEBREAK%%%%%%%PAGEBREAK%%%%%%%%%
%%%%%%%%%%%%%%%%%%%%%%%%%%%%%%%%%%%%%%%%%%
%%%%%%%%%%%%%%%%PAGEBREAK%%%%%%%%%%%%%%%%%
%%%%%%%%%%%%%%%%%%%%%%%%%%%%%%%%%%%%%%%%%%
%%%%%%%%PAGEBREAK%%%%%%%PAGEBREAK%%%%%%%%%
%%%%%%%%%%%%%%%%%%%%%%%%%%%%%%%%%%%%%%%%%%
%%%%%%%%%%%%%%%%%%%%%%%%%%%%%%%%%%%%%%%%%%
\begin{ekdosis}
  \begin{prose}
    \noindent
%------------------------------
%atha vā nāsāgrād ārabhya ṣaḍaṃgulapramāṇaṃ    pavanatattvaṃ dhūmrākāraṃ        lakṣyaṃ karttavyam// \E
%atha vā nāsāgrād ārabhya ṣaḍaṃgulapramāṇaṃ    pavanatatvaṃ  dhūmrākāraṃ        lakṣyaṃ karttavyam \P
%atha vā nāsāgrād ārabhya ṣaḍaṃgulaṃ pramāṇaṃ ?bi?ṣi?īnāvarṇaṃ .. .. .. ..??.  lakṣyaṃ kartavyam/  \B
%\om \L
%atha vā nāsāgrād ābhya   ṣadaṃgulapramāṇaṃ    pavanatatvaṃ dhūmrākāraṃ        lakṣaṃ karttavyaṃ/  \N1
%atha vā nāsāgrād ābhya   ṣadaṃgulapramāṇaṃ    pavanatatvaṃ dhūmrākāraṃ        lakṣaṃ karttavyaṃ// \D
%atha vā nāsāgrārabhya    ṣadaṃgulapramāṇaṃ    pavanatatvaṃ dhūmrākāraṃ        lakṣaṇaṃ karttavyaṃ// \N2
%atha vā nāsāgrād ārabhya dvadaśaṃgulapramāṇaṃ pavanatatvaṃ dhūmrākāraṃ        lakṣyaṃ karttavyaṃ \U1
%atha vā nāsāgrād ārabhya ṣaḍaṃgulapramāṇaṃ    pavanatatvaṃ dhūmrākāraṃ        lakṣaṃ karttavyaṃ// \U2
%------------------------------
%Or, a six finger wide distance from the tip of the nose, the wind-element whose appearance is greyish shall be made the object of fixation. 
%------------------------------
\note[type=source, labelb=152, lem={ṣadaṃgulapramāṇaṃ}]{Ysv (PT): līlayā bhāvayel līnaṃ jyotiḥpūrṇaṃ mahāparam | atha vā tatra deveśi dhūmrākāraṃ ṣaḍaṅgulam |}
\app{\lem[wit={ceteri}]{atha vā}
      \rdg[wit={L}]{\om}}
    \app{\lem[wit={ceteri}]{nāsāgrād\skp{-}ārabhya}
      \rdg[wit={N1,D}]{nāsāgrād ābhya}
      \rdg[wit={N2}]{nāsāgrārabhya}
      \rdg[wit={L}]{\om}}
    \app{\lem[wit={ceteri}]{ṣaḍaṃgulapramāṇaṃ}
      \rdg[wit={B}]{ṣaḍaṃgulaṃ pramāṇaṃ}
      \rdg[wit={U2}]{dvadaśaṃgulapramāṇaṃ}
      \rdg[wit={L}]{\om}}
     \app{\lem[wit={E}]{pavanatattvaṃ}
       \rdg[wit={ceteri}]{°tatvaṃ}
       \rdg[wit={L}]{\om}
       \rdg[wit={B}]{\illeg}}
     \app{\lem[wit={ceteri}]{dhūmrākāraṃ}
       \rdg[wit={B}]{\illeg}
       \rdg[wit={L}]{\om}}
     \app{\lem[wit={ceteri}]{lakṣyaṃ}
       \rdg[wit={N1,D,U2}]{lakṣaṃ}
       \rdg[wit={N2}]{lakṣaṇaṃ}}
     \app{\lem[wit={ceteri}]{karttavyaṃ}
       \rdg[wit={L}]{\om}}/
%------------------------------
%\om \E
%\om \P
%\om \B
%\om \L
%atha vā nāsāgrād ārabhyā  ṣaḍaṃgulapramāṇām atiraktaṃ        tejo lakṣaṃ karttavyaṃ \N1
%atha vā nāsāgrād ārabhya  ṣaḍaṃgulapramāṇām atirattaṃ        tejo lakṣaṃ karttavyaṃ// \D
%atha vā nāsāgrād ārabhyaṃ ṣṭāṃgulapramāṇam atirakṭaṃ         tejo lakṣaṇaṃ kartavyaṃ// \N2
%atha    nāsāgrād ārabhyāṣṭaṃgulapramāṇam     itiriktaṃ       tejo lakṣyaṃ karttavyaṃ/ \U1
%atha vā nāsāgrād ārabhyaṃ        ṣṭagulapramāṇaṃ matiraktaṃ  teja lakṣyaṃ karttavyaṃ// \U2 
%------------------------------
% Or, an eight finger wide distance from the tip of the nose, the very red fire[-element] shall be made the object of fixation. 
%------------------------------
\note[type=testium, labelb=153, lem={ārabhyāṣṭaṃgula°}]{Ysv (PT): atha vāṣṭāṅgulaṃ raktaṃ nāsikopari lakṣayet |}
atha
\app{\lem[wit={ceteri}]{vā}
  \rdg[wit={U1}]{\om}}
nāsāgrā\skp{d-ā}\app{\lem[wit={U1}, alt={ārabhyāṣṭāṃgula°}]{\skm{d-ā}rabhyāṣṭaṃgulapramāṇa\skp{m-a}}
  \rdg[wit={N1}]{ārabhyā ṣaḍaṃgulapramāṇām}
  \rdg[wit={D}]{ārabhya ṣaḍaṃgulapramāṇām}
  \rdg[wit={N2}]{ārabhyaṃ ṣṭāṃgulapramāṇam}
  \rdg[wit={U2}]{ārabhyaṃ ṣṭagulapramāṇaṃ}
  \rdg[wit={ceteri}]{\om}
}\app{\lem[wit={N1,N2}, alt={atiraktaṃ}]{\skm{m-a}tiraktaṃ}
  \rdg[wit={D}]{atirattaṃ}
  \rdg[wit={U1}]{itiriktaṃ}
  \rdg[wit={U2}]{matiraktaṃ}
  \rdg[wit={ceteri}]{\om}}
\app{\lem[wit={ceteri}]{tejo}
  \rdg[wit={U2}]{teja°}
  \rdg[wit={ceteri}]{\om}}
\app{\lem[wit={U1,U2}]{lakṣyaṃ}
  \rdg[wit={N1,N2}]{lakṣaṃ}
  \rdg[wit={N2}]{lakṣaṇaṃ}
  \rdg[wit={ceteri}]{\om}}
karttavyaṃ/
%------------------------------
%\om \E
%\om \P
%\om \B
%\om \L
%atha vā nāsāgrād ārabhya daśāṃgulapramāṇaṃ śuklaṃ caṃcalam    udakaṃ lakṣya   karttavyaṃ/  \N1
%atha vā nāsāgrād ārabhya daśāṃgulapramāṇaṃ śuklaṃ caṃcalam    udakaṃ lakṣya   karttavyaṃ// \D
%atha vā nāsāgrād ārabhya daśāṃgulapramāṇaṃ śuklaṃ caṃdrākāram udakaṃ lakṣyaṃ  kartavyaṃ    \U1
%atha vā nāsāgrād ārabhya daśāṃgulapramāṇaṃ śuklaṃ caṃcalam    udakaṃ lakṣaṇaṃ kartavyaṃ//  \N2 [S.7 Verso, Zeile 1]
%atha vā nāsāgrād ārabhya daśāṃgulapramāṇaṃ śuklaṃ caṃcalam    udakaṃ lakṣaṃ   kartavyaṃ//  \U2
%------------------------------
%Or, a ten finger wide distance from the tip of the nose, the white water[-element] being fickle shall be made the object of fixation. 
%------------------------------
\note[type=philcomm, labelb=154, lem={daśāṃgulapramāṇaṃ}]{The instruction of a ten finger wide distance is absent in the surviving testimonia of the Ysv.}
\app{\lem[wit={ceteri}]{atha vā}
  \rdg[wit={E,P,B,L}]{\om}}
\app{\lem[wit={ceteri}, alt={nāsāgrād ārabhya}]{nāsāgrād-ārabhya}
  \rdg[wit={E,P,B,L}]{\om}}
\app{\lem[wit={ceteri}]{daśāṃgulapramāṇaṃ}
  \rdg[wit={E,P,B,L}]{\om}}
\app{\lem[wit={ceteri}]{śuklaṃ}
  \rdg[wit={E,P,B,L}]{\om}}
\app{\lem[wit={ceteri}]{caṃcalam}
  \rdg[wit={U1}]{caṃdrākāram}
  \rdg[wit={E,P,B,L}]{\om}}
\app{\lem[wit={ceteri}]{udakaṃ}
  \rdg[wit={E,P,B,L}]{\om}}
\app{\lem[wit={U1}]{lakṣyaṃ}
  \rdg[wit={N1,D}]{lakṣya}
  \rdg[wit={N2}]{lakṣaṇaṃ}
  \rdg[wit={U2}]{lakṣaṃ}
  \rdg[wit={ceteri}]{\om}}
\app{\lem[wit={ceteri}]{kartavyaṃ}
  \rdg[wit={ceteri}]{\om}}
%------------------------------
%atha vā nāsāgrād ārabhya tattvaṃ dvādaśāṃgulapramāṇaṃ   pītavarṇaṃ  pṛthvītattvaṃ lakṣyaṃ  karttavyam/ \E
%atha vā nāsāgrād ārabhya         dvādaśāṃgulapramāṇaṃ   pītavarṇaṃ  pṛthvītatvaṃ  lakṣyaṃ  karttavyaṃ \P
%atha vā nāsāgrād ārabhya         dvadaśāṃgulapramāṇaṃ   pītavarṇaṃ  pṛthvītatvaṃ  lakṣaṃ   kartavyaṃ// \B
%atha vā nāsāgrād ārabhya         dvādaśāṃgulapramāṇaṃ   pītavarṇaṃ  pṛthvītatvaṃ  lakṣaṃ   kartavyaṃ/  \L
%atha vā nāsāgrād ārabhya         dvadaśāṃgulapramāṇaṃ   pītavarṇṇaṃ prthvītatvaṃ  lakṣaṃ   karttavyaṃ/   \N1
%atha vā nāsāgrād ārabhya         dvadaśāṃgulapramāṇaṃ   pītavarṇṇaṃ prthvītatvaṃ  lakṣaṃ   karttavyaṃ/   \D
%atha vā nāsāgrād ārabhya         dvadaśāṃgulapramāṇaṃ   pītavarṇaṃ  prthvītatvaṃ  lakṣaṇaṃ karttavyaṃ//  \N2
%atha vā nāsāgrād ārabhya         dvādaśā aṃgulapramāṇaṃ pītavarṇaṃ  prthvītatvaṃ  lakṣyaṃ  karttavyaṃ   \U1
%atha vā nāsāgrād ārabhya         dvādaśāṃgulapramāṇaṃ   pītavarṇaṃ  pṛthvītatvaṃ  lakṣaṃ   karttavyaṃ//    \U2
%------------------------------
%Or, a twelve finger wide distance from the tip of the nose, the yellow-colored earth-element shall be made the object of fixation.  
%------------------------------
\note[type=source, labelb=155, lem={dvādaśāṃgulapramāṇaṃ}]{Ysv (PT): dvādaśāṅgulamānaṃ vā pṛthvītattvan tu pītabham | lakṣayed atha vā tatra koṭisūryasamaprabham | tejaḥ puñjaṃ mahākāśaṃ tattad dhyānāc chivo bhavet | ākāśamadhye ākāśoparito dṛṣṭis usthiram | kṛtvā dhyānād vinā sūryaṃ caṇḍasūryan tu paśyati | atha vā lakṣam etat tu karttur vahiḥ śivopari |}
atha vā nāsāgrād-ārabhya
\app{\lem[wit={ceteri}]{dvādaśāṃgulapramāṇaṃ}
  \rdg[wit={E}]{tattvaṃ dvādaśāṃgulapramāṇaṃ}
  \rdg[wit={U1}]{dvādaśā aṃgulapramāṇaṃ}}
pītavarṇaṃ pṛthvītattvaṃ
\app{\lem[wit={E,P,U1}]{lakṣyaṃ}
  \rdg[wit={N2}]{lakṣaṇaṃ}
  \rdg[wit={ceteri}]{lakṣaṃ}}
kartavyaṃ/
%------------------------------
%atha vā nāsāgrād ārabhya koṭisūryasamaprabhaṃ tejaḥ/ pūrṇam ākāśatattvaṃ lakṣyaṃ karttavyam/                 \E
%atha vā nāsāgrād ārabhya koṭisūryasamaprabhaṃ tejaḥpūrṇam   ākāśatatvaṃ lakṣyaṃ karttavyaṃ                      \P
%atha vā nāsāgrād ārabhya koṭisūryasamaprabhaṃ tejaḥ/ pūrṇam ākāśatatvaṃ lakṣaṃ kartavyaṃ//                   \B %%%%DSCN7161.JPG letzte 3 Zeilen!
%atha vā nāsāgrād ārabhya koṭisūryasamaprabhāṃ tejaḥpūrṇam   ākāśatatvaṃ lakṣaṃ karttavyaṃ//                   \L
%atha vā nāsāgrād ārabhya koṭisūryasamaprabhaṃ tejaḥpūrṇaṃ   ākāśatatvaṃ lakṣyaṃ karttavyaṃ/                   \N1
%atha vā nāsāgrād ārabhya koṭisūryasamaprabhaṃ tejaḥpūrṇaṃ   ākāśatatvaṃ lakṣyaṃ karttavyaṃ//                   \D
%atha vā nāsāgrād ārabhya koṭisūryasamaprabhaṃ tejaḥpūrṇa    ākāśatatvaṃ lakṣaṇaṃ karttavyaṃ//                    \N2
%atha vā nāsāgrād ārabhya koṭisūryasamaprabhaṃ tejaḥpūrṇaṃ   ākāśatatvaṃ lakṣyaṃ karttavyaṃ                     \U1
%atha vā nāsāgrād ārabhya koṭisūryasamaprabhaṃ tejaḥpūrṇaṃ   ākāśatatvaṃ lakṣaṃ karttavyaṃ//                    \U2
%------------------------------
%Or beginning at the tip of the nose\footnote{Given the clear instructions of the respective distance of the exercise in the previous sentences, it is surprising that this instruction is lacking the mention of the distance.} the space-element full of fire shining like ten million suns shall be made the object of fixation.  
%------------------------------
atha vā nāsāgrād ārabhya koṭisūrya\app{\lem[wit={ceteri}]{samaprabhaṃ}
  \rdg[wit={L}]{°prabhāṃ}}
\app{\lem[wit={ceteri}, alt={tejaḥpūrṇam}]{tejaḥpūrṇa\skp{m-ā}}
  \rdg[wit={E,B}]{tejaḥ | pūrṇaṃ}
  \rdg[wit={N1,D,U1,U2}]{pūrṇaṃ}
  \rdg[wit={N2}]{pūrṇa}
}\skm{m-ā}kāśatattvaṃ
\app{\lem[wit={E,P,N1,D,U1}]{lakṣyaṃ}
  \rdg[wit={B,L,U2}]{lakṣaṃ}
  \rdg[wit={N2}]{lakṣaṇaṃ}}
karttavyaṃ/
%------------------------------
%ākāśamadhye  ākāśopari    dṛṣṭiṃ kṛtvā             dhyānakāraṇāt// sūryaṃ vinā sūryasambandhinī  sahasrakiraṇapaṅktīḥ   paśyati/ \E
%             ākāśopari    dṛṣtiṃ kṛtvā             dhyānakaraṇāt   sūryaṃ vinā sūryasaṃbaṃdhīnīṃ sahasrakiraṇāvalīṃ     pati   \P
%             ākāśopari    dṛṣti  kṛtvā ākāśamadhye dhyānakaraṇāt// sūryaṃ vinā sūryasaṃbaṃdhīnī  sahasrakiraṇāvali      paśyatī// \B
%                                       ākāśamadhye dhyānakaraṇāt// sūryaṃ vinā sūryasaṃbaṃdhinī  sahasrakiraṇāvali      paśyati// \L
%ākāśamadhye  ākāśoparī vā dṛṣṭiṃ kṛtvā             dhyānakaraṇāt   sūryaṃ vinā sūryasaṃbaṃdhinī  sahasrāṇy api kīraṇāṇi paśyatī/     \N1
%ākāśamadhye  ākāśopari vā dṛṣṭiṃ kṛtvā             dhyānakaraṇāt   sūryaṃ vinā sūryasaṃbaṃdhinī  sahasrāṇapi   kīraṇāṇi paśyatī//     \D
%ākāśamadhye  ākāśopari vā dṛṣṭiṃ kṛtvā             dhyānakaraṇāt   sūrya  vinā sūryasaṃbaṃdhinī  sahasrāṇapi   kiraṇāṇi paśyate//     \N2
%ākāśamadhye  ākāśopari vā dṛṣṭiṃ kṛtvā             dhyānakaraṇāt   sūryaṃ vinā sūryasaṃbaṃdhinī  sahasrāṇy api kiraṇāṇi paśyaṃti     \U1
%ākāśamadhye  ākāśopari vā dṛṣṭiṃ kṛtvā             dhyānakaraṇāt// sūrya  vinā sūryasaṃbaṃdhinī  sahasrakiraṇāvaliṃ     paśyati//     \U2
%------------------------------ 
%After having fixed the gaze on the space[-element?] or above the space[-element?], due to the execution of meditation he sees the sun without the group of thousand rays related to the sun. 
%------------------------------
\app{\lem[wit={ceteri}]{ākāśamadhye}
  \rdg[wit={P,B,L}]{\om}}
\app{\lem[wit={ceteri}]{ākāśopari}
  \rdg[wit={N1}]{ākāśoparī}}
\app{\lem[wit={ceteri}]{vā}
  \rdg[wit={E,P,B,L}]{\om}}
\app{\lem[wit={ceteri}]{dṛṣṭiṃ}
  \rdg[wit={B}]{dṛṣṭi}
  \rdg[wit={L}]{\om}}
\app{\lem[wit={ceteri}]{kṛtvā}
  \rdg[wit={B}]{kṛtvā ākāśamadhye}
  \rdg[wit={L}]{ākāśamadhye}}
dhyānakāraṇāt
\app{\lem[wit={ceteri}]{sūryaṃ}
  \rdg[wit={N2, U2}]{sūrya}}
vinā
\app{\lem[wit={P}]{sūryasaṃbaṃdhīnīṃ}
  \rdg[wit={ceteri}]{sūryasaṃbaṃdhinī}}
\app{\lem[wit={P}]{sahasrakiraṇāvalīṃ}
  \rdg[wit={U2}]{sahasrakiraṇāvaliṃ}
  \rdg[wit={B,L}]{sahasrakiraṇāvali}
  \rdg[wit={E}]{sahasrakiraṇapaṅktīḥ}
  \rdg[wit={N1,U1}]{sahasrāṇy api kīraṇāṇi}
  \rdg[wit={D,N2}]{sahasrāṇapi kiraṇāṇi}}
\app{\lem[wit={E,L,U2}]{paśyati}
  \rdg[wit={B,N1,D}]{paśyatī}
  \rdg[wit={N2}]{paśyate}
  \rdg[wit={P}]{pati}
  \rdg[wit={U1}]{paśyaṃti}}/
%-----------------------------
%atha vā śivopari vṛddhaṃ  saptadaśāṃgulapramāṇaṃ  tejaḥpuṃjalakṣyaṃ     karttavyam/ \E
%\om \P
%atha vā śiroparir urdhvaṃ saptadaśāṃgulapramāṇaṃ  tejaḥpūṃjaṃ lakṣaṇaṃ  kartavyaṃ/ \B
%atha vā śiropari ūrdhva---saptadaśāṃgulapramāṇaṃ  tejaḥpūṃjaṃ lakṣaṃ    kartavyaṃ/ \L
%atha kā śiropari ūrddhvaṃ saptadaśāṃgulapramāṇaṃ  tejā  puṃjalakṣaṃ      karttavyaṃ/ \N1
%atha vā śiropari ūrddhvaṃ saptadaśāṃgulapramāṇaṃ  tejā  puṃjalakṣyaṃ     karttavyaṃ// \D
%atha vā śiropari ūrddhvaṃ saptadaśāṃgulaṃ parāṇaṃ tejaḥpuṃjalakṣaṇaṃ    kartavyaṃ// \N2
%atha vā śiropari ūrddhaṃ  saptadaśāṃgulapramāṇaṃ  tejaḥpuṃjakaṃ lakṣyaṃ kartavyaṃ \U1 %%%278.jpg
%atha vā śiropari ūrddhaṃ  saptadaśāṃgulapramāṇa---tejaḥpuṃjaṃ lakṣyaṃ   karttavyaṃ// \U2
%-----------------------------
%Or the mass of light situated seventeen fingers wide distance above the head shall be made the fixation object. 
%-----------------------------
\note[type=source, labelb=156, lem={saptadaśāṃgulapramāṇaṃ}]{Ysv (PT): ūrddhvaṃ saptadaśāṅgulyaṃ pramāṇaṃ tejasā prabham |}
\app{\lem[wit={ceteri}]{atha vā}
  \rdg[wit={N1}]{atha kā}
  \rdg[wit={P}]{\om}}
\app{\lem[type=emendation, resp=egoscr, alt={śiropary}]{śiropar\skp{y-ū}}
  \rdg[wit={ceteri}]{\korr śiropari}
  \rdg[wit={E}]{śivopari}
  \rdg[wit={B}]{śiroparir}
  \rdg[wit={P}]{\om}
}\app{\lem[wit={ceteri}, alt={ūrddhvaṃ}]{\skm{y-ū}rddhvaṃ}
  \rdg[wit={L}]{ūrdhva°}
  \rdg[wit={B}]{urdhvam}
  \rdg[wit={U1,U2}]{ūrddhaṃ}
  \rdg[wit={E}]{vṛddhaṃ}
  \rdg[wit={P}]{\om}}
\app{\lem[wit={ceteri}]{saptadaśāṃgulapramāṇaṃ}
  \rdg[wit={N2}]{saptadaśāṃgulaṃ parāṇaṃ}
  \rdg[wit={U2}]{saptadaśāṃgulapramāṇa°}
  \rdg[wit={P}]{\om}}
\app{\lem[wit={E}]{tejaḥpuṃjalakṣyaṃ}
  \rdg[wit={P}]{tejaḥpūṃjaṃ lakṣaṇaṃ}
  \rdg[wit={L}]{tejaḥpūṃjaṃ lakṣaṃ}
  \rdg[wit={N1}]{tejā puṃjalakṣaṃ}
  \rdg[wit={D}]{tejā puṃjalakṣyaṃ}
  \rdg[wit={N2}]{tejaḥpuṃjalakṣaṇaṃ}
  \rdg[wit={U1}]{tejaḥpuṃjakaṃ lakṣyaṃ}
  \rdg[wit={U2}]{tejaḥpuṃjaṃ lakṣyaṃ}}
karttavyaṃ/
\end{prose}
\end{ekdosis}
\ekdpb*{}
%%%%%%%%%%%%%%%%%%%%%%%%%%%%%%%%%%%%%%%%%%
%%%%%%%%%%%%%%%%%%%%%%%%%%%%%%%%%%%%%%%%%%
%%%%%%%%PAGEBREAK%%%%%%%PAGEBREAK%%%%%%%%%
%%%%%%%%%%%%%%%%%%%%%%%%%%%%%%%%%%%%%%%%%%
%%%%%%%%%%%%%%%%PAGEBREAK%%%%%%%%%%%%%%%%%
%%%%%%%%%%%%%%%%%%%%%%%%%%%%%%%%%%%%%%%%%%
%%%%%%%%PAGEBREAK%%%%%%%PAGEBREAK%%%%%%%%%
%%%%%%%%%%%%%%%%%%%%%%%%%%%%%%%%%%%%%%%%%%
%%%%%%%%%%%%%%%%%%%%%%%%%%%%%%%%%%%%%%%%%%
%%%%%%%%%%%%%%%%%%%%%%%%%%%%%%%%%%%%%%%%%%
%%%%%%%%%%%%%%%%%%%%%%%%%%%%%%%%%%%%%%%%%%
%%%%%%%%PAGEBREAK%%%%%%%PAGEBREAK%%%%%%%%%
%%%%%%%%%%%%%%%%%%%%%%%%%%%%%%%%%%%%%%%%%%
%%%%%%%%%%%%%%%%PAGEBREAK%%%%%%%%%%%%%%%%%
%%%%%%%%%%%%%%%%%%%%%%%%%%%%%%%%%%%%%%%%%%
%%%%%%%%PAGEBREAK%%%%%%%PAGEBREAK%%%%%%%%%
%%%%%%%%%%%%%%%%%%%%%%%%%%%%%%%%%%%%%%%%%%
%%%%%%%%%%%%%%%%%%%%%%%%%%%%%%%%%%%%%%%%%%
%%%%%%%%%%%%%%%%%%%%%%%%%%%%%%%%%%%%%%%%%%
%%%%%%%%%%%%%%%%%%%%%%%%%%%%%%%%%%%%%%%%%%
%%%%%%%%PAGEBREAK%%%%%%%PAGEBREAK%%%%%%%%%
%%%%%%%%%%%%%%%%%%%%%%%%%%%%%%%%%%%%%%%%%%
%%%%%%%%%%%%%%%%PAGEBREAK%%%%%%%%%%%%%%%%%
%%%%%%%%%%%%%%%%%%%%%%%%%%%%%%%%%%%%%%%%%%
%%%%%%%%PAGEBREAK%%%%%%%PAGEBREAK%%%%%%%%%
%%%%%%%%%%%%%%%%%%%%%%%%%%%%%%%%%%%%%%%%%%
%%%%%%%%%%%%%%%%%%%%%%%%%%%%%%%%%%%%%%%%%%
\begin{ekdosis}
  \begin{prose}
%-----------------------------
%atha vā dṛṣṭer agre tatparaṃ svarṇākāraṃ  pṛthvītattvaṃ  lakṣyaṃ kartavyam/ \E
%atha vā dṛṣṭer agne taptasvarṇavarṇakāraṃ pṛthvītatvaṃ   lakṣyaṃ \P
%atha vā dṛṣṭer agne taptasuvarṇavarṇa-----pṛthivītatvaṃ  lakṣaṃ kartavyaṃ/ \B
%atha vā dṛṣṭer agne taptasuvarṇavarṇa-----pṛthītatvaṃ    lakṣaṃ kartavyaṃ/ \L
%atha vā dṛṣṭer ag..?taptavarṇākāraṃ       pṛthvītatvaṃ   lakṣaṃ karttavyaṃ/ \N1
%atha vā dṛṣṭer agre taptavarṇākāraṃ       pṛthvītatvaṃ   lakṣaṃ karttavyaṃ// \D %%%p.10 beginning
%atha vā dṛṣṭer agre taptavarṇākāraṃ       pṛthvītatvaṃ   lakṣaṇaṃ karttavyaṃ/ \N2
%atha vā dṛṣṭer agre taptavarṇākāraṃ       pṛthvītatvaṃ   lakṣyaṃ karttavyaṃ \U1
%atha vā dṛṣṭer agre taptasvarṇavarṇākāraṃ pṛthvīṃ tatvaṃ lakṣaṃ karttavyaṃ// \U2
%-----------------------------
%Or at the uppermost part of the focal point the earth-element appearing in the color of molten gold shall be made the fixation object.  
%-----------------------------
\noindent
\note[type=source, labelb=157, lem={pṛthvītattvaṃ}]{Ysv (PT): ūrddhvaṃ saptadaśāṅgulyaṃ pramāṇaṃ tejasā prabham | athavā pṛthivītattvaṃ taptakāñcanasannibham | dṛṣṭiragre tu karttavyaṃ lakṣam etad yat ātmanām | uktānāṃ yasya kasyaiva ekaśaḥ karaṇaṃ priye | balīpalitahīnaḥ syādauṣadhena vinā tathā |}
atha vā dṛṣṭe\skp{r-a}\app{\lem[wit={ceteri}, alt={agre}]{\skm{r-a}gre}
  \rdg[wit={P,B,L}]{agne}}
\app{\lem[wit={U2}]{taptasvarṇavarṇākāraṃ}
  \rdg[wit={P}]{taptasvarṇavarṇakāraṃ}
  \rdg[wit={E}]{tatparaṃ svarṇākāraṃ}
  \rdg[wit={B,L}]{taptasuvarṇavarṇa}
  \rdg[wit={N1,N2,D,U1}]{taptavarṇākāraṃ}}
\app{\lem[wit={E}]{pṛthvītattvaṃ}
  \rdg[wit={P}]{pṛthvītatvaṃ}
  \rdg[wit={B}]{pṛthivītatvaṃ}
  \rdg[wit={L}]{pṛthītatvaṃ}
  \rdg[wit={N1,N2,D,N2}]{pṛthvītatvaṃ}
  \rdg[wit={N2}]{pṛthvīṃ tatvaṃ}}
\app{\lem[wit={E,P,U1}]{lakṣyaṃ}
  \rdg[wit={B,L,N1,D,U2}]{lakṣaṃ}
  \rdg[wit={N2}]{lakṣaṇaṃ}}
\app{\lem[wit={ceteri}]{karttavyaṃ}
  \rdg[wit={P}]{\om}}/
%-----------------------------
%uktānāṃ lakṣyāṇāṃ  madhye yasya kasyāpy ekasya lakṣyakaraṇāt     valitapalitā dūre bhavanti/ \E
%uktānāṃ lakṣaṇānāṃ madhye yasya kasyāpy ekasya lakṣyakaraṇāt     valitapalitādidūre bhavati \P
%uktānāṃ lakṣaṇaṃ   madhye yasya kasyāpi kasya  lakṣakaraṇāt//    valitaṃ palitādi dūre bhavatī/ \B
%uktānāṃ lakṣaṇaṃ   madhye yasya kasyāpi kasya  lakṣakaraṇāt//    valitaṃ palitādi dūre bhavati// \L
%uktānāṃ lakṣyaṇāṃ  madhye yasya kasyāpy ekasya lakṣasya karaṇāt  valitapalitādidūre bhavati \N1
%uktānāṃ lakṣyaṇaṃ  madhye yasya kasyāp--ekasya lakṣasya karaṇāt  valitapalitādidūre bhavati// \D
%uktānāṃ lakṣāṇā----madhye yasya lasyāpy elasya lakṣaṇasya karaṇātvalitapalitādidūre bhavati/ \N2
%uktānāṃ lakṣyaṇāṃ  madhye yasya kasyāpi kasya  lakṣyasya karaṇā  valitapalitādidūre bhavati \U1
%uktānāṃ lakṣāṃ     madhye yasya kasyāpy ekasya lakṣyakaraṇāt     valitapalitādidūre bhavaṃti// \U2
%-----------------------------
%From the execution of [the yoga of] fixation onto the middle of anyone of the discussed fixation objects wrinkles and grey hair etc. are removed. 
%-----------------------------
uktānāṃ 
\app{\lem[wit={E}]{lakṣyāṇāṃ}
  \rdg[wit={U1,N1}]{lakṣyaṇāṃ}
  \rdg[wit={D}]{lakṣyaṇaṃ}
  \rdg[wit={P}]{lakṣaṇānāṃ}
  \rdg[wit={B,L}]{lakṣaṇaṃ}
  \rdg[wit={N2}]{lakṣāṇā°}
  \rdg[wit={U2}]{lakṣāṃ}}
madhye yasya
\app{\lem[wit={ceteri},alt={kasyāpy}]{kasyā\skp{py-e}}
  \rdg[wit={B,L,U1}]{kasyāpi}
  \rdg[wit={D}]{kasyāp°}
  \rdg[wit={N2}]{lasyāpy}
}\app{\lem[wit={ceteri}, alt={ekasya}]{\skm{py-e}kasya}
  \rdg[wit={B,L,U1}]{kasya}
  \rdg[wit={N2}]{elasya}}
\app{\lem[wit={ceteri},alt={lakṣya°}]{lakṣya}
  \rdg[wit={B,L}]{lakṣa°}
  \rdg[wit={N1,D}]{lakṣasya}
  \rdg[wit={N2}]{lakṣaṇasya}
  \rdg[wit={U1}]{lakṣyasya}
}\app{\lem[wit={ceteri}, alt={°karaṇāt}]{karaṇāt}
  \rdg[wit={U1}]{karaṇā}}
\app{\lem[wit={E}]{valitapalitā dūre}
  \rdg[wit={B,L}]{valitaṃ palitādi dūre}
  \rdg[wit={ceteri}]{valitapalitādidūre}}
\app{\lem[wit={E,U2}]{bhavanti}
  \rdg[wit={B}]{bhavatī}
  \rdg[wit={ceteri}]{bhavati}}/
%-----------------------------
%aṃgarogāḥ  vinauṣadhaṃ dūrī bhavanti/  samagrāḥ śatravaḥ  svapne pya mitran   nāyāṃti/     \E
%aṃgirogā   vinauṣadhaṃ dūre bhavati    samagrāḥ śatravaḥ  svapne pya mitratām ayāṃti  \P  %%%7646.jpg Z.1 
%aṃgirogādi vinauṣadhaṃ dūro bhavatī    samagrāḥ śatrave   svapne pya mitratām ayāṃti//   \B
%aṃgirogādi vinauṣadhaṃ dūro bhavati    samagrāḥ śatravo   svapne pya mitratām ayāṃtī     \L
%aṃgarogā   vinauṣadhaṃ dūre bhavaṃti/  samagrāḥ śatravaḥ  svapin eva mityaṃ   nāyāti/     \N1
%aṃgarogā   vinauṣadhaṃ dūre bhavaṃti// samagrāḥ śatravaḥ  svacan eva mityaṃ   nāyāti//     \D
%aṃgarogā   vinauṣadhaṃ dūre bhavati//  samagrā  śatravaḥ  svapin evan nityaṃ  nāyāti//    \N2
%aṃgarogā   vinauṣadhaṃ dūre bhavati    samagrāḥ śatravaḥ  svapin eva mitevaṃ  naiyati    sahasravarṣaparyaṃtam āyuṣyaṃ varddhate \U1
%aṃgarogā   vinauṣadhaṃ dūre bhavaṃti// samagra  śatravaḥ  svapne pi mitratām  āyāṃti//  sahasravarṣam āyur varddhate// \U2 ā-yānti= von ā-√yā=in einen Zustand ~, in eine Lage ~, in ein Verhältniss kommen, ~ gerathen; theilhaftig werden, erlangen; mit Acc. 
%-----------------------------
%Diseases of the limbs are removed without medical herbs. All enemies become friends while sleeping. The lifespan increases up to 1000 years. 
%-----------------------------
\note[type=source, labelb=158, lem={aṅgarogā}]{Ysv (PT): sarvarogāṇi naśyanti mitravac ca vaśī ripuḥ | jīved varṣasahasran tu sarvalokeṣu pūjitaḥ | jihvāgre prabhaved vidyā vinā śāstrāvalokanāt |}
\app{\lem[wit={ceteri}]{aṅgarogā}
  \rdg[wit={E}]{aṃgarogāḥ}
  \rdg[wit={B,L}]{aṃgirogādi}}
vinauṣadhaṃ
\app{\lem[wit={ceteri}]{dūre}
  \rdg[wit={E}]{dūrī}
  \rdg[wit={B,L}]{dūro}}
\app{\lem[wit={E,N1,D,U2}]{bhavanti}
  \rdg[wit={P,L,N2,U1}]{bhavati}
  \rdg[wit={B}]{bhavatī}}/
\app{\lem[wit={ceteri}]{samagrāḥ}
  \rdg[wit={N2}]{samagrā}
  \rdg[wit={U2}]{samagra°}}
\app{\lem[wit={ceteri}]{svapne}
  \rdg[wit={N1,N2,U1}]{svapin}
  \rdg[wit={D}]{svacan}}
\app{\lem[wit={U2}]{'pi}
  \rdg[wit={E,P,B,L}]{pya}
  \rdg[wit={N1,D,U1}]{eva}
  \rdg[wit={N2}]{evan}}
\app{\lem[wit={P,B,L,U2}]{mitratām}
  \rdg[wit={E}]{mitran}
  \rdg[wit={N1,D}]{mityaṃ}
  \rdg[wit={N2}]{nityaṃ}
  \rdg[wit={U1}]{mitevaṃ}}
\app{\lem[wit={P,B}]{ayāṃti}
  \rdg[wit={L}]{ayāṃtī}
  \rdg[wit={N2}]{āyāṃti}
  \rdg[wit={E}]{nāyāṃti}
  \rdg[wit={N1,N2,D}]{nāyāti}
  \rdg[wit={U1}]{naiyati}}/
%-----------------------------STEMMAPOINT!!!!------------------------------------
% sahasravarṣam āyur bhavati/ \E
% sahasravarṣam āyur varddhate \P  %%%7646.jpg Z.1 
% sahasravarṣam āyur vardhate// \B
% sahasravarṣam āyur vardhate// \L
% sahasravarṣaparyaṃtam āyuṣaṃ varddhate/ \N1
% sahasravarṣaparyaṃtam āyuṣaṃ varddhate// \D
% sahasravarṣaparyaṃtam āyuṣaṃ vardhate// \N2
% sahasravarṣaparyaṃtam āyuṣyaṃ varddhate \U1
% sahasravarṣam āyur varddhate// \U2 ā-yānti= von ā-√yā=in einen Zustand ~, in eine Lage ~, in ein Verhältniss kommen, ~ gerathen; theilhaftig werden, erlangen; mit Acc. 
%-----------------------------
% The lifespan increases up to 1000 years. 
%-----------------------------
\app{\lem[wit={N1,N2,D,U1}]{sahasravarṣaparyaṃtam}
  \rdg[wit={E,P,B,L,U2}]{sahasravarṣam}}
\app{\lem[wit={N1,N2,D}]{āyuṣaṃ}
  \rdg[wit={U1}]{āyuṣyaṃ}
  \rdg[wit={E,P,B,L,U2}]{āyur}}
vardhate/
%-----------------------------
%apaṭhitaṃ śāstraṃ jihvāgreṇoccarati/  etādṛśaṃ phalaṃ bahutaraṃ bhavati// \E
%apaṭhitaṃ śāstraṃ jihvāgreṇoccarati   etādṛśaṃ mitratāmāyāṃti sahasravarṣamāyur varddhate apaṭhitaṃ śāstraṃ jihvāgreṇoccarati etādṛśaṃ phalaṃ bahutaraṃ bhavati \P
%apaṭhitaṃ śāstraṃ jihvāgreṇoccaratī/  etādṛśaṃ phalaṃ bahutaraṃ bhavatī// \B
%apaṭhitaṃ śāstraṃ jihvāgreṇoccarati   etādṛśaṃ phalaṃ bahutaraṃ bhavaṃtī// \L
%apaṭhitaṃ śāstraṃ jihvāgreṇoccarate// etādṛśaṃ bahutaraṃ phalaṃ bhavati// \N1
%apaṭhitaṃ śāstraṃ jihvāgreṇoccarate// etādṛśaṃ bahutaraṃ phalaṃ bhavati// \D
%apathitaṃ śāstraṃ jihvāgreṇoccarate// etādṛśaṃ bahutaraṃ phalaṃ bhavati// \N2
%apathitaṃ śāstraṃ jihvāgreṇoccarate   etādṛśyaṃ bahutaraṃ phalaṃ bhavati \U1
%apathitaṃ śāstraṃ jihvāgreṇoccarati// etādṛśaṃ phalaṃ bahutaraṃ phalaṃ bhavati// \U2
%-----------------------------
%Non-recitable? teachings are enunciated by the tip of the tongue [of the practitioner?]. An abundance of such results arise.  
%-----------------------------
\app{\lem[wit={ceteri}]{apaṭhitaṃ}
  \rdg[wit={N2,U1,U2}]{apathitaṃ}}
śāstraṃ
jihvāgreṇocca\app{\lem[wit={N1,N2,D,U1}, alt={°rate}]{rate}
  \rdg[wit={E,B,L,U2}]{°rati}
  \rdg[wit={B}]{°ratī}}/
\app{\lem[wit={ceteri}]{etādṛśaṃ}
  \rdg[wit={U1}]{etādṛśyaṃ}}
\app{\lem[wit={N1,N2,D,U1}]{bahutaraṃ phalaṃ}
  \rdg[wit={E,P,B,L,U2}]{phalaṃ bahutaraṃ}}\dd{}
\note[type=philcomm, labelb=159, lem={etādṛśaṃ}]{Witness P includes a dittography of the previous lines here and reads: \textit{etādṛśaṃ mitratāmāyāṃti sahasravarṣamāyur varddhate apaṭhitaṃ śāstraṃ jihvāgreṇoccarati etādṛśaṃ phalaṃ bahutaraṃ bhavati |}.}
\end{prose}
\end{ekdosis}
%-----------------------------
%idānīm anyataraṃ lakṣyaṃ kathyate/ \E
%idānīṃ aṃtaraṃ lakṣyaṃ   kathyate \P
%idānīṃ antaralakṣaṃ      kartavyaṃ// \B
%idānīṃ aṃtaralakṣaṃ      kartavyaṃ// \L
%idānīṃ antaralakṣyakaṃ   kathyate// \N1
%idānīṃ antaralakṣyaṃ     kathyate// \D
%idānīṃ aṇtaralakṣyaṇaṃ   kathyate// \N2
%idānīṃ aṇtaralakṣyaṇaṃ   kathyate \U1
%idānīm ataraṃ lakṣyaṃ    kathyate// \U2
%-----------------------------
%Now the inner fixation objects are taught. 
%-----------------------------
\begin{ekdosis}
  \ekddiv{type=ed}
       \bigskip
       \centerline{\textrm{\small{[Antaralakṣya]}}}
          \bigskip
          \begin{prose}
    \app{\lem[wit={E,U2},alt={idānīm}]{idānī\skp{m-a}}
      \rdg[wit={ceteri}]{idānīṃ}
}\app{\lem[wit={D}, alt={antaralakṣyaṃ}]{\skm{m-a}ntaralakṣyaṃ}
  \rdg[wit={E}]{anyataraṃ lakṣyaṃ}
  \rdg[wit={P}]{aṃtaraṃ lakṣyaṃ}
  \rdg[wit={B,L}]{antaralakṣaṃ}
  \rdg[wit={N1}]{antaralakṣyakaṃ}
  \rdg[wit={N2,U1}]{aṇtaralakṣyaṇaṃ}
  \rdg[wit={U2}]{ataraṃ lakṣyaṃ}}
\app{\lem[wit={ceteri}]{kathyate}
  \rdg[wit={B,L}]{kartavyaṃ}}/    
%-----------------------------
%mūlakandasthāne brahmadaṇḍotpannā nāḍī śvetavarṇā   brahmadaṇḍaparyantam   ekā brahmanāḍī varttate/ \E
%mūlakaṃdasthāne brahmānaṃḍād utpannā   śvetavarṇā   brahmaraṃdhraparyaṃtaṃ ekā brahmanāḍī varttate   \P
%mūlakaṃ sthāne  brahmānaṃḍād utpannā   śvetāvarṇā   brahmaraṃdhraparyaṃtaṃ ekā nāḍī       vartate/     \B
%mūlakaṃdasthāne brahmānaṃdād utpannā   śvetāvarṇā   brahmaraṃdhraparyaṃtaṃ ekanāḍī        vartate/     \L
%mūlakaṃdasthāne brahmadaṃḍa ityannā    śvetavarṇā   brahmaraṃdhraparyaṃtaṃ ekā brahmanāḍī varttate/ \N1
%mūlakaṃdasthāne brahmadaṃḍād utpannā   śvetavarṇā// brahmaraṃdhraparyaṃtaṃ ekā brahmanāḍī varttate// \D
%mūlakaṃdasthāne brahmadaṇḍad ūtpannā   śvetavarṇā   brahmaraṃdhraparyaṃtaṃ ekā brahmanāḍī varttate/ \N2
%mūlakaṃdasthāne brahmadaṇād ūtpannaḥ   śvetavarṇāṃ  brahmaraṃdhraparyaṃtaṃ ekā brahmanāḍī varttate \U1
%mūlakaṃdasthāne brahmadaṇḍād utpannā   śvetavarṇā   brahmaraṃdhraparyaṃtaṃ ekā brahmanāḍī varttate// \U2
%-----------------------------
%At the location of the root bulp rising from the staff of Brahma up to the aperture of Brahma exists the one white coloured Brahma channel. 
%-----------------------------
\note[type=source, labelb=160, lem={mūlakandasthāne}]{Ysv (PT): mūlakandotthatalato brahmanāḍīsamudbhavā | śvetavarṇā brahmarandhraparyantam eva tiṣṭhati | eṣā tu brahmarandhrākhyā tanmadhye varttate parā |}
\app{\lem[wit={ceteri}]{mūlakandasthāne}
  \rdg[wit={P}]{mūlakaṃ sthāne}}
\app{\lem[wit={ceteri}]{brahmadaṇḍād\skp{-}utpannā}
  \rdg[wit={E}]{brahmadaṇḍotpannā nāḍī}
  \rdg[wit={N1}]{brahmadaṃḍa ityannā}
  \rdg[wit={N2}]{brahmadaṇḍad ūtpannā}
  \rdg[wit={U1}]{brahmadaṇād ūtpannaḥ}}
 śvetavarṇā
\app{\lem[wit={ceteri}]{brahmaraṃdhraparyaṃtaṃ}
  \rdg[wit={E}]{brahmadaṇḍaparyantam}} 
\app{\lem[wit={ceteri}]{ekā brahmanāḍī}
  \rdg[wit={B}]{ekā nāḍī}
  \rdg[wit={L}]{ekanāḍī}}
vartate/
%-----------------------------
%brahmanāḍīmadhye kamalatantusamānākārā koṭisūryavidyutsamaprabhā ūrdhvaṃ calati/ \E
%brahmanāḍīmadhye kamalataṃ samānākārā  koṭisūryavidyutsamaprabhā ūrdhvaṃ calati \P
%brahmanāḍīmadhye kamalataṃtusamānākārā koṭisūryavidyutsabhāprabhā ūrdhvaṃ calati/ \B
%brahmanāḍīmadhye kamalataṃtusamānākārā koṭisūryavidyutsabhāprabhā ūrdhvaṃ calati/ \L
%brahmanāḍīmadhye kamalatantusamānākārā koṭisūryavidyutsamaprabhā ūrdhvaṃ calati/ \N1
%brahmanāḍīmadhye kamalataṃtusamānākārā koṭisūryavidyutsamaprabhā ūrdhvaṃ calati// \D
%\om                                                                              \N2
%brahmanāḍīmadhye kamalatantusamānākārā koṭisūryavidyutsamaprabhā rdhvaṃ ccalati  \U1
%brahmanāḍīmadhye kamalataṃtusamānākārā koṭisūryavidyutsamaprabhā// urdhvaṃ calati  \U2 %%%412.jpg 
%-----------------------------
%The interior of the Brahma channel, which equals a pale-red string shining like 10 million suns, goes upwards. 
%-----------------------------
\note[type=source, labelb=161, lem={kamalatantusamānākārā}]{Ysv (PT): padmatantusamākārā koṭisūryataḍitprabhā | calaty ūrddhaṃ mahāmūrttir asya dhyānād bhavec chivaḥ | aṇimādy aṣṭasiddhis tu samagreṇa prasīdati |}
\app{\lem[wit={ceteri}]{brahmanāḍīmadhye}
  \rdg[wit={N2}]{\om}}
\app{\lem[wit={ceteri}]{kamalatantusamānākārā}
  \rdg[wit={P}]{kamalataṃ samānākārā}
  \rdg[wit={N2}]{\om}}
koṭisūryavidyutsa\app{\lem[wit={ceteri},alt={°maprabhā}]{maprabhā}
  \rdg[wit={B,L}]{°bhāprabhā}
  \rdg[wit={N2}]{\om}}
\app{\lem[wit={ceteri}]{ūrdhvaṃ}
  \rdg[wit={U1}]{°rdhvaṃ}
  \rdg[wit={N2}]{urdhvaṃ}}
\app{\lem[wit={ceteri}]{calati}
  \rdg[wit={N2}]{\om}}/
%-----------------------------
%etādṛśy ekā mūrttir varttate/  tan    mūrter dhyānakāraṇāt      aṣṭamahāsiddhayo  'ṇimādayas   tasya                                                                               puruṣasya samīpam āgatya tiṣṭhanti// \E
%etādṛśy ekā mūrttir vartate    tasyā  mūrter dhyānakaraṇāt      aṣṭamahāsiddhayo   ṇimādyāḥ    aṇimā-mahimā-laghīmā-girimā-dure dīya vā            dure  stutvā parakāyapraveśītā   puruṣasya samīm   āgatya tiṣṭhaṃti \P
%etādṛśy ekā mūrttir varttate/  tasyā  mūrte  dhyānakaraṇāt//    aṣṭamahāsiddhayo// aṇimādyāḥ// aṇimā-mahimā-laghimā-girimā-dure vā yadi vā yadi vā dure  śrutvā parakāyāpraveśitā// puruṣasya samīpem āgatya tiṣṭhati// \B
%etādṛśy ekā mūrttir varttate/  tasyā  mūrter dhyānakaraṇāt//    aṣṭamahāsiddhayo   aṇimādyāḥ// aṇimā-mahimā-laghimā-garimā-dure vā yadi         vā ddure śrutvā parakāyāpraveśitā   puruṣasya samīpam āgatya tiṣṭhati// \L
%etādṛśī ekā mūrttir varttate/  tasyāḥ mūrtter dhyānakāraṇāt/    aṇimādīsiddhiḥ                                                                                                     puruṣasya samīpe? āgatya tiṣṭhanti// \N1
%etādṛśī ekā mūrttir varttate// tasyā  mūrtter dhyānakāraṇāt//   aṇimādyaṣṭasiddhiḥ                                                                                                 puruṣasya samīpe  āgatya tiṣṭhati// \D
%\om                            tasyā  mūrtter dhyānakaraṇāc                                                                                        \N2
%                                                                aṇimādyaṣṭasiddhiḥ                                                                                                 puruṣasya sāmīpe  āgatya tiṣṭhati \U1
%etādṛśy ekā mūrttir varttate// tasyā   mūrter dhyānakaraṇāt//   aṣṭamahāsiddhayo aṇimādyāḥ//                                                                                       puruṣasya samīpam āgamya tiṣṭhati// \U2
%-----------------------------
%A particular manifestation exists as such. Due to the execution of meditation on this manifestation, the eight great siddhis of humans beginning with aṇima etc. become established after one has entered into [the manufestation's] imminence. 
%-----------------------------
\app{\lem[wit={ceteri}]{etādṛśy\skp{-}ekā}
  \rdg[wit={N1,D}]{etādṛśī ekā}
  \rdg[wit={U1,N2}]{\om}}
\app{\lem[wit={ceteri}, alt={mūrtir}]{mūrti\skp{r-va}}
  \rdg[wit={U1,N2}]{\om}
}\app{\lem[wit={ceteri}, alt={vartate}]{\skm{r-va}rtate}
  \rdg[wit={U1,N2}]{\om}}/
\end{prose}
\end{ekdosis}
\ekdpb*{}
%%%%%%%%%%%%%%%%%%%%%%%%%%%%%%%%%%%%%%%%%%
%%%%%%%%%%%%%%%%%%%%%%%%%%%%%%%%%%%%%%%%%%
%%%%%%%%PAGEBREAK%%%%%%%PAGEBREAK%%%%%%%%%
%%%%%%%%%%%%%%%%%%%%%%%%%%%%%%%%%%%%%%%%%%
%%%%%%%%%%%%%%%%PAGEBREAK%%%%%%%%%%%%%%%%%
%%%%%%%%%%%%%%%%%%%%%%%%%%%%%%%%%%%%%%%%%%
%%%%%%%%PAGEBREAK%%%%%%%PAGEBREAK%%%%%%%%%
%%%%%%%%%%%%%%%%%%%%%%%%%%%%%%%%%%%%%%%%%%
%%%%%%%%%%%%%%%%%%%%%%%%%%%%%%%%%%%%%%%%%%
%%%%%%%%%%%%%%%%%%%%%%%%%%%%%%%%%%%%%%%%%%
%%%%%%%%%%%%%%%%%%%%%%%%%%%%%%%%%%%%%%%%%%
%%%%%%%%PAGEBREAK%%%%%%%PAGEBREAK%%%%%%%%%
%%%%%%%%%%%%%%%%%%%%%%%%%%%%%%%%%%%%%%%%%%
%%%%%%%%%%%%%%%%PAGEBREAK%%%%%%%%%%%%%%%%%
%%%%%%%%%%%%%%%%%%%%%%%%%%%%%%%%%%%%%%%%%%
%%%%%%%%PAGEBREAK%%%%%%%PAGEBREAK%%%%%%%%%
%%%%%%%%%%%%%%%%%%%%%%%%%%%%%%%%%%%%%%%%%%
%%%%%%%%%%%%%%%%%%%%%%%%%%%%%%%%%%%%%%%%%%
%%%%%%%%%%%%%%%%%%%%%%%%%%%%%%%%%%%%%%%%%%
%%%%%%%%%%%%%%%%%%%%%%%%%%%%%%%%%%%%%%%%%%
%%%%%%%%PAGEBREAK%%%%%%%PAGEBREAK%%%%%%%%%
%%%%%%%%%%%%%%%%%%%%%%%%%%%%%%%%%%%%%%%%%%
%%%%%%%%%%%%%%%%PAGEBREAK%%%%%%%%%%%%%%%%%
%%%%%%%%%%%%%%%%%%%%%%%%%%%%%%%%%%%%%%%%%%
%%%%%%%%PAGEBREAK%%%%%%%PAGEBREAK%%%%%%%%%
%%%%%%%%%%%%%%%%%%%%%%%%%%%%%%%%%%%%%%%%%%
%%%%%%%%%%%%%%%%%%%%%%%%%%%%%%%%%%%%%%%%%%
\begin{ekdosis}
  \begin{prose}
    \noindent
\app{\lem[wit={ceteri}]{tasyā}
  \rdg[wit={N1}]{tasyāḥ}
  \rdg[wit={E}]{tan}
  \rdg[wit={U1}]{\om}}
\app{\lem[wit={ceteri}, alt={mūrter}]{mūrte\skp{r-dhyā}}
  \rdg[wit={B}]{mūrte}
  \rdg[wit={U1}]{\om}
}\app{\lem[wit={ceteri}, alt={dhyāna°}]{\skm{r-dhyā}na}
  \rdg[wit={U1}]{\om}
}\app{\lem[resp=egoscr, type=emendation, alt={kāraṇād}]{kāraṇā\skp{d-a}}
  \rdg[wit={ceteri}]{\korr karaṇāt}
  \rdg[wit={N2}]{dhyānakaraṇāc°}
  \rdg[wit={U1}]{\om}
}\app{\lem[wit={U2}, alt={aṣṭamahāsiddhayo}]{\skm{d-a}ṣṭamahāsiddhayo}
  \rdg[wit={U1,D}]{aṇimādyaṣṭasiddhiḥ}
  \rdg[wit={N1}]{aṇimādīsiddhiḥ}
  \rdg[wit={E,P,B,L}]{aṣṭamahāsiddhayo}
  \rdg[wit={N2}]{\om}}
\app{\lem[wit={P}]{'ṇimādyāḥ}
  \rdg[wit={E}]{'ṇimādayas tasya}
  \rdg[wit={B,L,U2}]{aṇimādyāḥ}
  \rdg[wit={ceteri}]{\om}}
\note[type=philcomm, labelb=162, lem={'ṇimādyāḥ}]{Witnesses P, B and L add a incomplete list of eight supernatural powers here: \textit{aṇimāmahimālaghimāgarimādure vā yadi vā dure śrutvā parakāyāpraveśitā} | Since the list is incomplete and corrupt and stemmatically a later addition, I have decided not to include it into the edition's text.}
\app{\lem[wit={ceteri}]{puruṣasya}
  \rdg[wit={N2}]{\om}}
\app{\lem[wit={N1,D}]{samīpe}
  \rdg[wit={U1}]{sāmīpe}
  \rdg[wit={B}]{samīpem}
  \rdg[wit={E,L,U2}]{samīpam}
  \rdg[wit={P}]{samīm}
  \rdg[wit={N2}]{\om}}
\app{\lem[wit={ceteri}]{āgatya}
  \rdg[wit={U2}]{āgamya}
  \rdg[wit={N2}]{\om}}
\app{\lem[wit={E,P,N1}]{tiṣṭhanti}
  \rdg[wit={ceteri}]{tiṣṭhati}
  \rdg[wit={N2}]{\om}}/%
%[Aṇima (the ability to reduce size to the size of the smallest particle), Mahimā (the ability to expand one's body to an infinitely large size), Laghimā (the ability to become weightless or lighter than air), Garimā (the ability to become heavy or dense), Dūraśravaṇa (Hearing things far away), Dūradarśanam (Seeing things far away) Parakāya praveśitā: Entering the bodies of others.
%-----------------------------
%atha vā lalāṭopary ākāśamadhye śuklasadṛśasya tejaso dhyānakāraṇāt       śarīrasambandhinaḥ  kuṣṭhādayo rogā  naśyanti/    āyur vṛddhir bhavati/  \E
%atha vā lalāṭopari ākāśamadhye śuklasadṛśasya tejaso dhyānakāraṇāt       śarīrasaṃbaṃdhinaḥ  kuṣṭhādayo rogāḥ naśyaṃtī     āyur vṛddhir bhavati   \P  %%%7647.jpg
%atha vā lalāṭopari ākāśamadhye śuklasadṛśasya tejaso dhyānakāraṇāt//     charīrasambandhinaḥ kuṣṭhādayo rogā  naśyaṃtī//   āyur vṛddhir bhavatī   \B 
%atha vā lalāṭopari ākāśamadhye śuklasadṛśasya tejaso dhyānakāraṇāt       charīrasambandhinaḥ kuṣṭhādayo rogā  naśyaṃti//   āyur vṛddhir bhavati// \L
%atha vā lalāṭopari ākāśamadhye śuklasadṛśasya tejaso dhyānakāraṇāt       śarīrasambandhī     kuṣṭhādayo rogāḥ naśyaṃti/    āyur vṛddhir bhavati/  \N1
%atha vā lalāṭopari ākāśamadhye śuklasadṛśasya tejaso dhyānakāraṇāt       śarīrasaṃbaṃdhī     kuṣṭādayo  rogāḥ naśyaṃti//   āyur vṛddhir bhavati//  \D
%                                                                         charīrasaṃbaṃdhi----kuṣṭadayo  rogāḥ naśyaṃti     āyur vṛddi   bhavati/  \N2
%atha vā lalāṭoparī ākāśamadhye śuklasadṛśasya tejo   dhyānakāraṇāt       śarīrasambaṃdhī     kuṣṭhādayo rogā  naśyaṃti     āyur vṛddhir bhavati   \U1 %%%279.jpg
%atha vā lalāṭoparī ākāśamadhye śuklasadṛśasya tejaso dhyānakāraṇāt//     śarīrasambaṃdhinaḥ  kuṣṭhādayo rogā  naśyaṃti//   āyur vṛddhir bhavati//  \U2
%-----------------------------
%Or from the execution of meditation onto the bright light at the centre within the space at the forehead diseases related to the body beginning with leprosy vanish. Lifeforce increases.   
%-----------------------------
\note[type=source, labelb=163, lem={lalāṭopari}]{Ysv (PT): lalāṭopari vā dhyātvā candraṃ vā jyotir īśvaram | nāśayet kuṣṭharogādīn mahāyuṣmān śivaḥ paraḥ |}
\app{\lem[wit={ceteri}]{atha vā}
  \rdg[wit={N2}]{\om}}
\app{\lem[wit={E}, alt={lalāṭopary}]{lalāṭopa\skp{ry-ā}}
  \rdg[wit={ceteri}]{lalāṭopari}
  \rdg[wit={N2}]{\om}
}\app{\lem[wit={ceteri},alt={ākāśamadhye}]{\skm{ry-ā}kāśamadhye}
  \rdg[wit={N2}]{\om}}
\app{\lem[wit={ceteri}]{śuklasadṛśasya}
  \rdg[wit={N2}]{\om}}
\app{\lem[wit={ceteri}]{tejaso}
  \rdg[wit={N2}]{\om}}
\app{\lem[resp=egoscr, type=emendation, alt={dhyānakaraṇāc}]{dhyānakāraṇā\skp{c-cha}}
  \rdg[wit={ceteri}]{\korr dhyānakāraṇāt}
  \rdg[wit={N2}]{\om}
}\app{\lem[wit={B,L}, alt={śarīra°}]{\skm{c-cha}rīra}
  \rdg[wit={ceteri}]{śarīra°}
}\app{\lem[wit={E,P,B,L,U2}, alt={°sambandhinaḥ}]{sambandhinaḥ}
    \rdg[wit={N1,D,U1}]{°sambandhī}
    \rdg[wit={U2}]{saṃbaṃdhi}}
\app{\lem[wit={ceteri}]{kuṣṭhādayo}
  \rdg[wit={D,N2}]{kuṣṭādayo}}
\app{\lem[wit={ceteri}]{rogā}
  \rdg[wit={P,N1,D,N2}]{rogāḥ}}
\app{\lem[wit={ceteri}]{naśyanti}
  \rdg[wit={P,B}]{naśyaṃtī}}
āyur-vṛddhir-bhavati/ 
%-----------------------------
%          bhruvor madhye  tiriktavarṇasyātisthūlasya     tejaso dhyānakāraṇād bahulānāṃ   pārthivānāṃ tatpuruṣāṇāṃ ca vallabho bhavati/ jagadvallabho pi bhavati/      \E
%atha vā   bhruvor madhye  tiraktavarṇasyātisthūlasya     tejaso dhyānakaraṇāt   sakalānāṃ pārthivapuruṣāṇāṃ           vallabho bhavati          \P
%atha vā// bhruvor madhye 'tiraktavarṇasyātisthūlasya     tejaso dhyānaṃ karaṇāt-sakalānāṃ pārthivapuruṣāṇāṃ vallabho bhavati/         \B DSCN7163.jpg Z.1
%atha vā// bhruvor madhye 'tiraktavarṇasyātisthūlasya     tejaso dhyānakaraṇāt   sakalānāṃ pārthivapuruṣāṇāṃ vallabho bhavati/         \L
%atha vā   bhruvor madhye 'tiraktavarṇasyātisthūlasya     tejaso dhyānakaraṇāt-sakalānāṃ   pārthivapuruṣāṇāṃ vallabho bhavati/           \N1
%atha vā   bhruvor madhye 'tiraktavarṇasyātisthūlasya     tejaso dhyānakaraṇāt-sakalānā    pārthivapuruṣāṇāṃ vallabho bhavati             \D %%%p.10 verso
%atha vā   bhruvor madhye  tiraktavarṇasyātisthūlasya     tejaso dhyānakaraṇāt-sakālānāṃ   pārthivapuruṣāṇāṃ vallabho bhavati/             \N2
%atha vā   bhruvor madhye  tiraktavarṇasyātī sthalasya    tejaso dhyānakaraṇāt sakalānāṃ   pārthivapuruṣāṇāṃ vallabho bhavati/          \U1
%atha vā   bṛvor   madhye atiraktavarṇasya 'tisthūlasyaḥ  tejāso dhyānakaraṇāt sakalānāṃ   pārthivapuruṣāṇāṃ vallabho bhavati         \U2
%-----------------------------
%Or because of executing meditation on the middle of the eyebrows [of which there is] a very subtle and red colored light, he is one who is beloved among all royal people.    
%-----------------------------
\note[type=source, labelb=164, lem={bhruvor madhye}]{Ysv (PT): bhruvor madhye 'thavā dhyātvā arkantu teja īśvaram |  sthiradṛṣṭau rājapūjyo jīvanmuktaḥ śivo yathā | ātmānam ātmarūpaṃ hi dhyātvā yo niṣkriyo bhavet | nirāśīryatatattvo 'yaṃ itaro na nṛpasthitiḥ |}
\app{\lem[wit={ceteri}]{atha vā}
  \rdg[wit={E}]{\om}}
\app{\lem[wit={ceteri}, alt={bhruvor}]{bhruvo\skp{r-ma}}
  \rdg[wit={U2}]{bṛvor}
}\skm{r-ma}dhye
\app{\lem[wit={ceteri}, alt={'tirakta°}]{'tirakta}
  \rdg[wit={U2}]{atirakta°}
  \rdg[wit={E}]{tirikta°}
}\app{\lem[wit={ceteri}]{varṇasyātisthūlasya}
  \rdg[wit={U1}]{varṇasyātī sthalasya}
  \rdg[wit={U2}]{'tisthūlasyaḥ}}
tejaso
\app{\lem[wit={ceteri}, alt={dhyānakaraṇāt}]{dhyānakaraṇā\skp{t-sa}}
  \rdg[wit={B}]{dhyānaṃ karaṇāt}
  \rdg[wit={E}]{dhyānakāraṇād}
}\app{\lem[wit={ceteri}]{\skm{t-sa}kālānāṃ}
  \rdg[wit={D}]{sakalānā}
  \rdg[wit={E}]{bahulānāṃ}}
pārthi\app{\lem[wit={ceteri},alt={°vapuruṣāṇāṃ}]{vapuruṣāṇāṃ}
  \rdg[wit={E}]{°vānāṃ tatpuruṣāṇāṃ ca}}
vallabho\app{\lem[wit={ceteri}]{bhavati}
  \rdg[wit={E}]{bhavati | jagad vallabho pi bhavati}}/
%-----------------------------
%asya puruṣasyāvalokanena sarveṣāṃ dṛṣṭiḥ sthirā bhavati// \E
%taṃ  puruṣaṃ        pratisarveṣāṃ dṛṣṭiḥ sthirā bhavati  \P
%taṃ  puruṣaṃ        pratisarveṣāṃ dṛṣṭisthirā bhavatī// \B
%taṃ  puruṣa         pratisarveṣāṃ dṛṣṭisthirā bhavati// \L
%taṃ  puruṣaṃ dṛṣṭvā      sarveṣāṃ dṛṣṭisthirā bhavati// \N1
%taṃ  puruṣaṃ dṛṣṭvā      sarveṣāṃ dṛṣṭisthirā bhavati// \D
%taṃ  puruṣaṃ dṛṣṭā       sarveṣāṃ dṛṣṭisthirā bhavati// \N2
%taṃ  puruṣaṃ dṛṣṭvā      sarveṣāṃ dṛṣṭisthirā bhavati \U1
%taṃ  puruṣaṃ        pratisarveṣāṃ dṛṣṭisthirā bhavati// \U2
%----------------------------
%Having seen this person, everybody's gaze is fixed onto him. 
%-----------------------------
\app{\lem[wit={ceteri}]{taṃ}
  \rdg[wit={E}]{asya}}
\app{\lem[wit={N1,D,U1}]{puruṣaṃ dṛṣṭvā}
  \rdg[wit={N2}]{puruṣaṃ dṛṣṭā}
  \rdg[wit={P,B}]{puruṣaṃ}
  \rdg[wit={L}]{puruṣa°}
  \rdg[wit={E}]{puruṣasyāvalokanena}}
\app{\lem[wit={E,N1,D,N2,U1}]{sarveṣāṃ}
  \rdg[wit={ceteri}]{pratisarveṣāṃ}}
\app{\lem[wit={ceteri}]{dṛṣṭisthirā}
  \rdg[wit={E,P}]{dṛṣṭiḥ sthirā}}
\app{\lem[wit={ceteri}]{bhavati}
  \rdg[wit={B}]{bhavati}}\dd{}
\end{prose}
\end{ekdosis}
%-----------------------------
%-----------------------------
%-----------------------------
\begin{ekdosis}
  \ekddiv{type=ed}
 \bigskip
 \centerline{\textrm{\small{[The Ten Main Bodily Channels]}}}
  \bigskip
  \begin{prose}   
\note[type=source, labelb=165, lem={nāḍīnāṃ}]{SSP: atha nāḍīnāṃ daśadvārāṇi iḍā piṅgalā ca nāsādvārayor vahataḥ | gāndhārī hastijihvikā ca cakṣurdvārayor vahataḥ | pūṣā yaśasvinī ca karṇadvārayor vahataḥ | alambuṣā ānane vahati | kuhūr gudadvāre vahati | śaṃkhinī liṅgadvāre vahati | suṣumṇā madhyadeśe vahati | sā daṇḍamārgeṇa brahmarandhraparyantaṃ vahati | evaṃ daśanāḍyo daśadvāreṣu vahanti | anyāḥ sarvanāḍyo romakūpeṣu vahanti ||1.66||}    
\note[type=source, labelb=166, lem={nāḍīnāṃ}]{Ysv (PT): idānīṃ śṛṇu nāḍīnāṃ bhedaṃ vakṣyāmi siddhidam | meruvāhye iḍānāmnī piṅgalayā samanvitā | suṣumnā bhānumārgeṇa brahmadvārāvadhi sthitā | sarasvatī sugandhā tu gāndhārī hastijihvakā | jñātavyā karṇayormadhye netrayoś ca tathāntimā | pūṣā cālambuṣā ceti mūlasthā kutracit tathā | liṅgadvārādiḍāmārge brahmasthānāvadhi priye | nāḍyantaṃ pratilomeṣu sahasrāṇāṃ dvisaptatiḥ |}
%-----------------------------   
%idānīṃ śarīramadhye nāḍīnāṃ       bhedāḥ   kathyante  daśamukhyanāḍyaḥ/ \E
%idānīṃ śarīramadhye nāḍīnāṃ       bhedāḥ   kathyaṃte  daśamukhyānāḍyaḥ \P
%idānī  śarīramadhye nāḍī----------bhedaḥ   kathyate// daśamukhyenāḍyā \B
%idānī  śarīramadhye nāḍī----------bhedaḥ   kathyate// daśamukhyānāḍayas... \L
%idānīṃ śarīramadhye nāḍīnām aparo bhedaḥ   kathyate// daśamukhyanādhyaḥ/ \N1
%idānīṃ śarīramadhye nāḍīnām aparo bhedaḥ// kathyaṃte  daśamukhyānādhyaḥ// \D
%idānī  śarīramadhye nāḍīnām aparo bhedāḥ   kathyate// daśamukhyanāḍyaḥ// \N2
%idānīṃ śarīramadhye nāḍīnām aparo bhedāḥ   kathyate   daśamukhyanāḍyas \U1
%idānīṃ śarīramadhye nāḍīnaṃ       bhedaḥ   kathyate   daśamukhyanāḍyaḥ// \U2
%-----------------------------
%Now the divisions of channels within the body are explained. There are ten primary channels. 
%-----------------------------
\app{\lem[wit={ceteri}]{idānīṃ}
  \rdg[wit={L,B,N2}]{idānī}}
śarīramadhye 
\app{\lem[wit={ceteri}]{nāḍīnāṃ}
  \rdg[wit={B,L}]{nāḍī°}
  \rdg[wit={N1,N2,D,U1}]{nāḍīnām aparo}}
\app{\lem[wit={ceteri}]{bhedāḥ}
  \rdg[wit={B,L,N1,D}]{bhedaḥ}}
\app{\lem[wit={E,P,N2,U1}]{kathyante}
  \rdg[wit={ceteri}]{kathyate}}/
\app{\lem[wit={E,N2,U1,U2}]{daśamukhyanāḍyaḥ}
  \rdg[wit={P}]{daśamukhyānāḍyaḥ}
  \rdg[wit={B}]{daśamukhyenāḍyā}
  \rdg[wit={L}]{daśamukhyānāḍayas}
  \rdg[wit={N1,D}]{daśamukhyanādhyaḥ}}/
\end{prose}
\end{ekdosis}
\ekdpb*{}
%%%%%%%%%%%%%%%%%%%%%%%%%%%%%%%%%%%%%%%%%%
%%%%%%%%%%%%%%%%%%%%%%%%%%%%%%%%%%%%%%%%%%
%%%%%%%%PAGEBREAK%%%%%%%PAGEBREAK%%%%%%%%%
%%%%%%%%%%%%%%%%%%%%%%%%%%%%%%%%%%%%%%%%%%
%%%%%%%%%%%%%%%%PAGEBREAK%%%%%%%%%%%%%%%%%
%%%%%%%%%%%%%%%%%%%%%%%%%%%%%%%%%%%%%%%%%%
%%%%%%%%PAGEBREAK%%%%%%%PAGEBREAK%%%%%%%%%
%%%%%%%%%%%%%%%%%%%%%%%%%%%%%%%%%%%%%%%%%%
%%%%%%%%%%%%%%%%%%%%%%%%%%%%%%%%%%%%%%%%%%
%%%%%%%%%%%%%%%%%%%%%%%%%%%%%%%%%%%%%%%%%%
%%%%%%%%%%%%%%%%%%%%%%%%%%%%%%%%%%%%%%%%%%
%%%%%%%%PAGEBREAK%%%%%%%PAGEBREAK%%%%%%%%%
%%%%%%%%%%%%%%%%%%%%%%%%%%%%%%%%%%%%%%%%%%
%%%%%%%%%%%%%%%%PAGEBREAK%%%%%%%%%%%%%%%%%
%%%%%%%%%%%%%%%%%%%%%%%%%%%%%%%%%%%%%%%%%%
%%%%%%%%PAGEBREAK%%%%%%%PAGEBREAK%%%%%%%%%
%%%%%%%%%%%%%%%%%%%%%%%%%%%%%%%%%%%%%%%%%%
%%%%%%%%%%%%%%%%%%%%%%%%%%%%%%%%%%%%%%%%%%
%%%%%%%%%%%%%%%%%%%%%%%%%%%%%%%%%%%%%%%%%%
%%%%%%%%%%%%%%%%%%%%%%%%%%%%%%%%%%%%%%%%%%
%%%%%%%%PAGEBREAK%%%%%%%PAGEBREAK%%%%%%%%%
%%%%%%%%%%%%%%%%%%%%%%%%%%%%%%%%%%%%%%%%%%
%%%%%%%%%%%%%%%%PAGEBREAK%%%%%%%%%%%%%%%%%
%%%%%%%%%%%%%%%%%%%%%%%%%%%%%%%%%%%%%%%%%%
%%%%%%%%PAGEBREAK%%%%%%%PAGEBREAK%%%%%%%%%
%%%%%%%%%%%%%%%%%%%%%%%%%%%%%%%%%%%%%%%%%%
%%%%%%%%%%%%%%%%%%%%%%%%%%%%%%%%%%%%%%%%%%
\begin{ekdosis}
  \begin{prose}
    \noindent
%----------------------------- 
%tanmadhye dvayam       iḍā  piṃgalāsaṃjñakaṃ       nāsādvāre tiṣṭhati/ \E
%tanmadhye nāḍīdvayaṃ   idāṃ piṃgalāsaṃjñakaṃ       nāsādvāre tiṣṭhati  \P
%tanmadhye nāḍīdvayaṃ/  idāpiṃgalāsaṃjñīkāḥ         nāsādvāre tiṣṭhati//  \B
%tanmadhye nāḍīdvayaṃ   idāpiṃgalāsaṃjñīkāḥ         nāsādvāre tiṣṭhati//  \L
%tanmadhye nāḍīdvayam/  idāpiṃgalāsaṃjñakaṃ         nāsādvāre tiṣṭhati//  \N1
%tanmadhye nāḍīdvayaṃ   idāpiṃgalāsaṃjñakaṃ         nāsānāsādvāre tiṣṭhati//  \D
%tanmadhye nāḍīdvayam/  idānīṃ piṃgalāsaṃjñakaṃ     nāsādvāre tiṣṭhati//  \N2
%tanmadhye nāḍīdvayaṃ   idāpiṃgalāsaṃjñākaṃ         nāsādvāre tiṣṭhati  \U1
%tanmadhye nāḍidvayaṃ// idā// piṃgalā// saṃjñākaṃ// nāsādvāre tiṣṭhati//  \U2
%-----------------------------
%Among them the pair of channels designated Idā and the Piṅgalā exists at the entrance of the nose. 
%-----------------------------
tanmadhye
\app{\lem[resp=egoscr, type=emendation, alt={nāḍīdvayam}]{nāḍīdvaya\skp{m-i}}
  \rdg[wit={E}]{dvayam}
  \rdg[wit={ceteri}]{nāḍīdvayaṃ}
}\app{\lem[wit={E}, alt={iḍāpiṅgalā}]{\skm{m-i}ḍāpiṃgalā}
  \rdg[wit={ceteri}]{idā piṃgalā}
  \rdg[wit={N2}]{idānīṃ piṃgalā}
  \rdg[wit={P}]{idāṃ piṃgalā}
}\app{\lem[wit={ceteri}]{saṃjñakaṃ}
  \rdg[wit={U1,U2}]{saṃjñākaṃ}
  \rdg[wit={ceteri}]{saṃjñīkāḥ}}
\app{\lem[wit={ceteri}]{nāsādvāre}
  \rdg[wit={D}]{nāsānāsādvāre}}
tiṣṭhati/
%-----------------------------
%suṣumṇā    tālumārge   brahmadvāraparyantaṃ   vahati tiṣṭhati/ \E
%suṣumṇā    tālumārgeṇa brahmaraṃdhraparyanta--vahati tiṣṭhati... \P
%suṣumṇā    tālumārge   brahmaraṃdhraparyantaṃ vahatī tiṣṭhati... \B
%suṣumṇā    tālumārge   brahmaraṃdhraparyantaṃ vahati tiṣṭhati... \L
%suṣumṇā tu tālumārgeṇa brahmadvāraparyantaṃ   vahatī tiṣṭhati... \N1
%suṣumṇā tu tālumārgeṇa brahmadvāraparyantaṃ   vahatī tiṣṭhati    \D
%suṣumṇā tu tālumārge   brahmadvāraparyantaṃ   vahatī tiṣṭhati// \N2
%suṣumṇā tu tālumārgeṇa brahmadvāraparyantaṃ   vahati tiṣṭhati \U1
%suṣumṇā    tālumārgeṇa brahmadvāraparyantaṃ   vahati// \U2
%-----------------------------
%The river being the central channel leads from the palate to the door of Brahma.  
%-----------------------------
\app{\lem[wit={ceteri}]{suṣumṇā}
  \rdg[wit={N1,N2,D,U1}]{suṣumṇā tu}}
\app{\lem[wit={ceteri}]{tālumārgeṇa}
  \rdg[wit={E,B,L,N2}]{tālumārge}}
brahma\app{\lem[wit={ceteri}, alt={°dvāra°}]{dvāra}
  \rdg[wit={P,B,L}]{°raṃdhra°}
}paryantaṃ
\app{\lem[wit={U2}]{vahati}
  \rdg[wit={E,P,L,U1}]{vahati tiṣṭhati}
  \rdg[wit={ceteri}]{vahati tiṣṭhatī}}/
%-----------------------------
%        sarasvatī mukhamadhye tiṣṭhati/ \E
%        sarasvatī mukhamadhye tiṣṭhati  \P
%        sarasvatī mukhamadhye tiṣṭhatī/ \B
%        sarasvatī mukhamadhye tiṣṭhati/ \L
%        sarasvatī mukhamadhye varttate/ \N1
%        sarasvatī mukhamadhye varttate// \D
%        sarasvatī mukhamadhye varttate/ \N2
%        sarasvatī mukhamadhye varttate \U1
%ti sraḥ sarasvati mukhamadhye tiṣṭhati// \U2
%-----------------------------
%The Sarasvatī[-channel] exists at the centre of the face. 
%-----------------------------
\app{\lem[wit={ceteri}]{sarasvatī}
  \rdg[wit={U2}]{ti sraḥ sarasvati}}
mukhamadhye
\app{\lem[wit={N1,N2,D,U1}]{varttate}
  \rdg[wit={E,P,L,U2}]{tiṣṭhati}
  \rdg[wit={B}]{tiṣṭhatī}}/
%-----------------------------
%gāṃdhārīhastijihvākarṇayor            madhye  vahalyau  tiṣṭhataḥ/    \E
%gāṃdhārīhastinījihve karṇayor         madhye  vahatyau  tiṣṭhataḥ   \P
%gāṃdhārīhastījihve karṇa----------------------vahatyo   tiṣṭhati//                \B
%gāṃdhārīhastijihve karṇa----------------------vahatyo   tiṣṭhati...               \L
%gāṃdhārīhastinījihve karṇayor         madhye  vahatyau  tiṣṭhataḥ// \N1
%gāṃdhārīhastinījihve karṇayor         madhye  vahatyau  tiṣṭhataḥ// \D
%gāṃdhārīhastinījihve karṇayor         madhye  vahatyau  tiṣṭhataḥ// \N2
%gāṃdhādīharratījihvakarṇayor          madhye            tiṣṭhataḥ              \U1
%gāṃdhārī// hastinī// jihve// netrayor madhye  vahaṃtyaḥ//    \U2
%-----------------------------
%The two rivers Gāṃdhārī and Hastjihvā exist within the centre of the two ears. 
%-----------------------------
gāṃdhārī\app{\lem[wit={E}, alt={hastijihvākarṇayor}]{hastijihvākarṇayo\skp{r-ma}}
  \rdg[wit={P,N1,N2,D}]{hastinījihve karṇayor}
  \rdg[wit={B,L}]{hastījihve karṇa°}
  \rdg[wit={U1}]{harratījihvakarṇayor}
  \rdg[wit={U2}]{hastinī || jihve || netrayor}}
\app{\lem[wit={ceteri}, alt={madhye}]{\skm{r-ma}dhye}
  \rdg[wit={L,B}]{\om}}
\app{\lem[wit={P,N1,D,D}]{vahatyau}
  \rdg[wit={E}]{vahalyau}
  \rdg[wit={B,L}]{vahatyo}
  \rdg[wit={U2}]{vahaṃtyaḥ}}
\app{\lem[wit={ceteri}]{tiṣṭhataḥ}
  \rdg[wit={B,L}]{tiṣṭhati}
  \rdg[wit={U2}]{\om}}/
%-----------------------------
%pūṣālambusemā         netrayor madhye rvahalyā tiṣṭhataḥ/ \E
%pūṣālaṃbuse           netrayor madhye vahatyau tiṣṭataḥ \P
%pūṣoḍalabuṣe----------netra----madhye vahatyo  tiṣṭhati/ \B
%pūṣo ulabuso          netra----madhye vahatyo  tiṣṭhaṃti// \L
%pūṣāṃalaṃbuṣe         netrayor madhye vahatyo  tiṣṭhataḥ/ \N1
%pūṣāṃalaṃbuṣe         netrayor madhye vahatyau tiṣṭhataḥ// \D
%pūṣāṃalaṃbuṣe         netayor  madhye vahatyo  tiṣṭhataḥ/ \N2
%pūṣālaṃbuṣe           netayor  madhye vahatyau tiṣṭhataḥ \U1
%pūṣāya śakhinī// karṇayor      madhye vahatyo  tiṣṭhata// alaṃbuṣā// bhu?madhye vaṃhatyo tiṣṭhati// \U2
%-----------------------------
%The two rivers Pūṣā and Ālaṃbuṣā are situated at the center of the two eyes. 
%-----------------------------
\app{\lem[resp=egoscr, type=emendation, alt={pūṣālaṃbuṣānetrayor}]{pūṣālaṃbuṣānetrayo\skp{r-ma}}
  \rdg[wit={E}]{pūṣālambusemā netrayor}
  \rdg[wit={P}]{pūṣālaṃbuse netrayor}
  \rdg[wit={B}]{pūṣoḍalabuṣe netra°}
  \rdg[wit={L}]{pūṣo ulabuso netra°}
  \rdg[wit={N1,D}]{pūṣāṃalaṃbuṣe netrayor}
  \rdg[wit={N2}]{pūṣāṃalaṃbuṣe netayor}
  \rdg[wit={U1}]{pūṣālaṃbuṣe netayor}
  \rdg[wit={U2}]{pūṣāya śakhinī || karṇayor}
}\skm{r-ma}dhye
\app{\lem[wit={ceteri}]{vahatyau}
  \rdg[wit={E}]{rvahalyā}
  \rdg[wit={B,L,N1,N2,U2}]{vahatyo}}
\app{\lem[wit={E,N1,N2,D,U1}]{tiṣṭhataḥ}
  \rdg[wit={P}]{tiṣṭataḥ}
  \rdg[wit={B}]{tiṣṭhati}
  \rdg[wit={L}]{tiṣṭhaṃti}
  \rdg[wit={U2}]{tiṣṭhata || alaṃbuṣā || bhrumadhye vaṃhatyo tiṣṭhati ||}}
%-----------------------------
%śaṃkhinī liṃgadvārād ārabhye--ḍāmārgeṇa     brahmasthānaparyaṃtaṃ tiṣṭhatīti/     \E
%śaṃkhinī liṃgadvārād ārabhya iḍāmārgeṇa     brahmasthānaparyaṃtaṃ tiṣṭhati      \P   %%%%%%%7648.jpg
%śaṃkhinī liṃgadvārād ārabhya iḍāmārgeṇa     brahmasthānaparyaṃtaṃ tiṣṭhati/     \B
%śaṃkhinī liṃgadvārād ārabhya iḍāmārgeṇa     brahmasthānaparyaṃtaṃ tiṣṭhati//    \L 
%śaṃkhanī liṃgadvārād ārabhya iḍāmārgeṇa     brahmasthānaparyaṃtaṃ tiṣṭhati/     \N1
%śaṃkhinī liṃgadvārād ārabhya iḍāmārgeṇa     brahmasthānaparyaṃtaṃ tiṣṭhati//     \D
%śaṃkhinī liṃgadvārād ārabhya iḍānīṃ mārgeṇa brahmasthānaparyaṃtaṃ tiṣṭhati/ \N2
%śaṃkhinī liṃgadvārārabhya    iḍāmārgeṇa     brahmasthānaparyaṃtaṃ tiṣṭhati      \U1
%kuhū     liṃgadvārād ārabhya iḍāmārgeṇa     brahmasthānaparyaṃtaṃ tiṣṭhati// śāṃkhinī mūladvārād arabhya piṃgalamargeṇa brahmasthānaparyaṃtaṃ tiṣṭhati// \U2
%-----------------------------
%The Śaṃkhinī channel strechtes from the the beginning of the opening of the penis through the Iḍā-channel. 
%-----------------------------
\app{\lem[wit={U2}]{kuhū}
  \rdg[wit={ceteri}]{śaṃkhinī}
  \rdg[wit={N1}]{śaṃkhanī}}
\app{\lem[wit={ceteri}, alt={liṃgadvārād}]{liṃgadvārā\skp{d-ā}}
  \rdg[wit={U1}]{liṃgadvārā°}}
\app{\lem[wit={ceteri}, alt={ārabhye}]{\skm{d-ā}rabhye}
  \rdg[wit={ceteri}]{ārabhya}
}\app{\lem[wit={E}]{iḍāmārgeṇa}
  \rdg[wit={ceteri}]{iḍāmārgeṇa}
  \rdg[wit={N2}]{iḍānīṃ mārgeṇa}}
brahmasthānaparyaṃtaṃ 
\app{\lem[wit={ceteri}]{tiṣṭhati}
  \rdg[wit={E}]{tiṣṭhatīti}}/
\extra{śāṃkhinī mūladvārād-arabhya
\app{\lem[resp=egoscr, type=emendation, alt={piṃgalā}]{piṃgalā}
  \rdg[wit={U2}]{piṃgala°}}margeṇa brahmasthānaparyaṃtaṃ tiṣṭhati/}
\note[type=philcomm, labelb=177, lem={kuhū}]{I followed Witness U\textsubscript{2} and chose the reading \textit{kuhū} instead of \textit{śaṃkhinī} to arrive at the complete ten channels. Due to similar structure of the sentences describing the channels, the channel \textit{kuhū} dropped in the process of copying in all other witnesses except U\textsubscript{2}.}
%-----------------------------
%etādṛśa  nāḍyo daśasu dvāreṣu tiṣṭhanti/    \E
%etādṛṣā  nāḍyo daśasu dvāreṣu tiṣṭhaṃti      \P
%etādṛṣyā nāḍyo daśasu dvāreṣu tiṣṭhaṃti/    \B
%etādṛṣyā nāḍyo daśa   dvāreṣu    tiṣṭhaṃti/    \L 5876_15.jpg
%etādaśa  nāḍyo daśasu dvāreṣu tiṣṭhaṃti/    \N1
%etādaśa  nāḍyo daśasu dvāreṣu tiṣṭhaṃti//   \D
%etā            daśasu  dvāreṣu tiṣṭhaṃti/                \N2
%etādṛśa  nāḍyo daśasv adhāreṣu  tiṣṭhati    \U1
%etādaśa  nāḍyo daśaśoṣu dvāreṣu tiṣṭhaṃti// \U2 %%%413.jpg
%-----------------------------
%In such a way the channels are situated at the 10 openings. 
%-----------------------------
\app{\lem[wit={P}]{etādṛṣā}
  \rdg[wit={E,N1,D,U1,U2}]{etādṛśa}
  \rdg[wit={B,L}]{etādṛṣyā}
  \rdg[wit={N2}]{etā}}
\app{\lem[wit={ceteri}]{nāḍyo}
  \rdg[wit={N2}]{\om}}
\app{\lem[wit={ceteri}]{daśasu dvāreṣu}
  \rdg[wit={L}]{daśa dvāreṣu}
  \rdg[wit={U1}]{daśasv adhāreṣu}}
\app{\lem[wit={ceteri}]{tiṣṭhanti}
  \rdg[wit={U1}]{tiṣṭhati}}/
%-----------------------------
%anyā dvisaptatisahasraparimitā                      nāḍayo lomnāṃ mūleṣu sūkṣmarūpeṇa tiṣṭanti// \E
%anyā dvisaptatisahasraparimitā                      nāḍyo  lomnā  mūleṣu sūkṣmarūpeṇa tiṣṭaṃti      \P
%anyā dvisaptatīsahasraparimitā                      nāḍyo  lomnā  mūleṣu sūkṣmarūpeṇa tiṣṭaṃti// \B
%anyā dvisaptatisahasraparimitā                      nāḍyo  lomnā  mūleṣu sūkṣmarūpeṇa tiṣṭaṃti// \L
%anyā dvisaptatisahasraparamitā                      nāḍyā  lomnāṃ mūleṣu sūkṣmarūpeṇa tiṣṭaṃti// \N1
%anyā dvisaptatisahasraparamitā                      nāḍyā  lomnāṃ mūleṣu sūkṣmarūpeṇa tiṣṭaṃti// \D
%anyā dvisaptatrisahasraparimitā                     nāḍyā  lomnāṃ mūleṣu sūkṣmarūpeṇa tiṣṭaṃti// \N2
%anyā dvisaptatisahasraparimitāgryo                         lomnā  mūleṣu sūkṣmarūpeṇa tiṣṭaṃti \U1
%anyā hidaśonā dvisatyati sahasraḥ//71110// parimitā nādhyo lomnāṃ mūleṣu sūkṣmarūpeṇa tiṣṭaṃti// \U2
%-----------------------------
%The other channels measured as 72000 are situated with a subtle form at the roots of the hairs.
%-----------------------------
anyā
\app{\lem[wit={ceteri}]{dvisaptatisahasraparimitā}
  \rdg[wit={U1}]{dvisaptatisahasraparimitāgryo}
  \rdg[wit={U2}]{hidaśonā dvisatyati sahasraḥ || 71110 || parimitā}}
\app{\lem[wit={P,B,L}]{nāḍyo}
  \rdg[wit={E}]{nāḍayo}
  \rdg[wit={U2}]{nādhyo}
  \rdg[wit={U1}]{\om}}
\app{\lem[wit={E,N1,N2,D,U2}]{lomnāṃ mūleṣu} %%%lomnāṃ = gen pl neutrum v.loman
  \rdg[wit={P,B,L,U1}]{lomnā}}
sūkṣmarūpeṇa tiṣṭaṃti\dd{}
\end{prose}
\end{ekdosis}
%-----------------------------
%-----------------------------
%-----------------------------
\begin{ekdosis}
  \ekddiv{type=ed}
 \centerline{\textrm{\small{[The Ten Vitalwinds]}}}
 \bigskip
 \begin{prose}
%-----------------------------
%[p.36]
%idānīṃ śarīramadhye vāyavo daśa tiṣṭhanti/ \E
%idānīṃ śarīramadhye vāyavo daśa tiṣṭhaṃti  \P
%idānīṃ śarīramadhye .....\om               \B
%idānīṃ śarīramadhye .....\om               \L
%idānīṃ śarīramadhye vāyavas tiṣṭhaṃti/     \N1
%idānīṃ śarīramadhye vāyavas tiṣṭhaṃti//    \D
%idānīṃ śarīramadhye vāyavas tiṣṭhaṃti/     \N2
%idānīṃ śarīramadhye vāyavas tiṣṭhaṃti      \U1
%idānīṃ śarīramadhye vāyavo daśa ṣṭaṃti//   \U2
%-----------------------------
%Now ten vitalwinds are situated within the body.  
%-----------------------------
\note[type=source, labelb=178, lem={vāyavo}]{Ysv (PT): idānīṃ śṛṇu nāḍīnāṃ bhedaṃ vakṣyāmi siddhidam | meruvāhye iḍānāmnī piṅgalayā samanvitā | suṣumnā bhānumārgeṇa brahmadvārāvadhi sthitā | sarasvatī sugandhā tu gāndhārī hastijihvakā | jñātavyā karṇayor madhye netrayoś ca tathāntimā | pūṣā cālambuṣā ceti mūlasthā kutracit tathā | liṅgadvārādiḍāmārge brahmasthānāvadhi priye | nāḍyantaṃ pratilomeṣu sahasrāṇāṃ dvisaptatiḥ |}
idānīṃ śarīramadhye
\app{\lem[wit={E,P,U2}]{vāyavo}
  \rdg[wit={N1,N2,D,U1}]{vāyavas}
  \rdg[wit={B,L}]{\om}}
\app{\lem[wit={E,P,U2}]{daśa}
  \rdg[wit={ceteri}]{\om}}
\app{\lem[wit={ceteri}]{tiṣṭhanti}
  \rdg[wit={U2}]{ṣṭaṃti}
  \rdg[wit={B,L}]{\om}}/
%-----------------------------
%teṣāṃ nāmāni kāryāṇi kathyante/ \E
%teṣāṃ nāmāni kārmāṇi kathyante/ \P
%\om  \B
%\om \L
%teṣāṃ kāryāṇi kathyante/ \N1
%teṣāṃ kāryāṇi kathyaṃte/ \D
%teṣāṃ kāryāṇi kathyate/ \N2
%teṣāṃ kāryāṇi kathyate \U1 %%%280.jpg
%teṣāṃ kāryāṇi kathyate \U2
%-----------------------------
%their functions are taught. 
%-----------------------------
\app{\lem[wit={ceteri}]{teṣāṃ}
  \rdg[wit={B,L}]{\om}}
\app{\lem[wit={ceteri}]{kāryāṇi}
  \rdg[wit={E}]{nāmāni kāryāṇi}
  \rdg[wit={P}]{nāmāni kārmāṇi}
  \rdg[wit={L,B}]{\om}}
\app{\lem[wit={ceteri}]{kathyante}
  \rdg[wit={N2,U1,U2}]{kathyate}
  \rdg[wit={L,B}]{\om}}/
%-----------------------------
%prāṇavāyur hṛdayamadhye śvāsocchāsaṃ karoti/ \E
%prāṇavāyur hṛdayamadhye śvāsochāsaṃ karoti       \P
%------------------------śvāsośvaroti/               \B
%                        śvāsośvareti...             \L 
%prāṇavāyuhṛdayamadhye  utsvāsaprasvāsasaṃ karoti//   \N1
%prāṇavāyuhṛdayamadhye  utsvāsaprasvāsaṃ karotī   \D
%prāṇavāyuhṛdayamadhye ūrdhvaśvāsapraśvāsaṃ karoti// \N2
%prāṇavāyuhṛdayamadhye ūdhvasaprasase karoti \U1
%prāṇavāyuhṛdayamadhye   svāsochvāsaṃ karoti \U2
%-----------------------------
%The Prāṇa vitalwind is located in the middle of the heart and causes inhalation and exhalation. 
%----------------------------
\app{\lem[wit={E,P}, alt={prāṇavāyur}]{prāṇavāyu\skp{r-hṛ}}
  \rdg[wit={N1,N2,D,U1,U2}]{prāṇavāyu°}
  \rdg[wit={B,L}]{\om}
}\app{\lem[wit={ceteri},alt={hṛdayamadhye}]{\skm{r-hṛ}dayamadhye}
  \rdg[wit={B,L}]{\om}}
\app{\lem[type=emendation, resp=egoscr]{ucchvaśvāsapraśvāsaṃ}
  \rdg[wit={N1}]{\korr utsvāsaprasvāsasaṃ}
  \rdg[wit={N2}]{ūrdhvaśvāsapraśvāsaṃ}
  \rdg[wit={D}]{utsvāsaprasvāsaṃ}
  \rdg[wit={U1}]{ūdhvasaprasase}
  \rdg[wit={E}]{śvāsocchāsaṃ}
  \rdg[wit={P}]{śvāsochāsaṃ}
  \rdg[wit={B}]{śvāsośvaroti}
  \rdg[wit={L}]{śvāsośvareti}}/
\note[type=source, labelb=179, lem={prāṇavāyur}]{SSP: hṛdaye prāṇavāyur ucchvāsaniḥśvāsakārako hakārasakārātmakaś ca | gude tv apānavāyuḥ recakakumbhakapūrakaś ca | nābhau samānavāyuḥ dīpakaḥ pācakaś ca| kaṇṭhe vyānavāyuḥ śoṣaṇāpyāyanakārakaś ca | tālau udānavāyuḥ grasanavamanajalpakārakaś ca| nāgavāyuḥ sarvāṅgavyāpakaḥ mocakaś cālakaś ca | kūrmavāyuḥ cakṣuṣor unmeṣakārakaś ca| kṛkalaḥ udgārakaḥ kṣutkārakaś ca | devadatto mukhavijṛmbhakaḥ | dhanañjayo nādaghoṣakah ||1.67|| iti daśavāyvavalokanena piṇḍotpattiḥ naranārīrūpam |}
\end{prose}
\end{ekdosis}
\ekdpb*{}
%%%%%%%%%%%%%%%%%%%%%%%%%%%%%%%%%%%%%%%%%%
%%%%%%%%%%%%%%%%%%%%%%%%%%%%%%%%%%%%%%%%%%
%%%%%%%%PAGEBREAK%%%%%%%PAGEBREAK%%%%%%%%%
%%%%%%%%%%%%%%%%%%%%%%%%%%%%%%%%%%%%%%%%%%
%%%%%%%%%%%%%%%%PAGEBREAK%%%%%%%%%%%%%%%%%
%%%%%%%%%%%%%%%%%%%%%%%%%%%%%%%%%%%%%%%%%%
%%%%%%%%PAGEBREAK%%%%%%%PAGEBREAK%%%%%%%%%
%%%%%%%%%%%%%%%%%%%%%%%%%%%%%%%%%%%%%%%%%%
%%%%%%%%%%%%%%%%%%%%%%%%%%%%%%%%%%%%%%%%%%
%%%%%%%%%%%%%%%%%%%%%%%%%%%%%%%%%%%%%%%%%%
%%%%%%%%%%%%%%%%%%%%%%%%%%%%%%%%%%%%%%%%%%
%%%%%%%%PAGEBREAK%%%%%%%PAGEBREAK%%%%%%%%%
%%%%%%%%%%%%%%%%%%%%%%%%%%%%%%%%%%%%%%%%%%
%%%%%%%%%%%%%%%%PAGEBREAK%%%%%%%%%%%%%%%%%
%%%%%%%%%%%%%%%%%%%%%%%%%%%%%%%%%%%%%%%%%%
%%%%%%%%PAGEBREAK%%%%%%%PAGEBREAK%%%%%%%%%
%%%%%%%%%%%%%%%%%%%%%%%%%%%%%%%%%%%%%%%%%%
%%%%%%%%%%%%%%%%%%%%%%%%%%%%%%%%%%%%%%%%%%
%%%%%%%%%%%%%%%%%%%%%%%%%%%%%%%%%%%%%%%%%%
%%%%%%%%%%%%%%%%%%%%%%%%%%%%%%%%%%%%%%%%%%
%%%%%%%%PAGEBREAK%%%%%%%PAGEBREAK%%%%%%%%%
%%%%%%%%%%%%%%%%%%%%%%%%%%%%%%%%%%%%%%%%%%
%%%%%%%%%%%%%%%%PAGEBREAK%%%%%%%%%%%%%%%%%
%%%%%%%%%%%%%%%%%%%%%%%%%%%%%%%%%%%%%%%%%%
%%%%%%%%PAGEBREAK%%%%%%%PAGEBREAK%%%%%%%%%
%%%%%%%%%%%%%%%%%%%%%%%%%%%%%%%%%%%%%%%%%%
%%%%%%%%%%%%%%%%%%%%%%%%%%%%%%%%%%%%%%%%%%
\begin{ekdosis}
  \begin{prose}
    \noindent
%-----------------------------
%aśanapānecchā bhavati/   gudamadhye                                                            \E
%aśanapānechā  bhavati    gudamadhye 'pānāvāyus  tiṣṭhati    sa āṃkucanastaṃbhanaṃ   karoti     \P
%aśanapānechā  bhavati//  gudamadhye apānāvāyor  tiṣṭhatī    sa āṃkucanastaṃbhanaṃ   karotī/    \B
%aśanapānecha  bhavati//  gudamadhye apānāvāyo   tiṣṭhati    sa āṃkucanastaṃbhanaṃ   karotī/    \L
%asitapittecha bhavati/   guḍamadhye apānavāyu   tiṣṭhati    sa ākuṃcanasthaṃbhanaṃ  karoti/ /  \N2
%aśitapiteccha bhavati/   gudamadhye apānavāyus  tiṣṭhati/   sa ākuṃcanaṃ staṃbhanaṃ karoti/    \N1
%aśitapiteccha bhavati//  gudamadhye apānavāyus  tiṣṭhati/   sa ākuṃcanaṃ staṃbhanaṃ karoti//   \D
%asīte pitechā bhavati    gudamadhye apānavāyu   tiṣṭhati    sa ākuṃcanaṃ staṃbhanaṃ karoti     \U1
%aśanapānechā  bhavati//  gudamadhye apānāvāyo   tiṣṭhati//     āṃkucanastabhanaṃ    karoti/    \U2
%-----------------------------
%The wish for eating an drinking exists. At the center of the anus the Apāna-Vitalwind exists. He does contraction and checking. 
%-----------------------------
\app{\lem[wit={E}]{aśanapānecchā}
  \rdg[wit={P,B,L,U2}]{aśanapānechā}
  \rdg[wit={N1,D}]{aśitapiteccha}
  \rdg[wit={N2}]{asitapittecha}
  \rdg[wit={U1}]{asīte pitechā}}
bhavati/
gudamadhye
\app{\lem[type=emendation, resp=egoscr, alt={'pānavāyus}]{'pānavāyu\skp{s-ti}}
  \rdg[wit={N1,D}]{\korr apānavāyus}
  \rdg[wit={B}]{apānāvāyor}
  \rdg[wit={L,U2}]{apānāvāyo}
  \rdg[wit={N2,U1}]{apānavāyu}
  \rdg[wit={E}]{\om}
}\app{\lem[wit={ceteri}, alt={tiṣṭhati}]{\skm{s-ti}ṣṭhati}
  \rdg[wit={B}]{tiṣṭhatī}
  \rdg[wit={E}]{\om}}/
\app{\lem[wit={ceteri}]{sa}
  \rdg[wit={E,U2}]{\om}}
\app{\lem[wit={N1,D,U1}]{ākuṃcanaṃ staṃbhanaṃ}
  \rdg[wit={P,B,L,U2}]{āṃkucanastaṃbhanaṃ}
  \rdg[wit={E}]{\om}}
\app{\lem[wit={ceteri}]{karoti}
  \rdg[wit={B}]{karotī}
  \rdg[wit={E}]{\om}}/ 
%-----------------------------
%              samāno vāyur vartate/ sapta samagrā nāḍīḥ śoṣayati/  \E
%  nābhīmadhye samāno varttate       sa samagrā    nāḍīḥ  śoṣayati    \P
%  nābhīmadhye smānā  vartate/       sa samagrā    nāḍī    śoṣayati//  \B
%  nābhīmadhye samānā vartate        sa samagrā    nāḍī    śoṣayatī//  \L
%  nābhimadhye samāno varttate/      sa samāgraṃ   nādhyaṃ śoṣayati/   \N2
%  nābhimadhye samāno varttate/      sa samagraṃ   nādhyaṃ śoṣayati//  \N1
%  nābhimadhye samāno varttate//     sa samagraṃ   nādhyaṃ śoṣayati//  \D
%  nābhimadhye samāno varttate       sa samagrāṃ   nāḍīṃ śoṣayati      \U1
%  nābhipadmamadhye samāno vartate// sa samagrā    nāḍī śoṣayati        \U2
%-----------------------------
%At the center of the navel the Samāna[-vitalwind] exists. He causes to dry up all the channels.
%-----------------------------
\app{\lem[wit={ceteri}]{nābhimadhye}
  \rdg[wit={U2}]{nābhipadmamadhye}
  \rdg[wit={E}]{\om}}
\app{\lem[wit={ceteri}]{samāno}
  \rdg[wit={E}]{samāno vāyur}
  \rdg[wit={B}]{smānā}}
  vartate/
  \app{\lem[wit={ceteri}]{sa}
    \rdg[wit={E}]{sapta}}
  \app{\lem[wit={E,P,B,L,U2}]{samagrā}
    \rdg[wit={N1,N2,D,U1}]{samāgraṃ}}
  \app{\lem[wit={E,P}]{nāḍīḥ}
    \rdg[wit={B,L,U2}]{nāḍī}
    \rdg[wit={U1}]{nāḍīṃ}
    \rdg[wit={N1,N2,D}]{nādhyaṃ}}
  \app{\lem[wit={ceteri}]{śoṣayati}
    \rdg[wit={L}]{śoṣayatī}}/
%-----------------------------
%  nābhīmadhye samāno varttate       sa samagrā    nāḍīḥ  śoṣayati    \P
%  nābhīmadhye smānā  vartate/       sa samagrā    nāḍī    śoṣayati//  \B
%  nābhīmadhye samānā vartate        sa samagrā    nāḍī    śoṣayatī//  \L
%  nābhimadhye samāno varttate/      sa samāgraṃ   nādhyaṃ śoṣayati/   \N2
%  nābhimadhye samāno varttate/      sa samagraṃ   nādhyaṃ śoṣayati//  \N1
%  nābhimadhye samāno varttate//     sa samagraṃ   nādhyaṃ śoṣayati//  \D
%  nābhimadhye samāno varttate       sa samagrāṃ   nāḍīṃ śoṣayati      \U1
%  nābhipadmamadhye samāno vartate// sa samagrā    nāḍī śoṣayati        \U2
%-----------------------------
%At the center of the navel the Samāna[-vitalwind] exists. He causes to dry up all the channels.
%-----------------------------
\app{\lem[wit={ceteri}]{nābhimadhye}
  \rdg[wit={U2}]{nābhipadmamadhye}
  \rdg[wit={E}]{\om}}
\app{\lem[wit={ceteri}]{samāno}
  \rdg[wit={E}]{samāno vāyur}
  \rdg[wit={B}]{smānā}}
  vartate/
  \app{\lem[wit={ceteri}]{sa}
    \rdg[wit={E}]{sapta}}
  \app{\lem[wit={E,P,B,L,U2}]{samagrā}
    \rdg[wit={N1,N2,D,U1}]{samāgraṃ}}
  \app{\lem[wit={E,P}]{nāḍīḥ}
    \rdg[wit={B,L,U2}]{nāḍī}
    \rdg[wit={U1}]{nāḍīṃ}
    \rdg[wit={N1,N2,D}]{śoṣayati}
    \rdg[wit={B}]{śoṣayatī}}/
%-----------------------------
%tathā nāḍīśoṣaṇāt                      rucim  utpādayati/  vahniṃ dīpayati/ \E
%tathā nāḍīḥ pośayati                   rucim  utpādayati   vahniṃ dīpayatī \P
%tathā       pośayatī/ tathā poṣayatī// rucir  utpādayatī   vahnī  dīpayatī/ \B
%tathā       pośayatī                   rucim  utpādayatī   vahnī  dīpayatī... \L
%tathā nāḍīṃ pośayati/                  kvacit-utpādayati/  āgniṃ  dīpayati \N1
%tathā nāḍīṃ pośayati//                 kvacit-utpādayati// āgniṃ  dīpayati \D %%%p. 11 recto
%tathā nāḍīṃ pośayati/                  kvacit-utpādayati/  āgniṃ  dīpayati \N2
%tathā nāḍīṃ pośa iti                   rucim  utpādayati    agnīṃ  dīpayati \U1
%            ṣoṣayati                   rucim  utpādayati//  vahniṃ dīpayati// \U2
%-----------------------------
%In this way the channels are caused to thrive, beauty is caused to be generated and the fire is caused to light up.  
%-----------------------------
\app{\lem[wit={ceteri}]{tathā}
  \rdg[wit={U2}]{\om}}
\app{\lem[wit={P}]{nāḍīḥ}
  \rdg[wit={E}]{nāḍī}
  \rdg[wit={N1,N2,D,U1}]{nāḍīṃ}
  \rdg[wit={B,L,U2}]{\om}}
\app{\lem[type=emendation, resp=egoscr]{poṣayati}
  \rdg[wit={P,N1,N2,D,U1}]{\korr pośayati}
  \rdg[wit={B}]{pośayatī | tathā poṣayatī}
  \rdg[wit={L}]{pośayatī}
  \rdg[wit={U1}]{pośa iti}
  \rdg[wit={U2}]{ṣoṣayati}
  \rdg[wit={E}]{°śoṣaṇāt}}/
\app{\lem[wit={ceteri}, alt={rucim}]{ruci\skp{m-u}}
  \rdg[wit={B}]{rucir}
  \rdg[wit={N1,N2,D}]{kvacit}}
\app{\lem[wit={ceteri}, alt={utpādayati}]{\skm{m-u}tpādayati}
  \rdg[wit={P}]{utpādayatī}}/
\app{\lem[type=emendation, resp=egoscr]{agniṃ}
  \rdg[wit={N1,N2,D}]{\korr āgniṃ}
  \rdg[wit={U1}]{agnīṃ}
  \rdg[wit={E,P,U2}]{vahniṃ}
  \rdg[wit={B,L}]{vahnī}}
\app{\lem[wit={ceteri}]{dīpayati}
  \rdg[wit={P,B,L}]{dīpayatī}}/ 
%-----------------------------
%tālumadhye udāno vāyus-tiṣṭhati/   sa vāyuḥ ratnaṃ līlati/    pānīyaṃ pibati/  nāgavāyuḥ   sarva--śarīre varttate/  tasmād-vāyoḥ śarīraṃ cālayati/ śokam āpnoti// vivilaḥ        \E
%tālumadhye udāno vāyus-tiṣṭhati    sa vāyu  ratnaṃ gilati     pānīyaṃ pībati   nāgavāyuḥ   sakale śarīre varttate   tasmād-vāyo śarīraṃ calayati   śopham āpnoti  vikṛtaḥ        \P %%%7649.jpg
%tālumadhye udānavāyus-tiṣṭhati/    sa vāyur annaṃ  galayatī/  pānīyaṃ pibatī/  nāgavāyuḥ   sakala-śarīre varttate   tasmād-vāyoḥ// śarīre cālatī/  śokam āpnoti   vi??kru??taḥ// \B DSCN7163.JPG Z.11
%tālumadhye udānavāyus tiṣṭhati//   sa vāyur annaṃ  galayati// pānīyaṃ pibatī// nāgavāyu----sakala-śarīre vartate    tasmād vāyoḥ// śarīre cālayatī śokam āpnoti   vikutaḥ...     \L
%tālumadhye udānavāyus-tiṣṭhati/    sa vāyuḥ ratnaṃ śilati/    pānīyaṃ pibati/  nāgavāyuḥ   sakale śarīre varttate// tasmād-vāyoḥ śarīraṃ calati/                                 \N1
%tālumadhye udāno vāyus-tiṣṭhati//  sa vāyur annaṃ  gilati/    pānīyaṃ pibati   nānāgavāyuḥ sakale śarīre varttate// tasmād-vāyoḥ śarīraṃ calati//                                \D
%tālumadhye udānāni vāyus-tiṣṭhati/ sa vāyur-annaṃ  gīlati/    pānīyaṃ pibati/  nāgavāyuḥ   sakale śarīre varttate// tasmād-vāyoḥ śarīraṃ calati/                                 \N2
%tālumadhye udānavāyus-tiṣṭhati     sa vāyur-annaṃ  gilati     pānīyaṃ pibati   nāgavāyu    sakale śarīre varttate   tasmād-vāyoḥ śarīraṃ calati                                  \U1
%tālumadhye udāno vāyus-tiṣṭhati//   sa vāyur annaṃ  gilati//  pānīyaṃ pibati// nāgavāyuḥ   sakale śarīre varttate// tasmād-vāyoḥ śarīraṃ calayati śokam āpnoti vikṛtaḥ//         \U2
%-----------------------------
%Within the throat the Udāna-vitalwind is situated. This wind swallows food, [and] it drinks water. The Nāga-vitalwind exists in the entire body. Through the vitalwind the body is caused to move. 
%em. nāgavāyu = vyānavāyuḥ ....
%-----------------------------
tālumadhye
\app{\lem[wit={B,L,N1,U1}, alt={udānavāyus}]{udānavāyu\skp{s-ti}}
  \rdg[wit={E,P,D,U2}]{udāno vāyus}
  \rdg[wit={N2}]{udānāni vāyus}
}\skm{s-ti}ṣṭhati/
sa \app{\lem[wit={ceteri}, alt={vāyur}]{vāyu\skp{r-a}}
  \rdg[wit={E}]{vāyuḥ}
  \rdg[wit={P}]{vāyu}
}\app{\lem[wit={ceteri}, alt={annaṃ}]{skm{r-a}nnaṃ}
  \rdg[wit={E,P,N1}]{ratnaṃ}}
\app{\lem[wit={ceteri}]{gilati}
  \rdg[wit={E}]{līlati}
  \rdg[wit={B}]{galayatī}
  \rdg[wit={L}]{galayati}
  \rdg[wit={N1}]{śilati}}/
pānīyaṃ \app{\lem[wit={ceteri}]{pibati}
  \rdg[wit={P}]{pībati}
  \rdg[wit={B,L}]{pibatī}}/
\app{\lem[wit={ceteri}]{nāgavāyuḥ}
  \rdg[wit={L}]{nāgavāyu°}
  \rdg[wit={D}]{nānāgavāyuḥ}}
\app{\lem[wit={ceteri}]{sakale}
  \rdg[wit={B,L}]{sakala°}
  \rdg[wit={E}]{sarva°}}
śarīre varttate/ 
%\note[type=philcomm, labelb=s35.z3a, lem={nāgavāyu}]{Only nine of the promised ten vitalwinds are described here. The missing vitalwind is \textit{vyānavāyu}. The description of \textit{nāgavāyu} matches rather the \textit{vyānavāyu}. Witnesses E, P, B, L and U2 preserve a nonsensical fragment after the description of \textit{nāgavāyu}: śokam āpnoti vikṛtaḥ. Possibly the description of \textit{vyānavāyu} was lost due to an eyeskip of a scribe.}
tasmā\skp{d-vā}\app{\lem[wit={ceteri},alt={vāyoḥ}]{skm{d-vā}yoḥ}
  \rdg[wit={P}]{vāyo}}
\app{\lem[wit={ceteri}]{śarīraṃ}
  \rdg[wit={B,L}]{śarīre}}
\app{\lem[type=emendation, resp=egoscr]{calayati}
  \rdg[wit={E}]{\korr cālayati| śokam āpnoti || vivilaḥ}
  \rdg[wit={P}]{calayati śopham āpnoti vikṛtaḥ}
  \rdg[wit={B}]{cālatī | śokam āpnoti vikrutaḥ ||}
  \rdg[wit={L}]{cālayatī śokam āpnoti vikutaḥ}
  \rdg[wit={U2}]{calayati śokam āpnoti vikṛtaḥ ||}
  \rdg[wit={ceteri}]{calati}}/
%-----------------------------
%kūrmavāyur netramadhye tiṣṭhati/ nimeṣonmeṣaṃ karoti/ \E
%kūrmavāyur netramadhye           nimeṣonmeṣaṃ karoti \P
%kūrmavāyoḥ netramadhye           nimeṣonmeṣaṃ karotī/ \B
%kūrmavāyoḥ netramadhye           nimiṣonmeṣaṃ karotī... \L
%kūrmo vāyunetramadhye tiṣṭhati/  unmeṣaṃ nimeṣaṃ karoti/ \N1
%kūrmo vāyunetramadhye tiṣṭhati/  unmeṣaṃ nimeṣaṃ ca karoti// \D
%kūrmo vāyunetramadhye tiṣṭhati/  unmeṣaṃ nimeṣaṃ karoti/ \N2
%\om                                                     \U1
%kūrmavāyur netramadhye           nimiṣonmeṣaṃ karoti//            \U2
%-----------------------------
%The Kūrma-vitalwind exists within the eyes. It causes [the] opening and closing [of the eyes]. 
%-----------------------------
\app{\lem[wit={E,P,U2}, alt={kūrmavāyur}]{kūrmavāyu\skp{r-ne}}
  \rdg[wit={B,L}]{kūrmavāyoḥ}
  \rdg[wit={N1,N2,D}]{kūrmo vāyu}
  \rdg[wit={U1}]{\om}}
\skm{r-ne}tramadhye
\app{\lem[wit={E,N1,N2,D}]{tiṣṭhati}
  \rdg[wit={ceteri}]{\om}}/  
\app{\lem[wit={E,P,B,U2}]{nimeṣonmeṣaṃ}
  \rdg[wit={N1,N2}]{unmeṣaṃ nimeṣaṃ}
  \rdg[wit={D}]{unmeṣaṃ nimeṣaṃ ca}
  \rdg[wit={U1}]{\om}}
\app{\lem[wit={ceteri}]{karoti}
  \rdg[wit={B,L}]{karotī}
  \rdg[wit={U1}]{\om}}/
%-----------------------------
%kṛkalakartāvāyur  udgāraṃ karoti      \E
%kṛkalavāyur       udhāraṃ karoti      \P
%kṛkalavāyur       udhāraṃ karotī      \B
%kṛkalavāyur       uhāraṃ karotī        \L
%kṛkalavāyor       ūdgāro bhavati//    \N1
%kṛkalavāyor-------ūdgāto bhavati/      \D
%kṛkaravāyor-------ūdgāro bhavati/      \N2
%                                       \U1
%puṣkaravāyur      udgāraṃ karoti//    \U2
%-----------------------------
%From the Kṛkala-vitalwind gagging arises. 
%-----------------------------
\app{\lem[wit={N1,N2,D},alt={kṛkalavāyor}]{kṛkalavāyo\skp{r-u}}
  \rdg[wit={P,B,L}]{kṛkalavāyur}
  \rdg[wit={E}]{kṛkalakartāvāyur}
  \rdg[wit={U2}]{puṣkaravāyur}
  \rdg[wit={U1}]{\om}
}\app{\lem[type=emendation, resp=egoscr, alt={udgāro}]{\skm{r-u}dgāro}
  \rdg[wit={E,U2}]{udgāraṃ}
  \rdg[wit={P,B}]{udhāraṃ}
  \rdg[wit={L}]{uhāraṃ}
  \rdg[wit={N1,N2}]{ūdgāro}
  \rdg[wit={D}]{ūdgāto}
  \rdg[wit={U1}]{\om}}
\app{\lem[wit={N1,N2,D}]{bhavati}
  \rdg[wit={E,P,U2}]{karoti}
  \rdg[wit={B,L}]{karotī}
  \rdg[wit={U1}]{\om}}/
%-----------------------------
% devadattavāyoḥ  jṛmbhaṇaṃ bhavati/ dhanaṃjayavāyoḥ śabda utpadyate// \E
% devadattavāyor  jumbhā bhavati     dhanaṃjayavāyo  śabdāḥ utpadyete  \P
% devadattavāyor  jumbhā bhavaṃtī    dhanaṃjayavāyoḥ śabda utpadyate// \B
% devadattavāyor  jṛṃbhā bhavatī     dhanaṃjayavāyoḥ śabdaḥ utpadyate// \L
% devadattavāyor  jṛṃbha utpadyate// dhanaṃjayavāyo  śabda utpadyate// \N1
% devadattavāyor  jṛṃbha utpadyate// dhanaṃjayavāyo  śabda utpadyate// \D
% devadattavāyo   jṛṃbhotpadyate/    dhanaṃjayavāyo  śabdotpadyate// \N2
% devadattavāyor  jaṃbhā utpadyate   dhanaṃjayavāyoḥ sabta utpadyate \U1
% devadattavāyo   jṛṃbhā bhavati//   dhanaṃjayavāyoḥ śabda utpadyate// \U2
%-----------------------------
%From the Devadatta-vitalwind jawning arises. From the Dhanaṃjaya-vitalwind speech arises. 
%-----------------------------
\app{\lem[wit={ceteri}, alt={devadattavāyor}]{devadattavāyo\skp{r-jṛ}}
  \rdg[wit={E}]{devadattavāyoḥ}
  \rdg[wit={N2,U2}]{devadattavāyo}
}\app{\lem[wit={N1,D,U2},alt={jṛmbha}]{\skm{r-jṛ}mbha}
  \rdg[wit={E}]{jṛmbhaṇaṃ}
  \rdg[wit={P,B}]{jumbhā}
  \rdg[wit={L}]{jṛṃbhā}
  \rdg[wit={N2}]{jṛṃbho°}
  \rdg[wit={U1}]{jaṃbhā}}
\app{\lem[wit={N1,D,U2,U1}]{utpadyate}
  \rdg[wit={E,P,U2}]{bhavati}
  \rdg[wit={B}]{bhavaṃtī}
  \rdg[wit={L}]{bhavatī}}/
\app{\lem[wit={ceteri}]{dhanaṃjayavāyoḥ}
  \rdg[wit={P,N1,N2,D}]{dhanaṃjayavāyo}}
\app{\lem[wit={ceteri}]{śabda}
  \rdg[wit={P}]{śabdāḥ}
  \rdg[wit={L}]{śabdaḥ}
  \rdg[wit={N2}]{śabdo°}
  \rdg[wit={U1}]{sabta}}
utpadyate\dd{}
\end{prose}
\end{ekdosis}
\ekdpb*{}
%%%%%%%%%%%%%%%%%%%%%%%%%%%%%%%%%%%%%%%%%%
%%%%%%%%%%%%%%%%%%%%%%%%%%%%%%%%%%%%%%%%%%
%%%%%%%%PAGEBREAK%%%%%%%PAGEBREAK%%%%%%%%%
%%%%%%%%%%%%%%%%%%%%%%%%%%%%%%%%%%%%%%%%%%
%%%%%%%%%%%%%%%%PAGEBREAK%%%%%%%%%%%%%%%%%
%%%%%%%%%%%%%%%%%%%%%%%%%%%%%%%%%%%%%%%%%%
%%%%%%%%PAGEBREAK%%%%%%%PAGEBREAK%%%%%%%%%
%%%%%%%%%%%%%%%%%%%%%%%%%%%%%%%%%%%%%%%%%%
%%%%%%%%%%%%%%%%%%%%%%%%%%%%%%%%%%%%%%%%%%
%%%%%%%%%%%%%%%%%%%%%%%%%%%%%%%%%%%%%%%%%%
%%%%%%%%%%%%%%%%%%%%%%%%%%%%%%%%%%%%%%%%%%
%%%%%%%%PAGEBREAK%%%%%%%PAGEBREAK%%%%%%%%%
%%%%%%%%%%%%%%%%%%%%%%%%%%%%%%%%%%%%%%%%%%
%%%%%%%%%%%%%%%%PAGEBREAK%%%%%%%%%%%%%%%%%
%%%%%%%%%%%%%%%%%%%%%%%%%%%%%%%%%%%%%%%%%%
%%%%%%%%PAGEBREAK%%%%%%%PAGEBREAK%%%%%%%%%
%%%%%%%%%%%%%%%%%%%%%%%%%%%%%%%%%%%%%%%%%%
%%%%%%%%%%%%%%%%%%%%%%%%%%%%%%%%%%%%%%%%%%
%%%%%%%%%%%%%%%%%%%%%%%%%%%%%%%%%%%%%%%%%%
%%%%%%%%%%%%%%%%%%%%%%%%%%%%%%%%%%%%%%%%%%
%%%%%%%%PAGEBREAK%%%%%%%PAGEBREAK%%%%%%%%%
%%%%%%%%%%%%%%%%%%%%%%%%%%%%%%%%%%%%%%%%%%
%%%%%%%%%%%%%%%%PAGEBREAK%%%%%%%%%%%%%%%%%
%%%%%%%%%%%%%%%%%%%%%%%%%%%%%%%%%%%%%%%%%%
%%%%%%%%PAGEBREAK%%%%%%%PAGEBREAK%%%%%%%%%
%%%%%%%%%%%%%%%%%%%%%%%%%%%%%%%%%%%%%%%%%%
%%%%%%%%%%%%%%%%%%%%%%%%%%%%%%%%%%%%%%%%%%
\begin{ekdosis}
  \ekddiv{type=ed}
 \centerline{\textrm{\small{[Madhyalakṣya]}}}
 \bigskip
 \begin{prose}
   \noindent
%----------------------------
%\om                               \E
%idānī  madhyalakṣaṃ   kathyate      \P
%idānīṃ madhyalakṣaṇaṃ kathyate//  \B DSCN7164 Z.1
%idānīṃ madhye lakṣaṃ  kathyate//   \L
%idānīṃ madhyalakṣyaṃ  kathyate//   \N1
%idānīṃ madhyalakṣyaṃ  kathyate//   \D
%idānīṃ madhyalakṣaṇaṃ kathyate//  \N2
%idānīṃ madhyalakṣyaṃ  kathyate     \U1
%idānīṃ madhye lakṣyaṃ kathyate//  \U2
%-----------------------------
%Now the central fixation is taught. 
%-----------------------------
\note[type=source, labelb=180, lem={madhyalakṣyaṃ}]{Ysv (PT): idānīṃ madhyalakṣan tu kathyate siddhikārakam | śvetaṃ raktaṃ tathā pītaṃ dhūmrākāran tu nīlabham |}
\app{\lem[wit={ceteri}]{idānīṃ}
  \rdg[wit={P}]{idānī}
  \rdg[wit={E}]{\om}}
\app{\lem[wit={N1,D,U1}]{madhyalakṣyaṃ}
  \rdg[wit={B,N2}]{madhyalakṣaṇaṃ}
  \rdg[wit={P}]{madhyalakṣaṃ}
  \rdg[wit={L}]{madhye lakṣaṃ}
  \rdg[wit={U2}]{madhye lakṣyaṃ}
  \rdg[wit={E}]{\om}}
kathyate/
%-----------------------------SSP. S41!!! almost identical! 
%              aṃtha ca pītavarṇaṃ   raktavarṇaṃ vā dhūmrākāraṃ yan  nīlavarṇaṃ vā   agniśikhāsadṛśaṃ vidyutsamānaṃ   sūryamaṇḍalasadṛśaṃ     arddhacandrasadṛśaṃ jvalad  ākāśasamākāraṃ  \E
%śvetavarṇaṃ   atha     pītavarṇaṃ   raktaṃ vā      dhūmrākāraṃ yan  nīlavarṇaṃ vā   'gniśikhāsadṛśaṃ vidyutsamānaṃ   sūryamaṇdalasadṛśaṃ     arddhacaṃdrasadṛśaṃ jvalad  ākāśasamākāraṃ  \P
%śvetavaraṃ    atha     pītavarṇaṃ// rakta  vā      dhūmrākāraṃ yan  nīlavarṇaṃ vā// agniśikhāsadṛśaṃ vidyutsamānaṃ   sūryamaṇdalasadṛśaṃ/    ūrdhvacaṃdrasadṛśaṃ jvalad  ākāśasamākāraṃ// \B
%śvetavarṇaṃ   atha     pītavarṇaṃ   raktaṃ vā      dhūmrākāraṃ yan  nīlavarṇaṃ vā// agniśikhāsadṛśaṃ vidyutsamāne    sūryamaṇdalasadṛśaṃ//   ardhacaṃdrasadṛśaṃ  jvalad  ākāśasamākāra   \L
%śvetavarṇā/   atha vā  pītavarṇaṃ   raktaṃ vā      dhūmāra     va   nīlavarṇaṃ vā   agniśikhāsadṛśaṃ vidyutsamānaṃ   sūryamaṇdalaṃ sadṛśaṃ/  ūrdhvacaṃdrasadṛśaṃ jvalad  ākāśasamānakāraṃ//  \N1
%śvetavarṇaṃ// atha vā  pītavarṇaṃ   raktaṃ vā      dhūmākāro   vā   nīlavarṇaṃ vā   agniśikhāsadṛśaṃ vidyutsamānaṃ// sūryamaṇdalaṃ sadṛśaṃ// ūrdhvacaṃdrasadṛśaṃ jvalad  ākāśasamānakāraṃ//  \D
%śvetavarṇā    atha vā  pītavarṇa    raktavarṇa     dhūmravarṇa      nīlavarṇaṃ vā   agniśikhāsadṛśaṃ vidyutsamānaṃ   sūryamaṇdalasadṛśaṃ     ūrdhvacaṃdrasadṛśaṃ jvalad  ākāśasamānakāraṃ//  \N2
%svetavarṇaṃ   atha vā  pītavarṇaṃ   raktaṃ vā      dhūmrākāra  van  nīlavarṇaṃ vā   agniśikhāsadṛśaṃ vidyutsamānaṃ   sūryamaṇdalasadṛśaṃ     ārdhacaṃdrasadṛśaṃ  jalad---ā----samānākāraṃ \U1
%svatavarṇaṃ   atha vā  pītavarṇaṃ// raktaṃ vā      dhūmrākāraṃ yan  nīlavarṇaṃ vā   agniśikhāsadṛśaṃ vidyutsamānaṃ   sūryamaṇdalasadṛśaṃ     arddhacaṃdrasadṛśaṃ jvalad--ākāraṃ samākāraṃ \U2
%-----------------------------
%White-colored, or yellow-colored or red-coloured or smoke-coloured or blue-coloured, like the flame of fire, equal to a lightning, like the orb of the sun, like a half-moon, appearing like flaming space, ...  
%-----------------------------
\note[type=source, labelb=181, lem={agniśikhāsadṛśaṃ}]{Ysv (PT): agnijvālāsamānābhā vidyutpuñjasamaprabhā | ādityamaṇḍalākāramathavā candramaṇḍalam |}
\app{\lem[wit={ceteri}, alt={°śveta}]{śveta}
  \rdg[wit={U1}]{sveta°}
  \rdg[wit={U2}]{svata°}
  \rdg[wit={E}]{\om}
}śveta\app{\lem[wit={P,L,U1,U2}, alt={°varṇaṃ}]{varṇaṃ}
  \rdg[wit={P}]{°varaṃ}
  \rdg[wit={N1}]{°varṇā |}
  \rdg[wit={D}]{°varṇaṃ ||}
  \rdg[wit={E}]{\om}}
\app{\lem[wit={ceteri}]{atha}
  \rdg[wit={E}]{aṃtha}}
\app{\lem[wit={ceteri}]{vā}
  \rdg[wit={E}]{ca}
  \rdg[wit={P,B,L}]{\om}}
pīta\app{\lem[wit={ceteri}, alt={°varṇaṃ}]{varṇaṃ}
  \rdg[wit={B,U2}]{°varṇaṃ ||}
  \rdg[wit={N2}]{°varṇa}}
rakta\app{\lem[wit={E}, alt={°varṇaṃ}]{varṇaṃ}
  \rdg[wit={N2}]{°varṇa}
  \rdg[wit={ceteri}]{°ṃ}
  \rdg[wit={B}]{\om}}
\app{\lem[wit={ceteri}]{vā}
  \rdg[wit={N2}]{\om}}
\app{\lem[type=emendation, resp=egoscr]{dhūmravarṇaṃ}
  \rdg[wit={N2}]{\korr dhūmravarṇa}
  \rdg[wit={D}]{dhūmākāro}
  \rdg[wit={N1}]{dhūmāra}
  \rdg[wit={U1}]{dhūmrākāra}
  \rdg[wit={ceteri}]{dhūmrākāraṃ}}
\note[type=philcomm, labelb=182, lem={dhūmra°}]{Given the repetetive mentioning of colours compounded with °\textit{varṇaṃ} before and after the mentioning of \textit{dhūmra}°, and previous usage in the same compound it is highly likely that \textit{dhūmravarṇaṃ} was the original reading.}
\app{\lem[wit={D}]{vā}
  \rdg[wit={N1}]{va}
  \rdg[wit={U1}]{van}
  \rdg[wit={ceteri}]{yan}
  \rdg[wit={N2}]{\om}}
nīlavarṇaṃ
\app{\lem[wit={ceteri}]{vā}
  \rdg[wit={B,L}]{vā ||}}
\app{\lem[wit={P}, alt={°gni}]{'gni}
  \rdg[wit={ceteri}]{agni°}
}śikhāsadṛśaṃ vidyut\app{\lem[wit={ceteri},alt={°samānaṃ}]{samānaṃ}
  \rdg[wit={L}]{°samāne}
  \rdg[wit={D}]{°samānaṃ ||}}
sūryamaṇdala\app{\lem[wit={ceteri}, alt={°sadṛśaṃ}]{sadṛśaṃ}
  \rdg[wit={N1,D}]{°ṃ sadṛśaṃ}}
\app{\lem[wit={ceteri},alt={ardha°}]{ardha}
  \rdg[wit={U1}]{ārdha°}
  \rdg[wit={B,N1,N2,D}]{ūrdhva°}
}candrasadṛśaṃ
\app{\lem[wit={ceteri}, alt={jvalad°}]{jvala\skp{d-ā}}
  \rdg[wit={U1}]{jalad}
}\app{\lem[wit={ceteri}, alt={°ākāśa°}]{\skm{d-ā}kāśa}
  \rdg[wit={U1}]{°ā°}
  \rdg[wit={U2}]{°ākāraṃ}
}\app{\lem[wit={ceteri}, alt={°samākāraṃ}]{samākāraṃ}
  \rdg[wit={N1,N2,D,U1}]{°samānakāraṃ}
  \rdg[wit={U2}]{samakāraṃ}
  \rdg[wit={L}]{°samākāra}}/
%-----------------------------
%svaśarīraparimitaṃ      tejomanomadhye tathyaṃ kartavyam// \E
%svaśarīraparimitaṃ      tejomanomadhye lakṣyaṃ karttavyaṃ\P
%svaśarīraparimitaṃ      tejomanomadhye lakṣaṃ kartavyaṃ//  \B
%svaśarīraparimitaṃ      tejomanomadhye lakṣaṃ kartavyaṃ//  \L
%svaśarīraparimitaṃ      tejomanomadhye lakṣyaṃ karttavyaṃ//  \N1
%svaśarīraparimitaṃ      tejomanomadhye lakṣyaṃ karttavyaṃ// \D
%svaśarīraparimitaṃ      tejomanomadhye lakṣaṇaṃ karttavyaṃ//  \N2
%svaśarīraparimanomittaṃ tejomadhye     lakṣyaṃ karttavyaṃ  \U1
%svaśarīraparimitaṃ      tejomanomadhye lakṣaṃ kartavyaṃ//  \U2
%-----------------------------
%measured according to ones own body, the fixation shall be directed onto the center of the glowing mind.  
%-----------------------------
\note[type=source, labelb=183, lem={tejomanomadhye}]{Ysv (PT): jvaladākāśatulyaṃvā bhāvayed rūpamātmanaḥ | etaj jyotirmayaṃ dehaṃ manomadhye tu lakṣayet |}
svaśarīrapari\app{\lem[wit={ceteri},alt={°mitaṃ}]{mitaṃ}
  \rdg[wit={U1}]{°manomittaṃ}}
tejo\app{\lem[wit={ceteri},alt={°mano}]{mano}
  \rdg[wit={U1}]{\om}
}madhye
\app{\lem[wit={P,N1,D,U1}]{lakṣyaṃ}
  \rdg[wit={E}]{tathyaṃ}
  \rdg[wit={B,L,U2}]{lakṣaṃ}
  \rdg[wit={N2}]{lakṣaṇaṃ}}
kartavyaṃ/
%-----------------------------
%ekasmin lakṣye    kṛte sati manomadhye sthitasya malasya dāho bhavati/ \E [p.38]%
%etasmil lakṣye    kṛte sati manomadhye sthitasya         dāho bhavati \P
%etasmin lakṣe     kṛte satī manomadhye sthitasya malasya dāho bhavati \B
%etasmil lakṣe     kṛte satī manomadhye sthitasya malasya dāho bhavati... \L
%ekasmin lakṣye    kṛte sati manomadhye sthitasya malasya dāho bhavati/ \N1
%ekasmin lakṣye    kṛte sati manomadhye sthitasya malasya dāho bhavati// \D
%ekasmin lakṣaṇo   kṛte sati manomadhye sthitasya malasya dāho bhavati/ \N2
%etasmin na lakṣye kṛte satī manomadhye sthitasya malasya dāho bhavati \U1 %%%281.jpg
%etasmil lakṣe     kṛte satī manomadhye sthitasya malasya dāho bhavati// \U2
%-----------------------------
%While abiding in this fixation the burning of the impurity in the center of the mind arises. 
%-----------------------------
\note[type=source, labelb=184, lem={malasya}]{Ysv (PT): eteṣāñ ca kṛte lakṣe nānāduḥkhaṃ praṇaśyati | manas astu malo yāti mahānando bhavet tataḥ |}
\app{\lem[wit={P,L,U2},alt={etasmil}]{etasmi\skp{ll-a}}
  \rdg[wit={U1}]{etasmin}
  \rdg[wit={ceteri}]{ekasmin}
}\app{\lem[wit={ceteri},alt={lakṣye}]{\skm{ll-a}kṣye}
  \rdg[wit={B,L,U2}]{lakṣe}
  \rdg[wit={U1}]{na lakṣye}
  \rdg[wit={N2}]{lakṣaṇo}}
kṛte
\app{\lem[wit={ceteri}]{sati}
  \rdg[wit={B,L,U1,U2}]{satī}}
manomadhye sthitasya
\app{\lem[wit={ceteri}]{malasya}
  \rdg[wit={P}]{\om}}
dāho bhavati/ 
%-----------------------------
%manasaḥ   sattvaguṇaprakāśo   bhavati/     puruṣa ānandamayo bhūtvā tiṣṭhati//   \E 
%manasaḥ   sattvaguṇaḥ prakaṭo bhavati      puruṣa ānandamayo bhūtvā tiṣṭhati    \P   %%%7650.jpg
%manasaḥ// sattvaguṇo  prakaṭo  bhavati//   puruṣa ānandamayo bhūtvā tiṣṭhati// \B
%manasaḥ// sattvaguṇaḥ prakaṭo bhavati      puruṣa ānandamayo bhūtvā tiṣṭhati//  \L
%manasaḥ   sattvaguṇe  prakaṭo  bhavati/    puruṣa ānandamayo bhūtvā tiṣṭhati//    \N1
%manaḥ saḥ sattvaguṇo  prakaṭo  bhavati//   puruṣa ānandamayo bhūtvā tiṣṭhati// \D
%manasaḥ   sattvaguṇo  prakaṭo  bhavati/    puruṣa ānandamayo bhūtvā tiṣṭhati//    \N2
%manasaḥ   sattvaguṇo  prakaṭo  bhavati     puruṣa ānandamayo bhūtvā tiṣṭhati       \U1
%manasaḥ   satvaguṇaprakaśo    bhavati//    puruṣa ānandamayo bhūtvā tiṣṭhati//    \U2 %%414.jpg
%-----------------------------
%The Sattva-quality of the mind becomes revealed. After this has happend the person abides supreme bliss. 
%-----------------------------
mana\app{\lem[wit={ceteri},alt={°saḥ}]{saḥ}
  \rdg[wit={B,L}]{°saḥ ||}
  \rdg[wit={D}]{manaḥ saḥ}}
sattva\app{\lem[wit={B,D,N2,U1},alt={°guṇo}]{guṇo}
  \rdg[wit={N1}]{°guṇe}
  \rdg[wit={E,U2}]{°guṇa°}
  \rdg[wit={P,L}]{°guṇaḥ}}
\app{\lem[wit={ceteri}]{prakaṭo}
  \rdg[wit={E,U2}]{°prakāśo}}
bhavati/ puruṣa ānandamayo bhūtvā tiṣṭhati\dd{}
\end{prose}
\end{ekdosis}
%-----------------------------
%-----------------------------
%-----------------------------
\begin{ekdosis}
  \ekddiv{type=ed}
 \bigskip
 \centerline{\textrm{\small{[The Divisions of Space]}}}
 \bigskip
 \begin{prose}
%-----------------------------
%idānīm-ākāśabhedāḥ kathyante/ \E
%idānīm ākaśabhedāḥ kathyaṃte   \P
%idānīṃ ākaśabhedāḥ kathyaṃte/ \B
%idānīṃ ākaśabhedāḥ kathyate/  \L
%idānīṃ ākaśabhedāḥ kathyaṃte/ \N1
%idānīṃ ākaśabhedāḥ kathyaṃte// \D
%idānīṃ ākāśabhedāḥ kathyate/  \N2
%idānīṃ ākāśabhedāḥ kathyaṃte  \U1
%idānīm ākāśabhedāḥ kathyate// \U2
%-----------------------------
%Now the divisions of space are taught. 
%-----------------------------
\note[type=source, labelb=185, lem={ākaśabhedāḥ}]{kathyate tu devyadhunākāśaṃ pañcabhirlakṣaṇaiḥ | ākāśan tu mahākāśaṃ parākāśaṃ parātparam | tattvākāśaṃ sūryakāśamākāśaṃ pañcalakṣaṇam |}
\app{\lem[wit={E,P,U2},alt={idānīm}]{idānī\skp{m-ā}}
  \rdg[wit={ceteri}]{idānīṃ}
}\skm{m-ā}kāśabhedāḥ
\app{\lem[wit={ceteri}]{kathyante}
  \rdg[wit={L,N2,U2 }]{kathyate}}/
%-----------------------------SSP!
%te                          ākāśaḥ paramākāśaḥ mahākāśaḥ tattvākāśaḥ sūryākāśaḥ/    bāhyābhyantare nirmalaṃ nirākāram ākāśa---lakṣyaṃ  karttavyam/ \E
%teṣāṃ lakṣyāni ca kathyaṃte ākāśaḥ parākāśaḥ mahākāśaḥ tatvākāśaḥ sūryakāśaḥ        bāhyābhyaṃtare nirmalaṃ nirākāram ākāśaṃ  lakṣyaṃ  karttavyaṃ  \P
%                            ākāśaḥ paramākāśaḥ// mahākāśa// tattvākāśaḥ sūryākāśa// bāhyābhyaṃtaro nirmalaṃ nirākāram ākāśaṃ  lakṣaṃ   kartavyaṃ// \B
%                            ākāśaḥ paramākāśaḥ// mahākāśaḥ tattvākāśaḥ sūryākāśaḥ   bāhyābhyaṃtare nirmalaṃ nirākāram ākāśaṃ  lakṣaṃ   kartavyaṃ// \L
%teṣāṃ lakṣyāni  kathyate//  ākāśa, parākāśa,mahākāśa,tatvākāśa,sūryakāśa//          bāhyābhyaṃtare nirmalaṃ nirākāraṃ ākāśa---lakṣyaṃ  kartavyaṃ// \N1
%teṣāṃ lakṣyāṇi  kathyaṃte// ākāśa--parākāśamahākāśatatvākāśasūryakāśa               bāhyābhyaṃtare nirmalaṃ nirākāraṃ ākāśa---lakṣyaṃ  karttavyaṃ// \D   %%%p.11 verso
%teṣāṃ lakṣaṇāni kathyate//  ākāśa--parākāśamahākāśatatvākāśasūryakāśaḥ              bāhyābhyaṃtare nirmalaṃ nirākāraṃ ākāśa---lakṣaṇaṃ kartavyaṃ// \N2
%ṣāṃ   lakṣyāṇi  kathyaṃte   ākāśa--parākāśamahākāśatatvākāśasūryakāśa---------------bāhyābhyaṃtare nirmalaṃ nirākāraṃ ākāśa---lakṣyaṃ  karttavyaṃ  \U1
%teṣāṃ lakṣyāni  kathyaṃte// ākāśaḥ parākāśa// mahākāśaḥ// tatvākāśaḥ// sūryakāśaḥ// bāhyābhyaṃtare nirmalaṃ nirākāraṃ mākāśaṃ lakṣyaṃ  karttavyaṃ// \U2
%-----------------------------
%The fixations of them are taught: Space, beyond space, great space, space of reality, the space of the sun. The fixation onto the pure and formless space \textit{akāśa} shall be done internally as well as externally.  
%-----------------------------SSP!
\note[type=source, labelb=186, lem={ākāśaḥ}]{SSP: ākāśaṃ parākāśaṃ mahākāśaṃ tatvākaśaṃ sūryākāśamiti vyomapañcakam | bāhyābhyantare 'tyantaṃ nirmalaṃ nirākāraṃ ākāśaṃ lakṣayet |}
\app{\lem[wit={ceteri}]{teṣāṃ}
  \rdg[wit={E}]{te}
  \rdg[wit={U1}]{ṣaṃ}
  \rdg[wit={B,L}]{\om}}
\app{\lem[wit={ceteri}]{lakṣyāni}
  \rdg[wit={N2}]{lakṣaṇāni}
  \rdg[wit={B,E,L}]{\om}}
\app{\lem[wit={D,U1,U2}]{kathyante}
  \rdg[wit={P}]{ca kathyante}
  \rdg[wit={N1,N2}]{kathyate}
  \rdg[wit={B,E,L}]{\om}}/
\app{\lem[wit={B,E,L,P}]{ākāśaḥ}
  \rdg[wit={D,N1,N2,U1}]{ākāśa°}}\dd{}
\app{\lem[wit={B,E,L}]{paramākāśaḥ}
  \rdg[wit={P,U2}]{parākāśaḥ}
  \rdg[wit={N1}]{parākāśa}
  \rdg[wit={D,N2,U1}]{parākāśa°}}\dd{}
\app{\lem[wit={E,L,P,U2}]{mahākāśaḥ}
  \rdg[wit={B,N1}]{mahākāśa}
  \rdg[wit={ceteri}]{mahākāśa°}}\dd{}
\app{\lem[wit={B,E,L,U2}]{tattvakāśaḥ}
  \rdg[wit={N1}]{tatvakāśa}
  \rdg[wit={ceteri}]{tatvakāśa°}}\dd{}
\app{\lem[wit={B,E,L}]{sūryākāśaḥ}
  \rdg[wit={N2,P,U2}]{sūryakāśaḥ}
  \rdg[wit={N1}]{sūryakāśa}
  \rdg[wit={ceteri}]{sūryakāśa°}}\dd{}
bāhyābhyantare nirmalaṃ nirākāram
\app{\lem[wit={ceteri},alt={ākāśa°}]{ākāśa}
  \rdg[wit={U2}]{mākāśaṃ}
  \rdg[wit={P,B,L}]{ākāśaṃ}
}\app{\lem[wit={ceteri}, alt={°lakṣyaṃ}]{lakṣyaṃ}
  \rdg[wit={B,L}]{lakṣaṃ}
  \rdg[wit={N2}]{°lakṣaṇaṃ}}
kartavya\app{\lem[wit={E}]{kartavyam}
  \rdg[wit={ceteri}]{kartavyaṃ}}\dd{}
\end{prose}
\end{ekdosis}
\ekdpb*{}
%%%%%%%%%%%%%%%%%%%%%%%%%%%%%%%%%%%%%%%%%%
%%%%%%%%%%%%%%%%%%%%%%%%%%%%%%%%%%%%%%%%%%
%%%%%%%%PAGEBREAK%%%%%%%PAGEBREAK%%%%%%%%%
%%%%%%%%%%%%%%%%%%%%%%%%%%%%%%%%%%%%%%%%%%
%%%%%%%%%%%%%%%%PAGEBREAK%%%%%%%%%%%%%%%%%
%%%%%%%%%%%%%%%%%%%%%%%%%%%%%%%%%%%%%%%%%%
%%%%%%%%PAGEBREAK%%%%%%%PAGEBREAK%%%%%%%%%
%%%%%%%%%%%%%%%%%%%%%%%%%%%%%%%%%%%%%%%%%%
%%%%%%%%%%%%%%%%%%%%%%%%%%%%%%%%%%%%%%%%%%
%%%%%%%%%%%%%%%%%%%%%%%%%%%%%%%%%%%%%%%%%%
%%%%%%%%%%%%%%%%%%%%%%%%%%%%%%%%%%%%%%%%%%
%%%%%%%%PAGEBREAK%%%%%%%PAGEBREAK%%%%%%%%%
%%%%%%%%%%%%%%%%%%%%%%%%%%%%%%%%%%%%%%%%%%
%%%%%%%%%%%%%%%%PAGEBREAK%%%%%%%%%%%%%%%%%
%%%%%%%%%%%%%%%%%%%%%%%%%%%%%%%%%%%%%%%%%%
%%%%%%%%PAGEBREAK%%%%%%%PAGEBREAK%%%%%%%%%
%%%%%%%%%%%%%%%%%%%%%%%%%%%%%%%%%%%%%%%%%%
%%%%%%%%%%%%%%%%%%%%%%%%%%%%%%%%%%%%%%%%%%
%%%%%%%%%%%%%%%%%%%%%%%%%%%%%%%%%%%%%%%%%%
%%%%%%%%%%%%%%%%%%%%%%%%%%%%%%%%%%%%%%%%%%
%%%%%%%%PAGEBREAK%%%%%%%PAGEBREAK%%%%%%%%%
%%%%%%%%%%%%%%%%%%%%%%%%%%%%%%%%%%%%%%%%%%
%%%%%%%%%%%%%%%%PAGEBREAK%%%%%%%%%%%%%%%%%
%%%%%%%%%%%%%%%%%%%%%%%%%%%%%%%%%%%%%%%%%%
%%%%%%%%PAGEBREAK%%%%%%%PAGEBREAK%%%%%%%%%
%%%%%%%%%%%%%%%%%%%%%%%%%%%%%%%%%%%%%%%%%%
%%%%%%%%%%%%%%%%%%%%%%%%%%%%%%%%%%%%%%%%%%
\begin{ekdosis}
  \begin{prose}
    \noindent
%-----------------------------
%tataḥ paraṃ bāhyābhyantare  ṣvanandhakārasadṛśaṃ   parākāśaikyaṃ lakṣyaṃ  karttavyam// \E
%tataḥ paraṃ bāhyābhyantarai ghanāṃdhakāraṃ sadṛśa--parākāśasya   lakṣyaṃ  karttavyam \P
%tataḥ paraṃ bāhyābhyaṃtare  ghanāṃghakārasadṛśaḥ   parākāśa------lakṣaṃ   kartavyaṃ// \B
%tataḥ paraṃ bāhyābhyaṃtare        dhakārasadṛśaḥ   parākāśa------lakṣaṃ   kartavyaṃ... \L %%%%%%%%%%%%%%%%%ghana hier = dunkel, schwarz%%%% andhakāra=  finster, dunkel. Finsterniss
%tataḥ paraṃ bāhyābhyantare  ghanāṃdhakārasadṛśa----parākāśasya   lakṣyaṃ  kattavyam// \N1
%tataḥ paraṃ bāhyābhyantare  ghanāṃdhakārasadṛśa----parākāśasya   lakṣyaṃ  kattavyaṃ// \D
%tataḥ paraṃ bāhyābhyantare  ghanāṃdhakārasadṛśa----parākāśasya   lakṣaṇaṃ karttavyam// \N2
%tataḥ paraṃ bāhyābhyantare  ghanāṃdhakārasadṛśa----parākāśasya   lakṣyaṃ  karttavyaṃ \U1
%tataḥ       bāhyābhyantare  ghanāṃdhakārasadṛśaṃ   parākāśasya   lakṣaṃ   karttavyaṃ// \U2
%-----------------------------
%Moreover, the fixation of the beyond-space \textit{parākāśa} which is equal to dense darkness shall be done internally and externally.
%-----------------------------SSP!
\note[type=source, labelb=187, lem={parākāśasya}]{Ysv (PT): savāhyābhyantare nityaṃ nirākāśantu nirmalam | karttavyaṃ lakṣam ākāśaṃ sādhayet sādhanaṃ vinā | ghanāntarālasadṛśaṃ parākāśaṃ tathaiva ca |}
tataḥ
\app{\lem[wit={ceteri}]{paraṃ}
  \rdg[wit={U2}]{\om}}
\note[type=source, labelb=188, lem={parākāśasya}]{SSP: atha vā bāhyābhyantare 'tyantāndhakāranibhaṃ parākāśam avalokayet |}
\note[type=philcomm, labelb=189, lem={ghanāṃdhakāra°}]{Instead of extreme brightness as in the SSP, Rāmacandra conspicuously choose dense darkness to characterize his \textit{parākāśa}-visualization.}
bāhyābhyanta\app{\lem[wit={ceteri},alt={°re}]{re}
  \rdg[wit={P}]{°rai}}
\app{\lem[wit={ceteri},alt={ghanāndha°}]{ghanāndha}
  \rdg[wit={B}]{ghanāṃgha°}
  \rdg[wit={E}]{ṣvanandha°}
  \rdg[wit={L}]{dha°}
}\app{\lem[wit={ceteri},alt={°kāra°}]{kāra}
  \rdg[wit={P}]{°kāraṃ}
}\app{\lem[wit={ceteri},alt={°sadṛśa°}]{sadṛśa}
  \rdg[wit={E,U2}]{sadṛśaṃ}
  \rdg[wit={B,L}]{sadṛśaḥ}
}parākāśa\app{\lem[wit={ceteri},alt={°sya}]{sya}
  \rdg[wit={E}]{°ikyaṃ}
  \rdg[wit={B,L}]{°}}
lakṣ\app{\lem[wit={ceteri},alt={°yaṃ}]{yaṃ}
  \rdg[wit={B,L,U2}]{°aṃ}
  \rdg[wit={N2}]{°aṇaṃ}}
kartavyaṃ/
%-----------------------------
%tataḥ paraṃ pralayakālīna--jvalad-dāvā---nala-pūrṇaṃ  bāhyābhyantare, mahākāśalakṣyaṃ karttavyam/ \E
%tataḥ paraṃ pralayakālīna--jalad--vaḍavā-nala-pūrṇaṃ  bāhyābhyaṃtare  mahākāśaṃ lakṣyaṃ karttavyaṃ \P
%tataḥ paraṃ pralayakālīnaḥ jalad--vaḍavā-nala-pūrṇaṃ  bāhyābhyaṃtare  mahākāśalakṣaṃ kartavyaṃ// \B
%tataḥ paraṃ pralayakālīnaḥ jvalad-vaḍavā-nala-pūrṇaṃ  bāhyābhyaṃtare  mahākāśalakṣaṃ kartavyaṃ// \L
%tataḥ paraṃ pralayakālīna--jvalad-vṛddha-nala-pūrṇa---bāhyābhyaṃtare  mahākāśalakṣyaṃ karttavyaṃ// \N1 ?[S.9 verso letzte Zeile] 
%tataḥ paraṃ pralayakālīna--jvalad-dāvā---nala-pūrṇaṃ  bāhyābhyaṃtare  mahākāśaṃ lakṣaṃ karttavyaṃ// \D
%tataḥ paraṃ pralayakālīna--jvalad-vṛ-----nala-pūrṇa---bāhyābhyaṃtare  mahākāśalakṣaṃ karttavyaṃ// \N2
%tataḥ paraṃ pralayakālīta--jjala--vaḍavā-nala-pūrṇaṃ  bāhyābhyaṃtare  mahākāśaṃ lakṣyaṃ kartavyaṃ \U1
%tataḥ       pralayakālīna--jvalad-vaḍavā-nala-pūrṇa---bāhyābhyaṃtare  ghanāṃ dhakārasadṛśaṃ mahākāśasya lakṣaṃ karttavyaṃ \U2
%-----------------------------
%Moreover, the fixation of the great space (\textit{mahākāśa}) which is the plethora of the burning fire of the time of dissolution shall be done internally and externally. 
%-----------------------------SSP!
\note[type=source, labelb=190, lem={mahākaśa°}]{Ysv (PT): kalpāntāgnisamaṃ jyotir mahākāśaṃ smaret tathā |}
\note[type=testium, labelb=191, lem={mahākāśa°}]{SSP: bāhyābhyantare kālānalasaṃkāśaṃ mahākāśam avalokayet |}
tataḥ
\app{\lem[wit={ceteri}]{paraṃ}
  \rdg[wit={ceteri}]{U2}}
pralayakālī\app{\lem[wit={ceteri},alt={°na}]{na}
  \rdg[wit={B,L}]{°naḥ}
}\app{\lem[wit={ceteri},alt={°jvalad°}]{jvalad}
  \rdg[wit={P,B}]{°jalad°}
  \rdg[wit={U1}]{°jjala°}
}\app{\lem[wit={E,D},alt={°dāvā°}]{dāvā}
    \rdg[wit={B,L,P,U1,U2}]{°vaḍavā°}
    \rdg[wit={N1}]{°vṛddha°}
    \rdg[wit={N2}]{°vṛ°}
}nalapū\app{\lem[wit={ceteri},alt={°rṇaṃ}]{rṇaṃ}
  \rdg[wit={N1,N2,U2}]{°rṇa}}
bāhyābhyantare
\app{\lem[wit={ceteri},alt={mahākāśa°}]{mahākaśa}
  \rdg[wit={P,D,U1}]{mahākāśaṃ}
  \rdg[wit={U2}]{ghanāṃ dhakārasadṛśaṃ mahākāśasya}
}\app{\lem[wit={ceteri}, alt={°lakṣyaṃ}]{lakṣyaṃ}
  \rdg[wit={B,D,L,N2,U2}]{°lakṣaṃ}}
kartavvyaṃ/
%-----------------------------
%\om                                                                                                                         \E
%tataḥ paraṃ bāhyābhyaṃtare koṭidīpānāṃ prakāśaprāptau  yādṛśam aujvalyaṃ bhavati   tādṛśaṃ   tatvākāśaṃ lakṣyaṃ karttavyaṃ  \P
%tataḥ paraṃ bāhyābhyaṃtare koṭidīpānāṃ prakāśaprāpto   yādṛśam aujvalaṃ  bhavatī/  tādṛśaṃ   tatvāśa----lakṣaṃ kartavyaṃ//  \B
%tataḥ paraṃ bāhyābhyaṃtare koṭidīpānāṃ prakāśaprāpto   yādṛśam  ujvalaṃ  bhavatī/  tādṛśaṃ   tatvāśa----lakṣaṃ kartavyaṃ    \L  
%tataḥ paraṃ bāhyābhyaṃtare koṭidīpānāṃ prakāśaprāptau  yādṛśam aujvalyaṃ bhavati/  tādṛśaṃ   tatvākāśaṃ lakṣyaṃ kartavyaṃ// \N1
%tataḥ paraṃ bāhyābhyaṃtare koṭidīpānāṃ prakāśaprāptau  yādṛśam aujvalyaṃ bhavati// tādṛśaṃ   tatvākāśaṃ lakṣaṃ kartavyaṃ//  \D
%tataḥ paraṃ bāhyābhyaṃtare koṭidīpānāṃ prakāśaprāptau  yādṛśam aujvala   bhavati/  tādṛśaṃ   tatvākāśaṃ lakṣaṃ kartavyaṃ//  \N2
%tataḥ paraṃ bāhyābhyaṃtare koṭidīpānāṃ prakāśaprāptau  yādṛśam aujvalaṃ  bhavati   tādṛśaṃ   tatvākāśaṃ lakṣyaṃ kartavyaṃ   \U1
%tataḥ paraṃ bāhyābhyaṃtare koṭidīpānāṃ prakāśaprāptau  yādṛśem aujvalyaṃ bhavati   tādṛśaṃ// tatvākāśaṃ lakṣaṃ karttavyaṃ// \U2
%-----------------------------
%Moreover, for whom internally and externally the brightness of millions of blazing lights arises, he shall execute the fixation [directed onto] the reality-space (\textit{tattvakāśa}).   
%-----------------------------SSP!
tataḥ paraṃ bāhyābhyaṃtare koṭidīpānāṃ
\note[type=philcomm, labelb=192, lem={tataḥ \ldots kartavyaṃ}]{The whole sentence is omitted in E.}
\note[type=testium, labelb=193, lem={tattvākāśaṃ}]{SSP: bāhyābhyantare nijatatvakharūpaṃ tatvākāśam avalokayet |}
\note[type=source, labelb=194, lem={tattvākāśaṃ}]{Ysv (PT): koṭikoṭipradīpābhaṃ tattvākāśaṃ smaret tathā |}
prakāśaprā\app{\lem[wit={ceteri},alt={°ptau}]{ptau}
  \rdg[wit={B,L}]{°pto}}
yādṛśaṃ
\app{\lem[wit={ceteri}]{aujvalyaṃ}
  \rdg[wit={L}]{ujvalaṃ}}
bhava\app{\lem[wit={ceteri},alt={°ti}]{ti}
  \rdg[wit={B,L}]{°tī}}/
tādṛśaṃ
tattvā\app{\lem[wit={ceteri}, alt={°kāśaṃ}]{kāśaṃ}
  \rdg[wit={B,L}]{°śa°}}
\app{\lem[wit={P,N1,U1}]{lakṣyaṃ}
  \rdg[wit={B,D,L,N2,U2}]{lakṣaṃ}}
kartavyaṃ/
%-----------------------------
%tataḥ        bāhyābhyantare  prakāśa-mānayarsūsahitaṃ        sūryākāśaṃ lakṣyaṃ karttavyam/ \E [p.39]
%tataḥ paścād bāhyābhyaṃtare  prakāśa-māgasūryaṃ biṃbasahitaṃ sūryākāśalakṣyaṃ   karttavyaṃ ... \P
%      paccā  bāhyābhyaṃtare  prakāśa-mān sūryabiṃbasahita----sūryakāśalakṣaṃ    kartavyaṃ mataḥ ... \B
%      paccā  bāhyābhyaṃtare  prakāśa-mān sūryabiṃbasahita----sūryakāśalakṣaṃ    kartavyaṃ mataḥ ... \L 
%tataḥ paścāt bāhyābhyaṃtare  prakāśa-mānasūryabiṃbasahitaṃ   sūryakāśaṃ lakṣyaṃ karttavyaṃ// \N1
%tataḥ paścāt bāhyābhyaṃtare  prakāśa-mānasūryabiṃbasahitaṃ   sūryakāśaṃ lakṣyaṃ karttavyaṃ// \D
%tataḥ paścād     ābhyaṃtare  prakāśa-mānasūryabiṃbasahitaṃ   sūryakāśaṃ lakṣaṃ  karttavyaṃ// \N2
%tataḥ paścāt bāhyabhyaṃttare prakāśa-mānasūryabiṃbasāhitaṃ   sūryakāśaṃ lakṣyaṃ karttavyaṃ \U1
%tataḥ paścād bāhyābhyaṃtare  prakāśa-mānasūryabiṃbasāhitaṃ   sūryākāśaṃ lakṣyaṃ karttavyaṃ// \U2
%-----------------------------
%After that the fixation of the sun-space (\textit{sūryakāśa}) which is associated with sundisk's appearance of light shall be done internally and externally.   
%-----------------------------SSP!
\note[type=source, labelb=195, lem={sūryakāśaṃ}]{SSP: atha vā bāhyābhyantare sūryakoṭisadṛśaṃ sūryākāśam avalokayet |}
\note[type=source, labelb=196, lem={sūryakāśaṃ}]{Ysv (PT): sūryākāśaṃ tathā koṭisūryavindusamaṃ smaret | savāhyābhyantare caivamākāśaṃ lakṣayettu yaḥ | śivavadviharedviśve pāpapuṇyavivarjitaḥ | eteṣāñ caiva lakṣeṇa karmadvārā 'ghamāharet}
\app{\lem[wit={ceteri}]{tataḥ}
  \rdg[wit={B,L}]{\om}}
\app{\lem[wit={ceteri}, alt={paścād}]{paścā\skp{d-bā}}
  \rdg[wit={N1,N2,U1}]{paścāt}
  \rdg[wit={B,L}]{paccā}
  \rdg[wit={E}]{\om}}
\app{\lem[wit={ceteri},alt={bāhyā°}]{\skm{d-bā}hyā}
  \rdg[wit={N2}]{ā°}
}bhyaṃtare
prakāśa\app{\lem[wit={ceteri},alt={°māna°}]{māna}
  \rdg[wit={P}]{°māga°}
  \rdg[wit={B,L}]{°mān}
}\app{\lem[wit={ceteri},alt={°sūrya°}]{sūrya}
  \rdg[wit={E}]{°yarsū°}
  \rdg[wit={P}]{°sūryaṃ}
}\app{\lem[wit={ceteri},alt={°bimba°}]{bimba}
  \rdg[wit={E}]{\om}
}\app{\lem[wit={ceteri},alt={°sahitaṃ}]{sahitaṃ}
  \rdg[wit={B,L}]{°sahita°}}
sūryakā\app{\lem[wit={ceteri},alt={°śaṃ}]{śaṃ}
  \rdg[wit={B,L,P}]{°śa°}}
lakṣ\app{\lem[wit={ceteri}, alt={°yaṃ}]{yaṃ}
  \rdg[wit={B,L,N2}]{°aṃ}}
\app{\lem[wit={ceteri}]{kartavyaṃ}
  \rdg[wit={B,L}]{kartavyaṃ mataḥ}}/
%-----------------------------
%eteṣāṃ lakṣyāṇāṃ kāraṇāt   śarīraṃ rogāsaṃsargi    bhavati// \E
%eteṣāṃ lakṣāṇāṃ  karaṇāt   śarīre  rogasaṃsargo na bhavati \P %%%7651.jpg
%eteṣāṃ lakṣaṇaṃ  karaṇāt// śarīre  rogasaṃsargo na bhavatī/ \B
%eteṣāṃ lakṣaṃ    karaṇāt   śarīre  rogasaṃsargo na bhavati... \L
%eteṣāṃ lakṣyaṇāṃ karaṇāt   śarīra--rohasaṃsarge na bhavati/ \N1
%eteṣāṃ lakṣyāṇāṃ karaṇāt   śarīra--rohasaṃsargo na bhavati// \D
%eteṣāṃ lakṣāṇā---kāraṇāc---charīra-rogāsaṃsargo na bhavati// \N2
%eteṣāṃ lakṣyāṇāṃ karaṇāt   śarīra--rogāsaṃsargo na bhavati \U1
%eteṣāṃ lakṣyāṇāṃ karaṇāt// śarīre  rogāsaṃsargo na bhavati \U2
%-----------------------------
%From the execution of these fixations contact of diseases does not arise within the body. 
%-----------------------------
eteṣāṃ la\app{\lem[wit={ceteri},alt={°kṣyāṇāṃ}]{kṣyāṇāṃ}
  \rdg[wit={P}]{°kṣāṇāṃ}
  \rdg[wit={B}]{°kṣaṇaṃ}
  \rdg[wit={L}]{°kṣaṃ}
  \rdg[wit={N2}]{°kṣāṇā}}
\app{\lem[wit={N2}, alt={kāraṇāc}]{kāraṇā\skp{c-cha}}
  \rdg[wit={E}]{kāraṇāt}
  \rdg[wit={ceteri}]{karaṇāt}
}\app{\lem[wit={N2}, alt={charīre}]{\skm{c-cha}rīre}
  \rdg[wit={N1,D}]{śarīra°}
  \rdg[wit={B,P,L,U2}]{śarīre}
  \rdg[wit={E}]{°śarīraṃ}}
\app{\lem[wit={ceteri}, alt={rogāsaṃsargo}]{rogāsaṃsargo}
  \rdg[wit={E}]{rogāsaṃsargi}}
\app{\lem[wit={ceteri}]{na}
  \rdg[wit={E}]{\om}}
bhava\app{\lem[wit={ceteri}, alt={°ti}]{ti}
  \rdg[wit={B}]{°tī}}/
% -----------------------------
%tathā valitapalitaṃ   puṇyaṃ pāpaṃ    na bhavati//    \E
%tathā valitapalitaṃ   puṇyāṃ pāpaṃ ca na bhavati   \P
%tathā// valitapalitaṃ puṇyāṃ pāpaṃ ca na bhavatī// \B
%tathā valitaṃ palitaṃ puṇyāṃ pāpaṃ ca na bhavatī// \L
%tathā valitaṃ palitaṃ puṇyaṃ pāpaṃ ca na bhavati// \N1
%tathā valitaṃ palitaṃ puṇyaṃ pāpaṃ ca na bhavati// \D
%tathā valitaṃ palitaṃ puṇyaṃ pāpaṃ ca na bhavati// \N2
%tathā valitaṃ palitaṃ puṇyaṃ pāpaṃ ca na bhati \U1
%tathā valīpalitaṃ     puṇyaṃ pāpaṃ ca na bhavati \U2
%-----------------------------
%Thus wrinkles and grey hair, sin or merit does not arise. 
%-----------------------------
tathā
\app{\lem[wit={L,D,N1,N2}, alt={valitaṃ palitaṃ}]{valitaṃ palitaṃ}
  \rdg[wit={N2}]{valīpalitaṃ}
  \rdg[wit={B,E,P}]{valitapalitaṃ}}
pu\app{\lem[wit={ceteri},alt={°ṇyaṃ}]{ṇyaṃ}
  \rdg[wit={B,L}]{°ṇyāṃ}}
pāpaṃ
\app{\lem[wit={ceteri}]{ca}
  \rdg[wit={E}]{\om}}
na
\app{\lem[wit={ceteri}]{bhavati}
  \rdg[wit={B,L}]{bhavatī}
  \rdg[wit={U1}]{bhati}}/
\end{prose}
\end{ekdosis}
\begin{ekdosis}
\begin{tlg}
% -----------------------------
%          navacakraṃ kalādhāraṃ trilakṣyaṃ vyomapaṃcakam/ \E
%          navacakraṃ kalādhāraṃ trilakṣyaṃ vyomapaṃcakaṃ  \P
%śloka     navacakraṃ kalādhāraṃ trilakṣaṃ  vyomapaṃcakam/ \B
%//śloka// navacakraṃ kalādhāraṃ trilakṣaṃ  vyomapaṃcakam... \L %%%%%%%%%%%%GREP THIS%%%%%%%%%%%%% SSP 2.31!!!
%          navacakra--kalādhāraṃ trilakṣyaṃ vyomapaṃcakaṃ/ \N1
%          navacakra--kalādhāraṃ trilakṣyaṃ vyomapaṃcakaṃ// \D
%          navacakra--kalādhāraṃ trilakṣaṃ  vyomapaṃcakaṃ/ \N2
%          navacakraṃ kalādhāraṃ trilakṣyaṃ vyomapaṃcakaṃ \U1 %%%282.jpg
%          navacakraṃ kalādhāraṃ trilakṣyaṃ vyomapaṃcakaṃ// \U2
%-----------------------------
%The nine cakras, the sixteen Adhāras, the three lakṣyas and die five spaces. 
%-----------------------------
\tl{
\app{\lem[wit={ceteri}]{navacakraṃ}
  \rdg[wit={B,L}]{śloka navacakraṃ}
  \rdg[wit={D,N1,N2}]{navacakra°}}
kalādhāraṃ
\note[type=source, labelb=197, lem={navacakraṃ}]{SSP: navacakraṃ kalādhāraṃ trilakṣyaṃ vyomapañcakam | samyag etan na jānāti sa yogī nāmadhārakaḥ||2.31|| NT: ataḥ paraṃ pravakṣyāmi dhyānaṃ sūkṣmam anuttamam | ṛtucakraṃ svarādhāraṃ trilakṣyaṃ vyomapañcakam ||7.1||}
\note[type=testium, labelb=198, lem={navacakraṃ}]{Ysv (PT): navacakraṃ kalādhāraṃ trilakṣaṃ vyomapañcakam | svadehe yo na jānāti sa yogī nāmadhārakaḥ |}
\app{\lem[wit={ceteri}, alt={°kṣyaṃ}]{trilakṣyaṃ}
    \rdg[wit={B,L,N2}]{trilakṣaṃ}}
  vyomapaṃcakaṃ/}\\
%-----------------------------
%svadehe yo na jānāti sa yogī nāmadhārakaḥ//       \E
%svadehe yo na jānāti sa yogī nāmadhārakaḥ 1       \P
%svadehe yo na jānāti sa yogī nāmadhārakaḥ//1//    \B
%svadehe yo na jānāti sa yogī nāmadhārakaḥ//1//   \L
%samakriyā  na jānāti sa yogī nāmadhāraka//           \N1
%samakriyā  na jānāti sa yogī nāmadhārakaḥ//           \D
%samakriyā  na jānāti sa yogī nāmadhāraka//           \N2
%samakriyā  na jānāti sa yogī nāmadhārakaḥ            \U1
%svadehe yo na jānāti sa yogī nāmadhārakaḥ        \U2
%-----------------------------
%Who does not know [them?] within ones own body, he is only a Yogin by name. 
%-----------------------------
\tl{\app{\lem[wit={ceteri}]{svadehe yo}
  \rdg[wit={D,N1,N2,U1}]{samakriyā}} 
na jānāti sa yogī nāmadhārakaḥ\dd{}}
\end{tlg}
\end{ekdosis}
\ekdpb*{}
%%%%%%%%%%%%%%%%%%%%%%%%%%%%%%%%%%%%%%%%%%
%%%%%%%%%%%%%%%%%%%%%%%%%%%%%%%%%%%%%%%%%%
%%%%%%%%PAGEBREAK%%%%%%%PAGEBREAK%%%%%%%%%
%%%%%%%%%%%%%%%%%%%%%%%%%%%%%%%%%%%%%%%%%%
%%%%%%%%%%%%%%%%PAGEBREAK%%%%%%%%%%%%%%%%%
%%%%%%%%%%%%%%%%%%%%%%%%%%%%%%%%%%%%%%%%%%
%%%%%%%%PAGEBREAK%%%%%%%PAGEBREAK%%%%%%%%%
%%%%%%%%%%%%%%%%%%%%%%%%%%%%%%%%%%%%%%%%%%
%%%%%%%%%%%%%%%%%%%%%%%%%%%%%%%%%%%%%%%%%%
%%%%%%%%%%%%%%%%%%%%%%%%%%%%%%%%%%%%%%%%%%
%%%%%%%%%%%%%%%%%%%%%%%%%%%%%%%%%%%%%%%%%%
%%%%%%%%PAGEBREAK%%%%%%%PAGEBREAK%%%%%%%%%
%%%%%%%%%%%%%%%%%%%%%%%%%%%%%%%%%%%%%%%%%%
%%%%%%%%%%%%%%%%PAGEBREAK%%%%%%%%%%%%%%%%%
%%%%%%%%%%%%%%%%%%%%%%%%%%%%%%%%%%%%%%%%%%
%%%%%%%%PAGEBREAK%%%%%%%PAGEBREAK%%%%%%%%%
%%%%%%%%%%%%%%%%%%%%%%%%%%%%%%%%%%%%%%%%%%
%%%%%%%%%%%%%%%%%%%%%%%%%%%%%%%%%%%%%%%%%%
%%%%%%%%%%%%%%%%%%%%%%%%%%%%%%%%%%%%%%%%%%
%%%%%%%%%%%%%%%%%%%%%%%%%%%%%%%%%%%%%%%%%%
%%%%%%%%PAGEBREAK%%%%%%%PAGEBREAK%%%%%%%%%
%%%%%%%%%%%%%%%%%%%%%%%%%%%%%%%%%%%%%%%%%%
%%%%%%%%%%%%%%%%PAGEBREAK%%%%%%%%%%%%%%%%%
%%%%%%%%%%%%%%%%%%%%%%%%%%%%%%%%%%%%%%%%%%
%%%%%%%%PAGEBREAK%%%%%%%PAGEBREAK%%%%%%%%%
%%%%%%%%%%%%%%%%%%%%%%%%%%%%%%%%%%%%%%%%%%
%%%%%%%%%%%%%%%%%%%%%%%%%%%%%%%%%%%%%%%%%%
\begin{ekdosis}
  \ekddiv{type=ed}
  \centerline{\textrm{\small{[The Order of Cakras]}}}
  \bigskip
 \begin{prose}
   \noindent
%-----------------------------
%idānīṃ cakrāṇām anukramaḥ  kathyate/    \E
%idānīṃ cakrāṇām anukramaḥ  kathyate     \P
%idānīṃ cakrāṇām anukramaḥ//             \B
%idānīṃ cakrāṇām anukramaḥ//             \L 19.jpg 
%idānīṃ cakrāṇām anukrama   kathyaṃte/   \N1
%idānīṃ cakrāṇām anukramā   kathyaṃte//  \D
%idānīṃ cakrānām-anukramā   kathyaṃte/   \N2
%idānīṃ cakrānām anukramaḥ  kathyate     \U1
%idānīṃ cakrānām anukramaḥ  kathyate//   \U2
%-----------------------------
%Now the practice of the cakras is explained. 
%-----------------------------
idānīṃ cakrānām-\app{\lem[wit={ceteri}, alt={anukramaḥ}]{anukramaḥ}
  \rdg[wit={N1}]{anukrama}
  \rdg[wit={D,N2}]{anukramā}}
\app{\lem[wit={ceteri}]{kathyate}
  \rdg[wit={D,N1,N2}]{kathyaṃte}}/ \\
\note[type=source, labelb=199, lem={cakrāṇāṃ}]{SSP: atha piṇḍavicāraḥ kathyate piṇḍe navacakrāṇi |}
\note[type=philcomm, labelb=200, lem={cakrāṇāṃ}]{Even tough Rāmacandra's descriptions of the \textit{cakra}s are more brief in this section, they are certainly based on the respective passage in the SSP, since what follows in both texts is the description of the 16 \textit{ādhāra}s. Structurally it seems redundant of Rāmacandra to add another account of the ninefold \textit{cakra}-system.}
%-----------------------------
%ādhāre brahmacakram/    ādhāropari liṃgamūle sbādhiṣṭhānacakram/     nābhau maṇipūrakacakram/     hṛdaye anāhatacakram/     kaṇṭhasthāne viśuddhicakram/     \E
%ādhāre brahmacakraṃ 1   ādhāropari liṃgamūle svādhiṣṭhānacakram 2    nābhau maṇipūrakacakraṃ      hṛdaye 'nāhatacakraṃ 4    kaṃṭhasthāne viśuddhicakraṃ 5    \P
%ādhāro brahmacakram/    ādhāropari liṃgamūle svādhiṣṭhānacakraṃ//2// nābhau maṇipūrakacakram//3   hṛdaye anāhatacakram// 4  kaṇṭhasthāne viśuddhicakraṃ//    \B
%ādhāro brahmacakram//   ādhāropari liṃgamūle svādhiṣṭhānacakraṃ//2// nābhau maṇipūrakacakram//3// hṛdaye anāhatacakram//4// kaṇṭhasthāne viśuddhacakraṃ//    \L
%ādhāre brahmacakraṃ                liṃge     svādhiṣṭhānacakram/     nābhau maṇipūrakacakram/     hṛdaye viśuddhacakraṃ/    kaṇṭhasthāne anāhatacakraṃ/      \N1
%ādhāre brahmacakraṃ                liṃge     svādhiṣṭhānacakram//    nābhau maṇipūrakacakraṃ//    hṛdaye viśuddhacakraṃ//   kaṃṭhasthāne anāhatacakraṃ//     \D
%ādhāre brahmacakraṃ                liṃge     svādhiṣṭhānacakram//    nābhau maṇipūrakacakram/     hṛdaye viśuddhacakraṃ/    kaṇṭhasthāne anāhatacakraṃ       \N2
%ādhāre brahmacakraṃ                liṃge     svādhiṣṭhānacakraṃ      nābhau maṇipūrakacakraṃ      hṛdaye viśuddhacakraṃ     kaṇṭhasthāne anāhatacakraṃ       \U1
%ādhāre brahmacakraṃ//1// ādhāropariliṃgamūle svādhiṣṭhānacakraṃ//2// nābhau maṇipūrakacakraṃ//3// hṛdaye anāhatacakraṃ//4// kaṇṭhasthāne viśuddhacakraṃ//5// \U2
%-----------------------------
%At the pelvic floor there is the Brahmacakra. Above the pelvic floor at the root of the gender is the Svadiṣṭhānacakra. At the navel there is the Maṇipūrakacakra. At the heart the Anāhatacakra. Situated within the throat is the Viśuddhicakra. 
%-----------------------------
\note[type=source, labelb=201, lem={brahmacakram}]{SSP: ādhāre brahmacakraṃ tridhāvartaṃ bhagamaṇḍalākāram | tatra mūlakandaḥ | tatra śaktiṃ pāvakākārāṃ dhyāyet | tatraiva kāmarūpapīṭhaṃ sarvakāmaphalapradaṃ bhavati ||2.1||}
\app{\lem[wit={ceteri}]{ādhāre}
  \rdg[wit={B,L}]{ādhāro}}
\app{\lem[wit={B,E,L}]{brahmacakram}
  \rdg[wit={ceteri}]{brahmacakraṃ}} 1 \dd{} 
\app{\lem[wit={ceteri}]{ādhāropari}
  \rdg[wit={D,N1,N2,U1}]{\om}}
\app{\lem[wit={ceteri}]{liṅgamūle}
  \rdg[wit={D,N1,N2,U1}]{liṅge}}
\note[type=source, labelb=202, lem={svādhiṣṭhāna°}]{SSP: dvitīyaṃ svādhiṣṭhānacakram | tanmadhye paścimābhimukhaṃ liṅgaṃ pravālāṅkurasadṛśaṃ dhyāyet | tatraivoḍyānapīṭhaṃ jagadākarṣaṇaṃ bhavati ||2.2||}
\app{\lem[wit={E,D,P,N1,N2}]{svādhiṣṭhānacakram}
  \rdg[wit={ceteri}]{svādhiṣṭhānacakraṃ}} 2 \dd{}
\note[type=source, labelb=203, lem={maṇipūraka°}]{SSP:tṛtīyaṃ nābhicakraṃ pañcāvartaṃ sarpavat kuṇḍalākāram | tanmadhye kuṇḍalinīṃ śaktiṃ bālārkakoṭisannibhāṃ dhyāyet | sā madhyā śaktiḥ sarvasiddhidā bhavati ||2.3||}
nābhau \app{\lem[wit={E,P,L,N1,N2}]{maṇipūrakacakram}
  \rdg[wit={ceteri}]{maṇipūrakacakraṃ}} 3 \dd{} \\
\note[type=source, labelb=204, lem={anāhata°}]{SSP: caturthaṃ hṛdayacakram aṣṭadalakamalam adhomukhaṃ tanmadhye karṇikāyāṃ liṅgākārāṃ jyotīrūpām dhyāyet | saiva haṃsakalā sarvendriyavaśyā bhavati ||2.4||}
hṛdaye
\app{\lem[wit={P}, alt={'nāhata°}]{'nāhata}
  \rdg[wit={E,B,L,U2}]{anāhata°}
  \rdg[wit={ceteri}]{viśuddha°}
}\app{\lem[wit={E,B,L}]{cakram}
  \rdg[wit={ceteri}]{cakraṃ}} 4 \dd{}
kaṇṭhasthāne
\app{\lem[wit={E,P,B,L,U2}]{viśuddhicakram}
  \rdg[wit={ceteri}]{anāhatacakraṃ}} 5 \dd{} 
\note[type=source, labelb=205, lem={viśuddhi°}]{SSP: pañcamaṃ kaṇṭhacakraṃ caturaṅgulam | tatra vāma iḍā candranāḍī | dakṣiṇe piṅgalā sūryanāḍī | tanmadhye suṣumnāṃ dhyāyet | saiva anāhatakalā anāhatasiddhidā bhavati ||2.5||}
%-----------------------------
%ṣaṣṭhaṃ tālucakram/     bhruvor madhye ājñācakram/       brahmasthāne        kālacakram/     navamam         ākāśacakram/       etat--paraṃ śūnyam/              \E
%ṣaṣṭhaṃ tālucakraṃ 6    bhruvor madhye agnejacakraṃ 7    brahmasthāne        kālacakraṃ 8    navamaṃ         ākāśacakraṃ 8      tataḥ paraṃ śūnyaṃ               \P
%ṣaṣṭhaṃ tālucakre/6     bhruvor madhye ājñāyacakraṃ/     brahmasthāne        kālacakraṃ// 8  navamaṃ         ākāśacakraṃ/9      tat---paraṃ śūnyam/              \B
%ṣaṣṭha  tālucakre//6//  bhruvor madhye āgneyacakraṃ//7// brahmasthāne        kālacakraṃ//8// navamaṃ         ākāśacakraṃ//9//   tat---paraṃ śūnyam//             \L
%ṣaṣṭhaṃ tālucakram/     bhruvor madhye ājñācakram        brahmaraṃdhrasthāne kālacakraṃ/     navamaṃ         ākāśacakram/       tat---paramaśūnyaṃ/              \N1
%ṣaṣṭhaṃ tālucakraṃ//    bhruvor madhye ājñācakraṃ//      brahmaraṃdhrasthāne kālacakraṃ//    navamaṃ         ākāśacakram/       tat---paraṃ// tatparamaśūnyaṃ// \D
%ṣaṣṭhaṃ tālucakram/     bhruvor madhye ājñācakram        brahmaraṃdhrasthāne kālacakraṃ/     navama          ākāśacakram       tata---paraśūnyaṃ/               \N2
%ṣaṣṭhaṃ tālucakraṃ      bhruvor madhye ājñācakram        brahmaraṃdhrasthāne brahmacakraṃ    navamaṃ rattu?! ākāśacakram         tat--paraśūnyaṃ                \U1
%        tālucakra //6// bhruvor madhye ājñācakram//7//   brahmaraṃdhrasthāne kalācakraṃ//8//                 ākāśacakram ūrdhvaṃ tat--paraṃ śūnyaṃ//9//         \U2
%-----------------------------
%The sixth is the cakra of the palate. In the center of the eyebrows is the Ājñācakra. At the opening of Brahma is the Kālacakra. The ninth is the Ākāśacakra. It is supreme emptiness. 
%-----------------------------
\note[type=source, labelb=206, lem={tālu°}]{SSP: ṣaṣṭhaṃ tālucakram | tatrāmṛtadhārāpravāhaḥ | ghaṃṭikāliṅgaṃ mūlarandhraṃ rājadantaṃ śaṃkhinīvivaraṃ daśamadvāram | tatra śūnyaṃ dhyāyet | cittalayo bhavati ||2.6||}
\app{\lem[wit={ceteri}]{ṣaṣṭhaṃ}
  \rdg[wit={L}]{ṣaṣṭha°}}
\app{\lem[wit={E,N1,N2}]{tālucakram}
  \rdg[wit={D,P,U1}]{tālucakraṃ}
  \rdg[wit={B,L}]{tālucakre}
  \rdg[wit={U2}]{tālucakra}} 6 \dd{} \\
bhruvor madhye
\note[type=source, labelb=207, lem={ājñā°}]{SSP: aptamaṃ bhrūcakraṃ madhyamāṅguṣṭhamatram | tatra jñānanetraṃ dīpaśikhākāraṃ dhyāyet | tatra vāksiddhir bhavati ||2.7||}
\app{\lem[wit={ceteri}, alt={°ājñā}]{ājñā}
  \rdg[wit={P}]{agneja}
  \rdg[wit={L}]{āgneya}
  \rdg[wit={B}]{ājñāya}
}\app{\lem[wit={E,D,N1,N2,U1,U2}]{cakraṃ}
  \rdg[wit={B,D,P,L}]{cakram}}  7 \dd{}
\note[type=source, labelb=208, lem={brahmarandhra°}]{SSP: aṣṭamaṃ brahmarandhraṃ nirvāṇacakraṃ sūcikāgrabhedyam | tatra dhūmaśikhākāraṃ dhyāyet | tatra jālandharapīṭhaṃ mokṣapradaṃ bhavati ||2.8||}
brahma\app{\lem[wit={ceteri}, alt={°randhra°}]{randhra}
  \rdg[wit={B,E,L,P}]{\om}}sthāne
\app{\lem[wit={ceteri}, alt={°kāla}]{kāla}
  \rdg[wit={U1}]{brahma°}
}\app{\lem[wit={E}]{cakram}
  \rdg[wit={ceteri}]{cakraṃ}} 8 \dd{}
\note[type=source, labelb=209, lem={ākāśa°}]{SSP: navamam ākāśacakraṃ soḍaśadalakamalam ūrdhvamukham | tanmadhye karṇikāyāṃ trikūṭākārāṃ tadūrdhvaśaktiṃ tāṃ paramaśunyāṃ dhyāyet | tatraiva pūrṇagiripīṭhaṃ sarveṣṭasiddhir bhavati ||2.9|| iti navacakravicāraḥ ||}
\app{\lem[wit={E}, alt={navamam}]{navama\skp{m-ā}}
  \rdg[wit={N2}]{navama}
  \rdg[wit={U1}]{navamaṃ rattu}
  \rdg[wit={ceteri}]{navamaṃ}}
\skm{m-ā}kāśa\app{\lem[wit={E,D,N1,N2,U1,U2}]{cakram}
  \rdg[wit={B,L,P}]{cakraṃ}} \dd{} 9\\
\app{\lem[wit={B,L,D,N1,U1,U2}, alt={tat°}]{ta\skp{t-pa}}
  \rdg[wit={E}]{etat}
  \rdg[wit={P}]{tataḥ}
  \rdg[wit={N2}]{tata}
}\app{\lem[wit={N1},alt={°parama°}]{\skm{t-pa}rama}
  \rdg[wit={E,P,B,L,D,U2}]{°paraṃ}
  \rdg[wit={N2,U1}]{para°}
}\app{\lem[wit={B,E,L}, alt={°śūnyam}]{śūnyam}
  \rdg[wit={P,N1,N2,U1,U2}]{°śūnyaṃ}
  \rdg[wit={D}]{tatparamaśūnyaṃ}}\dd{}\\
\end{prose}
\end{ekdosis}
\ekdpb*{}
%%%%%%%%%%%%%%%%%%%%%%%%%%%%%%%%%%%%%%%%%%
%%%%%%%%%%%%%%%%%%%%%%%%%%%%%%%%%%%%%%%%%%
%%%%%%%%PAGEBREAK%%%%%%%PAGEBREAK%%%%%%%%%
%%%%%%%%%%%%%%%%%%%%%%%%%%%%%%%%%%%%%%%%%%
%%%%%%%%%%%%%%%%PAGEBREAK%%%%%%%%%%%%%%%%%
%%%%%%%%%%%%%%%%%%%%%%%%%%%%%%%%%%%%%%%%%%
%%%%%%%%PAGEBREAK%%%%%%%PAGEBREAK%%%%%%%%%
%%%%%%%%%%%%%%%%%%%%%%%%%%%%%%%%%%%%%%%%%%
%%%%%%%%%%%%%%%%%%%%%%%%%%%%%%%%%%%%%%%%%%
%%%%%%%%%%%%%%%%%%%%%%%%%%%%%%%%%%%%%%%%%%
%%%%%%%%%%%%%%%%%%%%%%%%%%%%%%%%%%%%%%%%%%
%%%%%%%%PAGEBREAK%%%%%%%PAGEBREAK%%%%%%%%%
%%%%%%%%%%%%%%%%%%%%%%%%%%%%%%%%%%%%%%%%%%
%%%%%%%%%%%%%%%%PAGEBREAK%%%%%%%%%%%%%%%%%
%%%%%%%%%%%%%%%%%%%%%%%%%%%%%%%%%%%%%%%%%%
%%%%%%%%PAGEBREAK%%%%%%%PAGEBREAK%%%%%%%%%
%%%%%%%%%%%%%%%%%%%%%%%%%%%%%%%%%%%%%%%%%%
%%%%%%%%%%%%%%%%%%%%%%%%%%%%%%%%%%%%%%%%%%
%%%%%%%%%%%%%%%%%%%%%%%%%%%%%%%%%%%%%%%%%%
%%%%%%%%%%%%%%%%%%%%%%%%%%%%%%%%%%%%%%%%%%
%%%%%%%%PAGEBREAK%%%%%%%PAGEBREAK%%%%%%%%%
%%%%%%%%%%%%%%%%%%%%%%%%%%%%%%%%%%%%%%%%%%
%%%%%%%%%%%%%%%%PAGEBREAK%%%%%%%%%%%%%%%%%
%%%%%%%%%%%%%%%%%%%%%%%%%%%%%%%%%%%%%%%%%%
%%%%%%%%PAGEBREAK%%%%%%%PAGEBREAK%%%%%%%%%
%%%%%%%%%%%%%%%%%%%%%%%%%%%%%%%%%%%%%%%%%%
%%%%%%%%%%%%%%%%%%%%%%%%%%%%%%%%%%%%%%%%%%
\begin{ekdosis}
  \ekddiv{type=ed}
 \centerline{\textrm{\small{[The sixteen Container]}}}
 \bigskip
 \begin{prose}
   \noindent
%-----------------------------
%idānīm ādhāracakrasya bhedāḥ kathyanta/   \E
%idānīm ādhāracakrasya bhedaḥ kathyate     \P
%idānīm ādhāracakrasya bhedā  kathyaṃte/    \B DSCN7165.jpg Z.3
%idānīm ādhāracakrasya bhedā  kathyaṃte//   \L
%idānīm ādhāracakrasya bhedaḥ kathyate/    \N1
%idānīṃ ādhāracakrasya bhedaḥ kathyate//   \D
%idānī  ādhāracakrasya bhedaḥ kathyaṃte/   \N2
%idānīṃ ādhāracakrasya bhedāḥ kathyaṃte    \U1
%idānīṃ ādhāracakrasya bhedāḥ kathyaṃte // \U2
%-----------------------------
%Now the  divisions of the container-\textit{cakra}s are taught.
%-----------------------------
\note[type=source, labelb=210, lem={ādhāracakrasya}]{SSP: atha ṣoḍaśādhārāḥ kathyante |}
\note[type=source, labelb=211, lem={ādhāracakrasya}]{Ysv (PT=YK): ṣoḍaśādhārabhedan tu śṛṇu devi viśeṣataḥ |}
\app{\lem[wit={ceteri}, alt={idānīm}]{idānī\skp{m-ā}}
  \rdg[wit={N2}]{idānī}
}\skm{m-ā}dhāracakrasya
\app{\lem[wit={ceteri}]{bhedāḥ}
  \rdg[wit={B,L}]{bhedā}}
\app{\lem[wit={ceteri}]{kathyante}
  \rdg[wit={E}]{kathyanta}
  \rdg[wit={N1,D}]{kathyate}}/ 
%-----------------------------
%pādayor aṃguṣṭhe  tejaso  lakṣyakāraṇāt              dṛṣṭiḥ sthirā bhavati/ \E
%pādayor aṃguṣṭhe  tejaso  lakṣyakaraṇāt              dṛṣṭiḥ sthirā bhavati  \P
%pādayor aṃguṣṭhai tejasaṃ lakṣaṃ kartavyaṃ kāraṇāt// dṛṣṭiḥ sthirā bhavati/ \B
%pādayor aṃguṣṭhe  tejasaṃ lakṣaṃ karttavyaṃ kāraṇāt  dṛṣṭiḥ sthirā bhavatī/ \L
%pādayor aṃguṣṭhe  tejaso  lakṣyakāraṇāt              dṛṣṭisthirā   bhavati/ \N1
%pādayor aṃguṣṭhe  tejaso  lakṣyakāraṇāt              dṛṣṭiḥ sthirā bhavati \D
%pādayor aṃguṣṭhe  tejaso  lakṣakāraṇāt               dṛṣṭisthirā   bhavati/ \N2
%pādayor aṃguṣṭhe  tejaso  lakṣyakāraṇāt              dṛṣṭisthirā   bhavati \U1
%pādayor aṃguṣṭhe  tejaso  lakṣyakāraṇāt              dṛṣṭisthirā   bhavati// \U2 %%%415.jpg
%-----------------------------
%From the execution of the fixation onto the light at the big toes of the feet stability of the gaze arises.
%-----------------------------
\note[type=source, labelb=212, lem={ādhāracakrasya}]{SSP: tatra prathamaḥ pādāṅguṣṭhādhāraḥ | tatrāgratas tejomayaṃ dhyāyet | dṛṣṭiḥ sthirā bhavati ||2.10|||}
\note[type=source, labelb=213, lem={ādhāracakrasya}]{Ysv (PT): aṅguṣṭhapādayos tejaḥ salakṣasthiradṛṣṭimān | pādāṅguṣṭhe ya ādhāraḥ prathamo [prathamaṃ (YK)] yogatattvataḥ | }
\app{\lem[type=conjecture, resp=egoscr]{prathamaḥ pādāṅguṣṭhādhāraḥ}
  \rdg[wit={ceteri}]{\conj \om}}\dd{}
\note[type=philcomm, labelb=214, lem={pādāṅguṣṭhādhāraḥ}]{Judging by the source and parallels as well as the introductory statements in the following \textit{ādhāra}s, as well as previous passages that must have been dropped in the text's transmission to me it seems more likely than not that originally the first \textit{ādhāra} was introduced, too.}
pādayo\skp{r-aṃ}\app{\lem[wit={ceteri}, alt={aṃguṣṭhe}]{\skm{r-aṃ}guṣṭhe}
  \rdg[wit={B}]{aṃguṣṭhai}}
\app{\lem[wit={ceteri}]{tejaso}
  \rdg[wit={B,L}]{tejasaṃ}}
\app{\lem[wit={ceteri}, alt={lakṣya°}]{lakṣya}
  \rdg[wit={N2}]{lakṣa°}
  \rdg[wit={B,L}]{lakṣaṃ kartavyaṃ}
}\app{\lem[wit={ceteri}, alt={°kāraṇāt}]{kāraṇāt}
  \rdg[wit={P}]{°karaṇāt}} 
 \app{\lem[wit={ceteri}]{dṛṣṭiḥ}
   \rdg[wit={N1,N2,U1,U2}]{dṛṣṭi°}}sthirā
 \app{\lem[wit={ceteri}]{bhavati}
   \rdg[wit={L}]{bhavatī}}/
%-----------------------------
%dvitīyo mūlādhāraḥ/  pādāṃguṣṭhasya mūle parapādasya  pārṣṇiḥ                                         sthāpyate tadāgniḥ prabalo bhavati/ \E
%dvitīyo mūlādhāraḥ   pādāṃguṣṭhasya mūle 'parapādasya dhāraḥ pādāṃduṣṭhasya mūleḥ paradādasya pārṣṇiḥ sthāpyate tadāgniḥ prabalo bhavati \P
%dvitīyo mūlādhāraḥ/  pādāṃguṣṭhasya mūle aparasya pādapārṣṇiḥ                                         syāpyate tadāgniḥ  prabalo bhavatī/ \B
%dvitīyo mūlādhāraḥ   pādāṃguṣṭhasya mūle aparasya pādapārṣṇīḥ                                         syāpyate tadāgniḥ  prabalo bhavatī/ \L
%dvitīyo mūlādhāraḥ/  pādāṃguṣṭhasya mūle aparapādasya pārṣṇiḥ                                         sthāpyate agniḥ    prabalo bhavati/   \N1
%dvitīyo mūlādhāraḥ// pādāṃguṣṭhasya mūle aparapādasya pārṣṇiḥ                                         sthāpyate agni-----prabalo bhavati//   \D  %%%p.12 recto
%dvitīyo mūlādhāraḥ   pādāṃguṣṭhasya mūle aparapādasya pārṣṇiḥ                                         sthāpyate/ \om                     \N2
%dvitīyo mūlādharaḥ   pādāṃguṣṭhasya mūle aparapādasya pārṣṇiḥ                                         sthāpyate agniṃ ---prabalo bhavati    \U1
%dvitīyo mūlādhare    pādāṃguṣṭhasya mūle 'parapādasya pārṣṇiḥ                                         sthāyyaṃte//                       \U2
%-----------------------------
%The root-container is the second [one]. The heel of the backfoot is caused to be placed at the root of the big toe. As a result the fire is strengthened. 
%-----------------------------
 \note[type=source, labelb=215, lem={mūlādhāraḥ}]{SSP: dvitīyo mūlādhāras taṃ vāmapādapārṣṇinā niṣpīḍya sthātavyam | tatrāgnidīpanaṃ bhavati ||2.11||}
%The second is the Mūlādhara which is to be pressend with the left heel. This enhances the bodily fire. 
 \note[type=source, labelb=216, lem={mūlādhāraḥ}]{Ysv (PT): dvitīyaṃ pādamūlan tu pādamūlaparaṃ [pādamūlaṃ paraṃ (YK)] sa vai | pādasya pārṣṇī [pārṣṇi (YK)] saṃsthāpya balavān prabhaven muniḥ | pādamūle 'thavā pādāṅguṣṭhamūlaṃ [pṛṣṭhe pādāṅguṣṭhe (YK)] vidhārayet ||}
%The second is the root of the foot. That root of the foot is truly superior. Having placed himself on the heel of the foot the Muni becomes powerful. He shall hold [the gaze?] at the root of the foot or at the back or at the big toe.
dvitīyo
\app{\lem[wit={ceteri}]{mūlādhāraḥ}
  \rdg[wit={U1}]{mūlādharaḥ}
  \rdg[wit={U2}]{mūlādhare}}\dd{}
pādāṃguṣṭhasya mūle
\app{\lem[wit={ceteri},alt={'para°}]{'para}
  \rdg[wit={N1,N2,D,U1}]{apara°}
  \rdg[wit={B,L}]{aparasya}
}\app{\lem[wit={ceteri}]{pādasya}
  \rdg[wit={B,L}]{pāda°}}
\app{\lem[wit={ceteri}]{pārṣṇiḥ}
  \rdg[wit={L}]{°pārṣṇīḥ}
  \rdg[wit={P}]{dhāraḥ pādāṃduṣṭhasya mūleḥ paradādasya pārṣṇiḥ}}
\app{\lem[wit={ceteri}]{sthāpyate}
  \rdg[wit={B,L}]{syāpyate}
  \rdg[wit={U2}]{sthāyyaṃte}}
\app{\lem[wit={ceteri}]{tadāgniḥ}
  \rdg[wit={N1}]{agniḥ}
  \rdg[wit={D}]{agni°}
  \rdg[wit={U2}]{\om}}
\app{\lem[wit={ceteri}]{prabalo}
  \rdg[wit={N2,U2}]{\om}}
\app{\lem[wit={ceteri}]{bhavati}
  \rdg[wit={B,L}]{bhavatī}
  \rdg[wit={N2,U2}]{\om}}/
%-----------------------------
%ekaḥ  pārṣṇir ādau  mūlādhāre  sthāpyate/     tasya pādasyāṃguṣṭhamūle      parasya  pādasya pārṣṇiḥ sthāpyate// tadagniḥ pradīpyate// \E [P.41]
%ekā   pārṣṇir ādau  mūlādhāre  sthāpyate      tasya pādasyāṃguṣṭhamūle     'parasya  pādasya pārṣṇiḥ sthāpyate   tadagnīḥ pradipyate \P
%ekā   pārṣṇir ādau  mūlādhāra  sthāpyate      tasya pādasyāṃguṣṭhamūle     aparasya  pādasya pārṣṇiḥ sthāpyate// tadagnīḥ pradipyate// \B
%ekā   pārṣṇir ādau  mūlādhārā  sthāpyate      tasya pādasyāṃguṣṭhamūle     aparasya  pādasya pārṣṇiḥ sthāpyate// tadāgnīḥ pradivyate// \L
%ekā   pārṣṇiḥ       mūladdhāre sthāpyate/     tasya pādasya aṃguṣṭhamūlaṃ/ aparasya  pādasya pārṣṇiḥ sthāpyaṃ agnir dāpyate?!/ \N1
%ekā   pārṣṇiḥ       mūlādhārai sthāpyate//    tasya pādasyāṃguṣṭhamūle//   aparasya  pādasya pārṣṇiḥ sthāpyaṃ// agnir dīpyate// \D
% \om -------------------------------------    tasya pādasyāṃguṣṭhamūle//   aparasya  pādasya pārṇisthāpyaṃ agni dīpate// \N2
%ekāṃ pārṣṇir mūlādhāra sthāpyate              tasya pādasya aṃguṣṭhamūlaṃ  aparasya          pārṣṇo sthāpyate agni dīpyate  \U1
% \om                                                                                                         tadagnīḥ pradipyate// \U2
%-----------------------------
%One heel is caused to be placed at the Root-container. The heel of the other foot is caused to be placed at the root of the big toe of this foot. The fire of it is caused to be kindled. 
%-----------------------------
\app{\lem[wit={ceteri}]{ekā}
  \rdg[wit={E}]{ekaḥ}
  \rdg[wit={U1}]{ekāṃ}}
\app{\lem[wit={U1},alt={pārṣṇiḥ}]{pārṣṇi\skp{r-mū}}
  \rdg[wit={N1,D}]{pārṣṇiḥ}
  \rdg[wit={B,E,L,P}]{pārṣṇir ādau}
  \rdg[wit={N2,U2}]{\om}
}\app{\lem[wit={ceteri},alt={mūlādhāre}]{\skm{r-mū}lādhāre}
  \rdg[wit={B,U1}]{mūlādhāra}
  \rdg[wit={L}]{mūlādhārā}
  \rdg[wit={D}]{mūlādhārai}
  \rdg[wit={N2,U2}]{\om}}
%-----------------------------
%tṛtīyaṃ gudādhārasthānaṃ   tanmadhye saṃkocavikāsākuṃcana--kāraṇāt pavanaḥ sthiro bhavati// \E
%tṛtīyaṃ gudādhārasthānaṃ   tanmadhye saṃkocavikāśākuṃcana--kāraṇāt pavanaḥ sthiro bhavati   \P
%tṛtīyaṃ gudādhārasthāne    tanmadhye saṃkocavikāśākuṃcana--kāraṇāt pavanaḥ sthiro bhavati// \B
%tṛtīyaṃ gudādhārasthānaṃ   tanmadhye saṃkocavikāśa ākuṃcanakāraṇāt pavanasthiro   bhavatī// \L
%tṛtīyaṃ gudādhārasthānaṃ   tanmadhye saṃkocavikāśākuṃcana--kāraṇāt pavanaḥ sthiro bhavati// \N1
%tṛtīyaṃ gudādhārasthānaṃ   tanmadhye saṃkocavikāśākuṃcanaṃ kāraṇāt pavanasthiro   bhavati// \D
%tṛtīyaṃ gudādhārasthānaṃ   taṃmadhye saṃkocavikāśākuṃcanaṃ kāraṇāt pavanasthiro   bhavati// \N2
%tṛtīyaṃ gudādhārasthānaṃ   taṃmadhye saṃkocavikāśā akuṃcanakāraṇāt pavanasthiro   bhavati \U1
%tṛtīya  gudādhārasthānaṃ// tanmadhye saṃkocavikāśākuṃcana--kāraṇāt pavanasthiro   bhavati// \U2
%-----------------------------
%The third is the place of the anus-container. From the execution of expansion and contraction a stable vitalwind arises.   
%-----------------------------
\note[type=source, labelb=217, lem={gudādhāra°}]{SSP: tṛtīyo gudādhāra taṃ vikāsasaṃkocanena nirākuñcayet | apānavāyuḥ sthiro bhavati ||2.12||}
\note[type=source, labelb=218, lem={gudādhāra°}]{Ysv (PT): tṛtīyantu gudādhāro [gudādhāre (YK)] gudasaṅkocanakriyā | vikāśākuñcanaṃ tasya sthiravāyau ca mṛtyujit |}
\app{\lem[wit={ceteri}]{tṛṭīyaṃ}
  \rdg[wit={U2}]{tṛtīya}}
gudādhāra\app{\lem[wit={ceteri},alt={°sthānaṃ}]{sthānaṃ}
  \rdg[wit={B}]{°sthāne}}\dd{}
tanmadhye
saṃkoca\app{\lem[wit={ceteri},alt={°vikāśā}]{vikāśā}
  \rdg[wit={L}]{°vikāśa°}
}\app{\lem[wit={ceteri},alt={°kuṃcana}]{kuṃcana}
  \rdg[wit={L}]{ākuṃcana}
  \rdg[wit={U1}]{akuṃcana}
  \rdg[wit={D,N2}]{kuṃcanaṃ}
}kāraṇāt-\app{\lem[wit={ceteri}]{pavanaḥ}
  \rdg[wit={D,U1,U2,N2}]{pavana°}}
sthiro
\app{\lem[wit={ceteri}]{bhavati}
  \rdg[wit={B}]{bhavatī}}/
%-----------------------------
%anyac ca/ puruṣasya maraṇaṃ na bhavati/ \E
%anu ca puruṣasya maraṇaṃ bhavati  \P
%anucarapuruṣasya maraṇaṃ bhavatī/ \B
%anucakrapuruṣasya maraṇaṃ bhavatī/ \L
%anū ca puruṣasya maraṇaṃ na bhavati ve?/ \N1
%anu ca puruṣasya maraṇaṃ na bhavati// \D
%anū ca puruṣasya maraṇaṃ na bhavati// \N2
%anu ca puruṣasya maraṇaṃ na bhavati  \U1
%anu ca puruṣasya maraṇaṃ na bhavati//  \U2
%-----------------------------
%Additionally death of the person does not arise. Additionally the person does not die.
%-----------------------------
\app{\lem[wit={D,P,U1,U2}]{anu ca}
  \rdg[wit={E}]{anyac ca}
  \rdg[wit={N1,N2}]{anūca}
  \rdg[wit={B}]{anucara°}
  \rdg[wit={L}]{anucakra°}}
puruṣasya maraṇaṃ
\app{\lem[wit={ceteri}]{na}
  \rdg[wit={B,P,L}]{\om}}
\app{\lem[wit={ceteri}]{bhavati}
  \rdg[wit={B,L}]{bhavatī}}/
%-----------------------------
%caturthaṃ liṃgādhāraṃ   tanmadhye/ liṃgasaṃkocanābhyāsāt  paścimadaṇḍamadhye prajñā nāḍī bhavati/  tanmadhye punar abhyāsakaraṇān manaḥ pavanayoḥ saṃcāro bhavati/ \E
%caturthaṃ liṃgādhāraṃ   tanmadhye  liṃgasaṃkocanābhyāsāt  paścīmadaṇḍamadhye vajñā nāḍī  bhavati   tanmadhye punar abhyāsakaraṇān manaḥ pavanayoḥ saṃcāro bhavati \P
%caturtha--liṃgādhāraṃ   tanmadhye  liṃgasaṃkocanābhyāsāt  paścīmadaṇḍamadhye vajñā nāḍī  bhavatī/  tanmadhye punar abhyāsakaraṇāt punaḥ pavanayo  saṃcāro bhavatī/     \B
%caturtha--liṃgādhāraṃ// tanmadhye  liṃgasaṃkocanābhyāsāt  paścamadaṇḍamadhye vajñā nāḍī  bhavatī// tanmadhye punar abhyāsakaraṇāt punaḥ pavanayo  saṃcāro bhavatī//     \L %%%%%%%%%%%20.jpg
%caturthaṃ liṃgādhāraṃ   tanmadhye/ liṃgasaṃkocanābhyāsāt/ paścimadaṇḍamadhye vajranāḍī   bhavati/  tanmadhye punaḥ abhyāsakaraṇāt manaḥpavanayoḥ saṃcāro bhavati/ \N1
%caturtha--liṃgādhāraṃ// tanmadhye/ liṃgasaṃkocanābhyāsāt//paścimadaṇḍamadhye vajrānāḍī   bhavati// tanmadhye punaḥ abhyāsakaraṇāt manaḥpavanayoḥ saṃcoro bhavati// \D
%caturthaṃ liṃgādhāraṃ   tanmadhye  liṃgasakoṇābhyāsāt//   paścimadaṇḍamadhye vajranāḍī   bhavati/  tanmadhye punar ābhyāsakaraṇāt manaḥpavanayoḥ saṃcāro bhavati// \N2
%caturthaṃ liṃgādhāraṃ   tanmadhye  liṃgasaṃkocanābhyāsāt  paścimadaṇḍamadhye vajranāḍī   bhavati   tanmadhye punar ābhyāsakaraṇāt manaḥpavanayoḥ saṃcāro bhavati    \U1    %%%283.jpg
%caturthaṃ liṃgādhāraṃ   tanmadhye  liṃgasaṃkocanābhyāsāt  paścimadaṇḍamadhye vajranāḍī   bhavati   tanmadhye punar ābhyāsakaraṇān manaḥpavanayoḥ saṃcāro bhavati//   \U2
%-----------------------------
%The fourth is the penis-container. Due to the execution of repeated practice of contracting the penis in the midst of therof, the adamantine channel appears in the middle of the staff of the back. From the repeated practice again [and again] the transition of both breath and mind into its center arises.  
%-----------------------------
\note[type=source, labelb=219, lem={liṃgādhāraṃ}]{SSP: caturtho meḍhrādhāraḥ | liṅgasaṃkocanena brahmagranthitrayaṃ bhitvā bhramaraguhāyāṃ viśramya tata ūrdhvamukhe bindustambhanaṃ bhavati| eṣā vajrolī prasiddhā ||2.13||}
\note[type=source, labelb=220, lem={liṃgādhāraṃ}]{Ysv (PT): liṅgādhāraṃ caturthan tu liṅgasaṅkocanan tu ca | liṅgasaṅkocanābhyāsāt paścimādaṇḍamadhyagaḥ | vajranāḍīti [vajrānāḍī tu (YK)] tanmadhye punar abhyasayaṃs [abhyasanan (YK)] tathā | sañcāro vāyumanasor atisañcāra iti [ratiṃ sañcarati (YK)] tridhā | granthitrayavibhedas [°bhedan (YK)] tu tadbhedo brahmamārgataḥ | brahmapadmo [°padme (YK)] vāyupūrṇo [°pūrṇe (YK)] bhūtvā tiṣṭhati yogirāṭ | vīryastambho bhavet tena sādhayet tu sadā yuvā | mūlādhāre brahmapadme ṣaṭpadme ca tathā tathā |}
\app{\lem[wit={ceteri}]{caturthaṃ}
  \rdg[wit={B,L,D}]{caturtha°}}
liṅgādhāraṃ \dd{}
tanmadhye
liṃga\app{\lem[wit={ceteri},alt={saṃkocanā°}]{saṃkocanā}
  \rdg[wit={N2}]{sakoṇā°}
}bhyāsāt
\app{\lem[wit={ceteri}, alt={paścima°}]{paścima}
  \rdg[wit={B,P}]{paścīma°}
  \rdg[wit={L}]{paścama°}
}daṇḍamadhye
\app{\lem[wit={ceteri}, alt={vajra°}]{vajra}
  \rdg[wit={B,P,L}]{vajñā}
  \rdg[wit={E}]{prajñā}
}nāḍī
\app{\lem[wit={ceteri}]{bhavati}
  \rdg[wit={B,L}]{bhavatī}}/
\end{prose}
\end{ekdosis}
\ekdpb*{}
%%%%%%%%%%%%%%%%%%%%%%%%%%%%%%%%%%%%%%%%%%
%%%%%%%%%%%%%%%%%%%%%%%%%%%%%%%%%%%%%%%%%%
%%%%%%%%PAGEBREAK%%%%%%%PAGEBREAK%%%%%%%%%
%%%%%%%%%%%%%%%%%%%%%%%%%%%%%%%%%%%%%%%%%%
%%%%%%%%%%%%%%%%PAGEBREAK%%%%%%%%%%%%%%%%%
%%%%%%%%%%%%%%%%%%%%%%%%%%%%%%%%%%%%%%%%%%
%%%%%%%%PAGEBREAK%%%%%%%PAGEBREAK%%%%%%%%%
%%%%%%%%%%%%%%%%%%%%%%%%%%%%%%%%%%%%%%%%%%
%%%%%%%%%%%%%%%%%%%%%%%%%%%%%%%%%%%%%%%%%%
%%%%%%%%%%%%%%%%%%%%%%%%%%%%%%%%%%%%%%%%%%
%%%%%%%%%%%%%%%%%%%%%%%%%%%%%%%%%%%%%%%%%%
%%%%%%%%PAGEBREAK%%%%%%%PAGEBREAK%%%%%%%%%
%%%%%%%%%%%%%%%%%%%%%%%%%%%%%%%%%%%%%%%%%%
%%%%%%%%%%%%%%%%PAGEBREAK%%%%%%%%%%%%%%%%%
%%%%%%%%%%%%%%%%%%%%%%%%%%%%%%%%%%%%%%%%%%
%%%%%%%%PAGEBREAK%%%%%%%PAGEBREAK%%%%%%%%%
%%%%%%%%%%%%%%%%%%%%%%%%%%%%%%%%%%%%%%%%%%
%%%%%%%%%%%%%%%%%%%%%%%%%%%%%%%%%%%%%%%%%%
%%%%%%%%%%%%%%%%%%%%%%%%%%%%%%%%%%%%%%%%%%
%%%%%%%%%%%%%%%%%%%%%%%%%%%%%%%%%%%%%%%%%%
%%%%%%%%PAGEBREAK%%%%%%%PAGEBREAK%%%%%%%%%
%%%%%%%%%%%%%%%%%%%%%%%%%%%%%%%%%%%%%%%%%%
%%%%%%%%%%%%%%%%PAGEBREAK%%%%%%%%%%%%%%%%%
%%%%%%%%%%%%%%%%%%%%%%%%%%%%%%%%%%%%%%%%%%
%%%%%%%%PAGEBREAK%%%%%%%PAGEBREAK%%%%%%%%%
%%%%%%%%%%%%%%%%%%%%%%%%%%%%%%%%%%%%%%%%%%
%%%%%%%%%%%%%%%%%%%%%%%%%%%%%%%%%%%%%%%%%%
\begin{ekdosis}
  \begin{prose}
    \noindent
tanmadhye punar-ābhyāsa\app{\lem[wit={E,P,U2}, alt={°karaṇān}]{karaṇā\skp{n-ma}}
  \rdg[wit={ceteri}]{karaṇāt}
}\app{\lem[wit={ceteri}, alt={manaḥ}]{\skm{n-ma}naḥ}
  \rdg[wit={B,L}]{punaḥ}}
\app{\lem[wit={ceteri}]{pavanayoḥ}
  \rdg[wit={B,L}]{pavanayo}}
\app{\lem[wit={ceteri}]{saṃcāro}
  \rdg[wit={D}]{saṃcoro}}
\app{\lem[wit={ceteri}]{bhavati}
  \rdg[wit={B,L}]{bhavatī}}/
%-----------------------------
%tayoḥ saṃcārān  madhye granthitrayaṃ truṭyati/  tatroṭanāt        pavano  brahmakamalamadhye pūrṇo bhūtvā tiṣṭhati/  tato vīryastambho bhavati/  puruṣaḥ sadaiva   yuvā      bhavati/ \E
%tayoḥ saṃcārān  madhye graṃthitrayaṃ truṭyati                                                                        tato vīryastaṃbho bhavati   puruṣaḥ saṃdaivaṃ yuve   prabhavati  \P
%tayo  saṃcārān  madhye granthitrayaṃ truṭyatī/  tatroṭanāt        pavano  brahmakamadhye     pūrṇā bhūtvā tiṣṭhati// tato vīryastambho bhavatī// puruṣaḥ sadaiva   yuvai     bhavatī/ \B
%tayoḥ saṃcārān  madhye graṃthitrayaṃ truṭayatī  tatroṭanāt        pavano  brahmakamadhye     pūrṇā bhūtvā tiṣṭhati// tato vīryastaṃbho bhavati   puruṣaḥ sadaiva   yuvaiva   bhavati// \L
%tayoḥ saṃcārān  madhye granthitrayaṃ truṭyati/  tattroṭanāt       pavanaḥ brahmakamalamadhye pūrṇo bhūtvā tiṣṭhati/  tato vīryastambho bhavati/  puruṣaḥ sadaiva   yuvā/e va bhavati// \N1 %truṭyati="zerbrechen"
%tayoḥ saṃcārāt  madhye graṃthitrayaṃ truṭyati// tata troṭanāt     pavanaḥ brahmakamalamadhye pūrṇo bhūtvā tiṣṭhati// tato vīryastambho bhavati// puruṣaḥ sadaiva   yuvaiva   bhavati// \D 
%tayoḥ saṃcārān  madhye granthitrayaṃ ... ..ti/  tata troṭanāt     pavanaḥ brahmakamalamadhye pūrṇo bhūtvā tiṣṭhati/  tato vīryastambho bhavati/  puruṣa  sadaiva   yurvaiva  bhavati// \N2
%tayoḥ saṃccārāt madhye graṃthitrayaṃ trudyati   tatroṭaṇāt        pavanaḥ brahmakamalamadhye pūrṇo bhūtvā tiṣṭhati   tato vīryastaṃbho bhavati/  puruṣaḥ sadaiva   yuvaivaṃ  bhavati \U1
%tayoḥ saṃccārān madhye graṃthitrayaṃ truṭyati// tattroṭaṇāt       pavanaḥ brahmakamalamadhye pūrṇo bhūtvā tiṣṭhati// tato vīryastaṃbho bhavati   puruṣaḥ sadaiva   vaibhavo  bhavati// \U2
%-----------------------------
%Caused by the transition of them both into the center the trinity of knots breaks. There from the breaking of that, the vitalwind after having filled up (the central channel?) resides in the center og the Brahma-lotus. Then virility and strength arises. The person becomes youthful forever. 
%-----------------------------
\app{\lem[wit={ceteri}]{tayoḥ}
  \rdg[wit={B}]{tayo}}
\app{\lem[wit={ceteri},alt={saṃcārān}]{saṃcārā\skp{n-ma}}
  \rdg[wit={D,U1}]{saṃcārāt}
}\skm{n-ma}dhye
granthitrayaṃ
\app{\lem[wit={ceteri}]{truṭyati}
  \rdg[wit={B}]{truṭyatī}
  \rdg[wit={L}]{truṭayatī}
  \rdg[wit={U1}]{trudyati}
  \rdg[wit={N2}]{ti}}/
\app{\lem[wit={N1,U2},alt={°tattroṭanāt}]{tattroṭanā\skp{t-pa}}
  \rdg[wit={B,E,L,U1}]{tatroṭanāt}
  \rdg[wit={D,N2}]{tata troṭanāt}
  \rdg[wit={P}]{\om}}
\app{\lem[wit={ceteri},alt={pavano}]{\skm{t-pa}vano}
  \rdg[wit={ceteri}]{pavanaḥ}}
brahma\app{\lem[wit={ceteri}, alt={°kamala°}]{kamala}
  \rdg[wit={B,L}]{°ka°}
  \rdg[wit={P}]{\om}
}madhye
\app{\lem[wit={ceteri}]{pūrṇo}
  \rdg[wit={B,L}]{pūrṇā}
  \rdg[wit={P}]{\om}}
bhūtvā tiṣṭhati/
tato vīryastambho bhavati/
\app{\lem[wit={ceteri}]{puruṣaḥ}
  \rdg[wit={N2}]{puruṣa}}
\app{\lem[wit={ceteri}]{sadaiva}
  \rdg[wit={P}]{saṃdaivaṃ}}
\app{\lem[wit={D,L}]{yuvaiva}
  \rdg[wit={E}]{yuvā}
  \rdg[wit={P}]{yuve}
  \rdg[wit={B}]{yuvai}
  \rdg[wit={N1}]{yuve va}
  \rdg[wit={N2}]{yurvaiva}
  \rdg[wit={U1}]{yuvaivaṃ}
  \rdg[wit={U2}]{yuvaivaṃ}}
\app{\lem[wit={ceteri}]{bhavati}
  \rdg[wit={B}]{bhavatī}
  \rdg[wit={P}]{prabhavati}}/
%-----------------------------
%paṃcama  udgīryāṇāṃ svādhiṣṭhānaṃ tatra bandhanān      malamūtrayor nāśo   bhavati/  \E
%paṃcamaṃ uḍḍīyāṇāṃ  svādhiṣṭhānaṃ tatra baṃdhadānān    malamūtrayor nāśo   bhavati   \P
%paṃcama  uḍḍiyānāṃ  svādhiṣṭhānaṃ tatra baṃdha dīyate/ malamūtrayor nāśo   bhavatī// \B
%paṃcamaṃ uḍḍiyānāṃ  svādhiṣṭhānaṃ tatra baṃdha dīyate/ mūlamūcayor  nāśo   bhavati// \L 
%paṃcamaṃ udyānaṃ                  tatra baṃdhanāt      malamūtrayor nāśe/o bhavati// \N1 [s.10, verso, z4]
%paṃcamaṃ udyāṇāṃ                  tatra vaṃdhanāt      malamūtrayor nāśo   bhavati// \D
%paṃcam   odyānaṃ                  tatra baṃdhanāt      malamūtrayor nāśo   bhavati/  \N2
%paṃcamaṃ uddyānaṃ                 tatra baṃdhadānāt    malamūtrayor nāśo   bhavati   \U1
%paṃcamaṃ uḍḍīyāṇaṃ  svādhiṣṭhānaṃ tatra badhadānān     malamūtrayor nāśo   bhavati// \U2
%-----------------------------
%The fifth is Udyāna. From performing \textit{bandha} there, urine and faeces disappear.  
%-----------------------------
\note[type=source, labelb=221, lem={udyānaṃ°}]{SSP: pañcame oḍīyāṇādhārayor bandhanān malamūtrasaṃkocanaṃ bhavati ||2.14|| *uḍyānā° etc. in various mss.}
\note[type=source, labelb=222, lem={udyānaṃ°}]{Ysv (PT): pañcamaṃ jaṭharādhāraṃ tadā bandhayati kramāt | mṛtyunā bhaṅgasiddho 'yaṃ [mṛtyunāmāṅgasiddho 'yaṃ (YK)] mṛtyor [mṛtyur (YK)] eva kṣayaṅkaraḥ | anena paścimād ūrddhaṃ [mṛtyunāmāṅgasiddho'yaṃ (YK)] vāyuḥ kuryād viśāladhīḥ | bandho 'yaṃ buddhimanasoḥ pañcamādhārakālajit |}
\app{\lem[wit={ceteri}]{paṃcamaṃ}
  \rdg[wit={B}]{paṃcama}
  \rdg[wit={N2}]{paṃcam}}
\app{\lem[wit={N1,D}]{udyānaṃ}
  \rdg[wit={N2}]{odyānaṃ}
  \rdg[wit={U1}]{uddyānaṃ}
  \rdg[wit={P,U2}]{uḍḍīyāṇāṃ svādhiṣṭhānaṃ}
  \rdg[wit={B,L}]{uḍḍiyānāṃ svādhiṣṭhānaṃ}
  \rdg[wit={E}]{udgīryāṇāṃ svādhiṣṭhānaṃ}} \dd{}
\note[type=philcomm, labelb=223, lem={udyānaṃ}]{Spellings for this component of the yogic body vary dramatically across yogic literature. Since this sentence is clearly based on the SSP and the prevelant variant of the component is *\textit{uḍyānā}° etc., the reading of N\textsubscript{1} seems to be original. B,E,L,P,U\textsubscript{2} add the expression \textit{svādhiṣṭhānaṃ}. Since this reading is absent in the source and parallels it seems to be a later addition.}
tatra
\app{\lem[wit={E}]{bandhanā\skp{n-ma}}
  \rdg[wit={U2}]{badhadānān}
  \rdg[wit={N1,N2}]{baṃdhanāt}
  \rdg[wit={D}]{vaṃdhanāt}
  \rdg[wit={U1}]{baṃdhadānāt}
  \rdg[wit={P}]{baṃdhadānān}
  \rdg[wit={B,L}]{baṃdha dīyate}
}\app{\lem[wit={ceteri},alt={malamūtrayor}]{\skm{n-ma}lamūtrayo\skp{r-nā}}
  \rdg[wit={L}]{mūlamūcayor}}
\skm{r-nā}śo
\app{\lem[wit={ceteri}]{bhavati}
  \rdg[wit={B}]{bhavatī}}/
%-----------------------------
%ṣaṣṭho nābhyādhāraḥ/    \E
%ṣaṣṭho nābhyādhāraḥ   tatra         prāṇavābhyāsād  nāhato   nāraḥ   svayam utpadyate / \P
%ṣaṣṭho nābhyādhāraḥ   tatra         prāṇavābhyāsād  anāhato  nādaḥ// svayam utpadyate// \B
%ṣaṣṭho nābhyādhāraḥ   tatra         prāṇavābhyāsād  anāhato  nādaḥ// svayam utpadyate... \L 
%ṣaṣṭho nābhyādhāraḥ/  tatra         praṇavābhyāsāt  anāhato  nādaḥ   svayam ūtpadyate/  \N1
%ṣaṣṭho nābhyādhāraḥ// tatra         prāṇavābhyāsāt  anāhato  nādaḥ// svayam utpadyate// \D
%ṣaṣṭho nābhyādhāraḥ   tatra         praṇavābhyāsāt  anāhato  tādaḥ   svayaṃ utpadyate/ \N2
%ṣaṣṭho nābhyādhāras   tatra         praṇavābhyāṃsad ānāhato  nadaḥ   svayam utpadyate   \U1
%ṣaṣṭho nābhyādhāre//  tatra         prāṇavābhyāsād  anohato  nādaḥ   svayam utpadyate// \U2
%-----------------------------
%The sixth is the navel-container. From repeated practice of \textit{praṇava}, the unstruck sound arises by itself. 
%-----------------------------
\note[type=source, labelb=224, lem={nābhyādhāraḥ}]{SSP: ṣaṣṭhe nābhyādhāra oṃkāram ekacittenoccārayet | nādalayo bhavati ||2.15||}
\note[type=source, labelb=225, lem={nābhyādhāraḥ}]{Ysv (PT): nābhyādhāro bhavet ṣaṣṭhas [ṣaṣṭhaṃ (YK)] tatra prāṇaṃ samabhyaset | svayam utpadyate nādo nādato muktidantataḥ [muktidaṇḍataḥ (YK)]|}
ṣaṣṭho
\app{\lem[wit={ceteri}]{nābhyādhāraḥ}
  \rdg[wit={U1}]{nābhyādhāras}
  \rdg[wit={U2}]{nābhyādhāre}}\dd{}
\app{\lem[wit={ceteri}]{tatra}
  \rdg[wit={E}]{\om}}
\app{\lem[wit={P,B,L,U2}]{prāṇavābhyāsā\skp{d-a}}
  \rdg[wit={P,B,L,U2}]{prāṇavābhyāsād}
  \rdg[wit={U1}]{prāṇavābhyāṃsad}
  \rdg[wit={E}]{\om}
}\app{\lem[wit={ceteri},alt={°anāhato}]{\skm{d-a}nāhato}
    \rdg[wit={P}]{nāhato}
    \rdg[wit={U1}]{ānāhato}
    \rdg[wit={U2}]{anohato}}
  \app{\lem[wit={ceteri}]{nādaḥ}
    \rdg[wit={P}]{nāraḥ}
    \rdg[wit={N2}]{tādaḥ}}
  \app{\lem[wit={ceteri}]{svaya\skp{m-u}}
    \rdg[wit={N2}]{svayaṃ}
}\app{\lem[wit={ceteri},alt={utpadyate}]{\skm{m-u}tpadyate}
  \rdg[wit={N1}]{ūtpadyate}}/
%-----------------------------
%                             tasmin sthāne prāṇavāyor  nirodhāt            ṣaḍapi kamalāny ūrdhvamukhāni             vikasaṃti// \E                                                                        \E
%saptamo hṛdayarūpadhāraḥ     tasmin sthāne prāṇavāyor  nirodhāt            ṣadapi kamalāny ūrdhvamukhāni             vikasaṃti  \P  %%%7653.jpg 
%                             tasmin sthāne prāṇavāyo   nirodhāt/           ṣaḍapi kamalāny ūrdhvamukhāni             vikasaṃti// \B
%saptamo hṛdayarūpadhāraḥ//   tasmin sthāne prāṇavāyor  nirodhāt            ṣadapi kamalāny ūrdhvamukhāni             vikasaṃti// \L
%saptamo hṛdayarūpa ādhāraḥ   tasmin sthāne prāṇavāyor  nirūṃdhanāt/        ṣadapi kamalāny ūrdhvamukhaṃ              vikasaṃti// \N1
%saptamo hṛdayarūpa ādhāraḥ// tasmin sthāne prāṇavāyor  nir???ūṃ???dhanāt// ṣadapi kamalāny ūrdhvamukhaṃ              vikasaṃti// \D
%saptamo hṛdayarūpādhāraḥ     tasmin sthāne prāṇavāyor  nirūṃdhanāt/        ṣadapi kamalāny ūrdhvemukhaṃ              vikasaṃti// \N2 %%%%%%%%%[S.9, recto, z.4]
%saptamo hṛdayarūpādhāraḥ     tasmin sthāne prāṇavāyor  nirūṃdhanāt         ṣadapi kamalāny ūrusyordha mukhaṃ bhavati vikasaṃti  \U1
%saptamo hṛdayādhāraḥ         tasmin sthāne prāṇavāyor  nirodhāt//          ṣadapi kamalāny ūrddhvamukhāni            vikasaṃti//  \U2
%-----------------------------
%The seventh is the container of the heart-form. 
%-----------------------------
\note[type=source, labelb=226, lem={hṛdayarūpadhāraḥ}]{SSP: saptame hṛdayādhāre prāṇaṃ nirodhayet | kamalavikāso bhavati ||2.16||}
\note[type=source, labelb=227, lem={hṛdayarūpadhāraḥ}]{Ysv (YK): saptamo hṛdayādhāras tasmin vāyunibandhanāt | ūrdhvavaktrāṇi [ūrddhaktrāṇi (YK)] padmāni vikasanti mahān bhavet ||26||}
\app{\lem[wit={ceteri}]{saptamo}
  \rdg[wit={E,B}]{\om}}
\app{\lem[wit={ceteri}]{hṛdaya}
  \rdg[wit={U2}]{hṛdayā°}
}\app{\lem[wit={N2,U1},alt={°rūpādhāraḥ}]{rūpādhāraḥ}
  \rdg[wit={L}]{°rūpadhāraḥ}
  \rdg[wit={D,N1}]{rūpa ādhāraḥ}
  \rdg[wit={U2}]{°dhāraḥ}
  \rdg[wit={E,P}]{\om}}\dd{}
tasmin-sthāne
\app{\lem[wit={ceteri}]{prāṇavāyo\skp{r-ni}}
  \rdg[wit={B}]{prāṇavāyo}
}\app{\lem[wit={ceteri},alt={nirodhāt}]{skm{r-ni}rodhā\skp{t-ṣa}}
  \rdg[wit={D,N1,N2,U1}]{nirūṃdhanāt}
}\app{\lem[wit={ceteri},alt={ṣad api}]{\skm{t-ṣa}dapi}
  \rdg[wit={B}]{ṣaḍapi}}   
kamalā\skp{ny-ū}\app{\lem[wit={ceteri},alt={ūrdhvamukhāni}]{\skm{ny-ū}rdhvamukhāni}
  \rdg[wit={D,N1,N2}]{ūrdhvamukhaṃ}
  \rdg[wit={U1}]{ūrusyordha mukhaṃ bhavati}}
vikasaṃti/
%-----------------------------
%aṣṭamaṃ kaṇṭhādhāraḥ/  tatra  jālaṃdharo bandho dīyate/  tasmin satīḍāyāṃ   piṃgalāyāṃ pavanaḥ sthiro bhavati/  \E %%[p.43]
%aṣṭamaḥ kaṃṭhādhāraḥ   tatra  jālaṃdharo baṃdho dīyate   tasmin satīḍāyāṃ   piṃgalāyāṃ pavanaḥ sthiro bhavataḥ  \P
%aṣṭame  kaṇṭhādhāraḥ/  tatra  jalaṃ baṃdho      dīyate   tasmin satīyāṃ     piṃgalāyāṃ pavanaḥ sthiro bhavatī/ \B  %%%%DSCN7166.jpg Z.3
%aṣṭame  kaṇṭhādhāraḥ/  tatra  jalaṃ baṃdho      dīyate   tasmin satīyāṃ     piṃgalāyāṃ pavanaḥ sthiro bhavatī// \L
%aṣṭamaḥ kaṇṭhādhāraḥ/  tatra  jālaṃdharo baṃdho dīyate/  tasmin sati iḍāyāṃ piṃgalāyāṃ pavanaḥ sthiro bhavati/ \N1
%aṣṭamaḥ kaṃṭhādhāraḥ// tatraḥ jālaṃdharo baṃdho dīyate// tasmin sati iḍāyāṃ piṃgalāyāṃ pavanasthiro bhavati// \D  %%%p.12 verso
%aṣṭama--kaṇṭhādhāraḥ/  tatra  jālaṃdharabandho  dīyate// tasmin satiśadāyāṃ piṃgalāyāṃ pavanaḥ sthiro bhavati/ \N2
%aṣṭamaḥ kaṇṭhādhāraḥ   tatra  jālaṃdharo bandho dīpyate  tasmin sati iḍāyāṃ piṃgalāyāṃ pavanaḥ sthiro bhavati \U1
%aṣṭamaḥ kaṇṭhādhāraḥ   tatra  jālaṃdharo bandho dīyate   tasmin sati piḍāyā piṃgalāyāṃ pavanaḥ sthiro bhavati// \U2
%-----------------------------
%The throat-support is the eighth. There the contraction of Jālaṃdhara is produced. While abiding therein the vitalwind in the Iḍā and Piṅgalā channel becomes stable.   
%-----------------------------
\note[type=source, labelb=228, lem={kaṇṭhādhāraḥ}]{SSP: aṣṭame kaṇṭhādhāre kaṇṭhamūlaṃ cibukena nirodhayet | iḍāpiṅgalayor vāyuḥ sthiro bhavati ||2.17||}
\note[type=source, labelb=229, lem={kaṇṭhādhāraḥ}]{Ysv (PT=YK):kaṇṭhādhāro 'ṣṭamas tatra kaṇṭhasaṅkocalakṣaṇaḥ | jālandharākhyo bandhaḥ syāt tasmin sati marud dṛḍhaḥ ||27||}
\app{\lem[wit={P,N1,D,U1,U2}]{aṣṭamaḥ}
  \rdg[wit={B,L}]{aṣṭame}
  \rdg[wit={N2}]{aṣṭama°}}
kaṇṭhādhāraḥ/
\app{\lem[wit={ceteri}]{tatra}
  \rdg[wit={D}]{tatraḥ}}
\app{\lem[wit={ceteri}]{jālaṃdharo}
  \rdg[wit={N2}]{jālaṃdhara°}
  \rdg[wit={B,L}]{jalaṃ}}
bandho
\app{\lem[wit={ceteri}]{dīyate}
  \rdg[wit={U1}]{dīpyate}}/
tasmin \app{\lem[wit={E,P}]{satīḍāyāṃ}
  \rdg[wit={B,L}]{satīyāṃ}
  \rdg[wit={N1,D,U1,U2}]{sati iḍāyāṃ}
  \rdg[wit={N2}]{satiśadāyāṃ}}
piṅgalāyāṃ
\app{\lem[wit={ceteri}]{pavanaḥ}
  \rdg[wit={D}]{pavana°}}
sthiro
\app{\lem[wit={ceteri}]{bhavati}
  \rdg[wit={B,L}]{bhavatī}}/
\end{prose}
\end{ekdosis}
\ekdpb*{}
%%%%%%%%%%%%%%%%%%%%%%%%%%%%%%%%%%%%%%%%%%
%%%%%%%%%%%%%%%%%%%%%%%%%%%%%%%%%%%%%%%%%%
%%%%%%%%PAGEBREAK%%%%%%%PAGEBREAK%%%%%%%%%
%%%%%%%%%%%%%%%%%%%%%%%%%%%%%%%%%%%%%%%%%%
%%%%%%%%%%%%%%%%PAGEBREAK%%%%%%%%%%%%%%%%%
%%%%%%%%%%%%%%%%%%%%%%%%%%%%%%%%%%%%%%%%%%
%%%%%%%%PAGEBREAK%%%%%%%PAGEBREAK%%%%%%%%%
%%%%%%%%%%%%%%%%%%%%%%%%%%%%%%%%%%%%%%%%%%
%%%%%%%%%%%%%%%%%%%%%%%%%%%%%%%%%%%%%%%%%%
%%%%%%%%%%%%%%%%%%%%%%%%%%%%%%%%%%%%%%%%%%
%%%%%%%%%%%%%%%%%%%%%%%%%%%%%%%%%%%%%%%%%%
%%%%%%%%PAGEBREAK%%%%%%%PAGEBREAK%%%%%%%%%
%%%%%%%%%%%%%%%%%%%%%%%%%%%%%%%%%%%%%%%%%%
%%%%%%%%%%%%%%%%PAGEBREAK%%%%%%%%%%%%%%%%%
%%%%%%%%%%%%%%%%%%%%%%%%%%%%%%%%%%%%%%%%%%
%%%%%%%%PAGEBREAK%%%%%%%PAGEBREAK%%%%%%%%%
%%%%%%%%%%%%%%%%%%%%%%%%%%%%%%%%%%%%%%%%%%
%%%%%%%%%%%%%%%%%%%%%%%%%%%%%%%%%%%%%%%%%%
%%%%%%%%%%%%%%%%%%%%%%%%%%%%%%%%%%%%%%%%%%
%%%%%%%%%%%%%%%%%%%%%%%%%%%%%%%%%%%%%%%%%%
%%%%%%%%PAGEBREAK%%%%%%%PAGEBREAK%%%%%%%%%
%%%%%%%%%%%%%%%%%%%%%%%%%%%%%%%%%%%%%%%%%%
%%%%%%%%%%%%%%%%PAGEBREAK%%%%%%%%%%%%%%%%%
%%%%%%%%%%%%%%%%%%%%%%%%%%%%%%%%%%%%%%%%%%
%%%%%%%%PAGEBREAK%%%%%%%PAGEBREAK%%%%%%%%%
%%%%%%%%%%%%%%%%%%%%%%%%%%%%%%%%%%%%%%%%%%
%%%%%%%%%%%%%%%%%%%%%%%%%%%%%%%%%%%%%%%%%%
\begin{ekdosis}
  \begin{prose}
    \noindent
%-----------------------------
%navamo ghaṃṭikādhāraḥ/   tatra jihvāgraṃ   lagnaṃ bhavati/    tato mṛtakalāyā     amṛtaṃ sravati/  tadamṛtapānāt             śarīramadhye rogasaṃcāro na bhavati/ \E
%navamo ghaṭikādhāraḥ     tatra jihvāgraṃ   lagnaṃ bhavati     tato mṛtakakalāyā   amṛta  sravati   tadamṛtapānāc            charīramadhye rogasaṃcāro na bhavati  \P
%navo   ghaṃṭikādhāraḥ//  tatra jihvāgraṃ   lagnaṃ bhavatī/    tato mṛtakalāyā     amṛtaṃ sravati/  tadamṛtakalāyāṃ amṛtapānīcharīramadhye rogasaṃcāro bhavatī/ \B
%navamo ghaṃṭādhāraḥ//    tatra jihvāgraṃ   lagnaṃ bhavati//   tato mṛtakalāyāṃ                        amṛtapānā-------------charīramadhye rogasaṃcāro bhavati// \L %eyeskip in line.. :(
%navamo ghaṃṭikādhāraḥ/   tatra jihvāgraṃ   lagnaṃ bhavati/    tato mṛtakalāyā     amṛtaṃ sravati/  tadamṛtapānāt             śarīramadhye rogasaṃcāro na bhavati/ \N1
%navamo ghaṃṭikādhāraḥ//  tatra jihvāyāgraṃ lagnaṃ bhavati//   tataḥ amṛtakalāyāḥ  amṛtaṃ sravati// tadamṛtapānāc           charīramadhye  rogasaṃcāro na bhavati// \D
%navamo ghaṃṭikādhāraḥ/   tatra jihvāgraṃ   lagnaṃ bhavati/    tato mṛtakalāyā     amṛtaṃ sravati/  tadamṛtapānāt             śarīramadhye rogasaṃcāro na bhavati/ \N2
%navamo ghaṃṭikādhāras    tatra juhvāyāṃ    lagnaṃ bhavati vā  tataḥ amṛtakalāyāḥ  amṛtaṃ sravati   tadamṛtapānāt            charīramadhye rogasaṃcāro na bhavati \U1
%navamo ghaṃṭikādhāraḥ    tatra jihvāgraṃ   lagnaṃ bhavati//   tato mṛtakalāyāḥ    amṛtaṃ sravati// tadamṛtapānā             charīramadhye rogasaṃcāro na bhavati// \U2
%-----------------------------
%The ninth is the container of the uvula. There the tip of the tongue becomes attached [to the uvula]. Then the nectar of immortality flows from the immortality-digit. From drinking the nectar of immortality diseases do not spread in the body. 
%-----------------------------
\note[type=source, labelb=230, lem={ghaṃṭikādhāraḥ}]{SSP: navame ghaṇṭikādhāre jihvāgraṃ dhārayet | amṛtakalā sravati ||2.18||}
\note[type=source, labelb=231, lem={ghaṃṭikādhāraḥ}]{Ysv (PT): navamo ghaṇṭikādhāras tatra jihvāgramagrataḥ [jihvāgrataḥ kṛte (YK)] | sampivatyamṛtaṃ tasmād yogajinmṛtyujitparaḥ |}
\app{\lem[wit={ceteri}]{navamo}
  \rdg[wit={B}]{navo}}
\app{\lem[wit={ceteri},alt={ghaṃṭikā°}]{ghaṃṭikā}
  \rdg[wit={P}]{ghaṭikā°}
  \rdg[wit={L}]{ghaṃṭā°}
}\app{\lem[wit={ceteri},alt={°dhāraḥ}]{dhāraḥ}
  \rdg[wit={U1}]{dhāras}}/
tatra
\app{\lem[wit={ceteri}]{jihvāgraṃ}
  \rdg[wit={D}]{jihvāyāgraṃ}
  \rdg[wit={U1}]{juhvāyāṃ}}
lagnaṃ
\app{\lem[wit={ceteri}]{bhavati}
  \rdg[wit={B}]{bhavatī}
  \rdg[wit={U1}]{bhavati vā}}
\app{\lem[wit={ceteri}]{tato}
  \rdg[wit={N1,U1}]{tataḥ}}
\app{\lem[wit={E,B,N1,N2}]{'mṛtakalāyā}
  \rdg[wit={P}]{mṛtakakalāyā}
  \rdg[wit={L}]{mṛtakalāyāṃ}
  \rdg[wit={D,U1}]{amṛtakalāyāḥ}}
\app{\lem[wit={ceteri}]{amṛtaṃ}
  \rdg[wit={P}]{amṛta}
  \rdg[wit={L}]{\om}}
\app{\lem[wit={ceteri}]{sravati}
  \rdg[wit={L}]{\om}}/
\app{\lem[wit={P,D},alt={tadamṛtapānāc}]{tadamṛtapānā\skp{c-cha}}
  \rdg[wit={E,N1,N2,U1}]{tadamṛtapānāt}
  \rdg[wit={B}]{tadamṛtakalāyāṃ amṛtapānī°}
  \rdg[wit={L}]{amṛtapānā}
  \rdg[wit={U2}]{tadamṛtapānā}
}\app{\lem[wit={ceteri},alt={charīra°}]{\skm{c-cha}rīra}
  \rdg[wit={E,N1,N2}]{śarīra°}
}madhye 
rogasaṃcāro
\app{\lem[wit={ceteri}]{na}
  \rdg[wit={B,L}]{\om}}
\app{\lem[wit={ceteri}]{bhavati}
  \rdg[wit={B}]{bhavatī}}/  
%-----------------------------
%daśamaṃ tālvādhāraḥ/  tanmadhye    vānaṃ dollahanaṃ      ca kṛtvā              laṃbikāpraveśe sati    tāluni magnā jihvā tiṣṭhati/ \E
%daśamas tālvādhāraḥ   tanmadhye  cālanaṃ dohanaṃ         ca kratvā             laṃbikāpraveśe śe sati tālumagnā    jihvā tiṣṭhati  \P %%%7654.jpg
%daśamaṃ stālvādhāraḥ/ tanmadhye  cālanaṃ dohanaṃ         ca kratvā             laṃbikāpraveśe sati    tālumagnā    jihvā tiṣṭhati/ \B
%daśamas tālvādhāraḥ// tanmadhye  cālanaṃ dohanaṃ         ca kṛtvā              laṃbikāpraveśe sati    tālumagnā    jihvā tiṣṭhati ... \L
%daśama  tālvādhāraḥ// tanmadhye  cānanaṃ dohanaṃ         ca kṛtvā              laṃbikāpraveśe grati   tāluni magnā jihvā tiṣṭhati/ \N1
%daśamas tālvādhāraḥ   tanmadhye  cānanaṃ dohanaṃ         ca kṛtvā              laṃbikāpravese grati   tāluni magnā jihvā tiṣṭhati// \D
%daśama  tālvādhāraḥ   tanmadhye  cālanaṃ dohanaṃ         ca kṛtvā              laṃbikāpraveśe grati   tālūni magnā                    \N2
%daśamas tālvādhāraḥ  staṃnmadhye cālanaṃ dohanaṃ         ca sva/sca? kṛtvā cālaṃ vikā praveśe sati    tālūni lagnā juhvā tiṣṭhati \U1 %%%284.jpg
%daśamas tālvādhāraḥ   tanmadhye  cālanaṃ dohanaṃ chedanaṃ ca kṛtvā             laṃbikāpraveśe sati    tāluni magnā jihvā tiṣṭhati// \U2 %%416.jpg
%-----------------------------
%The tenth is the container of the palate. After the moving and milking has been done therein while abiding at the door of the uvula, the tongue resides inserted within the palate.  
%-----------------------------
\note[type=source, labelb=232, lem={tālvādhāraḥ}]{SSP: daśame tālvādhāre tālvantar garbhe lambikāṃ cālanadohanābhyāṃ dīrghīkṛtvā viparītena praveśayet | kāṣṭhībhavati ||2.19 ||}
\note[type=source, labelb=233, lem={tālvādhāraḥ}]{Ysv (PT): daśamas tālukādhāras tatra jihvāgrataḥ kṛte | calane dohane caiva jihvā jaḍati lambitā | nāsikāprāptajihveyaṃ tālulagnā bhavet tataḥ [jāyeta lambitam (YK)] |}
\app{\lem[wit={ceteri},alt={daśamas}]{daśama\skp{s-tā}}
  \rdg[wit={E}]{daśamaṃ}
  \rdg[wit={B}]{daśamaṃs}
  \rdg[wit={N1,N2}]{daśama}
}\skm{s-tā}lvādhāraḥ/
\app{\lem[wit={ceteri}]{tanmadhye}
  \rdg[wit={U1}]{staṃnmadhye}}
\app{\lem[wit={ceteri}]{cālanaṃ}
  \rdg[wit={D}]{cānanaṃ}
  \rdg[wit={E}]{vānaṃ}}
\app{\lem[wit={ceteri}]{dohanaṃ}
  \rdg[wit={E}]{dollahanaṃ}
  \rdg[wit={U2}]{dohanaṃ chedanaṃ}}
ca \app{\lem[wit={ceteri}]{kṛtvā}
  \rdg[wit={B,L}]{kratvā}
  \rdg[wit={U1}]{sva kṛtvā}}
\app{\lem[wit={ceteri}]{laṃbikā}
  \rdg[wit={U1}]{cālaṃ vikā}
}praveśe
\app{\lem[wit={ceteri}]{sati}
  \rdg[wit={P}]{śe sati}
  \rdg[wit={D,N1,N2}]{grati}}
\app{\lem[wit={ceteri}]{tālunimagnā}
  \rdg[wit={N2,U1,U2}]{tālūnimagnā}
  \rdg[wit={B,P,L}]{tālumagnā}}
\app{\lem[wit={ceteri}]{jihvā}
  \rdg[wit={U1}]{juhvā}
  \rdg[wit={N2}]{\om}}
\app{\lem[wit={ceteri}]{tiṣṭhati}
  \rdg[wit={N2}]{\om}}/
%-----------------------------
%ekādaśo           jihvādhāraḥ/  tasmin   jihvāgreṇa manthanaṃ kriyate   tasmin  kṛte   timadhuraṃ  pānīyaṃ sravati/  tadā                            ca kavitva------cchandonāṭakādiviṣayajñānam utpadyate/ \E
%ekādaśo jihvātale jihvādhāraḥ   tasmin   jihvāgreṇa mathanaṃ  kriyate   tasmin  kṛte   timadhuraṃ  pānīyaṃ sravati   tathā                           ca kavitva------chaṃdonāṭakādiviṣayajñānam  utpadyate  \P
%ekādaśo jihvātale jihvādhāraḥ// tasmin   jihvāgreṇa manthanaṃ kṛtvā//   tasmiṃ  kṛte satimadhuraṃ  pānīyaṃ sravatī// tathā                              kvacitva-----cchaṃdonāṭakādiviṣayapānam  utpadyaṃte/ \B
%ekādaśo jihvātale jihvādhāraḥ// tasmin   jihvāgreṇa mathanaṃ  kṛtvā//   tasmiṃ  kṛte satimadhuraṃ  pānīyaṃ sravati// tathā                              kvacitva-----chaṃdonāṭakādiviṣayajñānam  utpadyate// \L
%ekādaśo           jihvādhāraḥ/  tasmin   jihvāgreṇa manthanaṃ kriyate/  tasmin  kṛte atimadhuraṃ   pānīyaṃ sravati/  tathā                           ca kavitva--gītacchaṃdanāṭakādiviṣaye jñānam utpadyate/ \N1
%ekādaśo jihvātale jihvādhāraḥ// tasmin   jihvāgreṇa mathanaṃ  kriyate// tasmiṃ  kṛte satimadhuraṃ  pānīyaṃ sravati// tathā                           ca kvacitta-----chaṃdanāṭakādiviṣayajñānam   utpadyate \D 
%                                         jihvāgreṇa manthanaṃ kriyate// tasmin  kṛte atimadhuraṃ   pānīyaṃ sravati// kaminnāsikā phatkāravat// tathā ca kavitvagīta--chaṃdanāṭakādiviṣaye jñānam  utpadyate/ \N2
%ekādaśā jihvātale jihvādhāraḥ  tasmin na jihvāgreṇa manthanaṃ kriyate   tasminn kṛte  timadhuraṃ   pānīyaṃ sravati   tathā                           ca kavitvagīta--chaṃdavacchaṃdanāḍīviṣayaṃ jñānānam utpadyate \U1
%ekādaśo jihvātale jihvādhāraḥ   tasmin   jihvāgreṇa manthanaṃ kriyate// tasminn kṛte 'timadhuraṃ   pānīyaṃ sravati// tathā                           ca kavitvaṃ     chaṃdonāṭakādiviṣayajñānam  utpadyate// \U2
%-----------------------------
%The eleventh is the tongue-container at the surface of the tontue. Within it the tip of the tongue has to be churned. While doing a sweet drink flows out. And in that manner the knowledge of areas like poetry, singing, metric and dance is generated. 
%----------------------------
\note[type=source, labelb=234, lem={jihvādhāraḥ}]{SSP: ekādaśe atha jihvādhāre tatra jihvāgraṃ dhārayet | sarvaroganāśo bhavati ||2.20||}
\note[type=source, labelb=235, lem={jihvādhāraḥ}]{Ysv (PT): ekādaśī [ekādaśo (YK)] bhavej jihvā talajādhāra īśvari | jihvāgramathane tasmin pānīyaṃ madhuraṃ bhavet | tatpīteṣu kavir gītijyotiś [gītir (YK)] chandovidāṃ [chandovidur (YK)] varaḥ |}
\app{\lem[wit={ceteri}]{ekādaśo}
  \rdg[wit={N2}]{\om}}
\app{\lem[wit={ceteri}]{jihvātale}
  \rdg[wit={E,N1,N2}]{\om}}
\app{\lem[wit={ceteri}]{jihvādhāraḥ}
  \rdg[wit={N2}]{\om}}/
\app{\lem[wit={ceteri}]{tasmin}
  \rdg[wit={U1}]{tasmin na}
  \rdg[wit={N2}]{\om}}
jihvāgreṇa
\app{\lem[wit={ceteri}]{manthanaṃ}
  \rdg[wit={D,L,P}]{mathanaṃ}}
\app{\lem[wit={ceteri}]{kriyate}
  \rdg[wit={B,L}]{kṛtvā}}/
tasmin-kṛte
\app{\lem[wit={ceteri}]{'timadhuraṃ}
  \rdg[wit={N1,N2}]{atimadhuraṃ}
  \rdg[wit={B,L,D}]{satimadhuraṃ}}
pānīyaṃ
\app{\lem[wit={ceteri}]{sravati}
  \rdg[wit={B}]{sravatī}}/ 
\app{\lem[wit={ceteri}]{tathā}
  \rdg[wit={E}]{tadā}
  \rdg[wit={N2}]{kamin nāsikā phatkāravat || tathā}}
\app{\lem[wit={ceteri}]{ca}
  \rdg[wit={B,L}]{\om}}
\app{\lem[wit={ceteri},alt={kavitva°}]{kavitva}
  \rdg[wit={B,L}]{kvacitva°}
  \rdg[wit={D}]{kvacitta°}
  \rdg[wit={U2}]{kavitvaṃ}
}\app{\lem[wit={N1,N2,U1},alt={°gīta°}]{gīta}
  \rdg[wit={ceteri}]{\om}
}\app{\lem[wit={E,P,B,L,U2},alt={°chando°}]{chando}
  \rdg[wit={U1}]{°chaṃdavacchaṃda°}
  \rdg[wit={ceteri}]{°chaṃda°}
}\app{\lem[wit={ceteri},alt={°nāṭakādi°}]{nāṭakādi}
  \rdg[wit={U1}]{°nāḍī°}} 
\app{\lem[wit={B,E,L,P,D,U2},alt={°viṣaya°}]{viṣaya}
  \rdg[wit={N1,N2}]{°viṣaye}
  \rdg[wit={U1}]{viṣayaṃ}}
\app{\lem[wit={ceteri},alt={jñānam}]{jñāna\skp{m-u}}
  \rdg[wit={U1}]{jñānānam}
}\app{\lem[wit={ceteri},alt={utpadyate}]{\skm{m-u}tpadyate}
  \rdg[wit={B}]{utpadyaṃte}}/
%----------------------------
%tadupari dvādaśadantayo   madhye   dantādhāraḥ/  tasmin sthāne jihvāyā  agraṃ  ghaṭīmātraṃ                    balātkāreṇa  sthāpyate/  tasmin  sati sādhakasya samagrā rogā naśyanti// \E %%%[p.44]
%tadupari dvādaśo daṃtayor madhye   daṃtādhāraḥ   tasmin sthāne jihvāyā  agraṃ  ghaṭīmātram ārghaghaṭīmātraṃ   bālātkāreṇa  sthāpyate   tasmin  sati sādhakasya samagrā rogā naśyaṃti \P
%tadupari dvādaśo daṃtayor madhye// daṃtādhāraḥ// tasmin sthāne jihvāyā  'agnaṃ ghaṭīmātram ārghaghaṭimātraṃ   bālākāreṇa   sthāpyate// tasmiṃ       sādhakasya samagrā rogā naśyaṃtī// \B
%tadupari dvādaśo daṃtayor madhye// daṃtādhāraḥ   tasmin sthāne jihvāyā  agnaṃ  ghaṭīmātram ārddhaghaṭimātraṃ  bālākāreṇa   sthāpyate// tasmiṃ       sādhakasya samagrā rogā naśyaṃti... \L
%tadupari dvādaśayor       madhye   daṃtādhāraḥ/  tasmin sthāne jihvāyā  agraṃ  ghaṭīmātraṃ arddhaghaṭimātraṃ  balātkāreṇa  sthāpyate// tasmin  sati sādhakasya samagrā rogā naśyaṃti// \N1
%tadupari dvādaśayor       madhye   daṃtādhāraḥ// tasmin sthāne jihvāyā  agraṃ  ghaṭīmātraṃ arddhaghaṭimātraṃ  balātkāreṇa  sthāpyate// tasmin  sati sādhakasya samagrā rogā naśyaṃti// \D
%tadupari dvādaśayor       madhye   daṃtādhāraḥ// tasmin sthāne jihvāyā   graṃ  ghaṭīmātraṃ arddhaghaṭimātraṃ  balātkāreṇa  sthāpyate// tasmin  sati sādhakasya samagrā rogā naśyanti \N2
%tadupari dvādaśo daṃtayor madhye   daṃtādhāraḥ   tasmin sthāne jihvāyāṃ agraṃ  ghaṭīmātram ārdhaghaṭikāmātraṃ bālātkāreṇa  sthāpyate   tasminn sati sādhakasya samagra rogā naśyaṃti \U1
%tadupari dvādaśor daṃtayo madhye   daṃtādhāraḥ   tasmin sthāne jihvāyā  agraṃ  ghaṭīmātram ārghaghaṭīmātraṃ   bālātkāreṇa  sthāpyate// tasmin  sati sādhakasya samagrā rogā naśyaṃti// \U2
%-----------------------------
%On top thereof is the twelfth, being the teeth-support, which is situated inbetween the teeth. At this place the tip of the tongue is to be positioned with force for the duration of one and a half \textit{ghāṭī}s (24+12 = 36 minutes). Abiding therein the diseases of the practitioner will entirely disappear!
%----------------------------
\note[type=source, labelb=236, lem={dantādhāraḥ}]{SSP: dvādaśe bhrūmadhyādhāre tatra candramaṇḍalaṃ dhyāyet śītalatāṃ yāti ||2.21||}
\note[type=source, labelb=237, lem={dantādhāraḥ}]{Ysv (PT): dantādhāro [dvandvādhāro (PT)] dvādaśeti sarvarogakṣayaṅkaraḥ [sarvarogaḥ (YK)] | dhārayed dantayor madhye jihvāgrañ ca balād api | dhṛtvārddhaghaṭikāmātraṃ sarvarogan [sarvarogāṃs (YK)] tu nāśayet |}
tadupari
\app{\lem[wit={P,B,L,U1},alt={dvādaśo daṃtayor}]{dvādaśo daṃtayo\skp{r-ma}}
  \rdg[wit={E}]{dvādaśadantayo}
  \rdg[wit={U2}]{dvādaśor daṃtayo}
  \rdg[wit={D,N1,N2}]{dvādaśayor}
}\skm{r-ma}dhye
daṃtādhāraḥ/
tasmin sthāne
\app{\lem[wit={ceteri}]{jihvāyā}
  \rdg[wit={U1}]{jihvāyāṃ}}
\app{\lem[wit={ceteri}]{agraṃ}
  \rdg[wit={B,L}]{agnaṃ}
  \rdg[wit={N2}]{graṃ}}
\app{\lem[wit={ceteri},alt={ghaṭīmātraṃ}]{ghaṭīmātra\skp{m-a}}
  \rdg[wit={D,N1,N2}]{ghaṭīmātraṃ}
}\app{\lem[type=emendation, resp=egoscr,alt={ardhagaṭīmātraṃ}]{\skm{m-a}rdhagaṭīmātraṃ}
  \rdg[wit={D,N1,N2}]{\korr arddhaghaṭimātraṃ}
  \rdg[wit={U1}]{ārdhaghaṭikāmātraṃ}
  \rdg[wit={P,U2}]{ārghaghaṭīmātraṃ}
  \rdg[wit={B}]{ārghaghaṭimātraṃ}
  \rdg[wit={L}]{ārddhaghaṭimātraṃ}
  \rdg[wit={E}]{\om}}
\app{\lem[wit={E,D,N1,N2}]{balātkāreṇa}
  \rdg[wit={P,U1,U2}]{bālātkāreṇa}
  \rdg[wit={B,L}]{bālākāreṇa}}
sthāpyate/
\app{\lem[wit={ceteri}]{tasmin}
  \rdg[wit={B,L}]{tasmiṃ}}
\app{\lem[wit={ceteri}]{sati}
  \rdg[wit={B,L}]{\om}}
sādhakasya samagrā rogā
\app{\lem[wit={ceteri}]{naśyanti}
  \rdg[wit={B}]{naśyaṃtī}}/
%----------------------------
%trayodaśo nāsikāgrādhāraḥ/ tasmin lakṣye kṛte sati manaḥ sthiraṃ bhavati/ \E
%trayodaśo nāsikāgrādhāraḥ  tasmiṃ lakṣye kṛte sati manaḥ sthiraṃ bhavati \P
%trayodaso nāsikādhāraḥ/    tasmin ḍraṣṭe kṛte      minasthire    bhavati/ \B
%trayodaso nāśikādhāraḥ     tasmin ḍraṣṭe kṛte      manaḥ sthiro  bhavati/ \L
%trayodaśo nāsikādhāraḥ/    tasmin lakṣe  kṛte sati manasthiraṃ   bhavati/ \N1
%trayodaśo nāsikādhāraḥ//   tasmin lakṣe  kṛte sati manasthiraṃ   bhavati \D
%trayodaśo nāsikādhāraḥ/    tasmin lakṣe  kṛte sati manasthiraṃ   bhavati/ \N2
%trayodaśo nāsikādhāraḥ     tasmiṃ lakṣye kṛte sati manasthiraṃ   bhavati \U1
%trayodaśo nāsikādhāraḥ     tasmil lakṣe  kṛte sati manasthiraṃ   bhavati// \U2
%-----------------------------
%The thirteenth is the nose-container. While making it into the fixation object the mind becomes stable. 
%----------------------------
\note[type=source, labelb=238, lem={nāsikādhāraḥ}]{SSP: trayodaśe nāsādhāre tasyāgraṃ lakṣayet manaḥ sthiraṃ bhavati ||2.22||}
\note[type=source, labelb=239, lem={nāsikādhāraḥ}]{Ysv (PT): nāsādhāras tato [tataḥ (YK)] jñeyo nāsālakṣas trayodaśaḥ [trayodaśa (YK)]| manaḥsthirakaro yas tu [sthiraṃ karoty eva (YK)] vāyusthirakaro [vāyuḥ (YK)] mahān |}
\app{\lem[wit={ceteri}]{ nāśikādhāraḥ}
  \rdg[wit={E,P}]{nāsikāgrādhāraḥ}}/
\app{\lem[type=emendation, resp=egoscr]{tasmil-lakṣye}
  \rdg[wit={U2}]{\korr tasmil lakṣe}
  \rdg[wit={E,P,U1}]{tasmiṃ lakṣye}
  \rdg[wit={D,N1,N2}]{tasmin lakṣe}
  \rdg[wit={B,L}]{tasmin ḍraṣṭe}}
kṛte
\app{\lem[wit={ceteri}]{sati}
  \rdg[wit={B,L}]{\om}}
\app{\lem[wit={E,P}]{manaḥ sthiraṃ}
  \rdg[wit={B}]{minasthire}
  \rdg[wit={L}]{manaḥ sthiro}
  \rdg[wit={ceteri}]{manasthiraṃ}}
bhavati/
\end{prose}
\end{ekdosis}
\ekdpb*{}
%%%%%%%%%%%%%%%%%%%%%%%%%%%%%%%%%%%%%%%%%%
%%%%%%%%%%%%%%%%%%%%%%%%%%%%%%%%%%%%%%%%%%
%%%%%%%%PAGEBREAK%%%%%%%PAGEBREAK%%%%%%%%%
%%%%%%%%%%%%%%%%%%%%%%%%%%%%%%%%%%%%%%%%%%
%%%%%%%%%%%%%%%%PAGEBREAK%%%%%%%%%%%%%%%%%
%%%%%%%%%%%%%%%%%%%%%%%%%%%%%%%%%%%%%%%%%%
%%%%%%%%PAGEBREAK%%%%%%%PAGEBREAK%%%%%%%%%
%%%%%%%%%%%%%%%%%%%%%%%%%%%%%%%%%%%%%%%%%%
%%%%%%%%%%%%%%%%%%%%%%%%%%%%%%%%%%%%%%%%%%
%%%%%%%%%%%%%%%%%%%%%%%%%%%%%%%%%%%%%%%%%%
%%%%%%%%%%%%%%%%%%%%%%%%%%%%%%%%%%%%%%%%%%
%%%%%%%%PAGEBREAK%%%%%%%PAGEBREAK%%%%%%%%%
%%%%%%%%%%%%%%%%%%%%%%%%%%%%%%%%%%%%%%%%%%
%%%%%%%%%%%%%%%%PAGEBREAK%%%%%%%%%%%%%%%%%
%%%%%%%%%%%%%%%%%%%%%%%%%%%%%%%%%%%%%%%%%%
%%%%%%%%PAGEBREAK%%%%%%%PAGEBREAK%%%%%%%%%
%%%%%%%%%%%%%%%%%%%%%%%%%%%%%%%%%%%%%%%%%%
%%%%%%%%%%%%%%%%%%%%%%%%%%%%%%%%%%%%%%%%%%
%%%%%%%%%%%%%%%%%%%%%%%%%%%%%%%%%%%%%%%%%%
%%%%%%%%%%%%%%%%%%%%%%%%%%%%%%%%%%%%%%%%%%
%%%%%%%%PAGEBREAK%%%%%%%PAGEBREAK%%%%%%%%%
%%%%%%%%%%%%%%%%%%%%%%%%%%%%%%%%%%%%%%%%%%
%%%%%%%%%%%%%%%%PAGEBREAK%%%%%%%%%%%%%%%%%
%%%%%%%%%%%%%%%%%%%%%%%%%%%%%%%%%%%%%%%%%%
%%%%%%%%PAGEBREAK%%%%%%%PAGEBREAK%%%%%%%%%
%%%%%%%%%%%%%%%%%%%%%%%%%%%%%%%%%%%%%%%%%%
%%%%%%%%%%%%%%%%%%%%%%%%%%%%%%%%%%%%%%%%%%
\begin{ekdosis}
  \begin{prose}
    \noindent
%----------------------------
%caturdaśo nāsāmūlādhāraḥ/         tasmin dṛṣṭeḥ            sthairyakāraṇāt   ṣaṣṭhe māsi svīyan tejaḥ pratyakṣaṃ bhavati/  tejasaḥ pratyakṣatve pārthivaṃ sakalaṃ bandhanaṃ tuṭyati/   \E
%caturdaśo nāsāmūlādhāro           tasmin dṛṣṭeḥ            sthairyakāraṇāt   ṣaṣṭhe māsi svīyaṃ tejaḥ pratyakṣaṃ bhavati   tejasaḥ pratyakṣatve pārthivaṃ sakalaṃ baṃdhanaṃ truṭyati/ \P %%%7654.jpg vorletzte Zeile
%caturdaśo nāso mūlādhāraḥ//       tasmin llakṣe krute satī sthairyakāraṇāt// ṣaṣṭhe māse svayaṃ tejaḥ pratyakṣaṃ bhavati// tejasaḥ pratyakṣatve pārthivaṃ sakalaṃ baṃdhanaṃ truṭayati/ \B
%caturdaśo nāso mūlādhāraḥ         tasmin lakṣe kṛte satī   sthairyakāraṇāt   ṣaṣṭhe māse svayaṃ tejaḥ pratyakṣaṃ bhavati// tejasaḥ pratyakṣatve pārthivaṃ sakalaṃ baṃdhanaṃ truṭayati/ \L
%caturdaśo nāsāmūle vāyvādhāraḥ/   tasmin dṛṣṭeḥ            sthairyakāraṇāt   ṣaṣṭhe māsi svīyaṃ tejaḥ pratyakṣaṃ bhavati/  tejasaḥ pratyakṣatve pārthivaṃ sakalaṃ baṃdhanaṃ trudyati/  \N1
%caturdaśo nāsāmūle vāyvādhāraḥ//  tasmin dṛṣṭeḥ            sthairyakāraṇāt   ṣaṣṭhe māsi svīyaṃ tejaḥ pratyakṣaṃ bhavati// tejasaḥ pratyakṣatve pārthivaṃ sakalaṃ baṃdhanaṃ trudyati// \D  %%%p.13 recto 
%caturdaśo nāsāmūle vāyvādhāraḥ??/ tasmin dṛṣṭeḥ            sthairyakāraṇāt   ṣaṣṭhe māsi svayaṃ tejaḥ pratyakṣaṃ bhavati   tejasaḥ pratyakṣatve pārthiva  sakalaṃ bandhanaṃ trudyati// \N2
%caturdaśo nāsāmūle vādhāraḥ       tasmiṃ na dṛṣṭeḥ         sthairyakāraṇāt   ṣaṣṭhe māse svīyaṃ tejaḥ pratyakṣaṃ bhavati   tejasaḥ pratyakṣatve pārthivaṃ sakalaṃ baṃdhanaṃ truṭyati   \U1
%caturdaśo nāsāmūlādhāraḥ          tasmin laṣṭhe?           sthairyakāraṇāt   ṣaṣṭhe māsi svayaṃ tejaḥ pratyakṣaṃ bhavati// tejasaḥ pratyakṣatve pārthivaṃ sakalaṃ baṃdhanaṃ truṭyati// \U2
%-----------------------------
%The fourteenth is the container of breath at the root of the nose. From the execution of stabilizing of the gaze onto this the light of one's own becomes perceptible within 60 months. He breaks the mundane with regard to direct perception of the light. 
%-----------------------------
\note[type=source, labelb=240, lem={nāsikādhāraḥ}]{SSP: caturdaśe nāsāmūle kapāṭādhāre dṛṣṭiṃ dhārayet | ṣaṇmāsāj jyotiḥpuñjaṃ paśyati ||2.23||}
\note[type=source, labelb=241, lem={nāsikādhāraḥ}]{Ysv (PT=YK): nāsāpuṭe sthirā dṛṣṭir ādhāro 'yaṃ caturdaśaḥ | kṛte 'smin svīyatejaḥ syāt pratyakṣaṃ ṣaṭtrimāsataḥ | pārthivaṃ truṭati kṣipraṃ pratyakṣaṃ svīyatejasā |}
caturdaśo
\app{\lem[wit={D,N1,N2}]{nāsāmūle vāyvādhāraḥ}
  \rdg[wit={U1}]{nāsāmūle vādhāraḥ}
  \rdg[wit={P}]{nāsāmūlādhāro}
  \rdg[wit={B,L}]{nāso mūlādhāraḥ}
  \rdg[wit={E,U2}]{nāsāmūlādhāraḥ}}
\app{\lem[wit={ceteri}]{tasmin}
  \rdg[wit={ceteri}]{tasmiṃ na}}
\app{\lem[wit={ceteri}]{dṛṣṭeḥ}
  \rdg[wit={U1}]{na dṛṣṭeḥ}
  \rdg[wit={B}]{llakṣe krute satī}
  \rdg[wit={L}]{lakṣe kṛte satī}
  \rdg[wit={U2}]{laṣṭhe}}
sthairyakāraṇāt
ṣaṣṭhe
\app{\lem[wit={B,L,U1}]{māse}
  \rdg[wit={ceteri}]{māsi}} 
\app{\lem[wit={ceteri}]{svīyaṃ}
  \rdg[wit={B,L,N2,U2}]{svayaṃ}}
tejaḥ pratyakṣaṃ bhavati/
tejasaḥ pratyakṣatve
\app{\lem[wit={ceteri}]{pārthivaṃ}
  \rdg[wit={N2}]{pārthiva}}
bandhanaṃ 
\app{\lem[wit={P,U2,U1}]{truṭyati}
  \rdg[wit={E}]{tuṭyati}
  \rdg[wit={B,L}]{truṭayati}
  \rdg[wit={N1,N2,D}]{trudyati}}/
%----------------------------
%pañcadaśo bhruvormadhyādhāras        tasmin dṛṣṭeḥ sthirīkaraṇāt    koṭikiraṇāḥ  sphuraṃti/ \E
%paṃcadaśo bhruvormadhyādhāraḥ        tasmin ḍṛṣṭeḥ sthirīkaraṇāt    koṭikiraṇāḥ  sphuraṃti  \P  %%%7655.jpg
%paṃcadaśo bhruvormadhye dhāraḥ//     tasmin ḍṛṣṭeḥ sthirikaraṇāt//  koṭikiriṇā   sphuraṃti// \B
%paṃcadaśo bhruvormadhye dhāraḥ//     tasmin ḍṛṣṭe  sthirīkaraṇāt//  koṭikiriṇā   sphuraṃti// \L
%pañcadaśo bhruvormadhye ādhāraḥ/      asmin dṛṣṭeḥ sthirīkaraṇāt    koṭikiraṇāni sphuraṃti/ \N1
%pañcadaśo bhruvormadhye ājñādhāraḥ// ..smin dṛṣṭeḥ sthirīkaraṇāt    koṭikiraṇāni sphuraṃti// \D
%pañcadaśo bhruvormadhye ādhāraḥ      tasmin dṛṣṭeḥ sthirīkaraṇāt    koṭikiraṇāni sphuraṃti/ \N2 [S.9]
%pañcadaśo bhruvormadhye ādhāra         asin na dṛṣṭeḥ sthirīkaraṇāt koṭikiraṇāni sphuraṃti \U1
%pañcadaśo bhruvormadhyādhāra         tasmin dṛṣṭisthirīkaraṇāt      koṭikiraṇaḥ  sphuraṃti// \U2
%-----------------------------
%The fifteenth container is situated in the middle of the eyebrows. Due to stabilized the gaze therein 10 million rays of light sparkle. 
%----------------------------
\note[type=source, labelb=242, lem={nāsikādhāraḥ}]{SSP: pañcadaśe lalāṭādhāre tatra jyotiḥpuñjaṃ lakṣayet | tejasvī bhavati ||2.24||}
\note[type=source, labelb=243, lem={nāsikādhāraḥ}]{Ysv (PT): pañcadaśo bhruvormadhye sthira [sthirā (YK)] dṛṣṭis tathā dhruvam | asmin dṛṣṭiḥ sthirā koṭiḥ [koṭi° (YK)] kiraṇāni sphuranti hi |}
pañcadaśo
\app{\lem[type=emendation, resp=egoscr]{bhruvormadhya ādhāraḥ}
  \rdg[wit={N1,N2}]{\korr bhruvormadhye ādhāraḥ}
  \rdg[wit={U1}]{bhruvormadhye ādhāra}
  \rdg[wit={L,B}]{bhruvormadhye dhāraḥ}
  \rdg[wit={U2}]{bhruvormadhyādhāra}
  \rdg[wit={P}]{bhruvormadhyādhāraḥ}
  \rdg[wit={E}]{bhruvormadhyādhāras}
  \rdg[wit={D}]{bhruvormadhye ājñādhāraḥ}}/
\app{\lem[wit={ceteri}]{tasmin}
  \rdg[wit={N1}]{asmin}
  \rdg[wit={D}]{smin}
  \rdg[wit={U1}]{asin}
}\app{\lem[wit={ceteri}]{ḍṛṣṭeḥ}
  \rdg[wit={L}]{ḍṛṣṭe}
  \rdg[wit={U1}]{na dṛṣṭeḥ}
  \rdg[wit={U2}]{dṛṣṭi°}}
sthirīkaraṇāt
koṭi\app{\lem[wit={D,N1,N2,U1}]{kiraṇāni}
  \rdg[wit={E,P}]{koṭikiraṇāḥ}
  \rdg[wit={U2}]{koṭikiraṇaḥ}
  \rdg[wit={B,L}]{koṭikiriṇā}}
sphuranti/
\note[type=philcomm, labelb=244, lem={kiraṇāni}]{The better group of witnesses D\textsubscript{1},N\textsubscript{1},N\textsubscript{2} and U\textsubscript{1} support the uncommon neuter from of \textit{kiraṇa}. This is also supported by the Ysv and was hence adopted.}
%----------------------------
%ṣoḍaśo  netrādhāraḥ/  ayam aṃgulyagreṇa cālyate/  tadabhyāsāt/ pṛthvīmadhye  yatkiṃcin  tejo  varttate/  \E   %%%p.45
%ṣoḍaśo  netrādhāraḥ   ayam aṃgulyagreṇa cālyate   tadabhyāsāt  pṛthvīmadhye  yatkiṃcit  tejo  vartate... \P
%ṣoḍaśo  netrā//       ayam aṃgulyagreṇa cālyate// tadabhyāsāt  pṛthivīmadhye yatkiṃcit  tejo  vartate//  \B %%%%%%%%%%%%%%%%DSCN7167.jpg Z. 1
%ṣoḍaśo  netrā//       ayam aṃgulyagreṇa cālyate// tadabhyāsāt  pṛthivīmadhye yatkiṃcit  tejo  vartate... \L
%ṣoḍaśaḥ netrādhāraḥ/  ayaṃ agulyagreṇa  cālyate/  tadabhyāsāt  pṛthvīmadhye  yatkiṃcit  tejaḥ varttate/  \N1
%ṣoḍaśaḥ netrādhāraḥ// ayaṃ agulyagreṇa  cālyate// tadabhyāsāt  pṛthvīmadhye  yatkiṃcit  tejaḥ varttate \D
%ṣoḍaśaḥ netrādhāraḥ/  ayaṃ aṃgugreṇa    cālyate/  tadabhyāsāt  pṛthvīmadhye  yatkiṃcit  tejaḥ varttate/  \N2
%ṣoḍaśo  netrādhāraḥ   ayaṃ aṃgulyagreṇa cālyate   tadābhyāsāt  pṛthvīmadhye  yatkiṃcit        vatate     \U1 %%%%%%%%%%%%%%%%%%285.jpg
%ṣoḍaśo  netrādhāraḥ   ayam aṃgulyagreṇa cālyate// tadabhyāsāt  pṛthivīmadhye yatkiṃcit// tejo vartate//  \U2
%-----------------------------
%The sixteenth is the eye-container. Without wavering, the gaze [ayam] is to be held at the tip of the finger without wavering. From practicing this on earth any energy exists [for him].   
%-----------------------------
\note[type=source, labelb=245, lem={netrādhāraḥ}]{SSP: avaśiṣṭe ṣoḍaśe brahmarandhram ākāśacakram | tatra śrīgurucaraṇāmbujayugmaṃ sadāvalokayet | ākāśavat pūrṇo bhavati ||2.25||}
\note[type=source, labelb=246, lem={netrādhāraḥ}]{Ysv (PT): netrādhāraḥ ṣoḍaśo 'yam aṅgulyagreṇa cālayet | pṛthvīmadhye tu yatkiñcid varttate [sarvajñaḥ prabhavastena varddhate (YK)] jaṭharānalaḥ | pratyakṣaṃ tad bhavet sarvaṃ tadābhyāsān na saṃśayaḥ |}
\app{\lem[wit={ceteri}]{ṣoḍaśo}
  \rdg[wit={D,N1,N2}]{ṣoḍaśaḥ}}
\app{\lem[wit={ceteri}]{netrādhāraḥ}
  \rdg[wit={L,B}]{netrā}}/
\app{\lem[wit={ceteri},alt={ayam}]{aya\skp{m-a}}
  \rdg[wit={D,N1,N2,U1}]{ayaṃ}
}\app{\lem[type=emendation, resp=egoscr, alt={aṅgulyagre na}]{\skm{m-a}ṅgulyagre na}
  \rdg[wit={ceteri}]{\korr aṅgulyagreṇa}
  \rdg[wit={N1,D}]{agulyagreṇa}
  \rdg[wit={N2}]{aṃgugreṇa}}
cālyate/
tadabhyāsāt
\app{\lem[wit={ceteri},alt={pṛthvī°}]{pṛthvī}
  \rdg[wit={L,B,U2}]{pṛthivī°}}madhye
yatkiṃcit
\app{\lem[wit={ceteri}]{tejo}
  \rdg[wit={D,N1,N2}]{tejaḥ}
  \rdg[wit={U1}]{\om}}
\app{\lem[wit={ceteri}]{vartate}
  \rdg[wit={U1}]{vatate}}/
%----------------------------
%tatsarvaṃ tejo   dṛṣṭiviṣayaṃ bhavati/  taddarśanāt  puruṣaḥ sarvajño  bhavati// \E
%tatsarvaṃ tejo   dṛṣṭiviṣayaṃ bhavati   tadarśanāt   puruṣaḥ sarvajño  bhavati     \P
%tatsarvaṃ tejo   dṛṣṭiviṣayaṃ bhavatī// taddarśanāt  puruṣaḥ sarvajño  bhavatī// \B
%tatsarvaṃ tejo   dṛṣṭiviṣayaṃ bhavati// taddarśanāt  puruṣaḥ sarvajño  bhavati// \L
%tatsarvvatejo    dṛṣṭiviṣayaṃ bhavati   taddarśanāt  puruṣaḥ sarvvajño bhavati// \N1
%tatsarvatejo     dṛṣṭiviṣayaṃ bhavati   taddarśanāt  puruṣaḥ sarvvajño bhavati// \D
%tatsarvatejo     dṛṣṭiviṣayaṃ bhavati   taddarśanāt  puruṣaḥ sarvajño  bhavati// \N2
%tatsarvaṃ tejo   dṛṣṭīviṣayaṃ bhavati   tatdarśaḥ    puruṣaḥ sarvajño  bhavati \U1
%tatsarvaṃ tajaso dṛṣṭiviṣayaṃ bhavati// taddarśanāt  puruṣaḥ sarvajño  bhavati// \U2
%-----------------------------
%The light of everying that is arises as the object of sight. From that sight the person becomes omniscient. 
%-----------------------------
\app{\lem[wit={D,N1,N2}]{tatsarvatejo}
  \rdg[wit={ceteri}]{tatsarvaṃ}}
dṛṣṭiviṣayaṃ
\app{\lem[wit={ceteri}]{bhavati}
  \rdg[wit={B}]{bhavatī}}
\app{\lem[wit={ceteri}]{taddarśanāt}
  \rdg[wit={P}]{tadarśanāt}
  \rdg[wit={U1}]{tatdarśaḥ}}
puruṣaḥ
sarvajño 
\app{\lem[wit={ceteri}]{bhavati}
  \rdg[wit={B}]{bhavatī}}/\\
\end{prose}
\end{ekdosis}
\ekdpb*{}
%%%%%%%%%%%%%%%%%%%%%%%%%%%%%%%%%%%%%%%%%%
%%%%%%%%%%%%%%%%%%%%%%%%%%%%%%%%%%%%%%%%%%
%%%%%%%%PAGEBREAK%%%%%%%PAGEBREAK%%%%%%%%%
%%%%%%%%%%%%%%%%%%%%%%%%%%%%%%%%%%%%%%%%%%
%%%%%%%%%%%%%%%%PAGEBREAK%%%%%%%%%%%%%%%%%
%%%%%%%%%%%%%%%%%%%%%%%%%%%%%%%%%%%%%%%%%%
%%%%%%%%PAGEBREAK%%%%%%%PAGEBREAK%%%%%%%%%
%%%%%%%%%%%%%%%%%%%%%%%%%%%%%%%%%%%%%%%%%%
%%%%%%%%%%%%%%%%%%%%%%%%%%%%%%%%%%%%%%%%%%
%%%%%%%%%%%%%%%%%%%%%%%%%%%%%%%%%%%%%%%%%%
%%%%%%%%%%%%%%%%%%%%%%%%%%%%%%%%%%%%%%%%%%
%%%%%%%%PAGEBREAK%%%%%%%PAGEBREAK%%%%%%%%%
%%%%%%%%%%%%%%%%%%%%%%%%%%%%%%%%%%%%%%%%%%
%%%%%%%%%%%%%%%%PAGEBREAK%%%%%%%%%%%%%%%%%
%%%%%%%%%%%%%%%%%%%%%%%%%%%%%%%%%%%%%%%%%%
%%%%%%%%PAGEBREAK%%%%%%%PAGEBREAK%%%%%%%%%
%%%%%%%%%%%%%%%%%%%%%%%%%%%%%%%%%%%%%%%%%%
%%%%%%%%%%%%%%%%%%%%%%%%%%%%%%%%%%%%%%%%%%
%%%%%%%%%%%%%%%%%%%%%%%%%%%%%%%%%%%%%%%%%%
%%%%%%%%%%%%%%%%%%%%%%%%%%%%%%%%%%%%%%%%%%
%%%%%%%%PAGEBREAK%%%%%%%PAGEBREAK%%%%%%%%%
%%%%%%%%%%%%%%%%%%%%%%%%%%%%%%%%%%%%%%%%%%
%%%%%%%%%%%%%%%%PAGEBREAK%%%%%%%%%%%%%%%%%
%%%%%%%%%%%%%%%%%%%%%%%%%%%%%%%%%%%%%%%%%%
%%%%%%%%PAGEBREAK%%%%%%%PAGEBREAK%%%%%%%%%
%%%%%%%%%%%%%%%%%%%%%%%%%%%%%%%%%%%%%%%%%%
%%%%%%%%%%%%%%%%%%%%%%%%%%%%%%%%%%%%%%%%%%
\begin{ekdosis}
  \ekddiv{type=ed}
   \centerline{\textrm{\small{[Aṣṭāṅgayoga]}}}
      \bigskip
      \begin{prose}
        \noindent
%----------------------------
%Note: Rāmacandra does not adopt the yāmas and niyāmas from the Yogasvarodaya! 
%----------------------------
%idānīm aṣṭāṃgayoga----vicāraḥ kathyate/  yamaniyamāsanaprāṇāyāmapratyāhāradhyānadhāraṇāsamādhir iti/  eteṣāṃ lakṣaṇāni kathyante/     \E
%idānīm aṣṭāṃgayogasya vicāraḥ kathyate   yamaniyamāsanaprāṇāyāmapratyāhāradhyānadhāraṇāsamādhir iti   eteṣāṃ lakṣaṇāni kathyaṃte  \P
%idānīm aṣṭāṃgayogasya vicāraḥ kathyate/  yamaniyamāsanaprāṇāyāmapratyāhāradhāraṇādhyānasamādhir iti/  eteṣāṃ lakṣaṇāni kathyaṃte/ \B
%idānīm aṣṭāṃgayogasya vicāraḥ kathyate/  yamaniyamāsanaprāṇāyāmapratyāhāradhāraṇādhyānasamādhir iti/  eteṣāṃ lakṣaṇāni kathyaṃte/ \L
%idānīm aṣṭāṃgayogasya vicāraḥ kathyate// yamaniyamāsanaprāṇāyāmapratyāhāradhyānadhāraṇāsamādhiyaḥ     eteṣāṃ lakṣaṇāni kathyaṃte/   \N1
%idānīm aṣṭāṃgayogasya vicāraḥ kathyate// yamaniyamāsanaprāṇāyāmapratyāhāradhyānadhāraṇāsamādhi//      eteṣāṃ lakṣaṇāni kathyaṃte//   \D
%idānīṃ aṣṭāṃgayogasya vicāraḥ kathyate// yamaniyamāsanaprāṇāyāmapratyāhāradhyānadhāraṇāsamādhiyaḥ     eteṣāṃ lakṣaṇāni kathyaṃte/   \N2
%idānīṃ aṣṭāṅgayogasya vicāraḥ kathyate// yamaniyamāsanaprāṇāyāmapratyāhāradhyānadhāraṇāsamādhi        eteṣāṃ lakṣaṇāni kathyate   \U1
%idānīṃ aṣṭāṅgayogasya vicāra  kathyate// yamaniyamāsanaprāṇāyāmapratyāhāradhyānadhāraṇāsamādhir iti// eteṣāṃ lakṣaṇāni kathyaṃte//   \U2
%-----------------------------
%Now the procedure of the eightfold yoga (\textit{aṣṭāṅgayoga})is explained: "Yama, niyama, āsana, prāṇāyāma, pratyāhāra, dhyāna, dhāraṇā and samādhi." Their characteristics will be explained.   
%----------------------------
\note[type=source, labelb=247, lem={aṣṭāṃga°}]{SSP:yamaniyamāsanaprāṇāyāmapratyāhāradhāraṇādhyānasamādhayoḥ 'ṣṭāṅgāni|}
\note[type=source, labelb=248, lem={aṣṭāṃga°}]{Ysv (PT=YK): idānīṃ yogamaṣṭāṅgaṃ śṛṇu lakṣaṇasaṃyutam | yamaś ca niyamaś caiva cāsanaṃ prāṇasaṃyamaḥ | pratyāhāro dhāraṇā ca samādhiś ca viśeṣataḥ | aṣṭāṅgayoga ebhis tu caiteṣāṃ lakṣaṇaṃ śṛṇu |}
\app{\lem[wit={ceteri},alt={idānīm}]{idānī\skp{m-a}}
  \rdg[wit={N2,U1,U2}]{idānīṃ}
}\app{\lem[wit={ceteri},alt={aṣṭāṅgayogasya}]{\skm{m-a}ṣṭāṅgayogasya}
  \rdg[wit={E}]{aṣṭāṃgayoga°}}
\app{\lem[wit={ceteri}]{vicāraḥ}
  \rdg[wit={U2}]{vicāra}}
kathyate/
yamaniyamāsanaprāṇāyāmapratyāhāra\app{\lem[wit={ceteri},alt={°dhyānadhāraṇāsamādhir iti}]{dhyānadhāraṇāsamādhir\skp{-}iti}
  \rdg[wit={B,L}]{dhāraṇādhyānasamādhir iti}
  \rdg[wit={N1,N2}]{dhyānadhāraṇāsamādhiyaḥ}
  \rdg[wit={D,U1}]{dhyānadhāraṇāsamādhi}}
eteṣāṃ lakṣaṇāni
\app{\lem[wit={ceteri}]{kathyante}
  \rdg[wit={U1}]{kathyate}}/
%----------------------------
%śāntiḥ/ ṣaṇṇām  indriyāṇāṃ jayaḥ/ svalpāhāraḥ/            nidrājayaḥ/      śītoṣṇajayaḥ/               ete yamāḥ/ \E
%śāṃtiḥ  ṣaṇāṃ   iṃdriyāṇāṃ jayaḥ       ahāraḥ svalpaḥ     nidrājayaḥ       śaityajayaḥ   uṣṇa?jayaḥ    ete yamāniyamāḥ ...\P
%śāntiḥ  ṣaṇāṃ   iṃdriṇāṃ   jayaḥ//     ahāraḥ svalpaḥ     nidrāyā jayaḥ//  śaityajayaḥ/  uṣṇājayaḥ// ya te yamaḥ// \B
%śāntiḥ  ṣaṇṇāṃ  iṃdriyāṇāṃ jayaḥ//     ahāraḥ// svalpaḥ// nidrāyāḥ jayaḥ/  śaityajayaḥ   uṣṇajayaḥ   ya te yamaḥ... \L
%śānti---ṣaṇṇāṃ  indriyāṇāṃ jayaḥ/      svalpāḥ            nidrājayaḥ/      śītyajayaḥ/   uṣṇajayaḥ/    ete yamāḥ/ \N1
%śāṃti---ṣaṇṇāṃ  indriyāṇāṃ jayaḥ//     āhāraḥ svalpāḥ     nidrājayaḥ//     śaityajayaḥ// uṣṇajayaḥ/    ete yamāḥ \D
%śānti---ṣaṇṇāṃ  indriyāṇāṃ jayaḥ/      ahāraḥ svalpāḥ     nidrājayaḥ/      śaityajayaḥ   uṣṇajayaḥ/    ete yamāḥ/ \N2
%śāntiḥ  ṣaṇṇām  iṃdriyāṇāṃ jayaḥ       āhāraḥ sajayaḥ     nidrājayaḥ       śaityajayaḥ   auṣṇājayaḥ    ete yamāḥ \U1
%śānti---śaṇa    iṃdriyāṇāṃ jayaḥ//     āhāraḥ svalpaḥ//   nidrāyāḥ jayaḥ// śaityajayaḥ// uṣṇājayaḥ//   ete yamāḥ// \U2 %%%417.jpg 
%----------------------------
%These are the Yāmas: Peace, conquer of the six senses, little food, conquer of sleep, conquer of cold and heat.
%----------------------------
\note[type=source, labelb=249, lem={ete yamāḥ}]{SSP:yama iti upaśamaḥ sarvendriyajayaḥ āhāranidrāśītavātātapajayaś caivaṃ śanaiḥ śanaisādhayet ||2.32||}
\note[type=source, labelb=250, lem={ete yamāḥ}]{Ysv (PT): śāntiḥ santoṣa āhāro nidrālpā [nidrālpaṃ (YK)] manaso damaḥ | śūnyāntaḥ karaṇañceti [karaṇaś ceti (YK)] yamā iti prakīrttitāḥ |}
\app{\lem[wit={ceteri}]{śāntiḥ}
  \rdg[wit={D,N1,N2,U2}]{śānti°}}\dd{}
\app{\lem[wit={E,U1},alt={ṣaṇṇām}]{ṣaṇṇā\skp{m-i}}
  \rdg[wit={D,L,N1,N2}]{ṣaṇṇāṃ}
  \rdg[wit={B,P}]{ṣaṇāṃ}
  \rdg[wit={U2}]{śaṇa}
}\app{\lem[wit={ceteri},alt={indriyāṇāṃ}]{\skm{m-i}ndriyāṇāṃ}
  \rdg[wit={B}]{iṃdriṇāṃ}}
jayaḥ\dd{}
\app{\lem[wit={U2}]{āhāraḥ svalpaḥ}
  \rdg[wit={E}]{svalpāhāraḥ}
  \rdg[wit={B,P}]{ahāraḥ svalpaḥ}
  \rdg[wit={L}]{ahāraḥ|| svalpaḥ ||}
  \rdg[wit={N1}]{svalpāḥ}
  \rdg[wit={N2}]{ahāraḥ svalpāḥ}
  \rdg[wit={D}]{āhāraḥ svalpāḥ}
  \rdg[wit={U1}]{āhāraḥ sajayaḥ}}\dd{}
\app{\lem[wit={ceteri}]{nidrājayaḥ}
  \rdg[wit={B}]{nidrāyā jayaḥ}
  \rdg[wit={L,U2}]{nidrāyāḥ jayaḥ}}\dd{}
\app{\lem[wit={ceteri}]{śaityajayaḥ}
  \rdg[wit={N1}]{śītyajayaḥ}
  \rdg[wit={E}]{śītoṣṇajayaḥ}}\dd{}
\app{\lem[wit={ceteri}]{uṣṇajayaḥ}
  \rdg[wit={B,U2}]{uṣṇājayaḥ}
  \rdg[wit={U1}]{auṣṇājayaḥ}
  \rdg[wit={E}]{\om}}\dd{}
\app{\lem[wit={ceteri}]{ete}
  \rdg[wit={B,L}]{ya te}}
\app{\lem[wit={ceteri}]{yamāḥ}
  \rdg[wit={P}]{yamāniyamāḥ}
  \rdg[wit={B,L}]{yamaḥ}}\dd{}
%----------------------------
%niyamāḥ   khalu       cāpalabhāvān nivārya  sthairye  sthāpyate/  ekāṃte sevanam/ prāṇimātre samābuddhiḥ/ audāsīnyaṃ   kasyāpi vastuna    icchā na karttavyā    yathā lābhasaṃtoṣaḥ/   \E
%          khalu       cāpalābhāvān nirvārya sthairye  sthāpyate   ekāṃta sevānaṃ  prāṇimātre samābuddhiḥ   udāsīnyaṃ   kasyāpi vastuna    icchā na kartavyā     yathā lābhasaṃtoṣaḥ    \P %%%7656.jpg
%          khalu       cāpalabhāvān nirvārya           sthāpyate//           ekāṃta sevānāṃ  prāṇimātre samābuddhiḥ   udāsīnyaṃ   kasyāpi vastunaḥ// icchā na kartavyā     yathā lābhasaṃtoṣaḥ/   \B
%          ḱhalu       cāpalabhāvān nirvārya           sthāpyate//           ekāṃtasevānāṃ   prāṇimātre samābuddhiḥ/  udāsīnyaṃ   kasyāpi vastunaḥ/  icchā na kartavyā     yathā lābhasaṃtoṣaḥ    \L
%niyamaḥ   khalu       capalabhāvān nivārya  sthairye  sthāpyate/  ekāṃte sevanam/ prāṇimātre samābuddhiḥ/  udāsīnya/   kasyāpi vastunaḥ   icchā na karttavyā//  yathā lābhasaṃtoṣaḥ/   \N1
%niyamaḥ   khalu manaḥ capalabhāvān nivārye            sthāpyate//           ekāṃtasevanaṃ// prāṇimātre samābuddhiḥ// udāsīnya//  kasyāpi vastunaḥ   icchā na karttavyā//  yathā lābhasaṃtoṣaḥ//  \D
%niyamaḥ   khalū manaḥ capalabhāvān nivārya  sthairye  sthāpyate   ekāṃtasevanam/  prāṇimātre samābuddhiḥ   udāsīnya    kasyāpi vastunaḥ   icchā na karttavyā/   yathā lābhasaṃtoṣaḥ    \N2
%niyamaḥ   khalū manaḥ capalabhāvān nivāraya sthairye  sthāpyate   ekāṃtasevanaṃ   prāṇimātre samābuddhi    udāsīnyāṃ   kasyāpi vastunaḥ   icchā na karttavyaṃ   yathā lābhasaṃtoṣaḥ    \U1
%niyamaḥ// khalū       cāpalābhāvān nivārya            sthāpyate// ekāṃtasevanaṃ// prāṇimātre samābuddhi//  udāsīnyaṃ// kasyāpi vastuna    icchā na karttavyaṃ// yathā lābhasaṃtoṣaḥ//  \U2
%----------------------------
%parameśvaranāma na vismaraṇīyam/  manomadhye      dainyaṃ    karttavyam/ iti niyamāḥ// \E
%parameśvaranāma na vismaraṇīyaṃ   manomadhye      dainyaṃ    kartavyaṃ   iti niyamāḥ\P %%%7656.jpg
%parameśvaranāma na vismaraṇīyaṃ   manomadhye      dainyaṃ    kartavyaṃ// iti niyamaḥ// \B
%parameśvaranāma na vismaraṇīyaṃ   manomadhye      dainyaṃ    karttavyaṃ/ iti niyamaḥ// \L
%parameśvaranāma----vismaraṇīyam/  manomadhye      dainyaṃ na karttavyam/ //[S.11] \N1
%parameśvaranāma----vismaraṇīyaṃ// manomadhye      dainyaṃ na karttavyaṃ// \D
%parameśvaranāma----vismanīyam/    manomadhye      dainyaṃ na karttavyam// // \N2 \em zu vismāra
%parameśvaraḥ nāma na vismaraṇīyaṃ mano            dainyaṃ na karttavyaṃ  \U1
%parameśvaraḥ nāma na vismaraṇaṃ// yaṃ mano madhye dainyaṃ na karttavyaṃ iti niyamaḥ//  \U2
%----------------------------
%Niyamās are truly: Keeping the mind from the state of unsteadiness [and] ground it in calmness, retreating to a lonely place, refraining from contact to animals, unchanging intellect, keeping equanimous one shall not crave for things, as well as being contend with what is given, never forgetting the name of the highest lord, one shall not bring the mind into depression. 
%----------------------------
\note[type=source, labelb=251, lem={niyamāḥ}]{SSP:niyama iti manovṛttīnāṃ niyamanam iti ekāntavāso niḥsaṃgataudāsīnyaṃ yathāprāptisaṃtuṣṭir vairasyaṃ gurucaraṇāvarūḍhatvam iti niyamalakṣaṇam ||2.33||}
\note[type=source, labelb=252, lem={niyamāḥ}]{Ysv (YK): tyaktvā dūre tu cāpalyaṃ [cāpalyantu dūre tyaktvā (Ysv)] manaḥ sthairyyaṃ vidhāya ca ||31|| ekatra melanaṃ nityaṃ prāṇāmātre na sāmabhiḥ [sā matiḥ (PT)] | sadodāsīnabhāvas tu sarvatrecchāvivarjitaḥ [°vivarjanam (PT)] ||32|| yathālābhena santuṣṭaḥ parameśvaramānasaḥ | mānadānaparityāga ete tu niyamā iti || 33||}
\app{\lem[wit={E}]{niyamāḥ}
  \rdg[wit={D,N1,N2,U1,U2}]{niyamaḥ}
  \rdg[wit={B,P,L}]{\om}}\dd{}
\app{\lem[wit={ceteri}]{khalu}
  \rdg[wit={N1,N2,U2}]{khalū}} 
\app{\lem[wit={D,N2,U1}]{manaḥ}
  \rdg[wit={ceteri}]{\om}}
\app{\lem[wit={B,E,L},alt={cāpala°}]{cāpala}
  \rdg[wit={P,U2}]{cāpalā°}
  \rdg[wit={D,N1,N2,U1}]{capala°}
}bhāvā\skp{n-ni}
\app{\lem[wit={ceteri},alt={nivārya}]{\skm{n-ni}vārya}
  \rdg[wit={D}]{nivārye}
  \rdg[wit={B,L,P}]{nirvārya}
  \rdg[wit={U1}]{nivāraya}}
\app{\lem[wit={ceteri}]{sthairye}
  \rdg[wit={B,L,D,U2}]{\om}}
sthāpyate\dd{}
%----------------------------
%āsanalakṣaṇaṃ     bahuṣu grantheṣu nirūpitam     asti    tenātra na nirūpyate/ \E
%āsanalakṣaṇaṃ     bahuṣu graṃtheṣu nirūpitam     asti    tenātra na nirūpyate \P
%āsanaṃ lakṣaṇāṃ   bahūgraṃtheṣu    nirūpyam      asti    tenātra    nirūpyate/       \B
%āsanalakṣaṇāṃ     bahūgraṃtheṣu    nirūpyam      asti    tenātra    nirūpyate//     \L
%āsanasya lakṣaṇaṃ bahūgraṃthe      nirūpitam/    ataḥ    atrāyaṃ    nirūpyate/   \N1
%āsanasya lakṣaṇaṃ bahūgraṃthe      nirūpitaṃ//   ataḥ    atratyaṃ   nirūpyate// \D %%%p. 13 verso
%āsanasya lakṣaṇaṃ bahugraṃthe      nirūpitam//   ataḥ    atrāyaṃ    nirūpyate/  \N2
%āsanasya lakṣaṇaṃ bahugraṃthe      nirūpitam tan attaḥ   atra    na nirūpyate  \U1
%āsanalakṣaṇaṃ tu  bahugraṃtheṣu    nirūpitam     asti//  tenātra    nirūpyate// \U2
%----------------------------
%The characteristic of posture has been discussed in many works. Because of that it will not be discussed here.  
%----------------------------
\note[type=source, labelb=253, lem={āsanasya}]{SSP: āsanam iti svasvarūpe samāsannatā | svastikāsanaṃ padmāsanaṃ siddhāsanam eteṣāṃ madhye yatheṣṭam ekaṃ vidhāya sāvadhānena sthātavyam ity āsanalakṣaṇam ||2.34||}
\note[type=source, labelb=254, lem={āsanasya}]{Ysv (YK): āsanāni ca tāvanti yāvanto jīvajantavaḥ |[om. YK]}
\app{\lem[wit={D,N1,N2,U1}]{āsanasya lakṣaṇaṃ}
  \rdg[wit={E,P,L}]{āsanalakṣaṇaṃ}
  \rdg[wit={U2}]{āsanalakṣaṇaṃ tu}
  \rdg[wit={B}]{āsanaṃ lakṣaṇāṃ}}
\app{\lem[wit={B,L,U2}]{bahūgrantheṣu}
  \rdg[wit={E,P}]{bahuṣu graṃtheṣu}
  \rdg[wit={D,N1,N2,U1}]{bahūgraṃthe}}
\app{\lem[wit={E,P,U2},alt={nirūpitam}]{nirūpita\skp{m-a}}
  \rdg[wit={D}]{nirūpitaṃ ||}
  \rdg[wit={N1,N2}]{nirūpitam |}
  \rdg[wit={B,L}]{nirūpyam}
  \rdg[wit={U1}]{nirūpitam tan}}
\app{\lem[wit={B,E,L,P,U2},alt={asti}]{\skm{m-a}sti}
  \rdg[wit={D,N1,N2,U1}]{ataḥ}}
\app{\lem[wit={U2}]{/}
  \rdg[wit={ceteri}]{\om}} 
\app{\lem[wit={B,E,L,P,U2}]{tenātra}
  \rdg[wit={N1,N2}]{atrāyaṃ}
  \rdg[wit={D}]{atratyaṃ}
  \rdg[wit={U1}]{atra}}
\app{\lem[wit={E,P,U1}]{na}
  \rdg[wit={ceteri}]{\om}}
nirūpyate/
%---------------------------
%prāṇāyāmas tu sukumāreṇa        sādhituṃ na śakyate   atas tasya nāmamātraṃ kathyate/ \E
%prāṇāyāmas tu sukumāreṇa        sādhituṃ na śakyate   atas tasya nāmamātraṃ kathyate  \P
%prāṇāyāmas tu kumāreṇa          sādhituṃ na śakyate// ataḥ       nāma       kathyate/ \B
%prāṇāyāmas tu kumāreṇa          sādhituṃ na śakyate// ataḥ       nāma       kathyate// \L
%prāṇāyāmas tu kūmāreṇa puruṣeṇa sādhituṃ na śakyate/  ataḥ tasya nāmamātraṃ kathitaṃ/ \N1
%prāṇāyāmas tu kūmāreṇa puruṣeṇa sādhituṃ na śakyate// ataḥ tasya nāmamātre  kathitaṃ// \D
%prāṇāyāmas tu kūmāreṇa puruṣeṇa sādhituṃ na śakyate// ata  tasya nāmamātre  kathitaṃ/ \N2
%prāṇāyāmas tu kūmāreṇa puruṣeṇa sādhituṃ na śakyate   atas tasya nāmamātre  kathitaṃ \U1
%prāṇāyāmas tu kūmāreṇa          sādhituṃ na śakyate// atā  tasya nāmamātraṃ  kathyate// \U2
%----------------------------
%Breath-control can't be practiced by young persons. That's why it is just mentioned by name.
%Practicing breath-control can't be done by a young person. 
%----------------------------
\note[type=source, labelb=255, lem={prāṇāyāmas}]{SSP: prāṇāyāma iti prāṇasya sthiratā recakapūrakakumbhakasaṃghaṭṭakaraṇāni catvāri prāṇāyāmalakaṇam ||2.35||}
\note[type=source, labelb=256, lem={prāṇāyāmas}]{Ysv (YK): prāṇāyāmas tridhā ceti bahudhā prathamaṃ śṛṇu | āsane prāṇasaṃyāme na śaktāḥ sukumārakāḥ | mahāpuṇyaprabhāveṇa śakyate tu mahātmanā | iḍāṃ śaśiprabhāṃ dhyātvā mandendunā [yathāśakti (YK)] tu pūrayet [tu kumbhayet (YK)] | pūrayitvā yathāśakti dhyānayogī tu kumbhayet [sentence om. (YK)] | mahājyotir mano [mayo (YK)] bhūtvā vāyuḥ [vāyu° (YK)] pūrṇakalevaraḥ | śaktitrāsantu santrāsya recayed vāyum arhitaḥ | piṅgalām arkavarṇān [°varṇaṃ (YK)] tu tyajed dhyātvā śanaiḥ śanaiḥ | ayaṃ pataṅgaḥ kāṣṭhāgnipratyāsena punaḥ punaḥ | kṛtvā kalevaraṃ śuddhaṃ kuryād yatnair mahātmanā | mano nivārya saṃsāre viṣayakārye [viṣayeṣu (YK)] tathaiva ca | manovikārabhavañ caiva [manovikārān sarvāś ca (YK)] tyaktvā śūnyamayo bhavet |}
%----------------------------
prāṇāyāmas-tu
\app{\lem[wit={E,P}]{sukumāreṇa}
  \rdg[wit={B,L,U2}]{kumāreṇa}
  \rdg[wit={D,N1,N2,U1}]{kūmāreṇa puruṣeṇa}}
sādhituṃ na śakyate/
\end{prose}
\end{ekdosis}
\ekdpb*{}
%%%%%%%%%%%%%%%%%%%%%%%%%%%%%%%%%%%%%%%%%%
%%%%%%%%%%%%%%%%%%%%%%%%%%%%%%%%%%%%%%%%%%
%%%%%%%%PAGEBREAK%%%%%%%PAGEBREAK%%%%%%%%%
%%%%%%%%%%%%%%%%%%%%%%%%%%%%%%%%%%%%%%%%%%
%%%%%%%%%%%%%%%%PAGEBREAK%%%%%%%%%%%%%%%%%
%%%%%%%%%%%%%%%%%%%%%%%%%%%%%%%%%%%%%%%%%%
%%%%%%%%PAGEBREAK%%%%%%%PAGEBREAK%%%%%%%%%
%%%%%%%%%%%%%%%%%%%%%%%%%%%%%%%%%%%%%%%%%%
%%%%%%%%%%%%%%%%%%%%%%%%%%%%%%%%%%%%%%%%%%
%%%%%%%%%%%%%%%%%%%%%%%%%%%%%%%%%%%%%%%%%%
%%%%%%%%%%%%%%%%%%%%%%%%%%%%%%%%%%%%%%%%%%
%%%%%%%%PAGEBREAK%%%%%%%PAGEBREAK%%%%%%%%%
%%%%%%%%%%%%%%%%%%%%%%%%%%%%%%%%%%%%%%%%%%
%%%%%%%%%%%%%%%%PAGEBREAK%%%%%%%%%%%%%%%%%
%%%%%%%%%%%%%%%%%%%%%%%%%%%%%%%%%%%%%%%%%%
%%%%%%%%PAGEBREAK%%%%%%%PAGEBREAK%%%%%%%%%
%%%%%%%%%%%%%%%%%%%%%%%%%%%%%%%%%%%%%%%%%%
%%%%%%%%%%%%%%%%%%%%%%%%%%%%%%%%%%%%%%%%%%
%%%%%%%%%%%%%%%%%%%%%%%%%%%%%%%%%%%%%%%%%%
%%%%%%%%%%%%%%%%%%%%%%%%%%%%%%%%%%%%%%%%%%
%%%%%%%%PAGEBREAK%%%%%%%PAGEBREAK%%%%%%%%%
%%%%%%%%%%%%%%%%%%%%%%%%%%%%%%%%%%%%%%%%%%
%%%%%%%%%%%%%%%%PAGEBREAK%%%%%%%%%%%%%%%%%
%%%%%%%%%%%%%%%%%%%%%%%%%%%%%%%%%%%%%%%%%%
%%%%%%%%PAGEBREAK%%%%%%%PAGEBREAK%%%%%%%%%
%%%%%%%%%%%%%%%%%%%%%%%%%%%%%%%%%%%%%%%%%%
\begin{ekdosis}
  \begin{prose}
    \noindent
\app{\lem[wit={E,P,U1},alt={atas tasya}]{atas\skp{-}tasya}
  \rdg[wit={D,N1}]{ataḥ tasya}
  \rdg[wit={N2}]{ata tasya}
  \rdg[wit={U2}]{atā tasya}
  \rdg[wit={B,L}]{ataḥ}}
\app{\lem[wit={E,P,N1,U2}]{nāmamātraṃ}
  \rdg[wit={D,N2,U1}]{nāmamātre}
  \rdg[wit={B,L}]{nāma}}
\app{\lem[wit={ceteri}]{kathyate}
  \rdg[wit={D,N1,N2,U1}]{kathitaṃ}}/
%----------------------------
%prāṇāyāmastridhā ceti bahudhā prathamaṃ śrṛṇu।
%āsane prāṇasaṃyāme na śaktāḥ sukumārakāḥ।। 2।।
%mahāpuṇyaprabhāveṇa śakyate tu mahātmamā।
%iḍāṃ śaśiprabhāṃ dhyātvā yathāśakti tu kumbhayet।। 3।।
%mahājyotirmayo bhūtvā vāyupūrṇakalevaraḥ।
%śaktitrāsantu saṃtrāsya recayedvāyumarhitaḥ।। 4।।
%piṅgalāmarkavarṇaṃ tu tyajed dhyātvā śanaiḥ śanaiḥ।
%ayaṃ pataṅgakāṣṭhāgnipratyāsena punaḥ punaḥ।। 5।।
%kṛtvā kalevaraṃ śuddhaṃ kuryād yatnairmahātmanā।
%mano nivārya saṃsāre viṣayeṣu tathaiva ca।। 6।।
%manovikārān sarvāśca tyaktvā śūnyamayo bhavet।
%pratyāhāro bhavatyeṣu sarvanindācamatkṛtaḥ।। 7।।
%dhyānaṃ ca dvividhaṃ proktaṃ sthūlasūkṣmavibhedataḥ।
%sthūlaṃ mantramayaṃ viddhi sūkṣmantu mantravarjitam।। 8।।
%ityetatkathitaṃ sarvaṃ yogasaṅketamuttamam।
%adhunā cāṣṭakumbhasya lakṣaṇaṃ śrṛṇu kathyate।। 9।।
%śītkāraṃ sūryabhedaṃ ca uhyāyī śītalī tathā।
%bhastrikā bhrāmarī mūrcchā kevalī cāṣṭa kumbhakāḥ।। 10।।
%----------------------------
%pratyāhāraḥ pratyato   manaḥ saṃsārān nivartyātmani   sthāpyate// manomadhye ye vikārā  utpadyante/  tepi nivāraṇīyāḥ/  anekacamatkāriṇī         buddhir utpadyate/  sāṃgopāṃgaṃ  \E XX! this one?[P.47]
%pratyāhāraḥ kathyate   manaḥ saṃsārān nivṛtyātmanī    sthāpyate   manomadhye ye vikāraḥ utpadyaṃte   tepi nivāraṇīyāḥ   anekacamatkāriṇi         buddhir utpadyataraṃ  \P
%pratyāhāraḥ kathyate// manaḥ saṃsārān nivṛtyātmanī    sthāpyate// manomadhye ye vikārā  utpadyaṃte   tepi nivāraṇīyā    anekacamatkāriṇī         buddhir utpadyate/  sāgopyā// \B 
%pratyāhāraḥ kathyate   manaḥ saṃsārān nivṛttyātmanī   sthāpyate// manomadhye ye vikārā  utpadyaṃte   tepi nivāraṇīyā    anekacamatkāriṇi         buddhir utpadyate   sāgopyā//  \L %%%%0023.jpg
%pratyāhāraḥ kathyate// manaḥ saṃsārān nivṛtya ātmani  sthāpyate/  manomadhye ye vikārā  utpadyante/  tepi nivāraṇīyāḥ/  anekacamatkārakarakāraṇī buddhi  utpadyate   sāṃgopyāḥ/ \N1
%pratyāhāraḥ kathyate// manaḥ saṃsārān nivṛtya ātmani  sthāpyate// manomadhye ye vikārāḥ utpadyaṃte// tepi nivāraṇīyāḥ// anekacamatkārakāraṇī     buddhi  utpadyate// sāṃgopyāḥ// \D
%pratyāhāraḥ kathyate// manaḥ saṃsārān nivṛtya ātmani                                                        vāraṇīyāḥ// anekacamatkārakarakāraṇī buddhi  utpadyate   sāgopyāḥ/  \N2
%pratyāhāraḥ kathyate   manaḥ saṃsārān nivṛtyātmanī    sthāpyate   manomadhye ye vikārā  utpadyaṃte   tepi nivāraṇīyaḥ   anekacamatkāriṇī         buddhir utpadyate   sāgaupyā \U1 %%%286.jpg
%pratyāhāraḥ kathyate// manaḥ saṃsārān nivṛtyātmanī    sthāpyate// manomadhye ye vikārā  utpadyaṃte   tepi nivāraṇīyaḥ// anekacamatkāriṇī         buddhir utpadyate// sāgopyā// \U2
%-----------------------------
%Pratyāhāra [however] is taught. The mind is supposed to be turn away from the cyclic existence and caused to abide in the self. Changes within the mind arise, but they are kept off. The manyfold admirations the intellect generates are well hidden.    
%----------------------------
\note[type=source, labelb=257, lem={pratyāhāraḥ}]{SSP: pratyāhāram iti caitanyataraṅgānāṃ pratyāharaṇaṃ yathā nānāvikāragrasanotpannavikārasyāpi nivṛttiḥ nirbhātīti pratyāhāralakṣaṇam ||2.36||}
\note[type=source, labelb=258, lem={pratyāhāraḥ}]{Ysv (YK): ayaṃ pataṅgakāṣṭhāgnipratyāsena punaḥ punaḥ ||5|| kṛtvā kalevaraṃ śuddhaṃ kuryād yatnair mahātmanā | mano nivārya saṃsāre viṣayeṣu tathaiva ca ||6|| manovikārān sarvāś ca tyaktvā śūnyamayo bhavet | pratyāhāro bhavaty eṣu sarvanindācamatkṛtaḥ ||7||}
pratyāhāraḥ
\app{\lem[wit={ceteri}]{kathyate}
  \rdg[wit={E}]{pratyato}}/
manaḥ saṃsārā\skp{n-ni}\app{\lem[type=emendation, resp=egoscr, alt={nivṛtyātmani}]{\skm{n-ni}vṛtyātmani}
  \rdg[wit={B,L,P,U1,U2}]{\korr nivṛtyātmanī}
  \rdg[wit={E}]{nivartyātmani}
  \rdg[wit={D,N1,N2}]{nivṛtya ātmani}}
\app{\lem[wit={ceteri}]{sthāpyate}
  \rdg[wit={N2}]{\om}}/
manomadhye ye
\app{\lem[wit={ceteri}]{vikārā}
  \rdg[wit={P}]{vikāraḥ}
  \rdg[wit={D}]{vikārāḥ}
  \rdg[wit={N2}]{\om}}
\app{\lem[wit={ceteri}]{utpadyante}
  \rdg[wit={N2}]{\om}}/
 anekacama\skp{t-kā}\app{\lem[type=emendation, resp=egoscr, alt={°kārīṇi}]{\skm{t-kā}rīṇi}
   \rdg[wit={B,E,L,P,U1,U2}]{\korr kāriṇī}
   \rdg[wit={N1,N2}]{kārakarakāraṇī}
   \rdg[wit={D}]{kārakāraṇī}}
 \app{\lem[wit={ceteri},alt={buddhir}]{buddhi\skp{r-ut}}
   \rdg[wit={D,N1,N2}]{buddhi}
 }\app{\lem[wit={ceteri},alt={utpadyate}]{\skm{r-ut}padyate}
   \rdg[wit={E,B,D,U2}]{utpadyate |}
   \rdg[wit={P}]{utpadyataraṃ}}
 \app{\lem[type=emendation, resp=egoscr]{saṃgopyāḥ}
   \rdg[wit={D,N1}]{\korr sāṃgopyāḥ}
   \rdg[wit={N2}]{sāgopyāḥ}
   \rdg[wit={B,L,U2}]{sāgopyā}
   \rdg[wit={U1}]{sāgaupyā}
   \rdg[wit={E}]{sāṃgopāṃgaṃ}}/ 
%----------------------------
% dhyānaṃ ca bahutaraṃ prāg uktam/ tenātra       nocyate// \E XX! this one?
%                      prāg uktam  tenātra       nocyate  \P
% dhyānaṃ ca bahutaraṃ prāg uktam  tenātra       nocyate// \B 
% dhyānaṃ ca bahutaraṃ prāg uktam  tenātra       nocyate// \L %%%%0023.jpg
% dhyānaṃ ca bahutaraṃ      uktam  tena atra     nocyate/ \N1
% dhyānaṃ ca bahutaraṃ      uktaṃ  tena atra     nocyate// \D
% dhyānaṃ ca bahuttaraṃ     uktam  tenātra       nocyate// \N2
% dhyānaṃ    bahutaraṃ      uktaṃ  tena atra  na ucyate \U1 %%%286.jpg
% dhyānaṃ    bahutaraṃ prāg uktaṃ  tenātra       nocyate// \U2
%-----------------------------
%Dhyāna has been taught many times before. Because of that is not discussed here.
%-----------------------------
\note[type=source, labelb=259, lem={dhyānaṃ}]{SSP: atha dhyānam iti || asti kaś cana paramādvaitasya bhāvaḥ sa eva ātmeti yathā yadyat sphurati tattat svarūpam eveti bhāvayet sarvabhūteṣu samadṛṣṭiś ceti dhyānalakṣaṇam ||2.38||}
\note[type=source, labelb=260, lem={dhyānaṃ}]{Ysv (YK): dhyānan tu dvividhaṃ proktaṃ sthūlasūkṣmavibhedataḥ | sthūlaṃ mantramayaṃ viddhi sūkṣmantu mantravarjjitam | samādhir niścalā buddhiḥ śvāsocchvāsādivarjitaḥ |}
\app{\lem[wit={ceteri}]{dhyānaṃ}
  \rdg[wit={P}]{\om}}
\app{\lem[wit={ceteri}]{ca}
  \rdg[wit={P,U1,U2}]{\om}}
\app{\lem[wit={ceteri}]{bahutaraṃ}
  \rdg[wit={P}]{\om}}
\app{\lem[wit={B,E,L,P,U2},alt={prāg}]{prā\skp{g-u}}
  \rdg[wit={D,N1,N2,U1}]{\om}
}\app{\lem[wit={D,U1,U2},alt={uktaṃ}]{\skm{g-u}ktaṃ}
  \rdg[wit={E}]{uktam |}
  \rdg[wit={ceteri}]{uktam}}
\app{\lem[wit={ceteri}]{tenātra}
  \rdg[wit={D,N1,U1}]{tena atra}}
\app{\lem[wit={ceteri}]{nocyate}
  \rdg[wit={U1}]{na ucyate}}\dd{}
\end{prose}
\end{ekdosis}
%%%%%%%%%%%%%%
%%%%%%%%%%%%%%
%%%%%%%%%%%%%%
%%%%%%%%%%%%%
%%%%%%%%%%%%%%% 
\begin{ekdosis}
  \ekddiv{type=ed}
        \bigskip
  \centerline{\textrm{\small{[Internal and External Universe]}}}
      \bigskip
 \begin{prose}
%----------------------------
%idānīṃ piṃḍa-brahmāṃḍayor  aikyam asti    tasmāt   brahmāṇḍamadhye ye padārthās te pi     piṃḍamadhye santīti    kathyante/  \E %[P.48]
%idānīṃ piṃḍa-brahmāṃḍayor  aikyam asti    tasmād   brahmāṃḍamadhye ye padārthās te        piṃḍamadhye saṃti      kathyate    \P
%idānīṃ piṃḍa-brahmāṃḍayor  ekyam  asti//  tasmā    brahmāṃḍamadhye ye padārthās te        piṃḍamadhye sati       kathyate//  \B %%%%%%%%%%%%DSCN7168.jpg Z.2
%idānīṃ piṃḍa-brahmāṃḍayor  aikyam asti//  tasmāt   brahmāṃḍamadhye ye padārthās te        piṃḍamadhye saṃ        kathyaṃte// \L
%idānīṃ piḍa--brahmāḍayoḥ   aikyam asti//  tasmāt   brahmāṇḍamadhye ye padārthāḥ te pi     piṃḍamadhye saṃti// te kathyante// \N1
%idānīṃ piḍa--brahmāḍayoḥ   aikyam asti//  tasmāt   brahmāṇḍamadhye ye padārthāḥ te pi     piṃḍamadhye saṃti/  te kathyaṃte// \D
%idānīṃ piṇḍa-brahmāḍayoḥ   ekam   asti/   tasmānte brahmāṇḍamadhye ye padārthā  te pi     piṇḍamadhye saṃti   te kathyante// \N2
%idānīṃ piṇḍa-brahmāḍayor   aikam  asti    tasmāt   brahmāṇḍamadhye ye padārthā  sarve pi  piṇḍamadhye saṃti      kathyate    \U1
%idānīṃ piṇḍa-brahmāḍayor   aikam  asti//  tasmād   brahmāṇḍamadhye ye padārthās tanmadhye piṇḍamadhye sati       kathyaṃte// \U2
%-----------------------------
%Now there is the identity of the external universe and the body. Because of that, the objects which exist in the external universe are also in the body. They are taught.  
%----------------------------
   \note[type=source, labelb=261, lem={piṇḍa°}]{Ysv (PT): piṇḍabrahmāṇḍayor aikyaṃ śṛṇv idānīṃ prayatnataḥ | brahmāṇḍe santi ye cāṇḍāḥ piṇḍamadhye 'pi te sthitāḥ |}
   \note[type=testium, labelb=262, lem={piṇḍa°}]{SSP: piṇḍamadhye carācarau yo jānāti sa yogī piṇḍasaṃvittir bhavati||}
   \note[type=philcomm, labelb=263, lem={piṇḍa°}]{This section is not found in the quotes from the Ysv of the YK.}
   idānīṃ \app{\lem[wit={ceteri},alt={piṇḍa°}]{piṇḍa}
  \rdg[wit={D,N1}]{piḍa°}
}\app{\lem[wit={B,E,L,P},alt={brahmāṇḍayor}]{brahmāṇḍayo\skp{r-ai}}
  \rdg[wit={ceteri}]{°brahmāḍayoḥ}
}\app{\lem[wit={ceteri},alt={aikyam}]{\skm{r-ai}kya\skp{m-a}}
  \rdg[wit={B}]{ekyam}
  \rdg[wit={N2}]{ekam}
}\skm{m-a}sti/
\app{\lem[wit={ceteri},alt={tasmāt}]{tasmā\skp{t-bra}}
  \rdg[wit={B}]{tasmā}
  \rdg[wit={N2}]{tasmānte}}
\skm{t-bra}hmāṇḍamadhye ye
\app{\lem[wit={ceteri},alt={padārthās}]{padārthā\skp{s-te}}
  \rdg[wit={D,N1}]{padārthāḥ}
  \rdg[wit={N2,U1}]{padārthā}
}\app{\lem[wit={ceteri}, alt={te 'pi}]{\skm{s-te} 'pi}
  \rdg[wit={B,L,P}]{te}
  \rdg[wit={U1}]{sarve pi}
  \rdg[wit={U2}]{tanmadhye}}
piṇḍamadhye
\app{\lem[wit={ceteri}]{santi}
  \rdg[wit={E}]{santīti}
  \rdg[wit={B,U2}]{sati}
  \rdg[wit={L}]{saṃ°}}
\app{\lem[wit={D,N1,N2}]{te}
  \rdg[wit={ceteri}]{\om}}
\app{\lem[wit={ceteri}]{kathyante}
  \rdg[wit={B,P,U1}]{kathyate}}/ 
%----------------------------
%padas   tale        talaṃ           varttate/ pādopari talātalaṃ varttate/                         gulphayor mahātalaṃ   varttate/ jaṃghāmadhye sutalaṃ varttate/  jānumadhye   vitalaṃ varttate/ ūrvormadhye'talaṃ varttate// \E %[P.48]
%pādayos tele        talaṃ           varttate  pādopari talātalaṃ vartate    pādopari talaṃ vartate gulphayor mahātalaṃ  varttate                                   jānumadhye   vitalaṃ           ūrvormadhye atalaṃ \P
%pādayas talās       talaṃ           vartate// pādopari talātalaṃ vartate/                          gulphayor mahātalaṃ  vartate//  jaṃghāmadhye stutalaṃ vartate// jānubhyāṃ    vitalaṃ vartate// ūrvo madhye atalaṃ vartate//\7168.jpg Z.2
%pādayos talās       talaṃ           vartate// pādopari talātalaṃ vartate//                         gulphayor mahātalaṃ  vartate//  jaṃghāmadhye sutalaṃ varttate// jānubhyāṃ    vitalaṃ vartate// ūrvormadhye atalaṃ vartate// \L
%padayor aṃguṣṭale   talaṃ           varttate/ tādupari talātalaṃ varttate/                         gulpho parimahātalaṃ varttate/  jaṃghāmadhye sutalaṃ/           jānvomadhye  vitalaṃ/          ūrvormadhye atalaṃ//         \N1
%padayor aṃguṣṭale   talaṃ           varttate/ tādupari talātalaṃ varttate//                        gulpho parimahātalaṃ varttate// jaṃghāmadhye sutalaṃ//          jānvormadhye vitalaṃ//         ūrvormadhye atalaṃ//         \D
%padayor aṃguṣṭale   talaṃ           varttate  tādupari talātalaṃ varttate/                         gulpho parimahātalaṃ varttate   jaṃghāmadhye sutalaṃ/           jānvomadhye  vitalaṃ/          ūrvormadhye atalaṃ//         \N2
%pādayor aṃguṣṭatale talaṃ ca        vartate   taduparī talātalaṃ varttate                          gulpho parimahātalaṃ varttate   jaṃghāmadhye sutalaṃ            jānvormadhye vitalaṃ           ūrvormadhye atalaṃ           \U1
%pādayoṃguṣṭatale    mūlaṃ rasātalāt vartate// pādopari talātalaṃ varttate//                        gulphayor  mahātalaṃ varttate// jaghāmadhye  sutalaṃ vartate//  jānumadhye   vitalaṃ//         ūrvormadhye atalaṃ//         \U2
%-----------------------------
%Talam exists at the base of the big toe[s] of the feet. On top of the feet exists Talātala. Mahātala exists at the two ankles. Sutala exists in the center of the lower part of the leg between ankle and knee. Vitala exists in the middle of the knee. Atala exists in the middle of the two thighs.   
%----------------------------
\note[type=source, labelb=264, lem={talaṃ}]{Ysv (PT): talaṃ pādāṅguṣṭhatale tasyopari talātalam | mahātalaṃ gulphayor madhye gulphopari rasātalam | sutalaṃ jaṅghayor madhye vitalaṃ jānumadhyakam | ūrvormadhye 'talaṃ proktaṃ saptapātālam īritam | talaṃ talātalañ ceti mahātalarasātalam | saptapātālam etat tu sutalaṃ vitalātalam |}
\note[type=testium, labelb=265, lem={talaṃ}]{SSP 3.2: kūrmaḥ pādatale vasati pātālaṃ pādāṅguṣṭhe talātalam aṅguṣṭhāgre mahātalaṃ pṛṣṭhe rasātalaṃ hulphe sutalaṃ jaṅghāyāṃ vitalaṃ jānvoḥ atalam urvor evaṃ saptapātālaṃ rudradevatādhipatye tiṣṭhati piṇḍamadhye krodharūpī bhāvaḥ sa eva kālāgnirudraḥ mahātalaṃ pādapṛṣthe}
\app{\lem[wit={ceteri},alt={pādayor}]{pādayo\skp{r-a}}
  \rdg[wit={E}]{padas}
  \rdg[wit={P,L}]{pādayos}
  \rdg[wit={B}]{pādayas}
  \rdg[wit={U2}]{pādayo°}
}\app{\lem[type=emendation, resp=egoscr,alt={aṅguṣṭatale}]{\skm{r-a}ṅguṣṭatale}
  \rdg[wit={U1}]{\korr aṃguṣṭatale}
  \rdg[wit={D,N1,N2}]{aṃguṣṭale}
  \rdg[wit={U2}]{°ṃguṣṭatale}
  \rdg[wit={B,L}]{tālas}
  \rdg[wit={P}]{tele}
  \rdg[wit={E}]{tale}}
\app{\lem[wit={ceteri}]{talaṃ}
  \rdg[wit={U1}]{talaṃ ca}
  \rdg[wit={U2}]{mūlaṃ rasātalāt}}
vartate/
\app{\lem[type=emendation, resp=egoscr]{tadupari}
  \rdg[wit={U1}]{\korr taduparī}
  \rdg[wit={D,N1,N2}]{tādupari}
  \rdg[wit={B,E,L,P,U2}]{pādopari}}
talātalaṃ
\app{\lem[wit={ceteri}]{vartate}
  \rdg[wit={P}]{vartate | pādopari talaṃ vartate}}/
\app{\lem[wit={B,E,L,P,U2},alt={gulphayor}]{gulphayo\skp{r-ma}}
  \rdg[wit={D,N1,N2,U1}]{gulpho}
}\app{\lem[wit={B,E,L,P,U2},alt={mahātalaṃ}]{\skm{r-m}ahātalaṃ}
  \rdg[wit={D,N1,N2,U1}]{parimahātalaṃ}}
vartate/
\end{prose}
\end{ekdosis}
\ekdpb*{}
%%%%%%%%%%%%%%%%%%%%%%%%%%%%%%%%%%%%%%%%%%
%%%%%%%%%%%%%%%%%%%%%%%%%%%%%%%%%%%%%%%%%%
%%%%%%%%PAGEBREAK%%%%%%%PAGEBREAK%%%%%%%%%
%%%%%%%%%%%%%%%%%%%%%%%%%%%%%%%%%%%%%%%%%%
%%%%%%%%%%%%%%%%PAGEBREAK%%%%%%%%%%%%%%%%%
%%%%%%%%%%%%%%%%%%%%%%%%%%%%%%%%%%%%%%%%%%
%%%%%%%%PAGEBREAK%%%%%%%PAGEBREAK%%%%%%%%%
%%%%%%%%%%%%%%%%%%%%%%%%%%%%%%%%%%%%%%%%%%
%%%%%%%%%%%%%%%%%%%%%%%%%%%%%%%%%%%%%%%%%%
%%%%%%%%%%%%%%%%%%%%%%%%%%%%%%%%%%%%%%%%%%
%%%%%%%%%%%%%%%%%%%%%%%%%%%%%%%%%%%%%%%%%%
%%%%%%%%PAGEBREAK%%%%%%%PAGEBREAK%%%%%%%%%
%%%%%%%%%%%%%%%%%%%%%%%%%%%%%%%%%%%%%%%%%%
%%%%%%%%%%%%%%%%PAGEBREAK%%%%%%%%%%%%%%%%%
%%%%%%%%%%%%%%%%%%%%%%%%%%%%%%%%%%%%%%%%%%
%%%%%%%%PAGEBREAK%%%%%%%PAGEBREAK%%%%%%%%%
%%%%%%%%%%%%%%%%%%%%%%%%%%%%%%%%%%%%%%%%%%
%%%%%%%%%%%%%%%%%%%%%%%%%%%%%%%%%%%%%%%%%%
%%%%%%%%%%%%%%%%%%%%%%%%%%%%%%%%%%%%%%%%%%
%%%%%%%%%%%%%%%%%%%%%%%%%%%%%%%%%%%%%%%%%%
%%%%%%%%PAGEBREAK%%%%%%%PAGEBREAK%%%%%%%%%
%%%%%%%%%%%%%%%%%%%%%%%%%%%%%%%%%%%%%%%%%%
%%%%%%%%%%%%%%%%PAGEBREAK%%%%%%%%%%%%%%%%%
%%%%%%%%%%%%%%%%%%%%%%%%%%%%%%%%%%%%%%%%%%
%%%%%%%%PAGEBREAK%%%%%%%PAGEBREAK%%%%%%%%%
%%%%%%%%%%%%%%%%%%%%%%%%%%%%%%%%%%%%%%%%%%
%%%%%%%%%%%%%%%%%%%%%%%%%%%%%%%%%%%%%%%%%%
  \begin{ekdosis}
    \begin{prose}
      \noindent
    \app{\lem[wit={ceteri},alt={jaṅghā°}]{jaṅghā}
  \rdg[wit={U2}]{jaghā°}
  \rdg[wit={P}]{\om}}madhye
\app{\lem[wit={ceteri}]{sutalaṃ}
  \rdg[wit={B}]{stutalaṃ}
  \rdg[wit={P}]{\om}}
\app{\lem[wit={B,E,L,U2}]{vartate}
  \rdg[wit={ceteri}]{\om}}/
\app{\lem[wit={D,U1}]{jānvormadhye}
  \rdg[wit={N1,N2}]{jānvomadhye}
  \rdg[wit={E,P,U2}]{jānumadhye}
  \rdg[wit={B,L}]{jānubhyāṃ}}
vitalaṃ
\app{\lem[wit={E,B,L}]{vartate}
  \rdg[wit={ceteri}]{\om}}/ 
ūrvormadhye
\app{\lem[wit={E}]{'talaṃ}
  \rdg[wit={ceteri}]{atalaṃ}} 
\app{\lem[wit={E,L,B}]{vartate}
  \rdg[wit={ceteri}]{\om}}/
\end{prose}
\end{ekdosis}
\begin{ekdosis}
    \bigskip
    \centerline{\textrm{\small{[Triad of Worlds]}}}
    \bigskip
     \begin{prose}
%----------------------------
%idānīṃ                    śarīramadhye lokatrayaṃ kathyate/  mūlādhāre bhūrlokaḥ/  liṃgāgre  bhuvarlokaḥ/  liṃgamadhye  svarlokaḥ//    \E
%idānīṃ                    piṃḍamadhye  lokatrayaṃ kathyate   mūlādhāre bhūrlokaḥ   liṃgāgre  bhuvarlokaḥ   liṃgamūle    svarlokaḥ      \P
%idānīṃ                    piḍopiri     lokatrayaṃ kathyate// mūlādhāre bhūrlokaḥ   liṃgāgre  bhuvarloka----liṃgamadhye  svarlokaḥ//    \B
%idānīṃ                    piṃḍopari    lokatrayaṃ kathyate// mūlādhāre bhūrlokaḥ// liṃgāgre  bhuvarloka----liṃgamadhye  svarlokaḥ//   \L
%idānīṃ                    piṃḍamadhye  lokatrayaṃ kathyate/  mūlādhāre bhūrlokaḥ/  liṃgamūle                            svarlokaḥ     \N1
%idānīṃ                    piṃḍamadhye  lokatrayaṃ kathyate// mūlādhāre bhūrlokaḥ// liṃgāgre  bhuvarlokaḥ// liṃgamadhye  svarlokaḥ//   \D
%idānīṃ                    piṃḍamadhye  lokatrayaṃ kathyate/  mūlādhāre bhūrlokaḥ   liṃgamūle                            svargalokaḥ// \N2
%idānīṃ upari tataṃ  lokaṃ piṃḍamadhye  lokatrayaṃ kathyate   mūlādhāre bhūrlokaḥ   liṃgāgre  bhuvarlokaḥ   liṃgamūle    svaravarlokaḥ \U1
%idānīṃ                    piṃḍamadhye  lokatrayaṃ kathyate// mūlādhāre bhūrlokaḥ// liṃgāgre  bhuvarlokaḥ// liṃgamūle    svarlokaḥ//   \U2
%-----------------------------
%Now the threefold world within the body is taught. The earthen world is situated at the Root-cotainer (\textit{mūladhāra}). The air world is at the root of the gender. In the center of the gender is the heavenly world. 
%----------------------------
\note[type=source, labelb=266, lem={piṇḍamadhye}]{Ysv\textsuperscript{PT}: idānīṃ piṇḍamadhye tu saptalokaṃ śṛṇu priye | mūlādhāre tu bhūrloko liṅgāgre tu bhuvas tataḥ | svarloko liṅgamūle tu merumūle mahas tathā |}
\note[type=testium, labelb=267, lem={bhūrlokaḥ}]{SSP 3.3: bhūrlokaṃ guhyasthāne bhuvarlokaṃ liṅgasthāne svarlokaṃ nābhisthāne evaṃ lokatraye indro devatā piṇḍamadhye sarvendriyaniyāmakaḥ sa evendraḥ||}
\app{\lem[wit={ceteri}]{idānīṃ}
  \rdg[wit={U1}]{idānīṃ upati tataṃ lokaṃ}}
\app{\lem[wit={ceteri}]{piṇḍamadhye}
  \rdg[wit={L}]{piṃḍopari}
  \rdg[wit={B}]{piḍopiri}
  \rdg[wit={E}]{śarīramadhye}}
lokatrayaṃ kathyate/ \\
mūlādhāre bhūrlokaḥ/
\app{\lem[wit={ceteri}]{liṅgāgre}
  \rdg[wit={N1,N2}]{liṃgamūle}}
\app{\lem[wit={D,E,P,U1,U2}]{bhuvarlokaḥ}
  \rdg[wit={B,L}]{bhuvarloka°}
  \rdg[wit={N1,N2}]{\om}}/
\app{\lem[wit={ceteri}]{liṅgamadhye}
  \rdg[wit={P,U1,U2}]{liṃgamūle}
  \rdg[wit={N1,N2}]{\om}}
\app{\lem[wit={ceteri}]{svarlokaḥ}
  \rdg[wit={N2}]{svargalokaḥ}
  \rdg[wit={U1}]{svaravarlokaḥ}}\dd{}
\end{prose}
\end{ekdosis}
\begin{ekdosis}
    \bigskip
    \centerline{\textrm{\small{[Tetrad of Worlds]}}}
    \bigskip
    \begin{prose}
%----------------------------
%idānīm   uparitanaṃ lokacatuṣka      kathyate/  pṛṣṭhadaṃḍāṃkure  maharlokaḥ/   daṇḍacchidramadhye janalokaḥ/  taddaṇḍanāḍīmadhye   tapolokaḥ/  daṇḍamalamadhye   satyalokaḥ/ \E
%idānīm   uparitanu--lokacatuṣkaṃ     kathyate   pṛṣṭhadaṃḍākūre   maharlokaḥ    daṃḍaschidramadhye janalokaḥ   taddaṃḍanālimadhye   tapolokaḥ   daṃḍakamalamadhye satyalokaḥ \P
%idānīm   uparitanu--lokaḥ catuṣṭayaṃ kathyate// daṃḍaṣṭaṭheṃskure maharlokā/    daṇḍachidramadhye  janaloka    taddaṃḍanālikāmadhye ..polokaḥ   daṇḍakamalamadhye satyalokaḥ// \B
%idānīm   uparitana--lokaḥ catuṣṭayaṃ kathyate// daṃḍaṣṭaṭheṃkure  maharlokaḥ/   daṇḍachidramadhye  janaloka    taddaṃḍatālikāmadhye tapolokaḥ   daṇḍakamalamadhye satyalokaḥ// \L
%idānīṃ   uparijanaṃ lokacatuṣkaṃ     kathyate/  pṛṣṭhadaṃḍāṃkure  maharllokaḥ/  daṇḍacchidramadhye janalokaḥ/  taddaṇḍanālī \om                                               \N1!!!!!!!!!!!!!!!important omission stemmapoint S.11 verso
%idānīṃ// uparitanaṃ lokacatuṣkaṃ     kathyate// pṛṣṭhadaṃḍāṃkure  maharlokaḥ    daṇḍachidramadhye  janalokaḥ// taddaṇḍanālamadhye   tapolokaḥ// daṇḍakamalamadhye satyalokaḥ// \D
%idānīṃ   uparijanaṃ lokacatuṣkaṃ     kathyate// pṛṣṭhadaṃḍākūle   maharllokaḥ/  uchidramadhye      janalokaḥ/  taddaṇḍanālī                                              \om  \N2 !!!!!!!!!!!!!!!!!!important omission stemmapoint
%idānīṃ   uparitanaṃ lokaṃ catuṣkaṃ   kathyate   pṛṣṭhadaṃḍāṃkure  maharlokaḥ    daṃḍasthitamadhye  janalokaḥ   taddaṇḍanāḍīmadhye   tapolokaḥ   daṇḍamalamadhye   satyalokaḥ \U1
%idānīṃ   uparitana--lokacatuṣkaṃ     kathyate// pṛṣṭhadaṃḍāṃkure  maharlokaḥ//  daṃḍachidramadhye  janalokaḥ// daṇḍanālimadhye      tapolokaḥ// daṇḍakamalamadhye satyalokaḥ// \U2
%-----------------------------
%Now the quadruplet of worlds will be taught. The great world is at the shoot of the staff of the spine. The world of men is in the centre of the cavity of the spine. In the centre of the tube of that spine is the world of heat?. In the center of the lotus of the spine
%-----------------------------
idānīṃ
\app{\lem[wit={D,E,U1}]{uparitanaṃ}
  \rdg[wit={L,U2}]{uparitana°}
  \rdg[wit={N1,N2}]{uparijanaṃ}
  \rdg[wit={P,B}]{uparitanu°}}
\app{\lem[wit={P,D,N1,N2,U2}]{lokacatuṣkaṃ}
  \rdg[wit={E}]{lokacatuṣka}
  \rdg[wit={B,L}]{lokaḥ catuṣṭayaṃ}
  \rdg[wit={U1}]{lokaṃ catuṣkaṃ}}
kathyate/ \\
\note[type=testium, labelb=268, lem={lokacatuṣkaṃ}]{SSP 3.4: daṇḍāṅkure maharlokaḥ daṇḍakuhare jano lokaḥ daṇḍanāle tapo lokaḥ mūlakamale satyalokaḥ evaṃ lokacatuṣṭaye brahmādidevatā piṇḍamadhye anekamānābhimānasvarūpī tiṣṭhati||}
\note[type=source, labelb=269, lem={lokacatuṣkaṃ}]{Ysv\textsuperscript{PT}: merucchidre janoloko merunāḍyāṃ tapas tathā | kamale marttyalokas tu iti lokaḥ pṛthak pṛthak | bhūrbhuvaḥsvarmahaś ceti janaś caiva tapas tathā | saptamaḥ satyalokas tu saptaloka iti smṛtaḥ | saptalokais tu pātālair bhuvanāni caturdaśa |}
\app{\lem[wit={ceteri}]{pṛṣṭhadaṇḍāṅkure}
  \rdg[wit={N2}]{pṛṣṭhadaṃḍākūle}
  \rdg[wit={P}]{pṛṣṭhadaṃḍākūre}
  \rdg[wit={B}]{daṃḍaṣṭaṭheṃskure}
  \rdg[wit={L}]{daṃḍaṣṭaṭheṃkure}}
 maha\skp{r-lo}\app{\lem[wit={ceteri},alt={°lokaḥ}]{\skm{r-lo}kaḥ}
   \rdg[wit={B}]{°lokā}}/
 \app{\lem[wit={ceteri}, alt={daṇḍachidra°}]{daṇḍachidra}
   \rdg[wit={P}]{daṃḍaschidra}
   \rdg[wit={U1}]{daṃḍasthita}
   \rdg[wit={U2}]{uchidra}}madhye
 \app{\lem[wit={ceteri}]{janalokaḥ}
   \rdg[wit={B,L}]{janaloka}}/
\app{\lem[wit={ceteri},alt={taddaṇḍa°}]{taddaṇḍa}
  \rdg[wit={U2}]{daṇḍa°}
}\app{\lem[wit={E,U1},alt={°nāḍīmadhye}]{nāḍīmadhye}
  \rdg[wit={P,U2}]{nālimadhye}
  \rdg[wit={B}]{nālikāmadhye}
  \rdg[wit={L}]{tālikāmadhye}
  \rdg[wit={B}]{nālamadhye}
  \rdg[wit={N1,N2}]{nālī}}
\note[type=philcomm, labelb=270, lem={nāḍīmadhye}]{At this point of the text a huge gab of approximately 25\% of the full text starts in the two important and most reliable witnesses of \textit{Yogatattvabindu}. The two Nepalese manuscripts N\textsubscript{1} and N\textsubscript{2} indicate a large gap in their template, which makes it more than clear that N\textsubscript{1} and N\textsubscript{2} stematically belong closely together. They are undoubtedly either direct copies of each other or copies of the same template. The omissions of the reading of N\textsubscript{1} and N\textsubscript{2} will not be recorded in the apparatus until after their gap.}
\app{\lem[wit={ceteri}]{tapolokaḥ}
  \rdg[wit={B}]{polokaḥ}}/ \\
 daṇḍa\app{\lem[wit={ceteri},alt={°kamalamadhye}]{kamalamadhye}
   \rdg[wit={E,U1}]{°malamadhye}}
 satyalokaḥ/ 
 \end{prose}
\end{ekdosis}
\ekdpb*{}
%%%%%%%%%%%%%%%%%%%%%%%%%%%%%%%%%%%%%%%%%%
%%%%%%%%%%%%%%%%%%%%%%%%%%%%%%%%%%%%%%%%%%
%%%%%%%%PAGEBREAK%%%%%%%PAGEBREAK%%%%%%%%%
%%%%%%%%%%%%%%%%%%%%%%%%%%%%%%%%%%%%%%%%%%
%%%%%%%%%%%%%%%%PAGEBREAK%%%%%%%%%%%%%%%%%
%%%%%%%%%%%%%%%%%%%%%%%%%%%%%%%%%%%%%%%%%%
%%%%%%%%PAGEBREAK%%%%%%%PAGEBREAK%%%%%%%%%
%%%%%%%%%%%%%%%%%%%%%%%%%%%%%%%%%%%%%%%%%%
%%%%%%%%%%%%%%%%%%%%%%%%%%%%%%%%%%%%%%%%%%
%%%%%%%%%%%%%%%%%%%%%%%%%%%%%%%%%%%%%%%%%%
%%%%%%%%%%%%%%%%%%%%%%%%%%%%%%%%%%%%%%%%%%
%%%%%%%%PAGEBREAK%%%%%%%PAGEBREAK%%%%%%%%%
%%%%%%%%%%%%%%%%%%%%%%%%%%%%%%%%%%%%%%%%%%
%%%%%%%%%%%%%%%%PAGEBREAK%%%%%%%%%%%%%%%%%
%%%%%%%%%%%%%%%%%%%%%%%%%%%%%%%%%%%%%%%%%%
%%%%%%%%PAGEBREAK%%%%%%%PAGEBREAK%%%%%%%%%
%%%%%%%%%%%%%%%%%%%%%%%%%%%%%%%%%%%%%%%%%%
%%%%%%%%%%%%%%%%%%%%%%%%%%%%%%%%%%%%%%%%%%
%%%%%%%%%%%%%%%%%%%%%%%%%%%%%%%%%%%%%%%%%%
%%%%%%%%%%%%%%%%%%%%%%%%%%%%%%%%%%%%%%%%%%
%%%%%%%%PAGEBREAK%%%%%%%PAGEBREAK%%%%%%%%%
%%%%%%%%%%%%%%%%%%%%%%%%%%%%%%%%%%%%%%%%%%
%%%%%%%%%%%%%%%%PAGEBREAK%%%%%%%%%%%%%%%%%
%%%%%%%%%%%%%%%%%%%%%%%%%%%%%%%%%%%%%%%%%%
%%%%%%%%PAGEBREAK%%%%%%%PAGEBREAK%%%%%%%%%
%%%%%%%%%%%%%%%%%%%%%%%%%%%%%%%%%%%%%%%%%%
\begin{ekdosis}
    \bigskip
    \centerline{\textrm{\small{[Four Lords of the Worlds]}}}
    \bigskip
    \begin{prose}
      \noindent
%----------------------------
%atha brahmāṇḍamadhye caturdaśa-lokāni sthānāni tānyapi piṃḍe varttante// \E
%atha brahmāṇḍamadhye caturdaśa-lokāsthānāni    tānyapi piḍe  varttate  \P
%atha brahmāṇḍamadhye caturdaśa-lokasthānānī    tānyapi piṃḍo vartate... \B
%atha brahmāṇḍamadhye caturdaśa-lokasthānānī    tānyapi piṃḍe vartate... \L
%\om                                                                 \N1
%atha brahmāṇḍamadhye catvāro   lokasvāminaḥ//  te pi piṃḍamadhye varttate \D %%%p. 14 recto
%\om                                                                 \N2
%atha brahmāṇḍamadhye catvāro   lokāḥ svāminaḥ  te pi piṃḍamadhye vartate \U1
%atha brahmāṇḍamadhye caturddaśalokāḥ stānāni// tānyapi piṃḍe vartate// \U2 %%418.jpg
%-----------------------------
%Now the locations of the fourteen worlds within the universe exist in the body.
%Now the four lords of the worlds of the external universe also exist in the internal universe.       
%-----------------------------
\note[type=source, labelb=270, lem={catvāro}]{Ysv\textsuperscript{PT}: atha brahmāṇḍamadhyasthāś catvāro lokapālakāḥ |}
\note[type=philcomm, labelb=271 , lem={catvaro}]{Only the reading of witness D and U\textsubscript{1} is plausible and has to be considered as \textit{lectio dificilior}. This is confirmed by the reading of the source text, the Ysv\textsuperscript{PT} introducing the \textit{lokapālakāḥ} which become rewritten by Rāmacandra to \textit{lokasvāminah̤}. In the transmission of the text within the E,N,L,P and U\textsubscript{2}-group this subject has not been properly understood and in order to fix it the passage was rewritten, which probably resulted in the introduction of the \textit{caturdaśalokāsthānāni}.}
atha brahmāṇḍamadhye
\app{\lem[wit={D,U1}]{catvāro}
  \rdg[wit={ceteri}]{caturdaśa°}}
\app{\lem[wit={D}]{lokasvāminaḥ}
  \rdg[wit={U1}]{lokāḥ svāminaḥ}
  \rdg[wit={P,B,L}]{°lokāsthānāni}
  \rdg[wit={U2}]{°lokāḥ stānāni}
  \rdg[wit={E}]{°lokāni sthānāni}}/
\app{\lem[wit={E,U1}]{te 'pi}
  \rdg[wit={ceteri}]{tānyapi}}
\app{\lem[wit={E,U1}]{piṇḍamadhye}
  \rdg[wit={B,E,L,U2}]{piṇḍe}
  \rdg[wit={P}]{piḍe}}
\app{\lem[wit={E}]{vartante}
  \rdg[wit={ceteri}]{vartate}}/
%----------------------------
%śarīramadhye  dvau kukṣī  dve sakthinī   vakṣaḥsthalaṃ   kaṃṭhamūlaṃ    kaṃṭhamadhyaṃ laṃbikāmūlaṃ   tāludvāraṃ tālumadhyaṃ     lalāṭamadhye   śṛṃgāṭikā    kapolamadhye   kamalinīmadhye   brahmaraṃdhra             kamalinya---strikūṭasthānam/ \E
%śarīramadhye  dvau kukṣī  dve sakṭhi??nī vakṣaḥ schalaṃ  kaṃṭhamūlaṃ    kaṃṭhamadhyaḥ laṃbikāmūlaṃ   tāludvāraṃ tālumadhye      lalāṭamadhyaṃ  śṛṃgāṭikā    kapolamadhye   kamalinīmadhye   brahmaraṃdhraṃ   ūrddhvaṃ kamalinyā   strikūṭasthānam \P %%%7658.jpg
%śarīramadhye//dvau kukṣau dve sakṭhinī   vakṣaḥsthalaṃ   kaṃṭhamūlaṃ    kamardhye     laṃbikāmūlaṃ   tāludvāraṃ tālumadhyaṃ     lalāṭamadhyaṃ//śṛṃgāṭikā//  kapolamadhye// kamalinīmadhyaṃ  brahmaraṃdhraṃ            kamalīnyāṃ  strikūṭasthānam// \B
%śarīramadhye  dvau kukṣau dve sakthinī   vakṣasthalaṃ    kaṃṭhamūle     kaṃṭhamadhyaṃ laṃbikāmūlaṃ   tāludvāraṃ tālamadhyaṃ     lalāṭamadhyaṃ  śṛṃgāṭikā    karālamadhye   kamalinīmadhyaṃ  brahmaraṃdhraṃ   ūrdhvaṃ  kamalīnyā   trikūṭasthānam... \L
%\om                                                                 \N1
%śarīramadhye  dvau kukṣīnau vartatte//   vakṣasthale//   kaṃṭhasya mūle kaṃṭhamadhye  laṃbikāyāmūle/ tāludvāre tālumadhye       lalāṭe//       śṛṃgāṭikāyāṃ kapolamadhye// kamalinīmadhye// brahmaraṃdhre//  ūrdhva---kamalinyaḥ//trikūṭasthāne// saptapātāle//\D
%\om                                                                 \N2
%śarīramadhye  dvau kukṣīṇau    varttate  vakṣasthale     kaṃṭhasya mūle kaṃṭhamadhye   laṃbikāyāmūle tāludvāre tālumadhye       lalāṭe         śṛṃgāṭikāyāṃ kapolamadhye   kamalinīmadhye   brahmaraṃdhre    urdhva---kamalinyaḥ  trikūṭasthāne... \U1  %%%%287.jpg
%śarīramadhye  dvau kukṣī  dve sakthinī// vakṣassthalaṃ/  kaṃṭhamūle//   kaṃṭhamadhyaḥ//laṃbikāmūlaṃ//tāludvāraṃ// tālumadhyaṃ// lalāṭamadhyaṃ// śrṛṃgāṭikā  kapolamadhye// kamalinīmadhye// brahmaraṃdhraṃ// urdhva---kamalinyās  trikūṭasthānam// \U2
%-----------------------------
%Within the body in the two cavities (1), within the two thighs (2), at the location of the chest (3), at the root of throat (4), in the center of the throat (5) at the root of the uvula (6) at the entrance of the palate (7) at the forehead (8) at the crossroad of the center of the cheecks (9), at the center of the lotuspond (10?), at the aperture of Brahman (11), at the place of the three peaks above the lotusses (14), in the seven hells (21) \ldots
%----------------------------
\note[type=source, labelb=272, lem={śarīramadhye}]{Ysv\textsuperscript{PT}: piṇḍamadhye tu tān jñātvā sarvasiddhīśvaro bhavet | indro brahmā viṣṇurīśaścatvāraś cātmadevatāḥ | mūlādhāre catuṣpatre gajārūḍho mahān iti | sṛṣṭikarttā ca tatraiva svādhiṣṭhāne mahān hariḥ | maṇipūre śūlapāṇiraṣṭasiddhīśvaro mahān | tāludvāre tālumadhye lalāṭe vakṣakaṇṭhake | śṛṅgāṭikā kapāle ca lambikā brahmarandhrake | navacakram ūrddhvacakrañ ca trikūṭety ekaviṃśatiḥ | brahmāṇḍāni vasantīti jñātavyāni prayatnataḥ |}
\note[type=source, labelb=273, lem={kukṣau}]{SSP 3.5: viṣṇulokaḥ kukṣau tiṣṭhati tatra viṣṇur devatā piṇḍamadhye aṇekavyāpārakārako bhavati| hṛdaye rudralokaḥ tatra rudro devatā piṇḍamadhye ugrasvarūpī tiṣṭhati | vakṣaḥsthale īśvaralokaḥ tatra īśvaro devatā piṇḍamadhye tṛptisvarūpī tiṣṭhati | kaṇṭhamadhye nīlakaṇṭho lokas tatra nīlakaṇṭho devatā piṇḍamadhye nityaṃ tiṣṭhati | tāludvāre śivalokas tatra śivo devatā piṇḍamadhye 'nupamasvarūpī tiṣṭhati| lambikāmūle bhairavalokas tatra bhairavo devatā piṇḍamadhye sarvottamasvarūpī tiṣṭhati | lalāṭamadhye 'nādilokas tatrānādidevatā piṇḍamadhye ānandaparāhantāsvarūpī tiṣṭhati | śṛṅgāre kulalokas tatra kuleśvaro devatā piṇḍamadhye ānandasvarūpī tiṣṭhati | śaṃkhamadhye nalinīsthāne akuleśvaro devatā piṇḍamadhye nirabhimānāvasthā tiṣṭhati | brahmarandhre parabrahmalokas tatra parabrahma devatā piṇḍamadhye paripūrṇadaśā tiṣṭhati | ūrdhvakamale parāparalokas tatra parameśvaro devatā piṇḍamadhye parāparabhāvas tiṣṭhati | trikūṭasthāne śaktilokas tatra parāśaktir devatā sarvasaṃ sarvakartṛtvāvasthā tiṣṭhati| evaṃ piṇḍamadhye saptapātālasahitaikaviṃśatibrahmāṇḍasthānavicāraḥ |}
śarīramadhye
\app{\lem[type=conjecture, resp=egoscr]{dvāyoḥ kukṣayoḥ}
  \rdg[wit={E,P,U2}]{\conj dvau kukṣī}
  \rdg[wit={B,L}]{dvau kukṣau}
  \rdg[wit={D}]{dvau kukṣīnau}
  \rdg[wit={U1}]{dvau kukṣīṇau}}\dd{}
\app{\lem[type=conjecture, resp=egoscr]{dvāyoḥ sakthinoḥ}
  \rdg[wit={E,L,U2}]{\conj dve sakthinī}
  \rdg[wit={P,B}]{dve sakṭhinī}
  \rdg[wit={D,U1}]{vartate}}\dd{}
\note[type=philcomm, labelb=273, lem={śarīramadhye}]{This passage which lists the 21 locations is very problematic. The accusatives preserved in E,N,L,P and U\textsubscript{2} are clearly an attempt to fix the text according to the rewriting of the previous \textit{caturdaśalokāsthānāni}-sentence, which is seen also in the limitation of the elements of the list in those witnesses from 21 to just 14. It is more likely that the locatives in D and \textsubscript{1} are original. Since the text promises to account for 21 locations which all seem to have been listed as locatives, my best guess is to conjecture two more locatives for the cavities (\textit{dvāyoḥ kukṣau}) and for the two thighs (\textit{dvāyoḥ sakthinoḥ}) in order to arrive at a grammatically correct text and to accept the reading for the final seven locations given as \textit{saptapālāle} which is only preserved in witness D.}
\app{\lem[type=emendation, resp=egoscr]{vakṣaḥsthale}
  \rdg[wit={D,U1}]{\korr vakṣasthale}
  \rdg[wit={E,B}]{vakṣaḥ sthalaṃ}
  \rdg[wit={P}]{vakṣaḥschalaṃ}
  \rdg[wit={U2}]{vakṣassthalaṃ}}
\app{\lem[wit={L,U2}]{kaṇṭhamūle}
  \rdg[wit={E,P,B}]{kaṃṭhamūlaṃ}
  \rdg[wit={D,U1}]{kaṃṭhasya mūle}}\dd{}
\app{\lem[wit={D,U1}]{kaṇṭhamadhye}
  \rdg[wit={B}]{kamardhye}
  \rdg[wit={E,L}]{kaṃṭhamadhyaṃ}
  \rdg[wit={P,U2}]{kaṃṭhamadhyaḥ}}\d{}
\app{\lem[type=emendation, resp=egoscr]{lambikāmūle}
  \rdg[wit={D,U1}]{\korr laṃbikāyā mūle} %laṃbikāyām mūle = locatives
  \rdg[wit={ceteri}]{laṃbikāmūlaṃ}}\dd{}
\app{\lem[wit={D,U1}]{tāludvāre}
  \rdg[wit={ceteri}]{tāludvāraṃ}}\dd{}
\app{\lem[wit={D,U1}]{tālumadhye}
  \rdg[wit={ceteri}]{tālumadhyaṃ}}\dd{}
\app{\lem[wit={D,U1}]{lalāṭe}
  \rdg[wit={E}]{lalāṭamadhye}
  \rdg[wit={ceteri}]{lalāṭamadhyṃ}}\dd{}
\end{prose}
\end{ekdosis}
\ekdpb*{}
%%
\begin{ekdosis}
    \begin{prose}
\app{\lem[wit={D,U1}]{śṛṅgāṭikāyāṃ}
  \rdg[wit={ceteri}]{śṛṃgāṭikā}}
\app{\lem[type=conjecture, resp=egoscr]{kapālamadhye}
  \rdg[wit={L}]{\conj karālamadhye}
  \rdg[wit={ceteri}]{kapolamadhye}}
\app{\lem[wit={ceteri}]{kapolamadhye}
  \rdg[wit={L}]{karāla}}\dd{}
\app{\lem[wit={ceteri}]{kamalinīmadhye}
  \rdg[wit={B,L}]{kamalinīmadhyaṃ}}\dd{}
\app{\lem[wit={D,U1}]{brahmarandhre}
  \rdg[wit={E}]{brahmaraṃdhra°}
  \rdg[wit={ceteri}]{brahmaraṃdhraṃ}}\dd{}
\app{\lem[type=emendation, resp=egoscr]{ūrdhvakamalinyās\skp{-}trikūṭasthāne}
  \rdg[wit={U2}]{\korr urdhvakamalinyās trikūṭasthānam}
  \rdg[wit={U1}]{urdhvakamalinyaḥ trikūṭasthāne}
  \rdg[wit={D}]{ūrdhvakamalinyaḥ || trikūṭasthāne ||}
  \rdg[wit={L,P}]{ūrdhvaṃ kamalīnyā trikūṭasthānam}
  \rdg[wit={B}]{kamalīnyāṃ strikūṭasthānam}
  \rdg[wit={E}]{kamalinyas trikūṭasthānam}}\dd{}
\app{\lem[wit={D}]{saptapātāle}
  \rdg[wit={ceteri}]{\om}}\dd{} 
%----------------------------
%evam ekaviṃśatisthāne   ekaviṃśatibrahmāṃḍāni vasaṃti// \E
%evam ekaviṃśasthāneṣu   ekaviṃśabrahmāni vasaṃti \P
%ekam ekaṃ viṃśasthānek  ekaviṃśabrahmāḍānī vasaṃtī// \B
%ekam ekaṃ viṃśasthāneṣv ekaviṃśabrahmāḍānī vasaṃtī// \L %%%0024.jpg
%\om                                                                 \N1
%evaṃ ekaviṃśatisthāne   ekaviṃśatibrahmāṃḍāni vasaṃti// \D
%\om                                                                 \N2
%                        ekāviṃśatibrahmāṃḍāni vasaṃti \U1
%evam ekaviṃśasthān      ekaviṃśa---brahmāṃḍāni vasaṃti// \U2
%-----------------------------
%thus the 21 worlds reside in 21 locations.
%Thus they reside at the 21 worlds in the 21 locations. 
%----------------------------
\app{\lem[wit={ceteri},alt={evam}]{eva\skp{m-e}}
  \rdg[wit={D}]{evaṃ}}
\app{\lem[wit={P}, alt={ekaviṃśasthāneṣv}]{\skm{m-e}kaviṃśasthāneṣv}
  \rdg[wit={B}]{\korr viṃśasthānek°}
  \rdg[wit={L}]{ekaṃ viṃśasthāneṣv}
  \rdg[wit={E,D}]{ekaviṃśatisthāne}
  \rdg[wit={U2}]{ekaviṃśasthān}}
\app{\lem[wit={E,D,U1}]{ekaviṃśatibrahmāṃḍāni}
  \rdg[wit={B,P,L,U2}]{ekaviṃśabrahmāni}}
\app{\lem[wit={ceteri}]{vasanti}
  \rdg[wit={L,B}]{vasaṃtī}}/
\end{prose}
\end{ekdosis}
%%%%%%%%%%%%%%%%%%%%%%%%%%%%%%%%%%%%%%%%%%
%%%%%%%%%%%%%%%%%%%%%%%%%%%%%%%%%%%%%%%%%%
%%%%%%%%PAGEBREAK%%%%%%%PAGEBREAK%%%%%%%%%
%%%%%%%%%%%%%%%%%%%%%%%%%%%%%%%%%%%%%%%%%%
%%%%%%%%%%%%%%%%PAGEBREAK%%%%%%%%%%%%%%%%%
%%%%%%%%%%%%%%%%%%%%%%%%%%%%%%%%%%%%%%%%%%
%%%%%%%%PAGEBREAK%%%%%%%PAGEBREAK%%%%%%%%%
%%%%%%%%%%%%%%%%%%%%%%%%%%%%%%%%%%%%%%%%%%
%%%%%%%%%%%%%%%%%%%%%%%%%%%%%%%%%%%%%%%%%%
%%%%%%%%%%%%%%%%%%%%%%%%%%%%%%%%%%%%%%%%%%
%%%%%%%%%%%%%%%%%%%%%%%%%%%%%%%%%%%%%%%%%%
%%%%%%%%PAGEBREAK%%%%%%%PAGEBREAK%%%%%%%%%
%%%%%%%%%%%%%%%%%%%%%%%%%%%%%%%%%%%%%%%%%%
%%%%%%%%%%%%%%%%PAGEBREAK%%%%%%%%%%%%%%%%%
%%%%%%%%%%%%%%%%%%%%%%%%%%%%%%%%%%%%%%%%%%
%%%%%%%%PAGEBREAK%%%%%%%PAGEBREAK%%%%%%%%%
%%%%%%%%%%%%%%%%%%%%%%%%%%%%%%%%%%%%%%%%%%
%%%%%%%%%%%%%%%%%%%%%%%%%%%%%%%%%%%%%%%%%%
%%%%%%%%%%%%%%%%%%%%%%%%%%%%%%%%%%%%%%%%%%
%%%%%%%%%%%%%%%%%%%%%%%%%%%%%%%%%%%%%%%%%%
%%%%%%%%PAGEBREAK%%%%%%%PAGEBREAK%%%%%%%%%
%%%%%%%%%%%%%%%%%%%%%%%%%%%%%%%%%%%%%%%%%%
%%%%%%%%%%%%%%%%PAGEBREAK%%%%%%%%%%%%%%%%%
%%%%%%%%%%%%%%%%%%%%%%%%%%%%%%%%%%%%%%%%%%
%%%%%%%%PAGEBREAK%%%%%%%PAGEBREAK%%%%%%%%%
%%%%%%%%%%%%%%%%%%%%%%%%%%%%%%%%%%%%%%%%%%
\begin{ekdosis}
    \centerline{\textrm{\small{[Seven Islands]}}}
    \smallskip
    \begin{prose}
      \noindent
%----------------------------
%idānīṃ saptadvīpāni piṃḍamadhye kathyante// \E
%idānīṃ saptadvīpāni piṃḍamadhye kathyaṃte \P
%idānī  satyadvīpāni piṃḍamadhye kathyate// \B
%idānīṃ saptadvīpāni piṃḍamadhye kathyate \L
%\om                                                                 \N1
%idānīṃ saptadvīpāni piṃḍamadhye kathyaṃte// \D
%\om                                                                 \N2
%idānīṃ saptadvīpāni piṃḍamadhye kathyaṃte \U1
%idānīṃ saptadvīpāni piṃḍamadhye kathyaṃte// \U2
%-----------------------------
%----------------------------
\note[type=testium, labelb=274, lem={saptadvīpāni}]{SSP 3.7: majjāyaṃ jambūdvīpaḥ asthiṣu śaktidvīpaḥ śirāsu sūkṣmadvīpaḥ tvakṣu krauñcadvīpaḥ romasu gomayadvīpaḥ nakheṣu śvetadvīpaḥ māṃse (asthini) plakṣadvīpaḥ evaṃ saptadvīpāḥ |}
\note[type=source, labelb=275, lem={saptadvīpāni}]{Ysv\textsuperscript{PT}: sapta dvīpāni kathyante 'dhunā tāni śṛṇu priye  | jambūdvīpas tu majjāyāṃ śākadvīpas tu madhyamaḥ | śālmadvīpaḥ śiromadhye māṃsamadhye kuśas tathā | tvaci krauñco lomamadhye gomayadvīpa īritaḥ | nakhamadhye tathā śvetaḥ saptadvīpā vasundharā | jambūḥ śākastathā śālmaḥ kuśaḥ krauñcaś ca gomayaḥ | śvetaḥ sapteti khaṇḍāni saptakhaṇḍair vasundharā | guptāny etāni rūpāṇi dehamadhye sthirāṇi ca |}
idānīṃ saptadvīpāni piṃḍamadhye
\app{\lem[wit={ceteri}]{kathyante}
  \rdg[wit={B,L}]{kathyate}}/ 
%----------------------------
%majjāmadhye jaṃbudvīpaḥ/  asthimadhye śākadvīpaḥ      śirāmadhye   śālmalidvīpaḥ/    \E
%majjāmadhye jaṃbūdvīpaḥ/  asthīmadhye śākadvīpaḥ      śirāmadhye   śālmalidvīpaḥ     \P
%majjāmadhye jaṃbudvīpaḥ/  astimadhye  śākaladvīpaḥ//  śirāmadhye   śākaladvīpaḥ//    \B
%majjāmadhye jaṃbudvīpaḥ   astimadhye  śākaladvīpaḥ    śarīramadhye śākadvīpaḥ...     \L
%\om                                                                                  \N1
%majjāmadhye jaṃbudvīpaḥ// asthimadhye śākadvīpaḥ      śiromadhye   śālmalidvīpaḥ//   \D
%\om                                                                                  \N2
%majjāmadhye jaṃbudvīpaḥ   astimadhye  śāktidvīpaḥ     śīromadhye   śālmalidvīpaḥ     \U1
%majjāmadhye jaṃbudvīpaḥ// astimadhye  śākadvīpaḥ//    śīromadhye   śālmalīdvīpaḥ//   \U2
%-----------------------------
%Within the marrow is the island of Jambu. Within the bones is the island of Śāka. In the head is the island of Śālmali. 
%-----------------------------
majjāmadhye
\app{\lem[wit={ceteri}]{jambu}
  \rdg[wit={P}]{jaṃbū}}dvīpaḥ\dd{}
\app{\lem[wit={E,D}]{asthi}
  \rdg[wit={P}]{asthī}
  \rdg[wit={B,L,U1,U2}]{asti}}madhye
\app{\lem[wit={E,D,P,U2}]{śākadvīpaḥ}
  \rdg[wit={B,L}]{śākaladvīpaḥ}
  \rdg[wit={U1}]{śāktidvīpaḥ}}\dd{}
\app{\lem[wit={D,U1,U2}]{śiromadhye}
  \rdg[wit={B,E,P}]{śirāmadhye}
  \rdg[wit={L}]{śarīramadhye}}
\app{\lem[wit={ceteri}]{śālmalidvīpaḥ}
  \rdg[wit={U2}]{śālmalīdvīpaḥ}
  \rdg[wit={B}]{śākaladvīpaḥ}
  \rdg[wit={L}]{śākadvīpaḥ}}\dd{}
%----------------------------
%māṃsamadhye kuśadvīpaḥ/  tvacāmadhye krauṃcadvīpaḥ/  śarīrasthalomamadhye gomedadvīpaḥ/  nakhamadhye puṣkaradvīpaḥ//  etāni dvīpāni         madhye tiṣṭhanti// \E [p.50]
%māṃsamadhye kuśadvīpaḥ   tvacāmadhye krauṃcadvīpaḥ   śarīrasya lomamadhye gomedadvīpaḥ   nakhamadhye puṣkaradvīpaḥ    etāni dvīpāni guptāni madhye tiṣṭhaṃti \P
%māṃsamadhye kuśadvīpaḥ   tvacāmadhye krauṃcadvīpaḥ// śarīrasya lomamadhye gomedadvīpaḥ// nakhamadhye puṣkaradvīpaḥ//  etāni dvīpāni guptāni madhye tiṣṭhaṃti// \B
%māṃsamadhye kuśadvīpaḥ   tvacāmadhye krauṃcadvīpaḥ   śarīrasya lomamadhye gomedadvīpaḥ  taravamadhye puṣkaradvīpaḥ    etāni dvīpāni guptāni madhye tiṣṭhaṃti// \L
%\om                                                                                                                                                          \N1
%māṃsamadhye kuśadvīpaḥ// tvacāmadhye krauṃcadvīpaḥ   śarīrasya lomamadhye gomayadvīpaḥ/  nakhamadhye  śvetadvīpaḥ/    etāni rūpaṇi  guptamadhye   tiṣṭhaṃti// \D
%\om                                                                                                                                                            \N2
%māṃsamadhye kuśadvīpaḥ   tvacāmadhye krauṃcadvīpaḥ   śarīrasya lomadhye   gomayadvīpaḥ   taravamadhye svetadvīpaḥ     etāni rūpāṇī  guptamadhye   tiṣṭhaṃti \U1
%māṃsamadhye kuśadvīpaḥ// tvacāmadhye krauṃcadvīpaḥ// śarīrasya lomadhye   gomedadvīpaḥ// nakhamadhye  puṣkaradvīpaḥ// etāni dvīpāni guptāni madhye tiṣṭhaṃti// \U2
%-----------------------------
%In the flesh is the island of Kuśa. Within the skin is the island of Krauñca. At the hairy line between chest and navel (\textit{loma}) is the island of Gomaya. In dthe nails is the island of Śveta. These islands are situated are hidden within. 
%----------------------------
māṃsamadhye kuśadvīpaḥ\dd{} tvacāmadhye krauṃcadvīpaḥ\dd{} śarīrasya 
\app{\lem[wit={ceteri}]{lomamadhye}
  \rdg[wit={U1,U2}]{lomadhye}}
\app{\lem[wit={D,U1}]{gomayadvīpaḥ}
  \rdg[wit={ceteri}]{gomedadvīpaḥ}}\dd{}
\app{\lem[wit={ceteri}]{nakhamadhye}
  \rdg[wit={L,U1}]{taravamadhye}}
\app{\lem[wit={D,U1}]{śvetadvīpaḥ}
  \rdg[wit={ceteri}]{puṣkaradvīpaḥ}}
etāni \app{\lem[wit={ceteri}]{dvīpāni}
  \rdg[wit={D,U1}]{rūpaṇi}}
\app{\lem[wit={B,P,L,U2}]{guptāni}
  \rdg[wit={D,U1}]{gupta°}
  \rdg[wit={E}]{\om}}
madhye
tiṣṭhanti/
\end{prose}
\end{ekdosis}
%
\begin{ekdosis}
    \smallskip
    \centerline{\textrm{\small{[Seven Oceans]}}}
    \smallskip
    \begin{prose}
      \noindent
%----------------------------
%idānīṃ piṃḍamadhye saptasamudrāḥ kathyante// prasvedamadhye kṣārasamudraḥ/   \E
%idānīṃ piṃḍamadhye saptasamudrāḥ kathyaṃte   prasvedamadhye kṣārasamudraḥ    \P
%idānīṃ piṃḍamadhye samudrāḥ      kathyate//  prasvedamadhye kṣārasamudraḥ//  \B
%idānīṃ piṃḍamadhye samudrāḥ      kathyaṃte// prasvedamadhye sārasasamudraḥ// \L
%\om                                                                 \N1
%idānīṃ piṃḍamadhye saptasamudrāḥ kathyete//  prasvedamadhye kṣārasasamudra   \D
%\om                                                                 \N2
%idānīṃ piṃḍamadhye saptasamudrāḥ kathyaṃte      svedamadhye kṣārasasamudraḥ  \U1 %%%288.jpg
%idānīṃ piṃḍamadhye saptasamudrāḥ kathyaṃte// prasvedamadhye kṣārasāgaraḥ//   \U2
%-----------------------------
%Now the seven oceans within the body are taught. Within sweat is the salt ocean (1). 
%----------------------------
\note[type=source, labelb=276, lem={saptasamudrāḥ}]{Ysv\textsuperscript{PT}: samudrāḥ sapta kathyante piṇḍamadhye vyavasthitāḥ | lavaṇekṣusurāsarpirdadhidugdhajalāntakāḥ | lavaṇaṃ svedamadhye tu ikṣūrakte madhu tvaci | sarpir medo vasā madhye dadhi kṣīraṃ lalāṭake | vīryamadhye 'mṛto jñeyaḥ pāde kūrmaḥ sthito mahān |}
\note[type=source, labelb=277, lem={saptasamudrāḥ}]{SSP 3.8: mūrte kṣārasamudraḥ lālāyāṃ kṣīrasamudraḥ kaphe dadhisamudraḥ medasi ghṛtasamudraḥ vasāyāṃ madhusamudraḥ rakte ikṣusamudraḥ śukre 'mṛtasamudraḥ evaṃ saptasamudrāḥ||}
idānīṃ piṇḍamadhye
\app{\lem[wit={ceteri}]{saptasamudrāḥ}
  \rdg[wit={L,B}]{samidrāḥ}}
\app{\lem[wit={ceteri}]{kathyante}
  \rdg[wit={B}]{kathyate}
  \rdg[wit={D}]{kathyete}}/
\app{\lem[wit={ceteri}]{prasvedamadhye}
  \rdg[wit={U1}]{svedamadhye}}
\app{\lem[wit={ceteri}]{kṣārasamudraḥ}
  \rdg[wit={L}]{sārasasamudraḥ}
  \rdg[wit={U1}]{kṣārasasamudraḥ}
  \rdg[wit={U2}]{kṣārasāgaraḥ}}\dd{}
%----------------------------
%lalāṭamadhye kṣīraḥ samudraḥ/            vāṅmadhye                                 madhusamudraḥ/  kaphamadhye  dadhisamudraḥ/  medomadhye ghṛtasamudraḥ/  \E
%lālāmadhye   kṣīrasamudraḥ               vasāmadhye                                madhusamudraḥ   kaphamadhye  dadhisamudraḥ   medomadhye ghṛtasamudraḥ   \P
%lalāṭamadhye kṣīrasamudraḥ// raktamadhye vasāmadhye                                madasamudraḥ    kaphamadhye  dadhisamudraḥ// medomadhye ghṛtasamudraḥ// \B
%lalāṭamadhye kṣīrasamudraḥ// raktamadhye vasāmadhye                                madyasamudraḥ// kaphamadhye  dadhisamudraḥ// medamadhye ghṛtasamudraḥ// \L
%\om                                                                 \N1
%lalāṭamadhye kṣīrasamudraḥ/              vasāmadhye                                                             dadhisamudraḥ// medamadhye ghṛtasamudraḥ// \D
%\om                                                                                                                                                         \N2
%lalāṭamadhye kṣīrasamudraḥ               vasāmadhye                                                             dadhisamudraḥ   medamadhye ghṛtasamudraḥ   \U1 %%%288.jpg
%lalāṭamadhye kṣīrasamudraḥ//             vīryamadhye svāduḥ samudraḥ// majjāmadhye madhusamūdraḥ// kaphamadhye  dadhisamudraḥ// medamadhye ghṛtasamudraḥ// \U2
%-----------------------------
%Within the forehead is the milk ocean (2). Within the brain is the honey-ocean (3). In the phlegm is the sour milk ocean (4). In the fat is the butter ocean (5).  
%----------------------------
\app{\lem[wit={ceteri}]{lalāṭamadhye}
  \rdg[wit={P}]{lālāmadhye}}
\app{\lem[wit={ceteri}]{kṣīrasamudraḥ}
  \rdg[wit={E}]{kṣīraḥ samudraḥ}}\dd{}
\app{\lem[wit={ceteri}]{vasāmadhye}
  \rdg[wit={E}]{vāṅmadhye}
  \rdg[wit={U2}]{vīryamadhye svāduḥ samudraḥ || majjāmadhye}}
\app{\lem[wit={E,P}]{madhusamudraḥ}
  \rdg[wit={B}]{madasamudraḥ}
  \rdg[wit={L}]{madyasamudraḥ}
  \rdg[wit={U2}]{madhusamūdraḥ}}\dd{}
kaphamadhye dadhisamudraḥ\dd{}
\app{\lem[wit={ceteri}, alt={meda°}]{meda}
  \rdg[wit={B,E,P}]{medo°}}madhye ghṛtasamudraḥ\dd{}
%----------------------------
%                           rasamadhye   ikṣurasasamudraḥ// vīryamadhye svādusamudraḥ/                pādamadhye kūrmasthānam//   \E
%                           raktamadhye  ikṣurasasamudraḥ   vīryamadhye svādudakasamudraḥ             pādamadhye kūrmasthānam     \P
%                                        ikṣusamudraḥ/      vīryamadhye svādukasamudraḥ/  karmasthāna pādasamadhye/               \B
%                                        ikṣusamudraḥ//     vīryamadhye svādukasamudraḥ// karmasthāna pādamadhye                  \L
%\om                                                                                                                             \N1
%vasāmadhye madhusamudraḥ// raktamadhye  ikṣusamudraḥ//     vīryamadhye amṛtasamudraḥ/                pādamtale  kūrmasthānaṃ/    \D
%\om                                                                                                                             \N2
%vasāmadhye madhusamudraḥ   raktamadhye  ikṣurasamudraḥ     vīryamadhye mṛtasamudraḥ                  pādamadhye kūrmasthānaṃ     \U1 %%%288.jpg
%                           raktamadhye  ikṣurasamudraḥ//                                                                         \U2
%-----------------------------
%Within the forehead is the milk-ocean. Within the blood is the sugarcane ocean. Within the semen is the ocean of the nectar of immortality. Within the feet is the place of the turtle. 
%----------------------------
\app{\lem[wit={P,U1,U2}]{raktamadhye}
  \rdg[wit={D}]{vasāmadhye madhusamudraḥ || raktamadhye}
  \rdg[wit={U1}]{vasāmadhye madhusamudraḥ raktamadhye}
  \rdg[wit={E}]{rasamadhye}}
\app{\lem[wit={B,D,L}]{ikṣusamudraḥ}
  \rdg[wit={U1,U2}]{ikṣurasamudraḥ}
  \rdg[wit={E,P}]{ikṣurasasamudraḥ}}
\note[type=philcomm, labelb=278, lem={ikṣura°}]{Due to Sandhi \textit{akṣura°} would be exspected.}
vīryamadhye
\app{\lem[wit={U1}]{'mṛtasamudraḥ}
  \rdg[wit={D}]{amṛtasamudraḥ}
  \rdg[wit={E}]{svādusamudraḥ}
  \rdg[wit={B,L}]{svādukasamudraḥ}
  \rdg[wit={P}]{svādudakasamudraḥ}}\dd{}
\app{\lem[wit={ceteri}]{pādamadhye}
  \rdg[wit={B}]{karmasthāna pādasamadhye}
  \rdg[wit={L}]{karmasthāna pādamadhye}
  \rdg[wit={D}]{pādamtale}}
\app{\lem[wit={ceteri}]{kūrmasthānam}
  \rdg[wit={B,L}]{\om}}\dd{}
\note[type=philcomm, labelb=277, lem={kūrmasthānam}]{All witnesses preserve the statement of \textit{kūrmasthānam}, except for witness U\textsubscript{2} which places the statement two sentences later right after the introduction of the \textit{navadvāra}. In both cases it seems completely out of context. It must stem from the description of its source text, the Ysv\textsuperscript{PT} in which the statement seems likewise out of place.}
\end{prose}
\end{ekdosis}
\ekdpb*{}
%%%%%%%%%%%%%%%%%%%%%%%%%%%%%%%%%%%%%%%%%%
%%%%%%%%%%%%%%%%%%%%%%%%%%%%%%%%%%%%%%%%%%
%%%%%%%%PAGEBREAK%%%%%%%PAGEBREAK%%%%%%%%%
%%%%%%%%%%%%%%%%%%%%%%%%%%%%%%%%%%%%%%%%%%
%%%%%%%%%%%%%%%%PAGEBREAK%%%%%%%%%%%%%%%%%
%%%%%%%%%%%%%%%%%%%%%%%%%%%%%%%%%%%%%%%%%%
%%%%%%%%PAGEBREAK%%%%%%%PAGEBREAK%%%%%%%%%
%%%%%%%%%%%%%%%%%%%%%%%%%%%%%%%%%%%%%%%%%%
%%%%%%%%%%%%%%%%%%%%%%%%%%%%%%%%%%%%%%%%%%
%%%%%%%%%%%%%%%%%%%%%%%%%%%%%%%%%%%%%%%%%%
%%%%%%%%%%%%%%%%%%%%%%%%%%%%%%%%%%%%%%%%%%
%%%%%%%%PAGEBREAK%%%%%%%PAGEBREAK%%%%%%%%%
%%%%%%%%%%%%%%%%%%%%%%%%%%%%%%%%%%%%%%%%%%
%%%%%%%%%%%%%%%%PAGEBREAK%%%%%%%%%%%%%%%%%
%%%%%%%%%%%%%%%%%%%%%%%%%%%%%%%%%%%%%%%%%%
%%%%%%%%PAGEBREAK%%%%%%%PAGEBREAK%%%%%%%%%
%%%%%%%%%%%%%%%%%%%%%%%%%%%%%%%%%%%%%%%%%%
%%%%%%%%%%%%%%%%%%%%%%%%%%%%%%%%%%%%%%%%%%
%%%%%%%%%%%%%%%%%%%%%%%%%%%%%%%%%%%%%%%%%%
%%%%%%%%%%%%%%%%%%%%%%%%%%%%%%%%%%%%%%%%%%
%%%%%%%%PAGEBREAK%%%%%%%PAGEBREAK%%%%%%%%%
%%%%%%%%%%%%%%%%%%%%%%%%%%%%%%%%%%%%%%%%%%
%%%%%%%%%%%%%%%%PAGEBREAK%%%%%%%%%%%%%%%%%
%%%%%%%%%%%%%%%%%%%%%%%%%%%%%%%%%%%%%%%%%%
%%%%%%%%PAGEBREAK%%%%%%%PAGEBREAK%%%%%%%%%
%%%%%%%%%%%%%%%%%%%%%%%%%%%%%%%%%%%%%%%%%%